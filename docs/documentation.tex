\documentclass[12pt]{article}
\usepackage{pgf}
\usepackage{minted}
\title{Analysis of Various Sorting Algorithms}
\author{Deebakkarthi C R (CB.EN.U4CSE20613) \\ \and
Pravin Sabari Bala (CB.EN.U4CSE20648)}

\begin{document}
\begin{titlepage}
\maketitle
\pagebreak
\end{titlepage}
\tableofcontents
\pagebreak
\section{Comparisons of various sorting algorithms}
\subsection{Testing Data Generation}
The input data for this was using the \texttt{random.random()} function in 
python. The length of the input ranged from 100 to 10000.
\subsection{Taking Measurements}
\textit{Swaps, Comparisons and basic operations} were calculated by updating
a global variable. \textit{Time taken} was measured using
\texttt{perf\_counter\_ns()} and \textit{memory} was measured using
\texttt{tracemalloc()}
\subsection{Plot Generation}
The measurements were store in a \textit{csv} file which was then used by
\texttt{matplotlib} to produce the plots. The y-scale of certain plots which
were too large are presented in a \textit{log} scale.
\subsection{Theoretical time complexities}
\begin{table}[!h]
\begin{tabular}{|l|l|l|l|l|}
\hline
Algorithm & Best            & Average         & Worst      & Worst(Space)        \\ \hline
Quick     & $\Omega(nlogn)$ & $\Theta(nlogn)$ & $O(n^2)$   & $O(n) / O(1)$              \\ \hline
Merge     & $\Omega(nlogn)$ & $\Theta(nlogn)$ & $O(nlogn)$ & $O(n)$              \\ \hline
Heap      & $\Omega(nlogn)$ & $\Theta(nlogn)$ & $O(nlogn)$   & $O(n) / O(1)$ \\ \hline
Insertion & $\Omega(n)$     & $\Theta(nlogn)$ & $O(n^2)$   & $O(1)$              \\ \hline
Bucket    & $\Omega(n+k)$   & $\Theta(n+k)$   & $O(n^2)$   & $O(n)$              \\ \hline
\end{tabular}
\end{table}
Here the space complexity of \textit{Quick sort and Heap sort} varies depending
upon the implementation
\subsection{Plots}
%% Creator: Matplotlib, PGF backend
%%
%% To include the figure in your LaTeX document, write
%%   \input{<filename>.pgf}
%%
%% Make sure the required packages are loaded in your preamble
%%   \usepackage{pgf}
%%
%% Also ensure that all the required font packages are loaded; for instance,
%% the lmodern package is sometimes necessary when using math font.
%%   \usepackage{lmodern}
%%
%% Figures using additional raster images can only be included by \input if
%% they are in the same directory as the main LaTeX file. For loading figures
%% from other directories you can use the `import` package
%%   \usepackage{import}
%%
%% and then include the figures with
%%   \import{<path to file>}{<filename>.pgf}
%%
%% Matplotlib used the following preamble
%%   
%%   \makeatletter\@ifpackageloaded{underscore}{}{\usepackage[strings]{underscore}}\makeatother
%%
\begingroup%
\makeatletter%
\begin{pgfpicture}%
\pgfpathrectangle{\pgfpointorigin}{\pgfqpoint{6.400000in}{4.800000in}}%
\pgfusepath{use as bounding box, clip}%
\begin{pgfscope}%
\pgfsetbuttcap%
\pgfsetmiterjoin%
\definecolor{currentfill}{rgb}{1.000000,1.000000,1.000000}%
\pgfsetfillcolor{currentfill}%
\pgfsetlinewidth{0.000000pt}%
\definecolor{currentstroke}{rgb}{1.000000,1.000000,1.000000}%
\pgfsetstrokecolor{currentstroke}%
\pgfsetdash{}{0pt}%
\pgfpathmoveto{\pgfqpoint{0.000000in}{0.000000in}}%
\pgfpathlineto{\pgfqpoint{6.400000in}{0.000000in}}%
\pgfpathlineto{\pgfqpoint{6.400000in}{4.800000in}}%
\pgfpathlineto{\pgfqpoint{0.000000in}{4.800000in}}%
\pgfpathlineto{\pgfqpoint{0.000000in}{0.000000in}}%
\pgfpathclose%
\pgfusepath{fill}%
\end{pgfscope}%
\begin{pgfscope}%
\pgfsetbuttcap%
\pgfsetmiterjoin%
\definecolor{currentfill}{rgb}{1.000000,1.000000,1.000000}%
\pgfsetfillcolor{currentfill}%
\pgfsetlinewidth{0.000000pt}%
\definecolor{currentstroke}{rgb}{0.000000,0.000000,0.000000}%
\pgfsetstrokecolor{currentstroke}%
\pgfsetstrokeopacity{0.000000}%
\pgfsetdash{}{0pt}%
\pgfpathmoveto{\pgfqpoint{0.800000in}{0.528000in}}%
\pgfpathlineto{\pgfqpoint{5.760000in}{0.528000in}}%
\pgfpathlineto{\pgfqpoint{5.760000in}{4.224000in}}%
\pgfpathlineto{\pgfqpoint{0.800000in}{4.224000in}}%
\pgfpathlineto{\pgfqpoint{0.800000in}{0.528000in}}%
\pgfpathclose%
\pgfusepath{fill}%
\end{pgfscope}%
\begin{pgfscope}%
\pgfsetbuttcap%
\pgfsetroundjoin%
\definecolor{currentfill}{rgb}{0.000000,0.000000,0.000000}%
\pgfsetfillcolor{currentfill}%
\pgfsetlinewidth{0.803000pt}%
\definecolor{currentstroke}{rgb}{0.000000,0.000000,0.000000}%
\pgfsetstrokecolor{currentstroke}%
\pgfsetdash{}{0pt}%
\pgfsys@defobject{currentmarker}{\pgfqpoint{0.000000in}{-0.048611in}}{\pgfqpoint{0.000000in}{0.000000in}}{%
\pgfpathmoveto{\pgfqpoint{0.000000in}{0.000000in}}%
\pgfpathlineto{\pgfqpoint{0.000000in}{-0.048611in}}%
\pgfusepath{stroke,fill}%
}%
\begin{pgfscope}%
\pgfsys@transformshift{0.979443in}{0.528000in}%
\pgfsys@useobject{currentmarker}{}%
\end{pgfscope}%
\end{pgfscope}%
\begin{pgfscope}%
\definecolor{textcolor}{rgb}{0.000000,0.000000,0.000000}%
\pgfsetstrokecolor{textcolor}%
\pgfsetfillcolor{textcolor}%
\pgftext[x=0.979443in,y=0.430778in,,top]{\color{textcolor}\rmfamily\fontsize{10.000000}{12.000000}\selectfont \(\displaystyle {0}\)}%
\end{pgfscope}%
\begin{pgfscope}%
\pgfsetbuttcap%
\pgfsetroundjoin%
\definecolor{currentfill}{rgb}{0.000000,0.000000,0.000000}%
\pgfsetfillcolor{currentfill}%
\pgfsetlinewidth{0.803000pt}%
\definecolor{currentstroke}{rgb}{0.000000,0.000000,0.000000}%
\pgfsetstrokecolor{currentstroke}%
\pgfsetdash{}{0pt}%
\pgfsys@defobject{currentmarker}{\pgfqpoint{0.000000in}{-0.048611in}}{\pgfqpoint{0.000000in}{0.000000in}}{%
\pgfpathmoveto{\pgfqpoint{0.000000in}{0.000000in}}%
\pgfpathlineto{\pgfqpoint{0.000000in}{-0.048611in}}%
\pgfusepath{stroke,fill}%
}%
\begin{pgfscope}%
\pgfsys@transformshift{1.899666in}{0.528000in}%
\pgfsys@useobject{currentmarker}{}%
\end{pgfscope}%
\end{pgfscope}%
\begin{pgfscope}%
\definecolor{textcolor}{rgb}{0.000000,0.000000,0.000000}%
\pgfsetstrokecolor{textcolor}%
\pgfsetfillcolor{textcolor}%
\pgftext[x=1.899666in,y=0.430778in,,top]{\color{textcolor}\rmfamily\fontsize{10.000000}{12.000000}\selectfont \(\displaystyle {2000}\)}%
\end{pgfscope}%
\begin{pgfscope}%
\pgfsetbuttcap%
\pgfsetroundjoin%
\definecolor{currentfill}{rgb}{0.000000,0.000000,0.000000}%
\pgfsetfillcolor{currentfill}%
\pgfsetlinewidth{0.803000pt}%
\definecolor{currentstroke}{rgb}{0.000000,0.000000,0.000000}%
\pgfsetstrokecolor{currentstroke}%
\pgfsetdash{}{0pt}%
\pgfsys@defobject{currentmarker}{\pgfqpoint{0.000000in}{-0.048611in}}{\pgfqpoint{0.000000in}{0.000000in}}{%
\pgfpathmoveto{\pgfqpoint{0.000000in}{0.000000in}}%
\pgfpathlineto{\pgfqpoint{0.000000in}{-0.048611in}}%
\pgfusepath{stroke,fill}%
}%
\begin{pgfscope}%
\pgfsys@transformshift{2.819889in}{0.528000in}%
\pgfsys@useobject{currentmarker}{}%
\end{pgfscope}%
\end{pgfscope}%
\begin{pgfscope}%
\definecolor{textcolor}{rgb}{0.000000,0.000000,0.000000}%
\pgfsetstrokecolor{textcolor}%
\pgfsetfillcolor{textcolor}%
\pgftext[x=2.819889in,y=0.430778in,,top]{\color{textcolor}\rmfamily\fontsize{10.000000}{12.000000}\selectfont \(\displaystyle {4000}\)}%
\end{pgfscope}%
\begin{pgfscope}%
\pgfsetbuttcap%
\pgfsetroundjoin%
\definecolor{currentfill}{rgb}{0.000000,0.000000,0.000000}%
\pgfsetfillcolor{currentfill}%
\pgfsetlinewidth{0.803000pt}%
\definecolor{currentstroke}{rgb}{0.000000,0.000000,0.000000}%
\pgfsetstrokecolor{currentstroke}%
\pgfsetdash{}{0pt}%
\pgfsys@defobject{currentmarker}{\pgfqpoint{0.000000in}{-0.048611in}}{\pgfqpoint{0.000000in}{0.000000in}}{%
\pgfpathmoveto{\pgfqpoint{0.000000in}{0.000000in}}%
\pgfpathlineto{\pgfqpoint{0.000000in}{-0.048611in}}%
\pgfusepath{stroke,fill}%
}%
\begin{pgfscope}%
\pgfsys@transformshift{3.740111in}{0.528000in}%
\pgfsys@useobject{currentmarker}{}%
\end{pgfscope}%
\end{pgfscope}%
\begin{pgfscope}%
\definecolor{textcolor}{rgb}{0.000000,0.000000,0.000000}%
\pgfsetstrokecolor{textcolor}%
\pgfsetfillcolor{textcolor}%
\pgftext[x=3.740111in,y=0.430778in,,top]{\color{textcolor}\rmfamily\fontsize{10.000000}{12.000000}\selectfont \(\displaystyle {6000}\)}%
\end{pgfscope}%
\begin{pgfscope}%
\pgfsetbuttcap%
\pgfsetroundjoin%
\definecolor{currentfill}{rgb}{0.000000,0.000000,0.000000}%
\pgfsetfillcolor{currentfill}%
\pgfsetlinewidth{0.803000pt}%
\definecolor{currentstroke}{rgb}{0.000000,0.000000,0.000000}%
\pgfsetstrokecolor{currentstroke}%
\pgfsetdash{}{0pt}%
\pgfsys@defobject{currentmarker}{\pgfqpoint{0.000000in}{-0.048611in}}{\pgfqpoint{0.000000in}{0.000000in}}{%
\pgfpathmoveto{\pgfqpoint{0.000000in}{0.000000in}}%
\pgfpathlineto{\pgfqpoint{0.000000in}{-0.048611in}}%
\pgfusepath{stroke,fill}%
}%
\begin{pgfscope}%
\pgfsys@transformshift{4.660334in}{0.528000in}%
\pgfsys@useobject{currentmarker}{}%
\end{pgfscope}%
\end{pgfscope}%
\begin{pgfscope}%
\definecolor{textcolor}{rgb}{0.000000,0.000000,0.000000}%
\pgfsetstrokecolor{textcolor}%
\pgfsetfillcolor{textcolor}%
\pgftext[x=4.660334in,y=0.430778in,,top]{\color{textcolor}\rmfamily\fontsize{10.000000}{12.000000}\selectfont \(\displaystyle {8000}\)}%
\end{pgfscope}%
\begin{pgfscope}%
\pgfsetbuttcap%
\pgfsetroundjoin%
\definecolor{currentfill}{rgb}{0.000000,0.000000,0.000000}%
\pgfsetfillcolor{currentfill}%
\pgfsetlinewidth{0.803000pt}%
\definecolor{currentstroke}{rgb}{0.000000,0.000000,0.000000}%
\pgfsetstrokecolor{currentstroke}%
\pgfsetdash{}{0pt}%
\pgfsys@defobject{currentmarker}{\pgfqpoint{0.000000in}{-0.048611in}}{\pgfqpoint{0.000000in}{0.000000in}}{%
\pgfpathmoveto{\pgfqpoint{0.000000in}{0.000000in}}%
\pgfpathlineto{\pgfqpoint{0.000000in}{-0.048611in}}%
\pgfusepath{stroke,fill}%
}%
\begin{pgfscope}%
\pgfsys@transformshift{5.580557in}{0.528000in}%
\pgfsys@useobject{currentmarker}{}%
\end{pgfscope}%
\end{pgfscope}%
\begin{pgfscope}%
\definecolor{textcolor}{rgb}{0.000000,0.000000,0.000000}%
\pgfsetstrokecolor{textcolor}%
\pgfsetfillcolor{textcolor}%
\pgftext[x=5.580557in,y=0.430778in,,top]{\color{textcolor}\rmfamily\fontsize{10.000000}{12.000000}\selectfont \(\displaystyle {10000}\)}%
\end{pgfscope}%
\begin{pgfscope}%
\definecolor{textcolor}{rgb}{0.000000,0.000000,0.000000}%
\pgfsetstrokecolor{textcolor}%
\pgfsetfillcolor{textcolor}%
\pgftext[x=3.280000in,y=0.251766in,,top]{\color{textcolor}\rmfamily\fontsize{10.000000}{12.000000}\selectfont Input Size}%
\end{pgfscope}%
\begin{pgfscope}%
\pgfsetbuttcap%
\pgfsetroundjoin%
\definecolor{currentfill}{rgb}{0.000000,0.000000,0.000000}%
\pgfsetfillcolor{currentfill}%
\pgfsetlinewidth{0.803000pt}%
\definecolor{currentstroke}{rgb}{0.000000,0.000000,0.000000}%
\pgfsetstrokecolor{currentstroke}%
\pgfsetdash{}{0pt}%
\pgfsys@defobject{currentmarker}{\pgfqpoint{-0.048611in}{0.000000in}}{\pgfqpoint{-0.000000in}{0.000000in}}{%
\pgfpathmoveto{\pgfqpoint{-0.000000in}{0.000000in}}%
\pgfpathlineto{\pgfqpoint{-0.048611in}{0.000000in}}%
\pgfusepath{stroke,fill}%
}%
\begin{pgfscope}%
\pgfsys@transformshift{0.800000in}{0.896375in}%
\pgfsys@useobject{currentmarker}{}%
\end{pgfscope}%
\end{pgfscope}%
\begin{pgfscope}%
\definecolor{textcolor}{rgb}{0.000000,0.000000,0.000000}%
\pgfsetstrokecolor{textcolor}%
\pgfsetfillcolor{textcolor}%
\pgftext[x=0.501581in, y=0.848150in, left, base]{\color{textcolor}\rmfamily\fontsize{10.000000}{12.000000}\selectfont \(\displaystyle {10^{3}}\)}%
\end{pgfscope}%
\begin{pgfscope}%
\pgfsetbuttcap%
\pgfsetroundjoin%
\definecolor{currentfill}{rgb}{0.000000,0.000000,0.000000}%
\pgfsetfillcolor{currentfill}%
\pgfsetlinewidth{0.803000pt}%
\definecolor{currentstroke}{rgb}{0.000000,0.000000,0.000000}%
\pgfsetstrokecolor{currentstroke}%
\pgfsetdash{}{0pt}%
\pgfsys@defobject{currentmarker}{\pgfqpoint{-0.048611in}{0.000000in}}{\pgfqpoint{-0.000000in}{0.000000in}}{%
\pgfpathmoveto{\pgfqpoint{-0.000000in}{0.000000in}}%
\pgfpathlineto{\pgfqpoint{-0.048611in}{0.000000in}}%
\pgfusepath{stroke,fill}%
}%
\begin{pgfscope}%
\pgfsys@transformshift{0.800000in}{1.569747in}%
\pgfsys@useobject{currentmarker}{}%
\end{pgfscope}%
\end{pgfscope}%
\begin{pgfscope}%
\definecolor{textcolor}{rgb}{0.000000,0.000000,0.000000}%
\pgfsetstrokecolor{textcolor}%
\pgfsetfillcolor{textcolor}%
\pgftext[x=0.501581in, y=1.521521in, left, base]{\color{textcolor}\rmfamily\fontsize{10.000000}{12.000000}\selectfont \(\displaystyle {10^{4}}\)}%
\end{pgfscope}%
\begin{pgfscope}%
\pgfsetbuttcap%
\pgfsetroundjoin%
\definecolor{currentfill}{rgb}{0.000000,0.000000,0.000000}%
\pgfsetfillcolor{currentfill}%
\pgfsetlinewidth{0.803000pt}%
\definecolor{currentstroke}{rgb}{0.000000,0.000000,0.000000}%
\pgfsetstrokecolor{currentstroke}%
\pgfsetdash{}{0pt}%
\pgfsys@defobject{currentmarker}{\pgfqpoint{-0.048611in}{0.000000in}}{\pgfqpoint{-0.000000in}{0.000000in}}{%
\pgfpathmoveto{\pgfqpoint{-0.000000in}{0.000000in}}%
\pgfpathlineto{\pgfqpoint{-0.048611in}{0.000000in}}%
\pgfusepath{stroke,fill}%
}%
\begin{pgfscope}%
\pgfsys@transformshift{0.800000in}{2.243118in}%
\pgfsys@useobject{currentmarker}{}%
\end{pgfscope}%
\end{pgfscope}%
\begin{pgfscope}%
\definecolor{textcolor}{rgb}{0.000000,0.000000,0.000000}%
\pgfsetstrokecolor{textcolor}%
\pgfsetfillcolor{textcolor}%
\pgftext[x=0.501581in, y=2.194893in, left, base]{\color{textcolor}\rmfamily\fontsize{10.000000}{12.000000}\selectfont \(\displaystyle {10^{5}}\)}%
\end{pgfscope}%
\begin{pgfscope}%
\pgfsetbuttcap%
\pgfsetroundjoin%
\definecolor{currentfill}{rgb}{0.000000,0.000000,0.000000}%
\pgfsetfillcolor{currentfill}%
\pgfsetlinewidth{0.803000pt}%
\definecolor{currentstroke}{rgb}{0.000000,0.000000,0.000000}%
\pgfsetstrokecolor{currentstroke}%
\pgfsetdash{}{0pt}%
\pgfsys@defobject{currentmarker}{\pgfqpoint{-0.048611in}{0.000000in}}{\pgfqpoint{-0.000000in}{0.000000in}}{%
\pgfpathmoveto{\pgfqpoint{-0.000000in}{0.000000in}}%
\pgfpathlineto{\pgfqpoint{-0.048611in}{0.000000in}}%
\pgfusepath{stroke,fill}%
}%
\begin{pgfscope}%
\pgfsys@transformshift{0.800000in}{2.916490in}%
\pgfsys@useobject{currentmarker}{}%
\end{pgfscope}%
\end{pgfscope}%
\begin{pgfscope}%
\definecolor{textcolor}{rgb}{0.000000,0.000000,0.000000}%
\pgfsetstrokecolor{textcolor}%
\pgfsetfillcolor{textcolor}%
\pgftext[x=0.501581in, y=2.868265in, left, base]{\color{textcolor}\rmfamily\fontsize{10.000000}{12.000000}\selectfont \(\displaystyle {10^{6}}\)}%
\end{pgfscope}%
\begin{pgfscope}%
\pgfsetbuttcap%
\pgfsetroundjoin%
\definecolor{currentfill}{rgb}{0.000000,0.000000,0.000000}%
\pgfsetfillcolor{currentfill}%
\pgfsetlinewidth{0.803000pt}%
\definecolor{currentstroke}{rgb}{0.000000,0.000000,0.000000}%
\pgfsetstrokecolor{currentstroke}%
\pgfsetdash{}{0pt}%
\pgfsys@defobject{currentmarker}{\pgfqpoint{-0.048611in}{0.000000in}}{\pgfqpoint{-0.000000in}{0.000000in}}{%
\pgfpathmoveto{\pgfqpoint{-0.000000in}{0.000000in}}%
\pgfpathlineto{\pgfqpoint{-0.048611in}{0.000000in}}%
\pgfusepath{stroke,fill}%
}%
\begin{pgfscope}%
\pgfsys@transformshift{0.800000in}{3.589861in}%
\pgfsys@useobject{currentmarker}{}%
\end{pgfscope}%
\end{pgfscope}%
\begin{pgfscope}%
\definecolor{textcolor}{rgb}{0.000000,0.000000,0.000000}%
\pgfsetstrokecolor{textcolor}%
\pgfsetfillcolor{textcolor}%
\pgftext[x=0.501581in, y=3.541636in, left, base]{\color{textcolor}\rmfamily\fontsize{10.000000}{12.000000}\selectfont \(\displaystyle {10^{7}}\)}%
\end{pgfscope}%
\begin{pgfscope}%
\pgfsetbuttcap%
\pgfsetroundjoin%
\definecolor{currentfill}{rgb}{0.000000,0.000000,0.000000}%
\pgfsetfillcolor{currentfill}%
\pgfsetlinewidth{0.602250pt}%
\definecolor{currentstroke}{rgb}{0.000000,0.000000,0.000000}%
\pgfsetstrokecolor{currentstroke}%
\pgfsetdash{}{0pt}%
\pgfsys@defobject{currentmarker}{\pgfqpoint{-0.027778in}{0.000000in}}{\pgfqpoint{-0.000000in}{0.000000in}}{%
\pgfpathmoveto{\pgfqpoint{-0.000000in}{0.000000in}}%
\pgfpathlineto{\pgfqpoint{-0.027778in}{0.000000in}}%
\pgfusepath{stroke,fill}%
}%
\begin{pgfscope}%
\pgfsys@transformshift{0.800000in}{0.544283in}%
\pgfsys@useobject{currentmarker}{}%
\end{pgfscope}%
\end{pgfscope}%
\begin{pgfscope}%
\pgfsetbuttcap%
\pgfsetroundjoin%
\definecolor{currentfill}{rgb}{0.000000,0.000000,0.000000}%
\pgfsetfillcolor{currentfill}%
\pgfsetlinewidth{0.602250pt}%
\definecolor{currentstroke}{rgb}{0.000000,0.000000,0.000000}%
\pgfsetstrokecolor{currentstroke}%
\pgfsetdash{}{0pt}%
\pgfsys@defobject{currentmarker}{\pgfqpoint{-0.027778in}{0.000000in}}{\pgfqpoint{-0.000000in}{0.000000in}}{%
\pgfpathmoveto{\pgfqpoint{-0.000000in}{0.000000in}}%
\pgfpathlineto{\pgfqpoint{-0.027778in}{0.000000in}}%
\pgfusepath{stroke,fill}%
}%
\begin{pgfscope}%
\pgfsys@transformshift{0.800000in}{0.628413in}%
\pgfsys@useobject{currentmarker}{}%
\end{pgfscope}%
\end{pgfscope}%
\begin{pgfscope}%
\pgfsetbuttcap%
\pgfsetroundjoin%
\definecolor{currentfill}{rgb}{0.000000,0.000000,0.000000}%
\pgfsetfillcolor{currentfill}%
\pgfsetlinewidth{0.602250pt}%
\definecolor{currentstroke}{rgb}{0.000000,0.000000,0.000000}%
\pgfsetstrokecolor{currentstroke}%
\pgfsetdash{}{0pt}%
\pgfsys@defobject{currentmarker}{\pgfqpoint{-0.027778in}{0.000000in}}{\pgfqpoint{-0.000000in}{0.000000in}}{%
\pgfpathmoveto{\pgfqpoint{-0.000000in}{0.000000in}}%
\pgfpathlineto{\pgfqpoint{-0.027778in}{0.000000in}}%
\pgfusepath{stroke,fill}%
}%
\begin{pgfscope}%
\pgfsys@transformshift{0.800000in}{0.693670in}%
\pgfsys@useobject{currentmarker}{}%
\end{pgfscope}%
\end{pgfscope}%
\begin{pgfscope}%
\pgfsetbuttcap%
\pgfsetroundjoin%
\definecolor{currentfill}{rgb}{0.000000,0.000000,0.000000}%
\pgfsetfillcolor{currentfill}%
\pgfsetlinewidth{0.602250pt}%
\definecolor{currentstroke}{rgb}{0.000000,0.000000,0.000000}%
\pgfsetstrokecolor{currentstroke}%
\pgfsetdash{}{0pt}%
\pgfsys@defobject{currentmarker}{\pgfqpoint{-0.027778in}{0.000000in}}{\pgfqpoint{-0.000000in}{0.000000in}}{%
\pgfpathmoveto{\pgfqpoint{-0.000000in}{0.000000in}}%
\pgfpathlineto{\pgfqpoint{-0.027778in}{0.000000in}}%
\pgfusepath{stroke,fill}%
}%
\begin{pgfscope}%
\pgfsys@transformshift{0.800000in}{0.746988in}%
\pgfsys@useobject{currentmarker}{}%
\end{pgfscope}%
\end{pgfscope}%
\begin{pgfscope}%
\pgfsetbuttcap%
\pgfsetroundjoin%
\definecolor{currentfill}{rgb}{0.000000,0.000000,0.000000}%
\pgfsetfillcolor{currentfill}%
\pgfsetlinewidth{0.602250pt}%
\definecolor{currentstroke}{rgb}{0.000000,0.000000,0.000000}%
\pgfsetstrokecolor{currentstroke}%
\pgfsetdash{}{0pt}%
\pgfsys@defobject{currentmarker}{\pgfqpoint{-0.027778in}{0.000000in}}{\pgfqpoint{-0.000000in}{0.000000in}}{%
\pgfpathmoveto{\pgfqpoint{-0.000000in}{0.000000in}}%
\pgfpathlineto{\pgfqpoint{-0.027778in}{0.000000in}}%
\pgfusepath{stroke,fill}%
}%
\begin{pgfscope}%
\pgfsys@transformshift{0.800000in}{0.792068in}%
\pgfsys@useobject{currentmarker}{}%
\end{pgfscope}%
\end{pgfscope}%
\begin{pgfscope}%
\pgfsetbuttcap%
\pgfsetroundjoin%
\definecolor{currentfill}{rgb}{0.000000,0.000000,0.000000}%
\pgfsetfillcolor{currentfill}%
\pgfsetlinewidth{0.602250pt}%
\definecolor{currentstroke}{rgb}{0.000000,0.000000,0.000000}%
\pgfsetstrokecolor{currentstroke}%
\pgfsetdash{}{0pt}%
\pgfsys@defobject{currentmarker}{\pgfqpoint{-0.027778in}{0.000000in}}{\pgfqpoint{-0.000000in}{0.000000in}}{%
\pgfpathmoveto{\pgfqpoint{-0.000000in}{0.000000in}}%
\pgfpathlineto{\pgfqpoint{-0.027778in}{0.000000in}}%
\pgfusepath{stroke,fill}%
}%
\begin{pgfscope}%
\pgfsys@transformshift{0.800000in}{0.831118in}%
\pgfsys@useobject{currentmarker}{}%
\end{pgfscope}%
\end{pgfscope}%
\begin{pgfscope}%
\pgfsetbuttcap%
\pgfsetroundjoin%
\definecolor{currentfill}{rgb}{0.000000,0.000000,0.000000}%
\pgfsetfillcolor{currentfill}%
\pgfsetlinewidth{0.602250pt}%
\definecolor{currentstroke}{rgb}{0.000000,0.000000,0.000000}%
\pgfsetstrokecolor{currentstroke}%
\pgfsetdash{}{0pt}%
\pgfsys@defobject{currentmarker}{\pgfqpoint{-0.027778in}{0.000000in}}{\pgfqpoint{-0.000000in}{0.000000in}}{%
\pgfpathmoveto{\pgfqpoint{-0.000000in}{0.000000in}}%
\pgfpathlineto{\pgfqpoint{-0.027778in}{0.000000in}}%
\pgfusepath{stroke,fill}%
}%
\begin{pgfscope}%
\pgfsys@transformshift{0.800000in}{0.865563in}%
\pgfsys@useobject{currentmarker}{}%
\end{pgfscope}%
\end{pgfscope}%
\begin{pgfscope}%
\pgfsetbuttcap%
\pgfsetroundjoin%
\definecolor{currentfill}{rgb}{0.000000,0.000000,0.000000}%
\pgfsetfillcolor{currentfill}%
\pgfsetlinewidth{0.602250pt}%
\definecolor{currentstroke}{rgb}{0.000000,0.000000,0.000000}%
\pgfsetstrokecolor{currentstroke}%
\pgfsetdash{}{0pt}%
\pgfsys@defobject{currentmarker}{\pgfqpoint{-0.027778in}{0.000000in}}{\pgfqpoint{-0.000000in}{0.000000in}}{%
\pgfpathmoveto{\pgfqpoint{-0.000000in}{0.000000in}}%
\pgfpathlineto{\pgfqpoint{-0.027778in}{0.000000in}}%
\pgfusepath{stroke,fill}%
}%
\begin{pgfscope}%
\pgfsys@transformshift{0.800000in}{1.099080in}%
\pgfsys@useobject{currentmarker}{}%
\end{pgfscope}%
\end{pgfscope}%
\begin{pgfscope}%
\pgfsetbuttcap%
\pgfsetroundjoin%
\definecolor{currentfill}{rgb}{0.000000,0.000000,0.000000}%
\pgfsetfillcolor{currentfill}%
\pgfsetlinewidth{0.602250pt}%
\definecolor{currentstroke}{rgb}{0.000000,0.000000,0.000000}%
\pgfsetstrokecolor{currentstroke}%
\pgfsetdash{}{0pt}%
\pgfsys@defobject{currentmarker}{\pgfqpoint{-0.027778in}{0.000000in}}{\pgfqpoint{-0.000000in}{0.000000in}}{%
\pgfpathmoveto{\pgfqpoint{-0.000000in}{0.000000in}}%
\pgfpathlineto{\pgfqpoint{-0.027778in}{0.000000in}}%
\pgfusepath{stroke,fill}%
}%
\begin{pgfscope}%
\pgfsys@transformshift{0.800000in}{1.217655in}%
\pgfsys@useobject{currentmarker}{}%
\end{pgfscope}%
\end{pgfscope}%
\begin{pgfscope}%
\pgfsetbuttcap%
\pgfsetroundjoin%
\definecolor{currentfill}{rgb}{0.000000,0.000000,0.000000}%
\pgfsetfillcolor{currentfill}%
\pgfsetlinewidth{0.602250pt}%
\definecolor{currentstroke}{rgb}{0.000000,0.000000,0.000000}%
\pgfsetstrokecolor{currentstroke}%
\pgfsetdash{}{0pt}%
\pgfsys@defobject{currentmarker}{\pgfqpoint{-0.027778in}{0.000000in}}{\pgfqpoint{-0.000000in}{0.000000in}}{%
\pgfpathmoveto{\pgfqpoint{-0.000000in}{0.000000in}}%
\pgfpathlineto{\pgfqpoint{-0.027778in}{0.000000in}}%
\pgfusepath{stroke,fill}%
}%
\begin{pgfscope}%
\pgfsys@transformshift{0.800000in}{1.301785in}%
\pgfsys@useobject{currentmarker}{}%
\end{pgfscope}%
\end{pgfscope}%
\begin{pgfscope}%
\pgfsetbuttcap%
\pgfsetroundjoin%
\definecolor{currentfill}{rgb}{0.000000,0.000000,0.000000}%
\pgfsetfillcolor{currentfill}%
\pgfsetlinewidth{0.602250pt}%
\definecolor{currentstroke}{rgb}{0.000000,0.000000,0.000000}%
\pgfsetstrokecolor{currentstroke}%
\pgfsetdash{}{0pt}%
\pgfsys@defobject{currentmarker}{\pgfqpoint{-0.027778in}{0.000000in}}{\pgfqpoint{-0.000000in}{0.000000in}}{%
\pgfpathmoveto{\pgfqpoint{-0.000000in}{0.000000in}}%
\pgfpathlineto{\pgfqpoint{-0.027778in}{0.000000in}}%
\pgfusepath{stroke,fill}%
}%
\begin{pgfscope}%
\pgfsys@transformshift{0.800000in}{1.367041in}%
\pgfsys@useobject{currentmarker}{}%
\end{pgfscope}%
\end{pgfscope}%
\begin{pgfscope}%
\pgfsetbuttcap%
\pgfsetroundjoin%
\definecolor{currentfill}{rgb}{0.000000,0.000000,0.000000}%
\pgfsetfillcolor{currentfill}%
\pgfsetlinewidth{0.602250pt}%
\definecolor{currentstroke}{rgb}{0.000000,0.000000,0.000000}%
\pgfsetstrokecolor{currentstroke}%
\pgfsetdash{}{0pt}%
\pgfsys@defobject{currentmarker}{\pgfqpoint{-0.027778in}{0.000000in}}{\pgfqpoint{-0.000000in}{0.000000in}}{%
\pgfpathmoveto{\pgfqpoint{-0.000000in}{0.000000in}}%
\pgfpathlineto{\pgfqpoint{-0.027778in}{0.000000in}}%
\pgfusepath{stroke,fill}%
}%
\begin{pgfscope}%
\pgfsys@transformshift{0.800000in}{1.420360in}%
\pgfsys@useobject{currentmarker}{}%
\end{pgfscope}%
\end{pgfscope}%
\begin{pgfscope}%
\pgfsetbuttcap%
\pgfsetroundjoin%
\definecolor{currentfill}{rgb}{0.000000,0.000000,0.000000}%
\pgfsetfillcolor{currentfill}%
\pgfsetlinewidth{0.602250pt}%
\definecolor{currentstroke}{rgb}{0.000000,0.000000,0.000000}%
\pgfsetstrokecolor{currentstroke}%
\pgfsetdash{}{0pt}%
\pgfsys@defobject{currentmarker}{\pgfqpoint{-0.027778in}{0.000000in}}{\pgfqpoint{-0.000000in}{0.000000in}}{%
\pgfpathmoveto{\pgfqpoint{-0.000000in}{0.000000in}}%
\pgfpathlineto{\pgfqpoint{-0.027778in}{0.000000in}}%
\pgfusepath{stroke,fill}%
}%
\begin{pgfscope}%
\pgfsys@transformshift{0.800000in}{1.465440in}%
\pgfsys@useobject{currentmarker}{}%
\end{pgfscope}%
\end{pgfscope}%
\begin{pgfscope}%
\pgfsetbuttcap%
\pgfsetroundjoin%
\definecolor{currentfill}{rgb}{0.000000,0.000000,0.000000}%
\pgfsetfillcolor{currentfill}%
\pgfsetlinewidth{0.602250pt}%
\definecolor{currentstroke}{rgb}{0.000000,0.000000,0.000000}%
\pgfsetstrokecolor{currentstroke}%
\pgfsetdash{}{0pt}%
\pgfsys@defobject{currentmarker}{\pgfqpoint{-0.027778in}{0.000000in}}{\pgfqpoint{-0.000000in}{0.000000in}}{%
\pgfpathmoveto{\pgfqpoint{-0.000000in}{0.000000in}}%
\pgfpathlineto{\pgfqpoint{-0.027778in}{0.000000in}}%
\pgfusepath{stroke,fill}%
}%
\begin{pgfscope}%
\pgfsys@transformshift{0.800000in}{1.504490in}%
\pgfsys@useobject{currentmarker}{}%
\end{pgfscope}%
\end{pgfscope}%
\begin{pgfscope}%
\pgfsetbuttcap%
\pgfsetroundjoin%
\definecolor{currentfill}{rgb}{0.000000,0.000000,0.000000}%
\pgfsetfillcolor{currentfill}%
\pgfsetlinewidth{0.602250pt}%
\definecolor{currentstroke}{rgb}{0.000000,0.000000,0.000000}%
\pgfsetstrokecolor{currentstroke}%
\pgfsetdash{}{0pt}%
\pgfsys@defobject{currentmarker}{\pgfqpoint{-0.027778in}{0.000000in}}{\pgfqpoint{-0.000000in}{0.000000in}}{%
\pgfpathmoveto{\pgfqpoint{-0.000000in}{0.000000in}}%
\pgfpathlineto{\pgfqpoint{-0.027778in}{0.000000in}}%
\pgfusepath{stroke,fill}%
}%
\begin{pgfscope}%
\pgfsys@transformshift{0.800000in}{1.538935in}%
\pgfsys@useobject{currentmarker}{}%
\end{pgfscope}%
\end{pgfscope}%
\begin{pgfscope}%
\pgfsetbuttcap%
\pgfsetroundjoin%
\definecolor{currentfill}{rgb}{0.000000,0.000000,0.000000}%
\pgfsetfillcolor{currentfill}%
\pgfsetlinewidth{0.602250pt}%
\definecolor{currentstroke}{rgb}{0.000000,0.000000,0.000000}%
\pgfsetstrokecolor{currentstroke}%
\pgfsetdash{}{0pt}%
\pgfsys@defobject{currentmarker}{\pgfqpoint{-0.027778in}{0.000000in}}{\pgfqpoint{-0.000000in}{0.000000in}}{%
\pgfpathmoveto{\pgfqpoint{-0.000000in}{0.000000in}}%
\pgfpathlineto{\pgfqpoint{-0.027778in}{0.000000in}}%
\pgfusepath{stroke,fill}%
}%
\begin{pgfscope}%
\pgfsys@transformshift{0.800000in}{1.772452in}%
\pgfsys@useobject{currentmarker}{}%
\end{pgfscope}%
\end{pgfscope}%
\begin{pgfscope}%
\pgfsetbuttcap%
\pgfsetroundjoin%
\definecolor{currentfill}{rgb}{0.000000,0.000000,0.000000}%
\pgfsetfillcolor{currentfill}%
\pgfsetlinewidth{0.602250pt}%
\definecolor{currentstroke}{rgb}{0.000000,0.000000,0.000000}%
\pgfsetstrokecolor{currentstroke}%
\pgfsetdash{}{0pt}%
\pgfsys@defobject{currentmarker}{\pgfqpoint{-0.027778in}{0.000000in}}{\pgfqpoint{-0.000000in}{0.000000in}}{%
\pgfpathmoveto{\pgfqpoint{-0.000000in}{0.000000in}}%
\pgfpathlineto{\pgfqpoint{-0.027778in}{0.000000in}}%
\pgfusepath{stroke,fill}%
}%
\begin{pgfscope}%
\pgfsys@transformshift{0.800000in}{1.891026in}%
\pgfsys@useobject{currentmarker}{}%
\end{pgfscope}%
\end{pgfscope}%
\begin{pgfscope}%
\pgfsetbuttcap%
\pgfsetroundjoin%
\definecolor{currentfill}{rgb}{0.000000,0.000000,0.000000}%
\pgfsetfillcolor{currentfill}%
\pgfsetlinewidth{0.602250pt}%
\definecolor{currentstroke}{rgb}{0.000000,0.000000,0.000000}%
\pgfsetstrokecolor{currentstroke}%
\pgfsetdash{}{0pt}%
\pgfsys@defobject{currentmarker}{\pgfqpoint{-0.027778in}{0.000000in}}{\pgfqpoint{-0.000000in}{0.000000in}}{%
\pgfpathmoveto{\pgfqpoint{-0.000000in}{0.000000in}}%
\pgfpathlineto{\pgfqpoint{-0.027778in}{0.000000in}}%
\pgfusepath{stroke,fill}%
}%
\begin{pgfscope}%
\pgfsys@transformshift{0.800000in}{1.975157in}%
\pgfsys@useobject{currentmarker}{}%
\end{pgfscope}%
\end{pgfscope}%
\begin{pgfscope}%
\pgfsetbuttcap%
\pgfsetroundjoin%
\definecolor{currentfill}{rgb}{0.000000,0.000000,0.000000}%
\pgfsetfillcolor{currentfill}%
\pgfsetlinewidth{0.602250pt}%
\definecolor{currentstroke}{rgb}{0.000000,0.000000,0.000000}%
\pgfsetstrokecolor{currentstroke}%
\pgfsetdash{}{0pt}%
\pgfsys@defobject{currentmarker}{\pgfqpoint{-0.027778in}{0.000000in}}{\pgfqpoint{-0.000000in}{0.000000in}}{%
\pgfpathmoveto{\pgfqpoint{-0.000000in}{0.000000in}}%
\pgfpathlineto{\pgfqpoint{-0.027778in}{0.000000in}}%
\pgfusepath{stroke,fill}%
}%
\begin{pgfscope}%
\pgfsys@transformshift{0.800000in}{2.040413in}%
\pgfsys@useobject{currentmarker}{}%
\end{pgfscope}%
\end{pgfscope}%
\begin{pgfscope}%
\pgfsetbuttcap%
\pgfsetroundjoin%
\definecolor{currentfill}{rgb}{0.000000,0.000000,0.000000}%
\pgfsetfillcolor{currentfill}%
\pgfsetlinewidth{0.602250pt}%
\definecolor{currentstroke}{rgb}{0.000000,0.000000,0.000000}%
\pgfsetstrokecolor{currentstroke}%
\pgfsetdash{}{0pt}%
\pgfsys@defobject{currentmarker}{\pgfqpoint{-0.027778in}{0.000000in}}{\pgfqpoint{-0.000000in}{0.000000in}}{%
\pgfpathmoveto{\pgfqpoint{-0.000000in}{0.000000in}}%
\pgfpathlineto{\pgfqpoint{-0.027778in}{0.000000in}}%
\pgfusepath{stroke,fill}%
}%
\begin{pgfscope}%
\pgfsys@transformshift{0.800000in}{2.093732in}%
\pgfsys@useobject{currentmarker}{}%
\end{pgfscope}%
\end{pgfscope}%
\begin{pgfscope}%
\pgfsetbuttcap%
\pgfsetroundjoin%
\definecolor{currentfill}{rgb}{0.000000,0.000000,0.000000}%
\pgfsetfillcolor{currentfill}%
\pgfsetlinewidth{0.602250pt}%
\definecolor{currentstroke}{rgb}{0.000000,0.000000,0.000000}%
\pgfsetstrokecolor{currentstroke}%
\pgfsetdash{}{0pt}%
\pgfsys@defobject{currentmarker}{\pgfqpoint{-0.027778in}{0.000000in}}{\pgfqpoint{-0.000000in}{0.000000in}}{%
\pgfpathmoveto{\pgfqpoint{-0.000000in}{0.000000in}}%
\pgfpathlineto{\pgfqpoint{-0.027778in}{0.000000in}}%
\pgfusepath{stroke,fill}%
}%
\begin{pgfscope}%
\pgfsys@transformshift{0.800000in}{2.138812in}%
\pgfsys@useobject{currentmarker}{}%
\end{pgfscope}%
\end{pgfscope}%
\begin{pgfscope}%
\pgfsetbuttcap%
\pgfsetroundjoin%
\definecolor{currentfill}{rgb}{0.000000,0.000000,0.000000}%
\pgfsetfillcolor{currentfill}%
\pgfsetlinewidth{0.602250pt}%
\definecolor{currentstroke}{rgb}{0.000000,0.000000,0.000000}%
\pgfsetstrokecolor{currentstroke}%
\pgfsetdash{}{0pt}%
\pgfsys@defobject{currentmarker}{\pgfqpoint{-0.027778in}{0.000000in}}{\pgfqpoint{-0.000000in}{0.000000in}}{%
\pgfpathmoveto{\pgfqpoint{-0.000000in}{0.000000in}}%
\pgfpathlineto{\pgfqpoint{-0.027778in}{0.000000in}}%
\pgfusepath{stroke,fill}%
}%
\begin{pgfscope}%
\pgfsys@transformshift{0.800000in}{2.177862in}%
\pgfsys@useobject{currentmarker}{}%
\end{pgfscope}%
\end{pgfscope}%
\begin{pgfscope}%
\pgfsetbuttcap%
\pgfsetroundjoin%
\definecolor{currentfill}{rgb}{0.000000,0.000000,0.000000}%
\pgfsetfillcolor{currentfill}%
\pgfsetlinewidth{0.602250pt}%
\definecolor{currentstroke}{rgb}{0.000000,0.000000,0.000000}%
\pgfsetstrokecolor{currentstroke}%
\pgfsetdash{}{0pt}%
\pgfsys@defobject{currentmarker}{\pgfqpoint{-0.027778in}{0.000000in}}{\pgfqpoint{-0.000000in}{0.000000in}}{%
\pgfpathmoveto{\pgfqpoint{-0.000000in}{0.000000in}}%
\pgfpathlineto{\pgfqpoint{-0.027778in}{0.000000in}}%
\pgfusepath{stroke,fill}%
}%
\begin{pgfscope}%
\pgfsys@transformshift{0.800000in}{2.212306in}%
\pgfsys@useobject{currentmarker}{}%
\end{pgfscope}%
\end{pgfscope}%
\begin{pgfscope}%
\pgfsetbuttcap%
\pgfsetroundjoin%
\definecolor{currentfill}{rgb}{0.000000,0.000000,0.000000}%
\pgfsetfillcolor{currentfill}%
\pgfsetlinewidth{0.602250pt}%
\definecolor{currentstroke}{rgb}{0.000000,0.000000,0.000000}%
\pgfsetstrokecolor{currentstroke}%
\pgfsetdash{}{0pt}%
\pgfsys@defobject{currentmarker}{\pgfqpoint{-0.027778in}{0.000000in}}{\pgfqpoint{-0.000000in}{0.000000in}}{%
\pgfpathmoveto{\pgfqpoint{-0.000000in}{0.000000in}}%
\pgfpathlineto{\pgfqpoint{-0.027778in}{0.000000in}}%
\pgfusepath{stroke,fill}%
}%
\begin{pgfscope}%
\pgfsys@transformshift{0.800000in}{2.445823in}%
\pgfsys@useobject{currentmarker}{}%
\end{pgfscope}%
\end{pgfscope}%
\begin{pgfscope}%
\pgfsetbuttcap%
\pgfsetroundjoin%
\definecolor{currentfill}{rgb}{0.000000,0.000000,0.000000}%
\pgfsetfillcolor{currentfill}%
\pgfsetlinewidth{0.602250pt}%
\definecolor{currentstroke}{rgb}{0.000000,0.000000,0.000000}%
\pgfsetstrokecolor{currentstroke}%
\pgfsetdash{}{0pt}%
\pgfsys@defobject{currentmarker}{\pgfqpoint{-0.027778in}{0.000000in}}{\pgfqpoint{-0.000000in}{0.000000in}}{%
\pgfpathmoveto{\pgfqpoint{-0.000000in}{0.000000in}}%
\pgfpathlineto{\pgfqpoint{-0.027778in}{0.000000in}}%
\pgfusepath{stroke,fill}%
}%
\begin{pgfscope}%
\pgfsys@transformshift{0.800000in}{2.564398in}%
\pgfsys@useobject{currentmarker}{}%
\end{pgfscope}%
\end{pgfscope}%
\begin{pgfscope}%
\pgfsetbuttcap%
\pgfsetroundjoin%
\definecolor{currentfill}{rgb}{0.000000,0.000000,0.000000}%
\pgfsetfillcolor{currentfill}%
\pgfsetlinewidth{0.602250pt}%
\definecolor{currentstroke}{rgb}{0.000000,0.000000,0.000000}%
\pgfsetstrokecolor{currentstroke}%
\pgfsetdash{}{0pt}%
\pgfsys@defobject{currentmarker}{\pgfqpoint{-0.027778in}{0.000000in}}{\pgfqpoint{-0.000000in}{0.000000in}}{%
\pgfpathmoveto{\pgfqpoint{-0.000000in}{0.000000in}}%
\pgfpathlineto{\pgfqpoint{-0.027778in}{0.000000in}}%
\pgfusepath{stroke,fill}%
}%
\begin{pgfscope}%
\pgfsys@transformshift{0.800000in}{2.648528in}%
\pgfsys@useobject{currentmarker}{}%
\end{pgfscope}%
\end{pgfscope}%
\begin{pgfscope}%
\pgfsetbuttcap%
\pgfsetroundjoin%
\definecolor{currentfill}{rgb}{0.000000,0.000000,0.000000}%
\pgfsetfillcolor{currentfill}%
\pgfsetlinewidth{0.602250pt}%
\definecolor{currentstroke}{rgb}{0.000000,0.000000,0.000000}%
\pgfsetstrokecolor{currentstroke}%
\pgfsetdash{}{0pt}%
\pgfsys@defobject{currentmarker}{\pgfqpoint{-0.027778in}{0.000000in}}{\pgfqpoint{-0.000000in}{0.000000in}}{%
\pgfpathmoveto{\pgfqpoint{-0.000000in}{0.000000in}}%
\pgfpathlineto{\pgfqpoint{-0.027778in}{0.000000in}}%
\pgfusepath{stroke,fill}%
}%
\begin{pgfscope}%
\pgfsys@transformshift{0.800000in}{2.713785in}%
\pgfsys@useobject{currentmarker}{}%
\end{pgfscope}%
\end{pgfscope}%
\begin{pgfscope}%
\pgfsetbuttcap%
\pgfsetroundjoin%
\definecolor{currentfill}{rgb}{0.000000,0.000000,0.000000}%
\pgfsetfillcolor{currentfill}%
\pgfsetlinewidth{0.602250pt}%
\definecolor{currentstroke}{rgb}{0.000000,0.000000,0.000000}%
\pgfsetstrokecolor{currentstroke}%
\pgfsetdash{}{0pt}%
\pgfsys@defobject{currentmarker}{\pgfqpoint{-0.027778in}{0.000000in}}{\pgfqpoint{-0.000000in}{0.000000in}}{%
\pgfpathmoveto{\pgfqpoint{-0.000000in}{0.000000in}}%
\pgfpathlineto{\pgfqpoint{-0.027778in}{0.000000in}}%
\pgfusepath{stroke,fill}%
}%
\begin{pgfscope}%
\pgfsys@transformshift{0.800000in}{2.767103in}%
\pgfsys@useobject{currentmarker}{}%
\end{pgfscope}%
\end{pgfscope}%
\begin{pgfscope}%
\pgfsetbuttcap%
\pgfsetroundjoin%
\definecolor{currentfill}{rgb}{0.000000,0.000000,0.000000}%
\pgfsetfillcolor{currentfill}%
\pgfsetlinewidth{0.602250pt}%
\definecolor{currentstroke}{rgb}{0.000000,0.000000,0.000000}%
\pgfsetstrokecolor{currentstroke}%
\pgfsetdash{}{0pt}%
\pgfsys@defobject{currentmarker}{\pgfqpoint{-0.027778in}{0.000000in}}{\pgfqpoint{-0.000000in}{0.000000in}}{%
\pgfpathmoveto{\pgfqpoint{-0.000000in}{0.000000in}}%
\pgfpathlineto{\pgfqpoint{-0.027778in}{0.000000in}}%
\pgfusepath{stroke,fill}%
}%
\begin{pgfscope}%
\pgfsys@transformshift{0.800000in}{2.812183in}%
\pgfsys@useobject{currentmarker}{}%
\end{pgfscope}%
\end{pgfscope}%
\begin{pgfscope}%
\pgfsetbuttcap%
\pgfsetroundjoin%
\definecolor{currentfill}{rgb}{0.000000,0.000000,0.000000}%
\pgfsetfillcolor{currentfill}%
\pgfsetlinewidth{0.602250pt}%
\definecolor{currentstroke}{rgb}{0.000000,0.000000,0.000000}%
\pgfsetstrokecolor{currentstroke}%
\pgfsetdash{}{0pt}%
\pgfsys@defobject{currentmarker}{\pgfqpoint{-0.027778in}{0.000000in}}{\pgfqpoint{-0.000000in}{0.000000in}}{%
\pgfpathmoveto{\pgfqpoint{-0.000000in}{0.000000in}}%
\pgfpathlineto{\pgfqpoint{-0.027778in}{0.000000in}}%
\pgfusepath{stroke,fill}%
}%
\begin{pgfscope}%
\pgfsys@transformshift{0.800000in}{2.851233in}%
\pgfsys@useobject{currentmarker}{}%
\end{pgfscope}%
\end{pgfscope}%
\begin{pgfscope}%
\pgfsetbuttcap%
\pgfsetroundjoin%
\definecolor{currentfill}{rgb}{0.000000,0.000000,0.000000}%
\pgfsetfillcolor{currentfill}%
\pgfsetlinewidth{0.602250pt}%
\definecolor{currentstroke}{rgb}{0.000000,0.000000,0.000000}%
\pgfsetstrokecolor{currentstroke}%
\pgfsetdash{}{0pt}%
\pgfsys@defobject{currentmarker}{\pgfqpoint{-0.027778in}{0.000000in}}{\pgfqpoint{-0.000000in}{0.000000in}}{%
\pgfpathmoveto{\pgfqpoint{-0.000000in}{0.000000in}}%
\pgfpathlineto{\pgfqpoint{-0.027778in}{0.000000in}}%
\pgfusepath{stroke,fill}%
}%
\begin{pgfscope}%
\pgfsys@transformshift{0.800000in}{2.885678in}%
\pgfsys@useobject{currentmarker}{}%
\end{pgfscope}%
\end{pgfscope}%
\begin{pgfscope}%
\pgfsetbuttcap%
\pgfsetroundjoin%
\definecolor{currentfill}{rgb}{0.000000,0.000000,0.000000}%
\pgfsetfillcolor{currentfill}%
\pgfsetlinewidth{0.602250pt}%
\definecolor{currentstroke}{rgb}{0.000000,0.000000,0.000000}%
\pgfsetstrokecolor{currentstroke}%
\pgfsetdash{}{0pt}%
\pgfsys@defobject{currentmarker}{\pgfqpoint{-0.027778in}{0.000000in}}{\pgfqpoint{-0.000000in}{0.000000in}}{%
\pgfpathmoveto{\pgfqpoint{-0.000000in}{0.000000in}}%
\pgfpathlineto{\pgfqpoint{-0.027778in}{0.000000in}}%
\pgfusepath{stroke,fill}%
}%
\begin{pgfscope}%
\pgfsys@transformshift{0.800000in}{3.119195in}%
\pgfsys@useobject{currentmarker}{}%
\end{pgfscope}%
\end{pgfscope}%
\begin{pgfscope}%
\pgfsetbuttcap%
\pgfsetroundjoin%
\definecolor{currentfill}{rgb}{0.000000,0.000000,0.000000}%
\pgfsetfillcolor{currentfill}%
\pgfsetlinewidth{0.602250pt}%
\definecolor{currentstroke}{rgb}{0.000000,0.000000,0.000000}%
\pgfsetstrokecolor{currentstroke}%
\pgfsetdash{}{0pt}%
\pgfsys@defobject{currentmarker}{\pgfqpoint{-0.027778in}{0.000000in}}{\pgfqpoint{-0.000000in}{0.000000in}}{%
\pgfpathmoveto{\pgfqpoint{-0.000000in}{0.000000in}}%
\pgfpathlineto{\pgfqpoint{-0.027778in}{0.000000in}}%
\pgfusepath{stroke,fill}%
}%
\begin{pgfscope}%
\pgfsys@transformshift{0.800000in}{3.237770in}%
\pgfsys@useobject{currentmarker}{}%
\end{pgfscope}%
\end{pgfscope}%
\begin{pgfscope}%
\pgfsetbuttcap%
\pgfsetroundjoin%
\definecolor{currentfill}{rgb}{0.000000,0.000000,0.000000}%
\pgfsetfillcolor{currentfill}%
\pgfsetlinewidth{0.602250pt}%
\definecolor{currentstroke}{rgb}{0.000000,0.000000,0.000000}%
\pgfsetstrokecolor{currentstroke}%
\pgfsetdash{}{0pt}%
\pgfsys@defobject{currentmarker}{\pgfqpoint{-0.027778in}{0.000000in}}{\pgfqpoint{-0.000000in}{0.000000in}}{%
\pgfpathmoveto{\pgfqpoint{-0.000000in}{0.000000in}}%
\pgfpathlineto{\pgfqpoint{-0.027778in}{0.000000in}}%
\pgfusepath{stroke,fill}%
}%
\begin{pgfscope}%
\pgfsys@transformshift{0.800000in}{3.321900in}%
\pgfsys@useobject{currentmarker}{}%
\end{pgfscope}%
\end{pgfscope}%
\begin{pgfscope}%
\pgfsetbuttcap%
\pgfsetroundjoin%
\definecolor{currentfill}{rgb}{0.000000,0.000000,0.000000}%
\pgfsetfillcolor{currentfill}%
\pgfsetlinewidth{0.602250pt}%
\definecolor{currentstroke}{rgb}{0.000000,0.000000,0.000000}%
\pgfsetstrokecolor{currentstroke}%
\pgfsetdash{}{0pt}%
\pgfsys@defobject{currentmarker}{\pgfqpoint{-0.027778in}{0.000000in}}{\pgfqpoint{-0.000000in}{0.000000in}}{%
\pgfpathmoveto{\pgfqpoint{-0.000000in}{0.000000in}}%
\pgfpathlineto{\pgfqpoint{-0.027778in}{0.000000in}}%
\pgfusepath{stroke,fill}%
}%
\begin{pgfscope}%
\pgfsys@transformshift{0.800000in}{3.387156in}%
\pgfsys@useobject{currentmarker}{}%
\end{pgfscope}%
\end{pgfscope}%
\begin{pgfscope}%
\pgfsetbuttcap%
\pgfsetroundjoin%
\definecolor{currentfill}{rgb}{0.000000,0.000000,0.000000}%
\pgfsetfillcolor{currentfill}%
\pgfsetlinewidth{0.602250pt}%
\definecolor{currentstroke}{rgb}{0.000000,0.000000,0.000000}%
\pgfsetstrokecolor{currentstroke}%
\pgfsetdash{}{0pt}%
\pgfsys@defobject{currentmarker}{\pgfqpoint{-0.027778in}{0.000000in}}{\pgfqpoint{-0.000000in}{0.000000in}}{%
\pgfpathmoveto{\pgfqpoint{-0.000000in}{0.000000in}}%
\pgfpathlineto{\pgfqpoint{-0.027778in}{0.000000in}}%
\pgfusepath{stroke,fill}%
}%
\begin{pgfscope}%
\pgfsys@transformshift{0.800000in}{3.440475in}%
\pgfsys@useobject{currentmarker}{}%
\end{pgfscope}%
\end{pgfscope}%
\begin{pgfscope}%
\pgfsetbuttcap%
\pgfsetroundjoin%
\definecolor{currentfill}{rgb}{0.000000,0.000000,0.000000}%
\pgfsetfillcolor{currentfill}%
\pgfsetlinewidth{0.602250pt}%
\definecolor{currentstroke}{rgb}{0.000000,0.000000,0.000000}%
\pgfsetstrokecolor{currentstroke}%
\pgfsetdash{}{0pt}%
\pgfsys@defobject{currentmarker}{\pgfqpoint{-0.027778in}{0.000000in}}{\pgfqpoint{-0.000000in}{0.000000in}}{%
\pgfpathmoveto{\pgfqpoint{-0.000000in}{0.000000in}}%
\pgfpathlineto{\pgfqpoint{-0.027778in}{0.000000in}}%
\pgfusepath{stroke,fill}%
}%
\begin{pgfscope}%
\pgfsys@transformshift{0.800000in}{3.485555in}%
\pgfsys@useobject{currentmarker}{}%
\end{pgfscope}%
\end{pgfscope}%
\begin{pgfscope}%
\pgfsetbuttcap%
\pgfsetroundjoin%
\definecolor{currentfill}{rgb}{0.000000,0.000000,0.000000}%
\pgfsetfillcolor{currentfill}%
\pgfsetlinewidth{0.602250pt}%
\definecolor{currentstroke}{rgb}{0.000000,0.000000,0.000000}%
\pgfsetstrokecolor{currentstroke}%
\pgfsetdash{}{0pt}%
\pgfsys@defobject{currentmarker}{\pgfqpoint{-0.027778in}{0.000000in}}{\pgfqpoint{-0.000000in}{0.000000in}}{%
\pgfpathmoveto{\pgfqpoint{-0.000000in}{0.000000in}}%
\pgfpathlineto{\pgfqpoint{-0.027778in}{0.000000in}}%
\pgfusepath{stroke,fill}%
}%
\begin{pgfscope}%
\pgfsys@transformshift{0.800000in}{3.524605in}%
\pgfsys@useobject{currentmarker}{}%
\end{pgfscope}%
\end{pgfscope}%
\begin{pgfscope}%
\pgfsetbuttcap%
\pgfsetroundjoin%
\definecolor{currentfill}{rgb}{0.000000,0.000000,0.000000}%
\pgfsetfillcolor{currentfill}%
\pgfsetlinewidth{0.602250pt}%
\definecolor{currentstroke}{rgb}{0.000000,0.000000,0.000000}%
\pgfsetstrokecolor{currentstroke}%
\pgfsetdash{}{0pt}%
\pgfsys@defobject{currentmarker}{\pgfqpoint{-0.027778in}{0.000000in}}{\pgfqpoint{-0.000000in}{0.000000in}}{%
\pgfpathmoveto{\pgfqpoint{-0.000000in}{0.000000in}}%
\pgfpathlineto{\pgfqpoint{-0.027778in}{0.000000in}}%
\pgfusepath{stroke,fill}%
}%
\begin{pgfscope}%
\pgfsys@transformshift{0.800000in}{3.559050in}%
\pgfsys@useobject{currentmarker}{}%
\end{pgfscope}%
\end{pgfscope}%
\begin{pgfscope}%
\pgfsetbuttcap%
\pgfsetroundjoin%
\definecolor{currentfill}{rgb}{0.000000,0.000000,0.000000}%
\pgfsetfillcolor{currentfill}%
\pgfsetlinewidth{0.602250pt}%
\definecolor{currentstroke}{rgb}{0.000000,0.000000,0.000000}%
\pgfsetstrokecolor{currentstroke}%
\pgfsetdash{}{0pt}%
\pgfsys@defobject{currentmarker}{\pgfqpoint{-0.027778in}{0.000000in}}{\pgfqpoint{-0.000000in}{0.000000in}}{%
\pgfpathmoveto{\pgfqpoint{-0.000000in}{0.000000in}}%
\pgfpathlineto{\pgfqpoint{-0.027778in}{0.000000in}}%
\pgfusepath{stroke,fill}%
}%
\begin{pgfscope}%
\pgfsys@transformshift{0.800000in}{3.792567in}%
\pgfsys@useobject{currentmarker}{}%
\end{pgfscope}%
\end{pgfscope}%
\begin{pgfscope}%
\pgfsetbuttcap%
\pgfsetroundjoin%
\definecolor{currentfill}{rgb}{0.000000,0.000000,0.000000}%
\pgfsetfillcolor{currentfill}%
\pgfsetlinewidth{0.602250pt}%
\definecolor{currentstroke}{rgb}{0.000000,0.000000,0.000000}%
\pgfsetstrokecolor{currentstroke}%
\pgfsetdash{}{0pt}%
\pgfsys@defobject{currentmarker}{\pgfqpoint{-0.027778in}{0.000000in}}{\pgfqpoint{-0.000000in}{0.000000in}}{%
\pgfpathmoveto{\pgfqpoint{-0.000000in}{0.000000in}}%
\pgfpathlineto{\pgfqpoint{-0.027778in}{0.000000in}}%
\pgfusepath{stroke,fill}%
}%
\begin{pgfscope}%
\pgfsys@transformshift{0.800000in}{3.911141in}%
\pgfsys@useobject{currentmarker}{}%
\end{pgfscope}%
\end{pgfscope}%
\begin{pgfscope}%
\pgfsetbuttcap%
\pgfsetroundjoin%
\definecolor{currentfill}{rgb}{0.000000,0.000000,0.000000}%
\pgfsetfillcolor{currentfill}%
\pgfsetlinewidth{0.602250pt}%
\definecolor{currentstroke}{rgb}{0.000000,0.000000,0.000000}%
\pgfsetstrokecolor{currentstroke}%
\pgfsetdash{}{0pt}%
\pgfsys@defobject{currentmarker}{\pgfqpoint{-0.027778in}{0.000000in}}{\pgfqpoint{-0.000000in}{0.000000in}}{%
\pgfpathmoveto{\pgfqpoint{-0.000000in}{0.000000in}}%
\pgfpathlineto{\pgfqpoint{-0.027778in}{0.000000in}}%
\pgfusepath{stroke,fill}%
}%
\begin{pgfscope}%
\pgfsys@transformshift{0.800000in}{3.995272in}%
\pgfsys@useobject{currentmarker}{}%
\end{pgfscope}%
\end{pgfscope}%
\begin{pgfscope}%
\pgfsetbuttcap%
\pgfsetroundjoin%
\definecolor{currentfill}{rgb}{0.000000,0.000000,0.000000}%
\pgfsetfillcolor{currentfill}%
\pgfsetlinewidth{0.602250pt}%
\definecolor{currentstroke}{rgb}{0.000000,0.000000,0.000000}%
\pgfsetstrokecolor{currentstroke}%
\pgfsetdash{}{0pt}%
\pgfsys@defobject{currentmarker}{\pgfqpoint{-0.027778in}{0.000000in}}{\pgfqpoint{-0.000000in}{0.000000in}}{%
\pgfpathmoveto{\pgfqpoint{-0.000000in}{0.000000in}}%
\pgfpathlineto{\pgfqpoint{-0.027778in}{0.000000in}}%
\pgfusepath{stroke,fill}%
}%
\begin{pgfscope}%
\pgfsys@transformshift{0.800000in}{4.060528in}%
\pgfsys@useobject{currentmarker}{}%
\end{pgfscope}%
\end{pgfscope}%
\begin{pgfscope}%
\pgfsetbuttcap%
\pgfsetroundjoin%
\definecolor{currentfill}{rgb}{0.000000,0.000000,0.000000}%
\pgfsetfillcolor{currentfill}%
\pgfsetlinewidth{0.602250pt}%
\definecolor{currentstroke}{rgb}{0.000000,0.000000,0.000000}%
\pgfsetstrokecolor{currentstroke}%
\pgfsetdash{}{0pt}%
\pgfsys@defobject{currentmarker}{\pgfqpoint{-0.027778in}{0.000000in}}{\pgfqpoint{-0.000000in}{0.000000in}}{%
\pgfpathmoveto{\pgfqpoint{-0.000000in}{0.000000in}}%
\pgfpathlineto{\pgfqpoint{-0.027778in}{0.000000in}}%
\pgfusepath{stroke,fill}%
}%
\begin{pgfscope}%
\pgfsys@transformshift{0.800000in}{4.113846in}%
\pgfsys@useobject{currentmarker}{}%
\end{pgfscope}%
\end{pgfscope}%
\begin{pgfscope}%
\pgfsetbuttcap%
\pgfsetroundjoin%
\definecolor{currentfill}{rgb}{0.000000,0.000000,0.000000}%
\pgfsetfillcolor{currentfill}%
\pgfsetlinewidth{0.602250pt}%
\definecolor{currentstroke}{rgb}{0.000000,0.000000,0.000000}%
\pgfsetstrokecolor{currentstroke}%
\pgfsetdash{}{0pt}%
\pgfsys@defobject{currentmarker}{\pgfqpoint{-0.027778in}{0.000000in}}{\pgfqpoint{-0.000000in}{0.000000in}}{%
\pgfpathmoveto{\pgfqpoint{-0.000000in}{0.000000in}}%
\pgfpathlineto{\pgfqpoint{-0.027778in}{0.000000in}}%
\pgfusepath{stroke,fill}%
}%
\begin{pgfscope}%
\pgfsys@transformshift{0.800000in}{4.158927in}%
\pgfsys@useobject{currentmarker}{}%
\end{pgfscope}%
\end{pgfscope}%
\begin{pgfscope}%
\pgfsetbuttcap%
\pgfsetroundjoin%
\definecolor{currentfill}{rgb}{0.000000,0.000000,0.000000}%
\pgfsetfillcolor{currentfill}%
\pgfsetlinewidth{0.602250pt}%
\definecolor{currentstroke}{rgb}{0.000000,0.000000,0.000000}%
\pgfsetstrokecolor{currentstroke}%
\pgfsetdash{}{0pt}%
\pgfsys@defobject{currentmarker}{\pgfqpoint{-0.027778in}{0.000000in}}{\pgfqpoint{-0.000000in}{0.000000in}}{%
\pgfpathmoveto{\pgfqpoint{-0.000000in}{0.000000in}}%
\pgfpathlineto{\pgfqpoint{-0.027778in}{0.000000in}}%
\pgfusepath{stroke,fill}%
}%
\begin{pgfscope}%
\pgfsys@transformshift{0.800000in}{4.197977in}%
\pgfsys@useobject{currentmarker}{}%
\end{pgfscope}%
\end{pgfscope}%
\begin{pgfscope}%
\definecolor{textcolor}{rgb}{0.000000,0.000000,0.000000}%
\pgfsetstrokecolor{textcolor}%
\pgfsetfillcolor{textcolor}%
\pgftext[x=0.446026in,y=2.376000in,,bottom,rotate=90.000000]{\color{textcolor}\rmfamily\fontsize{10.000000}{12.000000}\selectfont COMPARISONS in log}%
\end{pgfscope}%
\begin{pgfscope}%
\pgfpathrectangle{\pgfqpoint{0.800000in}{0.528000in}}{\pgfqpoint{4.960000in}{3.696000in}}%
\pgfusepath{clip}%
\pgfsetrectcap%
\pgfsetroundjoin%
\pgfsetlinewidth{1.505625pt}%
\definecolor{currentstroke}{rgb}{0.121569,0.466667,0.705882}%
\pgfsetstrokecolor{currentstroke}%
\pgfsetdash{}{0pt}%
\pgfpathmoveto{\pgfqpoint{1.025455in}{0.977566in}}%
\pgfpathlineto{\pgfqpoint{1.071466in}{1.240883in}}%
\pgfpathlineto{\pgfqpoint{1.117477in}{1.365752in}}%
\pgfpathlineto{\pgfqpoint{1.163488in}{1.459702in}}%
\pgfpathlineto{\pgfqpoint{1.209499in}{1.530863in}}%
\pgfpathlineto{\pgfqpoint{1.255510in}{1.602784in}}%
\pgfpathlineto{\pgfqpoint{1.301521in}{1.655946in}}%
\pgfpathlineto{\pgfqpoint{1.347532in}{1.695029in}}%
\pgfpathlineto{\pgfqpoint{1.393544in}{1.735549in}}%
\pgfpathlineto{\pgfqpoint{1.439555in}{1.782696in}}%
\pgfpathlineto{\pgfqpoint{1.485566in}{1.802641in}}%
\pgfpathlineto{\pgfqpoint{1.531577in}{1.893792in}}%
\pgfpathlineto{\pgfqpoint{1.577588in}{1.860929in}}%
\pgfpathlineto{\pgfqpoint{1.623599in}{1.864974in}}%
\pgfpathlineto{\pgfqpoint{1.669610in}{1.899434in}}%
\pgfpathlineto{\pgfqpoint{1.715622in}{1.925722in}}%
\pgfpathlineto{\pgfqpoint{1.761633in}{1.943050in}}%
\pgfpathlineto{\pgfqpoint{1.807644in}{1.973661in}}%
\pgfpathlineto{\pgfqpoint{1.853655in}{1.990197in}}%
\pgfpathlineto{\pgfqpoint{1.899666in}{2.008507in}}%
\pgfpathlineto{\pgfqpoint{1.945677in}{2.038153in}}%
\pgfpathlineto{\pgfqpoint{1.991688in}{2.035684in}}%
\pgfpathlineto{\pgfqpoint{2.037699in}{2.056436in}}%
\pgfpathlineto{\pgfqpoint{2.083711in}{2.082007in}}%
\pgfpathlineto{\pgfqpoint{2.129722in}{2.067631in}}%
\pgfpathlineto{\pgfqpoint{2.175733in}{2.112838in}}%
\pgfpathlineto{\pgfqpoint{2.221744in}{2.107521in}}%
\pgfpathlineto{\pgfqpoint{2.267755in}{2.132810in}}%
\pgfpathlineto{\pgfqpoint{2.313766in}{2.128912in}}%
\pgfpathlineto{\pgfqpoint{2.359777in}{2.140490in}}%
\pgfpathlineto{\pgfqpoint{2.405788in}{2.151720in}}%
\pgfpathlineto{\pgfqpoint{2.451800in}{2.143396in}}%
\pgfpathlineto{\pgfqpoint{2.497811in}{2.209830in}}%
\pgfpathlineto{\pgfqpoint{2.543822in}{2.190594in}}%
\pgfpathlineto{\pgfqpoint{2.589833in}{2.209256in}}%
\pgfpathlineto{\pgfqpoint{2.635844in}{2.199623in}}%
\pgfpathlineto{\pgfqpoint{2.681855in}{2.227680in}}%
\pgfpathlineto{\pgfqpoint{2.727866in}{2.220692in}}%
\pgfpathlineto{\pgfqpoint{2.773878in}{2.227103in}}%
\pgfpathlineto{\pgfqpoint{2.819889in}{2.258074in}}%
\pgfpathlineto{\pgfqpoint{2.865900in}{2.245593in}}%
\pgfpathlineto{\pgfqpoint{2.911911in}{2.243451in}}%
\pgfpathlineto{\pgfqpoint{2.957922in}{2.256223in}}%
\pgfpathlineto{\pgfqpoint{3.003933in}{2.260456in}}%
\pgfpathlineto{\pgfqpoint{3.049944in}{2.299489in}}%
\pgfpathlineto{\pgfqpoint{3.095955in}{2.295524in}}%
\pgfpathlineto{\pgfqpoint{3.141967in}{2.298672in}}%
\pgfpathlineto{\pgfqpoint{3.187978in}{2.329811in}}%
\pgfpathlineto{\pgfqpoint{3.233989in}{2.330883in}}%
\pgfpathlineto{\pgfqpoint{3.280000in}{2.318742in}}%
\pgfpathlineto{\pgfqpoint{3.326011in}{2.329439in}}%
\pgfpathlineto{\pgfqpoint{3.372022in}{2.314725in}}%
\pgfpathlineto{\pgfqpoint{3.418033in}{2.322839in}}%
\pgfpathlineto{\pgfqpoint{3.464045in}{2.323284in}}%
\pgfpathlineto{\pgfqpoint{3.510056in}{2.358283in}}%
\pgfpathlineto{\pgfqpoint{3.556067in}{2.341140in}}%
\pgfpathlineto{\pgfqpoint{3.602078in}{2.347506in}}%
\pgfpathlineto{\pgfqpoint{3.648089in}{2.380141in}}%
\pgfpathlineto{\pgfqpoint{3.694100in}{2.364083in}}%
\pgfpathlineto{\pgfqpoint{3.740111in}{2.372990in}}%
\pgfpathlineto{\pgfqpoint{3.786122in}{2.373373in}}%
\pgfpathlineto{\pgfqpoint{3.832134in}{2.370748in}}%
\pgfpathlineto{\pgfqpoint{3.878145in}{2.386927in}}%
\pgfpathlineto{\pgfqpoint{3.924156in}{2.394112in}}%
\pgfpathlineto{\pgfqpoint{3.970167in}{2.438145in}}%
\pgfpathlineto{\pgfqpoint{4.016178in}{2.389385in}}%
\pgfpathlineto{\pgfqpoint{4.062189in}{2.428460in}}%
\pgfpathlineto{\pgfqpoint{4.108200in}{2.410794in}}%
\pgfpathlineto{\pgfqpoint{4.154212in}{2.411898in}}%
\pgfpathlineto{\pgfqpoint{4.200223in}{2.411330in}}%
\pgfpathlineto{\pgfqpoint{4.246234in}{2.423913in}}%
\pgfpathlineto{\pgfqpoint{4.292245in}{2.464802in}}%
\pgfpathlineto{\pgfqpoint{4.338256in}{2.427927in}}%
\pgfpathlineto{\pgfqpoint{4.384267in}{2.427310in}}%
\pgfpathlineto{\pgfqpoint{4.430278in}{2.443339in}}%
\pgfpathlineto{\pgfqpoint{4.476289in}{2.450402in}}%
\pgfpathlineto{\pgfqpoint{4.522301in}{2.456816in}}%
\pgfpathlineto{\pgfqpoint{4.568312in}{2.460776in}}%
\pgfpathlineto{\pgfqpoint{4.614323in}{2.467925in}}%
\pgfpathlineto{\pgfqpoint{4.660334in}{2.473889in}}%
\pgfpathlineto{\pgfqpoint{4.706345in}{2.467751in}}%
\pgfpathlineto{\pgfqpoint{4.752356in}{2.494284in}}%
\pgfpathlineto{\pgfqpoint{4.798367in}{2.481249in}}%
\pgfpathlineto{\pgfqpoint{4.844378in}{2.492127in}}%
\pgfpathlineto{\pgfqpoint{4.890390in}{2.466746in}}%
\pgfpathlineto{\pgfqpoint{4.936401in}{2.488203in}}%
\pgfpathlineto{\pgfqpoint{4.982412in}{2.504039in}}%
\pgfpathlineto{\pgfqpoint{5.028423in}{2.492265in}}%
\pgfpathlineto{\pgfqpoint{5.074434in}{2.497318in}}%
\pgfpathlineto{\pgfqpoint{5.120445in}{2.511918in}}%
\pgfpathlineto{\pgfqpoint{5.166456in}{2.492611in}}%
\pgfpathlineto{\pgfqpoint{5.212468in}{2.515925in}}%
\pgfpathlineto{\pgfqpoint{5.258479in}{2.537670in}}%
\pgfpathlineto{\pgfqpoint{5.304490in}{2.520522in}}%
\pgfpathlineto{\pgfqpoint{5.350501in}{2.532660in}}%
\pgfpathlineto{\pgfqpoint{5.396512in}{2.525987in}}%
\pgfpathlineto{\pgfqpoint{5.442523in}{2.529931in}}%
\pgfpathlineto{\pgfqpoint{5.488534in}{2.565278in}}%
\pgfpathlineto{\pgfqpoint{5.534545in}{2.543630in}}%
\pgfusepath{stroke}%
\end{pgfscope}%
\begin{pgfscope}%
\pgfpathrectangle{\pgfqpoint{0.800000in}{0.528000in}}{\pgfqpoint{4.960000in}{3.696000in}}%
\pgfusepath{clip}%
\pgfsetrectcap%
\pgfsetroundjoin%
\pgfsetlinewidth{1.505625pt}%
\definecolor{currentstroke}{rgb}{1.000000,0.498039,0.054902}%
\pgfsetstrokecolor{currentstroke}%
\pgfsetdash{}{0pt}%
\pgfpathmoveto{\pgfqpoint{1.025455in}{1.093619in}}%
\pgfpathlineto{\pgfqpoint{1.071466in}{1.338624in}}%
\pgfpathlineto{\pgfqpoint{1.117477in}{1.477431in}}%
\pgfpathlineto{\pgfqpoint{1.163488in}{1.575394in}}%
\pgfpathlineto{\pgfqpoint{1.209499in}{1.650871in}}%
\pgfpathlineto{\pgfqpoint{1.255510in}{1.714042in}}%
\pgfpathlineto{\pgfqpoint{1.301521in}{1.766708in}}%
\pgfpathlineto{\pgfqpoint{1.347532in}{1.810859in}}%
\pgfpathlineto{\pgfqpoint{1.393544in}{1.850110in}}%
\pgfpathlineto{\pgfqpoint{1.439555in}{1.885625in}}%
\pgfpathlineto{\pgfqpoint{1.485566in}{1.916934in}}%
\pgfpathlineto{\pgfqpoint{1.531577in}{1.947198in}}%
\pgfpathlineto{\pgfqpoint{1.577588in}{1.973816in}}%
\pgfpathlineto{\pgfqpoint{1.623599in}{1.999425in}}%
\pgfpathlineto{\pgfqpoint{1.669610in}{2.021951in}}%
\pgfpathlineto{\pgfqpoint{1.715622in}{2.042749in}}%
\pgfpathlineto{\pgfqpoint{1.761633in}{2.062524in}}%
\pgfpathlineto{\pgfqpoint{1.807644in}{2.081616in}}%
\pgfpathlineto{\pgfqpoint{1.853655in}{2.098858in}}%
\pgfpathlineto{\pgfqpoint{1.899666in}{2.115981in}}%
\pgfpathlineto{\pgfqpoint{1.945677in}{2.132532in}}%
\pgfpathlineto{\pgfqpoint{1.991688in}{2.148043in}}%
\pgfpathlineto{\pgfqpoint{2.037699in}{2.163403in}}%
\pgfpathlineto{\pgfqpoint{2.083711in}{2.177229in}}%
\pgfpathlineto{\pgfqpoint{2.129722in}{2.190926in}}%
\pgfpathlineto{\pgfqpoint{2.175733in}{2.203857in}}%
\pgfpathlineto{\pgfqpoint{2.221744in}{2.216555in}}%
\pgfpathlineto{\pgfqpoint{2.267755in}{2.228244in}}%
\pgfpathlineto{\pgfqpoint{2.313766in}{2.239774in}}%
\pgfpathlineto{\pgfqpoint{2.359777in}{2.250758in}}%
\pgfpathlineto{\pgfqpoint{2.405788in}{2.261620in}}%
\pgfpathlineto{\pgfqpoint{2.451800in}{2.272314in}}%
\pgfpathlineto{\pgfqpoint{2.497811in}{2.282274in}}%
\pgfpathlineto{\pgfqpoint{2.543822in}{2.292039in}}%
\pgfpathlineto{\pgfqpoint{2.589833in}{2.301517in}}%
\pgfpathlineto{\pgfqpoint{2.635844in}{2.310563in}}%
\pgfpathlineto{\pgfqpoint{2.681855in}{2.319856in}}%
\pgfpathlineto{\pgfqpoint{2.727866in}{2.328132in}}%
\pgfpathlineto{\pgfqpoint{2.773878in}{2.336573in}}%
\pgfpathlineto{\pgfqpoint{2.819889in}{2.344817in}}%
\pgfpathlineto{\pgfqpoint{2.865900in}{2.352619in}}%
\pgfpathlineto{\pgfqpoint{2.911911in}{2.360742in}}%
\pgfpathlineto{\pgfqpoint{2.957922in}{2.368589in}}%
\pgfpathlineto{\pgfqpoint{3.003933in}{2.376334in}}%
\pgfpathlineto{\pgfqpoint{3.049944in}{2.383645in}}%
\pgfpathlineto{\pgfqpoint{3.095955in}{2.391035in}}%
\pgfpathlineto{\pgfqpoint{3.141967in}{2.398360in}}%
\pgfpathlineto{\pgfqpoint{3.187978in}{2.405230in}}%
\pgfpathlineto{\pgfqpoint{3.233989in}{2.412192in}}%
\pgfpathlineto{\pgfqpoint{3.280000in}{2.418729in}}%
\pgfpathlineto{\pgfqpoint{3.326011in}{2.425161in}}%
\pgfpathlineto{\pgfqpoint{3.372022in}{2.431264in}}%
\pgfpathlineto{\pgfqpoint{3.418033in}{2.437745in}}%
\pgfpathlineto{\pgfqpoint{3.464045in}{2.443800in}}%
\pgfpathlineto{\pgfqpoint{3.510056in}{2.449810in}}%
\pgfpathlineto{\pgfqpoint{3.556067in}{2.455902in}}%
\pgfpathlineto{\pgfqpoint{3.602078in}{2.461463in}}%
\pgfpathlineto{\pgfqpoint{3.648089in}{2.467165in}}%
\pgfpathlineto{\pgfqpoint{3.694100in}{2.472859in}}%
\pgfpathlineto{\pgfqpoint{3.740111in}{2.478235in}}%
\pgfpathlineto{\pgfqpoint{3.786122in}{2.483621in}}%
\pgfpathlineto{\pgfqpoint{3.832134in}{2.489113in}}%
\pgfpathlineto{\pgfqpoint{3.878145in}{2.494196in}}%
\pgfpathlineto{\pgfqpoint{3.924156in}{2.499194in}}%
\pgfpathlineto{\pgfqpoint{3.970167in}{2.504372in}}%
\pgfpathlineto{\pgfqpoint{4.016178in}{2.509215in}}%
\pgfpathlineto{\pgfqpoint{4.062189in}{2.513998in}}%
\pgfpathlineto{\pgfqpoint{4.108200in}{2.518797in}}%
\pgfpathlineto{\pgfqpoint{4.154212in}{2.523687in}}%
\pgfpathlineto{\pgfqpoint{4.200223in}{2.528390in}}%
\pgfpathlineto{\pgfqpoint{4.246234in}{2.533087in}}%
\pgfpathlineto{\pgfqpoint{4.292245in}{2.537241in}}%
\pgfpathlineto{\pgfqpoint{4.338256in}{2.541872in}}%
\pgfpathlineto{\pgfqpoint{4.384267in}{2.546184in}}%
\pgfpathlineto{\pgfqpoint{4.430278in}{2.550476in}}%
\pgfpathlineto{\pgfqpoint{4.476289in}{2.554695in}}%
\pgfpathlineto{\pgfqpoint{4.522301in}{2.558978in}}%
\pgfpathlineto{\pgfqpoint{4.568312in}{2.563166in}}%
\pgfpathlineto{\pgfqpoint{4.614323in}{2.567193in}}%
\pgfpathlineto{\pgfqpoint{4.660334in}{2.571360in}}%
\pgfpathlineto{\pgfqpoint{4.706345in}{2.575062in}}%
\pgfpathlineto{\pgfqpoint{4.752356in}{2.579012in}}%
\pgfpathlineto{\pgfqpoint{4.798367in}{2.582989in}}%
\pgfpathlineto{\pgfqpoint{4.844378in}{2.587117in}}%
\pgfpathlineto{\pgfqpoint{4.890390in}{2.591062in}}%
\pgfpathlineto{\pgfqpoint{4.936401in}{2.594934in}}%
\pgfpathlineto{\pgfqpoint{4.982412in}{2.598811in}}%
\pgfpathlineto{\pgfqpoint{5.028423in}{2.602614in}}%
\pgfpathlineto{\pgfqpoint{5.074434in}{2.606268in}}%
\pgfpathlineto{\pgfqpoint{5.120445in}{2.609888in}}%
\pgfpathlineto{\pgfqpoint{5.166456in}{2.613570in}}%
\pgfpathlineto{\pgfqpoint{5.212468in}{2.617207in}}%
\pgfpathlineto{\pgfqpoint{5.258479in}{2.620642in}}%
\pgfpathlineto{\pgfqpoint{5.304490in}{2.624189in}}%
\pgfpathlineto{\pgfqpoint{5.350501in}{2.627756in}}%
\pgfpathlineto{\pgfqpoint{5.396512in}{2.631090in}}%
\pgfpathlineto{\pgfqpoint{5.442523in}{2.634540in}}%
\pgfpathlineto{\pgfqpoint{5.488534in}{2.638010in}}%
\pgfpathlineto{\pgfqpoint{5.534545in}{2.641224in}}%
\pgfusepath{stroke}%
\end{pgfscope}%
\begin{pgfscope}%
\pgfpathrectangle{\pgfqpoint{0.800000in}{0.528000in}}{\pgfqpoint{4.960000in}{3.696000in}}%
\pgfusepath{clip}%
\pgfsetrectcap%
\pgfsetroundjoin%
\pgfsetlinewidth{1.505625pt}%
\definecolor{currentstroke}{rgb}{0.172549,0.627451,0.172549}%
\pgfsetstrokecolor{currentstroke}%
\pgfsetdash{}{0pt}%
\pgfpathmoveto{\pgfqpoint{1.025455in}{1.105730in}}%
\pgfpathlineto{\pgfqpoint{1.071466in}{1.361193in}}%
\pgfpathlineto{\pgfqpoint{1.117477in}{1.505074in}}%
\pgfpathlineto{\pgfqpoint{1.163488in}{1.606392in}}%
\pgfpathlineto{\pgfqpoint{1.209499in}{1.684396in}}%
\pgfpathlineto{\pgfqpoint{1.255510in}{1.748036in}}%
\pgfpathlineto{\pgfqpoint{1.301521in}{1.802259in}}%
\pgfpathlineto{\pgfqpoint{1.347532in}{1.849009in}}%
\pgfpathlineto{\pgfqpoint{1.393544in}{1.889110in}}%
\pgfpathlineto{\pgfqpoint{1.439555in}{1.923550in}}%
\pgfpathlineto{\pgfqpoint{1.485566in}{1.956999in}}%
\pgfpathlineto{\pgfqpoint{1.531577in}{1.987013in}}%
\pgfpathlineto{\pgfqpoint{1.577588in}{2.014646in}}%
\pgfpathlineto{\pgfqpoint{1.623599in}{2.040267in}}%
\pgfpathlineto{\pgfqpoint{1.669610in}{2.063299in}}%
\pgfpathlineto{\pgfqpoint{1.715622in}{2.085436in}}%
\pgfpathlineto{\pgfqpoint{1.761633in}{2.105309in}}%
\pgfpathlineto{\pgfqpoint{1.807644in}{2.124668in}}%
\pgfpathlineto{\pgfqpoint{1.853655in}{2.142737in}}%
\pgfpathlineto{\pgfqpoint{1.899666in}{2.159677in}}%
\pgfpathlineto{\pgfqpoint{1.945677in}{2.176297in}}%
\pgfpathlineto{\pgfqpoint{1.991688in}{2.192032in}}%
\pgfpathlineto{\pgfqpoint{2.037699in}{2.207054in}}%
\pgfpathlineto{\pgfqpoint{2.083711in}{2.221628in}}%
\pgfpathlineto{\pgfqpoint{2.129722in}{2.235444in}}%
\pgfpathlineto{\pgfqpoint{2.175733in}{2.248559in}}%
\pgfpathlineto{\pgfqpoint{2.221744in}{2.261315in}}%
\pgfpathlineto{\pgfqpoint{2.267755in}{2.273643in}}%
\pgfpathlineto{\pgfqpoint{2.313766in}{2.284915in}}%
\pgfpathlineto{\pgfqpoint{2.359777in}{2.297123in}}%
\pgfpathlineto{\pgfqpoint{2.405788in}{2.307864in}}%
\pgfpathlineto{\pgfqpoint{2.451800in}{2.318184in}}%
\pgfpathlineto{\pgfqpoint{2.497811in}{2.328434in}}%
\pgfpathlineto{\pgfqpoint{2.543822in}{2.338356in}}%
\pgfpathlineto{\pgfqpoint{2.589833in}{2.348138in}}%
\pgfpathlineto{\pgfqpoint{2.635844in}{2.357085in}}%
\pgfpathlineto{\pgfqpoint{2.681855in}{2.366780in}}%
\pgfpathlineto{\pgfqpoint{2.727866in}{2.375130in}}%
\pgfpathlineto{\pgfqpoint{2.773878in}{2.383773in}}%
\pgfpathlineto{\pgfqpoint{2.819889in}{2.391822in}}%
\pgfpathlineto{\pgfqpoint{2.865900in}{2.400322in}}%
\pgfpathlineto{\pgfqpoint{2.911911in}{2.408443in}}%
\pgfpathlineto{\pgfqpoint{2.957922in}{2.415915in}}%
\pgfpathlineto{\pgfqpoint{3.003933in}{2.423969in}}%
\pgfpathlineto{\pgfqpoint{3.049944in}{2.431450in}}%
\pgfpathlineto{\pgfqpoint{3.095955in}{2.438677in}}%
\pgfpathlineto{\pgfqpoint{3.141967in}{2.445927in}}%
\pgfpathlineto{\pgfqpoint{3.187978in}{2.453036in}}%
\pgfpathlineto{\pgfqpoint{3.233989in}{2.459775in}}%
\pgfpathlineto{\pgfqpoint{3.280000in}{2.466709in}}%
\pgfpathlineto{\pgfqpoint{3.326011in}{2.473078in}}%
\pgfpathlineto{\pgfqpoint{3.372022in}{2.479826in}}%
\pgfpathlineto{\pgfqpoint{3.418033in}{2.486474in}}%
\pgfpathlineto{\pgfqpoint{3.464045in}{2.492376in}}%
\pgfpathlineto{\pgfqpoint{3.510056in}{2.498511in}}%
\pgfpathlineto{\pgfqpoint{3.556067in}{2.504444in}}%
\pgfpathlineto{\pgfqpoint{3.602078in}{2.510463in}}%
\pgfpathlineto{\pgfqpoint{3.648089in}{2.516101in}}%
\pgfpathlineto{\pgfqpoint{3.694100in}{2.521523in}}%
\pgfpathlineto{\pgfqpoint{3.740111in}{2.527538in}}%
\pgfpathlineto{\pgfqpoint{3.786122in}{2.532422in}}%
\pgfpathlineto{\pgfqpoint{3.832134in}{2.538266in}}%
\pgfpathlineto{\pgfqpoint{3.878145in}{2.543445in}}%
\pgfpathlineto{\pgfqpoint{3.924156in}{2.548631in}}%
\pgfpathlineto{\pgfqpoint{3.970167in}{2.553562in}}%
\pgfpathlineto{\pgfqpoint{4.016178in}{2.558820in}}%
\pgfpathlineto{\pgfqpoint{4.062189in}{2.563434in}}%
\pgfpathlineto{\pgfqpoint{4.108200in}{2.568542in}}%
\pgfpathlineto{\pgfqpoint{4.154212in}{2.573348in}}%
\pgfpathlineto{\pgfqpoint{4.200223in}{2.577937in}}%
\pgfpathlineto{\pgfqpoint{4.246234in}{2.582724in}}%
\pgfpathlineto{\pgfqpoint{4.292245in}{2.587148in}}%
\pgfpathlineto{\pgfqpoint{4.338256in}{2.591635in}}%
\pgfpathlineto{\pgfqpoint{4.384267in}{2.596179in}}%
\pgfpathlineto{\pgfqpoint{4.430278in}{2.600445in}}%
\pgfpathlineto{\pgfqpoint{4.476289in}{2.605024in}}%
\pgfpathlineto{\pgfqpoint{4.522301in}{2.608953in}}%
\pgfpathlineto{\pgfqpoint{4.568312in}{2.613406in}}%
\pgfpathlineto{\pgfqpoint{4.614323in}{2.617338in}}%
\pgfpathlineto{\pgfqpoint{4.660334in}{2.621630in}}%
\pgfpathlineto{\pgfqpoint{4.706345in}{2.625576in}}%
\pgfpathlineto{\pgfqpoint{4.752356in}{2.629592in}}%
\pgfpathlineto{\pgfqpoint{4.798367in}{2.633562in}}%
\pgfpathlineto{\pgfqpoint{4.844378in}{2.637533in}}%
\pgfpathlineto{\pgfqpoint{4.890390in}{2.641548in}}%
\pgfpathlineto{\pgfqpoint{4.936401in}{2.645563in}}%
\pgfpathlineto{\pgfqpoint{4.982412in}{2.649283in}}%
\pgfpathlineto{\pgfqpoint{5.028423in}{2.652917in}}%
\pgfpathlineto{\pgfqpoint{5.074434in}{2.656992in}}%
\pgfpathlineto{\pgfqpoint{5.120445in}{2.660318in}}%
\pgfpathlineto{\pgfqpoint{5.166456in}{2.664198in}}%
\pgfpathlineto{\pgfqpoint{5.212468in}{2.667728in}}%
\pgfpathlineto{\pgfqpoint{5.258479in}{2.671412in}}%
\pgfpathlineto{\pgfqpoint{5.304490in}{2.674959in}}%
\pgfpathlineto{\pgfqpoint{5.350501in}{2.678350in}}%
\pgfpathlineto{\pgfqpoint{5.396512in}{2.681828in}}%
\pgfpathlineto{\pgfqpoint{5.442523in}{2.685230in}}%
\pgfpathlineto{\pgfqpoint{5.488534in}{2.688857in}}%
\pgfpathlineto{\pgfqpoint{5.534545in}{2.692112in}}%
\pgfusepath{stroke}%
\end{pgfscope}%
\begin{pgfscope}%
\pgfpathrectangle{\pgfqpoint{0.800000in}{0.528000in}}{\pgfqpoint{4.960000in}{3.696000in}}%
\pgfusepath{clip}%
\pgfsetrectcap%
\pgfsetroundjoin%
\pgfsetlinewidth{1.505625pt}%
\definecolor{currentstroke}{rgb}{0.839216,0.152941,0.156863}%
\pgfsetstrokecolor{currentstroke}%
\pgfsetdash{}{0pt}%
\pgfpathmoveto{\pgfqpoint{1.025455in}{1.389006in}}%
\pgfpathlineto{\pgfqpoint{1.071466in}{1.762275in}}%
\pgfpathlineto{\pgfqpoint{1.117477in}{2.001830in}}%
\pgfpathlineto{\pgfqpoint{1.163488in}{2.170503in}}%
\pgfpathlineto{\pgfqpoint{1.209499in}{2.311895in}}%
\pgfpathlineto{\pgfqpoint{1.255510in}{2.410364in}}%
\pgfpathlineto{\pgfqpoint{1.301521in}{2.512184in}}%
\pgfpathlineto{\pgfqpoint{1.347532in}{2.579121in}}%
\pgfpathlineto{\pgfqpoint{1.393544in}{2.647830in}}%
\pgfpathlineto{\pgfqpoint{1.439555in}{2.709829in}}%
\pgfpathlineto{\pgfqpoint{1.485566in}{2.764701in}}%
\pgfpathlineto{\pgfqpoint{1.531577in}{2.826032in}}%
\pgfpathlineto{\pgfqpoint{1.577588in}{2.873940in}}%
\pgfpathlineto{\pgfqpoint{1.623599in}{2.913573in}}%
\pgfpathlineto{\pgfqpoint{1.669610in}{2.953127in}}%
\pgfpathlineto{\pgfqpoint{1.715622in}{2.999586in}}%
\pgfpathlineto{\pgfqpoint{1.761633in}{3.021852in}}%
\pgfpathlineto{\pgfqpoint{1.807644in}{3.054034in}}%
\pgfpathlineto{\pgfqpoint{1.853655in}{3.085514in}}%
\pgfpathlineto{\pgfqpoint{1.899666in}{3.118677in}}%
\pgfpathlineto{\pgfqpoint{1.945677in}{3.142035in}}%
\pgfpathlineto{\pgfqpoint{1.991688in}{3.173439in}}%
\pgfpathlineto{\pgfqpoint{2.037699in}{3.201595in}}%
\pgfpathlineto{\pgfqpoint{2.083711in}{3.230635in}}%
\pgfpathlineto{\pgfqpoint{2.129722in}{3.249080in}}%
\pgfpathlineto{\pgfqpoint{2.175733in}{3.276796in}}%
\pgfpathlineto{\pgfqpoint{2.221744in}{3.288953in}}%
\pgfpathlineto{\pgfqpoint{2.267755in}{3.317072in}}%
\pgfpathlineto{\pgfqpoint{2.313766in}{3.337826in}}%
\pgfpathlineto{\pgfqpoint{2.359777in}{3.358016in}}%
\pgfpathlineto{\pgfqpoint{2.405788in}{3.376221in}}%
\pgfpathlineto{\pgfqpoint{2.451800in}{3.390802in}}%
\pgfpathlineto{\pgfqpoint{2.497811in}{3.415931in}}%
\pgfpathlineto{\pgfqpoint{2.543822in}{3.432769in}}%
\pgfpathlineto{\pgfqpoint{2.589833in}{3.445618in}}%
\pgfpathlineto{\pgfqpoint{2.635844in}{3.465624in}}%
\pgfpathlineto{\pgfqpoint{2.681855in}{3.478755in}}%
\pgfpathlineto{\pgfqpoint{2.727866in}{3.496061in}}%
\pgfpathlineto{\pgfqpoint{2.773878in}{3.511945in}}%
\pgfpathlineto{\pgfqpoint{2.819889in}{3.526144in}}%
\pgfpathlineto{\pgfqpoint{2.865900in}{3.540067in}}%
\pgfpathlineto{\pgfqpoint{2.911911in}{3.552312in}}%
\pgfpathlineto{\pgfqpoint{2.957922in}{3.563935in}}%
\pgfpathlineto{\pgfqpoint{3.003933in}{3.580974in}}%
\pgfpathlineto{\pgfqpoint{3.049944in}{3.597044in}}%
\pgfpathlineto{\pgfqpoint{3.095955in}{3.608054in}}%
\pgfpathlineto{\pgfqpoint{3.141967in}{3.618427in}}%
\pgfpathlineto{\pgfqpoint{3.187978in}{3.634244in}}%
\pgfpathlineto{\pgfqpoint{3.233989in}{3.639407in}}%
\pgfpathlineto{\pgfqpoint{3.280000in}{3.655653in}}%
\pgfpathlineto{\pgfqpoint{3.326011in}{3.669745in}}%
\pgfpathlineto{\pgfqpoint{3.372022in}{3.683436in}}%
\pgfpathlineto{\pgfqpoint{3.418033in}{3.685679in}}%
\pgfpathlineto{\pgfqpoint{3.464045in}{3.701755in}}%
\pgfpathlineto{\pgfqpoint{3.510056in}{3.706215in}}%
\pgfpathlineto{\pgfqpoint{3.556067in}{3.718466in}}%
\pgfpathlineto{\pgfqpoint{3.602078in}{3.732379in}}%
\pgfpathlineto{\pgfqpoint{3.648089in}{3.742807in}}%
\pgfpathlineto{\pgfqpoint{3.694100in}{3.752460in}}%
\pgfpathlineto{\pgfqpoint{3.740111in}{3.760412in}}%
\pgfpathlineto{\pgfqpoint{3.786122in}{3.775248in}}%
\pgfpathlineto{\pgfqpoint{3.832134in}{3.779170in}}%
\pgfpathlineto{\pgfqpoint{3.878145in}{3.790587in}}%
\pgfpathlineto{\pgfqpoint{3.924156in}{3.795334in}}%
\pgfpathlineto{\pgfqpoint{3.970167in}{3.805558in}}%
\pgfpathlineto{\pgfqpoint{4.016178in}{3.815595in}}%
\pgfpathlineto{\pgfqpoint{4.062189in}{3.824948in}}%
\pgfpathlineto{\pgfqpoint{4.108200in}{3.836913in}}%
\pgfpathlineto{\pgfqpoint{4.108200in}{3.830988in}}%
\pgfpathlineto{\pgfqpoint{4.154212in}{3.843330in}}%
\pgfpathlineto{\pgfqpoint{4.200223in}{3.852606in}}%
\pgfpathlineto{\pgfqpoint{4.246234in}{3.860651in}}%
\pgfpathlineto{\pgfqpoint{4.292245in}{3.864456in}}%
\pgfpathlineto{\pgfqpoint{4.338256in}{3.875028in}}%
\pgfpathlineto{\pgfqpoint{4.384267in}{3.888467in}}%
\pgfpathlineto{\pgfqpoint{4.430278in}{3.892400in}}%
\pgfpathlineto{\pgfqpoint{4.476289in}{3.897540in}}%
\pgfpathlineto{\pgfqpoint{4.522301in}{3.905423in}}%
\pgfpathlineto{\pgfqpoint{4.568312in}{3.914122in}}%
\pgfpathlineto{\pgfqpoint{4.614323in}{3.923107in}}%
\pgfpathlineto{\pgfqpoint{4.660334in}{3.930948in}}%
\pgfpathlineto{\pgfqpoint{4.706345in}{3.937888in}}%
\pgfpathlineto{\pgfqpoint{4.752356in}{3.944930in}}%
\pgfpathlineto{\pgfqpoint{4.798367in}{3.952019in}}%
\pgfpathlineto{\pgfqpoint{4.844378in}{3.960954in}}%
\pgfpathlineto{\pgfqpoint{4.890390in}{3.966329in}}%
\pgfpathlineto{\pgfqpoint{4.936401in}{3.971727in}}%
\pgfpathlineto{\pgfqpoint{4.982412in}{3.979286in}}%
\pgfpathlineto{\pgfqpoint{5.028423in}{3.985457in}}%
\pgfpathlineto{\pgfqpoint{5.074434in}{3.992710in}}%
\pgfpathlineto{\pgfqpoint{5.120445in}{3.998850in}}%
\pgfpathlineto{\pgfqpoint{5.166456in}{4.004852in}}%
\pgfpathlineto{\pgfqpoint{5.212468in}{4.011093in}}%
\pgfpathlineto{\pgfqpoint{5.258479in}{4.020134in}}%
\pgfpathlineto{\pgfqpoint{5.304490in}{4.024085in}}%
\pgfpathlineto{\pgfqpoint{5.350501in}{4.031741in}}%
\pgfpathlineto{\pgfqpoint{5.396512in}{4.035263in}}%
\pgfpathlineto{\pgfqpoint{5.442523in}{4.042178in}}%
\pgfpathlineto{\pgfqpoint{5.488534in}{4.050059in}}%
\pgfpathlineto{\pgfqpoint{5.534545in}{4.056000in}}%
\pgfusepath{stroke}%
\end{pgfscope}%
\begin{pgfscope}%
\pgfpathrectangle{\pgfqpoint{0.800000in}{0.528000in}}{\pgfqpoint{4.960000in}{3.696000in}}%
\pgfusepath{clip}%
\pgfsetrectcap%
\pgfsetroundjoin%
\pgfsetlinewidth{1.505625pt}%
\definecolor{currentstroke}{rgb}{0.580392,0.403922,0.741176}%
\pgfsetstrokecolor{currentstroke}%
\pgfsetdash{}{0pt}%
\pgfpathmoveto{\pgfqpoint{1.025455in}{0.696000in}}%
\pgfpathlineto{\pgfqpoint{1.071466in}{0.980212in}}%
\pgfpathlineto{\pgfqpoint{1.117477in}{1.191357in}}%
\pgfpathlineto{\pgfqpoint{1.163488in}{1.284156in}}%
\pgfpathlineto{\pgfqpoint{1.209499in}{1.403507in}}%
\pgfpathlineto{\pgfqpoint{1.255510in}{1.483110in}}%
\pgfpathlineto{\pgfqpoint{1.301521in}{1.552149in}}%
\pgfpathlineto{\pgfqpoint{1.347532in}{1.591874in}}%
\pgfpathlineto{\pgfqpoint{1.393544in}{1.656468in}}%
\pgfpathlineto{\pgfqpoint{1.439555in}{1.717079in}}%
\pgfpathlineto{\pgfqpoint{1.485566in}{1.737813in}}%
\pgfpathlineto{\pgfqpoint{1.531577in}{1.791691in}}%
\pgfpathlineto{\pgfqpoint{1.577588in}{1.816888in}}%
\pgfpathlineto{\pgfqpoint{1.623599in}{1.851753in}}%
\pgfpathlineto{\pgfqpoint{1.669610in}{1.876559in}}%
\pgfpathlineto{\pgfqpoint{1.715622in}{1.913677in}}%
\pgfpathlineto{\pgfqpoint{1.761633in}{1.948859in}}%
\pgfpathlineto{\pgfqpoint{1.807644in}{1.969845in}}%
\pgfpathlineto{\pgfqpoint{1.853655in}{2.004528in}}%
\pgfpathlineto{\pgfqpoint{1.899666in}{2.018296in}}%
\pgfpathlineto{\pgfqpoint{1.945677in}{2.035030in}}%
\pgfpathlineto{\pgfqpoint{1.991688in}{2.049443in}}%
\pgfpathlineto{\pgfqpoint{2.037699in}{2.075481in}}%
\pgfpathlineto{\pgfqpoint{2.083711in}{2.095907in}}%
\pgfpathlineto{\pgfqpoint{2.129722in}{2.104864in}}%
\pgfpathlineto{\pgfqpoint{2.175733in}{2.129154in}}%
\pgfpathlineto{\pgfqpoint{2.221744in}{2.146587in}}%
\pgfpathlineto{\pgfqpoint{2.267755in}{2.168939in}}%
\pgfpathlineto{\pgfqpoint{2.313766in}{2.174390in}}%
\pgfpathlineto{\pgfqpoint{2.359777in}{2.187257in}}%
\pgfpathlineto{\pgfqpoint{2.405788in}{2.200009in}}%
\pgfpathlineto{\pgfqpoint{2.451800in}{2.220281in}}%
\pgfpathlineto{\pgfqpoint{2.497811in}{2.236565in}}%
\pgfpathlineto{\pgfqpoint{2.543822in}{2.245379in}}%
\pgfpathlineto{\pgfqpoint{2.589833in}{2.258537in}}%
\pgfpathlineto{\pgfqpoint{2.635844in}{2.269525in}}%
\pgfpathlineto{\pgfqpoint{2.681855in}{2.286321in}}%
\pgfpathlineto{\pgfqpoint{2.727866in}{2.285117in}}%
\pgfpathlineto{\pgfqpoint{2.773878in}{2.307808in}}%
\pgfpathlineto{\pgfqpoint{2.819889in}{2.324105in}}%
\pgfpathlineto{\pgfqpoint{2.865900in}{2.319862in}}%
\pgfpathlineto{\pgfqpoint{2.911911in}{2.339387in}}%
\pgfpathlineto{\pgfqpoint{2.957922in}{2.342068in}}%
\pgfpathlineto{\pgfqpoint{3.003933in}{2.356576in}}%
\pgfpathlineto{\pgfqpoint{3.049944in}{2.366112in}}%
\pgfpathlineto{\pgfqpoint{3.095955in}{2.377679in}}%
\pgfpathlineto{\pgfqpoint{3.141967in}{2.389846in}}%
\pgfpathlineto{\pgfqpoint{3.187978in}{2.399755in}}%
\pgfpathlineto{\pgfqpoint{3.233989in}{2.403712in}}%
\pgfpathlineto{\pgfqpoint{3.280000in}{2.413395in}}%
\pgfpathlineto{\pgfqpoint{3.326011in}{2.421458in}}%
\pgfpathlineto{\pgfqpoint{3.372022in}{2.428558in}}%
\pgfpathlineto{\pgfqpoint{3.418033in}{2.443165in}}%
\pgfpathlineto{\pgfqpoint{3.464045in}{2.447642in}}%
\pgfpathlineto{\pgfqpoint{3.510056in}{2.457464in}}%
\pgfpathlineto{\pgfqpoint{3.556067in}{2.466150in}}%
\pgfpathlineto{\pgfqpoint{3.602078in}{2.469972in}}%
\pgfpathlineto{\pgfqpoint{3.648089in}{2.475014in}}%
\pgfpathlineto{\pgfqpoint{3.694100in}{2.490076in}}%
\pgfpathlineto{\pgfqpoint{3.740111in}{2.494551in}}%
\pgfpathlineto{\pgfqpoint{3.786122in}{2.498827in}}%
\pgfpathlineto{\pgfqpoint{3.832134in}{2.509306in}}%
\pgfpathlineto{\pgfqpoint{3.878145in}{2.515549in}}%
\pgfpathlineto{\pgfqpoint{3.924156in}{2.519444in}}%
\pgfpathlineto{\pgfqpoint{3.970167in}{2.527570in}}%
\pgfpathlineto{\pgfqpoint{4.016178in}{2.532929in}}%
\pgfpathlineto{\pgfqpoint{4.062189in}{2.544082in}}%
\pgfpathlineto{\pgfqpoint{4.108200in}{2.548646in}}%
\pgfpathlineto{\pgfqpoint{4.154212in}{2.555798in}}%
\pgfpathlineto{\pgfqpoint{4.200223in}{2.561106in}}%
\pgfpathlineto{\pgfqpoint{4.246234in}{2.564881in}}%
\pgfpathlineto{\pgfqpoint{4.292245in}{2.570741in}}%
\pgfpathlineto{\pgfqpoint{4.338256in}{2.577850in}}%
\pgfpathlineto{\pgfqpoint{4.384267in}{2.582767in}}%
\pgfpathlineto{\pgfqpoint{4.430278in}{2.589849in}}%
\pgfpathlineto{\pgfqpoint{4.476289in}{2.592536in}}%
\pgfpathlineto{\pgfqpoint{4.522301in}{2.603708in}}%
\pgfpathlineto{\pgfqpoint{4.568312in}{2.608100in}}%
\pgfpathlineto{\pgfqpoint{4.614323in}{2.619162in}}%
\pgfpathlineto{\pgfqpoint{4.660334in}{2.621176in}}%
\pgfpathlineto{\pgfqpoint{4.706345in}{2.626132in}}%
\pgfpathlineto{\pgfqpoint{4.752356in}{2.628775in}}%
\pgfpathlineto{\pgfqpoint{4.798367in}{2.633289in}}%
\pgfpathlineto{\pgfqpoint{4.844378in}{2.640383in}}%
\pgfpathlineto{\pgfqpoint{4.890390in}{2.647267in}}%
\pgfpathlineto{\pgfqpoint{4.936401in}{2.654572in}}%
\pgfpathlineto{\pgfqpoint{4.982412in}{2.655989in}}%
\pgfpathlineto{\pgfqpoint{5.028423in}{2.667086in}}%
\pgfpathlineto{\pgfqpoint{5.074434in}{2.661485in}}%
\pgfpathlineto{\pgfqpoint{5.120445in}{2.670365in}}%
\pgfpathlineto{\pgfqpoint{5.166456in}{2.677797in}}%
\pgfpathlineto{\pgfqpoint{5.212468in}{2.682765in}}%
\pgfpathlineto{\pgfqpoint{5.258479in}{2.688128in}}%
\pgfpathlineto{\pgfqpoint{5.304490in}{2.690450in}}%
\pgfpathlineto{\pgfqpoint{5.350501in}{2.697945in}}%
\pgfpathlineto{\pgfqpoint{5.396512in}{2.697998in}}%
\pgfpathlineto{\pgfqpoint{5.442523in}{2.708991in}}%
\pgfpathlineto{\pgfqpoint{5.488534in}{2.713174in}}%
\pgfpathlineto{\pgfqpoint{5.534545in}{2.713049in}}%
\pgfusepath{stroke}%
\end{pgfscope}%
\begin{pgfscope}%
\pgfsetrectcap%
\pgfsetmiterjoin%
\pgfsetlinewidth{0.803000pt}%
\definecolor{currentstroke}{rgb}{0.000000,0.000000,0.000000}%
\pgfsetstrokecolor{currentstroke}%
\pgfsetdash{}{0pt}%
\pgfpathmoveto{\pgfqpoint{0.800000in}{0.528000in}}%
\pgfpathlineto{\pgfqpoint{0.800000in}{4.224000in}}%
\pgfusepath{stroke}%
\end{pgfscope}%
\begin{pgfscope}%
\pgfsetrectcap%
\pgfsetmiterjoin%
\pgfsetlinewidth{0.803000pt}%
\definecolor{currentstroke}{rgb}{0.000000,0.000000,0.000000}%
\pgfsetstrokecolor{currentstroke}%
\pgfsetdash{}{0pt}%
\pgfpathmoveto{\pgfqpoint{5.760000in}{0.528000in}}%
\pgfpathlineto{\pgfqpoint{5.760000in}{4.224000in}}%
\pgfusepath{stroke}%
\end{pgfscope}%
\begin{pgfscope}%
\pgfsetrectcap%
\pgfsetmiterjoin%
\pgfsetlinewidth{0.803000pt}%
\definecolor{currentstroke}{rgb}{0.000000,0.000000,0.000000}%
\pgfsetstrokecolor{currentstroke}%
\pgfsetdash{}{0pt}%
\pgfpathmoveto{\pgfqpoint{0.800000in}{0.528000in}}%
\pgfpathlineto{\pgfqpoint{5.760000in}{0.528000in}}%
\pgfusepath{stroke}%
\end{pgfscope}%
\begin{pgfscope}%
\pgfsetrectcap%
\pgfsetmiterjoin%
\pgfsetlinewidth{0.803000pt}%
\definecolor{currentstroke}{rgb}{0.000000,0.000000,0.000000}%
\pgfsetstrokecolor{currentstroke}%
\pgfsetdash{}{0pt}%
\pgfpathmoveto{\pgfqpoint{0.800000in}{4.224000in}}%
\pgfpathlineto{\pgfqpoint{5.760000in}{4.224000in}}%
\pgfusepath{stroke}%
\end{pgfscope}%
\begin{pgfscope}%
\pgfsetbuttcap%
\pgfsetmiterjoin%
\definecolor{currentfill}{rgb}{1.000000,1.000000,1.000000}%
\pgfsetfillcolor{currentfill}%
\pgfsetfillopacity{0.800000}%
\pgfsetlinewidth{1.003750pt}%
\definecolor{currentstroke}{rgb}{0.800000,0.800000,0.800000}%
\pgfsetstrokecolor{currentstroke}%
\pgfsetstrokeopacity{0.800000}%
\pgfsetdash{}{0pt}%
\pgfpathmoveto{\pgfqpoint{0.897222in}{3.144525in}}%
\pgfpathlineto{\pgfqpoint{1.739044in}{3.144525in}}%
\pgfpathquadraticcurveto{\pgfqpoint{1.766822in}{3.144525in}}{\pgfqpoint{1.766822in}{3.172303in}}%
\pgfpathlineto{\pgfqpoint{1.766822in}{4.126778in}}%
\pgfpathquadraticcurveto{\pgfqpoint{1.766822in}{4.154556in}}{\pgfqpoint{1.739044in}{4.154556in}}%
\pgfpathlineto{\pgfqpoint{0.897222in}{4.154556in}}%
\pgfpathquadraticcurveto{\pgfqpoint{0.869444in}{4.154556in}}{\pgfqpoint{0.869444in}{4.126778in}}%
\pgfpathlineto{\pgfqpoint{0.869444in}{3.172303in}}%
\pgfpathquadraticcurveto{\pgfqpoint{0.869444in}{3.144525in}}{\pgfqpoint{0.897222in}{3.144525in}}%
\pgfpathlineto{\pgfqpoint{0.897222in}{3.144525in}}%
\pgfpathclose%
\pgfusepath{stroke,fill}%
\end{pgfscope}%
\begin{pgfscope}%
\pgfsetrectcap%
\pgfsetroundjoin%
\pgfsetlinewidth{1.505625pt}%
\definecolor{currentstroke}{rgb}{0.121569,0.466667,0.705882}%
\pgfsetstrokecolor{currentstroke}%
\pgfsetdash{}{0pt}%
\pgfpathmoveto{\pgfqpoint{0.925000in}{4.050389in}}%
\pgfpathlineto{\pgfqpoint{1.063889in}{4.050389in}}%
\pgfpathlineto{\pgfqpoint{1.202778in}{4.050389in}}%
\pgfusepath{stroke}%
\end{pgfscope}%
\begin{pgfscope}%
\definecolor{textcolor}{rgb}{0.000000,0.000000,0.000000}%
\pgfsetstrokecolor{textcolor}%
\pgfsetfillcolor{textcolor}%
\pgftext[x=1.313889in,y=4.001778in,left,base]{\color{textcolor}\rmfamily\fontsize{10.000000}{12.000000}\selectfont quick}%
\end{pgfscope}%
\begin{pgfscope}%
\pgfsetrectcap%
\pgfsetroundjoin%
\pgfsetlinewidth{1.505625pt}%
\definecolor{currentstroke}{rgb}{1.000000,0.498039,0.054902}%
\pgfsetstrokecolor{currentstroke}%
\pgfsetdash{}{0pt}%
\pgfpathmoveto{\pgfqpoint{0.925000in}{3.856716in}}%
\pgfpathlineto{\pgfqpoint{1.063889in}{3.856716in}}%
\pgfpathlineto{\pgfqpoint{1.202778in}{3.856716in}}%
\pgfusepath{stroke}%
\end{pgfscope}%
\begin{pgfscope}%
\definecolor{textcolor}{rgb}{0.000000,0.000000,0.000000}%
\pgfsetstrokecolor{textcolor}%
\pgfsetfillcolor{textcolor}%
\pgftext[x=1.313889in,y=3.808105in,left,base]{\color{textcolor}\rmfamily\fontsize{10.000000}{12.000000}\selectfont merge}%
\end{pgfscope}%
\begin{pgfscope}%
\pgfsetrectcap%
\pgfsetroundjoin%
\pgfsetlinewidth{1.505625pt}%
\definecolor{currentstroke}{rgb}{0.172549,0.627451,0.172549}%
\pgfsetstrokecolor{currentstroke}%
\pgfsetdash{}{0pt}%
\pgfpathmoveto{\pgfqpoint{0.925000in}{3.663043in}}%
\pgfpathlineto{\pgfqpoint{1.063889in}{3.663043in}}%
\pgfpathlineto{\pgfqpoint{1.202778in}{3.663043in}}%
\pgfusepath{stroke}%
\end{pgfscope}%
\begin{pgfscope}%
\definecolor{textcolor}{rgb}{0.000000,0.000000,0.000000}%
\pgfsetstrokecolor{textcolor}%
\pgfsetfillcolor{textcolor}%
\pgftext[x=1.313889in,y=3.614432in,left,base]{\color{textcolor}\rmfamily\fontsize{10.000000}{12.000000}\selectfont heap}%
\end{pgfscope}%
\begin{pgfscope}%
\pgfsetrectcap%
\pgfsetroundjoin%
\pgfsetlinewidth{1.505625pt}%
\definecolor{currentstroke}{rgb}{0.839216,0.152941,0.156863}%
\pgfsetstrokecolor{currentstroke}%
\pgfsetdash{}{0pt}%
\pgfpathmoveto{\pgfqpoint{0.925000in}{3.469371in}}%
\pgfpathlineto{\pgfqpoint{1.063889in}{3.469371in}}%
\pgfpathlineto{\pgfqpoint{1.202778in}{3.469371in}}%
\pgfusepath{stroke}%
\end{pgfscope}%
\begin{pgfscope}%
\definecolor{textcolor}{rgb}{0.000000,0.000000,0.000000}%
\pgfsetstrokecolor{textcolor}%
\pgfsetfillcolor{textcolor}%
\pgftext[x=1.313889in,y=3.420759in,left,base]{\color{textcolor}\rmfamily\fontsize{10.000000}{12.000000}\selectfont insert}%
\end{pgfscope}%
\begin{pgfscope}%
\pgfsetrectcap%
\pgfsetroundjoin%
\pgfsetlinewidth{1.505625pt}%
\definecolor{currentstroke}{rgb}{0.580392,0.403922,0.741176}%
\pgfsetstrokecolor{currentstroke}%
\pgfsetdash{}{0pt}%
\pgfpathmoveto{\pgfqpoint{0.925000in}{3.275698in}}%
\pgfpathlineto{\pgfqpoint{1.063889in}{3.275698in}}%
\pgfpathlineto{\pgfqpoint{1.202778in}{3.275698in}}%
\pgfusepath{stroke}%
\end{pgfscope}%
\begin{pgfscope}%
\definecolor{textcolor}{rgb}{0.000000,0.000000,0.000000}%
\pgfsetstrokecolor{textcolor}%
\pgfsetfillcolor{textcolor}%
\pgftext[x=1.313889in,y=3.227087in,left,base]{\color{textcolor}\rmfamily\fontsize{10.000000}{12.000000}\selectfont bucket}%
\end{pgfscope}%
\end{pgfpicture}%
\makeatother%
\endgroup%

%% Creator: Matplotlib, PGF backend
%%
%% To include the figure in your LaTeX document, write
%%   \input{<filename>.pgf}
%%
%% Make sure the required packages are loaded in your preamble
%%   \usepackage{pgf}
%%
%% Also ensure that all the required font packages are loaded; for instance,
%% the lmodern package is sometimes necessary when using math font.
%%   \usepackage{lmodern}
%%
%% Figures using additional raster images can only be included by \input if
%% they are in the same directory as the main LaTeX file. For loading figures
%% from other directories you can use the `import` package
%%   \usepackage{import}
%%
%% and then include the figures with
%%   \import{<path to file>}{<filename>.pgf}
%%
%% Matplotlib used the following preamble
%%   
%%   \makeatletter\@ifpackageloaded{underscore}{}{\usepackage[strings]{underscore}}\makeatother
%%
\begingroup%
\makeatletter%
\begin{pgfpicture}%
\pgfpathrectangle{\pgfpointorigin}{\pgfqpoint{6.400000in}{4.800000in}}%
\pgfusepath{use as bounding box, clip}%
\begin{pgfscope}%
\pgfsetbuttcap%
\pgfsetmiterjoin%
\definecolor{currentfill}{rgb}{1.000000,1.000000,1.000000}%
\pgfsetfillcolor{currentfill}%
\pgfsetlinewidth{0.000000pt}%
\definecolor{currentstroke}{rgb}{1.000000,1.000000,1.000000}%
\pgfsetstrokecolor{currentstroke}%
\pgfsetdash{}{0pt}%
\pgfpathmoveto{\pgfqpoint{0.000000in}{0.000000in}}%
\pgfpathlineto{\pgfqpoint{6.400000in}{0.000000in}}%
\pgfpathlineto{\pgfqpoint{6.400000in}{4.800000in}}%
\pgfpathlineto{\pgfqpoint{0.000000in}{4.800000in}}%
\pgfpathlineto{\pgfqpoint{0.000000in}{0.000000in}}%
\pgfpathclose%
\pgfusepath{fill}%
\end{pgfscope}%
\begin{pgfscope}%
\pgfsetbuttcap%
\pgfsetmiterjoin%
\definecolor{currentfill}{rgb}{1.000000,1.000000,1.000000}%
\pgfsetfillcolor{currentfill}%
\pgfsetlinewidth{0.000000pt}%
\definecolor{currentstroke}{rgb}{0.000000,0.000000,0.000000}%
\pgfsetstrokecolor{currentstroke}%
\pgfsetstrokeopacity{0.000000}%
\pgfsetdash{}{0pt}%
\pgfpathmoveto{\pgfqpoint{0.800000in}{0.528000in}}%
\pgfpathlineto{\pgfqpoint{5.760000in}{0.528000in}}%
\pgfpathlineto{\pgfqpoint{5.760000in}{4.224000in}}%
\pgfpathlineto{\pgfqpoint{0.800000in}{4.224000in}}%
\pgfpathlineto{\pgfqpoint{0.800000in}{0.528000in}}%
\pgfpathclose%
\pgfusepath{fill}%
\end{pgfscope}%
\begin{pgfscope}%
\pgfsetbuttcap%
\pgfsetroundjoin%
\definecolor{currentfill}{rgb}{0.000000,0.000000,0.000000}%
\pgfsetfillcolor{currentfill}%
\pgfsetlinewidth{0.803000pt}%
\definecolor{currentstroke}{rgb}{0.000000,0.000000,0.000000}%
\pgfsetstrokecolor{currentstroke}%
\pgfsetdash{}{0pt}%
\pgfsys@defobject{currentmarker}{\pgfqpoint{0.000000in}{-0.048611in}}{\pgfqpoint{0.000000in}{0.000000in}}{%
\pgfpathmoveto{\pgfqpoint{0.000000in}{0.000000in}}%
\pgfpathlineto{\pgfqpoint{0.000000in}{-0.048611in}}%
\pgfusepath{stroke,fill}%
}%
\begin{pgfscope}%
\pgfsys@transformshift{0.979443in}{0.528000in}%
\pgfsys@useobject{currentmarker}{}%
\end{pgfscope}%
\end{pgfscope}%
\begin{pgfscope}%
\definecolor{textcolor}{rgb}{0.000000,0.000000,0.000000}%
\pgfsetstrokecolor{textcolor}%
\pgfsetfillcolor{textcolor}%
\pgftext[x=0.979443in,y=0.430778in,,top]{\color{textcolor}\rmfamily\fontsize{10.000000}{12.000000}\selectfont \(\displaystyle {0}\)}%
\end{pgfscope}%
\begin{pgfscope}%
\pgfsetbuttcap%
\pgfsetroundjoin%
\definecolor{currentfill}{rgb}{0.000000,0.000000,0.000000}%
\pgfsetfillcolor{currentfill}%
\pgfsetlinewidth{0.803000pt}%
\definecolor{currentstroke}{rgb}{0.000000,0.000000,0.000000}%
\pgfsetstrokecolor{currentstroke}%
\pgfsetdash{}{0pt}%
\pgfsys@defobject{currentmarker}{\pgfqpoint{0.000000in}{-0.048611in}}{\pgfqpoint{0.000000in}{0.000000in}}{%
\pgfpathmoveto{\pgfqpoint{0.000000in}{0.000000in}}%
\pgfpathlineto{\pgfqpoint{0.000000in}{-0.048611in}}%
\pgfusepath{stroke,fill}%
}%
\begin{pgfscope}%
\pgfsys@transformshift{1.899666in}{0.528000in}%
\pgfsys@useobject{currentmarker}{}%
\end{pgfscope}%
\end{pgfscope}%
\begin{pgfscope}%
\definecolor{textcolor}{rgb}{0.000000,0.000000,0.000000}%
\pgfsetstrokecolor{textcolor}%
\pgfsetfillcolor{textcolor}%
\pgftext[x=1.899666in,y=0.430778in,,top]{\color{textcolor}\rmfamily\fontsize{10.000000}{12.000000}\selectfont \(\displaystyle {2000}\)}%
\end{pgfscope}%
\begin{pgfscope}%
\pgfsetbuttcap%
\pgfsetroundjoin%
\definecolor{currentfill}{rgb}{0.000000,0.000000,0.000000}%
\pgfsetfillcolor{currentfill}%
\pgfsetlinewidth{0.803000pt}%
\definecolor{currentstroke}{rgb}{0.000000,0.000000,0.000000}%
\pgfsetstrokecolor{currentstroke}%
\pgfsetdash{}{0pt}%
\pgfsys@defobject{currentmarker}{\pgfqpoint{0.000000in}{-0.048611in}}{\pgfqpoint{0.000000in}{0.000000in}}{%
\pgfpathmoveto{\pgfqpoint{0.000000in}{0.000000in}}%
\pgfpathlineto{\pgfqpoint{0.000000in}{-0.048611in}}%
\pgfusepath{stroke,fill}%
}%
\begin{pgfscope}%
\pgfsys@transformshift{2.819889in}{0.528000in}%
\pgfsys@useobject{currentmarker}{}%
\end{pgfscope}%
\end{pgfscope}%
\begin{pgfscope}%
\definecolor{textcolor}{rgb}{0.000000,0.000000,0.000000}%
\pgfsetstrokecolor{textcolor}%
\pgfsetfillcolor{textcolor}%
\pgftext[x=2.819889in,y=0.430778in,,top]{\color{textcolor}\rmfamily\fontsize{10.000000}{12.000000}\selectfont \(\displaystyle {4000}\)}%
\end{pgfscope}%
\begin{pgfscope}%
\pgfsetbuttcap%
\pgfsetroundjoin%
\definecolor{currentfill}{rgb}{0.000000,0.000000,0.000000}%
\pgfsetfillcolor{currentfill}%
\pgfsetlinewidth{0.803000pt}%
\definecolor{currentstroke}{rgb}{0.000000,0.000000,0.000000}%
\pgfsetstrokecolor{currentstroke}%
\pgfsetdash{}{0pt}%
\pgfsys@defobject{currentmarker}{\pgfqpoint{0.000000in}{-0.048611in}}{\pgfqpoint{0.000000in}{0.000000in}}{%
\pgfpathmoveto{\pgfqpoint{0.000000in}{0.000000in}}%
\pgfpathlineto{\pgfqpoint{0.000000in}{-0.048611in}}%
\pgfusepath{stroke,fill}%
}%
\begin{pgfscope}%
\pgfsys@transformshift{3.740111in}{0.528000in}%
\pgfsys@useobject{currentmarker}{}%
\end{pgfscope}%
\end{pgfscope}%
\begin{pgfscope}%
\definecolor{textcolor}{rgb}{0.000000,0.000000,0.000000}%
\pgfsetstrokecolor{textcolor}%
\pgfsetfillcolor{textcolor}%
\pgftext[x=3.740111in,y=0.430778in,,top]{\color{textcolor}\rmfamily\fontsize{10.000000}{12.000000}\selectfont \(\displaystyle {6000}\)}%
\end{pgfscope}%
\begin{pgfscope}%
\pgfsetbuttcap%
\pgfsetroundjoin%
\definecolor{currentfill}{rgb}{0.000000,0.000000,0.000000}%
\pgfsetfillcolor{currentfill}%
\pgfsetlinewidth{0.803000pt}%
\definecolor{currentstroke}{rgb}{0.000000,0.000000,0.000000}%
\pgfsetstrokecolor{currentstroke}%
\pgfsetdash{}{0pt}%
\pgfsys@defobject{currentmarker}{\pgfqpoint{0.000000in}{-0.048611in}}{\pgfqpoint{0.000000in}{0.000000in}}{%
\pgfpathmoveto{\pgfqpoint{0.000000in}{0.000000in}}%
\pgfpathlineto{\pgfqpoint{0.000000in}{-0.048611in}}%
\pgfusepath{stroke,fill}%
}%
\begin{pgfscope}%
\pgfsys@transformshift{4.660334in}{0.528000in}%
\pgfsys@useobject{currentmarker}{}%
\end{pgfscope}%
\end{pgfscope}%
\begin{pgfscope}%
\definecolor{textcolor}{rgb}{0.000000,0.000000,0.000000}%
\pgfsetstrokecolor{textcolor}%
\pgfsetfillcolor{textcolor}%
\pgftext[x=4.660334in,y=0.430778in,,top]{\color{textcolor}\rmfamily\fontsize{10.000000}{12.000000}\selectfont \(\displaystyle {8000}\)}%
\end{pgfscope}%
\begin{pgfscope}%
\pgfsetbuttcap%
\pgfsetroundjoin%
\definecolor{currentfill}{rgb}{0.000000,0.000000,0.000000}%
\pgfsetfillcolor{currentfill}%
\pgfsetlinewidth{0.803000pt}%
\definecolor{currentstroke}{rgb}{0.000000,0.000000,0.000000}%
\pgfsetstrokecolor{currentstroke}%
\pgfsetdash{}{0pt}%
\pgfsys@defobject{currentmarker}{\pgfqpoint{0.000000in}{-0.048611in}}{\pgfqpoint{0.000000in}{0.000000in}}{%
\pgfpathmoveto{\pgfqpoint{0.000000in}{0.000000in}}%
\pgfpathlineto{\pgfqpoint{0.000000in}{-0.048611in}}%
\pgfusepath{stroke,fill}%
}%
\begin{pgfscope}%
\pgfsys@transformshift{5.580557in}{0.528000in}%
\pgfsys@useobject{currentmarker}{}%
\end{pgfscope}%
\end{pgfscope}%
\begin{pgfscope}%
\definecolor{textcolor}{rgb}{0.000000,0.000000,0.000000}%
\pgfsetstrokecolor{textcolor}%
\pgfsetfillcolor{textcolor}%
\pgftext[x=5.580557in,y=0.430778in,,top]{\color{textcolor}\rmfamily\fontsize{10.000000}{12.000000}\selectfont \(\displaystyle {10000}\)}%
\end{pgfscope}%
\begin{pgfscope}%
\definecolor{textcolor}{rgb}{0.000000,0.000000,0.000000}%
\pgfsetstrokecolor{textcolor}%
\pgfsetfillcolor{textcolor}%
\pgftext[x=3.280000in,y=0.251766in,,top]{\color{textcolor}\rmfamily\fontsize{10.000000}{12.000000}\selectfont Input Size}%
\end{pgfscope}%
\begin{pgfscope}%
\pgfsetbuttcap%
\pgfsetroundjoin%
\definecolor{currentfill}{rgb}{0.000000,0.000000,0.000000}%
\pgfsetfillcolor{currentfill}%
\pgfsetlinewidth{0.803000pt}%
\definecolor{currentstroke}{rgb}{0.000000,0.000000,0.000000}%
\pgfsetstrokecolor{currentstroke}%
\pgfsetdash{}{0pt}%
\pgfsys@defobject{currentmarker}{\pgfqpoint{-0.048611in}{0.000000in}}{\pgfqpoint{-0.000000in}{0.000000in}}{%
\pgfpathmoveto{\pgfqpoint{-0.000000in}{0.000000in}}%
\pgfpathlineto{\pgfqpoint{-0.048611in}{0.000000in}}%
\pgfusepath{stroke,fill}%
}%
\begin{pgfscope}%
\pgfsys@transformshift{0.800000in}{0.551214in}%
\pgfsys@useobject{currentmarker}{}%
\end{pgfscope}%
\end{pgfscope}%
\begin{pgfscope}%
\definecolor{textcolor}{rgb}{0.000000,0.000000,0.000000}%
\pgfsetstrokecolor{textcolor}%
\pgfsetfillcolor{textcolor}%
\pgftext[x=0.501581in, y=0.502989in, left, base]{\color{textcolor}\rmfamily\fontsize{10.000000}{12.000000}\selectfont \(\displaystyle {10^{2}}\)}%
\end{pgfscope}%
\begin{pgfscope}%
\pgfsetbuttcap%
\pgfsetroundjoin%
\definecolor{currentfill}{rgb}{0.000000,0.000000,0.000000}%
\pgfsetfillcolor{currentfill}%
\pgfsetlinewidth{0.803000pt}%
\definecolor{currentstroke}{rgb}{0.000000,0.000000,0.000000}%
\pgfsetstrokecolor{currentstroke}%
\pgfsetdash{}{0pt}%
\pgfsys@defobject{currentmarker}{\pgfqpoint{-0.048611in}{0.000000in}}{\pgfqpoint{-0.000000in}{0.000000in}}{%
\pgfpathmoveto{\pgfqpoint{-0.000000in}{0.000000in}}%
\pgfpathlineto{\pgfqpoint{-0.048611in}{0.000000in}}%
\pgfusepath{stroke,fill}%
}%
\begin{pgfscope}%
\pgfsys@transformshift{0.800000in}{1.201306in}%
\pgfsys@useobject{currentmarker}{}%
\end{pgfscope}%
\end{pgfscope}%
\begin{pgfscope}%
\definecolor{textcolor}{rgb}{0.000000,0.000000,0.000000}%
\pgfsetstrokecolor{textcolor}%
\pgfsetfillcolor{textcolor}%
\pgftext[x=0.501581in, y=1.153081in, left, base]{\color{textcolor}\rmfamily\fontsize{10.000000}{12.000000}\selectfont \(\displaystyle {10^{3}}\)}%
\end{pgfscope}%
\begin{pgfscope}%
\pgfsetbuttcap%
\pgfsetroundjoin%
\definecolor{currentfill}{rgb}{0.000000,0.000000,0.000000}%
\pgfsetfillcolor{currentfill}%
\pgfsetlinewidth{0.803000pt}%
\definecolor{currentstroke}{rgb}{0.000000,0.000000,0.000000}%
\pgfsetstrokecolor{currentstroke}%
\pgfsetdash{}{0pt}%
\pgfsys@defobject{currentmarker}{\pgfqpoint{-0.048611in}{0.000000in}}{\pgfqpoint{-0.000000in}{0.000000in}}{%
\pgfpathmoveto{\pgfqpoint{-0.000000in}{0.000000in}}%
\pgfpathlineto{\pgfqpoint{-0.048611in}{0.000000in}}%
\pgfusepath{stroke,fill}%
}%
\begin{pgfscope}%
\pgfsys@transformshift{0.800000in}{1.851398in}%
\pgfsys@useobject{currentmarker}{}%
\end{pgfscope}%
\end{pgfscope}%
\begin{pgfscope}%
\definecolor{textcolor}{rgb}{0.000000,0.000000,0.000000}%
\pgfsetstrokecolor{textcolor}%
\pgfsetfillcolor{textcolor}%
\pgftext[x=0.501581in, y=1.803173in, left, base]{\color{textcolor}\rmfamily\fontsize{10.000000}{12.000000}\selectfont \(\displaystyle {10^{4}}\)}%
\end{pgfscope}%
\begin{pgfscope}%
\pgfsetbuttcap%
\pgfsetroundjoin%
\definecolor{currentfill}{rgb}{0.000000,0.000000,0.000000}%
\pgfsetfillcolor{currentfill}%
\pgfsetlinewidth{0.803000pt}%
\definecolor{currentstroke}{rgb}{0.000000,0.000000,0.000000}%
\pgfsetstrokecolor{currentstroke}%
\pgfsetdash{}{0pt}%
\pgfsys@defobject{currentmarker}{\pgfqpoint{-0.048611in}{0.000000in}}{\pgfqpoint{-0.000000in}{0.000000in}}{%
\pgfpathmoveto{\pgfqpoint{-0.000000in}{0.000000in}}%
\pgfpathlineto{\pgfqpoint{-0.048611in}{0.000000in}}%
\pgfusepath{stroke,fill}%
}%
\begin{pgfscope}%
\pgfsys@transformshift{0.800000in}{2.501490in}%
\pgfsys@useobject{currentmarker}{}%
\end{pgfscope}%
\end{pgfscope}%
\begin{pgfscope}%
\definecolor{textcolor}{rgb}{0.000000,0.000000,0.000000}%
\pgfsetstrokecolor{textcolor}%
\pgfsetfillcolor{textcolor}%
\pgftext[x=0.501581in, y=2.453265in, left, base]{\color{textcolor}\rmfamily\fontsize{10.000000}{12.000000}\selectfont \(\displaystyle {10^{5}}\)}%
\end{pgfscope}%
\begin{pgfscope}%
\pgfsetbuttcap%
\pgfsetroundjoin%
\definecolor{currentfill}{rgb}{0.000000,0.000000,0.000000}%
\pgfsetfillcolor{currentfill}%
\pgfsetlinewidth{0.803000pt}%
\definecolor{currentstroke}{rgb}{0.000000,0.000000,0.000000}%
\pgfsetstrokecolor{currentstroke}%
\pgfsetdash{}{0pt}%
\pgfsys@defobject{currentmarker}{\pgfqpoint{-0.048611in}{0.000000in}}{\pgfqpoint{-0.000000in}{0.000000in}}{%
\pgfpathmoveto{\pgfqpoint{-0.000000in}{0.000000in}}%
\pgfpathlineto{\pgfqpoint{-0.048611in}{0.000000in}}%
\pgfusepath{stroke,fill}%
}%
\begin{pgfscope}%
\pgfsys@transformshift{0.800000in}{3.151582in}%
\pgfsys@useobject{currentmarker}{}%
\end{pgfscope}%
\end{pgfscope}%
\begin{pgfscope}%
\definecolor{textcolor}{rgb}{0.000000,0.000000,0.000000}%
\pgfsetstrokecolor{textcolor}%
\pgfsetfillcolor{textcolor}%
\pgftext[x=0.501581in, y=3.103357in, left, base]{\color{textcolor}\rmfamily\fontsize{10.000000}{12.000000}\selectfont \(\displaystyle {10^{6}}\)}%
\end{pgfscope}%
\begin{pgfscope}%
\pgfsetbuttcap%
\pgfsetroundjoin%
\definecolor{currentfill}{rgb}{0.000000,0.000000,0.000000}%
\pgfsetfillcolor{currentfill}%
\pgfsetlinewidth{0.803000pt}%
\definecolor{currentstroke}{rgb}{0.000000,0.000000,0.000000}%
\pgfsetstrokecolor{currentstroke}%
\pgfsetdash{}{0pt}%
\pgfsys@defobject{currentmarker}{\pgfqpoint{-0.048611in}{0.000000in}}{\pgfqpoint{-0.000000in}{0.000000in}}{%
\pgfpathmoveto{\pgfqpoint{-0.000000in}{0.000000in}}%
\pgfpathlineto{\pgfqpoint{-0.048611in}{0.000000in}}%
\pgfusepath{stroke,fill}%
}%
\begin{pgfscope}%
\pgfsys@transformshift{0.800000in}{3.801674in}%
\pgfsys@useobject{currentmarker}{}%
\end{pgfscope}%
\end{pgfscope}%
\begin{pgfscope}%
\definecolor{textcolor}{rgb}{0.000000,0.000000,0.000000}%
\pgfsetstrokecolor{textcolor}%
\pgfsetfillcolor{textcolor}%
\pgftext[x=0.501581in, y=3.753449in, left, base]{\color{textcolor}\rmfamily\fontsize{10.000000}{12.000000}\selectfont \(\displaystyle {10^{7}}\)}%
\end{pgfscope}%
\begin{pgfscope}%
\pgfsetbuttcap%
\pgfsetroundjoin%
\definecolor{currentfill}{rgb}{0.000000,0.000000,0.000000}%
\pgfsetfillcolor{currentfill}%
\pgfsetlinewidth{0.602250pt}%
\definecolor{currentstroke}{rgb}{0.000000,0.000000,0.000000}%
\pgfsetstrokecolor{currentstroke}%
\pgfsetdash{}{0pt}%
\pgfsys@defobject{currentmarker}{\pgfqpoint{-0.027778in}{0.000000in}}{\pgfqpoint{-0.000000in}{0.000000in}}{%
\pgfpathmoveto{\pgfqpoint{-0.000000in}{0.000000in}}%
\pgfpathlineto{\pgfqpoint{-0.027778in}{0.000000in}}%
\pgfusepath{stroke,fill}%
}%
\begin{pgfscope}%
\pgfsys@transformshift{0.800000in}{0.746911in}%
\pgfsys@useobject{currentmarker}{}%
\end{pgfscope}%
\end{pgfscope}%
\begin{pgfscope}%
\pgfsetbuttcap%
\pgfsetroundjoin%
\definecolor{currentfill}{rgb}{0.000000,0.000000,0.000000}%
\pgfsetfillcolor{currentfill}%
\pgfsetlinewidth{0.602250pt}%
\definecolor{currentstroke}{rgb}{0.000000,0.000000,0.000000}%
\pgfsetstrokecolor{currentstroke}%
\pgfsetdash{}{0pt}%
\pgfsys@defobject{currentmarker}{\pgfqpoint{-0.027778in}{0.000000in}}{\pgfqpoint{-0.000000in}{0.000000in}}{%
\pgfpathmoveto{\pgfqpoint{-0.000000in}{0.000000in}}%
\pgfpathlineto{\pgfqpoint{-0.027778in}{0.000000in}}%
\pgfusepath{stroke,fill}%
}%
\begin{pgfscope}%
\pgfsys@transformshift{0.800000in}{0.861387in}%
\pgfsys@useobject{currentmarker}{}%
\end{pgfscope}%
\end{pgfscope}%
\begin{pgfscope}%
\pgfsetbuttcap%
\pgfsetroundjoin%
\definecolor{currentfill}{rgb}{0.000000,0.000000,0.000000}%
\pgfsetfillcolor{currentfill}%
\pgfsetlinewidth{0.602250pt}%
\definecolor{currentstroke}{rgb}{0.000000,0.000000,0.000000}%
\pgfsetstrokecolor{currentstroke}%
\pgfsetdash{}{0pt}%
\pgfsys@defobject{currentmarker}{\pgfqpoint{-0.027778in}{0.000000in}}{\pgfqpoint{-0.000000in}{0.000000in}}{%
\pgfpathmoveto{\pgfqpoint{-0.000000in}{0.000000in}}%
\pgfpathlineto{\pgfqpoint{-0.027778in}{0.000000in}}%
\pgfusepath{stroke,fill}%
}%
\begin{pgfscope}%
\pgfsys@transformshift{0.800000in}{0.942608in}%
\pgfsys@useobject{currentmarker}{}%
\end{pgfscope}%
\end{pgfscope}%
\begin{pgfscope}%
\pgfsetbuttcap%
\pgfsetroundjoin%
\definecolor{currentfill}{rgb}{0.000000,0.000000,0.000000}%
\pgfsetfillcolor{currentfill}%
\pgfsetlinewidth{0.602250pt}%
\definecolor{currentstroke}{rgb}{0.000000,0.000000,0.000000}%
\pgfsetstrokecolor{currentstroke}%
\pgfsetdash{}{0pt}%
\pgfsys@defobject{currentmarker}{\pgfqpoint{-0.027778in}{0.000000in}}{\pgfqpoint{-0.000000in}{0.000000in}}{%
\pgfpathmoveto{\pgfqpoint{-0.000000in}{0.000000in}}%
\pgfpathlineto{\pgfqpoint{-0.027778in}{0.000000in}}%
\pgfusepath{stroke,fill}%
}%
\begin{pgfscope}%
\pgfsys@transformshift{0.800000in}{1.005609in}%
\pgfsys@useobject{currentmarker}{}%
\end{pgfscope}%
\end{pgfscope}%
\begin{pgfscope}%
\pgfsetbuttcap%
\pgfsetroundjoin%
\definecolor{currentfill}{rgb}{0.000000,0.000000,0.000000}%
\pgfsetfillcolor{currentfill}%
\pgfsetlinewidth{0.602250pt}%
\definecolor{currentstroke}{rgb}{0.000000,0.000000,0.000000}%
\pgfsetstrokecolor{currentstroke}%
\pgfsetdash{}{0pt}%
\pgfsys@defobject{currentmarker}{\pgfqpoint{-0.027778in}{0.000000in}}{\pgfqpoint{-0.000000in}{0.000000in}}{%
\pgfpathmoveto{\pgfqpoint{-0.000000in}{0.000000in}}%
\pgfpathlineto{\pgfqpoint{-0.027778in}{0.000000in}}%
\pgfusepath{stroke,fill}%
}%
\begin{pgfscope}%
\pgfsys@transformshift{0.800000in}{1.057084in}%
\pgfsys@useobject{currentmarker}{}%
\end{pgfscope}%
\end{pgfscope}%
\begin{pgfscope}%
\pgfsetbuttcap%
\pgfsetroundjoin%
\definecolor{currentfill}{rgb}{0.000000,0.000000,0.000000}%
\pgfsetfillcolor{currentfill}%
\pgfsetlinewidth{0.602250pt}%
\definecolor{currentstroke}{rgb}{0.000000,0.000000,0.000000}%
\pgfsetstrokecolor{currentstroke}%
\pgfsetdash{}{0pt}%
\pgfsys@defobject{currentmarker}{\pgfqpoint{-0.027778in}{0.000000in}}{\pgfqpoint{-0.000000in}{0.000000in}}{%
\pgfpathmoveto{\pgfqpoint{-0.000000in}{0.000000in}}%
\pgfpathlineto{\pgfqpoint{-0.027778in}{0.000000in}}%
\pgfusepath{stroke,fill}%
}%
\begin{pgfscope}%
\pgfsys@transformshift{0.800000in}{1.100605in}%
\pgfsys@useobject{currentmarker}{}%
\end{pgfscope}%
\end{pgfscope}%
\begin{pgfscope}%
\pgfsetbuttcap%
\pgfsetroundjoin%
\definecolor{currentfill}{rgb}{0.000000,0.000000,0.000000}%
\pgfsetfillcolor{currentfill}%
\pgfsetlinewidth{0.602250pt}%
\definecolor{currentstroke}{rgb}{0.000000,0.000000,0.000000}%
\pgfsetstrokecolor{currentstroke}%
\pgfsetdash{}{0pt}%
\pgfsys@defobject{currentmarker}{\pgfqpoint{-0.027778in}{0.000000in}}{\pgfqpoint{-0.000000in}{0.000000in}}{%
\pgfpathmoveto{\pgfqpoint{-0.000000in}{0.000000in}}%
\pgfpathlineto{\pgfqpoint{-0.027778in}{0.000000in}}%
\pgfusepath{stroke,fill}%
}%
\begin{pgfscope}%
\pgfsys@transformshift{0.800000in}{1.138305in}%
\pgfsys@useobject{currentmarker}{}%
\end{pgfscope}%
\end{pgfscope}%
\begin{pgfscope}%
\pgfsetbuttcap%
\pgfsetroundjoin%
\definecolor{currentfill}{rgb}{0.000000,0.000000,0.000000}%
\pgfsetfillcolor{currentfill}%
\pgfsetlinewidth{0.602250pt}%
\definecolor{currentstroke}{rgb}{0.000000,0.000000,0.000000}%
\pgfsetstrokecolor{currentstroke}%
\pgfsetdash{}{0pt}%
\pgfsys@defobject{currentmarker}{\pgfqpoint{-0.027778in}{0.000000in}}{\pgfqpoint{-0.000000in}{0.000000in}}{%
\pgfpathmoveto{\pgfqpoint{-0.000000in}{0.000000in}}%
\pgfpathlineto{\pgfqpoint{-0.027778in}{0.000000in}}%
\pgfusepath{stroke,fill}%
}%
\begin{pgfscope}%
\pgfsys@transformshift{0.800000in}{1.171559in}%
\pgfsys@useobject{currentmarker}{}%
\end{pgfscope}%
\end{pgfscope}%
\begin{pgfscope}%
\pgfsetbuttcap%
\pgfsetroundjoin%
\definecolor{currentfill}{rgb}{0.000000,0.000000,0.000000}%
\pgfsetfillcolor{currentfill}%
\pgfsetlinewidth{0.602250pt}%
\definecolor{currentstroke}{rgb}{0.000000,0.000000,0.000000}%
\pgfsetstrokecolor{currentstroke}%
\pgfsetdash{}{0pt}%
\pgfsys@defobject{currentmarker}{\pgfqpoint{-0.027778in}{0.000000in}}{\pgfqpoint{-0.000000in}{0.000000in}}{%
\pgfpathmoveto{\pgfqpoint{-0.000000in}{0.000000in}}%
\pgfpathlineto{\pgfqpoint{-0.027778in}{0.000000in}}%
\pgfusepath{stroke,fill}%
}%
\begin{pgfscope}%
\pgfsys@transformshift{0.800000in}{1.397003in}%
\pgfsys@useobject{currentmarker}{}%
\end{pgfscope}%
\end{pgfscope}%
\begin{pgfscope}%
\pgfsetbuttcap%
\pgfsetroundjoin%
\definecolor{currentfill}{rgb}{0.000000,0.000000,0.000000}%
\pgfsetfillcolor{currentfill}%
\pgfsetlinewidth{0.602250pt}%
\definecolor{currentstroke}{rgb}{0.000000,0.000000,0.000000}%
\pgfsetstrokecolor{currentstroke}%
\pgfsetdash{}{0pt}%
\pgfsys@defobject{currentmarker}{\pgfqpoint{-0.027778in}{0.000000in}}{\pgfqpoint{-0.000000in}{0.000000in}}{%
\pgfpathmoveto{\pgfqpoint{-0.000000in}{0.000000in}}%
\pgfpathlineto{\pgfqpoint{-0.027778in}{0.000000in}}%
\pgfusepath{stroke,fill}%
}%
\begin{pgfscope}%
\pgfsys@transformshift{0.800000in}{1.511479in}%
\pgfsys@useobject{currentmarker}{}%
\end{pgfscope}%
\end{pgfscope}%
\begin{pgfscope}%
\pgfsetbuttcap%
\pgfsetroundjoin%
\definecolor{currentfill}{rgb}{0.000000,0.000000,0.000000}%
\pgfsetfillcolor{currentfill}%
\pgfsetlinewidth{0.602250pt}%
\definecolor{currentstroke}{rgb}{0.000000,0.000000,0.000000}%
\pgfsetstrokecolor{currentstroke}%
\pgfsetdash{}{0pt}%
\pgfsys@defobject{currentmarker}{\pgfqpoint{-0.027778in}{0.000000in}}{\pgfqpoint{-0.000000in}{0.000000in}}{%
\pgfpathmoveto{\pgfqpoint{-0.000000in}{0.000000in}}%
\pgfpathlineto{\pgfqpoint{-0.027778in}{0.000000in}}%
\pgfusepath{stroke,fill}%
}%
\begin{pgfscope}%
\pgfsys@transformshift{0.800000in}{1.592700in}%
\pgfsys@useobject{currentmarker}{}%
\end{pgfscope}%
\end{pgfscope}%
\begin{pgfscope}%
\pgfsetbuttcap%
\pgfsetroundjoin%
\definecolor{currentfill}{rgb}{0.000000,0.000000,0.000000}%
\pgfsetfillcolor{currentfill}%
\pgfsetlinewidth{0.602250pt}%
\definecolor{currentstroke}{rgb}{0.000000,0.000000,0.000000}%
\pgfsetstrokecolor{currentstroke}%
\pgfsetdash{}{0pt}%
\pgfsys@defobject{currentmarker}{\pgfqpoint{-0.027778in}{0.000000in}}{\pgfqpoint{-0.000000in}{0.000000in}}{%
\pgfpathmoveto{\pgfqpoint{-0.000000in}{0.000000in}}%
\pgfpathlineto{\pgfqpoint{-0.027778in}{0.000000in}}%
\pgfusepath{stroke,fill}%
}%
\begin{pgfscope}%
\pgfsys@transformshift{0.800000in}{1.655701in}%
\pgfsys@useobject{currentmarker}{}%
\end{pgfscope}%
\end{pgfscope}%
\begin{pgfscope}%
\pgfsetbuttcap%
\pgfsetroundjoin%
\definecolor{currentfill}{rgb}{0.000000,0.000000,0.000000}%
\pgfsetfillcolor{currentfill}%
\pgfsetlinewidth{0.602250pt}%
\definecolor{currentstroke}{rgb}{0.000000,0.000000,0.000000}%
\pgfsetstrokecolor{currentstroke}%
\pgfsetdash{}{0pt}%
\pgfsys@defobject{currentmarker}{\pgfqpoint{-0.027778in}{0.000000in}}{\pgfqpoint{-0.000000in}{0.000000in}}{%
\pgfpathmoveto{\pgfqpoint{-0.000000in}{0.000000in}}%
\pgfpathlineto{\pgfqpoint{-0.027778in}{0.000000in}}%
\pgfusepath{stroke,fill}%
}%
\begin{pgfscope}%
\pgfsys@transformshift{0.800000in}{1.707176in}%
\pgfsys@useobject{currentmarker}{}%
\end{pgfscope}%
\end{pgfscope}%
\begin{pgfscope}%
\pgfsetbuttcap%
\pgfsetroundjoin%
\definecolor{currentfill}{rgb}{0.000000,0.000000,0.000000}%
\pgfsetfillcolor{currentfill}%
\pgfsetlinewidth{0.602250pt}%
\definecolor{currentstroke}{rgb}{0.000000,0.000000,0.000000}%
\pgfsetstrokecolor{currentstroke}%
\pgfsetdash{}{0pt}%
\pgfsys@defobject{currentmarker}{\pgfqpoint{-0.027778in}{0.000000in}}{\pgfqpoint{-0.000000in}{0.000000in}}{%
\pgfpathmoveto{\pgfqpoint{-0.000000in}{0.000000in}}%
\pgfpathlineto{\pgfqpoint{-0.027778in}{0.000000in}}%
\pgfusepath{stroke,fill}%
}%
\begin{pgfscope}%
\pgfsys@transformshift{0.800000in}{1.750697in}%
\pgfsys@useobject{currentmarker}{}%
\end{pgfscope}%
\end{pgfscope}%
\begin{pgfscope}%
\pgfsetbuttcap%
\pgfsetroundjoin%
\definecolor{currentfill}{rgb}{0.000000,0.000000,0.000000}%
\pgfsetfillcolor{currentfill}%
\pgfsetlinewidth{0.602250pt}%
\definecolor{currentstroke}{rgb}{0.000000,0.000000,0.000000}%
\pgfsetstrokecolor{currentstroke}%
\pgfsetdash{}{0pt}%
\pgfsys@defobject{currentmarker}{\pgfqpoint{-0.027778in}{0.000000in}}{\pgfqpoint{-0.000000in}{0.000000in}}{%
\pgfpathmoveto{\pgfqpoint{-0.000000in}{0.000000in}}%
\pgfpathlineto{\pgfqpoint{-0.027778in}{0.000000in}}%
\pgfusepath{stroke,fill}%
}%
\begin{pgfscope}%
\pgfsys@transformshift{0.800000in}{1.788397in}%
\pgfsys@useobject{currentmarker}{}%
\end{pgfscope}%
\end{pgfscope}%
\begin{pgfscope}%
\pgfsetbuttcap%
\pgfsetroundjoin%
\definecolor{currentfill}{rgb}{0.000000,0.000000,0.000000}%
\pgfsetfillcolor{currentfill}%
\pgfsetlinewidth{0.602250pt}%
\definecolor{currentstroke}{rgb}{0.000000,0.000000,0.000000}%
\pgfsetstrokecolor{currentstroke}%
\pgfsetdash{}{0pt}%
\pgfsys@defobject{currentmarker}{\pgfqpoint{-0.027778in}{0.000000in}}{\pgfqpoint{-0.000000in}{0.000000in}}{%
\pgfpathmoveto{\pgfqpoint{-0.000000in}{0.000000in}}%
\pgfpathlineto{\pgfqpoint{-0.027778in}{0.000000in}}%
\pgfusepath{stroke,fill}%
}%
\begin{pgfscope}%
\pgfsys@transformshift{0.800000in}{1.821651in}%
\pgfsys@useobject{currentmarker}{}%
\end{pgfscope}%
\end{pgfscope}%
\begin{pgfscope}%
\pgfsetbuttcap%
\pgfsetroundjoin%
\definecolor{currentfill}{rgb}{0.000000,0.000000,0.000000}%
\pgfsetfillcolor{currentfill}%
\pgfsetlinewidth{0.602250pt}%
\definecolor{currentstroke}{rgb}{0.000000,0.000000,0.000000}%
\pgfsetstrokecolor{currentstroke}%
\pgfsetdash{}{0pt}%
\pgfsys@defobject{currentmarker}{\pgfqpoint{-0.027778in}{0.000000in}}{\pgfqpoint{-0.000000in}{0.000000in}}{%
\pgfpathmoveto{\pgfqpoint{-0.000000in}{0.000000in}}%
\pgfpathlineto{\pgfqpoint{-0.027778in}{0.000000in}}%
\pgfusepath{stroke,fill}%
}%
\begin{pgfscope}%
\pgfsys@transformshift{0.800000in}{2.047095in}%
\pgfsys@useobject{currentmarker}{}%
\end{pgfscope}%
\end{pgfscope}%
\begin{pgfscope}%
\pgfsetbuttcap%
\pgfsetroundjoin%
\definecolor{currentfill}{rgb}{0.000000,0.000000,0.000000}%
\pgfsetfillcolor{currentfill}%
\pgfsetlinewidth{0.602250pt}%
\definecolor{currentstroke}{rgb}{0.000000,0.000000,0.000000}%
\pgfsetstrokecolor{currentstroke}%
\pgfsetdash{}{0pt}%
\pgfsys@defobject{currentmarker}{\pgfqpoint{-0.027778in}{0.000000in}}{\pgfqpoint{-0.000000in}{0.000000in}}{%
\pgfpathmoveto{\pgfqpoint{-0.000000in}{0.000000in}}%
\pgfpathlineto{\pgfqpoint{-0.027778in}{0.000000in}}%
\pgfusepath{stroke,fill}%
}%
\begin{pgfscope}%
\pgfsys@transformshift{0.800000in}{2.161571in}%
\pgfsys@useobject{currentmarker}{}%
\end{pgfscope}%
\end{pgfscope}%
\begin{pgfscope}%
\pgfsetbuttcap%
\pgfsetroundjoin%
\definecolor{currentfill}{rgb}{0.000000,0.000000,0.000000}%
\pgfsetfillcolor{currentfill}%
\pgfsetlinewidth{0.602250pt}%
\definecolor{currentstroke}{rgb}{0.000000,0.000000,0.000000}%
\pgfsetstrokecolor{currentstroke}%
\pgfsetdash{}{0pt}%
\pgfsys@defobject{currentmarker}{\pgfqpoint{-0.027778in}{0.000000in}}{\pgfqpoint{-0.000000in}{0.000000in}}{%
\pgfpathmoveto{\pgfqpoint{-0.000000in}{0.000000in}}%
\pgfpathlineto{\pgfqpoint{-0.027778in}{0.000000in}}%
\pgfusepath{stroke,fill}%
}%
\begin{pgfscope}%
\pgfsys@transformshift{0.800000in}{2.242792in}%
\pgfsys@useobject{currentmarker}{}%
\end{pgfscope}%
\end{pgfscope}%
\begin{pgfscope}%
\pgfsetbuttcap%
\pgfsetroundjoin%
\definecolor{currentfill}{rgb}{0.000000,0.000000,0.000000}%
\pgfsetfillcolor{currentfill}%
\pgfsetlinewidth{0.602250pt}%
\definecolor{currentstroke}{rgb}{0.000000,0.000000,0.000000}%
\pgfsetstrokecolor{currentstroke}%
\pgfsetdash{}{0pt}%
\pgfsys@defobject{currentmarker}{\pgfqpoint{-0.027778in}{0.000000in}}{\pgfqpoint{-0.000000in}{0.000000in}}{%
\pgfpathmoveto{\pgfqpoint{-0.000000in}{0.000000in}}%
\pgfpathlineto{\pgfqpoint{-0.027778in}{0.000000in}}%
\pgfusepath{stroke,fill}%
}%
\begin{pgfscope}%
\pgfsys@transformshift{0.800000in}{2.305793in}%
\pgfsys@useobject{currentmarker}{}%
\end{pgfscope}%
\end{pgfscope}%
\begin{pgfscope}%
\pgfsetbuttcap%
\pgfsetroundjoin%
\definecolor{currentfill}{rgb}{0.000000,0.000000,0.000000}%
\pgfsetfillcolor{currentfill}%
\pgfsetlinewidth{0.602250pt}%
\definecolor{currentstroke}{rgb}{0.000000,0.000000,0.000000}%
\pgfsetstrokecolor{currentstroke}%
\pgfsetdash{}{0pt}%
\pgfsys@defobject{currentmarker}{\pgfqpoint{-0.027778in}{0.000000in}}{\pgfqpoint{-0.000000in}{0.000000in}}{%
\pgfpathmoveto{\pgfqpoint{-0.000000in}{0.000000in}}%
\pgfpathlineto{\pgfqpoint{-0.027778in}{0.000000in}}%
\pgfusepath{stroke,fill}%
}%
\begin{pgfscope}%
\pgfsys@transformshift{0.800000in}{2.357268in}%
\pgfsys@useobject{currentmarker}{}%
\end{pgfscope}%
\end{pgfscope}%
\begin{pgfscope}%
\pgfsetbuttcap%
\pgfsetroundjoin%
\definecolor{currentfill}{rgb}{0.000000,0.000000,0.000000}%
\pgfsetfillcolor{currentfill}%
\pgfsetlinewidth{0.602250pt}%
\definecolor{currentstroke}{rgb}{0.000000,0.000000,0.000000}%
\pgfsetstrokecolor{currentstroke}%
\pgfsetdash{}{0pt}%
\pgfsys@defobject{currentmarker}{\pgfqpoint{-0.027778in}{0.000000in}}{\pgfqpoint{-0.000000in}{0.000000in}}{%
\pgfpathmoveto{\pgfqpoint{-0.000000in}{0.000000in}}%
\pgfpathlineto{\pgfqpoint{-0.027778in}{0.000000in}}%
\pgfusepath{stroke,fill}%
}%
\begin{pgfscope}%
\pgfsys@transformshift{0.800000in}{2.400789in}%
\pgfsys@useobject{currentmarker}{}%
\end{pgfscope}%
\end{pgfscope}%
\begin{pgfscope}%
\pgfsetbuttcap%
\pgfsetroundjoin%
\definecolor{currentfill}{rgb}{0.000000,0.000000,0.000000}%
\pgfsetfillcolor{currentfill}%
\pgfsetlinewidth{0.602250pt}%
\definecolor{currentstroke}{rgb}{0.000000,0.000000,0.000000}%
\pgfsetstrokecolor{currentstroke}%
\pgfsetdash{}{0pt}%
\pgfsys@defobject{currentmarker}{\pgfqpoint{-0.027778in}{0.000000in}}{\pgfqpoint{-0.000000in}{0.000000in}}{%
\pgfpathmoveto{\pgfqpoint{-0.000000in}{0.000000in}}%
\pgfpathlineto{\pgfqpoint{-0.027778in}{0.000000in}}%
\pgfusepath{stroke,fill}%
}%
\begin{pgfscope}%
\pgfsys@transformshift{0.800000in}{2.438489in}%
\pgfsys@useobject{currentmarker}{}%
\end{pgfscope}%
\end{pgfscope}%
\begin{pgfscope}%
\pgfsetbuttcap%
\pgfsetroundjoin%
\definecolor{currentfill}{rgb}{0.000000,0.000000,0.000000}%
\pgfsetfillcolor{currentfill}%
\pgfsetlinewidth{0.602250pt}%
\definecolor{currentstroke}{rgb}{0.000000,0.000000,0.000000}%
\pgfsetstrokecolor{currentstroke}%
\pgfsetdash{}{0pt}%
\pgfsys@defobject{currentmarker}{\pgfqpoint{-0.027778in}{0.000000in}}{\pgfqpoint{-0.000000in}{0.000000in}}{%
\pgfpathmoveto{\pgfqpoint{-0.000000in}{0.000000in}}%
\pgfpathlineto{\pgfqpoint{-0.027778in}{0.000000in}}%
\pgfusepath{stroke,fill}%
}%
\begin{pgfscope}%
\pgfsys@transformshift{0.800000in}{2.471743in}%
\pgfsys@useobject{currentmarker}{}%
\end{pgfscope}%
\end{pgfscope}%
\begin{pgfscope}%
\pgfsetbuttcap%
\pgfsetroundjoin%
\definecolor{currentfill}{rgb}{0.000000,0.000000,0.000000}%
\pgfsetfillcolor{currentfill}%
\pgfsetlinewidth{0.602250pt}%
\definecolor{currentstroke}{rgb}{0.000000,0.000000,0.000000}%
\pgfsetstrokecolor{currentstroke}%
\pgfsetdash{}{0pt}%
\pgfsys@defobject{currentmarker}{\pgfqpoint{-0.027778in}{0.000000in}}{\pgfqpoint{-0.000000in}{0.000000in}}{%
\pgfpathmoveto{\pgfqpoint{-0.000000in}{0.000000in}}%
\pgfpathlineto{\pgfqpoint{-0.027778in}{0.000000in}}%
\pgfusepath{stroke,fill}%
}%
\begin{pgfscope}%
\pgfsys@transformshift{0.800000in}{2.697187in}%
\pgfsys@useobject{currentmarker}{}%
\end{pgfscope}%
\end{pgfscope}%
\begin{pgfscope}%
\pgfsetbuttcap%
\pgfsetroundjoin%
\definecolor{currentfill}{rgb}{0.000000,0.000000,0.000000}%
\pgfsetfillcolor{currentfill}%
\pgfsetlinewidth{0.602250pt}%
\definecolor{currentstroke}{rgb}{0.000000,0.000000,0.000000}%
\pgfsetstrokecolor{currentstroke}%
\pgfsetdash{}{0pt}%
\pgfsys@defobject{currentmarker}{\pgfqpoint{-0.027778in}{0.000000in}}{\pgfqpoint{-0.000000in}{0.000000in}}{%
\pgfpathmoveto{\pgfqpoint{-0.000000in}{0.000000in}}%
\pgfpathlineto{\pgfqpoint{-0.027778in}{0.000000in}}%
\pgfusepath{stroke,fill}%
}%
\begin{pgfscope}%
\pgfsys@transformshift{0.800000in}{2.811663in}%
\pgfsys@useobject{currentmarker}{}%
\end{pgfscope}%
\end{pgfscope}%
\begin{pgfscope}%
\pgfsetbuttcap%
\pgfsetroundjoin%
\definecolor{currentfill}{rgb}{0.000000,0.000000,0.000000}%
\pgfsetfillcolor{currentfill}%
\pgfsetlinewidth{0.602250pt}%
\definecolor{currentstroke}{rgb}{0.000000,0.000000,0.000000}%
\pgfsetstrokecolor{currentstroke}%
\pgfsetdash{}{0pt}%
\pgfsys@defobject{currentmarker}{\pgfqpoint{-0.027778in}{0.000000in}}{\pgfqpoint{-0.000000in}{0.000000in}}{%
\pgfpathmoveto{\pgfqpoint{-0.000000in}{0.000000in}}%
\pgfpathlineto{\pgfqpoint{-0.027778in}{0.000000in}}%
\pgfusepath{stroke,fill}%
}%
\begin{pgfscope}%
\pgfsys@transformshift{0.800000in}{2.892884in}%
\pgfsys@useobject{currentmarker}{}%
\end{pgfscope}%
\end{pgfscope}%
\begin{pgfscope}%
\pgfsetbuttcap%
\pgfsetroundjoin%
\definecolor{currentfill}{rgb}{0.000000,0.000000,0.000000}%
\pgfsetfillcolor{currentfill}%
\pgfsetlinewidth{0.602250pt}%
\definecolor{currentstroke}{rgb}{0.000000,0.000000,0.000000}%
\pgfsetstrokecolor{currentstroke}%
\pgfsetdash{}{0pt}%
\pgfsys@defobject{currentmarker}{\pgfqpoint{-0.027778in}{0.000000in}}{\pgfqpoint{-0.000000in}{0.000000in}}{%
\pgfpathmoveto{\pgfqpoint{-0.000000in}{0.000000in}}%
\pgfpathlineto{\pgfqpoint{-0.027778in}{0.000000in}}%
\pgfusepath{stroke,fill}%
}%
\begin{pgfscope}%
\pgfsys@transformshift{0.800000in}{2.955885in}%
\pgfsys@useobject{currentmarker}{}%
\end{pgfscope}%
\end{pgfscope}%
\begin{pgfscope}%
\pgfsetbuttcap%
\pgfsetroundjoin%
\definecolor{currentfill}{rgb}{0.000000,0.000000,0.000000}%
\pgfsetfillcolor{currentfill}%
\pgfsetlinewidth{0.602250pt}%
\definecolor{currentstroke}{rgb}{0.000000,0.000000,0.000000}%
\pgfsetstrokecolor{currentstroke}%
\pgfsetdash{}{0pt}%
\pgfsys@defobject{currentmarker}{\pgfqpoint{-0.027778in}{0.000000in}}{\pgfqpoint{-0.000000in}{0.000000in}}{%
\pgfpathmoveto{\pgfqpoint{-0.000000in}{0.000000in}}%
\pgfpathlineto{\pgfqpoint{-0.027778in}{0.000000in}}%
\pgfusepath{stroke,fill}%
}%
\begin{pgfscope}%
\pgfsys@transformshift{0.800000in}{3.007360in}%
\pgfsys@useobject{currentmarker}{}%
\end{pgfscope}%
\end{pgfscope}%
\begin{pgfscope}%
\pgfsetbuttcap%
\pgfsetroundjoin%
\definecolor{currentfill}{rgb}{0.000000,0.000000,0.000000}%
\pgfsetfillcolor{currentfill}%
\pgfsetlinewidth{0.602250pt}%
\definecolor{currentstroke}{rgb}{0.000000,0.000000,0.000000}%
\pgfsetstrokecolor{currentstroke}%
\pgfsetdash{}{0pt}%
\pgfsys@defobject{currentmarker}{\pgfqpoint{-0.027778in}{0.000000in}}{\pgfqpoint{-0.000000in}{0.000000in}}{%
\pgfpathmoveto{\pgfqpoint{-0.000000in}{0.000000in}}%
\pgfpathlineto{\pgfqpoint{-0.027778in}{0.000000in}}%
\pgfusepath{stroke,fill}%
}%
\begin{pgfscope}%
\pgfsys@transformshift{0.800000in}{3.050881in}%
\pgfsys@useobject{currentmarker}{}%
\end{pgfscope}%
\end{pgfscope}%
\begin{pgfscope}%
\pgfsetbuttcap%
\pgfsetroundjoin%
\definecolor{currentfill}{rgb}{0.000000,0.000000,0.000000}%
\pgfsetfillcolor{currentfill}%
\pgfsetlinewidth{0.602250pt}%
\definecolor{currentstroke}{rgb}{0.000000,0.000000,0.000000}%
\pgfsetstrokecolor{currentstroke}%
\pgfsetdash{}{0pt}%
\pgfsys@defobject{currentmarker}{\pgfqpoint{-0.027778in}{0.000000in}}{\pgfqpoint{-0.000000in}{0.000000in}}{%
\pgfpathmoveto{\pgfqpoint{-0.000000in}{0.000000in}}%
\pgfpathlineto{\pgfqpoint{-0.027778in}{0.000000in}}%
\pgfusepath{stroke,fill}%
}%
\begin{pgfscope}%
\pgfsys@transformshift{0.800000in}{3.088581in}%
\pgfsys@useobject{currentmarker}{}%
\end{pgfscope}%
\end{pgfscope}%
\begin{pgfscope}%
\pgfsetbuttcap%
\pgfsetroundjoin%
\definecolor{currentfill}{rgb}{0.000000,0.000000,0.000000}%
\pgfsetfillcolor{currentfill}%
\pgfsetlinewidth{0.602250pt}%
\definecolor{currentstroke}{rgb}{0.000000,0.000000,0.000000}%
\pgfsetstrokecolor{currentstroke}%
\pgfsetdash{}{0pt}%
\pgfsys@defobject{currentmarker}{\pgfqpoint{-0.027778in}{0.000000in}}{\pgfqpoint{-0.000000in}{0.000000in}}{%
\pgfpathmoveto{\pgfqpoint{-0.000000in}{0.000000in}}%
\pgfpathlineto{\pgfqpoint{-0.027778in}{0.000000in}}%
\pgfusepath{stroke,fill}%
}%
\begin{pgfscope}%
\pgfsys@transformshift{0.800000in}{3.121835in}%
\pgfsys@useobject{currentmarker}{}%
\end{pgfscope}%
\end{pgfscope}%
\begin{pgfscope}%
\pgfsetbuttcap%
\pgfsetroundjoin%
\definecolor{currentfill}{rgb}{0.000000,0.000000,0.000000}%
\pgfsetfillcolor{currentfill}%
\pgfsetlinewidth{0.602250pt}%
\definecolor{currentstroke}{rgb}{0.000000,0.000000,0.000000}%
\pgfsetstrokecolor{currentstroke}%
\pgfsetdash{}{0pt}%
\pgfsys@defobject{currentmarker}{\pgfqpoint{-0.027778in}{0.000000in}}{\pgfqpoint{-0.000000in}{0.000000in}}{%
\pgfpathmoveto{\pgfqpoint{-0.000000in}{0.000000in}}%
\pgfpathlineto{\pgfqpoint{-0.027778in}{0.000000in}}%
\pgfusepath{stroke,fill}%
}%
\begin{pgfscope}%
\pgfsys@transformshift{0.800000in}{3.347279in}%
\pgfsys@useobject{currentmarker}{}%
\end{pgfscope}%
\end{pgfscope}%
\begin{pgfscope}%
\pgfsetbuttcap%
\pgfsetroundjoin%
\definecolor{currentfill}{rgb}{0.000000,0.000000,0.000000}%
\pgfsetfillcolor{currentfill}%
\pgfsetlinewidth{0.602250pt}%
\definecolor{currentstroke}{rgb}{0.000000,0.000000,0.000000}%
\pgfsetstrokecolor{currentstroke}%
\pgfsetdash{}{0pt}%
\pgfsys@defobject{currentmarker}{\pgfqpoint{-0.027778in}{0.000000in}}{\pgfqpoint{-0.000000in}{0.000000in}}{%
\pgfpathmoveto{\pgfqpoint{-0.000000in}{0.000000in}}%
\pgfpathlineto{\pgfqpoint{-0.027778in}{0.000000in}}%
\pgfusepath{stroke,fill}%
}%
\begin{pgfscope}%
\pgfsys@transformshift{0.800000in}{3.461755in}%
\pgfsys@useobject{currentmarker}{}%
\end{pgfscope}%
\end{pgfscope}%
\begin{pgfscope}%
\pgfsetbuttcap%
\pgfsetroundjoin%
\definecolor{currentfill}{rgb}{0.000000,0.000000,0.000000}%
\pgfsetfillcolor{currentfill}%
\pgfsetlinewidth{0.602250pt}%
\definecolor{currentstroke}{rgb}{0.000000,0.000000,0.000000}%
\pgfsetstrokecolor{currentstroke}%
\pgfsetdash{}{0pt}%
\pgfsys@defobject{currentmarker}{\pgfqpoint{-0.027778in}{0.000000in}}{\pgfqpoint{-0.000000in}{0.000000in}}{%
\pgfpathmoveto{\pgfqpoint{-0.000000in}{0.000000in}}%
\pgfpathlineto{\pgfqpoint{-0.027778in}{0.000000in}}%
\pgfusepath{stroke,fill}%
}%
\begin{pgfscope}%
\pgfsys@transformshift{0.800000in}{3.542976in}%
\pgfsys@useobject{currentmarker}{}%
\end{pgfscope}%
\end{pgfscope}%
\begin{pgfscope}%
\pgfsetbuttcap%
\pgfsetroundjoin%
\definecolor{currentfill}{rgb}{0.000000,0.000000,0.000000}%
\pgfsetfillcolor{currentfill}%
\pgfsetlinewidth{0.602250pt}%
\definecolor{currentstroke}{rgb}{0.000000,0.000000,0.000000}%
\pgfsetstrokecolor{currentstroke}%
\pgfsetdash{}{0pt}%
\pgfsys@defobject{currentmarker}{\pgfqpoint{-0.027778in}{0.000000in}}{\pgfqpoint{-0.000000in}{0.000000in}}{%
\pgfpathmoveto{\pgfqpoint{-0.000000in}{0.000000in}}%
\pgfpathlineto{\pgfqpoint{-0.027778in}{0.000000in}}%
\pgfusepath{stroke,fill}%
}%
\begin{pgfscope}%
\pgfsys@transformshift{0.800000in}{3.605977in}%
\pgfsys@useobject{currentmarker}{}%
\end{pgfscope}%
\end{pgfscope}%
\begin{pgfscope}%
\pgfsetbuttcap%
\pgfsetroundjoin%
\definecolor{currentfill}{rgb}{0.000000,0.000000,0.000000}%
\pgfsetfillcolor{currentfill}%
\pgfsetlinewidth{0.602250pt}%
\definecolor{currentstroke}{rgb}{0.000000,0.000000,0.000000}%
\pgfsetstrokecolor{currentstroke}%
\pgfsetdash{}{0pt}%
\pgfsys@defobject{currentmarker}{\pgfqpoint{-0.027778in}{0.000000in}}{\pgfqpoint{-0.000000in}{0.000000in}}{%
\pgfpathmoveto{\pgfqpoint{-0.000000in}{0.000000in}}%
\pgfpathlineto{\pgfqpoint{-0.027778in}{0.000000in}}%
\pgfusepath{stroke,fill}%
}%
\begin{pgfscope}%
\pgfsys@transformshift{0.800000in}{3.657452in}%
\pgfsys@useobject{currentmarker}{}%
\end{pgfscope}%
\end{pgfscope}%
\begin{pgfscope}%
\pgfsetbuttcap%
\pgfsetroundjoin%
\definecolor{currentfill}{rgb}{0.000000,0.000000,0.000000}%
\pgfsetfillcolor{currentfill}%
\pgfsetlinewidth{0.602250pt}%
\definecolor{currentstroke}{rgb}{0.000000,0.000000,0.000000}%
\pgfsetstrokecolor{currentstroke}%
\pgfsetdash{}{0pt}%
\pgfsys@defobject{currentmarker}{\pgfqpoint{-0.027778in}{0.000000in}}{\pgfqpoint{-0.000000in}{0.000000in}}{%
\pgfpathmoveto{\pgfqpoint{-0.000000in}{0.000000in}}%
\pgfpathlineto{\pgfqpoint{-0.027778in}{0.000000in}}%
\pgfusepath{stroke,fill}%
}%
\begin{pgfscope}%
\pgfsys@transformshift{0.800000in}{3.700973in}%
\pgfsys@useobject{currentmarker}{}%
\end{pgfscope}%
\end{pgfscope}%
\begin{pgfscope}%
\pgfsetbuttcap%
\pgfsetroundjoin%
\definecolor{currentfill}{rgb}{0.000000,0.000000,0.000000}%
\pgfsetfillcolor{currentfill}%
\pgfsetlinewidth{0.602250pt}%
\definecolor{currentstroke}{rgb}{0.000000,0.000000,0.000000}%
\pgfsetstrokecolor{currentstroke}%
\pgfsetdash{}{0pt}%
\pgfsys@defobject{currentmarker}{\pgfqpoint{-0.027778in}{0.000000in}}{\pgfqpoint{-0.000000in}{0.000000in}}{%
\pgfpathmoveto{\pgfqpoint{-0.000000in}{0.000000in}}%
\pgfpathlineto{\pgfqpoint{-0.027778in}{0.000000in}}%
\pgfusepath{stroke,fill}%
}%
\begin{pgfscope}%
\pgfsys@transformshift{0.800000in}{3.738673in}%
\pgfsys@useobject{currentmarker}{}%
\end{pgfscope}%
\end{pgfscope}%
\begin{pgfscope}%
\pgfsetbuttcap%
\pgfsetroundjoin%
\definecolor{currentfill}{rgb}{0.000000,0.000000,0.000000}%
\pgfsetfillcolor{currentfill}%
\pgfsetlinewidth{0.602250pt}%
\definecolor{currentstroke}{rgb}{0.000000,0.000000,0.000000}%
\pgfsetstrokecolor{currentstroke}%
\pgfsetdash{}{0pt}%
\pgfsys@defobject{currentmarker}{\pgfqpoint{-0.027778in}{0.000000in}}{\pgfqpoint{-0.000000in}{0.000000in}}{%
\pgfpathmoveto{\pgfqpoint{-0.000000in}{0.000000in}}%
\pgfpathlineto{\pgfqpoint{-0.027778in}{0.000000in}}%
\pgfusepath{stroke,fill}%
}%
\begin{pgfscope}%
\pgfsys@transformshift{0.800000in}{3.771927in}%
\pgfsys@useobject{currentmarker}{}%
\end{pgfscope}%
\end{pgfscope}%
\begin{pgfscope}%
\pgfsetbuttcap%
\pgfsetroundjoin%
\definecolor{currentfill}{rgb}{0.000000,0.000000,0.000000}%
\pgfsetfillcolor{currentfill}%
\pgfsetlinewidth{0.602250pt}%
\definecolor{currentstroke}{rgb}{0.000000,0.000000,0.000000}%
\pgfsetstrokecolor{currentstroke}%
\pgfsetdash{}{0pt}%
\pgfsys@defobject{currentmarker}{\pgfqpoint{-0.027778in}{0.000000in}}{\pgfqpoint{-0.000000in}{0.000000in}}{%
\pgfpathmoveto{\pgfqpoint{-0.000000in}{0.000000in}}%
\pgfpathlineto{\pgfqpoint{-0.027778in}{0.000000in}}%
\pgfusepath{stroke,fill}%
}%
\begin{pgfscope}%
\pgfsys@transformshift{0.800000in}{3.997371in}%
\pgfsys@useobject{currentmarker}{}%
\end{pgfscope}%
\end{pgfscope}%
\begin{pgfscope}%
\pgfsetbuttcap%
\pgfsetroundjoin%
\definecolor{currentfill}{rgb}{0.000000,0.000000,0.000000}%
\pgfsetfillcolor{currentfill}%
\pgfsetlinewidth{0.602250pt}%
\definecolor{currentstroke}{rgb}{0.000000,0.000000,0.000000}%
\pgfsetstrokecolor{currentstroke}%
\pgfsetdash{}{0pt}%
\pgfsys@defobject{currentmarker}{\pgfqpoint{-0.027778in}{0.000000in}}{\pgfqpoint{-0.000000in}{0.000000in}}{%
\pgfpathmoveto{\pgfqpoint{-0.000000in}{0.000000in}}%
\pgfpathlineto{\pgfqpoint{-0.027778in}{0.000000in}}%
\pgfusepath{stroke,fill}%
}%
\begin{pgfscope}%
\pgfsys@transformshift{0.800000in}{4.111847in}%
\pgfsys@useobject{currentmarker}{}%
\end{pgfscope}%
\end{pgfscope}%
\begin{pgfscope}%
\pgfsetbuttcap%
\pgfsetroundjoin%
\definecolor{currentfill}{rgb}{0.000000,0.000000,0.000000}%
\pgfsetfillcolor{currentfill}%
\pgfsetlinewidth{0.602250pt}%
\definecolor{currentstroke}{rgb}{0.000000,0.000000,0.000000}%
\pgfsetstrokecolor{currentstroke}%
\pgfsetdash{}{0pt}%
\pgfsys@defobject{currentmarker}{\pgfqpoint{-0.027778in}{0.000000in}}{\pgfqpoint{-0.000000in}{0.000000in}}{%
\pgfpathmoveto{\pgfqpoint{-0.000000in}{0.000000in}}%
\pgfpathlineto{\pgfqpoint{-0.027778in}{0.000000in}}%
\pgfusepath{stroke,fill}%
}%
\begin{pgfscope}%
\pgfsys@transformshift{0.800000in}{4.193068in}%
\pgfsys@useobject{currentmarker}{}%
\end{pgfscope}%
\end{pgfscope}%
\begin{pgfscope}%
\definecolor{textcolor}{rgb}{0.000000,0.000000,0.000000}%
\pgfsetstrokecolor{textcolor}%
\pgfsetfillcolor{textcolor}%
\pgftext[x=0.446026in,y=2.376000in,,bottom,rotate=90.000000]{\color{textcolor}\rmfamily\fontsize{10.000000}{12.000000}\selectfont SWAPS in log}%
\end{pgfscope}%
\begin{pgfscope}%
\pgfpathrectangle{\pgfqpoint{0.800000in}{0.528000in}}{\pgfqpoint{4.960000in}{3.696000in}}%
\pgfusepath{clip}%
\pgfsetrectcap%
\pgfsetroundjoin%
\pgfsetlinewidth{1.505625pt}%
\definecolor{currentstroke}{rgb}{0.121569,0.466667,0.705882}%
\pgfsetstrokecolor{currentstroke}%
\pgfsetdash{}{0pt}%
\pgfpathmoveto{\pgfqpoint{1.025455in}{0.696000in}}%
\pgfpathlineto{\pgfqpoint{1.071466in}{0.915981in}}%
\pgfpathlineto{\pgfqpoint{1.117477in}{1.054721in}}%
\pgfpathlineto{\pgfqpoint{1.163488in}{1.153756in}}%
\pgfpathlineto{\pgfqpoint{1.209499in}{1.218820in}}%
\pgfpathlineto{\pgfqpoint{1.255510in}{1.278189in}}%
\pgfpathlineto{\pgfqpoint{1.301521in}{1.335586in}}%
\pgfpathlineto{\pgfqpoint{1.347532in}{1.378179in}}%
\pgfpathlineto{\pgfqpoint{1.393544in}{1.417159in}}%
\pgfpathlineto{\pgfqpoint{1.439555in}{1.443972in}}%
\pgfpathlineto{\pgfqpoint{1.485566in}{1.481313in}}%
\pgfpathlineto{\pgfqpoint{1.531577in}{1.495208in}}%
\pgfpathlineto{\pgfqpoint{1.577588in}{1.529347in}}%
\pgfpathlineto{\pgfqpoint{1.623599in}{1.557809in}}%
\pgfpathlineto{\pgfqpoint{1.669610in}{1.580144in}}%
\pgfpathlineto{\pgfqpoint{1.715622in}{1.599465in}}%
\pgfpathlineto{\pgfqpoint{1.761633in}{1.620058in}}%
\pgfpathlineto{\pgfqpoint{1.807644in}{1.636544in}}%
\pgfpathlineto{\pgfqpoint{1.853655in}{1.652006in}}%
\pgfpathlineto{\pgfqpoint{1.899666in}{1.667804in}}%
\pgfpathlineto{\pgfqpoint{1.945677in}{1.680342in}}%
\pgfpathlineto{\pgfqpoint{1.991688in}{1.700847in}}%
\pgfpathlineto{\pgfqpoint{2.037699in}{1.711935in}}%
\pgfpathlineto{\pgfqpoint{2.083711in}{1.725176in}}%
\pgfpathlineto{\pgfqpoint{2.129722in}{1.748063in}}%
\pgfpathlineto{\pgfqpoint{2.175733in}{1.746225in}}%
\pgfpathlineto{\pgfqpoint{2.221744in}{1.760683in}}%
\pgfpathlineto{\pgfqpoint{2.267755in}{1.776982in}}%
\pgfpathlineto{\pgfqpoint{2.313766in}{1.786592in}}%
\pgfpathlineto{\pgfqpoint{2.359777in}{1.795851in}}%
\pgfpathlineto{\pgfqpoint{2.405788in}{1.811430in}}%
\pgfpathlineto{\pgfqpoint{2.451800in}{1.823434in}}%
\pgfpathlineto{\pgfqpoint{2.497811in}{1.820646in}}%
\pgfpathlineto{\pgfqpoint{2.543822in}{1.837688in}}%
\pgfpathlineto{\pgfqpoint{2.589833in}{1.842245in}}%
\pgfpathlineto{\pgfqpoint{2.635844in}{1.853479in}}%
\pgfpathlineto{\pgfqpoint{2.681855in}{1.861138in}}%
\pgfpathlineto{\pgfqpoint{2.727866in}{1.872786in}}%
\pgfpathlineto{\pgfqpoint{2.773878in}{1.880251in}}%
\pgfpathlineto{\pgfqpoint{2.819889in}{1.884024in}}%
\pgfpathlineto{\pgfqpoint{2.865900in}{1.898343in}}%
\pgfpathlineto{\pgfqpoint{2.911911in}{1.905053in}}%
\pgfpathlineto{\pgfqpoint{2.957922in}{1.912426in}}%
\pgfpathlineto{\pgfqpoint{3.003933in}{1.920984in}}%
\pgfpathlineto{\pgfqpoint{3.049944in}{1.925233in}}%
\pgfpathlineto{\pgfqpoint{3.095955in}{1.930892in}}%
\pgfpathlineto{\pgfqpoint{3.141967in}{1.939495in}}%
\pgfpathlineto{\pgfqpoint{3.187978in}{1.941697in}}%
\pgfpathlineto{\pgfqpoint{3.233989in}{1.949344in}}%
\pgfpathlineto{\pgfqpoint{3.280000in}{1.956613in}}%
\pgfpathlineto{\pgfqpoint{3.326011in}{1.961855in}}%
\pgfpathlineto{\pgfqpoint{3.372022in}{1.973487in}}%
\pgfpathlineto{\pgfqpoint{3.418033in}{1.977417in}}%
\pgfpathlineto{\pgfqpoint{3.464045in}{1.984976in}}%
\pgfpathlineto{\pgfqpoint{3.510056in}{1.985222in}}%
\pgfpathlineto{\pgfqpoint{3.556067in}{1.994743in}}%
\pgfpathlineto{\pgfqpoint{3.602078in}{2.002503in}}%
\pgfpathlineto{\pgfqpoint{3.648089in}{2.004562in}}%
\pgfpathlineto{\pgfqpoint{3.694100in}{2.011853in}}%
\pgfpathlineto{\pgfqpoint{3.740111in}{2.016201in}}%
\pgfpathlineto{\pgfqpoint{3.786122in}{2.022262in}}%
\pgfpathlineto{\pgfqpoint{3.832134in}{2.028693in}}%
\pgfpathlineto{\pgfqpoint{3.878145in}{2.031183in}}%
\pgfpathlineto{\pgfqpoint{3.924156in}{2.035481in}}%
\pgfpathlineto{\pgfqpoint{3.970167in}{2.036143in}}%
\pgfpathlineto{\pgfqpoint{4.016178in}{2.047363in}}%
\pgfpathlineto{\pgfqpoint{4.062189in}{2.048321in}}%
\pgfpathlineto{\pgfqpoint{4.108200in}{2.054521in}}%
\pgfpathlineto{\pgfqpoint{4.154212in}{2.061890in}}%
\pgfpathlineto{\pgfqpoint{4.200223in}{2.066435in}}%
\pgfpathlineto{\pgfqpoint{4.246234in}{2.070102in}}%
\pgfpathlineto{\pgfqpoint{4.292245in}{2.069971in}}%
\pgfpathlineto{\pgfqpoint{4.338256in}{2.077638in}}%
\pgfpathlineto{\pgfqpoint{4.384267in}{2.084138in}}%
\pgfpathlineto{\pgfqpoint{4.430278in}{2.089120in}}%
\pgfpathlineto{\pgfqpoint{4.476289in}{2.089473in}}%
\pgfpathlineto{\pgfqpoint{4.522301in}{2.091603in}}%
\pgfpathlineto{\pgfqpoint{4.568312in}{2.098664in}}%
\pgfpathlineto{\pgfqpoint{4.614323in}{2.101973in}}%
\pgfpathlineto{\pgfqpoint{4.660334in}{2.105163in}}%
\pgfpathlineto{\pgfqpoint{4.706345in}{2.111391in}}%
\pgfpathlineto{\pgfqpoint{4.752356in}{2.112692in}}%
\pgfpathlineto{\pgfqpoint{4.798367in}{2.117859in}}%
\pgfpathlineto{\pgfqpoint{4.844378in}{2.120560in}}%
\pgfpathlineto{\pgfqpoint{4.890390in}{2.128490in}}%
\pgfpathlineto{\pgfqpoint{4.936401in}{2.127907in}}%
\pgfpathlineto{\pgfqpoint{4.982412in}{2.129082in}}%
\pgfpathlineto{\pgfqpoint{5.028423in}{2.138070in}}%
\pgfpathlineto{\pgfqpoint{5.074434in}{2.141345in}}%
\pgfpathlineto{\pgfqpoint{5.120445in}{2.141809in}}%
\pgfpathlineto{\pgfqpoint{5.166456in}{2.148403in}}%
\pgfpathlineto{\pgfqpoint{5.212468in}{2.150437in}}%
\pgfpathlineto{\pgfqpoint{5.258479in}{2.147821in}}%
\pgfpathlineto{\pgfqpoint{5.304490in}{2.157933in}}%
\pgfpathlineto{\pgfqpoint{5.350501in}{2.157876in}}%
\pgfpathlineto{\pgfqpoint{5.396512in}{2.162801in}}%
\pgfpathlineto{\pgfqpoint{5.442523in}{2.165922in}}%
\pgfpathlineto{\pgfqpoint{5.488534in}{2.167715in}}%
\pgfpathlineto{\pgfqpoint{5.534545in}{2.174178in}}%
\pgfusepath{stroke}%
\end{pgfscope}%
\begin{pgfscope}%
\pgfpathrectangle{\pgfqpoint{0.800000in}{0.528000in}}{\pgfqpoint{4.960000in}{3.696000in}}%
\pgfusepath{clip}%
\pgfsetrectcap%
\pgfsetroundjoin%
\pgfsetlinewidth{1.505625pt}%
\definecolor{currentstroke}{rgb}{1.000000,0.498039,0.054902}%
\pgfsetstrokecolor{currentstroke}%
\pgfsetdash{}{0pt}%
\pgfpathmoveto{\pgfqpoint{1.025455in}{0.839884in}}%
\pgfpathlineto{\pgfqpoint{1.071466in}{1.093250in}}%
\pgfpathlineto{\pgfqpoint{1.117477in}{1.223818in}}%
\pgfpathlineto{\pgfqpoint{1.163488in}{1.317470in}}%
\pgfpathlineto{\pgfqpoint{1.209499in}{1.387237in}}%
\pgfpathlineto{\pgfqpoint{1.255510in}{1.458645in}}%
\pgfpathlineto{\pgfqpoint{1.301521in}{1.512324in}}%
\pgfpathlineto{\pgfqpoint{1.347532in}{1.549296in}}%
\pgfpathlineto{\pgfqpoint{1.393544in}{1.586275in}}%
\pgfpathlineto{\pgfqpoint{1.439555in}{1.620698in}}%
\pgfpathlineto{\pgfqpoint{1.485566in}{1.650572in}}%
\pgfpathlineto{\pgfqpoint{1.531577in}{1.685317in}}%
\pgfpathlineto{\pgfqpoint{1.577588in}{1.711240in}}%
\pgfpathlineto{\pgfqpoint{1.623599in}{1.737824in}}%
\pgfpathlineto{\pgfqpoint{1.669610in}{1.755774in}}%
\pgfpathlineto{\pgfqpoint{1.715622in}{1.776578in}}%
\pgfpathlineto{\pgfqpoint{1.761633in}{1.795816in}}%
\pgfpathlineto{\pgfqpoint{1.807644in}{1.812631in}}%
\pgfpathlineto{\pgfqpoint{1.853655in}{1.831334in}}%
\pgfpathlineto{\pgfqpoint{1.899666in}{1.846701in}}%
\pgfpathlineto{\pgfqpoint{1.945677in}{1.862199in}}%
\pgfpathlineto{\pgfqpoint{1.991688in}{1.878127in}}%
\pgfpathlineto{\pgfqpoint{2.037699in}{1.894806in}}%
\pgfpathlineto{\pgfqpoint{2.083711in}{1.907308in}}%
\pgfpathlineto{\pgfqpoint{2.129722in}{1.921755in}}%
\pgfpathlineto{\pgfqpoint{2.175733in}{1.934070in}}%
\pgfpathlineto{\pgfqpoint{2.221744in}{1.946677in}}%
\pgfpathlineto{\pgfqpoint{2.267755in}{1.957894in}}%
\pgfpathlineto{\pgfqpoint{2.313766in}{1.968347in}}%
\pgfpathlineto{\pgfqpoint{2.359777in}{1.977688in}}%
\pgfpathlineto{\pgfqpoint{2.405788in}{1.989010in}}%
\pgfpathlineto{\pgfqpoint{2.451800in}{1.999879in}}%
\pgfpathlineto{\pgfqpoint{2.497811in}{2.008878in}}%
\pgfpathlineto{\pgfqpoint{2.543822in}{2.019505in}}%
\pgfpathlineto{\pgfqpoint{2.589833in}{2.029370in}}%
\pgfpathlineto{\pgfqpoint{2.635844in}{2.036524in}}%
\pgfpathlineto{\pgfqpoint{2.681855in}{2.047264in}}%
\pgfpathlineto{\pgfqpoint{2.727866in}{2.053405in}}%
\pgfpathlineto{\pgfqpoint{2.773878in}{2.061501in}}%
\pgfpathlineto{\pgfqpoint{2.819889in}{2.068222in}}%
\pgfpathlineto{\pgfqpoint{2.865900in}{2.074607in}}%
\pgfpathlineto{\pgfqpoint{2.911911in}{2.084645in}}%
\pgfpathlineto{\pgfqpoint{2.957922in}{2.092590in}}%
\pgfpathlineto{\pgfqpoint{3.003933in}{2.101135in}}%
\pgfpathlineto{\pgfqpoint{3.049944in}{2.108169in}}%
\pgfpathlineto{\pgfqpoint{3.095955in}{2.115565in}}%
\pgfpathlineto{\pgfqpoint{3.141967in}{2.123924in}}%
\pgfpathlineto{\pgfqpoint{3.187978in}{2.130125in}}%
\pgfpathlineto{\pgfqpoint{3.233989in}{2.137312in}}%
\pgfpathlineto{\pgfqpoint{3.280000in}{2.143941in}}%
\pgfpathlineto{\pgfqpoint{3.326011in}{2.149810in}}%
\pgfpathlineto{\pgfqpoint{3.372022in}{2.156040in}}%
\pgfpathlineto{\pgfqpoint{3.418033in}{2.162613in}}%
\pgfpathlineto{\pgfqpoint{3.464045in}{2.168285in}}%
\pgfpathlineto{\pgfqpoint{3.510056in}{2.174277in}}%
\pgfpathlineto{\pgfqpoint{3.556067in}{2.179748in}}%
\pgfpathlineto{\pgfqpoint{3.602078in}{2.185530in}}%
\pgfpathlineto{\pgfqpoint{3.648089in}{2.190475in}}%
\pgfpathlineto{\pgfqpoint{3.694100in}{2.195985in}}%
\pgfpathlineto{\pgfqpoint{3.740111in}{2.200514in}}%
\pgfpathlineto{\pgfqpoint{3.786122in}{2.204737in}}%
\pgfpathlineto{\pgfqpoint{3.832134in}{2.211236in}}%
\pgfpathlineto{\pgfqpoint{3.878145in}{2.216010in}}%
\pgfpathlineto{\pgfqpoint{3.924156in}{2.221025in}}%
\pgfpathlineto{\pgfqpoint{3.970167in}{2.227097in}}%
\pgfpathlineto{\pgfqpoint{4.016178in}{2.231068in}}%
\pgfpathlineto{\pgfqpoint{4.062189in}{2.235948in}}%
\pgfpathlineto{\pgfqpoint{4.108200in}{2.240894in}}%
\pgfpathlineto{\pgfqpoint{4.154212in}{2.245371in}}%
\pgfpathlineto{\pgfqpoint{4.200223in}{2.249791in}}%
\pgfpathlineto{\pgfqpoint{4.246234in}{2.253940in}}%
\pgfpathlineto{\pgfqpoint{4.292245in}{2.257567in}}%
\pgfpathlineto{\pgfqpoint{4.338256in}{2.262507in}}%
\pgfpathlineto{\pgfqpoint{4.384267in}{2.266591in}}%
\pgfpathlineto{\pgfqpoint{4.430278in}{2.270624in}}%
\pgfpathlineto{\pgfqpoint{4.476289in}{2.274297in}}%
\pgfpathlineto{\pgfqpoint{4.522301in}{2.278252in}}%
\pgfpathlineto{\pgfqpoint{4.568312in}{2.282846in}}%
\pgfpathlineto{\pgfqpoint{4.614323in}{2.286152in}}%
\pgfpathlineto{\pgfqpoint{4.660334in}{2.289833in}}%
\pgfpathlineto{\pgfqpoint{4.706345in}{2.291863in}}%
\pgfpathlineto{\pgfqpoint{4.752356in}{2.295019in}}%
\pgfpathlineto{\pgfqpoint{4.798367in}{2.299829in}}%
\pgfpathlineto{\pgfqpoint{4.844378in}{2.304763in}}%
\pgfpathlineto{\pgfqpoint{4.890390in}{2.309188in}}%
\pgfpathlineto{\pgfqpoint{4.936401in}{2.313337in}}%
\pgfpathlineto{\pgfqpoint{4.982412in}{2.317202in}}%
\pgfpathlineto{\pgfqpoint{5.028423in}{2.321182in}}%
\pgfpathlineto{\pgfqpoint{5.074434in}{2.325448in}}%
\pgfpathlineto{\pgfqpoint{5.120445in}{2.329345in}}%
\pgfpathlineto{\pgfqpoint{5.166456in}{2.332275in}}%
\pgfpathlineto{\pgfqpoint{5.212468in}{2.335643in}}%
\pgfpathlineto{\pgfqpoint{5.258479in}{2.339668in}}%
\pgfpathlineto{\pgfqpoint{5.304490in}{2.343419in}}%
\pgfpathlineto{\pgfqpoint{5.350501in}{2.347940in}}%
\pgfpathlineto{\pgfqpoint{5.396512in}{2.350303in}}%
\pgfpathlineto{\pgfqpoint{5.442523in}{2.354126in}}%
\pgfpathlineto{\pgfqpoint{5.488534in}{2.357710in}}%
\pgfpathlineto{\pgfqpoint{5.534545in}{2.360426in}}%
\pgfusepath{stroke}%
\end{pgfscope}%
\begin{pgfscope}%
\pgfpathrectangle{\pgfqpoint{0.800000in}{0.528000in}}{\pgfqpoint{4.960000in}{3.696000in}}%
\pgfusepath{clip}%
\pgfsetrectcap%
\pgfsetroundjoin%
\pgfsetlinewidth{1.505625pt}%
\definecolor{currentstroke}{rgb}{0.172549,0.627451,0.172549}%
\pgfsetstrokecolor{currentstroke}%
\pgfsetdash{}{0pt}%
\pgfpathmoveto{\pgfqpoint{1.025455in}{1.043097in}}%
\pgfpathlineto{\pgfqpoint{1.071466in}{1.287495in}}%
\pgfpathlineto{\pgfqpoint{1.117477in}{1.424937in}}%
\pgfpathlineto{\pgfqpoint{1.163488in}{1.521918in}}%
\pgfpathlineto{\pgfqpoint{1.209499in}{1.596486in}}%
\pgfpathlineto{\pgfqpoint{1.255510in}{1.655418in}}%
\pgfpathlineto{\pgfqpoint{1.301521in}{1.707740in}}%
\pgfpathlineto{\pgfqpoint{1.347532in}{1.752506in}}%
\pgfpathlineto{\pgfqpoint{1.393544in}{1.790752in}}%
\pgfpathlineto{\pgfqpoint{1.439555in}{1.823247in}}%
\pgfpathlineto{\pgfqpoint{1.485566in}{1.855407in}}%
\pgfpathlineto{\pgfqpoint{1.531577in}{1.884325in}}%
\pgfpathlineto{\pgfqpoint{1.577588in}{1.910647in}}%
\pgfpathlineto{\pgfqpoint{1.623599in}{1.935289in}}%
\pgfpathlineto{\pgfqpoint{1.669610in}{1.956788in}}%
\pgfpathlineto{\pgfqpoint{1.715622in}{1.978877in}}%
\pgfpathlineto{\pgfqpoint{1.761633in}{1.996978in}}%
\pgfpathlineto{\pgfqpoint{1.807644in}{2.015697in}}%
\pgfpathlineto{\pgfqpoint{1.853655in}{2.033014in}}%
\pgfpathlineto{\pgfqpoint{1.899666in}{2.049149in}}%
\pgfpathlineto{\pgfqpoint{1.945677in}{2.065775in}}%
\pgfpathlineto{\pgfqpoint{1.991688in}{2.079884in}}%
\pgfpathlineto{\pgfqpoint{2.037699in}{2.094673in}}%
\pgfpathlineto{\pgfqpoint{2.083711in}{2.108680in}}%
\pgfpathlineto{\pgfqpoint{2.129722in}{2.121809in}}%
\pgfpathlineto{\pgfqpoint{2.175733in}{2.134323in}}%
\pgfpathlineto{\pgfqpoint{2.221744in}{2.146395in}}%
\pgfpathlineto{\pgfqpoint{2.267755in}{2.158305in}}%
\pgfpathlineto{\pgfqpoint{2.313766in}{2.168845in}}%
\pgfpathlineto{\pgfqpoint{2.359777in}{2.180717in}}%
\pgfpathlineto{\pgfqpoint{2.405788in}{2.191170in}}%
\pgfpathlineto{\pgfqpoint{2.451800in}{2.200809in}}%
\pgfpathlineto{\pgfqpoint{2.497811in}{2.210825in}}%
\pgfpathlineto{\pgfqpoint{2.543822in}{2.220407in}}%
\pgfpathlineto{\pgfqpoint{2.589833in}{2.229741in}}%
\pgfpathlineto{\pgfqpoint{2.635844in}{2.238124in}}%
\pgfpathlineto{\pgfqpoint{2.681855in}{2.247947in}}%
\pgfpathlineto{\pgfqpoint{2.727866in}{2.255375in}}%
\pgfpathlineto{\pgfqpoint{2.773878in}{2.264083in}}%
\pgfpathlineto{\pgfqpoint{2.819889in}{2.271715in}}%
\pgfpathlineto{\pgfqpoint{2.865900in}{2.279978in}}%
\pgfpathlineto{\pgfqpoint{2.911911in}{2.287848in}}%
\pgfpathlineto{\pgfqpoint{2.957922in}{2.294867in}}%
\pgfpathlineto{\pgfqpoint{3.003933in}{2.302613in}}%
\pgfpathlineto{\pgfqpoint{3.049944in}{2.309300in}}%
\pgfpathlineto{\pgfqpoint{3.095955in}{2.316399in}}%
\pgfpathlineto{\pgfqpoint{3.141967in}{2.323392in}}%
\pgfpathlineto{\pgfqpoint{3.187978in}{2.330123in}}%
\pgfpathlineto{\pgfqpoint{3.233989in}{2.336526in}}%
\pgfpathlineto{\pgfqpoint{3.280000in}{2.343409in}}%
\pgfpathlineto{\pgfqpoint{3.326011in}{2.349395in}}%
\pgfpathlineto{\pgfqpoint{3.372022in}{2.355829in}}%
\pgfpathlineto{\pgfqpoint{3.418033in}{2.362388in}}%
\pgfpathlineto{\pgfqpoint{3.464045in}{2.367934in}}%
\pgfpathlineto{\pgfqpoint{3.510056in}{2.373941in}}%
\pgfpathlineto{\pgfqpoint{3.556067in}{2.379636in}}%
\pgfpathlineto{\pgfqpoint{3.602078in}{2.385457in}}%
\pgfpathlineto{\pgfqpoint{3.648089in}{2.390601in}}%
\pgfpathlineto{\pgfqpoint{3.694100in}{2.395842in}}%
\pgfpathlineto{\pgfqpoint{3.740111in}{2.401756in}}%
\pgfpathlineto{\pgfqpoint{3.786122in}{2.406111in}}%
\pgfpathlineto{\pgfqpoint{3.832134in}{2.412219in}}%
\pgfpathlineto{\pgfqpoint{3.878145in}{2.416898in}}%
\pgfpathlineto{\pgfqpoint{3.924156in}{2.421856in}}%
\pgfpathlineto{\pgfqpoint{3.970167in}{2.426552in}}%
\pgfpathlineto{\pgfqpoint{4.016178in}{2.431902in}}%
\pgfpathlineto{\pgfqpoint{4.062189in}{2.435880in}}%
\pgfpathlineto{\pgfqpoint{4.108200in}{2.440809in}}%
\pgfpathlineto{\pgfqpoint{4.154212in}{2.445468in}}%
\pgfpathlineto{\pgfqpoint{4.200223in}{2.450132in}}%
\pgfpathlineto{\pgfqpoint{4.246234in}{2.454584in}}%
\pgfpathlineto{\pgfqpoint{4.292245in}{2.458648in}}%
\pgfpathlineto{\pgfqpoint{4.338256in}{2.463237in}}%
\pgfpathlineto{\pgfqpoint{4.384267in}{2.467632in}}%
\pgfpathlineto{\pgfqpoint{4.430278in}{2.471815in}}%
\pgfpathlineto{\pgfqpoint{4.476289in}{2.476126in}}%
\pgfpathlineto{\pgfqpoint{4.522301in}{2.479821in}}%
\pgfpathlineto{\pgfqpoint{4.568312in}{2.483948in}}%
\pgfpathlineto{\pgfqpoint{4.614323in}{2.487715in}}%
\pgfpathlineto{\pgfqpoint{4.660334in}{2.491943in}}%
\pgfpathlineto{\pgfqpoint{4.706345in}{2.495604in}}%
\pgfpathlineto{\pgfqpoint{4.752356in}{2.499430in}}%
\pgfpathlineto{\pgfqpoint{4.798367in}{2.503285in}}%
\pgfpathlineto{\pgfqpoint{4.844378in}{2.506959in}}%
\pgfpathlineto{\pgfqpoint{4.890390in}{2.511063in}}%
\pgfpathlineto{\pgfqpoint{4.936401in}{2.514980in}}%
\pgfpathlineto{\pgfqpoint{4.982412in}{2.518375in}}%
\pgfpathlineto{\pgfqpoint{5.028423in}{2.521927in}}%
\pgfpathlineto{\pgfqpoint{5.074434in}{2.526004in}}%
\pgfpathlineto{\pgfqpoint{5.120445in}{2.528830in}}%
\pgfpathlineto{\pgfqpoint{5.166456in}{2.532832in}}%
\pgfpathlineto{\pgfqpoint{5.212468in}{2.536123in}}%
\pgfpathlineto{\pgfqpoint{5.258479in}{2.539561in}}%
\pgfpathlineto{\pgfqpoint{5.304490in}{2.543065in}}%
\pgfpathlineto{\pgfqpoint{5.350501in}{2.546386in}}%
\pgfpathlineto{\pgfqpoint{5.396512in}{2.549437in}}%
\pgfpathlineto{\pgfqpoint{5.442523in}{2.552836in}}%
\pgfpathlineto{\pgfqpoint{5.488534in}{2.556426in}}%
\pgfpathlineto{\pgfqpoint{5.534545in}{2.559555in}}%
\pgfusepath{stroke}%
\end{pgfscope}%
\begin{pgfscope}%
\pgfpathrectangle{\pgfqpoint{0.800000in}{0.528000in}}{\pgfqpoint{4.960000in}{3.696000in}}%
\pgfusepath{clip}%
\pgfsetrectcap%
\pgfsetroundjoin%
\pgfsetlinewidth{1.505625pt}%
\definecolor{currentstroke}{rgb}{0.839216,0.152941,0.156863}%
\pgfsetstrokecolor{currentstroke}%
\pgfsetdash{}{0pt}%
\pgfpathmoveto{\pgfqpoint{1.025455in}{1.481209in}}%
\pgfpathlineto{\pgfqpoint{1.071466in}{1.841573in}}%
\pgfpathlineto{\pgfqpoint{1.117477in}{2.072847in}}%
\pgfpathlineto{\pgfqpoint{1.163488in}{2.235688in}}%
\pgfpathlineto{\pgfqpoint{1.209499in}{2.372192in}}%
\pgfpathlineto{\pgfqpoint{1.255510in}{2.467256in}}%
\pgfpathlineto{\pgfqpoint{1.301521in}{2.565557in}}%
\pgfpathlineto{\pgfqpoint{1.347532in}{2.630179in}}%
\pgfpathlineto{\pgfqpoint{1.393544in}{2.696513in}}%
\pgfpathlineto{\pgfqpoint{1.439555in}{2.756368in}}%
\pgfpathlineto{\pgfqpoint{1.485566in}{2.809344in}}%
\pgfpathlineto{\pgfqpoint{1.531577in}{2.868554in}}%
\pgfpathlineto{\pgfqpoint{1.577588in}{2.914806in}}%
\pgfpathlineto{\pgfqpoint{1.623599in}{2.953068in}}%
\pgfpathlineto{\pgfqpoint{1.669610in}{2.991256in}}%
\pgfpathlineto{\pgfqpoint{1.715622in}{3.036108in}}%
\pgfpathlineto{\pgfqpoint{1.761633in}{3.057604in}}%
\pgfpathlineto{\pgfqpoint{1.807644in}{3.088674in}}%
\pgfpathlineto{\pgfqpoint{1.853655in}{3.119065in}}%
\pgfpathlineto{\pgfqpoint{1.899666in}{3.151082in}}%
\pgfpathlineto{\pgfqpoint{1.945677in}{3.173633in}}%
\pgfpathlineto{\pgfqpoint{1.991688in}{3.203951in}}%
\pgfpathlineto{\pgfqpoint{2.037699in}{3.231133in}}%
\pgfpathlineto{\pgfqpoint{2.083711in}{3.259169in}}%
\pgfpathlineto{\pgfqpoint{2.129722in}{3.276976in}}%
\pgfpathlineto{\pgfqpoint{2.175733in}{3.303735in}}%
\pgfpathlineto{\pgfqpoint{2.221744in}{3.315472in}}%
\pgfpathlineto{\pgfqpoint{2.267755in}{3.342618in}}%
\pgfpathlineto{\pgfqpoint{2.313766in}{3.362654in}}%
\pgfpathlineto{\pgfqpoint{2.359777in}{3.382147in}}%
\pgfpathlineto{\pgfqpoint{2.405788in}{3.399722in}}%
\pgfpathlineto{\pgfqpoint{2.451800in}{3.413799in}}%
\pgfpathlineto{\pgfqpoint{2.497811in}{3.438059in}}%
\pgfpathlineto{\pgfqpoint{2.543822in}{3.454315in}}%
\pgfpathlineto{\pgfqpoint{2.589833in}{3.466720in}}%
\pgfpathlineto{\pgfqpoint{2.635844in}{3.486034in}}%
\pgfpathlineto{\pgfqpoint{2.681855in}{3.498711in}}%
\pgfpathlineto{\pgfqpoint{2.727866in}{3.515419in}}%
\pgfpathlineto{\pgfqpoint{2.773878in}{3.530754in}}%
\pgfpathlineto{\pgfqpoint{2.819889in}{3.544462in}}%
\pgfpathlineto{\pgfqpoint{2.865900in}{3.557904in}}%
\pgfpathlineto{\pgfqpoint{2.911911in}{3.569725in}}%
\pgfpathlineto{\pgfqpoint{2.957922in}{3.580946in}}%
\pgfpathlineto{\pgfqpoint{3.003933in}{3.597396in}}%
\pgfpathlineto{\pgfqpoint{3.049944in}{3.612911in}}%
\pgfpathlineto{\pgfqpoint{3.095955in}{3.623540in}}%
\pgfpathlineto{\pgfqpoint{3.141967in}{3.633555in}}%
\pgfpathlineto{\pgfqpoint{3.187978in}{3.648825in}}%
\pgfpathlineto{\pgfqpoint{3.233989in}{3.653809in}}%
\pgfpathlineto{\pgfqpoint{3.280000in}{3.669494in}}%
\pgfpathlineto{\pgfqpoint{3.326011in}{3.683098in}}%
\pgfpathlineto{\pgfqpoint{3.372022in}{3.696316in}}%
\pgfpathlineto{\pgfqpoint{3.418033in}{3.698481in}}%
\pgfpathlineto{\pgfqpoint{3.464045in}{3.714002in}}%
\pgfpathlineto{\pgfqpoint{3.510056in}{3.718308in}}%
\pgfpathlineto{\pgfqpoint{3.556067in}{3.730135in}}%
\pgfpathlineto{\pgfqpoint{3.602078in}{3.743567in}}%
\pgfpathlineto{\pgfqpoint{3.648089in}{3.753634in}}%
\pgfpathlineto{\pgfqpoint{3.694100in}{3.762954in}}%
\pgfpathlineto{\pgfqpoint{3.740111in}{3.770631in}}%
\pgfpathlineto{\pgfqpoint{3.786122in}{3.784954in}}%
\pgfpathlineto{\pgfqpoint{3.832134in}{3.788741in}}%
\pgfpathlineto{\pgfqpoint{3.878145in}{3.799763in}}%
\pgfpathlineto{\pgfqpoint{3.924156in}{3.804345in}}%
\pgfpathlineto{\pgfqpoint{3.970167in}{3.814216in}}%
\pgfpathlineto{\pgfqpoint{4.016178in}{3.823906in}}%
\pgfpathlineto{\pgfqpoint{4.062189in}{3.832936in}}%
\pgfpathlineto{\pgfqpoint{4.108200in}{3.844487in}}%
\pgfpathlineto{\pgfqpoint{4.108200in}{3.838767in}}%
\pgfpathlineto{\pgfqpoint{4.154212in}{3.850683in}}%
\pgfpathlineto{\pgfqpoint{4.200223in}{3.859637in}}%
\pgfpathlineto{\pgfqpoint{4.246234in}{3.867405in}}%
\pgfpathlineto{\pgfqpoint{4.292245in}{3.871078in}}%
\pgfpathlineto{\pgfqpoint{4.338256in}{3.881284in}}%
\pgfpathlineto{\pgfqpoint{4.384267in}{3.894259in}}%
\pgfpathlineto{\pgfqpoint{4.430278in}{3.898055in}}%
\pgfpathlineto{\pgfqpoint{4.476289in}{3.903019in}}%
\pgfpathlineto{\pgfqpoint{4.522301in}{3.910628in}}%
\pgfpathlineto{\pgfqpoint{4.568312in}{3.919027in}}%
\pgfpathlineto{\pgfqpoint{4.614323in}{3.927701in}}%
\pgfpathlineto{\pgfqpoint{4.660334in}{3.935272in}}%
\pgfpathlineto{\pgfqpoint{4.706345in}{3.941972in}}%
\pgfpathlineto{\pgfqpoint{4.752356in}{3.948770in}}%
\pgfpathlineto{\pgfqpoint{4.798367in}{3.955614in}}%
\pgfpathlineto{\pgfqpoint{4.844378in}{3.964240in}}%
\pgfpathlineto{\pgfqpoint{4.890390in}{3.969429in}}%
\pgfpathlineto{\pgfqpoint{4.936401in}{3.974640in}}%
\pgfpathlineto{\pgfqpoint{4.982412in}{3.981938in}}%
\pgfpathlineto{\pgfqpoint{5.028423in}{3.987895in}}%
\pgfpathlineto{\pgfqpoint{5.074434in}{3.994899in}}%
\pgfpathlineto{\pgfqpoint{5.120445in}{4.000826in}}%
\pgfpathlineto{\pgfqpoint{5.166456in}{4.006621in}}%
\pgfpathlineto{\pgfqpoint{5.212468in}{4.012645in}}%
\pgfpathlineto{\pgfqpoint{5.258479in}{4.021374in}}%
\pgfpathlineto{\pgfqpoint{5.304490in}{4.025188in}}%
\pgfpathlineto{\pgfqpoint{5.350501in}{4.032580in}}%
\pgfpathlineto{\pgfqpoint{5.396512in}{4.035980in}}%
\pgfpathlineto{\pgfqpoint{5.442523in}{4.042656in}}%
\pgfpathlineto{\pgfqpoint{5.488534in}{4.050265in}}%
\pgfpathlineto{\pgfqpoint{5.534545in}{4.056000in}}%
\pgfusepath{stroke}%
\end{pgfscope}%
\begin{pgfscope}%
\pgfpathrectangle{\pgfqpoint{0.800000in}{0.528000in}}{\pgfqpoint{4.960000in}{3.696000in}}%
\pgfusepath{clip}%
\pgfsetrectcap%
\pgfsetroundjoin%
\pgfsetlinewidth{1.505625pt}%
\definecolor{currentstroke}{rgb}{0.580392,0.403922,0.741176}%
\pgfsetstrokecolor{currentstroke}%
\pgfsetdash{}{0pt}%
\pgfpathmoveto{\pgfqpoint{1.025455in}{0.812161in}}%
\pgfpathlineto{\pgfqpoint{1.071466in}{1.086548in}}%
\pgfpathlineto{\pgfqpoint{1.117477in}{1.290393in}}%
\pgfpathlineto{\pgfqpoint{1.163488in}{1.379984in}}%
\pgfpathlineto{\pgfqpoint{1.209499in}{1.495208in}}%
\pgfpathlineto{\pgfqpoint{1.255510in}{1.572059in}}%
\pgfpathlineto{\pgfqpoint{1.301521in}{1.638711in}}%
\pgfpathlineto{\pgfqpoint{1.347532in}{1.677063in}}%
\pgfpathlineto{\pgfqpoint{1.393544in}{1.739424in}}%
\pgfpathlineto{\pgfqpoint{1.439555in}{1.797939in}}%
\pgfpathlineto{\pgfqpoint{1.485566in}{1.817957in}}%
\pgfpathlineto{\pgfqpoint{1.531577in}{1.869972in}}%
\pgfpathlineto{\pgfqpoint{1.577588in}{1.894298in}}%
\pgfpathlineto{\pgfqpoint{1.623599in}{1.927958in}}%
\pgfpathlineto{\pgfqpoint{1.669610in}{1.951906in}}%
\pgfpathlineto{\pgfqpoint{1.715622in}{1.987741in}}%
\pgfpathlineto{\pgfqpoint{1.761633in}{2.021706in}}%
\pgfpathlineto{\pgfqpoint{1.807644in}{2.041967in}}%
\pgfpathlineto{\pgfqpoint{1.853655in}{2.075451in}}%
\pgfpathlineto{\pgfqpoint{1.899666in}{2.088743in}}%
\pgfpathlineto{\pgfqpoint{1.945677in}{2.104898in}}%
\pgfpathlineto{\pgfqpoint{1.991688in}{2.118813in}}%
\pgfpathlineto{\pgfqpoint{2.037699in}{2.143951in}}%
\pgfpathlineto{\pgfqpoint{2.083711in}{2.163671in}}%
\pgfpathlineto{\pgfqpoint{2.129722in}{2.172318in}}%
\pgfpathlineto{\pgfqpoint{2.175733in}{2.195768in}}%
\pgfpathlineto{\pgfqpoint{2.221744in}{2.212598in}}%
\pgfpathlineto{\pgfqpoint{2.267755in}{2.234178in}}%
\pgfpathlineto{\pgfqpoint{2.313766in}{2.239441in}}%
\pgfpathlineto{\pgfqpoint{2.359777in}{2.251863in}}%
\pgfpathlineto{\pgfqpoint{2.405788in}{2.264174in}}%
\pgfpathlineto{\pgfqpoint{2.451800in}{2.283745in}}%
\pgfpathlineto{\pgfqpoint{2.497811in}{2.299466in}}%
\pgfpathlineto{\pgfqpoint{2.543822in}{2.307975in}}%
\pgfpathlineto{\pgfqpoint{2.589833in}{2.320679in}}%
\pgfpathlineto{\pgfqpoint{2.635844in}{2.331287in}}%
\pgfpathlineto{\pgfqpoint{2.681855in}{2.347502in}}%
\pgfpathlineto{\pgfqpoint{2.727866in}{2.346340in}}%
\pgfpathlineto{\pgfqpoint{2.773878in}{2.368246in}}%
\pgfpathlineto{\pgfqpoint{2.819889in}{2.383980in}}%
\pgfpathlineto{\pgfqpoint{2.865900in}{2.379884in}}%
\pgfpathlineto{\pgfqpoint{2.911911in}{2.398733in}}%
\pgfpathlineto{\pgfqpoint{2.957922in}{2.401321in}}%
\pgfpathlineto{\pgfqpoint{3.003933in}{2.415328in}}%
\pgfpathlineto{\pgfqpoint{3.049944in}{2.424535in}}%
\pgfpathlineto{\pgfqpoint{3.095955in}{2.435702in}}%
\pgfpathlineto{\pgfqpoint{3.141967in}{2.447447in}}%
\pgfpathlineto{\pgfqpoint{3.187978in}{2.457014in}}%
\pgfpathlineto{\pgfqpoint{3.233989in}{2.460835in}}%
\pgfpathlineto{\pgfqpoint{3.280000in}{2.470183in}}%
\pgfpathlineto{\pgfqpoint{3.326011in}{2.477967in}}%
\pgfpathlineto{\pgfqpoint{3.372022in}{2.484821in}}%
\pgfpathlineto{\pgfqpoint{3.418033in}{2.498923in}}%
\pgfpathlineto{\pgfqpoint{3.464045in}{2.503246in}}%
\pgfpathlineto{\pgfqpoint{3.510056in}{2.512729in}}%
\pgfpathlineto{\pgfqpoint{3.556067in}{2.521114in}}%
\pgfpathlineto{\pgfqpoint{3.602078in}{2.524803in}}%
\pgfpathlineto{\pgfqpoint{3.648089in}{2.529672in}}%
\pgfpathlineto{\pgfqpoint{3.694100in}{2.544213in}}%
\pgfpathlineto{\pgfqpoint{3.740111in}{2.548533in}}%
\pgfpathlineto{\pgfqpoint{3.786122in}{2.552661in}}%
\pgfpathlineto{\pgfqpoint{3.832134in}{2.562778in}}%
\pgfpathlineto{\pgfqpoint{3.878145in}{2.568805in}}%
\pgfpathlineto{\pgfqpoint{3.924156in}{2.572566in}}%
\pgfpathlineto{\pgfqpoint{3.970167in}{2.580410in}}%
\pgfpathlineto{\pgfqpoint{4.016178in}{2.585584in}}%
\pgfpathlineto{\pgfqpoint{4.062189in}{2.596351in}}%
\pgfpathlineto{\pgfqpoint{4.108200in}{2.600758in}}%
\pgfpathlineto{\pgfqpoint{4.154212in}{2.607663in}}%
\pgfpathlineto{\pgfqpoint{4.200223in}{2.612787in}}%
\pgfpathlineto{\pgfqpoint{4.246234in}{2.616432in}}%
\pgfpathlineto{\pgfqpoint{4.292245in}{2.622089in}}%
\pgfpathlineto{\pgfqpoint{4.338256in}{2.628952in}}%
\pgfpathlineto{\pgfqpoint{4.384267in}{2.633699in}}%
\pgfpathlineto{\pgfqpoint{4.430278in}{2.640536in}}%
\pgfpathlineto{\pgfqpoint{4.476289in}{2.643131in}}%
\pgfpathlineto{\pgfqpoint{4.522301in}{2.653916in}}%
\pgfpathlineto{\pgfqpoint{4.568312in}{2.658156in}}%
\pgfpathlineto{\pgfqpoint{4.614323in}{2.668836in}}%
\pgfpathlineto{\pgfqpoint{4.660334in}{2.670780in}}%
\pgfpathlineto{\pgfqpoint{4.706345in}{2.675565in}}%
\pgfpathlineto{\pgfqpoint{4.752356in}{2.678117in}}%
\pgfpathlineto{\pgfqpoint{4.798367in}{2.682475in}}%
\pgfpathlineto{\pgfqpoint{4.844378in}{2.689323in}}%
\pgfpathlineto{\pgfqpoint{4.890390in}{2.695969in}}%
\pgfpathlineto{\pgfqpoint{4.936401in}{2.703021in}}%
\pgfpathlineto{\pgfqpoint{4.982412in}{2.704390in}}%
\pgfpathlineto{\pgfqpoint{5.028423in}{2.715103in}}%
\pgfpathlineto{\pgfqpoint{5.074434in}{2.709695in}}%
\pgfpathlineto{\pgfqpoint{5.120445in}{2.718269in}}%
\pgfpathlineto{\pgfqpoint{5.166456in}{2.725444in}}%
\pgfpathlineto{\pgfqpoint{5.212468in}{2.730240in}}%
\pgfpathlineto{\pgfqpoint{5.258479in}{2.735417in}}%
\pgfpathlineto{\pgfqpoint{5.304490in}{2.737660in}}%
\pgfpathlineto{\pgfqpoint{5.350501in}{2.744895in}}%
\pgfpathlineto{\pgfqpoint{5.396512in}{2.744946in}}%
\pgfpathlineto{\pgfqpoint{5.442523in}{2.755559in}}%
\pgfpathlineto{\pgfqpoint{5.488534in}{2.759597in}}%
\pgfpathlineto{\pgfqpoint{5.534545in}{2.759477in}}%
\pgfusepath{stroke}%
\end{pgfscope}%
\begin{pgfscope}%
\pgfsetrectcap%
\pgfsetmiterjoin%
\pgfsetlinewidth{0.803000pt}%
\definecolor{currentstroke}{rgb}{0.000000,0.000000,0.000000}%
\pgfsetstrokecolor{currentstroke}%
\pgfsetdash{}{0pt}%
\pgfpathmoveto{\pgfqpoint{0.800000in}{0.528000in}}%
\pgfpathlineto{\pgfqpoint{0.800000in}{4.224000in}}%
\pgfusepath{stroke}%
\end{pgfscope}%
\begin{pgfscope}%
\pgfsetrectcap%
\pgfsetmiterjoin%
\pgfsetlinewidth{0.803000pt}%
\definecolor{currentstroke}{rgb}{0.000000,0.000000,0.000000}%
\pgfsetstrokecolor{currentstroke}%
\pgfsetdash{}{0pt}%
\pgfpathmoveto{\pgfqpoint{5.760000in}{0.528000in}}%
\pgfpathlineto{\pgfqpoint{5.760000in}{4.224000in}}%
\pgfusepath{stroke}%
\end{pgfscope}%
\begin{pgfscope}%
\pgfsetrectcap%
\pgfsetmiterjoin%
\pgfsetlinewidth{0.803000pt}%
\definecolor{currentstroke}{rgb}{0.000000,0.000000,0.000000}%
\pgfsetstrokecolor{currentstroke}%
\pgfsetdash{}{0pt}%
\pgfpathmoveto{\pgfqpoint{0.800000in}{0.528000in}}%
\pgfpathlineto{\pgfqpoint{5.760000in}{0.528000in}}%
\pgfusepath{stroke}%
\end{pgfscope}%
\begin{pgfscope}%
\pgfsetrectcap%
\pgfsetmiterjoin%
\pgfsetlinewidth{0.803000pt}%
\definecolor{currentstroke}{rgb}{0.000000,0.000000,0.000000}%
\pgfsetstrokecolor{currentstroke}%
\pgfsetdash{}{0pt}%
\pgfpathmoveto{\pgfqpoint{0.800000in}{4.224000in}}%
\pgfpathlineto{\pgfqpoint{5.760000in}{4.224000in}}%
\pgfusepath{stroke}%
\end{pgfscope}%
\begin{pgfscope}%
\pgfsetbuttcap%
\pgfsetmiterjoin%
\definecolor{currentfill}{rgb}{1.000000,1.000000,1.000000}%
\pgfsetfillcolor{currentfill}%
\pgfsetfillopacity{0.800000}%
\pgfsetlinewidth{1.003750pt}%
\definecolor{currentstroke}{rgb}{0.800000,0.800000,0.800000}%
\pgfsetstrokecolor{currentstroke}%
\pgfsetstrokeopacity{0.800000}%
\pgfsetdash{}{0pt}%
\pgfpathmoveto{\pgfqpoint{0.897222in}{3.144525in}}%
\pgfpathlineto{\pgfqpoint{1.739044in}{3.144525in}}%
\pgfpathquadraticcurveto{\pgfqpoint{1.766822in}{3.144525in}}{\pgfqpoint{1.766822in}{3.172303in}}%
\pgfpathlineto{\pgfqpoint{1.766822in}{4.126778in}}%
\pgfpathquadraticcurveto{\pgfqpoint{1.766822in}{4.154556in}}{\pgfqpoint{1.739044in}{4.154556in}}%
\pgfpathlineto{\pgfqpoint{0.897222in}{4.154556in}}%
\pgfpathquadraticcurveto{\pgfqpoint{0.869444in}{4.154556in}}{\pgfqpoint{0.869444in}{4.126778in}}%
\pgfpathlineto{\pgfqpoint{0.869444in}{3.172303in}}%
\pgfpathquadraticcurveto{\pgfqpoint{0.869444in}{3.144525in}}{\pgfqpoint{0.897222in}{3.144525in}}%
\pgfpathlineto{\pgfqpoint{0.897222in}{3.144525in}}%
\pgfpathclose%
\pgfusepath{stroke,fill}%
\end{pgfscope}%
\begin{pgfscope}%
\pgfsetrectcap%
\pgfsetroundjoin%
\pgfsetlinewidth{1.505625pt}%
\definecolor{currentstroke}{rgb}{0.121569,0.466667,0.705882}%
\pgfsetstrokecolor{currentstroke}%
\pgfsetdash{}{0pt}%
\pgfpathmoveto{\pgfqpoint{0.925000in}{4.050389in}}%
\pgfpathlineto{\pgfqpoint{1.063889in}{4.050389in}}%
\pgfpathlineto{\pgfqpoint{1.202778in}{4.050389in}}%
\pgfusepath{stroke}%
\end{pgfscope}%
\begin{pgfscope}%
\definecolor{textcolor}{rgb}{0.000000,0.000000,0.000000}%
\pgfsetstrokecolor{textcolor}%
\pgfsetfillcolor{textcolor}%
\pgftext[x=1.313889in,y=4.001778in,left,base]{\color{textcolor}\rmfamily\fontsize{10.000000}{12.000000}\selectfont quick}%
\end{pgfscope}%
\begin{pgfscope}%
\pgfsetrectcap%
\pgfsetroundjoin%
\pgfsetlinewidth{1.505625pt}%
\definecolor{currentstroke}{rgb}{1.000000,0.498039,0.054902}%
\pgfsetstrokecolor{currentstroke}%
\pgfsetdash{}{0pt}%
\pgfpathmoveto{\pgfqpoint{0.925000in}{3.856716in}}%
\pgfpathlineto{\pgfqpoint{1.063889in}{3.856716in}}%
\pgfpathlineto{\pgfqpoint{1.202778in}{3.856716in}}%
\pgfusepath{stroke}%
\end{pgfscope}%
\begin{pgfscope}%
\definecolor{textcolor}{rgb}{0.000000,0.000000,0.000000}%
\pgfsetstrokecolor{textcolor}%
\pgfsetfillcolor{textcolor}%
\pgftext[x=1.313889in,y=3.808105in,left,base]{\color{textcolor}\rmfamily\fontsize{10.000000}{12.000000}\selectfont merge}%
\end{pgfscope}%
\begin{pgfscope}%
\pgfsetrectcap%
\pgfsetroundjoin%
\pgfsetlinewidth{1.505625pt}%
\definecolor{currentstroke}{rgb}{0.172549,0.627451,0.172549}%
\pgfsetstrokecolor{currentstroke}%
\pgfsetdash{}{0pt}%
\pgfpathmoveto{\pgfqpoint{0.925000in}{3.663043in}}%
\pgfpathlineto{\pgfqpoint{1.063889in}{3.663043in}}%
\pgfpathlineto{\pgfqpoint{1.202778in}{3.663043in}}%
\pgfusepath{stroke}%
\end{pgfscope}%
\begin{pgfscope}%
\definecolor{textcolor}{rgb}{0.000000,0.000000,0.000000}%
\pgfsetstrokecolor{textcolor}%
\pgfsetfillcolor{textcolor}%
\pgftext[x=1.313889in,y=3.614432in,left,base]{\color{textcolor}\rmfamily\fontsize{10.000000}{12.000000}\selectfont heap}%
\end{pgfscope}%
\begin{pgfscope}%
\pgfsetrectcap%
\pgfsetroundjoin%
\pgfsetlinewidth{1.505625pt}%
\definecolor{currentstroke}{rgb}{0.839216,0.152941,0.156863}%
\pgfsetstrokecolor{currentstroke}%
\pgfsetdash{}{0pt}%
\pgfpathmoveto{\pgfqpoint{0.925000in}{3.469371in}}%
\pgfpathlineto{\pgfqpoint{1.063889in}{3.469371in}}%
\pgfpathlineto{\pgfqpoint{1.202778in}{3.469371in}}%
\pgfusepath{stroke}%
\end{pgfscope}%
\begin{pgfscope}%
\definecolor{textcolor}{rgb}{0.000000,0.000000,0.000000}%
\pgfsetstrokecolor{textcolor}%
\pgfsetfillcolor{textcolor}%
\pgftext[x=1.313889in,y=3.420759in,left,base]{\color{textcolor}\rmfamily\fontsize{10.000000}{12.000000}\selectfont insert}%
\end{pgfscope}%
\begin{pgfscope}%
\pgfsetrectcap%
\pgfsetroundjoin%
\pgfsetlinewidth{1.505625pt}%
\definecolor{currentstroke}{rgb}{0.580392,0.403922,0.741176}%
\pgfsetstrokecolor{currentstroke}%
\pgfsetdash{}{0pt}%
\pgfpathmoveto{\pgfqpoint{0.925000in}{3.275698in}}%
\pgfpathlineto{\pgfqpoint{1.063889in}{3.275698in}}%
\pgfpathlineto{\pgfqpoint{1.202778in}{3.275698in}}%
\pgfusepath{stroke}%
\end{pgfscope}%
\begin{pgfscope}%
\definecolor{textcolor}{rgb}{0.000000,0.000000,0.000000}%
\pgfsetstrokecolor{textcolor}%
\pgfsetfillcolor{textcolor}%
\pgftext[x=1.313889in,y=3.227087in,left,base]{\color{textcolor}\rmfamily\fontsize{10.000000}{12.000000}\selectfont bucket}%
\end{pgfscope}%
\end{pgfpicture}%
\makeatother%
\endgroup%

%% Creator: Matplotlib, PGF backend
%%
%% To include the figure in your LaTeX document, write
%%   \input{<filename>.pgf}
%%
%% Make sure the required packages are loaded in your preamble
%%   \usepackage{pgf}
%%
%% Also ensure that all the required font packages are loaded; for instance,
%% the lmodern package is sometimes necessary when using math font.
%%   \usepackage{lmodern}
%%
%% Figures using additional raster images can only be included by \input if
%% they are in the same directory as the main LaTeX file. For loading figures
%% from other directories you can use the `import` package
%%   \usepackage{import}
%%
%% and then include the figures with
%%   \import{<path to file>}{<filename>.pgf}
%%
%% Matplotlib used the following preamble
%%   
%%   \makeatletter\@ifpackageloaded{underscore}{}{\usepackage[strings]{underscore}}\makeatother
%%
\begingroup%
\makeatletter%
\begin{pgfpicture}%
\pgfpathrectangle{\pgfpointorigin}{\pgfqpoint{6.400000in}{4.800000in}}%
\pgfusepath{use as bounding box, clip}%
\begin{pgfscope}%
\pgfsetbuttcap%
\pgfsetmiterjoin%
\definecolor{currentfill}{rgb}{1.000000,1.000000,1.000000}%
\pgfsetfillcolor{currentfill}%
\pgfsetlinewidth{0.000000pt}%
\definecolor{currentstroke}{rgb}{1.000000,1.000000,1.000000}%
\pgfsetstrokecolor{currentstroke}%
\pgfsetdash{}{0pt}%
\pgfpathmoveto{\pgfqpoint{0.000000in}{0.000000in}}%
\pgfpathlineto{\pgfqpoint{6.400000in}{0.000000in}}%
\pgfpathlineto{\pgfqpoint{6.400000in}{4.800000in}}%
\pgfpathlineto{\pgfqpoint{0.000000in}{4.800000in}}%
\pgfpathlineto{\pgfqpoint{0.000000in}{0.000000in}}%
\pgfpathclose%
\pgfusepath{fill}%
\end{pgfscope}%
\begin{pgfscope}%
\pgfsetbuttcap%
\pgfsetmiterjoin%
\definecolor{currentfill}{rgb}{1.000000,1.000000,1.000000}%
\pgfsetfillcolor{currentfill}%
\pgfsetlinewidth{0.000000pt}%
\definecolor{currentstroke}{rgb}{0.000000,0.000000,0.000000}%
\pgfsetstrokecolor{currentstroke}%
\pgfsetstrokeopacity{0.000000}%
\pgfsetdash{}{0pt}%
\pgfpathmoveto{\pgfqpoint{0.800000in}{0.528000in}}%
\pgfpathlineto{\pgfqpoint{5.760000in}{0.528000in}}%
\pgfpathlineto{\pgfqpoint{5.760000in}{4.224000in}}%
\pgfpathlineto{\pgfqpoint{0.800000in}{4.224000in}}%
\pgfpathlineto{\pgfqpoint{0.800000in}{0.528000in}}%
\pgfpathclose%
\pgfusepath{fill}%
\end{pgfscope}%
\begin{pgfscope}%
\pgfsetbuttcap%
\pgfsetroundjoin%
\definecolor{currentfill}{rgb}{0.000000,0.000000,0.000000}%
\pgfsetfillcolor{currentfill}%
\pgfsetlinewidth{0.803000pt}%
\definecolor{currentstroke}{rgb}{0.000000,0.000000,0.000000}%
\pgfsetstrokecolor{currentstroke}%
\pgfsetdash{}{0pt}%
\pgfsys@defobject{currentmarker}{\pgfqpoint{0.000000in}{-0.048611in}}{\pgfqpoint{0.000000in}{0.000000in}}{%
\pgfpathmoveto{\pgfqpoint{0.000000in}{0.000000in}}%
\pgfpathlineto{\pgfqpoint{0.000000in}{-0.048611in}}%
\pgfusepath{stroke,fill}%
}%
\begin{pgfscope}%
\pgfsys@transformshift{0.979443in}{0.528000in}%
\pgfsys@useobject{currentmarker}{}%
\end{pgfscope}%
\end{pgfscope}%
\begin{pgfscope}%
\definecolor{textcolor}{rgb}{0.000000,0.000000,0.000000}%
\pgfsetstrokecolor{textcolor}%
\pgfsetfillcolor{textcolor}%
\pgftext[x=0.979443in,y=0.430778in,,top]{\color{textcolor}\rmfamily\fontsize{10.000000}{12.000000}\selectfont \(\displaystyle {0}\)}%
\end{pgfscope}%
\begin{pgfscope}%
\pgfsetbuttcap%
\pgfsetroundjoin%
\definecolor{currentfill}{rgb}{0.000000,0.000000,0.000000}%
\pgfsetfillcolor{currentfill}%
\pgfsetlinewidth{0.803000pt}%
\definecolor{currentstroke}{rgb}{0.000000,0.000000,0.000000}%
\pgfsetstrokecolor{currentstroke}%
\pgfsetdash{}{0pt}%
\pgfsys@defobject{currentmarker}{\pgfqpoint{0.000000in}{-0.048611in}}{\pgfqpoint{0.000000in}{0.000000in}}{%
\pgfpathmoveto{\pgfqpoint{0.000000in}{0.000000in}}%
\pgfpathlineto{\pgfqpoint{0.000000in}{-0.048611in}}%
\pgfusepath{stroke,fill}%
}%
\begin{pgfscope}%
\pgfsys@transformshift{1.899666in}{0.528000in}%
\pgfsys@useobject{currentmarker}{}%
\end{pgfscope}%
\end{pgfscope}%
\begin{pgfscope}%
\definecolor{textcolor}{rgb}{0.000000,0.000000,0.000000}%
\pgfsetstrokecolor{textcolor}%
\pgfsetfillcolor{textcolor}%
\pgftext[x=1.899666in,y=0.430778in,,top]{\color{textcolor}\rmfamily\fontsize{10.000000}{12.000000}\selectfont \(\displaystyle {2000}\)}%
\end{pgfscope}%
\begin{pgfscope}%
\pgfsetbuttcap%
\pgfsetroundjoin%
\definecolor{currentfill}{rgb}{0.000000,0.000000,0.000000}%
\pgfsetfillcolor{currentfill}%
\pgfsetlinewidth{0.803000pt}%
\definecolor{currentstroke}{rgb}{0.000000,0.000000,0.000000}%
\pgfsetstrokecolor{currentstroke}%
\pgfsetdash{}{0pt}%
\pgfsys@defobject{currentmarker}{\pgfqpoint{0.000000in}{-0.048611in}}{\pgfqpoint{0.000000in}{0.000000in}}{%
\pgfpathmoveto{\pgfqpoint{0.000000in}{0.000000in}}%
\pgfpathlineto{\pgfqpoint{0.000000in}{-0.048611in}}%
\pgfusepath{stroke,fill}%
}%
\begin{pgfscope}%
\pgfsys@transformshift{2.819889in}{0.528000in}%
\pgfsys@useobject{currentmarker}{}%
\end{pgfscope}%
\end{pgfscope}%
\begin{pgfscope}%
\definecolor{textcolor}{rgb}{0.000000,0.000000,0.000000}%
\pgfsetstrokecolor{textcolor}%
\pgfsetfillcolor{textcolor}%
\pgftext[x=2.819889in,y=0.430778in,,top]{\color{textcolor}\rmfamily\fontsize{10.000000}{12.000000}\selectfont \(\displaystyle {4000}\)}%
\end{pgfscope}%
\begin{pgfscope}%
\pgfsetbuttcap%
\pgfsetroundjoin%
\definecolor{currentfill}{rgb}{0.000000,0.000000,0.000000}%
\pgfsetfillcolor{currentfill}%
\pgfsetlinewidth{0.803000pt}%
\definecolor{currentstroke}{rgb}{0.000000,0.000000,0.000000}%
\pgfsetstrokecolor{currentstroke}%
\pgfsetdash{}{0pt}%
\pgfsys@defobject{currentmarker}{\pgfqpoint{0.000000in}{-0.048611in}}{\pgfqpoint{0.000000in}{0.000000in}}{%
\pgfpathmoveto{\pgfqpoint{0.000000in}{0.000000in}}%
\pgfpathlineto{\pgfqpoint{0.000000in}{-0.048611in}}%
\pgfusepath{stroke,fill}%
}%
\begin{pgfscope}%
\pgfsys@transformshift{3.740111in}{0.528000in}%
\pgfsys@useobject{currentmarker}{}%
\end{pgfscope}%
\end{pgfscope}%
\begin{pgfscope}%
\definecolor{textcolor}{rgb}{0.000000,0.000000,0.000000}%
\pgfsetstrokecolor{textcolor}%
\pgfsetfillcolor{textcolor}%
\pgftext[x=3.740111in,y=0.430778in,,top]{\color{textcolor}\rmfamily\fontsize{10.000000}{12.000000}\selectfont \(\displaystyle {6000}\)}%
\end{pgfscope}%
\begin{pgfscope}%
\pgfsetbuttcap%
\pgfsetroundjoin%
\definecolor{currentfill}{rgb}{0.000000,0.000000,0.000000}%
\pgfsetfillcolor{currentfill}%
\pgfsetlinewidth{0.803000pt}%
\definecolor{currentstroke}{rgb}{0.000000,0.000000,0.000000}%
\pgfsetstrokecolor{currentstroke}%
\pgfsetdash{}{0pt}%
\pgfsys@defobject{currentmarker}{\pgfqpoint{0.000000in}{-0.048611in}}{\pgfqpoint{0.000000in}{0.000000in}}{%
\pgfpathmoveto{\pgfqpoint{0.000000in}{0.000000in}}%
\pgfpathlineto{\pgfqpoint{0.000000in}{-0.048611in}}%
\pgfusepath{stroke,fill}%
}%
\begin{pgfscope}%
\pgfsys@transformshift{4.660334in}{0.528000in}%
\pgfsys@useobject{currentmarker}{}%
\end{pgfscope}%
\end{pgfscope}%
\begin{pgfscope}%
\definecolor{textcolor}{rgb}{0.000000,0.000000,0.000000}%
\pgfsetstrokecolor{textcolor}%
\pgfsetfillcolor{textcolor}%
\pgftext[x=4.660334in,y=0.430778in,,top]{\color{textcolor}\rmfamily\fontsize{10.000000}{12.000000}\selectfont \(\displaystyle {8000}\)}%
\end{pgfscope}%
\begin{pgfscope}%
\pgfsetbuttcap%
\pgfsetroundjoin%
\definecolor{currentfill}{rgb}{0.000000,0.000000,0.000000}%
\pgfsetfillcolor{currentfill}%
\pgfsetlinewidth{0.803000pt}%
\definecolor{currentstroke}{rgb}{0.000000,0.000000,0.000000}%
\pgfsetstrokecolor{currentstroke}%
\pgfsetdash{}{0pt}%
\pgfsys@defobject{currentmarker}{\pgfqpoint{0.000000in}{-0.048611in}}{\pgfqpoint{0.000000in}{0.000000in}}{%
\pgfpathmoveto{\pgfqpoint{0.000000in}{0.000000in}}%
\pgfpathlineto{\pgfqpoint{0.000000in}{-0.048611in}}%
\pgfusepath{stroke,fill}%
}%
\begin{pgfscope}%
\pgfsys@transformshift{5.580557in}{0.528000in}%
\pgfsys@useobject{currentmarker}{}%
\end{pgfscope}%
\end{pgfscope}%
\begin{pgfscope}%
\definecolor{textcolor}{rgb}{0.000000,0.000000,0.000000}%
\pgfsetstrokecolor{textcolor}%
\pgfsetfillcolor{textcolor}%
\pgftext[x=5.580557in,y=0.430778in,,top]{\color{textcolor}\rmfamily\fontsize{10.000000}{12.000000}\selectfont \(\displaystyle {10000}\)}%
\end{pgfscope}%
\begin{pgfscope}%
\definecolor{textcolor}{rgb}{0.000000,0.000000,0.000000}%
\pgfsetstrokecolor{textcolor}%
\pgfsetfillcolor{textcolor}%
\pgftext[x=3.280000in,y=0.251766in,,top]{\color{textcolor}\rmfamily\fontsize{10.000000}{12.000000}\selectfont Input Size}%
\end{pgfscope}%
\begin{pgfscope}%
\pgfsetbuttcap%
\pgfsetroundjoin%
\definecolor{currentfill}{rgb}{0.000000,0.000000,0.000000}%
\pgfsetfillcolor{currentfill}%
\pgfsetlinewidth{0.803000pt}%
\definecolor{currentstroke}{rgb}{0.000000,0.000000,0.000000}%
\pgfsetstrokecolor{currentstroke}%
\pgfsetdash{}{0pt}%
\pgfsys@defobject{currentmarker}{\pgfqpoint{-0.048611in}{0.000000in}}{\pgfqpoint{-0.000000in}{0.000000in}}{%
\pgfpathmoveto{\pgfqpoint{-0.000000in}{0.000000in}}%
\pgfpathlineto{\pgfqpoint{-0.048611in}{0.000000in}}%
\pgfusepath{stroke,fill}%
}%
\begin{pgfscope}%
\pgfsys@transformshift{0.800000in}{1.047732in}%
\pgfsys@useobject{currentmarker}{}%
\end{pgfscope}%
\end{pgfscope}%
\begin{pgfscope}%
\definecolor{textcolor}{rgb}{0.000000,0.000000,0.000000}%
\pgfsetstrokecolor{textcolor}%
\pgfsetfillcolor{textcolor}%
\pgftext[x=0.501581in, y=0.999507in, left, base]{\color{textcolor}\rmfamily\fontsize{10.000000}{12.000000}\selectfont \(\displaystyle {10^{4}}\)}%
\end{pgfscope}%
\begin{pgfscope}%
\pgfsetbuttcap%
\pgfsetroundjoin%
\definecolor{currentfill}{rgb}{0.000000,0.000000,0.000000}%
\pgfsetfillcolor{currentfill}%
\pgfsetlinewidth{0.803000pt}%
\definecolor{currentstroke}{rgb}{0.000000,0.000000,0.000000}%
\pgfsetstrokecolor{currentstroke}%
\pgfsetdash{}{0pt}%
\pgfsys@defobject{currentmarker}{\pgfqpoint{-0.048611in}{0.000000in}}{\pgfqpoint{-0.000000in}{0.000000in}}{%
\pgfpathmoveto{\pgfqpoint{-0.000000in}{0.000000in}}%
\pgfpathlineto{\pgfqpoint{-0.048611in}{0.000000in}}%
\pgfusepath{stroke,fill}%
}%
\begin{pgfscope}%
\pgfsys@transformshift{0.800000in}{1.726398in}%
\pgfsys@useobject{currentmarker}{}%
\end{pgfscope}%
\end{pgfscope}%
\begin{pgfscope}%
\definecolor{textcolor}{rgb}{0.000000,0.000000,0.000000}%
\pgfsetstrokecolor{textcolor}%
\pgfsetfillcolor{textcolor}%
\pgftext[x=0.501581in, y=1.678172in, left, base]{\color{textcolor}\rmfamily\fontsize{10.000000}{12.000000}\selectfont \(\displaystyle {10^{5}}\)}%
\end{pgfscope}%
\begin{pgfscope}%
\pgfsetbuttcap%
\pgfsetroundjoin%
\definecolor{currentfill}{rgb}{0.000000,0.000000,0.000000}%
\pgfsetfillcolor{currentfill}%
\pgfsetlinewidth{0.803000pt}%
\definecolor{currentstroke}{rgb}{0.000000,0.000000,0.000000}%
\pgfsetstrokecolor{currentstroke}%
\pgfsetdash{}{0pt}%
\pgfsys@defobject{currentmarker}{\pgfqpoint{-0.048611in}{0.000000in}}{\pgfqpoint{-0.000000in}{0.000000in}}{%
\pgfpathmoveto{\pgfqpoint{-0.000000in}{0.000000in}}%
\pgfpathlineto{\pgfqpoint{-0.048611in}{0.000000in}}%
\pgfusepath{stroke,fill}%
}%
\begin{pgfscope}%
\pgfsys@transformshift{0.800000in}{2.405063in}%
\pgfsys@useobject{currentmarker}{}%
\end{pgfscope}%
\end{pgfscope}%
\begin{pgfscope}%
\definecolor{textcolor}{rgb}{0.000000,0.000000,0.000000}%
\pgfsetstrokecolor{textcolor}%
\pgfsetfillcolor{textcolor}%
\pgftext[x=0.501581in, y=2.356837in, left, base]{\color{textcolor}\rmfamily\fontsize{10.000000}{12.000000}\selectfont \(\displaystyle {10^{6}}\)}%
\end{pgfscope}%
\begin{pgfscope}%
\pgfsetbuttcap%
\pgfsetroundjoin%
\definecolor{currentfill}{rgb}{0.000000,0.000000,0.000000}%
\pgfsetfillcolor{currentfill}%
\pgfsetlinewidth{0.803000pt}%
\definecolor{currentstroke}{rgb}{0.000000,0.000000,0.000000}%
\pgfsetstrokecolor{currentstroke}%
\pgfsetdash{}{0pt}%
\pgfsys@defobject{currentmarker}{\pgfqpoint{-0.048611in}{0.000000in}}{\pgfqpoint{-0.000000in}{0.000000in}}{%
\pgfpathmoveto{\pgfqpoint{-0.000000in}{0.000000in}}%
\pgfpathlineto{\pgfqpoint{-0.048611in}{0.000000in}}%
\pgfusepath{stroke,fill}%
}%
\begin{pgfscope}%
\pgfsys@transformshift{0.800000in}{3.083728in}%
\pgfsys@useobject{currentmarker}{}%
\end{pgfscope}%
\end{pgfscope}%
\begin{pgfscope}%
\definecolor{textcolor}{rgb}{0.000000,0.000000,0.000000}%
\pgfsetstrokecolor{textcolor}%
\pgfsetfillcolor{textcolor}%
\pgftext[x=0.501581in, y=3.035503in, left, base]{\color{textcolor}\rmfamily\fontsize{10.000000}{12.000000}\selectfont \(\displaystyle {10^{7}}\)}%
\end{pgfscope}%
\begin{pgfscope}%
\pgfsetbuttcap%
\pgfsetroundjoin%
\definecolor{currentfill}{rgb}{0.000000,0.000000,0.000000}%
\pgfsetfillcolor{currentfill}%
\pgfsetlinewidth{0.803000pt}%
\definecolor{currentstroke}{rgb}{0.000000,0.000000,0.000000}%
\pgfsetstrokecolor{currentstroke}%
\pgfsetdash{}{0pt}%
\pgfsys@defobject{currentmarker}{\pgfqpoint{-0.048611in}{0.000000in}}{\pgfqpoint{-0.000000in}{0.000000in}}{%
\pgfpathmoveto{\pgfqpoint{-0.000000in}{0.000000in}}%
\pgfpathlineto{\pgfqpoint{-0.048611in}{0.000000in}}%
\pgfusepath{stroke,fill}%
}%
\begin{pgfscope}%
\pgfsys@transformshift{0.800000in}{3.762393in}%
\pgfsys@useobject{currentmarker}{}%
\end{pgfscope}%
\end{pgfscope}%
\begin{pgfscope}%
\definecolor{textcolor}{rgb}{0.000000,0.000000,0.000000}%
\pgfsetstrokecolor{textcolor}%
\pgfsetfillcolor{textcolor}%
\pgftext[x=0.501581in, y=3.714168in, left, base]{\color{textcolor}\rmfamily\fontsize{10.000000}{12.000000}\selectfont \(\displaystyle {10^{8}}\)}%
\end{pgfscope}%
\begin{pgfscope}%
\pgfsetbuttcap%
\pgfsetroundjoin%
\definecolor{currentfill}{rgb}{0.000000,0.000000,0.000000}%
\pgfsetfillcolor{currentfill}%
\pgfsetlinewidth{0.602250pt}%
\definecolor{currentstroke}{rgb}{0.000000,0.000000,0.000000}%
\pgfsetstrokecolor{currentstroke}%
\pgfsetdash{}{0pt}%
\pgfsys@defobject{currentmarker}{\pgfqpoint{-0.027778in}{0.000000in}}{\pgfqpoint{-0.000000in}{0.000000in}}{%
\pgfpathmoveto{\pgfqpoint{-0.000000in}{0.000000in}}%
\pgfpathlineto{\pgfqpoint{-0.027778in}{0.000000in}}%
\pgfusepath{stroke,fill}%
}%
\begin{pgfscope}%
\pgfsys@transformshift{0.800000in}{0.573366in}%
\pgfsys@useobject{currentmarker}{}%
\end{pgfscope}%
\end{pgfscope}%
\begin{pgfscope}%
\pgfsetbuttcap%
\pgfsetroundjoin%
\definecolor{currentfill}{rgb}{0.000000,0.000000,0.000000}%
\pgfsetfillcolor{currentfill}%
\pgfsetlinewidth{0.602250pt}%
\definecolor{currentstroke}{rgb}{0.000000,0.000000,0.000000}%
\pgfsetstrokecolor{currentstroke}%
\pgfsetdash{}{0pt}%
\pgfsys@defobject{currentmarker}{\pgfqpoint{-0.027778in}{0.000000in}}{\pgfqpoint{-0.000000in}{0.000000in}}{%
\pgfpathmoveto{\pgfqpoint{-0.000000in}{0.000000in}}%
\pgfpathlineto{\pgfqpoint{-0.027778in}{0.000000in}}%
\pgfusepath{stroke,fill}%
}%
\begin{pgfscope}%
\pgfsys@transformshift{0.800000in}{0.692873in}%
\pgfsys@useobject{currentmarker}{}%
\end{pgfscope}%
\end{pgfscope}%
\begin{pgfscope}%
\pgfsetbuttcap%
\pgfsetroundjoin%
\definecolor{currentfill}{rgb}{0.000000,0.000000,0.000000}%
\pgfsetfillcolor{currentfill}%
\pgfsetlinewidth{0.602250pt}%
\definecolor{currentstroke}{rgb}{0.000000,0.000000,0.000000}%
\pgfsetstrokecolor{currentstroke}%
\pgfsetdash{}{0pt}%
\pgfsys@defobject{currentmarker}{\pgfqpoint{-0.027778in}{0.000000in}}{\pgfqpoint{-0.000000in}{0.000000in}}{%
\pgfpathmoveto{\pgfqpoint{-0.000000in}{0.000000in}}%
\pgfpathlineto{\pgfqpoint{-0.027778in}{0.000000in}}%
\pgfusepath{stroke,fill}%
}%
\begin{pgfscope}%
\pgfsys@transformshift{0.800000in}{0.777664in}%
\pgfsys@useobject{currentmarker}{}%
\end{pgfscope}%
\end{pgfscope}%
\begin{pgfscope}%
\pgfsetbuttcap%
\pgfsetroundjoin%
\definecolor{currentfill}{rgb}{0.000000,0.000000,0.000000}%
\pgfsetfillcolor{currentfill}%
\pgfsetlinewidth{0.602250pt}%
\definecolor{currentstroke}{rgb}{0.000000,0.000000,0.000000}%
\pgfsetstrokecolor{currentstroke}%
\pgfsetdash{}{0pt}%
\pgfsys@defobject{currentmarker}{\pgfqpoint{-0.027778in}{0.000000in}}{\pgfqpoint{-0.000000in}{0.000000in}}{%
\pgfpathmoveto{\pgfqpoint{-0.000000in}{0.000000in}}%
\pgfpathlineto{\pgfqpoint{-0.027778in}{0.000000in}}%
\pgfusepath{stroke,fill}%
}%
\begin{pgfscope}%
\pgfsys@transformshift{0.800000in}{0.843434in}%
\pgfsys@useobject{currentmarker}{}%
\end{pgfscope}%
\end{pgfscope}%
\begin{pgfscope}%
\pgfsetbuttcap%
\pgfsetroundjoin%
\definecolor{currentfill}{rgb}{0.000000,0.000000,0.000000}%
\pgfsetfillcolor{currentfill}%
\pgfsetlinewidth{0.602250pt}%
\definecolor{currentstroke}{rgb}{0.000000,0.000000,0.000000}%
\pgfsetstrokecolor{currentstroke}%
\pgfsetdash{}{0pt}%
\pgfsys@defobject{currentmarker}{\pgfqpoint{-0.027778in}{0.000000in}}{\pgfqpoint{-0.000000in}{0.000000in}}{%
\pgfpathmoveto{\pgfqpoint{-0.000000in}{0.000000in}}%
\pgfpathlineto{\pgfqpoint{-0.027778in}{0.000000in}}%
\pgfusepath{stroke,fill}%
}%
\begin{pgfscope}%
\pgfsys@transformshift{0.800000in}{0.897171in}%
\pgfsys@useobject{currentmarker}{}%
\end{pgfscope}%
\end{pgfscope}%
\begin{pgfscope}%
\pgfsetbuttcap%
\pgfsetroundjoin%
\definecolor{currentfill}{rgb}{0.000000,0.000000,0.000000}%
\pgfsetfillcolor{currentfill}%
\pgfsetlinewidth{0.602250pt}%
\definecolor{currentstroke}{rgb}{0.000000,0.000000,0.000000}%
\pgfsetstrokecolor{currentstroke}%
\pgfsetdash{}{0pt}%
\pgfsys@defobject{currentmarker}{\pgfqpoint{-0.027778in}{0.000000in}}{\pgfqpoint{-0.000000in}{0.000000in}}{%
\pgfpathmoveto{\pgfqpoint{-0.000000in}{0.000000in}}%
\pgfpathlineto{\pgfqpoint{-0.027778in}{0.000000in}}%
\pgfusepath{stroke,fill}%
}%
\begin{pgfscope}%
\pgfsys@transformshift{0.800000in}{0.942606in}%
\pgfsys@useobject{currentmarker}{}%
\end{pgfscope}%
\end{pgfscope}%
\begin{pgfscope}%
\pgfsetbuttcap%
\pgfsetroundjoin%
\definecolor{currentfill}{rgb}{0.000000,0.000000,0.000000}%
\pgfsetfillcolor{currentfill}%
\pgfsetlinewidth{0.602250pt}%
\definecolor{currentstroke}{rgb}{0.000000,0.000000,0.000000}%
\pgfsetstrokecolor{currentstroke}%
\pgfsetdash{}{0pt}%
\pgfsys@defobject{currentmarker}{\pgfqpoint{-0.027778in}{0.000000in}}{\pgfqpoint{-0.000000in}{0.000000in}}{%
\pgfpathmoveto{\pgfqpoint{-0.000000in}{0.000000in}}%
\pgfpathlineto{\pgfqpoint{-0.027778in}{0.000000in}}%
\pgfusepath{stroke,fill}%
}%
\begin{pgfscope}%
\pgfsys@transformshift{0.800000in}{0.981963in}%
\pgfsys@useobject{currentmarker}{}%
\end{pgfscope}%
\end{pgfscope}%
\begin{pgfscope}%
\pgfsetbuttcap%
\pgfsetroundjoin%
\definecolor{currentfill}{rgb}{0.000000,0.000000,0.000000}%
\pgfsetfillcolor{currentfill}%
\pgfsetlinewidth{0.602250pt}%
\definecolor{currentstroke}{rgb}{0.000000,0.000000,0.000000}%
\pgfsetstrokecolor{currentstroke}%
\pgfsetdash{}{0pt}%
\pgfsys@defobject{currentmarker}{\pgfqpoint{-0.027778in}{0.000000in}}{\pgfqpoint{-0.000000in}{0.000000in}}{%
\pgfpathmoveto{\pgfqpoint{-0.000000in}{0.000000in}}%
\pgfpathlineto{\pgfqpoint{-0.027778in}{0.000000in}}%
\pgfusepath{stroke,fill}%
}%
\begin{pgfscope}%
\pgfsys@transformshift{0.800000in}{1.016678in}%
\pgfsys@useobject{currentmarker}{}%
\end{pgfscope}%
\end{pgfscope}%
\begin{pgfscope}%
\pgfsetbuttcap%
\pgfsetroundjoin%
\definecolor{currentfill}{rgb}{0.000000,0.000000,0.000000}%
\pgfsetfillcolor{currentfill}%
\pgfsetlinewidth{0.602250pt}%
\definecolor{currentstroke}{rgb}{0.000000,0.000000,0.000000}%
\pgfsetstrokecolor{currentstroke}%
\pgfsetdash{}{0pt}%
\pgfsys@defobject{currentmarker}{\pgfqpoint{-0.027778in}{0.000000in}}{\pgfqpoint{-0.000000in}{0.000000in}}{%
\pgfpathmoveto{\pgfqpoint{-0.000000in}{0.000000in}}%
\pgfpathlineto{\pgfqpoint{-0.027778in}{0.000000in}}%
\pgfusepath{stroke,fill}%
}%
\begin{pgfscope}%
\pgfsys@transformshift{0.800000in}{1.252031in}%
\pgfsys@useobject{currentmarker}{}%
\end{pgfscope}%
\end{pgfscope}%
\begin{pgfscope}%
\pgfsetbuttcap%
\pgfsetroundjoin%
\definecolor{currentfill}{rgb}{0.000000,0.000000,0.000000}%
\pgfsetfillcolor{currentfill}%
\pgfsetlinewidth{0.602250pt}%
\definecolor{currentstroke}{rgb}{0.000000,0.000000,0.000000}%
\pgfsetstrokecolor{currentstroke}%
\pgfsetdash{}{0pt}%
\pgfsys@defobject{currentmarker}{\pgfqpoint{-0.027778in}{0.000000in}}{\pgfqpoint{-0.000000in}{0.000000in}}{%
\pgfpathmoveto{\pgfqpoint{-0.000000in}{0.000000in}}%
\pgfpathlineto{\pgfqpoint{-0.027778in}{0.000000in}}%
\pgfusepath{stroke,fill}%
}%
\begin{pgfscope}%
\pgfsys@transformshift{0.800000in}{1.371538in}%
\pgfsys@useobject{currentmarker}{}%
\end{pgfscope}%
\end{pgfscope}%
\begin{pgfscope}%
\pgfsetbuttcap%
\pgfsetroundjoin%
\definecolor{currentfill}{rgb}{0.000000,0.000000,0.000000}%
\pgfsetfillcolor{currentfill}%
\pgfsetlinewidth{0.602250pt}%
\definecolor{currentstroke}{rgb}{0.000000,0.000000,0.000000}%
\pgfsetstrokecolor{currentstroke}%
\pgfsetdash{}{0pt}%
\pgfsys@defobject{currentmarker}{\pgfqpoint{-0.027778in}{0.000000in}}{\pgfqpoint{-0.000000in}{0.000000in}}{%
\pgfpathmoveto{\pgfqpoint{-0.000000in}{0.000000in}}%
\pgfpathlineto{\pgfqpoint{-0.027778in}{0.000000in}}%
\pgfusepath{stroke,fill}%
}%
\begin{pgfscope}%
\pgfsys@transformshift{0.800000in}{1.456330in}%
\pgfsys@useobject{currentmarker}{}%
\end{pgfscope}%
\end{pgfscope}%
\begin{pgfscope}%
\pgfsetbuttcap%
\pgfsetroundjoin%
\definecolor{currentfill}{rgb}{0.000000,0.000000,0.000000}%
\pgfsetfillcolor{currentfill}%
\pgfsetlinewidth{0.602250pt}%
\definecolor{currentstroke}{rgb}{0.000000,0.000000,0.000000}%
\pgfsetstrokecolor{currentstroke}%
\pgfsetdash{}{0pt}%
\pgfsys@defobject{currentmarker}{\pgfqpoint{-0.027778in}{0.000000in}}{\pgfqpoint{-0.000000in}{0.000000in}}{%
\pgfpathmoveto{\pgfqpoint{-0.000000in}{0.000000in}}%
\pgfpathlineto{\pgfqpoint{-0.027778in}{0.000000in}}%
\pgfusepath{stroke,fill}%
}%
\begin{pgfscope}%
\pgfsys@transformshift{0.800000in}{1.522099in}%
\pgfsys@useobject{currentmarker}{}%
\end{pgfscope}%
\end{pgfscope}%
\begin{pgfscope}%
\pgfsetbuttcap%
\pgfsetroundjoin%
\definecolor{currentfill}{rgb}{0.000000,0.000000,0.000000}%
\pgfsetfillcolor{currentfill}%
\pgfsetlinewidth{0.602250pt}%
\definecolor{currentstroke}{rgb}{0.000000,0.000000,0.000000}%
\pgfsetstrokecolor{currentstroke}%
\pgfsetdash{}{0pt}%
\pgfsys@defobject{currentmarker}{\pgfqpoint{-0.027778in}{0.000000in}}{\pgfqpoint{-0.000000in}{0.000000in}}{%
\pgfpathmoveto{\pgfqpoint{-0.000000in}{0.000000in}}%
\pgfpathlineto{\pgfqpoint{-0.027778in}{0.000000in}}%
\pgfusepath{stroke,fill}%
}%
\begin{pgfscope}%
\pgfsys@transformshift{0.800000in}{1.575837in}%
\pgfsys@useobject{currentmarker}{}%
\end{pgfscope}%
\end{pgfscope}%
\begin{pgfscope}%
\pgfsetbuttcap%
\pgfsetroundjoin%
\definecolor{currentfill}{rgb}{0.000000,0.000000,0.000000}%
\pgfsetfillcolor{currentfill}%
\pgfsetlinewidth{0.602250pt}%
\definecolor{currentstroke}{rgb}{0.000000,0.000000,0.000000}%
\pgfsetstrokecolor{currentstroke}%
\pgfsetdash{}{0pt}%
\pgfsys@defobject{currentmarker}{\pgfqpoint{-0.027778in}{0.000000in}}{\pgfqpoint{-0.000000in}{0.000000in}}{%
\pgfpathmoveto{\pgfqpoint{-0.000000in}{0.000000in}}%
\pgfpathlineto{\pgfqpoint{-0.027778in}{0.000000in}}%
\pgfusepath{stroke,fill}%
}%
\begin{pgfscope}%
\pgfsys@transformshift{0.800000in}{1.621271in}%
\pgfsys@useobject{currentmarker}{}%
\end{pgfscope}%
\end{pgfscope}%
\begin{pgfscope}%
\pgfsetbuttcap%
\pgfsetroundjoin%
\definecolor{currentfill}{rgb}{0.000000,0.000000,0.000000}%
\pgfsetfillcolor{currentfill}%
\pgfsetlinewidth{0.602250pt}%
\definecolor{currentstroke}{rgb}{0.000000,0.000000,0.000000}%
\pgfsetstrokecolor{currentstroke}%
\pgfsetdash{}{0pt}%
\pgfsys@defobject{currentmarker}{\pgfqpoint{-0.027778in}{0.000000in}}{\pgfqpoint{-0.000000in}{0.000000in}}{%
\pgfpathmoveto{\pgfqpoint{-0.000000in}{0.000000in}}%
\pgfpathlineto{\pgfqpoint{-0.027778in}{0.000000in}}%
\pgfusepath{stroke,fill}%
}%
\begin{pgfscope}%
\pgfsys@transformshift{0.800000in}{1.660628in}%
\pgfsys@useobject{currentmarker}{}%
\end{pgfscope}%
\end{pgfscope}%
\begin{pgfscope}%
\pgfsetbuttcap%
\pgfsetroundjoin%
\definecolor{currentfill}{rgb}{0.000000,0.000000,0.000000}%
\pgfsetfillcolor{currentfill}%
\pgfsetlinewidth{0.602250pt}%
\definecolor{currentstroke}{rgb}{0.000000,0.000000,0.000000}%
\pgfsetstrokecolor{currentstroke}%
\pgfsetdash{}{0pt}%
\pgfsys@defobject{currentmarker}{\pgfqpoint{-0.027778in}{0.000000in}}{\pgfqpoint{-0.000000in}{0.000000in}}{%
\pgfpathmoveto{\pgfqpoint{-0.000000in}{0.000000in}}%
\pgfpathlineto{\pgfqpoint{-0.027778in}{0.000000in}}%
\pgfusepath{stroke,fill}%
}%
\begin{pgfscope}%
\pgfsys@transformshift{0.800000in}{1.695344in}%
\pgfsys@useobject{currentmarker}{}%
\end{pgfscope}%
\end{pgfscope}%
\begin{pgfscope}%
\pgfsetbuttcap%
\pgfsetroundjoin%
\definecolor{currentfill}{rgb}{0.000000,0.000000,0.000000}%
\pgfsetfillcolor{currentfill}%
\pgfsetlinewidth{0.602250pt}%
\definecolor{currentstroke}{rgb}{0.000000,0.000000,0.000000}%
\pgfsetstrokecolor{currentstroke}%
\pgfsetdash{}{0pt}%
\pgfsys@defobject{currentmarker}{\pgfqpoint{-0.027778in}{0.000000in}}{\pgfqpoint{-0.000000in}{0.000000in}}{%
\pgfpathmoveto{\pgfqpoint{-0.000000in}{0.000000in}}%
\pgfpathlineto{\pgfqpoint{-0.027778in}{0.000000in}}%
\pgfusepath{stroke,fill}%
}%
\begin{pgfscope}%
\pgfsys@transformshift{0.800000in}{1.930696in}%
\pgfsys@useobject{currentmarker}{}%
\end{pgfscope}%
\end{pgfscope}%
\begin{pgfscope}%
\pgfsetbuttcap%
\pgfsetroundjoin%
\definecolor{currentfill}{rgb}{0.000000,0.000000,0.000000}%
\pgfsetfillcolor{currentfill}%
\pgfsetlinewidth{0.602250pt}%
\definecolor{currentstroke}{rgb}{0.000000,0.000000,0.000000}%
\pgfsetstrokecolor{currentstroke}%
\pgfsetdash{}{0pt}%
\pgfsys@defobject{currentmarker}{\pgfqpoint{-0.027778in}{0.000000in}}{\pgfqpoint{-0.000000in}{0.000000in}}{%
\pgfpathmoveto{\pgfqpoint{-0.000000in}{0.000000in}}%
\pgfpathlineto{\pgfqpoint{-0.027778in}{0.000000in}}%
\pgfusepath{stroke,fill}%
}%
\begin{pgfscope}%
\pgfsys@transformshift{0.800000in}{2.050203in}%
\pgfsys@useobject{currentmarker}{}%
\end{pgfscope}%
\end{pgfscope}%
\begin{pgfscope}%
\pgfsetbuttcap%
\pgfsetroundjoin%
\definecolor{currentfill}{rgb}{0.000000,0.000000,0.000000}%
\pgfsetfillcolor{currentfill}%
\pgfsetlinewidth{0.602250pt}%
\definecolor{currentstroke}{rgb}{0.000000,0.000000,0.000000}%
\pgfsetstrokecolor{currentstroke}%
\pgfsetdash{}{0pt}%
\pgfsys@defobject{currentmarker}{\pgfqpoint{-0.027778in}{0.000000in}}{\pgfqpoint{-0.000000in}{0.000000in}}{%
\pgfpathmoveto{\pgfqpoint{-0.000000in}{0.000000in}}%
\pgfpathlineto{\pgfqpoint{-0.027778in}{0.000000in}}%
\pgfusepath{stroke,fill}%
}%
\begin{pgfscope}%
\pgfsys@transformshift{0.800000in}{2.134995in}%
\pgfsys@useobject{currentmarker}{}%
\end{pgfscope}%
\end{pgfscope}%
\begin{pgfscope}%
\pgfsetbuttcap%
\pgfsetroundjoin%
\definecolor{currentfill}{rgb}{0.000000,0.000000,0.000000}%
\pgfsetfillcolor{currentfill}%
\pgfsetlinewidth{0.602250pt}%
\definecolor{currentstroke}{rgb}{0.000000,0.000000,0.000000}%
\pgfsetstrokecolor{currentstroke}%
\pgfsetdash{}{0pt}%
\pgfsys@defobject{currentmarker}{\pgfqpoint{-0.027778in}{0.000000in}}{\pgfqpoint{-0.000000in}{0.000000in}}{%
\pgfpathmoveto{\pgfqpoint{-0.000000in}{0.000000in}}%
\pgfpathlineto{\pgfqpoint{-0.027778in}{0.000000in}}%
\pgfusepath{stroke,fill}%
}%
\begin{pgfscope}%
\pgfsys@transformshift{0.800000in}{2.200764in}%
\pgfsys@useobject{currentmarker}{}%
\end{pgfscope}%
\end{pgfscope}%
\begin{pgfscope}%
\pgfsetbuttcap%
\pgfsetroundjoin%
\definecolor{currentfill}{rgb}{0.000000,0.000000,0.000000}%
\pgfsetfillcolor{currentfill}%
\pgfsetlinewidth{0.602250pt}%
\definecolor{currentstroke}{rgb}{0.000000,0.000000,0.000000}%
\pgfsetstrokecolor{currentstroke}%
\pgfsetdash{}{0pt}%
\pgfsys@defobject{currentmarker}{\pgfqpoint{-0.027778in}{0.000000in}}{\pgfqpoint{-0.000000in}{0.000000in}}{%
\pgfpathmoveto{\pgfqpoint{-0.000000in}{0.000000in}}%
\pgfpathlineto{\pgfqpoint{-0.027778in}{0.000000in}}%
\pgfusepath{stroke,fill}%
}%
\begin{pgfscope}%
\pgfsys@transformshift{0.800000in}{2.254502in}%
\pgfsys@useobject{currentmarker}{}%
\end{pgfscope}%
\end{pgfscope}%
\begin{pgfscope}%
\pgfsetbuttcap%
\pgfsetroundjoin%
\definecolor{currentfill}{rgb}{0.000000,0.000000,0.000000}%
\pgfsetfillcolor{currentfill}%
\pgfsetlinewidth{0.602250pt}%
\definecolor{currentstroke}{rgb}{0.000000,0.000000,0.000000}%
\pgfsetstrokecolor{currentstroke}%
\pgfsetdash{}{0pt}%
\pgfsys@defobject{currentmarker}{\pgfqpoint{-0.027778in}{0.000000in}}{\pgfqpoint{-0.000000in}{0.000000in}}{%
\pgfpathmoveto{\pgfqpoint{-0.000000in}{0.000000in}}%
\pgfpathlineto{\pgfqpoint{-0.027778in}{0.000000in}}%
\pgfusepath{stroke,fill}%
}%
\begin{pgfscope}%
\pgfsys@transformshift{0.800000in}{2.299936in}%
\pgfsys@useobject{currentmarker}{}%
\end{pgfscope}%
\end{pgfscope}%
\begin{pgfscope}%
\pgfsetbuttcap%
\pgfsetroundjoin%
\definecolor{currentfill}{rgb}{0.000000,0.000000,0.000000}%
\pgfsetfillcolor{currentfill}%
\pgfsetlinewidth{0.602250pt}%
\definecolor{currentstroke}{rgb}{0.000000,0.000000,0.000000}%
\pgfsetstrokecolor{currentstroke}%
\pgfsetdash{}{0pt}%
\pgfsys@defobject{currentmarker}{\pgfqpoint{-0.027778in}{0.000000in}}{\pgfqpoint{-0.000000in}{0.000000in}}{%
\pgfpathmoveto{\pgfqpoint{-0.000000in}{0.000000in}}%
\pgfpathlineto{\pgfqpoint{-0.027778in}{0.000000in}}%
\pgfusepath{stroke,fill}%
}%
\begin{pgfscope}%
\pgfsys@transformshift{0.800000in}{2.339293in}%
\pgfsys@useobject{currentmarker}{}%
\end{pgfscope}%
\end{pgfscope}%
\begin{pgfscope}%
\pgfsetbuttcap%
\pgfsetroundjoin%
\definecolor{currentfill}{rgb}{0.000000,0.000000,0.000000}%
\pgfsetfillcolor{currentfill}%
\pgfsetlinewidth{0.602250pt}%
\definecolor{currentstroke}{rgb}{0.000000,0.000000,0.000000}%
\pgfsetstrokecolor{currentstroke}%
\pgfsetdash{}{0pt}%
\pgfsys@defobject{currentmarker}{\pgfqpoint{-0.027778in}{0.000000in}}{\pgfqpoint{-0.000000in}{0.000000in}}{%
\pgfpathmoveto{\pgfqpoint{-0.000000in}{0.000000in}}%
\pgfpathlineto{\pgfqpoint{-0.027778in}{0.000000in}}%
\pgfusepath{stroke,fill}%
}%
\begin{pgfscope}%
\pgfsys@transformshift{0.800000in}{2.374009in}%
\pgfsys@useobject{currentmarker}{}%
\end{pgfscope}%
\end{pgfscope}%
\begin{pgfscope}%
\pgfsetbuttcap%
\pgfsetroundjoin%
\definecolor{currentfill}{rgb}{0.000000,0.000000,0.000000}%
\pgfsetfillcolor{currentfill}%
\pgfsetlinewidth{0.602250pt}%
\definecolor{currentstroke}{rgb}{0.000000,0.000000,0.000000}%
\pgfsetstrokecolor{currentstroke}%
\pgfsetdash{}{0pt}%
\pgfsys@defobject{currentmarker}{\pgfqpoint{-0.027778in}{0.000000in}}{\pgfqpoint{-0.000000in}{0.000000in}}{%
\pgfpathmoveto{\pgfqpoint{-0.000000in}{0.000000in}}%
\pgfpathlineto{\pgfqpoint{-0.027778in}{0.000000in}}%
\pgfusepath{stroke,fill}%
}%
\begin{pgfscope}%
\pgfsys@transformshift{0.800000in}{2.609361in}%
\pgfsys@useobject{currentmarker}{}%
\end{pgfscope}%
\end{pgfscope}%
\begin{pgfscope}%
\pgfsetbuttcap%
\pgfsetroundjoin%
\definecolor{currentfill}{rgb}{0.000000,0.000000,0.000000}%
\pgfsetfillcolor{currentfill}%
\pgfsetlinewidth{0.602250pt}%
\definecolor{currentstroke}{rgb}{0.000000,0.000000,0.000000}%
\pgfsetstrokecolor{currentstroke}%
\pgfsetdash{}{0pt}%
\pgfsys@defobject{currentmarker}{\pgfqpoint{-0.027778in}{0.000000in}}{\pgfqpoint{-0.000000in}{0.000000in}}{%
\pgfpathmoveto{\pgfqpoint{-0.000000in}{0.000000in}}%
\pgfpathlineto{\pgfqpoint{-0.027778in}{0.000000in}}%
\pgfusepath{stroke,fill}%
}%
\begin{pgfscope}%
\pgfsys@transformshift{0.800000in}{2.728868in}%
\pgfsys@useobject{currentmarker}{}%
\end{pgfscope}%
\end{pgfscope}%
\begin{pgfscope}%
\pgfsetbuttcap%
\pgfsetroundjoin%
\definecolor{currentfill}{rgb}{0.000000,0.000000,0.000000}%
\pgfsetfillcolor{currentfill}%
\pgfsetlinewidth{0.602250pt}%
\definecolor{currentstroke}{rgb}{0.000000,0.000000,0.000000}%
\pgfsetstrokecolor{currentstroke}%
\pgfsetdash{}{0pt}%
\pgfsys@defobject{currentmarker}{\pgfqpoint{-0.027778in}{0.000000in}}{\pgfqpoint{-0.000000in}{0.000000in}}{%
\pgfpathmoveto{\pgfqpoint{-0.000000in}{0.000000in}}%
\pgfpathlineto{\pgfqpoint{-0.027778in}{0.000000in}}%
\pgfusepath{stroke,fill}%
}%
\begin{pgfscope}%
\pgfsys@transformshift{0.800000in}{2.813660in}%
\pgfsys@useobject{currentmarker}{}%
\end{pgfscope}%
\end{pgfscope}%
\begin{pgfscope}%
\pgfsetbuttcap%
\pgfsetroundjoin%
\definecolor{currentfill}{rgb}{0.000000,0.000000,0.000000}%
\pgfsetfillcolor{currentfill}%
\pgfsetlinewidth{0.602250pt}%
\definecolor{currentstroke}{rgb}{0.000000,0.000000,0.000000}%
\pgfsetstrokecolor{currentstroke}%
\pgfsetdash{}{0pt}%
\pgfsys@defobject{currentmarker}{\pgfqpoint{-0.027778in}{0.000000in}}{\pgfqpoint{-0.000000in}{0.000000in}}{%
\pgfpathmoveto{\pgfqpoint{-0.000000in}{0.000000in}}%
\pgfpathlineto{\pgfqpoint{-0.027778in}{0.000000in}}%
\pgfusepath{stroke,fill}%
}%
\begin{pgfscope}%
\pgfsys@transformshift{0.800000in}{2.879429in}%
\pgfsys@useobject{currentmarker}{}%
\end{pgfscope}%
\end{pgfscope}%
\begin{pgfscope}%
\pgfsetbuttcap%
\pgfsetroundjoin%
\definecolor{currentfill}{rgb}{0.000000,0.000000,0.000000}%
\pgfsetfillcolor{currentfill}%
\pgfsetlinewidth{0.602250pt}%
\definecolor{currentstroke}{rgb}{0.000000,0.000000,0.000000}%
\pgfsetstrokecolor{currentstroke}%
\pgfsetdash{}{0pt}%
\pgfsys@defobject{currentmarker}{\pgfqpoint{-0.027778in}{0.000000in}}{\pgfqpoint{-0.000000in}{0.000000in}}{%
\pgfpathmoveto{\pgfqpoint{-0.000000in}{0.000000in}}%
\pgfpathlineto{\pgfqpoint{-0.027778in}{0.000000in}}%
\pgfusepath{stroke,fill}%
}%
\begin{pgfscope}%
\pgfsys@transformshift{0.800000in}{2.933167in}%
\pgfsys@useobject{currentmarker}{}%
\end{pgfscope}%
\end{pgfscope}%
\begin{pgfscope}%
\pgfsetbuttcap%
\pgfsetroundjoin%
\definecolor{currentfill}{rgb}{0.000000,0.000000,0.000000}%
\pgfsetfillcolor{currentfill}%
\pgfsetlinewidth{0.602250pt}%
\definecolor{currentstroke}{rgb}{0.000000,0.000000,0.000000}%
\pgfsetstrokecolor{currentstroke}%
\pgfsetdash{}{0pt}%
\pgfsys@defobject{currentmarker}{\pgfqpoint{-0.027778in}{0.000000in}}{\pgfqpoint{-0.000000in}{0.000000in}}{%
\pgfpathmoveto{\pgfqpoint{-0.000000in}{0.000000in}}%
\pgfpathlineto{\pgfqpoint{-0.027778in}{0.000000in}}%
\pgfusepath{stroke,fill}%
}%
\begin{pgfscope}%
\pgfsys@transformshift{0.800000in}{2.978601in}%
\pgfsys@useobject{currentmarker}{}%
\end{pgfscope}%
\end{pgfscope}%
\begin{pgfscope}%
\pgfsetbuttcap%
\pgfsetroundjoin%
\definecolor{currentfill}{rgb}{0.000000,0.000000,0.000000}%
\pgfsetfillcolor{currentfill}%
\pgfsetlinewidth{0.602250pt}%
\definecolor{currentstroke}{rgb}{0.000000,0.000000,0.000000}%
\pgfsetstrokecolor{currentstroke}%
\pgfsetdash{}{0pt}%
\pgfsys@defobject{currentmarker}{\pgfqpoint{-0.027778in}{0.000000in}}{\pgfqpoint{-0.000000in}{0.000000in}}{%
\pgfpathmoveto{\pgfqpoint{-0.000000in}{0.000000in}}%
\pgfpathlineto{\pgfqpoint{-0.027778in}{0.000000in}}%
\pgfusepath{stroke,fill}%
}%
\begin{pgfscope}%
\pgfsys@transformshift{0.800000in}{3.017958in}%
\pgfsys@useobject{currentmarker}{}%
\end{pgfscope}%
\end{pgfscope}%
\begin{pgfscope}%
\pgfsetbuttcap%
\pgfsetroundjoin%
\definecolor{currentfill}{rgb}{0.000000,0.000000,0.000000}%
\pgfsetfillcolor{currentfill}%
\pgfsetlinewidth{0.602250pt}%
\definecolor{currentstroke}{rgb}{0.000000,0.000000,0.000000}%
\pgfsetstrokecolor{currentstroke}%
\pgfsetdash{}{0pt}%
\pgfsys@defobject{currentmarker}{\pgfqpoint{-0.027778in}{0.000000in}}{\pgfqpoint{-0.000000in}{0.000000in}}{%
\pgfpathmoveto{\pgfqpoint{-0.000000in}{0.000000in}}%
\pgfpathlineto{\pgfqpoint{-0.027778in}{0.000000in}}%
\pgfusepath{stroke,fill}%
}%
\begin{pgfscope}%
\pgfsys@transformshift{0.800000in}{3.052674in}%
\pgfsys@useobject{currentmarker}{}%
\end{pgfscope}%
\end{pgfscope}%
\begin{pgfscope}%
\pgfsetbuttcap%
\pgfsetroundjoin%
\definecolor{currentfill}{rgb}{0.000000,0.000000,0.000000}%
\pgfsetfillcolor{currentfill}%
\pgfsetlinewidth{0.602250pt}%
\definecolor{currentstroke}{rgb}{0.000000,0.000000,0.000000}%
\pgfsetstrokecolor{currentstroke}%
\pgfsetdash{}{0pt}%
\pgfsys@defobject{currentmarker}{\pgfqpoint{-0.027778in}{0.000000in}}{\pgfqpoint{-0.000000in}{0.000000in}}{%
\pgfpathmoveto{\pgfqpoint{-0.000000in}{0.000000in}}%
\pgfpathlineto{\pgfqpoint{-0.027778in}{0.000000in}}%
\pgfusepath{stroke,fill}%
}%
\begin{pgfscope}%
\pgfsys@transformshift{0.800000in}{3.288026in}%
\pgfsys@useobject{currentmarker}{}%
\end{pgfscope}%
\end{pgfscope}%
\begin{pgfscope}%
\pgfsetbuttcap%
\pgfsetroundjoin%
\definecolor{currentfill}{rgb}{0.000000,0.000000,0.000000}%
\pgfsetfillcolor{currentfill}%
\pgfsetlinewidth{0.602250pt}%
\definecolor{currentstroke}{rgb}{0.000000,0.000000,0.000000}%
\pgfsetstrokecolor{currentstroke}%
\pgfsetdash{}{0pt}%
\pgfsys@defobject{currentmarker}{\pgfqpoint{-0.027778in}{0.000000in}}{\pgfqpoint{-0.000000in}{0.000000in}}{%
\pgfpathmoveto{\pgfqpoint{-0.000000in}{0.000000in}}%
\pgfpathlineto{\pgfqpoint{-0.027778in}{0.000000in}}%
\pgfusepath{stroke,fill}%
}%
\begin{pgfscope}%
\pgfsys@transformshift{0.800000in}{3.407534in}%
\pgfsys@useobject{currentmarker}{}%
\end{pgfscope}%
\end{pgfscope}%
\begin{pgfscope}%
\pgfsetbuttcap%
\pgfsetroundjoin%
\definecolor{currentfill}{rgb}{0.000000,0.000000,0.000000}%
\pgfsetfillcolor{currentfill}%
\pgfsetlinewidth{0.602250pt}%
\definecolor{currentstroke}{rgb}{0.000000,0.000000,0.000000}%
\pgfsetstrokecolor{currentstroke}%
\pgfsetdash{}{0pt}%
\pgfsys@defobject{currentmarker}{\pgfqpoint{-0.027778in}{0.000000in}}{\pgfqpoint{-0.000000in}{0.000000in}}{%
\pgfpathmoveto{\pgfqpoint{-0.000000in}{0.000000in}}%
\pgfpathlineto{\pgfqpoint{-0.027778in}{0.000000in}}%
\pgfusepath{stroke,fill}%
}%
\begin{pgfscope}%
\pgfsys@transformshift{0.800000in}{3.492325in}%
\pgfsys@useobject{currentmarker}{}%
\end{pgfscope}%
\end{pgfscope}%
\begin{pgfscope}%
\pgfsetbuttcap%
\pgfsetroundjoin%
\definecolor{currentfill}{rgb}{0.000000,0.000000,0.000000}%
\pgfsetfillcolor{currentfill}%
\pgfsetlinewidth{0.602250pt}%
\definecolor{currentstroke}{rgb}{0.000000,0.000000,0.000000}%
\pgfsetstrokecolor{currentstroke}%
\pgfsetdash{}{0pt}%
\pgfsys@defobject{currentmarker}{\pgfqpoint{-0.027778in}{0.000000in}}{\pgfqpoint{-0.000000in}{0.000000in}}{%
\pgfpathmoveto{\pgfqpoint{-0.000000in}{0.000000in}}%
\pgfpathlineto{\pgfqpoint{-0.027778in}{0.000000in}}%
\pgfusepath{stroke,fill}%
}%
\begin{pgfscope}%
\pgfsys@transformshift{0.800000in}{3.558095in}%
\pgfsys@useobject{currentmarker}{}%
\end{pgfscope}%
\end{pgfscope}%
\begin{pgfscope}%
\pgfsetbuttcap%
\pgfsetroundjoin%
\definecolor{currentfill}{rgb}{0.000000,0.000000,0.000000}%
\pgfsetfillcolor{currentfill}%
\pgfsetlinewidth{0.602250pt}%
\definecolor{currentstroke}{rgb}{0.000000,0.000000,0.000000}%
\pgfsetstrokecolor{currentstroke}%
\pgfsetdash{}{0pt}%
\pgfsys@defobject{currentmarker}{\pgfqpoint{-0.027778in}{0.000000in}}{\pgfqpoint{-0.000000in}{0.000000in}}{%
\pgfpathmoveto{\pgfqpoint{-0.000000in}{0.000000in}}%
\pgfpathlineto{\pgfqpoint{-0.027778in}{0.000000in}}%
\pgfusepath{stroke,fill}%
}%
\begin{pgfscope}%
\pgfsys@transformshift{0.800000in}{3.611832in}%
\pgfsys@useobject{currentmarker}{}%
\end{pgfscope}%
\end{pgfscope}%
\begin{pgfscope}%
\pgfsetbuttcap%
\pgfsetroundjoin%
\definecolor{currentfill}{rgb}{0.000000,0.000000,0.000000}%
\pgfsetfillcolor{currentfill}%
\pgfsetlinewidth{0.602250pt}%
\definecolor{currentstroke}{rgb}{0.000000,0.000000,0.000000}%
\pgfsetstrokecolor{currentstroke}%
\pgfsetdash{}{0pt}%
\pgfsys@defobject{currentmarker}{\pgfqpoint{-0.027778in}{0.000000in}}{\pgfqpoint{-0.000000in}{0.000000in}}{%
\pgfpathmoveto{\pgfqpoint{-0.000000in}{0.000000in}}%
\pgfpathlineto{\pgfqpoint{-0.027778in}{0.000000in}}%
\pgfusepath{stroke,fill}%
}%
\begin{pgfscope}%
\pgfsys@transformshift{0.800000in}{3.657267in}%
\pgfsys@useobject{currentmarker}{}%
\end{pgfscope}%
\end{pgfscope}%
\begin{pgfscope}%
\pgfsetbuttcap%
\pgfsetroundjoin%
\definecolor{currentfill}{rgb}{0.000000,0.000000,0.000000}%
\pgfsetfillcolor{currentfill}%
\pgfsetlinewidth{0.602250pt}%
\definecolor{currentstroke}{rgb}{0.000000,0.000000,0.000000}%
\pgfsetstrokecolor{currentstroke}%
\pgfsetdash{}{0pt}%
\pgfsys@defobject{currentmarker}{\pgfqpoint{-0.027778in}{0.000000in}}{\pgfqpoint{-0.000000in}{0.000000in}}{%
\pgfpathmoveto{\pgfqpoint{-0.000000in}{0.000000in}}%
\pgfpathlineto{\pgfqpoint{-0.027778in}{0.000000in}}%
\pgfusepath{stroke,fill}%
}%
\begin{pgfscope}%
\pgfsys@transformshift{0.800000in}{3.696624in}%
\pgfsys@useobject{currentmarker}{}%
\end{pgfscope}%
\end{pgfscope}%
\begin{pgfscope}%
\pgfsetbuttcap%
\pgfsetroundjoin%
\definecolor{currentfill}{rgb}{0.000000,0.000000,0.000000}%
\pgfsetfillcolor{currentfill}%
\pgfsetlinewidth{0.602250pt}%
\definecolor{currentstroke}{rgb}{0.000000,0.000000,0.000000}%
\pgfsetstrokecolor{currentstroke}%
\pgfsetdash{}{0pt}%
\pgfsys@defobject{currentmarker}{\pgfqpoint{-0.027778in}{0.000000in}}{\pgfqpoint{-0.000000in}{0.000000in}}{%
\pgfpathmoveto{\pgfqpoint{-0.000000in}{0.000000in}}%
\pgfpathlineto{\pgfqpoint{-0.027778in}{0.000000in}}%
\pgfusepath{stroke,fill}%
}%
\begin{pgfscope}%
\pgfsys@transformshift{0.800000in}{3.731339in}%
\pgfsys@useobject{currentmarker}{}%
\end{pgfscope}%
\end{pgfscope}%
\begin{pgfscope}%
\pgfsetbuttcap%
\pgfsetroundjoin%
\definecolor{currentfill}{rgb}{0.000000,0.000000,0.000000}%
\pgfsetfillcolor{currentfill}%
\pgfsetlinewidth{0.602250pt}%
\definecolor{currentstroke}{rgb}{0.000000,0.000000,0.000000}%
\pgfsetstrokecolor{currentstroke}%
\pgfsetdash{}{0pt}%
\pgfsys@defobject{currentmarker}{\pgfqpoint{-0.027778in}{0.000000in}}{\pgfqpoint{-0.000000in}{0.000000in}}{%
\pgfpathmoveto{\pgfqpoint{-0.000000in}{0.000000in}}%
\pgfpathlineto{\pgfqpoint{-0.027778in}{0.000000in}}%
\pgfusepath{stroke,fill}%
}%
\begin{pgfscope}%
\pgfsys@transformshift{0.800000in}{3.966692in}%
\pgfsys@useobject{currentmarker}{}%
\end{pgfscope}%
\end{pgfscope}%
\begin{pgfscope}%
\pgfsetbuttcap%
\pgfsetroundjoin%
\definecolor{currentfill}{rgb}{0.000000,0.000000,0.000000}%
\pgfsetfillcolor{currentfill}%
\pgfsetlinewidth{0.602250pt}%
\definecolor{currentstroke}{rgb}{0.000000,0.000000,0.000000}%
\pgfsetstrokecolor{currentstroke}%
\pgfsetdash{}{0pt}%
\pgfsys@defobject{currentmarker}{\pgfqpoint{-0.027778in}{0.000000in}}{\pgfqpoint{-0.000000in}{0.000000in}}{%
\pgfpathmoveto{\pgfqpoint{-0.000000in}{0.000000in}}%
\pgfpathlineto{\pgfqpoint{-0.027778in}{0.000000in}}%
\pgfusepath{stroke,fill}%
}%
\begin{pgfscope}%
\pgfsys@transformshift{0.800000in}{4.086199in}%
\pgfsys@useobject{currentmarker}{}%
\end{pgfscope}%
\end{pgfscope}%
\begin{pgfscope}%
\pgfsetbuttcap%
\pgfsetroundjoin%
\definecolor{currentfill}{rgb}{0.000000,0.000000,0.000000}%
\pgfsetfillcolor{currentfill}%
\pgfsetlinewidth{0.602250pt}%
\definecolor{currentstroke}{rgb}{0.000000,0.000000,0.000000}%
\pgfsetstrokecolor{currentstroke}%
\pgfsetdash{}{0pt}%
\pgfsys@defobject{currentmarker}{\pgfqpoint{-0.027778in}{0.000000in}}{\pgfqpoint{-0.000000in}{0.000000in}}{%
\pgfpathmoveto{\pgfqpoint{-0.000000in}{0.000000in}}%
\pgfpathlineto{\pgfqpoint{-0.027778in}{0.000000in}}%
\pgfusepath{stroke,fill}%
}%
\begin{pgfscope}%
\pgfsys@transformshift{0.800000in}{4.170990in}%
\pgfsys@useobject{currentmarker}{}%
\end{pgfscope}%
\end{pgfscope}%
\begin{pgfscope}%
\definecolor{textcolor}{rgb}{0.000000,0.000000,0.000000}%
\pgfsetstrokecolor{textcolor}%
\pgfsetfillcolor{textcolor}%
\pgftext[x=0.446026in,y=2.376000in,,bottom,rotate=90.000000]{\color{textcolor}\rmfamily\fontsize{10.000000}{12.000000}\selectfont BASIC OPERATIONS in log}%
\end{pgfscope}%
\begin{pgfscope}%
\pgfpathrectangle{\pgfqpoint{0.800000in}{0.528000in}}{\pgfqpoint{4.960000in}{3.696000in}}%
\pgfusepath{clip}%
\pgfsetrectcap%
\pgfsetroundjoin%
\pgfsetlinewidth{1.505625pt}%
\definecolor{currentstroke}{rgb}{0.121569,0.466667,0.705882}%
\pgfsetstrokecolor{currentstroke}%
\pgfsetdash{}{0pt}%
\pgfpathmoveto{\pgfqpoint{1.025455in}{0.696000in}}%
\pgfpathlineto{\pgfqpoint{1.071466in}{0.948649in}}%
\pgfpathlineto{\pgfqpoint{1.117477in}{1.080476in}}%
\pgfpathlineto{\pgfqpoint{1.163488in}{1.177948in}}%
\pgfpathlineto{\pgfqpoint{1.209499in}{1.247995in}}%
\pgfpathlineto{\pgfqpoint{1.255510in}{1.317163in}}%
\pgfpathlineto{\pgfqpoint{1.301521in}{1.372744in}}%
\pgfpathlineto{\pgfqpoint{1.347532in}{1.413636in}}%
\pgfpathlineto{\pgfqpoint{1.393544in}{1.454430in}}%
\pgfpathlineto{\pgfqpoint{1.439555in}{1.495601in}}%
\pgfpathlineto{\pgfqpoint{1.485566in}{1.521845in}}%
\pgfpathlineto{\pgfqpoint{1.531577in}{1.590039in}}%
\pgfpathlineto{\pgfqpoint{1.577588in}{1.577614in}}%
\pgfpathlineto{\pgfqpoint{1.623599in}{1.590339in}}%
\pgfpathlineto{\pgfqpoint{1.669610in}{1.621090in}}%
\pgfpathlineto{\pgfqpoint{1.715622in}{1.645425in}}%
\pgfpathlineto{\pgfqpoint{1.761633in}{1.664224in}}%
\pgfpathlineto{\pgfqpoint{1.807644in}{1.690626in}}%
\pgfpathlineto{\pgfqpoint{1.853655in}{1.707011in}}%
\pgfpathlineto{\pgfqpoint{1.899666in}{1.724816in}}%
\pgfpathlineto{\pgfqpoint{1.945677in}{1.749406in}}%
\pgfpathlineto{\pgfqpoint{1.991688in}{1.754511in}}%
\pgfpathlineto{\pgfqpoint{2.037699in}{1.772393in}}%
\pgfpathlineto{\pgfqpoint{2.083711in}{1.794408in}}%
\pgfpathlineto{\pgfqpoint{2.129722in}{1.792339in}}%
\pgfpathlineto{\pgfqpoint{2.175733in}{1.822578in}}%
\pgfpathlineto{\pgfqpoint{2.221744in}{1.823618in}}%
\pgfpathlineto{\pgfqpoint{2.267755in}{1.846337in}}%
\pgfpathlineto{\pgfqpoint{2.313766in}{1.846895in}}%
\pgfpathlineto{\pgfqpoint{2.359777in}{1.857950in}}%
\pgfpathlineto{\pgfqpoint{2.405788in}{1.870834in}}%
\pgfpathlineto{\pgfqpoint{2.451800in}{1.869340in}}%
\pgfpathlineto{\pgfqpoint{2.497811in}{1.914322in}}%
\pgfpathlineto{\pgfqpoint{2.543822in}{1.906183in}}%
\pgfpathlineto{\pgfqpoint{2.589833in}{1.920615in}}%
\pgfpathlineto{\pgfqpoint{2.635844in}{1.917651in}}%
\pgfpathlineto{\pgfqpoint{2.681855in}{1.939567in}}%
\pgfpathlineto{\pgfqpoint{2.727866in}{1.938573in}}%
\pgfpathlineto{\pgfqpoint{2.773878in}{1.945450in}}%
\pgfpathlineto{\pgfqpoint{2.819889in}{1.968113in}}%
\pgfpathlineto{\pgfqpoint{2.865900in}{1.964054in}}%
\pgfpathlineto{\pgfqpoint{2.911911in}{1.964968in}}%
\pgfpathlineto{\pgfqpoint{2.957922in}{1.976094in}}%
\pgfpathlineto{\pgfqpoint{3.003933in}{1.981913in}}%
\pgfpathlineto{\pgfqpoint{3.049944in}{2.010168in}}%
\pgfpathlineto{\pgfqpoint{3.095955in}{2.009269in}}%
\pgfpathlineto{\pgfqpoint{3.141967in}{2.014200in}}%
\pgfpathlineto{\pgfqpoint{3.187978in}{2.036683in}}%
\pgfpathlineto{\pgfqpoint{3.233989in}{2.039822in}}%
\pgfpathlineto{\pgfqpoint{3.280000in}{2.033685in}}%
\pgfpathlineto{\pgfqpoint{3.326011in}{2.042806in}}%
\pgfpathlineto{\pgfqpoint{3.372022in}{2.036609in}}%
\pgfpathlineto{\pgfqpoint{3.418033in}{2.043470in}}%
\pgfpathlineto{\pgfqpoint{3.464045in}{2.046320in}}%
\pgfpathlineto{\pgfqpoint{3.510056in}{2.070417in}}%
\pgfpathlineto{\pgfqpoint{3.556067in}{2.061776in}}%
\pgfpathlineto{\pgfqpoint{3.602078in}{2.068698in}}%
\pgfpathlineto{\pgfqpoint{3.648089in}{2.091853in}}%
\pgfpathlineto{\pgfqpoint{3.694100in}{2.083143in}}%
\pgfpathlineto{\pgfqpoint{3.740111in}{2.090711in}}%
\pgfpathlineto{\pgfqpoint{3.786122in}{2.092999in}}%
\pgfpathlineto{\pgfqpoint{3.832134in}{2.093361in}}%
\pgfpathlineto{\pgfqpoint{3.878145in}{2.105200in}}%
\pgfpathlineto{\pgfqpoint{3.924156in}{2.111570in}}%
\pgfpathlineto{\pgfqpoint{3.970167in}{2.142708in}}%
\pgfpathlineto{\pgfqpoint{4.016178in}{2.112386in}}%
\pgfpathlineto{\pgfqpoint{4.062189in}{2.139630in}}%
\pgfpathlineto{\pgfqpoint{4.108200in}{2.129330in}}%
\pgfpathlineto{\pgfqpoint{4.154212in}{2.132582in}}%
\pgfpathlineto{\pgfqpoint{4.200223in}{2.133800in}}%
\pgfpathlineto{\pgfqpoint{4.246234in}{2.143534in}}%
\pgfpathlineto{\pgfqpoint{4.292245in}{2.172039in}}%
\pgfpathlineto{\pgfqpoint{4.338256in}{2.148863in}}%
\pgfpathlineto{\pgfqpoint{4.384267in}{2.150636in}}%
\pgfpathlineto{\pgfqpoint{4.430278in}{2.163182in}}%
\pgfpathlineto{\pgfqpoint{4.476289in}{2.168096in}}%
\pgfpathlineto{\pgfqpoint{4.522301in}{2.173267in}}%
\pgfpathlineto{\pgfqpoint{4.568312in}{2.178334in}}%
\pgfpathlineto{\pgfqpoint{4.614323in}{2.184324in}}%
\pgfpathlineto{\pgfqpoint{4.660334in}{2.189504in}}%
\pgfpathlineto{\pgfqpoint{4.706345in}{2.187362in}}%
\pgfpathlineto{\pgfqpoint{4.752356in}{2.206135in}}%
\pgfpathlineto{\pgfqpoint{4.798367in}{2.198766in}}%
\pgfpathlineto{\pgfqpoint{4.844378in}{2.207172in}}%
\pgfpathlineto{\pgfqpoint{4.890390in}{2.192550in}}%
\pgfpathlineto{\pgfqpoint{4.936401in}{2.206897in}}%
\pgfpathlineto{\pgfqpoint{4.982412in}{2.218225in}}%
\pgfpathlineto{\pgfqpoint{5.028423in}{2.213023in}}%
\pgfpathlineto{\pgfqpoint{5.074434in}{2.217621in}}%
\pgfpathlineto{\pgfqpoint{5.120445in}{2.227804in}}%
\pgfpathlineto{\pgfqpoint{5.166456in}{2.216837in}}%
\pgfpathlineto{\pgfqpoint{5.212468in}{2.233380in}}%
\pgfpathlineto{\pgfqpoint{5.258479in}{2.247779in}}%
\pgfpathlineto{\pgfqpoint{5.304490in}{2.239045in}}%
\pgfpathlineto{\pgfqpoint{5.350501in}{2.247414in}}%
\pgfpathlineto{\pgfqpoint{5.396512in}{2.244443in}}%
\pgfpathlineto{\pgfqpoint{5.442523in}{2.248128in}}%
\pgfpathlineto{\pgfqpoint{5.488534in}{2.273426in}}%
\pgfpathlineto{\pgfqpoint{5.534545in}{2.260277in}}%
\pgfusepath{stroke}%
\end{pgfscope}%
\begin{pgfscope}%
\pgfpathrectangle{\pgfqpoint{0.800000in}{0.528000in}}{\pgfqpoint{4.960000in}{3.696000in}}%
\pgfusepath{clip}%
\pgfsetrectcap%
\pgfsetroundjoin%
\pgfsetlinewidth{1.505625pt}%
\definecolor{currentstroke}{rgb}{1.000000,0.498039,0.054902}%
\pgfsetstrokecolor{currentstroke}%
\pgfsetdash{}{0pt}%
\pgfpathmoveto{\pgfqpoint{1.025455in}{0.929123in}}%
\pgfpathlineto{\pgfqpoint{1.071466in}{1.171007in}}%
\pgfpathlineto{\pgfqpoint{1.117477in}{1.309588in}}%
\pgfpathlineto{\pgfqpoint{1.163488in}{1.407691in}}%
\pgfpathlineto{\pgfqpoint{1.209499in}{1.481844in}}%
\pgfpathlineto{\pgfqpoint{1.255510in}{1.544930in}}%
\pgfpathlineto{\pgfqpoint{1.301521in}{1.597422in}}%
\pgfpathlineto{\pgfqpoint{1.347532in}{1.641822in}}%
\pgfpathlineto{\pgfqpoint{1.393544in}{1.680677in}}%
\pgfpathlineto{\pgfqpoint{1.439555in}{1.715282in}}%
\pgfpathlineto{\pgfqpoint{1.485566in}{1.747392in}}%
\pgfpathlineto{\pgfqpoint{1.531577in}{1.777314in}}%
\pgfpathlineto{\pgfqpoint{1.577588in}{1.804228in}}%
\pgfpathlineto{\pgfqpoint{1.623599in}{1.829262in}}%
\pgfpathlineto{\pgfqpoint{1.669610in}{1.852020in}}%
\pgfpathlineto{\pgfqpoint{1.715622in}{1.873111in}}%
\pgfpathlineto{\pgfqpoint{1.761633in}{1.892903in}}%
\pgfpathlineto{\pgfqpoint{1.807644in}{1.911625in}}%
\pgfpathlineto{\pgfqpoint{1.853655in}{1.929019in}}%
\pgfpathlineto{\pgfqpoint{1.899666in}{1.945703in}}%
\pgfpathlineto{\pgfqpoint{1.945677in}{1.962024in}}%
\pgfpathlineto{\pgfqpoint{1.991688in}{1.977803in}}%
\pgfpathlineto{\pgfqpoint{2.037699in}{1.992975in}}%
\pgfpathlineto{\pgfqpoint{2.083711in}{2.007166in}}%
\pgfpathlineto{\pgfqpoint{2.129722in}{2.020859in}}%
\pgfpathlineto{\pgfqpoint{2.175733in}{2.033896in}}%
\pgfpathlineto{\pgfqpoint{2.221744in}{2.046479in}}%
\pgfpathlineto{\pgfqpoint{2.267755in}{2.058397in}}%
\pgfpathlineto{\pgfqpoint{2.313766in}{2.069942in}}%
\pgfpathlineto{\pgfqpoint{2.359777in}{2.081018in}}%
\pgfpathlineto{\pgfqpoint{2.405788in}{2.091778in}}%
\pgfpathlineto{\pgfqpoint{2.451800in}{2.102229in}}%
\pgfpathlineto{\pgfqpoint{2.497811in}{2.112210in}}%
\pgfpathlineto{\pgfqpoint{2.543822in}{2.121906in}}%
\pgfpathlineto{\pgfqpoint{2.589833in}{2.131301in}}%
\pgfpathlineto{\pgfqpoint{2.635844in}{2.140365in}}%
\pgfpathlineto{\pgfqpoint{2.681855in}{2.149320in}}%
\pgfpathlineto{\pgfqpoint{2.727866in}{2.157783in}}%
\pgfpathlineto{\pgfqpoint{2.773878in}{2.166131in}}%
\pgfpathlineto{\pgfqpoint{2.819889in}{2.174263in}}%
\pgfpathlineto{\pgfqpoint{2.865900in}{2.182123in}}%
\pgfpathlineto{\pgfqpoint{2.911911in}{2.190296in}}%
\pgfpathlineto{\pgfqpoint{2.957922in}{2.198230in}}%
\pgfpathlineto{\pgfqpoint{3.003933in}{2.205988in}}%
\pgfpathlineto{\pgfqpoint{3.049944in}{2.213474in}}%
\pgfpathlineto{\pgfqpoint{3.095955in}{2.220855in}}%
\pgfpathlineto{\pgfqpoint{3.141967in}{2.228092in}}%
\pgfpathlineto{\pgfqpoint{3.187978in}{2.235070in}}%
\pgfpathlineto{\pgfqpoint{3.233989in}{2.241964in}}%
\pgfpathlineto{\pgfqpoint{3.280000in}{2.248619in}}%
\pgfpathlineto{\pgfqpoint{3.326011in}{2.255138in}}%
\pgfpathlineto{\pgfqpoint{3.372022in}{2.261457in}}%
\pgfpathlineto{\pgfqpoint{3.418033in}{2.267800in}}%
\pgfpathlineto{\pgfqpoint{3.464045in}{2.273920in}}%
\pgfpathlineto{\pgfqpoint{3.510056in}{2.279940in}}%
\pgfpathlineto{\pgfqpoint{3.556067in}{2.285903in}}%
\pgfpathlineto{\pgfqpoint{3.602078in}{2.291621in}}%
\pgfpathlineto{\pgfqpoint{3.648089in}{2.297306in}}%
\pgfpathlineto{\pgfqpoint{3.694100in}{2.302915in}}%
\pgfpathlineto{\pgfqpoint{3.740111in}{2.308354in}}%
\pgfpathlineto{\pgfqpoint{3.786122in}{2.313729in}}%
\pgfpathlineto{\pgfqpoint{3.832134in}{2.319070in}}%
\pgfpathlineto{\pgfqpoint{3.878145in}{2.324220in}}%
\pgfpathlineto{\pgfqpoint{3.924156in}{2.329282in}}%
\pgfpathlineto{\pgfqpoint{3.970167in}{2.334341in}}%
\pgfpathlineto{\pgfqpoint{4.016178in}{2.339238in}}%
\pgfpathlineto{\pgfqpoint{4.062189in}{2.344061in}}%
\pgfpathlineto{\pgfqpoint{4.108200in}{2.348836in}}%
\pgfpathlineto{\pgfqpoint{4.154212in}{2.353587in}}%
\pgfpathlineto{\pgfqpoint{4.200223in}{2.358230in}}%
\pgfpathlineto{\pgfqpoint{4.246234in}{2.362821in}}%
\pgfpathlineto{\pgfqpoint{4.292245in}{2.367196in}}%
\pgfpathlineto{\pgfqpoint{4.338256in}{2.371674in}}%
\pgfpathlineto{\pgfqpoint{4.384267in}{2.376008in}}%
\pgfpathlineto{\pgfqpoint{4.430278in}{2.380292in}}%
\pgfpathlineto{\pgfqpoint{4.476289in}{2.384512in}}%
\pgfpathlineto{\pgfqpoint{4.522301in}{2.388711in}}%
\pgfpathlineto{\pgfqpoint{4.568312in}{2.392840in}}%
\pgfpathlineto{\pgfqpoint{4.614323in}{2.396880in}}%
\pgfpathlineto{\pgfqpoint{4.660334in}{2.400927in}}%
\pgfpathlineto{\pgfqpoint{4.706345in}{2.404792in}}%
\pgfpathlineto{\pgfqpoint{4.752356in}{2.408713in}}%
\pgfpathlineto{\pgfqpoint{4.798367in}{2.412766in}}%
\pgfpathlineto{\pgfqpoint{4.844378in}{2.416827in}}%
\pgfpathlineto{\pgfqpoint{4.890390in}{2.420795in}}%
\pgfpathlineto{\pgfqpoint{4.936401in}{2.424703in}}%
\pgfpathlineto{\pgfqpoint{4.982412in}{2.428578in}}%
\pgfpathlineto{\pgfqpoint{5.028423in}{2.432395in}}%
\pgfpathlineto{\pgfqpoint{5.074434in}{2.436132in}}%
\pgfpathlineto{\pgfqpoint{5.120445in}{2.439825in}}%
\pgfpathlineto{\pgfqpoint{5.166456in}{2.443506in}}%
\pgfpathlineto{\pgfqpoint{5.212468in}{2.447142in}}%
\pgfpathlineto{\pgfqpoint{5.258479in}{2.450684in}}%
\pgfpathlineto{\pgfqpoint{5.304490in}{2.454232in}}%
\pgfpathlineto{\pgfqpoint{5.350501in}{2.457757in}}%
\pgfpathlineto{\pgfqpoint{5.396512in}{2.461181in}}%
\pgfpathlineto{\pgfqpoint{5.442523in}{2.464614in}}%
\pgfpathlineto{\pgfqpoint{5.488534in}{2.468026in}}%
\pgfpathlineto{\pgfqpoint{5.534545in}{2.471332in}}%
\pgfusepath{stroke}%
\end{pgfscope}%
\begin{pgfscope}%
\pgfpathrectangle{\pgfqpoint{0.800000in}{0.528000in}}{\pgfqpoint{4.960000in}{3.696000in}}%
\pgfusepath{clip}%
\pgfsetrectcap%
\pgfsetroundjoin%
\pgfsetlinewidth{1.505625pt}%
\definecolor{currentstroke}{rgb}{0.172549,0.627451,0.172549}%
\pgfsetstrokecolor{currentstroke}%
\pgfsetdash{}{0pt}%
\pgfpathmoveto{\pgfqpoint{1.025455in}{0.918396in}}%
\pgfpathlineto{\pgfqpoint{1.071466in}{1.169217in}}%
\pgfpathlineto{\pgfqpoint{1.117477in}{1.311557in}}%
\pgfpathlineto{\pgfqpoint{1.163488in}{1.411824in}}%
\pgfpathlineto{\pgfqpoint{1.209499in}{1.489159in}}%
\pgfpathlineto{\pgfqpoint{1.255510in}{1.552975in}}%
\pgfpathlineto{\pgfqpoint{1.301521in}{1.606627in}}%
\pgfpathlineto{\pgfqpoint{1.347532in}{1.652988in}}%
\pgfpathlineto{\pgfqpoint{1.393544in}{1.692930in}}%
\pgfpathlineto{\pgfqpoint{1.439555in}{1.727545in}}%
\pgfpathlineto{\pgfqpoint{1.485566in}{1.760733in}}%
\pgfpathlineto{\pgfqpoint{1.531577in}{1.790547in}}%
\pgfpathlineto{\pgfqpoint{1.577588in}{1.818079in}}%
\pgfpathlineto{\pgfqpoint{1.623599in}{1.843517in}}%
\pgfpathlineto{\pgfqpoint{1.669610in}{1.866693in}}%
\pgfpathlineto{\pgfqpoint{1.715622in}{1.888445in}}%
\pgfpathlineto{\pgfqpoint{1.761633in}{1.908637in}}%
\pgfpathlineto{\pgfqpoint{1.807644in}{1.927889in}}%
\pgfpathlineto{\pgfqpoint{1.853655in}{1.945947in}}%
\pgfpathlineto{\pgfqpoint{1.899666in}{1.962870in}}%
\pgfpathlineto{\pgfqpoint{1.945677in}{1.979225in}}%
\pgfpathlineto{\pgfqpoint{1.991688in}{1.995255in}}%
\pgfpathlineto{\pgfqpoint{2.037699in}{2.010113in}}%
\pgfpathlineto{\pgfqpoint{2.083711in}{2.024599in}}%
\pgfpathlineto{\pgfqpoint{2.129722in}{2.038437in}}%
\pgfpathlineto{\pgfqpoint{2.175733in}{2.051555in}}%
\pgfpathlineto{\pgfqpoint{2.221744in}{2.064348in}}%
\pgfpathlineto{\pgfqpoint{2.267755in}{2.076591in}}%
\pgfpathlineto{\pgfqpoint{2.313766in}{2.087981in}}%
\pgfpathlineto{\pgfqpoint{2.359777in}{2.100066in}}%
\pgfpathlineto{\pgfqpoint{2.405788in}{2.110751in}}%
\pgfpathlineto{\pgfqpoint{2.451800in}{2.121175in}}%
\pgfpathlineto{\pgfqpoint{2.497811in}{2.131354in}}%
\pgfpathlineto{\pgfqpoint{2.543822in}{2.141243in}}%
\pgfpathlineto{\pgfqpoint{2.589833in}{2.151022in}}%
\pgfpathlineto{\pgfqpoint{2.635844in}{2.160069in}}%
\pgfpathlineto{\pgfqpoint{2.681855in}{2.169521in}}%
\pgfpathlineto{\pgfqpoint{2.727866in}{2.178113in}}%
\pgfpathlineto{\pgfqpoint{2.773878in}{2.186602in}}%
\pgfpathlineto{\pgfqpoint{2.819889in}{2.194715in}}%
\pgfpathlineto{\pgfqpoint{2.865900in}{2.203148in}}%
\pgfpathlineto{\pgfqpoint{2.911911in}{2.211224in}}%
\pgfpathlineto{\pgfqpoint{2.957922in}{2.218772in}}%
\pgfpathlineto{\pgfqpoint{3.003933in}{2.226789in}}%
\pgfpathlineto{\pgfqpoint{3.049944in}{2.234433in}}%
\pgfpathlineto{\pgfqpoint{3.095955in}{2.241616in}}%
\pgfpathlineto{\pgfqpoint{3.141967in}{2.248840in}}%
\pgfpathlineto{\pgfqpoint{3.187978in}{2.255966in}}%
\pgfpathlineto{\pgfqpoint{3.233989in}{2.262730in}}%
\pgfpathlineto{\pgfqpoint{3.280000in}{2.269566in}}%
\pgfpathlineto{\pgfqpoint{3.326011in}{2.275997in}}%
\pgfpathlineto{\pgfqpoint{3.372022in}{2.282731in}}%
\pgfpathlineto{\pgfqpoint{3.418033in}{2.289292in}}%
\pgfpathlineto{\pgfqpoint{3.464045in}{2.295259in}}%
\pgfpathlineto{\pgfqpoint{3.510056in}{2.301337in}}%
\pgfpathlineto{\pgfqpoint{3.556067in}{2.307278in}}%
\pgfpathlineto{\pgfqpoint{3.602078in}{2.313277in}}%
\pgfpathlineto{\pgfqpoint{3.648089in}{2.319018in}}%
\pgfpathlineto{\pgfqpoint{3.694100in}{2.324437in}}%
\pgfpathlineto{\pgfqpoint{3.740111in}{2.330378in}}%
\pgfpathlineto{\pgfqpoint{3.786122in}{2.335420in}}%
\pgfpathlineto{\pgfqpoint{3.832134in}{2.341058in}}%
\pgfpathlineto{\pgfqpoint{3.878145in}{2.346346in}}%
\pgfpathlineto{\pgfqpoint{3.924156in}{2.351546in}}%
\pgfpathlineto{\pgfqpoint{3.970167in}{2.356504in}}%
\pgfpathlineto{\pgfqpoint{4.016178in}{2.361643in}}%
\pgfpathlineto{\pgfqpoint{4.062189in}{2.366439in}}%
\pgfpathlineto{\pgfqpoint{4.108200in}{2.371523in}}%
\pgfpathlineto{\pgfqpoint{4.154212in}{2.376316in}}%
\pgfpathlineto{\pgfqpoint{4.200223in}{2.380826in}}%
\pgfpathlineto{\pgfqpoint{4.246234in}{2.385665in}}%
\pgfpathlineto{\pgfqpoint{4.292245in}{2.390167in}}%
\pgfpathlineto{\pgfqpoint{4.338256in}{2.394563in}}%
\pgfpathlineto{\pgfqpoint{4.384267in}{2.399092in}}%
\pgfpathlineto{\pgfqpoint{4.430278in}{2.403345in}}%
\pgfpathlineto{\pgfqpoint{4.476289in}{2.407942in}}%
\pgfpathlineto{\pgfqpoint{4.522301in}{2.411927in}}%
\pgfpathlineto{\pgfqpoint{4.568312in}{2.416419in}}%
\pgfpathlineto{\pgfqpoint{4.614323in}{2.420375in}}%
\pgfpathlineto{\pgfqpoint{4.660334in}{2.424617in}}%
\pgfpathlineto{\pgfqpoint{4.706345in}{2.428625in}}%
\pgfpathlineto{\pgfqpoint{4.752356in}{2.432657in}}%
\pgfpathlineto{\pgfqpoint{4.798367in}{2.436616in}}%
\pgfpathlineto{\pgfqpoint{4.844378in}{2.440637in}}%
\pgfpathlineto{\pgfqpoint{4.890390in}{2.444559in}}%
\pgfpathlineto{\pgfqpoint{4.936401in}{2.448543in}}%
\pgfpathlineto{\pgfqpoint{4.982412in}{2.452333in}}%
\pgfpathlineto{\pgfqpoint{5.028423in}{2.455963in}}%
\pgfpathlineto{\pgfqpoint{5.074434in}{2.459953in}}%
\pgfpathlineto{\pgfqpoint{5.120445in}{2.463436in}}%
\pgfpathlineto{\pgfqpoint{5.166456in}{2.467204in}}%
\pgfpathlineto{\pgfqpoint{5.212468in}{2.470780in}}%
\pgfpathlineto{\pgfqpoint{5.258479in}{2.474489in}}%
\pgfpathlineto{\pgfqpoint{5.304490in}{2.478005in}}%
\pgfpathlineto{\pgfqpoint{5.350501in}{2.481385in}}%
\pgfpathlineto{\pgfqpoint{5.396512in}{2.484960in}}%
\pgfpathlineto{\pgfqpoint{5.442523in}{2.488319in}}%
\pgfpathlineto{\pgfqpoint{5.488534in}{2.491897in}}%
\pgfpathlineto{\pgfqpoint{5.534545in}{2.495160in}}%
\pgfusepath{stroke}%
\end{pgfscope}%
\begin{pgfscope}%
\pgfpathrectangle{\pgfqpoint{0.800000in}{0.528000in}}{\pgfqpoint{4.960000in}{3.696000in}}%
\pgfusepath{clip}%
\pgfsetrectcap%
\pgfsetroundjoin%
\pgfsetlinewidth{1.505625pt}%
\definecolor{currentstroke}{rgb}{0.839216,0.152941,0.156863}%
\pgfsetstrokecolor{currentstroke}%
\pgfsetdash{}{0pt}%
\pgfpathmoveto{\pgfqpoint{1.025455in}{1.369012in}}%
\pgfpathlineto{\pgfqpoint{1.071466in}{1.744784in}}%
\pgfpathlineto{\pgfqpoint{1.117477in}{1.986037in}}%
\pgfpathlineto{\pgfqpoint{1.163488in}{2.155944in}}%
\pgfpathlineto{\pgfqpoint{1.209499in}{2.298385in}}%
\pgfpathlineto{\pgfqpoint{1.255510in}{2.397598in}}%
\pgfpathlineto{\pgfqpoint{1.301521in}{2.500186in}}%
\pgfpathlineto{\pgfqpoint{1.347532in}{2.567636in}}%
\pgfpathlineto{\pgfqpoint{1.393544in}{2.636870in}}%
\pgfpathlineto{\pgfqpoint{1.439555in}{2.699344in}}%
\pgfpathlineto{\pgfqpoint{1.485566in}{2.754638in}}%
\pgfpathlineto{\pgfqpoint{1.531577in}{2.816440in}}%
\pgfpathlineto{\pgfqpoint{1.577588in}{2.864718in}}%
\pgfpathlineto{\pgfqpoint{1.623599in}{2.904657in}}%
\pgfpathlineto{\pgfqpoint{1.669610in}{2.944518in}}%
\pgfpathlineto{\pgfqpoint{1.715622in}{2.991335in}}%
\pgfpathlineto{\pgfqpoint{1.761633in}{3.013775in}}%
\pgfpathlineto{\pgfqpoint{1.807644in}{3.046207in}}%
\pgfpathlineto{\pgfqpoint{1.853655in}{3.077931in}}%
\pgfpathlineto{\pgfqpoint{1.899666in}{3.111352in}}%
\pgfpathlineto{\pgfqpoint{1.945677in}{3.134892in}}%
\pgfpathlineto{\pgfqpoint{1.991688in}{3.166539in}}%
\pgfpathlineto{\pgfqpoint{2.037699in}{3.194914in}}%
\pgfpathlineto{\pgfqpoint{2.083711in}{3.224180in}}%
\pgfpathlineto{\pgfqpoint{2.129722in}{3.242769in}}%
\pgfpathlineto{\pgfqpoint{2.175733in}{3.270701in}}%
\pgfpathlineto{\pgfqpoint{2.221744in}{3.282953in}}%
\pgfpathlineto{\pgfqpoint{2.267755in}{3.311290in}}%
\pgfpathlineto{\pgfqpoint{2.313766in}{3.332206in}}%
\pgfpathlineto{\pgfqpoint{2.359777in}{3.352554in}}%
\pgfpathlineto{\pgfqpoint{2.405788in}{3.370901in}}%
\pgfpathlineto{\pgfqpoint{2.451800in}{3.385596in}}%
\pgfpathlineto{\pgfqpoint{2.497811in}{3.410920in}}%
\pgfpathlineto{\pgfqpoint{2.543822in}{3.427890in}}%
\pgfpathlineto{\pgfqpoint{2.589833in}{3.440840in}}%
\pgfpathlineto{\pgfqpoint{2.635844in}{3.461001in}}%
\pgfpathlineto{\pgfqpoint{2.681855in}{3.474235in}}%
\pgfpathlineto{\pgfqpoint{2.727866in}{3.491676in}}%
\pgfpathlineto{\pgfqpoint{2.773878in}{3.507684in}}%
\pgfpathlineto{\pgfqpoint{2.819889in}{3.521995in}}%
\pgfpathlineto{\pgfqpoint{2.865900in}{3.536027in}}%
\pgfpathlineto{\pgfqpoint{2.911911in}{3.548367in}}%
\pgfpathlineto{\pgfqpoint{2.957922in}{3.560081in}}%
\pgfpathlineto{\pgfqpoint{3.003933in}{3.577253in}}%
\pgfpathlineto{\pgfqpoint{3.049944in}{3.593448in}}%
\pgfpathlineto{\pgfqpoint{3.095955in}{3.604545in}}%
\pgfpathlineto{\pgfqpoint{3.141967in}{3.614999in}}%
\pgfpathlineto{\pgfqpoint{3.187978in}{3.630940in}}%
\pgfpathlineto{\pgfqpoint{3.233989in}{3.636144in}}%
\pgfpathlineto{\pgfqpoint{3.280000in}{3.652516in}}%
\pgfpathlineto{\pgfqpoint{3.326011in}{3.666718in}}%
\pgfpathlineto{\pgfqpoint{3.372022in}{3.680516in}}%
\pgfpathlineto{\pgfqpoint{3.418033in}{3.682777in}}%
\pgfpathlineto{\pgfqpoint{3.464045in}{3.698980in}}%
\pgfpathlineto{\pgfqpoint{3.510056in}{3.703474in}}%
\pgfpathlineto{\pgfqpoint{3.556067in}{3.715821in}}%
\pgfpathlineto{\pgfqpoint{3.602078in}{3.729843in}}%
\pgfpathlineto{\pgfqpoint{3.648089in}{3.740352in}}%
\pgfpathlineto{\pgfqpoint{3.694100in}{3.750081in}}%
\pgfpathlineto{\pgfqpoint{3.740111in}{3.758095in}}%
\pgfpathlineto{\pgfqpoint{3.786122in}{3.773047in}}%
\pgfpathlineto{\pgfqpoint{3.832134in}{3.777001in}}%
\pgfpathlineto{\pgfqpoint{3.878145in}{3.788507in}}%
\pgfpathlineto{\pgfqpoint{3.924156in}{3.793291in}}%
\pgfpathlineto{\pgfqpoint{3.970167in}{3.803595in}}%
\pgfpathlineto{\pgfqpoint{4.016178in}{3.813711in}}%
\pgfpathlineto{\pgfqpoint{4.062189in}{3.823137in}}%
\pgfpathlineto{\pgfqpoint{4.108200in}{3.835196in}}%
\pgfpathlineto{\pgfqpoint{4.108200in}{3.829224in}}%
\pgfpathlineto{\pgfqpoint{4.154212in}{3.841663in}}%
\pgfpathlineto{\pgfqpoint{4.200223in}{3.851011in}}%
\pgfpathlineto{\pgfqpoint{4.246234in}{3.859120in}}%
\pgfpathlineto{\pgfqpoint{4.292245in}{3.862955in}}%
\pgfpathlineto{\pgfqpoint{4.338256in}{3.873609in}}%
\pgfpathlineto{\pgfqpoint{4.384267in}{3.887154in}}%
\pgfpathlineto{\pgfqpoint{4.430278in}{3.891117in}}%
\pgfpathlineto{\pgfqpoint{4.476289in}{3.896298in}}%
\pgfpathlineto{\pgfqpoint{4.522301in}{3.904242in}}%
\pgfpathlineto{\pgfqpoint{4.568312in}{3.913010in}}%
\pgfpathlineto{\pgfqpoint{4.614323in}{3.922065in}}%
\pgfpathlineto{\pgfqpoint{4.660334in}{3.929968in}}%
\pgfpathlineto{\pgfqpoint{4.706345in}{3.936962in}}%
\pgfpathlineto{\pgfqpoint{4.752356in}{3.944059in}}%
\pgfpathlineto{\pgfqpoint{4.798367in}{3.951204in}}%
\pgfpathlineto{\pgfqpoint{4.844378in}{3.960209in}}%
\pgfpathlineto{\pgfqpoint{4.890390in}{3.965626in}}%
\pgfpathlineto{\pgfqpoint{4.936401in}{3.971066in}}%
\pgfpathlineto{\pgfqpoint{4.982412in}{3.978685in}}%
\pgfpathlineto{\pgfqpoint{5.028423in}{3.984903in}}%
\pgfpathlineto{\pgfqpoint{5.074434in}{3.992214in}}%
\pgfpathlineto{\pgfqpoint{5.120445in}{3.998402in}}%
\pgfpathlineto{\pgfqpoint{5.166456in}{4.004451in}}%
\pgfpathlineto{\pgfqpoint{5.212468in}{4.010741in}}%
\pgfpathlineto{\pgfqpoint{5.258479in}{4.019853in}}%
\pgfpathlineto{\pgfqpoint{5.304490in}{4.023835in}}%
\pgfpathlineto{\pgfqpoint{5.350501in}{4.031551in}}%
\pgfpathlineto{\pgfqpoint{5.396512in}{4.035101in}}%
\pgfpathlineto{\pgfqpoint{5.442523in}{4.042070in}}%
\pgfpathlineto{\pgfqpoint{5.488534in}{4.050013in}}%
\pgfpathlineto{\pgfqpoint{5.534545in}{4.056000in}}%
\pgfusepath{stroke}%
\end{pgfscope}%
\begin{pgfscope}%
\pgfpathrectangle{\pgfqpoint{0.800000in}{0.528000in}}{\pgfqpoint{4.960000in}{3.696000in}}%
\pgfusepath{clip}%
\pgfsetrectcap%
\pgfsetroundjoin%
\pgfsetlinewidth{1.505625pt}%
\definecolor{currentstroke}{rgb}{0.580392,0.403922,0.741176}%
\pgfsetstrokecolor{currentstroke}%
\pgfsetdash{}{0pt}%
\pgfpathmoveto{\pgfqpoint{1.025455in}{0.736110in}}%
\pgfpathlineto{\pgfqpoint{1.071466in}{1.007633in}}%
\pgfpathlineto{\pgfqpoint{1.117477in}{1.207286in}}%
\pgfpathlineto{\pgfqpoint{1.163488in}{1.299751in}}%
\pgfpathlineto{\pgfqpoint{1.209499in}{1.414044in}}%
\pgfpathlineto{\pgfqpoint{1.255510in}{1.491699in}}%
\pgfpathlineto{\pgfqpoint{1.301521in}{1.559111in}}%
\pgfpathlineto{\pgfqpoint{1.347532in}{1.599088in}}%
\pgfpathlineto{\pgfqpoint{1.393544in}{1.661687in}}%
\pgfpathlineto{\pgfqpoint{1.439555in}{1.720524in}}%
\pgfpathlineto{\pgfqpoint{1.485566in}{1.741941in}}%
\pgfpathlineto{\pgfqpoint{1.531577in}{1.794239in}}%
\pgfpathlineto{\pgfqpoint{1.577588in}{1.819514in}}%
\pgfpathlineto{\pgfqpoint{1.623599in}{1.853790in}}%
\pgfpathlineto{\pgfqpoint{1.669610in}{1.878495in}}%
\pgfpathlineto{\pgfqpoint{1.715622in}{1.914786in}}%
\pgfpathlineto{\pgfqpoint{1.761633in}{1.949232in}}%
\pgfpathlineto{\pgfqpoint{1.807644in}{1.970144in}}%
\pgfpathlineto{\pgfqpoint{1.853655in}{2.004083in}}%
\pgfpathlineto{\pgfqpoint{1.899666in}{2.018024in}}%
\pgfpathlineto{\pgfqpoint{1.945677in}{2.034760in}}%
\pgfpathlineto{\pgfqpoint{1.991688in}{2.049245in}}%
\pgfpathlineto{\pgfqpoint{2.037699in}{2.074824in}}%
\pgfpathlineto{\pgfqpoint{2.083711in}{2.095019in}}%
\pgfpathlineto{\pgfqpoint{2.129722in}{2.104191in}}%
\pgfpathlineto{\pgfqpoint{2.175733in}{2.128057in}}%
\pgfpathlineto{\pgfqpoint{2.221744in}{2.145330in}}%
\pgfpathlineto{\pgfqpoint{2.267755in}{2.167330in}}%
\pgfpathlineto{\pgfqpoint{2.313766in}{2.173038in}}%
\pgfpathlineto{\pgfqpoint{2.359777in}{2.185874in}}%
\pgfpathlineto{\pgfqpoint{2.405788in}{2.198586in}}%
\pgfpathlineto{\pgfqpoint{2.451800in}{2.218542in}}%
\pgfpathlineto{\pgfqpoint{2.497811in}{2.234647in}}%
\pgfpathlineto{\pgfqpoint{2.543822in}{2.243527in}}%
\pgfpathlineto{\pgfqpoint{2.589833in}{2.256596in}}%
\pgfpathlineto{\pgfqpoint{2.635844in}{2.267558in}}%
\pgfpathlineto{\pgfqpoint{2.681855in}{2.284135in}}%
\pgfpathlineto{\pgfqpoint{2.727866in}{2.283282in}}%
\pgfpathlineto{\pgfqpoint{2.773878in}{2.305554in}}%
\pgfpathlineto{\pgfqpoint{2.819889in}{2.321642in}}%
\pgfpathlineto{\pgfqpoint{2.865900in}{2.317803in}}%
\pgfpathlineto{\pgfqpoint{2.911911in}{2.337005in}}%
\pgfpathlineto{\pgfqpoint{2.957922in}{2.339865in}}%
\pgfpathlineto{\pgfqpoint{3.003933in}{2.354196in}}%
\pgfpathlineto{\pgfqpoint{3.049944in}{2.363699in}}%
\pgfpathlineto{\pgfqpoint{3.095955in}{2.375171in}}%
\pgfpathlineto{\pgfqpoint{3.141967in}{2.387224in}}%
\pgfpathlineto{\pgfqpoint{3.187978in}{2.397079in}}%
\pgfpathlineto{\pgfqpoint{3.233989in}{2.401141in}}%
\pgfpathlineto{\pgfqpoint{3.280000in}{2.410768in}}%
\pgfpathlineto{\pgfqpoint{3.326011in}{2.418816in}}%
\pgfpathlineto{\pgfqpoint{3.372022in}{2.425923in}}%
\pgfpathlineto{\pgfqpoint{3.418033in}{2.440340in}}%
\pgfpathlineto{\pgfqpoint{3.464045in}{2.444886in}}%
\pgfpathlineto{\pgfqpoint{3.510056in}{2.454638in}}%
\pgfpathlineto{\pgfqpoint{3.556067in}{2.463281in}}%
\pgfpathlineto{\pgfqpoint{3.602078in}{2.467176in}}%
\pgfpathlineto{\pgfqpoint{3.648089in}{2.472260in}}%
\pgfpathlineto{\pgfqpoint{3.694100in}{2.487117in}}%
\pgfpathlineto{\pgfqpoint{3.740111in}{2.491642in}}%
\pgfpathlineto{\pgfqpoint{3.786122in}{2.495969in}}%
\pgfpathlineto{\pgfqpoint{3.832134in}{2.506351in}}%
\pgfpathlineto{\pgfqpoint{3.878145in}{2.512595in}}%
\pgfpathlineto{\pgfqpoint{3.924156in}{2.516543in}}%
\pgfpathlineto{\pgfqpoint{3.970167in}{2.524621in}}%
\pgfpathlineto{\pgfqpoint{4.016178in}{2.529996in}}%
\pgfpathlineto{\pgfqpoint{4.062189in}{2.541032in}}%
\pgfpathlineto{\pgfqpoint{4.108200in}{2.545625in}}%
\pgfpathlineto{\pgfqpoint{4.154212in}{2.552748in}}%
\pgfpathlineto{\pgfqpoint{4.200223in}{2.558066in}}%
\pgfpathlineto{\pgfqpoint{4.246234in}{2.561881in}}%
\pgfpathlineto{\pgfqpoint{4.292245in}{2.567735in}}%
\pgfpathlineto{\pgfqpoint{4.338256in}{2.574811in}}%
\pgfpathlineto{\pgfqpoint{4.384267in}{2.579739in}}%
\pgfpathlineto{\pgfqpoint{4.430278in}{2.586786in}}%
\pgfpathlineto{\pgfqpoint{4.476289in}{2.589529in}}%
\pgfpathlineto{\pgfqpoint{4.522301in}{2.600580in}}%
\pgfpathlineto{\pgfqpoint{4.568312in}{2.604989in}}%
\pgfpathlineto{\pgfqpoint{4.614323in}{2.615936in}}%
\pgfpathlineto{\pgfqpoint{4.660334in}{2.618011in}}%
\pgfpathlineto{\pgfqpoint{4.706345in}{2.622970in}}%
\pgfpathlineto{\pgfqpoint{4.752356in}{2.625660in}}%
\pgfpathlineto{\pgfqpoint{4.798367in}{2.630183in}}%
\pgfpathlineto{\pgfqpoint{4.844378in}{2.637234in}}%
\pgfpathlineto{\pgfqpoint{4.890390in}{2.644081in}}%
\pgfpathlineto{\pgfqpoint{4.936401in}{2.651340in}}%
\pgfpathlineto{\pgfqpoint{4.982412in}{2.652821in}}%
\pgfpathlineto{\pgfqpoint{5.028423in}{2.663802in}}%
\pgfpathlineto{\pgfqpoint{5.074434in}{2.658390in}}%
\pgfpathlineto{\pgfqpoint{5.120445in}{2.667193in}}%
\pgfpathlineto{\pgfqpoint{5.166456in}{2.674574in}}%
\pgfpathlineto{\pgfqpoint{5.212468in}{2.679536in}}%
\pgfpathlineto{\pgfqpoint{5.258479in}{2.684884in}}%
\pgfpathlineto{\pgfqpoint{5.304490in}{2.687246in}}%
\pgfpathlineto{\pgfqpoint{5.350501in}{2.694688in}}%
\pgfpathlineto{\pgfqpoint{5.396512in}{2.694817in}}%
\pgfpathlineto{\pgfqpoint{5.442523in}{2.705697in}}%
\pgfpathlineto{\pgfqpoint{5.488534in}{2.709884in}}%
\pgfpathlineto{\pgfqpoint{5.534545in}{2.709835in}}%
\pgfusepath{stroke}%
\end{pgfscope}%
\begin{pgfscope}%
\pgfsetrectcap%
\pgfsetmiterjoin%
\pgfsetlinewidth{0.803000pt}%
\definecolor{currentstroke}{rgb}{0.000000,0.000000,0.000000}%
\pgfsetstrokecolor{currentstroke}%
\pgfsetdash{}{0pt}%
\pgfpathmoveto{\pgfqpoint{0.800000in}{0.528000in}}%
\pgfpathlineto{\pgfqpoint{0.800000in}{4.224000in}}%
\pgfusepath{stroke}%
\end{pgfscope}%
\begin{pgfscope}%
\pgfsetrectcap%
\pgfsetmiterjoin%
\pgfsetlinewidth{0.803000pt}%
\definecolor{currentstroke}{rgb}{0.000000,0.000000,0.000000}%
\pgfsetstrokecolor{currentstroke}%
\pgfsetdash{}{0pt}%
\pgfpathmoveto{\pgfqpoint{5.760000in}{0.528000in}}%
\pgfpathlineto{\pgfqpoint{5.760000in}{4.224000in}}%
\pgfusepath{stroke}%
\end{pgfscope}%
\begin{pgfscope}%
\pgfsetrectcap%
\pgfsetmiterjoin%
\pgfsetlinewidth{0.803000pt}%
\definecolor{currentstroke}{rgb}{0.000000,0.000000,0.000000}%
\pgfsetstrokecolor{currentstroke}%
\pgfsetdash{}{0pt}%
\pgfpathmoveto{\pgfqpoint{0.800000in}{0.528000in}}%
\pgfpathlineto{\pgfqpoint{5.760000in}{0.528000in}}%
\pgfusepath{stroke}%
\end{pgfscope}%
\begin{pgfscope}%
\pgfsetrectcap%
\pgfsetmiterjoin%
\pgfsetlinewidth{0.803000pt}%
\definecolor{currentstroke}{rgb}{0.000000,0.000000,0.000000}%
\pgfsetstrokecolor{currentstroke}%
\pgfsetdash{}{0pt}%
\pgfpathmoveto{\pgfqpoint{0.800000in}{4.224000in}}%
\pgfpathlineto{\pgfqpoint{5.760000in}{4.224000in}}%
\pgfusepath{stroke}%
\end{pgfscope}%
\begin{pgfscope}%
\pgfsetbuttcap%
\pgfsetmiterjoin%
\definecolor{currentfill}{rgb}{1.000000,1.000000,1.000000}%
\pgfsetfillcolor{currentfill}%
\pgfsetfillopacity{0.800000}%
\pgfsetlinewidth{1.003750pt}%
\definecolor{currentstroke}{rgb}{0.800000,0.800000,0.800000}%
\pgfsetstrokecolor{currentstroke}%
\pgfsetstrokeopacity{0.800000}%
\pgfsetdash{}{0pt}%
\pgfpathmoveto{\pgfqpoint{0.897222in}{3.144525in}}%
\pgfpathlineto{\pgfqpoint{1.739044in}{3.144525in}}%
\pgfpathquadraticcurveto{\pgfqpoint{1.766822in}{3.144525in}}{\pgfqpoint{1.766822in}{3.172303in}}%
\pgfpathlineto{\pgfqpoint{1.766822in}{4.126778in}}%
\pgfpathquadraticcurveto{\pgfqpoint{1.766822in}{4.154556in}}{\pgfqpoint{1.739044in}{4.154556in}}%
\pgfpathlineto{\pgfqpoint{0.897222in}{4.154556in}}%
\pgfpathquadraticcurveto{\pgfqpoint{0.869444in}{4.154556in}}{\pgfqpoint{0.869444in}{4.126778in}}%
\pgfpathlineto{\pgfqpoint{0.869444in}{3.172303in}}%
\pgfpathquadraticcurveto{\pgfqpoint{0.869444in}{3.144525in}}{\pgfqpoint{0.897222in}{3.144525in}}%
\pgfpathlineto{\pgfqpoint{0.897222in}{3.144525in}}%
\pgfpathclose%
\pgfusepath{stroke,fill}%
\end{pgfscope}%
\begin{pgfscope}%
\pgfsetrectcap%
\pgfsetroundjoin%
\pgfsetlinewidth{1.505625pt}%
\definecolor{currentstroke}{rgb}{0.121569,0.466667,0.705882}%
\pgfsetstrokecolor{currentstroke}%
\pgfsetdash{}{0pt}%
\pgfpathmoveto{\pgfqpoint{0.925000in}{4.050389in}}%
\pgfpathlineto{\pgfqpoint{1.063889in}{4.050389in}}%
\pgfpathlineto{\pgfqpoint{1.202778in}{4.050389in}}%
\pgfusepath{stroke}%
\end{pgfscope}%
\begin{pgfscope}%
\definecolor{textcolor}{rgb}{0.000000,0.000000,0.000000}%
\pgfsetstrokecolor{textcolor}%
\pgfsetfillcolor{textcolor}%
\pgftext[x=1.313889in,y=4.001778in,left,base]{\color{textcolor}\rmfamily\fontsize{10.000000}{12.000000}\selectfont quick}%
\end{pgfscope}%
\begin{pgfscope}%
\pgfsetrectcap%
\pgfsetroundjoin%
\pgfsetlinewidth{1.505625pt}%
\definecolor{currentstroke}{rgb}{1.000000,0.498039,0.054902}%
\pgfsetstrokecolor{currentstroke}%
\pgfsetdash{}{0pt}%
\pgfpathmoveto{\pgfqpoint{0.925000in}{3.856716in}}%
\pgfpathlineto{\pgfqpoint{1.063889in}{3.856716in}}%
\pgfpathlineto{\pgfqpoint{1.202778in}{3.856716in}}%
\pgfusepath{stroke}%
\end{pgfscope}%
\begin{pgfscope}%
\definecolor{textcolor}{rgb}{0.000000,0.000000,0.000000}%
\pgfsetstrokecolor{textcolor}%
\pgfsetfillcolor{textcolor}%
\pgftext[x=1.313889in,y=3.808105in,left,base]{\color{textcolor}\rmfamily\fontsize{10.000000}{12.000000}\selectfont merge}%
\end{pgfscope}%
\begin{pgfscope}%
\pgfsetrectcap%
\pgfsetroundjoin%
\pgfsetlinewidth{1.505625pt}%
\definecolor{currentstroke}{rgb}{0.172549,0.627451,0.172549}%
\pgfsetstrokecolor{currentstroke}%
\pgfsetdash{}{0pt}%
\pgfpathmoveto{\pgfqpoint{0.925000in}{3.663043in}}%
\pgfpathlineto{\pgfqpoint{1.063889in}{3.663043in}}%
\pgfpathlineto{\pgfqpoint{1.202778in}{3.663043in}}%
\pgfusepath{stroke}%
\end{pgfscope}%
\begin{pgfscope}%
\definecolor{textcolor}{rgb}{0.000000,0.000000,0.000000}%
\pgfsetstrokecolor{textcolor}%
\pgfsetfillcolor{textcolor}%
\pgftext[x=1.313889in,y=3.614432in,left,base]{\color{textcolor}\rmfamily\fontsize{10.000000}{12.000000}\selectfont heap}%
\end{pgfscope}%
\begin{pgfscope}%
\pgfsetrectcap%
\pgfsetroundjoin%
\pgfsetlinewidth{1.505625pt}%
\definecolor{currentstroke}{rgb}{0.839216,0.152941,0.156863}%
\pgfsetstrokecolor{currentstroke}%
\pgfsetdash{}{0pt}%
\pgfpathmoveto{\pgfqpoint{0.925000in}{3.469371in}}%
\pgfpathlineto{\pgfqpoint{1.063889in}{3.469371in}}%
\pgfpathlineto{\pgfqpoint{1.202778in}{3.469371in}}%
\pgfusepath{stroke}%
\end{pgfscope}%
\begin{pgfscope}%
\definecolor{textcolor}{rgb}{0.000000,0.000000,0.000000}%
\pgfsetstrokecolor{textcolor}%
\pgfsetfillcolor{textcolor}%
\pgftext[x=1.313889in,y=3.420759in,left,base]{\color{textcolor}\rmfamily\fontsize{10.000000}{12.000000}\selectfont insert}%
\end{pgfscope}%
\begin{pgfscope}%
\pgfsetrectcap%
\pgfsetroundjoin%
\pgfsetlinewidth{1.505625pt}%
\definecolor{currentstroke}{rgb}{0.580392,0.403922,0.741176}%
\pgfsetstrokecolor{currentstroke}%
\pgfsetdash{}{0pt}%
\pgfpathmoveto{\pgfqpoint{0.925000in}{3.275698in}}%
\pgfpathlineto{\pgfqpoint{1.063889in}{3.275698in}}%
\pgfpathlineto{\pgfqpoint{1.202778in}{3.275698in}}%
\pgfusepath{stroke}%
\end{pgfscope}%
\begin{pgfscope}%
\definecolor{textcolor}{rgb}{0.000000,0.000000,0.000000}%
\pgfsetstrokecolor{textcolor}%
\pgfsetfillcolor{textcolor}%
\pgftext[x=1.313889in,y=3.227087in,left,base]{\color{textcolor}\rmfamily\fontsize{10.000000}{12.000000}\selectfont bucket}%
\end{pgfscope}%
\end{pgfpicture}%
\makeatother%
\endgroup%

%% Creator: Matplotlib, PGF backend
%%
%% To include the figure in your LaTeX document, write
%%   \input{<filename>.pgf}
%%
%% Make sure the required packages are loaded in your preamble
%%   \usepackage{pgf}
%%
%% Also ensure that all the required font packages are loaded; for instance,
%% the lmodern package is sometimes necessary when using math font.
%%   \usepackage{lmodern}
%%
%% Figures using additional raster images can only be included by \input if
%% they are in the same directory as the main LaTeX file. For loading figures
%% from other directories you can use the `import` package
%%   \usepackage{import}
%%
%% and then include the figures with
%%   \import{<path to file>}{<filename>.pgf}
%%
%% Matplotlib used the following preamble
%%   
%%   \makeatletter\@ifpackageloaded{underscore}{}{\usepackage[strings]{underscore}}\makeatother
%%
\begingroup%
\makeatletter%
\begin{pgfpicture}%
\pgfpathrectangle{\pgfpointorigin}{\pgfqpoint{6.400000in}{4.800000in}}%
\pgfusepath{use as bounding box, clip}%
\begin{pgfscope}%
\pgfsetbuttcap%
\pgfsetmiterjoin%
\definecolor{currentfill}{rgb}{1.000000,1.000000,1.000000}%
\pgfsetfillcolor{currentfill}%
\pgfsetlinewidth{0.000000pt}%
\definecolor{currentstroke}{rgb}{1.000000,1.000000,1.000000}%
\pgfsetstrokecolor{currentstroke}%
\pgfsetdash{}{0pt}%
\pgfpathmoveto{\pgfqpoint{0.000000in}{0.000000in}}%
\pgfpathlineto{\pgfqpoint{6.400000in}{0.000000in}}%
\pgfpathlineto{\pgfqpoint{6.400000in}{4.800000in}}%
\pgfpathlineto{\pgfqpoint{0.000000in}{4.800000in}}%
\pgfpathlineto{\pgfqpoint{0.000000in}{0.000000in}}%
\pgfpathclose%
\pgfusepath{fill}%
\end{pgfscope}%
\begin{pgfscope}%
\pgfsetbuttcap%
\pgfsetmiterjoin%
\definecolor{currentfill}{rgb}{1.000000,1.000000,1.000000}%
\pgfsetfillcolor{currentfill}%
\pgfsetlinewidth{0.000000pt}%
\definecolor{currentstroke}{rgb}{0.000000,0.000000,0.000000}%
\pgfsetstrokecolor{currentstroke}%
\pgfsetstrokeopacity{0.000000}%
\pgfsetdash{}{0pt}%
\pgfpathmoveto{\pgfqpoint{0.800000in}{0.528000in}}%
\pgfpathlineto{\pgfqpoint{5.760000in}{0.528000in}}%
\pgfpathlineto{\pgfqpoint{5.760000in}{4.224000in}}%
\pgfpathlineto{\pgfqpoint{0.800000in}{4.224000in}}%
\pgfpathlineto{\pgfqpoint{0.800000in}{0.528000in}}%
\pgfpathclose%
\pgfusepath{fill}%
\end{pgfscope}%
\begin{pgfscope}%
\pgfsetbuttcap%
\pgfsetroundjoin%
\definecolor{currentfill}{rgb}{0.000000,0.000000,0.000000}%
\pgfsetfillcolor{currentfill}%
\pgfsetlinewidth{0.803000pt}%
\definecolor{currentstroke}{rgb}{0.000000,0.000000,0.000000}%
\pgfsetstrokecolor{currentstroke}%
\pgfsetdash{}{0pt}%
\pgfsys@defobject{currentmarker}{\pgfqpoint{0.000000in}{-0.048611in}}{\pgfqpoint{0.000000in}{0.000000in}}{%
\pgfpathmoveto{\pgfqpoint{0.000000in}{0.000000in}}%
\pgfpathlineto{\pgfqpoint{0.000000in}{-0.048611in}}%
\pgfusepath{stroke,fill}%
}%
\begin{pgfscope}%
\pgfsys@transformshift{0.979443in}{0.528000in}%
\pgfsys@useobject{currentmarker}{}%
\end{pgfscope}%
\end{pgfscope}%
\begin{pgfscope}%
\definecolor{textcolor}{rgb}{0.000000,0.000000,0.000000}%
\pgfsetstrokecolor{textcolor}%
\pgfsetfillcolor{textcolor}%
\pgftext[x=0.979443in,y=0.430778in,,top]{\color{textcolor}\rmfamily\fontsize{10.000000}{12.000000}\selectfont \(\displaystyle {0}\)}%
\end{pgfscope}%
\begin{pgfscope}%
\pgfsetbuttcap%
\pgfsetroundjoin%
\definecolor{currentfill}{rgb}{0.000000,0.000000,0.000000}%
\pgfsetfillcolor{currentfill}%
\pgfsetlinewidth{0.803000pt}%
\definecolor{currentstroke}{rgb}{0.000000,0.000000,0.000000}%
\pgfsetstrokecolor{currentstroke}%
\pgfsetdash{}{0pt}%
\pgfsys@defobject{currentmarker}{\pgfqpoint{0.000000in}{-0.048611in}}{\pgfqpoint{0.000000in}{0.000000in}}{%
\pgfpathmoveto{\pgfqpoint{0.000000in}{0.000000in}}%
\pgfpathlineto{\pgfqpoint{0.000000in}{-0.048611in}}%
\pgfusepath{stroke,fill}%
}%
\begin{pgfscope}%
\pgfsys@transformshift{1.899666in}{0.528000in}%
\pgfsys@useobject{currentmarker}{}%
\end{pgfscope}%
\end{pgfscope}%
\begin{pgfscope}%
\definecolor{textcolor}{rgb}{0.000000,0.000000,0.000000}%
\pgfsetstrokecolor{textcolor}%
\pgfsetfillcolor{textcolor}%
\pgftext[x=1.899666in,y=0.430778in,,top]{\color{textcolor}\rmfamily\fontsize{10.000000}{12.000000}\selectfont \(\displaystyle {2000}\)}%
\end{pgfscope}%
\begin{pgfscope}%
\pgfsetbuttcap%
\pgfsetroundjoin%
\definecolor{currentfill}{rgb}{0.000000,0.000000,0.000000}%
\pgfsetfillcolor{currentfill}%
\pgfsetlinewidth{0.803000pt}%
\definecolor{currentstroke}{rgb}{0.000000,0.000000,0.000000}%
\pgfsetstrokecolor{currentstroke}%
\pgfsetdash{}{0pt}%
\pgfsys@defobject{currentmarker}{\pgfqpoint{0.000000in}{-0.048611in}}{\pgfqpoint{0.000000in}{0.000000in}}{%
\pgfpathmoveto{\pgfqpoint{0.000000in}{0.000000in}}%
\pgfpathlineto{\pgfqpoint{0.000000in}{-0.048611in}}%
\pgfusepath{stroke,fill}%
}%
\begin{pgfscope}%
\pgfsys@transformshift{2.819889in}{0.528000in}%
\pgfsys@useobject{currentmarker}{}%
\end{pgfscope}%
\end{pgfscope}%
\begin{pgfscope}%
\definecolor{textcolor}{rgb}{0.000000,0.000000,0.000000}%
\pgfsetstrokecolor{textcolor}%
\pgfsetfillcolor{textcolor}%
\pgftext[x=2.819889in,y=0.430778in,,top]{\color{textcolor}\rmfamily\fontsize{10.000000}{12.000000}\selectfont \(\displaystyle {4000}\)}%
\end{pgfscope}%
\begin{pgfscope}%
\pgfsetbuttcap%
\pgfsetroundjoin%
\definecolor{currentfill}{rgb}{0.000000,0.000000,0.000000}%
\pgfsetfillcolor{currentfill}%
\pgfsetlinewidth{0.803000pt}%
\definecolor{currentstroke}{rgb}{0.000000,0.000000,0.000000}%
\pgfsetstrokecolor{currentstroke}%
\pgfsetdash{}{0pt}%
\pgfsys@defobject{currentmarker}{\pgfqpoint{0.000000in}{-0.048611in}}{\pgfqpoint{0.000000in}{0.000000in}}{%
\pgfpathmoveto{\pgfqpoint{0.000000in}{0.000000in}}%
\pgfpathlineto{\pgfqpoint{0.000000in}{-0.048611in}}%
\pgfusepath{stroke,fill}%
}%
\begin{pgfscope}%
\pgfsys@transformshift{3.740111in}{0.528000in}%
\pgfsys@useobject{currentmarker}{}%
\end{pgfscope}%
\end{pgfscope}%
\begin{pgfscope}%
\definecolor{textcolor}{rgb}{0.000000,0.000000,0.000000}%
\pgfsetstrokecolor{textcolor}%
\pgfsetfillcolor{textcolor}%
\pgftext[x=3.740111in,y=0.430778in,,top]{\color{textcolor}\rmfamily\fontsize{10.000000}{12.000000}\selectfont \(\displaystyle {6000}\)}%
\end{pgfscope}%
\begin{pgfscope}%
\pgfsetbuttcap%
\pgfsetroundjoin%
\definecolor{currentfill}{rgb}{0.000000,0.000000,0.000000}%
\pgfsetfillcolor{currentfill}%
\pgfsetlinewidth{0.803000pt}%
\definecolor{currentstroke}{rgb}{0.000000,0.000000,0.000000}%
\pgfsetstrokecolor{currentstroke}%
\pgfsetdash{}{0pt}%
\pgfsys@defobject{currentmarker}{\pgfqpoint{0.000000in}{-0.048611in}}{\pgfqpoint{0.000000in}{0.000000in}}{%
\pgfpathmoveto{\pgfqpoint{0.000000in}{0.000000in}}%
\pgfpathlineto{\pgfqpoint{0.000000in}{-0.048611in}}%
\pgfusepath{stroke,fill}%
}%
\begin{pgfscope}%
\pgfsys@transformshift{4.660334in}{0.528000in}%
\pgfsys@useobject{currentmarker}{}%
\end{pgfscope}%
\end{pgfscope}%
\begin{pgfscope}%
\definecolor{textcolor}{rgb}{0.000000,0.000000,0.000000}%
\pgfsetstrokecolor{textcolor}%
\pgfsetfillcolor{textcolor}%
\pgftext[x=4.660334in,y=0.430778in,,top]{\color{textcolor}\rmfamily\fontsize{10.000000}{12.000000}\selectfont \(\displaystyle {8000}\)}%
\end{pgfscope}%
\begin{pgfscope}%
\pgfsetbuttcap%
\pgfsetroundjoin%
\definecolor{currentfill}{rgb}{0.000000,0.000000,0.000000}%
\pgfsetfillcolor{currentfill}%
\pgfsetlinewidth{0.803000pt}%
\definecolor{currentstroke}{rgb}{0.000000,0.000000,0.000000}%
\pgfsetstrokecolor{currentstroke}%
\pgfsetdash{}{0pt}%
\pgfsys@defobject{currentmarker}{\pgfqpoint{0.000000in}{-0.048611in}}{\pgfqpoint{0.000000in}{0.000000in}}{%
\pgfpathmoveto{\pgfqpoint{0.000000in}{0.000000in}}%
\pgfpathlineto{\pgfqpoint{0.000000in}{-0.048611in}}%
\pgfusepath{stroke,fill}%
}%
\begin{pgfscope}%
\pgfsys@transformshift{5.580557in}{0.528000in}%
\pgfsys@useobject{currentmarker}{}%
\end{pgfscope}%
\end{pgfscope}%
\begin{pgfscope}%
\definecolor{textcolor}{rgb}{0.000000,0.000000,0.000000}%
\pgfsetstrokecolor{textcolor}%
\pgfsetfillcolor{textcolor}%
\pgftext[x=5.580557in,y=0.430778in,,top]{\color{textcolor}\rmfamily\fontsize{10.000000}{12.000000}\selectfont \(\displaystyle {10000}\)}%
\end{pgfscope}%
\begin{pgfscope}%
\definecolor{textcolor}{rgb}{0.000000,0.000000,0.000000}%
\pgfsetstrokecolor{textcolor}%
\pgfsetfillcolor{textcolor}%
\pgftext[x=3.280000in,y=0.251766in,,top]{\color{textcolor}\rmfamily\fontsize{10.000000}{12.000000}\selectfont Input Size}%
\end{pgfscope}%
\begin{pgfscope}%
\pgfsetbuttcap%
\pgfsetroundjoin%
\definecolor{currentfill}{rgb}{0.000000,0.000000,0.000000}%
\pgfsetfillcolor{currentfill}%
\pgfsetlinewidth{0.803000pt}%
\definecolor{currentstroke}{rgb}{0.000000,0.000000,0.000000}%
\pgfsetstrokecolor{currentstroke}%
\pgfsetdash{}{0pt}%
\pgfsys@defobject{currentmarker}{\pgfqpoint{-0.048611in}{0.000000in}}{\pgfqpoint{-0.000000in}{0.000000in}}{%
\pgfpathmoveto{\pgfqpoint{-0.000000in}{0.000000in}}%
\pgfpathlineto{\pgfqpoint{-0.048611in}{0.000000in}}%
\pgfusepath{stroke,fill}%
}%
\begin{pgfscope}%
\pgfsys@transformshift{0.800000in}{0.959459in}%
\pgfsys@useobject{currentmarker}{}%
\end{pgfscope}%
\end{pgfscope}%
\begin{pgfscope}%
\definecolor{textcolor}{rgb}{0.000000,0.000000,0.000000}%
\pgfsetstrokecolor{textcolor}%
\pgfsetfillcolor{textcolor}%
\pgftext[x=0.501581in, y=0.911234in, left, base]{\color{textcolor}\rmfamily\fontsize{10.000000}{12.000000}\selectfont \(\displaystyle {10^{6}}\)}%
\end{pgfscope}%
\begin{pgfscope}%
\pgfsetbuttcap%
\pgfsetroundjoin%
\definecolor{currentfill}{rgb}{0.000000,0.000000,0.000000}%
\pgfsetfillcolor{currentfill}%
\pgfsetlinewidth{0.803000pt}%
\definecolor{currentstroke}{rgb}{0.000000,0.000000,0.000000}%
\pgfsetstrokecolor{currentstroke}%
\pgfsetdash{}{0pt}%
\pgfsys@defobject{currentmarker}{\pgfqpoint{-0.048611in}{0.000000in}}{\pgfqpoint{-0.000000in}{0.000000in}}{%
\pgfpathmoveto{\pgfqpoint{-0.000000in}{0.000000in}}%
\pgfpathlineto{\pgfqpoint{-0.048611in}{0.000000in}}%
\pgfusepath{stroke,fill}%
}%
\begin{pgfscope}%
\pgfsys@transformshift{0.800000in}{1.639311in}%
\pgfsys@useobject{currentmarker}{}%
\end{pgfscope}%
\end{pgfscope}%
\begin{pgfscope}%
\definecolor{textcolor}{rgb}{0.000000,0.000000,0.000000}%
\pgfsetstrokecolor{textcolor}%
\pgfsetfillcolor{textcolor}%
\pgftext[x=0.501581in, y=1.591086in, left, base]{\color{textcolor}\rmfamily\fontsize{10.000000}{12.000000}\selectfont \(\displaystyle {10^{7}}\)}%
\end{pgfscope}%
\begin{pgfscope}%
\pgfsetbuttcap%
\pgfsetroundjoin%
\definecolor{currentfill}{rgb}{0.000000,0.000000,0.000000}%
\pgfsetfillcolor{currentfill}%
\pgfsetlinewidth{0.803000pt}%
\definecolor{currentstroke}{rgb}{0.000000,0.000000,0.000000}%
\pgfsetstrokecolor{currentstroke}%
\pgfsetdash{}{0pt}%
\pgfsys@defobject{currentmarker}{\pgfqpoint{-0.048611in}{0.000000in}}{\pgfqpoint{-0.000000in}{0.000000in}}{%
\pgfpathmoveto{\pgfqpoint{-0.000000in}{0.000000in}}%
\pgfpathlineto{\pgfqpoint{-0.048611in}{0.000000in}}%
\pgfusepath{stroke,fill}%
}%
\begin{pgfscope}%
\pgfsys@transformshift{0.800000in}{2.319163in}%
\pgfsys@useobject{currentmarker}{}%
\end{pgfscope}%
\end{pgfscope}%
\begin{pgfscope}%
\definecolor{textcolor}{rgb}{0.000000,0.000000,0.000000}%
\pgfsetstrokecolor{textcolor}%
\pgfsetfillcolor{textcolor}%
\pgftext[x=0.501581in, y=2.270938in, left, base]{\color{textcolor}\rmfamily\fontsize{10.000000}{12.000000}\selectfont \(\displaystyle {10^{8}}\)}%
\end{pgfscope}%
\begin{pgfscope}%
\pgfsetbuttcap%
\pgfsetroundjoin%
\definecolor{currentfill}{rgb}{0.000000,0.000000,0.000000}%
\pgfsetfillcolor{currentfill}%
\pgfsetlinewidth{0.803000pt}%
\definecolor{currentstroke}{rgb}{0.000000,0.000000,0.000000}%
\pgfsetstrokecolor{currentstroke}%
\pgfsetdash{}{0pt}%
\pgfsys@defobject{currentmarker}{\pgfqpoint{-0.048611in}{0.000000in}}{\pgfqpoint{-0.000000in}{0.000000in}}{%
\pgfpathmoveto{\pgfqpoint{-0.000000in}{0.000000in}}%
\pgfpathlineto{\pgfqpoint{-0.048611in}{0.000000in}}%
\pgfusepath{stroke,fill}%
}%
\begin{pgfscope}%
\pgfsys@transformshift{0.800000in}{2.999015in}%
\pgfsys@useobject{currentmarker}{}%
\end{pgfscope}%
\end{pgfscope}%
\begin{pgfscope}%
\definecolor{textcolor}{rgb}{0.000000,0.000000,0.000000}%
\pgfsetstrokecolor{textcolor}%
\pgfsetfillcolor{textcolor}%
\pgftext[x=0.501581in, y=2.950790in, left, base]{\color{textcolor}\rmfamily\fontsize{10.000000}{12.000000}\selectfont \(\displaystyle {10^{9}}\)}%
\end{pgfscope}%
\begin{pgfscope}%
\pgfsetbuttcap%
\pgfsetroundjoin%
\definecolor{currentfill}{rgb}{0.000000,0.000000,0.000000}%
\pgfsetfillcolor{currentfill}%
\pgfsetlinewidth{0.803000pt}%
\definecolor{currentstroke}{rgb}{0.000000,0.000000,0.000000}%
\pgfsetstrokecolor{currentstroke}%
\pgfsetdash{}{0pt}%
\pgfsys@defobject{currentmarker}{\pgfqpoint{-0.048611in}{0.000000in}}{\pgfqpoint{-0.000000in}{0.000000in}}{%
\pgfpathmoveto{\pgfqpoint{-0.000000in}{0.000000in}}%
\pgfpathlineto{\pgfqpoint{-0.048611in}{0.000000in}}%
\pgfusepath{stroke,fill}%
}%
\begin{pgfscope}%
\pgfsys@transformshift{0.800000in}{3.678868in}%
\pgfsys@useobject{currentmarker}{}%
\end{pgfscope}%
\end{pgfscope}%
\begin{pgfscope}%
\definecolor{textcolor}{rgb}{0.000000,0.000000,0.000000}%
\pgfsetstrokecolor{textcolor}%
\pgfsetfillcolor{textcolor}%
\pgftext[x=0.446218in, y=3.630642in, left, base]{\color{textcolor}\rmfamily\fontsize{10.000000}{12.000000}\selectfont \(\displaystyle {10^{10}}\)}%
\end{pgfscope}%
\begin{pgfscope}%
\pgfsetbuttcap%
\pgfsetroundjoin%
\definecolor{currentfill}{rgb}{0.000000,0.000000,0.000000}%
\pgfsetfillcolor{currentfill}%
\pgfsetlinewidth{0.602250pt}%
\definecolor{currentstroke}{rgb}{0.000000,0.000000,0.000000}%
\pgfsetstrokecolor{currentstroke}%
\pgfsetdash{}{0pt}%
\pgfsys@defobject{currentmarker}{\pgfqpoint{-0.027778in}{0.000000in}}{\pgfqpoint{-0.000000in}{0.000000in}}{%
\pgfpathmoveto{\pgfqpoint{-0.000000in}{0.000000in}}%
\pgfpathlineto{\pgfqpoint{-0.027778in}{0.000000in}}%
\pgfusepath{stroke,fill}%
}%
\begin{pgfscope}%
\pgfsys@transformshift{0.800000in}{0.603979in}%
\pgfsys@useobject{currentmarker}{}%
\end{pgfscope}%
\end{pgfscope}%
\begin{pgfscope}%
\pgfsetbuttcap%
\pgfsetroundjoin%
\definecolor{currentfill}{rgb}{0.000000,0.000000,0.000000}%
\pgfsetfillcolor{currentfill}%
\pgfsetlinewidth{0.602250pt}%
\definecolor{currentstroke}{rgb}{0.000000,0.000000,0.000000}%
\pgfsetstrokecolor{currentstroke}%
\pgfsetdash{}{0pt}%
\pgfsys@defobject{currentmarker}{\pgfqpoint{-0.027778in}{0.000000in}}{\pgfqpoint{-0.000000in}{0.000000in}}{%
\pgfpathmoveto{\pgfqpoint{-0.000000in}{0.000000in}}%
\pgfpathlineto{\pgfqpoint{-0.027778in}{0.000000in}}%
\pgfusepath{stroke,fill}%
}%
\begin{pgfscope}%
\pgfsys@transformshift{0.800000in}{0.688919in}%
\pgfsys@useobject{currentmarker}{}%
\end{pgfscope}%
\end{pgfscope}%
\begin{pgfscope}%
\pgfsetbuttcap%
\pgfsetroundjoin%
\definecolor{currentfill}{rgb}{0.000000,0.000000,0.000000}%
\pgfsetfillcolor{currentfill}%
\pgfsetlinewidth{0.602250pt}%
\definecolor{currentstroke}{rgb}{0.000000,0.000000,0.000000}%
\pgfsetstrokecolor{currentstroke}%
\pgfsetdash{}{0pt}%
\pgfsys@defobject{currentmarker}{\pgfqpoint{-0.027778in}{0.000000in}}{\pgfqpoint{-0.000000in}{0.000000in}}{%
\pgfpathmoveto{\pgfqpoint{-0.000000in}{0.000000in}}%
\pgfpathlineto{\pgfqpoint{-0.027778in}{0.000000in}}%
\pgfusepath{stroke,fill}%
}%
\begin{pgfscope}%
\pgfsys@transformshift{0.800000in}{0.754803in}%
\pgfsys@useobject{currentmarker}{}%
\end{pgfscope}%
\end{pgfscope}%
\begin{pgfscope}%
\pgfsetbuttcap%
\pgfsetroundjoin%
\definecolor{currentfill}{rgb}{0.000000,0.000000,0.000000}%
\pgfsetfillcolor{currentfill}%
\pgfsetlinewidth{0.602250pt}%
\definecolor{currentstroke}{rgb}{0.000000,0.000000,0.000000}%
\pgfsetstrokecolor{currentstroke}%
\pgfsetdash{}{0pt}%
\pgfsys@defobject{currentmarker}{\pgfqpoint{-0.027778in}{0.000000in}}{\pgfqpoint{-0.000000in}{0.000000in}}{%
\pgfpathmoveto{\pgfqpoint{-0.000000in}{0.000000in}}%
\pgfpathlineto{\pgfqpoint{-0.027778in}{0.000000in}}%
\pgfusepath{stroke,fill}%
}%
\begin{pgfscope}%
\pgfsys@transformshift{0.800000in}{0.808635in}%
\pgfsys@useobject{currentmarker}{}%
\end{pgfscope}%
\end{pgfscope}%
\begin{pgfscope}%
\pgfsetbuttcap%
\pgfsetroundjoin%
\definecolor{currentfill}{rgb}{0.000000,0.000000,0.000000}%
\pgfsetfillcolor{currentfill}%
\pgfsetlinewidth{0.602250pt}%
\definecolor{currentstroke}{rgb}{0.000000,0.000000,0.000000}%
\pgfsetstrokecolor{currentstroke}%
\pgfsetdash{}{0pt}%
\pgfsys@defobject{currentmarker}{\pgfqpoint{-0.027778in}{0.000000in}}{\pgfqpoint{-0.000000in}{0.000000in}}{%
\pgfpathmoveto{\pgfqpoint{-0.000000in}{0.000000in}}%
\pgfpathlineto{\pgfqpoint{-0.027778in}{0.000000in}}%
\pgfusepath{stroke,fill}%
}%
\begin{pgfscope}%
\pgfsys@transformshift{0.800000in}{0.854149in}%
\pgfsys@useobject{currentmarker}{}%
\end{pgfscope}%
\end{pgfscope}%
\begin{pgfscope}%
\pgfsetbuttcap%
\pgfsetroundjoin%
\definecolor{currentfill}{rgb}{0.000000,0.000000,0.000000}%
\pgfsetfillcolor{currentfill}%
\pgfsetlinewidth{0.602250pt}%
\definecolor{currentstroke}{rgb}{0.000000,0.000000,0.000000}%
\pgfsetstrokecolor{currentstroke}%
\pgfsetdash{}{0pt}%
\pgfsys@defobject{currentmarker}{\pgfqpoint{-0.027778in}{0.000000in}}{\pgfqpoint{-0.000000in}{0.000000in}}{%
\pgfpathmoveto{\pgfqpoint{-0.000000in}{0.000000in}}%
\pgfpathlineto{\pgfqpoint{-0.027778in}{0.000000in}}%
\pgfusepath{stroke,fill}%
}%
\begin{pgfscope}%
\pgfsys@transformshift{0.800000in}{0.893575in}%
\pgfsys@useobject{currentmarker}{}%
\end{pgfscope}%
\end{pgfscope}%
\begin{pgfscope}%
\pgfsetbuttcap%
\pgfsetroundjoin%
\definecolor{currentfill}{rgb}{0.000000,0.000000,0.000000}%
\pgfsetfillcolor{currentfill}%
\pgfsetlinewidth{0.602250pt}%
\definecolor{currentstroke}{rgb}{0.000000,0.000000,0.000000}%
\pgfsetstrokecolor{currentstroke}%
\pgfsetdash{}{0pt}%
\pgfsys@defobject{currentmarker}{\pgfqpoint{-0.027778in}{0.000000in}}{\pgfqpoint{-0.000000in}{0.000000in}}{%
\pgfpathmoveto{\pgfqpoint{-0.000000in}{0.000000in}}%
\pgfpathlineto{\pgfqpoint{-0.027778in}{0.000000in}}%
\pgfusepath{stroke,fill}%
}%
\begin{pgfscope}%
\pgfsys@transformshift{0.800000in}{0.928351in}%
\pgfsys@useobject{currentmarker}{}%
\end{pgfscope}%
\end{pgfscope}%
\begin{pgfscope}%
\pgfsetbuttcap%
\pgfsetroundjoin%
\definecolor{currentfill}{rgb}{0.000000,0.000000,0.000000}%
\pgfsetfillcolor{currentfill}%
\pgfsetlinewidth{0.602250pt}%
\definecolor{currentstroke}{rgb}{0.000000,0.000000,0.000000}%
\pgfsetstrokecolor{currentstroke}%
\pgfsetdash{}{0pt}%
\pgfsys@defobject{currentmarker}{\pgfqpoint{-0.027778in}{0.000000in}}{\pgfqpoint{-0.000000in}{0.000000in}}{%
\pgfpathmoveto{\pgfqpoint{-0.000000in}{0.000000in}}%
\pgfpathlineto{\pgfqpoint{-0.027778in}{0.000000in}}%
\pgfusepath{stroke,fill}%
}%
\begin{pgfscope}%
\pgfsys@transformshift{0.800000in}{1.164115in}%
\pgfsys@useobject{currentmarker}{}%
\end{pgfscope}%
\end{pgfscope}%
\begin{pgfscope}%
\pgfsetbuttcap%
\pgfsetroundjoin%
\definecolor{currentfill}{rgb}{0.000000,0.000000,0.000000}%
\pgfsetfillcolor{currentfill}%
\pgfsetlinewidth{0.602250pt}%
\definecolor{currentstroke}{rgb}{0.000000,0.000000,0.000000}%
\pgfsetstrokecolor{currentstroke}%
\pgfsetdash{}{0pt}%
\pgfsys@defobject{currentmarker}{\pgfqpoint{-0.027778in}{0.000000in}}{\pgfqpoint{-0.000000in}{0.000000in}}{%
\pgfpathmoveto{\pgfqpoint{-0.000000in}{0.000000in}}%
\pgfpathlineto{\pgfqpoint{-0.027778in}{0.000000in}}%
\pgfusepath{stroke,fill}%
}%
\begin{pgfscope}%
\pgfsys@transformshift{0.800000in}{1.283831in}%
\pgfsys@useobject{currentmarker}{}%
\end{pgfscope}%
\end{pgfscope}%
\begin{pgfscope}%
\pgfsetbuttcap%
\pgfsetroundjoin%
\definecolor{currentfill}{rgb}{0.000000,0.000000,0.000000}%
\pgfsetfillcolor{currentfill}%
\pgfsetlinewidth{0.602250pt}%
\definecolor{currentstroke}{rgb}{0.000000,0.000000,0.000000}%
\pgfsetstrokecolor{currentstroke}%
\pgfsetdash{}{0pt}%
\pgfsys@defobject{currentmarker}{\pgfqpoint{-0.027778in}{0.000000in}}{\pgfqpoint{-0.000000in}{0.000000in}}{%
\pgfpathmoveto{\pgfqpoint{-0.000000in}{0.000000in}}%
\pgfpathlineto{\pgfqpoint{-0.027778in}{0.000000in}}%
\pgfusepath{stroke,fill}%
}%
\begin{pgfscope}%
\pgfsys@transformshift{0.800000in}{1.368771in}%
\pgfsys@useobject{currentmarker}{}%
\end{pgfscope}%
\end{pgfscope}%
\begin{pgfscope}%
\pgfsetbuttcap%
\pgfsetroundjoin%
\definecolor{currentfill}{rgb}{0.000000,0.000000,0.000000}%
\pgfsetfillcolor{currentfill}%
\pgfsetlinewidth{0.602250pt}%
\definecolor{currentstroke}{rgb}{0.000000,0.000000,0.000000}%
\pgfsetstrokecolor{currentstroke}%
\pgfsetdash{}{0pt}%
\pgfsys@defobject{currentmarker}{\pgfqpoint{-0.027778in}{0.000000in}}{\pgfqpoint{-0.000000in}{0.000000in}}{%
\pgfpathmoveto{\pgfqpoint{-0.000000in}{0.000000in}}%
\pgfpathlineto{\pgfqpoint{-0.027778in}{0.000000in}}%
\pgfusepath{stroke,fill}%
}%
\begin{pgfscope}%
\pgfsys@transformshift{0.800000in}{1.434656in}%
\pgfsys@useobject{currentmarker}{}%
\end{pgfscope}%
\end{pgfscope}%
\begin{pgfscope}%
\pgfsetbuttcap%
\pgfsetroundjoin%
\definecolor{currentfill}{rgb}{0.000000,0.000000,0.000000}%
\pgfsetfillcolor{currentfill}%
\pgfsetlinewidth{0.602250pt}%
\definecolor{currentstroke}{rgb}{0.000000,0.000000,0.000000}%
\pgfsetstrokecolor{currentstroke}%
\pgfsetdash{}{0pt}%
\pgfsys@defobject{currentmarker}{\pgfqpoint{-0.027778in}{0.000000in}}{\pgfqpoint{-0.000000in}{0.000000in}}{%
\pgfpathmoveto{\pgfqpoint{-0.000000in}{0.000000in}}%
\pgfpathlineto{\pgfqpoint{-0.027778in}{0.000000in}}%
\pgfusepath{stroke,fill}%
}%
\begin{pgfscope}%
\pgfsys@transformshift{0.800000in}{1.488487in}%
\pgfsys@useobject{currentmarker}{}%
\end{pgfscope}%
\end{pgfscope}%
\begin{pgfscope}%
\pgfsetbuttcap%
\pgfsetroundjoin%
\definecolor{currentfill}{rgb}{0.000000,0.000000,0.000000}%
\pgfsetfillcolor{currentfill}%
\pgfsetlinewidth{0.602250pt}%
\definecolor{currentstroke}{rgb}{0.000000,0.000000,0.000000}%
\pgfsetstrokecolor{currentstroke}%
\pgfsetdash{}{0pt}%
\pgfsys@defobject{currentmarker}{\pgfqpoint{-0.027778in}{0.000000in}}{\pgfqpoint{-0.000000in}{0.000000in}}{%
\pgfpathmoveto{\pgfqpoint{-0.000000in}{0.000000in}}%
\pgfpathlineto{\pgfqpoint{-0.027778in}{0.000000in}}%
\pgfusepath{stroke,fill}%
}%
\begin{pgfscope}%
\pgfsys@transformshift{0.800000in}{1.534001in}%
\pgfsys@useobject{currentmarker}{}%
\end{pgfscope}%
\end{pgfscope}%
\begin{pgfscope}%
\pgfsetbuttcap%
\pgfsetroundjoin%
\definecolor{currentfill}{rgb}{0.000000,0.000000,0.000000}%
\pgfsetfillcolor{currentfill}%
\pgfsetlinewidth{0.602250pt}%
\definecolor{currentstroke}{rgb}{0.000000,0.000000,0.000000}%
\pgfsetstrokecolor{currentstroke}%
\pgfsetdash{}{0pt}%
\pgfsys@defobject{currentmarker}{\pgfqpoint{-0.027778in}{0.000000in}}{\pgfqpoint{-0.000000in}{0.000000in}}{%
\pgfpathmoveto{\pgfqpoint{-0.000000in}{0.000000in}}%
\pgfpathlineto{\pgfqpoint{-0.027778in}{0.000000in}}%
\pgfusepath{stroke,fill}%
}%
\begin{pgfscope}%
\pgfsys@transformshift{0.800000in}{1.573427in}%
\pgfsys@useobject{currentmarker}{}%
\end{pgfscope}%
\end{pgfscope}%
\begin{pgfscope}%
\pgfsetbuttcap%
\pgfsetroundjoin%
\definecolor{currentfill}{rgb}{0.000000,0.000000,0.000000}%
\pgfsetfillcolor{currentfill}%
\pgfsetlinewidth{0.602250pt}%
\definecolor{currentstroke}{rgb}{0.000000,0.000000,0.000000}%
\pgfsetstrokecolor{currentstroke}%
\pgfsetdash{}{0pt}%
\pgfsys@defobject{currentmarker}{\pgfqpoint{-0.027778in}{0.000000in}}{\pgfqpoint{-0.000000in}{0.000000in}}{%
\pgfpathmoveto{\pgfqpoint{-0.000000in}{0.000000in}}%
\pgfpathlineto{\pgfqpoint{-0.027778in}{0.000000in}}%
\pgfusepath{stroke,fill}%
}%
\begin{pgfscope}%
\pgfsys@transformshift{0.800000in}{1.608203in}%
\pgfsys@useobject{currentmarker}{}%
\end{pgfscope}%
\end{pgfscope}%
\begin{pgfscope}%
\pgfsetbuttcap%
\pgfsetroundjoin%
\definecolor{currentfill}{rgb}{0.000000,0.000000,0.000000}%
\pgfsetfillcolor{currentfill}%
\pgfsetlinewidth{0.602250pt}%
\definecolor{currentstroke}{rgb}{0.000000,0.000000,0.000000}%
\pgfsetstrokecolor{currentstroke}%
\pgfsetdash{}{0pt}%
\pgfsys@defobject{currentmarker}{\pgfqpoint{-0.027778in}{0.000000in}}{\pgfqpoint{-0.000000in}{0.000000in}}{%
\pgfpathmoveto{\pgfqpoint{-0.000000in}{0.000000in}}%
\pgfpathlineto{\pgfqpoint{-0.027778in}{0.000000in}}%
\pgfusepath{stroke,fill}%
}%
\begin{pgfscope}%
\pgfsys@transformshift{0.800000in}{1.843967in}%
\pgfsys@useobject{currentmarker}{}%
\end{pgfscope}%
\end{pgfscope}%
\begin{pgfscope}%
\pgfsetbuttcap%
\pgfsetroundjoin%
\definecolor{currentfill}{rgb}{0.000000,0.000000,0.000000}%
\pgfsetfillcolor{currentfill}%
\pgfsetlinewidth{0.602250pt}%
\definecolor{currentstroke}{rgb}{0.000000,0.000000,0.000000}%
\pgfsetstrokecolor{currentstroke}%
\pgfsetdash{}{0pt}%
\pgfsys@defobject{currentmarker}{\pgfqpoint{-0.027778in}{0.000000in}}{\pgfqpoint{-0.000000in}{0.000000in}}{%
\pgfpathmoveto{\pgfqpoint{-0.000000in}{0.000000in}}%
\pgfpathlineto{\pgfqpoint{-0.027778in}{0.000000in}}%
\pgfusepath{stroke,fill}%
}%
\begin{pgfscope}%
\pgfsys@transformshift{0.800000in}{1.963683in}%
\pgfsys@useobject{currentmarker}{}%
\end{pgfscope}%
\end{pgfscope}%
\begin{pgfscope}%
\pgfsetbuttcap%
\pgfsetroundjoin%
\definecolor{currentfill}{rgb}{0.000000,0.000000,0.000000}%
\pgfsetfillcolor{currentfill}%
\pgfsetlinewidth{0.602250pt}%
\definecolor{currentstroke}{rgb}{0.000000,0.000000,0.000000}%
\pgfsetstrokecolor{currentstroke}%
\pgfsetdash{}{0pt}%
\pgfsys@defobject{currentmarker}{\pgfqpoint{-0.027778in}{0.000000in}}{\pgfqpoint{-0.000000in}{0.000000in}}{%
\pgfpathmoveto{\pgfqpoint{-0.000000in}{0.000000in}}%
\pgfpathlineto{\pgfqpoint{-0.027778in}{0.000000in}}%
\pgfusepath{stroke,fill}%
}%
\begin{pgfscope}%
\pgfsys@transformshift{0.800000in}{2.048623in}%
\pgfsys@useobject{currentmarker}{}%
\end{pgfscope}%
\end{pgfscope}%
\begin{pgfscope}%
\pgfsetbuttcap%
\pgfsetroundjoin%
\definecolor{currentfill}{rgb}{0.000000,0.000000,0.000000}%
\pgfsetfillcolor{currentfill}%
\pgfsetlinewidth{0.602250pt}%
\definecolor{currentstroke}{rgb}{0.000000,0.000000,0.000000}%
\pgfsetstrokecolor{currentstroke}%
\pgfsetdash{}{0pt}%
\pgfsys@defobject{currentmarker}{\pgfqpoint{-0.027778in}{0.000000in}}{\pgfqpoint{-0.000000in}{0.000000in}}{%
\pgfpathmoveto{\pgfqpoint{-0.000000in}{0.000000in}}%
\pgfpathlineto{\pgfqpoint{-0.027778in}{0.000000in}}%
\pgfusepath{stroke,fill}%
}%
\begin{pgfscope}%
\pgfsys@transformshift{0.800000in}{2.114508in}%
\pgfsys@useobject{currentmarker}{}%
\end{pgfscope}%
\end{pgfscope}%
\begin{pgfscope}%
\pgfsetbuttcap%
\pgfsetroundjoin%
\definecolor{currentfill}{rgb}{0.000000,0.000000,0.000000}%
\pgfsetfillcolor{currentfill}%
\pgfsetlinewidth{0.602250pt}%
\definecolor{currentstroke}{rgb}{0.000000,0.000000,0.000000}%
\pgfsetstrokecolor{currentstroke}%
\pgfsetdash{}{0pt}%
\pgfsys@defobject{currentmarker}{\pgfqpoint{-0.027778in}{0.000000in}}{\pgfqpoint{-0.000000in}{0.000000in}}{%
\pgfpathmoveto{\pgfqpoint{-0.000000in}{0.000000in}}%
\pgfpathlineto{\pgfqpoint{-0.027778in}{0.000000in}}%
\pgfusepath{stroke,fill}%
}%
\begin{pgfscope}%
\pgfsys@transformshift{0.800000in}{2.168339in}%
\pgfsys@useobject{currentmarker}{}%
\end{pgfscope}%
\end{pgfscope}%
\begin{pgfscope}%
\pgfsetbuttcap%
\pgfsetroundjoin%
\definecolor{currentfill}{rgb}{0.000000,0.000000,0.000000}%
\pgfsetfillcolor{currentfill}%
\pgfsetlinewidth{0.602250pt}%
\definecolor{currentstroke}{rgb}{0.000000,0.000000,0.000000}%
\pgfsetstrokecolor{currentstroke}%
\pgfsetdash{}{0pt}%
\pgfsys@defobject{currentmarker}{\pgfqpoint{-0.027778in}{0.000000in}}{\pgfqpoint{-0.000000in}{0.000000in}}{%
\pgfpathmoveto{\pgfqpoint{-0.000000in}{0.000000in}}%
\pgfpathlineto{\pgfqpoint{-0.027778in}{0.000000in}}%
\pgfusepath{stroke,fill}%
}%
\begin{pgfscope}%
\pgfsys@transformshift{0.800000in}{2.213853in}%
\pgfsys@useobject{currentmarker}{}%
\end{pgfscope}%
\end{pgfscope}%
\begin{pgfscope}%
\pgfsetbuttcap%
\pgfsetroundjoin%
\definecolor{currentfill}{rgb}{0.000000,0.000000,0.000000}%
\pgfsetfillcolor{currentfill}%
\pgfsetlinewidth{0.602250pt}%
\definecolor{currentstroke}{rgb}{0.000000,0.000000,0.000000}%
\pgfsetstrokecolor{currentstroke}%
\pgfsetdash{}{0pt}%
\pgfsys@defobject{currentmarker}{\pgfqpoint{-0.027778in}{0.000000in}}{\pgfqpoint{-0.000000in}{0.000000in}}{%
\pgfpathmoveto{\pgfqpoint{-0.000000in}{0.000000in}}%
\pgfpathlineto{\pgfqpoint{-0.027778in}{0.000000in}}%
\pgfusepath{stroke,fill}%
}%
\begin{pgfscope}%
\pgfsys@transformshift{0.800000in}{2.253279in}%
\pgfsys@useobject{currentmarker}{}%
\end{pgfscope}%
\end{pgfscope}%
\begin{pgfscope}%
\pgfsetbuttcap%
\pgfsetroundjoin%
\definecolor{currentfill}{rgb}{0.000000,0.000000,0.000000}%
\pgfsetfillcolor{currentfill}%
\pgfsetlinewidth{0.602250pt}%
\definecolor{currentstroke}{rgb}{0.000000,0.000000,0.000000}%
\pgfsetstrokecolor{currentstroke}%
\pgfsetdash{}{0pt}%
\pgfsys@defobject{currentmarker}{\pgfqpoint{-0.027778in}{0.000000in}}{\pgfqpoint{-0.000000in}{0.000000in}}{%
\pgfpathmoveto{\pgfqpoint{-0.000000in}{0.000000in}}%
\pgfpathlineto{\pgfqpoint{-0.027778in}{0.000000in}}%
\pgfusepath{stroke,fill}%
}%
\begin{pgfscope}%
\pgfsys@transformshift{0.800000in}{2.288055in}%
\pgfsys@useobject{currentmarker}{}%
\end{pgfscope}%
\end{pgfscope}%
\begin{pgfscope}%
\pgfsetbuttcap%
\pgfsetroundjoin%
\definecolor{currentfill}{rgb}{0.000000,0.000000,0.000000}%
\pgfsetfillcolor{currentfill}%
\pgfsetlinewidth{0.602250pt}%
\definecolor{currentstroke}{rgb}{0.000000,0.000000,0.000000}%
\pgfsetstrokecolor{currentstroke}%
\pgfsetdash{}{0pt}%
\pgfsys@defobject{currentmarker}{\pgfqpoint{-0.027778in}{0.000000in}}{\pgfqpoint{-0.000000in}{0.000000in}}{%
\pgfpathmoveto{\pgfqpoint{-0.000000in}{0.000000in}}%
\pgfpathlineto{\pgfqpoint{-0.027778in}{0.000000in}}%
\pgfusepath{stroke,fill}%
}%
\begin{pgfscope}%
\pgfsys@transformshift{0.800000in}{2.523819in}%
\pgfsys@useobject{currentmarker}{}%
\end{pgfscope}%
\end{pgfscope}%
\begin{pgfscope}%
\pgfsetbuttcap%
\pgfsetroundjoin%
\definecolor{currentfill}{rgb}{0.000000,0.000000,0.000000}%
\pgfsetfillcolor{currentfill}%
\pgfsetlinewidth{0.602250pt}%
\definecolor{currentstroke}{rgb}{0.000000,0.000000,0.000000}%
\pgfsetstrokecolor{currentstroke}%
\pgfsetdash{}{0pt}%
\pgfsys@defobject{currentmarker}{\pgfqpoint{-0.027778in}{0.000000in}}{\pgfqpoint{-0.000000in}{0.000000in}}{%
\pgfpathmoveto{\pgfqpoint{-0.000000in}{0.000000in}}%
\pgfpathlineto{\pgfqpoint{-0.027778in}{0.000000in}}%
\pgfusepath{stroke,fill}%
}%
\begin{pgfscope}%
\pgfsys@transformshift{0.800000in}{2.643535in}%
\pgfsys@useobject{currentmarker}{}%
\end{pgfscope}%
\end{pgfscope}%
\begin{pgfscope}%
\pgfsetbuttcap%
\pgfsetroundjoin%
\definecolor{currentfill}{rgb}{0.000000,0.000000,0.000000}%
\pgfsetfillcolor{currentfill}%
\pgfsetlinewidth{0.602250pt}%
\definecolor{currentstroke}{rgb}{0.000000,0.000000,0.000000}%
\pgfsetstrokecolor{currentstroke}%
\pgfsetdash{}{0pt}%
\pgfsys@defobject{currentmarker}{\pgfqpoint{-0.027778in}{0.000000in}}{\pgfqpoint{-0.000000in}{0.000000in}}{%
\pgfpathmoveto{\pgfqpoint{-0.000000in}{0.000000in}}%
\pgfpathlineto{\pgfqpoint{-0.027778in}{0.000000in}}%
\pgfusepath{stroke,fill}%
}%
\begin{pgfscope}%
\pgfsys@transformshift{0.800000in}{2.728475in}%
\pgfsys@useobject{currentmarker}{}%
\end{pgfscope}%
\end{pgfscope}%
\begin{pgfscope}%
\pgfsetbuttcap%
\pgfsetroundjoin%
\definecolor{currentfill}{rgb}{0.000000,0.000000,0.000000}%
\pgfsetfillcolor{currentfill}%
\pgfsetlinewidth{0.602250pt}%
\definecolor{currentstroke}{rgb}{0.000000,0.000000,0.000000}%
\pgfsetstrokecolor{currentstroke}%
\pgfsetdash{}{0pt}%
\pgfsys@defobject{currentmarker}{\pgfqpoint{-0.027778in}{0.000000in}}{\pgfqpoint{-0.000000in}{0.000000in}}{%
\pgfpathmoveto{\pgfqpoint{-0.000000in}{0.000000in}}%
\pgfpathlineto{\pgfqpoint{-0.027778in}{0.000000in}}%
\pgfusepath{stroke,fill}%
}%
\begin{pgfscope}%
\pgfsys@transformshift{0.800000in}{2.794360in}%
\pgfsys@useobject{currentmarker}{}%
\end{pgfscope}%
\end{pgfscope}%
\begin{pgfscope}%
\pgfsetbuttcap%
\pgfsetroundjoin%
\definecolor{currentfill}{rgb}{0.000000,0.000000,0.000000}%
\pgfsetfillcolor{currentfill}%
\pgfsetlinewidth{0.602250pt}%
\definecolor{currentstroke}{rgb}{0.000000,0.000000,0.000000}%
\pgfsetstrokecolor{currentstroke}%
\pgfsetdash{}{0pt}%
\pgfsys@defobject{currentmarker}{\pgfqpoint{-0.027778in}{0.000000in}}{\pgfqpoint{-0.000000in}{0.000000in}}{%
\pgfpathmoveto{\pgfqpoint{-0.000000in}{0.000000in}}%
\pgfpathlineto{\pgfqpoint{-0.027778in}{0.000000in}}%
\pgfusepath{stroke,fill}%
}%
\begin{pgfscope}%
\pgfsys@transformshift{0.800000in}{2.848191in}%
\pgfsys@useobject{currentmarker}{}%
\end{pgfscope}%
\end{pgfscope}%
\begin{pgfscope}%
\pgfsetbuttcap%
\pgfsetroundjoin%
\definecolor{currentfill}{rgb}{0.000000,0.000000,0.000000}%
\pgfsetfillcolor{currentfill}%
\pgfsetlinewidth{0.602250pt}%
\definecolor{currentstroke}{rgb}{0.000000,0.000000,0.000000}%
\pgfsetstrokecolor{currentstroke}%
\pgfsetdash{}{0pt}%
\pgfsys@defobject{currentmarker}{\pgfqpoint{-0.027778in}{0.000000in}}{\pgfqpoint{-0.000000in}{0.000000in}}{%
\pgfpathmoveto{\pgfqpoint{-0.000000in}{0.000000in}}%
\pgfpathlineto{\pgfqpoint{-0.027778in}{0.000000in}}%
\pgfusepath{stroke,fill}%
}%
\begin{pgfscope}%
\pgfsys@transformshift{0.800000in}{2.893705in}%
\pgfsys@useobject{currentmarker}{}%
\end{pgfscope}%
\end{pgfscope}%
\begin{pgfscope}%
\pgfsetbuttcap%
\pgfsetroundjoin%
\definecolor{currentfill}{rgb}{0.000000,0.000000,0.000000}%
\pgfsetfillcolor{currentfill}%
\pgfsetlinewidth{0.602250pt}%
\definecolor{currentstroke}{rgb}{0.000000,0.000000,0.000000}%
\pgfsetstrokecolor{currentstroke}%
\pgfsetdash{}{0pt}%
\pgfsys@defobject{currentmarker}{\pgfqpoint{-0.027778in}{0.000000in}}{\pgfqpoint{-0.000000in}{0.000000in}}{%
\pgfpathmoveto{\pgfqpoint{-0.000000in}{0.000000in}}%
\pgfpathlineto{\pgfqpoint{-0.027778in}{0.000000in}}%
\pgfusepath{stroke,fill}%
}%
\begin{pgfscope}%
\pgfsys@transformshift{0.800000in}{2.933131in}%
\pgfsys@useobject{currentmarker}{}%
\end{pgfscope}%
\end{pgfscope}%
\begin{pgfscope}%
\pgfsetbuttcap%
\pgfsetroundjoin%
\definecolor{currentfill}{rgb}{0.000000,0.000000,0.000000}%
\pgfsetfillcolor{currentfill}%
\pgfsetlinewidth{0.602250pt}%
\definecolor{currentstroke}{rgb}{0.000000,0.000000,0.000000}%
\pgfsetstrokecolor{currentstroke}%
\pgfsetdash{}{0pt}%
\pgfsys@defobject{currentmarker}{\pgfqpoint{-0.027778in}{0.000000in}}{\pgfqpoint{-0.000000in}{0.000000in}}{%
\pgfpathmoveto{\pgfqpoint{-0.000000in}{0.000000in}}%
\pgfpathlineto{\pgfqpoint{-0.027778in}{0.000000in}}%
\pgfusepath{stroke,fill}%
}%
\begin{pgfscope}%
\pgfsys@transformshift{0.800000in}{2.967907in}%
\pgfsys@useobject{currentmarker}{}%
\end{pgfscope}%
\end{pgfscope}%
\begin{pgfscope}%
\pgfsetbuttcap%
\pgfsetroundjoin%
\definecolor{currentfill}{rgb}{0.000000,0.000000,0.000000}%
\pgfsetfillcolor{currentfill}%
\pgfsetlinewidth{0.602250pt}%
\definecolor{currentstroke}{rgb}{0.000000,0.000000,0.000000}%
\pgfsetstrokecolor{currentstroke}%
\pgfsetdash{}{0pt}%
\pgfsys@defobject{currentmarker}{\pgfqpoint{-0.027778in}{0.000000in}}{\pgfqpoint{-0.000000in}{0.000000in}}{%
\pgfpathmoveto{\pgfqpoint{-0.000000in}{0.000000in}}%
\pgfpathlineto{\pgfqpoint{-0.027778in}{0.000000in}}%
\pgfusepath{stroke,fill}%
}%
\begin{pgfscope}%
\pgfsys@transformshift{0.800000in}{3.203671in}%
\pgfsys@useobject{currentmarker}{}%
\end{pgfscope}%
\end{pgfscope}%
\begin{pgfscope}%
\pgfsetbuttcap%
\pgfsetroundjoin%
\definecolor{currentfill}{rgb}{0.000000,0.000000,0.000000}%
\pgfsetfillcolor{currentfill}%
\pgfsetlinewidth{0.602250pt}%
\definecolor{currentstroke}{rgb}{0.000000,0.000000,0.000000}%
\pgfsetstrokecolor{currentstroke}%
\pgfsetdash{}{0pt}%
\pgfsys@defobject{currentmarker}{\pgfqpoint{-0.027778in}{0.000000in}}{\pgfqpoint{-0.000000in}{0.000000in}}{%
\pgfpathmoveto{\pgfqpoint{-0.000000in}{0.000000in}}%
\pgfpathlineto{\pgfqpoint{-0.027778in}{0.000000in}}%
\pgfusepath{stroke,fill}%
}%
\begin{pgfscope}%
\pgfsys@transformshift{0.800000in}{3.323387in}%
\pgfsys@useobject{currentmarker}{}%
\end{pgfscope}%
\end{pgfscope}%
\begin{pgfscope}%
\pgfsetbuttcap%
\pgfsetroundjoin%
\definecolor{currentfill}{rgb}{0.000000,0.000000,0.000000}%
\pgfsetfillcolor{currentfill}%
\pgfsetlinewidth{0.602250pt}%
\definecolor{currentstroke}{rgb}{0.000000,0.000000,0.000000}%
\pgfsetstrokecolor{currentstroke}%
\pgfsetdash{}{0pt}%
\pgfsys@defobject{currentmarker}{\pgfqpoint{-0.027778in}{0.000000in}}{\pgfqpoint{-0.000000in}{0.000000in}}{%
\pgfpathmoveto{\pgfqpoint{-0.000000in}{0.000000in}}%
\pgfpathlineto{\pgfqpoint{-0.027778in}{0.000000in}}%
\pgfusepath{stroke,fill}%
}%
\begin{pgfscope}%
\pgfsys@transformshift{0.800000in}{3.408327in}%
\pgfsys@useobject{currentmarker}{}%
\end{pgfscope}%
\end{pgfscope}%
\begin{pgfscope}%
\pgfsetbuttcap%
\pgfsetroundjoin%
\definecolor{currentfill}{rgb}{0.000000,0.000000,0.000000}%
\pgfsetfillcolor{currentfill}%
\pgfsetlinewidth{0.602250pt}%
\definecolor{currentstroke}{rgb}{0.000000,0.000000,0.000000}%
\pgfsetstrokecolor{currentstroke}%
\pgfsetdash{}{0pt}%
\pgfsys@defobject{currentmarker}{\pgfqpoint{-0.027778in}{0.000000in}}{\pgfqpoint{-0.000000in}{0.000000in}}{%
\pgfpathmoveto{\pgfqpoint{-0.000000in}{0.000000in}}%
\pgfpathlineto{\pgfqpoint{-0.027778in}{0.000000in}}%
\pgfusepath{stroke,fill}%
}%
\begin{pgfscope}%
\pgfsys@transformshift{0.800000in}{3.474212in}%
\pgfsys@useobject{currentmarker}{}%
\end{pgfscope}%
\end{pgfscope}%
\begin{pgfscope}%
\pgfsetbuttcap%
\pgfsetroundjoin%
\definecolor{currentfill}{rgb}{0.000000,0.000000,0.000000}%
\pgfsetfillcolor{currentfill}%
\pgfsetlinewidth{0.602250pt}%
\definecolor{currentstroke}{rgb}{0.000000,0.000000,0.000000}%
\pgfsetstrokecolor{currentstroke}%
\pgfsetdash{}{0pt}%
\pgfsys@defobject{currentmarker}{\pgfqpoint{-0.027778in}{0.000000in}}{\pgfqpoint{-0.000000in}{0.000000in}}{%
\pgfpathmoveto{\pgfqpoint{-0.000000in}{0.000000in}}%
\pgfpathlineto{\pgfqpoint{-0.027778in}{0.000000in}}%
\pgfusepath{stroke,fill}%
}%
\begin{pgfscope}%
\pgfsys@transformshift{0.800000in}{3.528043in}%
\pgfsys@useobject{currentmarker}{}%
\end{pgfscope}%
\end{pgfscope}%
\begin{pgfscope}%
\pgfsetbuttcap%
\pgfsetroundjoin%
\definecolor{currentfill}{rgb}{0.000000,0.000000,0.000000}%
\pgfsetfillcolor{currentfill}%
\pgfsetlinewidth{0.602250pt}%
\definecolor{currentstroke}{rgb}{0.000000,0.000000,0.000000}%
\pgfsetstrokecolor{currentstroke}%
\pgfsetdash{}{0pt}%
\pgfsys@defobject{currentmarker}{\pgfqpoint{-0.027778in}{0.000000in}}{\pgfqpoint{-0.000000in}{0.000000in}}{%
\pgfpathmoveto{\pgfqpoint{-0.000000in}{0.000000in}}%
\pgfpathlineto{\pgfqpoint{-0.027778in}{0.000000in}}%
\pgfusepath{stroke,fill}%
}%
\begin{pgfscope}%
\pgfsys@transformshift{0.800000in}{3.573557in}%
\pgfsys@useobject{currentmarker}{}%
\end{pgfscope}%
\end{pgfscope}%
\begin{pgfscope}%
\pgfsetbuttcap%
\pgfsetroundjoin%
\definecolor{currentfill}{rgb}{0.000000,0.000000,0.000000}%
\pgfsetfillcolor{currentfill}%
\pgfsetlinewidth{0.602250pt}%
\definecolor{currentstroke}{rgb}{0.000000,0.000000,0.000000}%
\pgfsetstrokecolor{currentstroke}%
\pgfsetdash{}{0pt}%
\pgfsys@defobject{currentmarker}{\pgfqpoint{-0.027778in}{0.000000in}}{\pgfqpoint{-0.000000in}{0.000000in}}{%
\pgfpathmoveto{\pgfqpoint{-0.000000in}{0.000000in}}%
\pgfpathlineto{\pgfqpoint{-0.027778in}{0.000000in}}%
\pgfusepath{stroke,fill}%
}%
\begin{pgfscope}%
\pgfsys@transformshift{0.800000in}{3.612983in}%
\pgfsys@useobject{currentmarker}{}%
\end{pgfscope}%
\end{pgfscope}%
\begin{pgfscope}%
\pgfsetbuttcap%
\pgfsetroundjoin%
\definecolor{currentfill}{rgb}{0.000000,0.000000,0.000000}%
\pgfsetfillcolor{currentfill}%
\pgfsetlinewidth{0.602250pt}%
\definecolor{currentstroke}{rgb}{0.000000,0.000000,0.000000}%
\pgfsetstrokecolor{currentstroke}%
\pgfsetdash{}{0pt}%
\pgfsys@defobject{currentmarker}{\pgfqpoint{-0.027778in}{0.000000in}}{\pgfqpoint{-0.000000in}{0.000000in}}{%
\pgfpathmoveto{\pgfqpoint{-0.000000in}{0.000000in}}%
\pgfpathlineto{\pgfqpoint{-0.027778in}{0.000000in}}%
\pgfusepath{stroke,fill}%
}%
\begin{pgfscope}%
\pgfsys@transformshift{0.800000in}{3.647759in}%
\pgfsys@useobject{currentmarker}{}%
\end{pgfscope}%
\end{pgfscope}%
\begin{pgfscope}%
\pgfsetbuttcap%
\pgfsetroundjoin%
\definecolor{currentfill}{rgb}{0.000000,0.000000,0.000000}%
\pgfsetfillcolor{currentfill}%
\pgfsetlinewidth{0.602250pt}%
\definecolor{currentstroke}{rgb}{0.000000,0.000000,0.000000}%
\pgfsetstrokecolor{currentstroke}%
\pgfsetdash{}{0pt}%
\pgfsys@defobject{currentmarker}{\pgfqpoint{-0.027778in}{0.000000in}}{\pgfqpoint{-0.000000in}{0.000000in}}{%
\pgfpathmoveto{\pgfqpoint{-0.000000in}{0.000000in}}%
\pgfpathlineto{\pgfqpoint{-0.027778in}{0.000000in}}%
\pgfusepath{stroke,fill}%
}%
\begin{pgfscope}%
\pgfsys@transformshift{0.800000in}{3.883523in}%
\pgfsys@useobject{currentmarker}{}%
\end{pgfscope}%
\end{pgfscope}%
\begin{pgfscope}%
\pgfsetbuttcap%
\pgfsetroundjoin%
\definecolor{currentfill}{rgb}{0.000000,0.000000,0.000000}%
\pgfsetfillcolor{currentfill}%
\pgfsetlinewidth{0.602250pt}%
\definecolor{currentstroke}{rgb}{0.000000,0.000000,0.000000}%
\pgfsetstrokecolor{currentstroke}%
\pgfsetdash{}{0pt}%
\pgfsys@defobject{currentmarker}{\pgfqpoint{-0.027778in}{0.000000in}}{\pgfqpoint{-0.000000in}{0.000000in}}{%
\pgfpathmoveto{\pgfqpoint{-0.000000in}{0.000000in}}%
\pgfpathlineto{\pgfqpoint{-0.027778in}{0.000000in}}%
\pgfusepath{stroke,fill}%
}%
\begin{pgfscope}%
\pgfsys@transformshift{0.800000in}{4.003239in}%
\pgfsys@useobject{currentmarker}{}%
\end{pgfscope}%
\end{pgfscope}%
\begin{pgfscope}%
\pgfsetbuttcap%
\pgfsetroundjoin%
\definecolor{currentfill}{rgb}{0.000000,0.000000,0.000000}%
\pgfsetfillcolor{currentfill}%
\pgfsetlinewidth{0.602250pt}%
\definecolor{currentstroke}{rgb}{0.000000,0.000000,0.000000}%
\pgfsetstrokecolor{currentstroke}%
\pgfsetdash{}{0pt}%
\pgfsys@defobject{currentmarker}{\pgfqpoint{-0.027778in}{0.000000in}}{\pgfqpoint{-0.000000in}{0.000000in}}{%
\pgfpathmoveto{\pgfqpoint{-0.000000in}{0.000000in}}%
\pgfpathlineto{\pgfqpoint{-0.027778in}{0.000000in}}%
\pgfusepath{stroke,fill}%
}%
\begin{pgfscope}%
\pgfsys@transformshift{0.800000in}{4.088179in}%
\pgfsys@useobject{currentmarker}{}%
\end{pgfscope}%
\end{pgfscope}%
\begin{pgfscope}%
\pgfsetbuttcap%
\pgfsetroundjoin%
\definecolor{currentfill}{rgb}{0.000000,0.000000,0.000000}%
\pgfsetfillcolor{currentfill}%
\pgfsetlinewidth{0.602250pt}%
\definecolor{currentstroke}{rgb}{0.000000,0.000000,0.000000}%
\pgfsetstrokecolor{currentstroke}%
\pgfsetdash{}{0pt}%
\pgfsys@defobject{currentmarker}{\pgfqpoint{-0.027778in}{0.000000in}}{\pgfqpoint{-0.000000in}{0.000000in}}{%
\pgfpathmoveto{\pgfqpoint{-0.000000in}{0.000000in}}%
\pgfpathlineto{\pgfqpoint{-0.027778in}{0.000000in}}%
\pgfusepath{stroke,fill}%
}%
\begin{pgfscope}%
\pgfsys@transformshift{0.800000in}{4.154064in}%
\pgfsys@useobject{currentmarker}{}%
\end{pgfscope}%
\end{pgfscope}%
\begin{pgfscope}%
\pgfsetbuttcap%
\pgfsetroundjoin%
\definecolor{currentfill}{rgb}{0.000000,0.000000,0.000000}%
\pgfsetfillcolor{currentfill}%
\pgfsetlinewidth{0.602250pt}%
\definecolor{currentstroke}{rgb}{0.000000,0.000000,0.000000}%
\pgfsetstrokecolor{currentstroke}%
\pgfsetdash{}{0pt}%
\pgfsys@defobject{currentmarker}{\pgfqpoint{-0.027778in}{0.000000in}}{\pgfqpoint{-0.000000in}{0.000000in}}{%
\pgfpathmoveto{\pgfqpoint{-0.000000in}{0.000000in}}%
\pgfpathlineto{\pgfqpoint{-0.027778in}{0.000000in}}%
\pgfusepath{stroke,fill}%
}%
\begin{pgfscope}%
\pgfsys@transformshift{0.800000in}{4.207895in}%
\pgfsys@useobject{currentmarker}{}%
\end{pgfscope}%
\end{pgfscope}%
\begin{pgfscope}%
\definecolor{textcolor}{rgb}{0.000000,0.000000,0.000000}%
\pgfsetstrokecolor{textcolor}%
\pgfsetfillcolor{textcolor}%
\pgftext[x=0.390663in,y=2.376000in,,bottom,rotate=90.000000]{\color{textcolor}\rmfamily\fontsize{10.000000}{12.000000}\selectfont TIME in log}%
\end{pgfscope}%
\begin{pgfscope}%
\pgfpathrectangle{\pgfqpoint{0.800000in}{0.528000in}}{\pgfqpoint{4.960000in}{3.696000in}}%
\pgfusepath{clip}%
\pgfsetrectcap%
\pgfsetroundjoin%
\pgfsetlinewidth{1.505625pt}%
\definecolor{currentstroke}{rgb}{0.121569,0.466667,0.705882}%
\pgfsetstrokecolor{currentstroke}%
\pgfsetdash{}{0pt}%
\pgfpathmoveto{\pgfqpoint{1.025455in}{0.893051in}}%
\pgfpathlineto{\pgfqpoint{1.071466in}{1.151479in}}%
\pgfpathlineto{\pgfqpoint{1.117477in}{1.301438in}}%
\pgfpathlineto{\pgfqpoint{1.163488in}{1.423189in}}%
\pgfpathlineto{\pgfqpoint{1.209499in}{1.501823in}}%
\pgfpathlineto{\pgfqpoint{1.255510in}{1.594019in}}%
\pgfpathlineto{\pgfqpoint{1.301521in}{1.648471in}}%
\pgfpathlineto{\pgfqpoint{1.347532in}{1.685613in}}%
\pgfpathlineto{\pgfqpoint{1.393544in}{1.734632in}}%
\pgfpathlineto{\pgfqpoint{1.439555in}{1.778057in}}%
\pgfpathlineto{\pgfqpoint{1.485566in}{1.808345in}}%
\pgfpathlineto{\pgfqpoint{1.531577in}{1.888531in}}%
\pgfpathlineto{\pgfqpoint{1.577588in}{1.860667in}}%
\pgfpathlineto{\pgfqpoint{1.623599in}{1.873311in}}%
\pgfpathlineto{\pgfqpoint{1.669610in}{1.908309in}}%
\pgfpathlineto{\pgfqpoint{1.715622in}{1.939378in}}%
\pgfpathlineto{\pgfqpoint{1.761633in}{1.952628in}}%
\pgfpathlineto{\pgfqpoint{1.807644in}{1.983621in}}%
\pgfpathlineto{\pgfqpoint{1.853655in}{2.001483in}}%
\pgfpathlineto{\pgfqpoint{1.899666in}{2.017123in}}%
\pgfpathlineto{\pgfqpoint{1.945677in}{2.047299in}}%
\pgfpathlineto{\pgfqpoint{1.991688in}{2.046563in}}%
\pgfpathlineto{\pgfqpoint{2.037699in}{2.068270in}}%
\pgfpathlineto{\pgfqpoint{2.083711in}{2.094436in}}%
\pgfpathlineto{\pgfqpoint{2.129722in}{2.086275in}}%
\pgfpathlineto{\pgfqpoint{2.175733in}{2.124379in}}%
\pgfpathlineto{\pgfqpoint{2.221744in}{2.121001in}}%
\pgfpathlineto{\pgfqpoint{2.267755in}{2.149088in}}%
\pgfpathlineto{\pgfqpoint{2.313766in}{2.146291in}}%
\pgfpathlineto{\pgfqpoint{2.359777in}{2.164448in}}%
\pgfpathlineto{\pgfqpoint{2.405788in}{2.170503in}}%
\pgfpathlineto{\pgfqpoint{2.451800in}{2.176974in}}%
\pgfpathlineto{\pgfqpoint{2.497811in}{2.225641in}}%
\pgfpathlineto{\pgfqpoint{2.543822in}{2.205798in}}%
\pgfpathlineto{\pgfqpoint{2.589833in}{2.227250in}}%
\pgfpathlineto{\pgfqpoint{2.635844in}{2.220079in}}%
\pgfpathlineto{\pgfqpoint{2.681855in}{2.241890in}}%
\pgfpathlineto{\pgfqpoint{2.727866in}{2.237374in}}%
\pgfpathlineto{\pgfqpoint{2.773878in}{2.249372in}}%
\pgfpathlineto{\pgfqpoint{2.819889in}{2.274844in}}%
\pgfpathlineto{\pgfqpoint{2.865900in}{2.271726in}}%
\pgfpathlineto{\pgfqpoint{2.911911in}{2.264920in}}%
\pgfpathlineto{\pgfqpoint{2.957922in}{2.277979in}}%
\pgfpathlineto{\pgfqpoint{3.003933in}{2.283441in}}%
\pgfpathlineto{\pgfqpoint{3.049944in}{2.317281in}}%
\pgfpathlineto{\pgfqpoint{3.095955in}{2.311631in}}%
\pgfpathlineto{\pgfqpoint{3.141967in}{2.322259in}}%
\pgfpathlineto{\pgfqpoint{3.187978in}{2.341876in}}%
\pgfpathlineto{\pgfqpoint{3.233989in}{2.361381in}}%
\pgfpathlineto{\pgfqpoint{3.280000in}{2.340960in}}%
\pgfpathlineto{\pgfqpoint{3.326011in}{2.348442in}}%
\pgfpathlineto{\pgfqpoint{3.372022in}{2.348196in}}%
\pgfpathlineto{\pgfqpoint{3.418033in}{2.364151in}}%
\pgfpathlineto{\pgfqpoint{3.464045in}{2.345236in}}%
\pgfpathlineto{\pgfqpoint{3.510056in}{2.380002in}}%
\pgfpathlineto{\pgfqpoint{3.556067in}{2.361440in}}%
\pgfpathlineto{\pgfqpoint{3.602078in}{2.372732in}}%
\pgfpathlineto{\pgfqpoint{3.648089in}{2.394971in}}%
\pgfpathlineto{\pgfqpoint{3.694100in}{2.384595in}}%
\pgfpathlineto{\pgfqpoint{3.740111in}{2.391545in}}%
\pgfpathlineto{\pgfqpoint{3.786122in}{2.401445in}}%
\pgfpathlineto{\pgfqpoint{3.832134in}{2.393278in}}%
\pgfpathlineto{\pgfqpoint{3.878145in}{2.407270in}}%
\pgfpathlineto{\pgfqpoint{3.924156in}{2.413239in}}%
\pgfpathlineto{\pgfqpoint{3.970167in}{2.450592in}}%
\pgfpathlineto{\pgfqpoint{4.016178in}{2.411982in}}%
\pgfpathlineto{\pgfqpoint{4.062189in}{2.446408in}}%
\pgfpathlineto{\pgfqpoint{4.108200in}{2.431160in}}%
\pgfpathlineto{\pgfqpoint{4.154212in}{2.436465in}}%
\pgfpathlineto{\pgfqpoint{4.200223in}{2.437345in}}%
\pgfpathlineto{\pgfqpoint{4.246234in}{2.445560in}}%
\pgfpathlineto{\pgfqpoint{4.292245in}{2.480263in}}%
\pgfpathlineto{\pgfqpoint{4.338256in}{2.453633in}}%
\pgfpathlineto{\pgfqpoint{4.384267in}{2.453177in}}%
\pgfpathlineto{\pgfqpoint{4.430278in}{2.467645in}}%
\pgfpathlineto{\pgfqpoint{4.476289in}{2.471696in}}%
\pgfpathlineto{\pgfqpoint{4.522301in}{2.477929in}}%
\pgfpathlineto{\pgfqpoint{4.568312in}{2.482382in}}%
\pgfpathlineto{\pgfqpoint{4.614323in}{2.489202in}}%
\pgfpathlineto{\pgfqpoint{4.660334in}{2.497399in}}%
\pgfpathlineto{\pgfqpoint{4.706345in}{2.490774in}}%
\pgfpathlineto{\pgfqpoint{4.752356in}{2.511991in}}%
\pgfpathlineto{\pgfqpoint{4.798367in}{2.502176in}}%
\pgfpathlineto{\pgfqpoint{4.844378in}{2.512426in}}%
\pgfpathlineto{\pgfqpoint{4.890390in}{2.496493in}}%
\pgfpathlineto{\pgfqpoint{4.936401in}{2.515694in}}%
\pgfpathlineto{\pgfqpoint{4.982412in}{2.524908in}}%
\pgfpathlineto{\pgfqpoint{5.028423in}{2.546384in}}%
\pgfpathlineto{\pgfqpoint{5.074434in}{2.521714in}}%
\pgfpathlineto{\pgfqpoint{5.120445in}{2.534794in}}%
\pgfpathlineto{\pgfqpoint{5.166456in}{2.517182in}}%
\pgfpathlineto{\pgfqpoint{5.212468in}{2.551688in}}%
\pgfpathlineto{\pgfqpoint{5.258479in}{2.555696in}}%
\pgfpathlineto{\pgfqpoint{5.304490in}{2.544261in}}%
\pgfpathlineto{\pgfqpoint{5.350501in}{2.554705in}}%
\pgfpathlineto{\pgfqpoint{5.396512in}{2.549830in}}%
\pgfpathlineto{\pgfqpoint{5.442523in}{2.554009in}}%
\pgfpathlineto{\pgfqpoint{5.488534in}{2.582228in}}%
\pgfpathlineto{\pgfqpoint{5.534545in}{2.564471in}}%
\pgfusepath{stroke}%
\end{pgfscope}%
\begin{pgfscope}%
\pgfpathrectangle{\pgfqpoint{0.800000in}{0.528000in}}{\pgfqpoint{4.960000in}{3.696000in}}%
\pgfusepath{clip}%
\pgfsetrectcap%
\pgfsetroundjoin%
\pgfsetlinewidth{1.505625pt}%
\definecolor{currentstroke}{rgb}{1.000000,0.498039,0.054902}%
\pgfsetstrokecolor{currentstroke}%
\pgfsetdash{}{0pt}%
\pgfpathmoveto{\pgfqpoint{1.025455in}{0.968497in}}%
\pgfpathlineto{\pgfqpoint{1.071466in}{1.215889in}}%
\pgfpathlineto{\pgfqpoint{1.117477in}{1.354973in}}%
\pgfpathlineto{\pgfqpoint{1.163488in}{1.449099in}}%
\pgfpathlineto{\pgfqpoint{1.209499in}{1.520631in}}%
\pgfpathlineto{\pgfqpoint{1.255510in}{1.598404in}}%
\pgfpathlineto{\pgfqpoint{1.301521in}{1.647898in}}%
\pgfpathlineto{\pgfqpoint{1.347532in}{1.691784in}}%
\pgfpathlineto{\pgfqpoint{1.393544in}{1.731735in}}%
\pgfpathlineto{\pgfqpoint{1.439555in}{1.765330in}}%
\pgfpathlineto{\pgfqpoint{1.485566in}{1.809646in}}%
\pgfpathlineto{\pgfqpoint{1.531577in}{1.841034in}}%
\pgfpathlineto{\pgfqpoint{1.577588in}{1.866939in}}%
\pgfpathlineto{\pgfqpoint{1.623599in}{1.894041in}}%
\pgfpathlineto{\pgfqpoint{1.669610in}{1.924889in}}%
\pgfpathlineto{\pgfqpoint{1.715622in}{1.936367in}}%
\pgfpathlineto{\pgfqpoint{1.761633in}{1.957051in}}%
\pgfpathlineto{\pgfqpoint{1.807644in}{1.976426in}}%
\pgfpathlineto{\pgfqpoint{1.853655in}{2.002157in}}%
\pgfpathlineto{\pgfqpoint{1.899666in}{2.011051in}}%
\pgfpathlineto{\pgfqpoint{1.945677in}{2.038969in}}%
\pgfpathlineto{\pgfqpoint{1.991688in}{2.055293in}}%
\pgfpathlineto{\pgfqpoint{2.037699in}{2.069636in}}%
\pgfpathlineto{\pgfqpoint{2.083711in}{2.084831in}}%
\pgfpathlineto{\pgfqpoint{2.129722in}{2.096814in}}%
\pgfpathlineto{\pgfqpoint{2.175733in}{2.117761in}}%
\pgfpathlineto{\pgfqpoint{2.221744in}{2.122791in}}%
\pgfpathlineto{\pgfqpoint{2.267755in}{2.137398in}}%
\pgfpathlineto{\pgfqpoint{2.313766in}{2.154480in}}%
\pgfpathlineto{\pgfqpoint{2.359777in}{2.158336in}}%
\pgfpathlineto{\pgfqpoint{2.405788in}{2.169541in}}%
\pgfpathlineto{\pgfqpoint{2.451800in}{2.180605in}}%
\pgfpathlineto{\pgfqpoint{2.497811in}{2.190221in}}%
\pgfpathlineto{\pgfqpoint{2.543822in}{2.222287in}}%
\pgfpathlineto{\pgfqpoint{2.589833in}{2.209342in}}%
\pgfpathlineto{\pgfqpoint{2.635844in}{2.217945in}}%
\pgfpathlineto{\pgfqpoint{2.681855in}{2.230156in}}%
\pgfpathlineto{\pgfqpoint{2.727866in}{2.234323in}}%
\pgfpathlineto{\pgfqpoint{2.773878in}{2.243776in}}%
\pgfpathlineto{\pgfqpoint{2.819889in}{2.252604in}}%
\pgfpathlineto{\pgfqpoint{2.865900in}{2.260284in}}%
\pgfpathlineto{\pgfqpoint{2.911911in}{2.276166in}}%
\pgfpathlineto{\pgfqpoint{2.957922in}{2.283761in}}%
\pgfpathlineto{\pgfqpoint{3.003933in}{2.297629in}}%
\pgfpathlineto{\pgfqpoint{3.049944in}{2.298334in}}%
\pgfpathlineto{\pgfqpoint{3.095955in}{2.305654in}}%
\pgfpathlineto{\pgfqpoint{3.141967in}{2.314444in}}%
\pgfpathlineto{\pgfqpoint{3.187978in}{2.321240in}}%
\pgfpathlineto{\pgfqpoint{3.233989in}{2.327374in}}%
\pgfpathlineto{\pgfqpoint{3.280000in}{2.343091in}}%
\pgfpathlineto{\pgfqpoint{3.326011in}{2.342086in}}%
\pgfpathlineto{\pgfqpoint{3.372022in}{2.351955in}}%
\pgfpathlineto{\pgfqpoint{3.418033in}{2.354792in}}%
\pgfpathlineto{\pgfqpoint{3.464045in}{2.360110in}}%
\pgfpathlineto{\pgfqpoint{3.510056in}{2.366173in}}%
\pgfpathlineto{\pgfqpoint{3.556067in}{2.372848in}}%
\pgfpathlineto{\pgfqpoint{3.602078in}{2.382767in}}%
\pgfpathlineto{\pgfqpoint{3.648089in}{2.384148in}}%
\pgfpathlineto{\pgfqpoint{3.694100in}{2.389035in}}%
\pgfpathlineto{\pgfqpoint{3.740111in}{2.396262in}}%
\pgfpathlineto{\pgfqpoint{3.786122in}{2.401358in}}%
\pgfpathlineto{\pgfqpoint{3.832134in}{2.408075in}}%
\pgfpathlineto{\pgfqpoint{3.878145in}{2.409914in}}%
\pgfpathlineto{\pgfqpoint{3.924156in}{2.423909in}}%
\pgfpathlineto{\pgfqpoint{3.970167in}{2.421959in}}%
\pgfpathlineto{\pgfqpoint{4.016178in}{2.427068in}}%
\pgfpathlineto{\pgfqpoint{4.062189in}{2.435257in}}%
\pgfpathlineto{\pgfqpoint{4.108200in}{2.437267in}}%
\pgfpathlineto{\pgfqpoint{4.154212in}{2.450371in}}%
\pgfpathlineto{\pgfqpoint{4.200223in}{2.453555in}}%
\pgfpathlineto{\pgfqpoint{4.246234in}{2.470001in}}%
\pgfpathlineto{\pgfqpoint{4.292245in}{2.456875in}}%
\pgfpathlineto{\pgfqpoint{4.338256in}{2.459811in}}%
\pgfpathlineto{\pgfqpoint{4.384267in}{2.465795in}}%
\pgfpathlineto{\pgfqpoint{4.430278in}{2.468914in}}%
\pgfpathlineto{\pgfqpoint{4.476289in}{2.480546in}}%
\pgfpathlineto{\pgfqpoint{4.522301in}{2.477081in}}%
\pgfpathlineto{\pgfqpoint{4.568312in}{2.487535in}}%
\pgfpathlineto{\pgfqpoint{4.614323in}{2.486198in}}%
\pgfpathlineto{\pgfqpoint{4.660334in}{2.489012in}}%
\pgfpathlineto{\pgfqpoint{4.706345in}{2.497008in}}%
\pgfpathlineto{\pgfqpoint{4.752356in}{2.498234in}}%
\pgfpathlineto{\pgfqpoint{4.798367in}{2.508638in}}%
\pgfpathlineto{\pgfqpoint{4.844378in}{2.512250in}}%
\pgfpathlineto{\pgfqpoint{4.890390in}{2.521078in}}%
\pgfpathlineto{\pgfqpoint{4.936401in}{2.521458in}}%
\pgfpathlineto{\pgfqpoint{4.982412in}{2.535155in}}%
\pgfpathlineto{\pgfqpoint{5.028423in}{2.529724in}}%
\pgfpathlineto{\pgfqpoint{5.074434in}{2.537015in}}%
\pgfpathlineto{\pgfqpoint{5.120445in}{2.536481in}}%
\pgfpathlineto{\pgfqpoint{5.166456in}{2.549214in}}%
\pgfpathlineto{\pgfqpoint{5.212468in}{2.543241in}}%
\pgfpathlineto{\pgfqpoint{5.258479in}{2.561005in}}%
\pgfpathlineto{\pgfqpoint{5.304490in}{2.550003in}}%
\pgfpathlineto{\pgfqpoint{5.350501in}{2.563603in}}%
\pgfpathlineto{\pgfqpoint{5.396512in}{2.556231in}}%
\pgfpathlineto{\pgfqpoint{5.442523in}{2.560339in}}%
\pgfpathlineto{\pgfqpoint{5.488534in}{2.564021in}}%
\pgfpathlineto{\pgfqpoint{5.534545in}{2.567357in}}%
\pgfusepath{stroke}%
\end{pgfscope}%
\begin{pgfscope}%
\pgfpathrectangle{\pgfqpoint{0.800000in}{0.528000in}}{\pgfqpoint{4.960000in}{3.696000in}}%
\pgfusepath{clip}%
\pgfsetrectcap%
\pgfsetroundjoin%
\pgfsetlinewidth{1.505625pt}%
\definecolor{currentstroke}{rgb}{0.172549,0.627451,0.172549}%
\pgfsetstrokecolor{currentstroke}%
\pgfsetdash{}{0pt}%
\pgfpathmoveto{\pgfqpoint{1.025455in}{1.137618in}}%
\pgfpathlineto{\pgfqpoint{1.071466in}{1.399481in}}%
\pgfpathlineto{\pgfqpoint{1.117477in}{1.550174in}}%
\pgfpathlineto{\pgfqpoint{1.163488in}{1.658805in}}%
\pgfpathlineto{\pgfqpoint{1.209499in}{1.742103in}}%
\pgfpathlineto{\pgfqpoint{1.255510in}{1.811520in}}%
\pgfpathlineto{\pgfqpoint{1.301521in}{1.869715in}}%
\pgfpathlineto{\pgfqpoint{1.347532in}{1.919945in}}%
\pgfpathlineto{\pgfqpoint{1.393544in}{1.962280in}}%
\pgfpathlineto{\pgfqpoint{1.439555in}{1.999250in}}%
\pgfpathlineto{\pgfqpoint{1.485566in}{2.032012in}}%
\pgfpathlineto{\pgfqpoint{1.531577in}{2.063388in}}%
\pgfpathlineto{\pgfqpoint{1.577588in}{2.093032in}}%
\pgfpathlineto{\pgfqpoint{1.623599in}{2.121032in}}%
\pgfpathlineto{\pgfqpoint{1.669610in}{2.144464in}}%
\pgfpathlineto{\pgfqpoint{1.715622in}{2.167605in}}%
\pgfpathlineto{\pgfqpoint{1.761633in}{2.188790in}}%
\pgfpathlineto{\pgfqpoint{1.807644in}{2.210013in}}%
\pgfpathlineto{\pgfqpoint{1.853655in}{2.234386in}}%
\pgfpathlineto{\pgfqpoint{1.899666in}{2.248126in}}%
\pgfpathlineto{\pgfqpoint{1.945677in}{2.263954in}}%
\pgfpathlineto{\pgfqpoint{1.991688in}{2.280393in}}%
\pgfpathlineto{\pgfqpoint{2.037699in}{2.295894in}}%
\pgfpathlineto{\pgfqpoint{2.083711in}{2.312086in}}%
\pgfpathlineto{\pgfqpoint{2.129722in}{2.326238in}}%
\pgfpathlineto{\pgfqpoint{2.175733in}{2.339709in}}%
\pgfpathlineto{\pgfqpoint{2.221744in}{2.353671in}}%
\pgfpathlineto{\pgfqpoint{2.267755in}{2.366501in}}%
\pgfpathlineto{\pgfqpoint{2.313766in}{2.377595in}}%
\pgfpathlineto{\pgfqpoint{2.359777in}{2.391662in}}%
\pgfpathlineto{\pgfqpoint{2.405788in}{2.401890in}}%
\pgfpathlineto{\pgfqpoint{2.451800in}{2.412550in}}%
\pgfpathlineto{\pgfqpoint{2.497811in}{2.423810in}}%
\pgfpathlineto{\pgfqpoint{2.543822in}{2.435091in}}%
\pgfpathlineto{\pgfqpoint{2.589833in}{2.444200in}}%
\pgfpathlineto{\pgfqpoint{2.635844in}{2.453534in}}%
\pgfpathlineto{\pgfqpoint{2.681855in}{2.464193in}}%
\pgfpathlineto{\pgfqpoint{2.727866in}{2.472193in}}%
\pgfpathlineto{\pgfqpoint{2.773878in}{2.481141in}}%
\pgfpathlineto{\pgfqpoint{2.819889in}{2.489838in}}%
\pgfpathlineto{\pgfqpoint{2.865900in}{2.498400in}}%
\pgfpathlineto{\pgfqpoint{2.911911in}{2.507390in}}%
\pgfpathlineto{\pgfqpoint{2.957922in}{2.514530in}}%
\pgfpathlineto{\pgfqpoint{3.003933in}{2.524200in}}%
\pgfpathlineto{\pgfqpoint{3.049944in}{2.530915in}}%
\pgfpathlineto{\pgfqpoint{3.095955in}{2.538606in}}%
\pgfpathlineto{\pgfqpoint{3.141967in}{2.546861in}}%
\pgfpathlineto{\pgfqpoint{3.187978in}{2.553499in}}%
\pgfpathlineto{\pgfqpoint{3.233989in}{2.560169in}}%
\pgfpathlineto{\pgfqpoint{3.280000in}{2.569222in}}%
\pgfpathlineto{\pgfqpoint{3.326011in}{2.574168in}}%
\pgfpathlineto{\pgfqpoint{3.372022in}{2.581815in}}%
\pgfpathlineto{\pgfqpoint{3.418033in}{2.588387in}}%
\pgfpathlineto{\pgfqpoint{3.464045in}{2.594436in}}%
\pgfpathlineto{\pgfqpoint{3.510056in}{2.602084in}}%
\pgfpathlineto{\pgfqpoint{3.556067in}{2.610493in}}%
\pgfpathlineto{\pgfqpoint{3.602078in}{2.615538in}}%
\pgfpathlineto{\pgfqpoint{3.648089in}{2.622274in}}%
\pgfpathlineto{\pgfqpoint{3.694100in}{2.627029in}}%
\pgfpathlineto{\pgfqpoint{3.740111in}{2.632980in}}%
\pgfpathlineto{\pgfqpoint{3.786122in}{2.638376in}}%
\pgfpathlineto{\pgfqpoint{3.832134in}{2.646004in}}%
\pgfpathlineto{\pgfqpoint{3.878145in}{2.650408in}}%
\pgfpathlineto{\pgfqpoint{3.924156in}{2.656035in}}%
\pgfpathlineto{\pgfqpoint{3.970167in}{2.660276in}}%
\pgfpathlineto{\pgfqpoint{4.016178in}{2.666864in}}%
\pgfpathlineto{\pgfqpoint{4.062189in}{2.669863in}}%
\pgfpathlineto{\pgfqpoint{4.108200in}{2.687365in}}%
\pgfpathlineto{\pgfqpoint{4.154212in}{2.681167in}}%
\pgfpathlineto{\pgfqpoint{4.200223in}{2.693778in}}%
\pgfpathlineto{\pgfqpoint{4.246234in}{2.690617in}}%
\pgfpathlineto{\pgfqpoint{4.292245in}{2.696384in}}%
\pgfpathlineto{\pgfqpoint{4.338256in}{2.700804in}}%
\pgfpathlineto{\pgfqpoint{4.384267in}{2.707920in}}%
\pgfpathlineto{\pgfqpoint{4.430278in}{2.709002in}}%
\pgfpathlineto{\pgfqpoint{4.476289in}{2.714415in}}%
\pgfpathlineto{\pgfqpoint{4.522301in}{2.717866in}}%
\pgfpathlineto{\pgfqpoint{4.568312in}{2.722354in}}%
\pgfpathlineto{\pgfqpoint{4.614323in}{2.731269in}}%
\pgfpathlineto{\pgfqpoint{4.660334in}{2.731948in}}%
\pgfpathlineto{\pgfqpoint{4.706345in}{2.734964in}}%
\pgfpathlineto{\pgfqpoint{4.752356in}{2.738645in}}%
\pgfpathlineto{\pgfqpoint{4.798367in}{2.742537in}}%
\pgfpathlineto{\pgfqpoint{4.844378in}{2.747148in}}%
\pgfpathlineto{\pgfqpoint{4.890390in}{2.749225in}}%
\pgfpathlineto{\pgfqpoint{4.936401in}{2.754446in}}%
\pgfpathlineto{\pgfqpoint{4.982412in}{2.757602in}}%
\pgfpathlineto{\pgfqpoint{5.028423in}{2.765731in}}%
\pgfpathlineto{\pgfqpoint{5.074434in}{2.767687in}}%
\pgfpathlineto{\pgfqpoint{5.120445in}{2.770482in}}%
\pgfpathlineto{\pgfqpoint{5.166456in}{2.773618in}}%
\pgfpathlineto{\pgfqpoint{5.212468in}{2.777132in}}%
\pgfpathlineto{\pgfqpoint{5.258479in}{2.780528in}}%
\pgfpathlineto{\pgfqpoint{5.304490in}{2.785473in}}%
\pgfpathlineto{\pgfqpoint{5.350501in}{2.788992in}}%
\pgfpathlineto{\pgfqpoint{5.396512in}{2.791419in}}%
\pgfpathlineto{\pgfqpoint{5.442523in}{2.794900in}}%
\pgfpathlineto{\pgfqpoint{5.488534in}{2.799251in}}%
\pgfpathlineto{\pgfqpoint{5.534545in}{2.802507in}}%
\pgfusepath{stroke}%
\end{pgfscope}%
\begin{pgfscope}%
\pgfpathrectangle{\pgfqpoint{0.800000in}{0.528000in}}{\pgfqpoint{4.960000in}{3.696000in}}%
\pgfusepath{clip}%
\pgfsetrectcap%
\pgfsetroundjoin%
\pgfsetlinewidth{1.505625pt}%
\definecolor{currentstroke}{rgb}{0.839216,0.152941,0.156863}%
\pgfsetstrokecolor{currentstroke}%
\pgfsetdash{}{0pt}%
\pgfpathmoveto{\pgfqpoint{1.025455in}{1.193317in}}%
\pgfpathlineto{\pgfqpoint{1.071466in}{1.581325in}}%
\pgfpathlineto{\pgfqpoint{1.117477in}{1.836120in}}%
\pgfpathlineto{\pgfqpoint{1.163488in}{2.051493in}}%
\pgfpathlineto{\pgfqpoint{1.209499in}{2.221044in}}%
\pgfpathlineto{\pgfqpoint{1.255510in}{2.348981in}}%
\pgfpathlineto{\pgfqpoint{1.301521in}{2.470511in}}%
\pgfpathlineto{\pgfqpoint{1.347532in}{2.545577in}}%
\pgfpathlineto{\pgfqpoint{1.393544in}{2.624182in}}%
\pgfpathlineto{\pgfqpoint{1.439555in}{2.694752in}}%
\pgfpathlineto{\pgfqpoint{1.485566in}{2.756248in}}%
\pgfpathlineto{\pgfqpoint{1.531577in}{2.821725in}}%
\pgfpathlineto{\pgfqpoint{1.577588in}{2.874060in}}%
\pgfpathlineto{\pgfqpoint{1.623599in}{2.914272in}}%
\pgfpathlineto{\pgfqpoint{1.669610in}{2.960408in}}%
\pgfpathlineto{\pgfqpoint{1.715622in}{3.010599in}}%
\pgfpathlineto{\pgfqpoint{1.761633in}{3.058530in}}%
\pgfpathlineto{\pgfqpoint{1.807644in}{3.088004in}}%
\pgfpathlineto{\pgfqpoint{1.853655in}{3.135068in}}%
\pgfpathlineto{\pgfqpoint{1.899666in}{3.157910in}}%
\pgfpathlineto{\pgfqpoint{1.945677in}{3.182870in}}%
\pgfpathlineto{\pgfqpoint{1.991688in}{3.216526in}}%
\pgfpathlineto{\pgfqpoint{2.037699in}{3.243867in}}%
\pgfpathlineto{\pgfqpoint{2.083711in}{3.273280in}}%
\pgfpathlineto{\pgfqpoint{2.129722in}{3.282576in}}%
\pgfpathlineto{\pgfqpoint{2.175733in}{3.301627in}}%
\pgfpathlineto{\pgfqpoint{2.221744in}{3.312483in}}%
\pgfpathlineto{\pgfqpoint{2.267755in}{3.344160in}}%
\pgfpathlineto{\pgfqpoint{2.313766in}{3.363837in}}%
\pgfpathlineto{\pgfqpoint{2.359777in}{3.405686in}}%
\pgfpathlineto{\pgfqpoint{2.405788in}{3.399068in}}%
\pgfpathlineto{\pgfqpoint{2.451800in}{3.414933in}}%
\pgfpathlineto{\pgfqpoint{2.497811in}{3.442557in}}%
\pgfpathlineto{\pgfqpoint{2.543822in}{3.479642in}}%
\pgfpathlineto{\pgfqpoint{2.589833in}{3.477351in}}%
\pgfpathlineto{\pgfqpoint{2.635844in}{3.495561in}}%
\pgfpathlineto{\pgfqpoint{2.681855in}{3.501961in}}%
\pgfpathlineto{\pgfqpoint{2.727866in}{3.534697in}}%
\pgfpathlineto{\pgfqpoint{2.773878in}{3.559232in}}%
\pgfpathlineto{\pgfqpoint{2.819889in}{3.581912in}}%
\pgfpathlineto{\pgfqpoint{2.865900in}{3.578775in}}%
\pgfpathlineto{\pgfqpoint{2.911911in}{3.584612in}}%
\pgfpathlineto{\pgfqpoint{2.957922in}{3.604182in}}%
\pgfpathlineto{\pgfqpoint{3.003933in}{3.619809in}}%
\pgfpathlineto{\pgfqpoint{3.049944in}{3.644802in}}%
\pgfpathlineto{\pgfqpoint{3.095955in}{3.655465in}}%
\pgfpathlineto{\pgfqpoint{3.141967in}{3.659522in}}%
\pgfpathlineto{\pgfqpoint{3.187978in}{3.685853in}}%
\pgfpathlineto{\pgfqpoint{3.233989in}{3.660708in}}%
\pgfpathlineto{\pgfqpoint{3.280000in}{3.696606in}}%
\pgfpathlineto{\pgfqpoint{3.326011in}{3.730294in}}%
\pgfpathlineto{\pgfqpoint{3.372022in}{3.729503in}}%
\pgfpathlineto{\pgfqpoint{3.418033in}{3.732597in}}%
\pgfpathlineto{\pgfqpoint{3.464045in}{3.746413in}}%
\pgfpathlineto{\pgfqpoint{3.510056in}{3.745479in}}%
\pgfpathlineto{\pgfqpoint{3.556067in}{3.769568in}}%
\pgfpathlineto{\pgfqpoint{3.602078in}{3.769546in}}%
\pgfpathlineto{\pgfqpoint{3.648089in}{3.789465in}}%
\pgfpathlineto{\pgfqpoint{3.694100in}{3.800953in}}%
\pgfpathlineto{\pgfqpoint{3.740111in}{3.816060in}}%
\pgfpathlineto{\pgfqpoint{3.786122in}{3.838546in}}%
\pgfpathlineto{\pgfqpoint{3.832134in}{3.834643in}}%
\pgfpathlineto{\pgfqpoint{3.878145in}{3.848871in}}%
\pgfpathlineto{\pgfqpoint{3.924156in}{3.837361in}}%
\pgfpathlineto{\pgfqpoint{3.970167in}{3.844054in}}%
\pgfpathlineto{\pgfqpoint{4.016178in}{3.874585in}}%
\pgfpathlineto{\pgfqpoint{4.062189in}{3.883285in}}%
\pgfpathlineto{\pgfqpoint{4.108200in}{3.889111in}}%
\pgfpathlineto{\pgfqpoint{4.108200in}{3.871611in}}%
\pgfpathlineto{\pgfqpoint{4.154212in}{3.897791in}}%
\pgfpathlineto{\pgfqpoint{4.200223in}{3.895936in}}%
\pgfpathlineto{\pgfqpoint{4.246234in}{3.916882in}}%
\pgfpathlineto{\pgfqpoint{4.292245in}{3.914868in}}%
\pgfpathlineto{\pgfqpoint{4.338256in}{3.925933in}}%
\pgfpathlineto{\pgfqpoint{4.384267in}{3.942652in}}%
\pgfpathlineto{\pgfqpoint{4.430278in}{3.942904in}}%
\pgfpathlineto{\pgfqpoint{4.476289in}{3.946679in}}%
\pgfpathlineto{\pgfqpoint{4.522301in}{3.951145in}}%
\pgfpathlineto{\pgfqpoint{4.568312in}{3.976064in}}%
\pgfpathlineto{\pgfqpoint{4.614323in}{3.967863in}}%
\pgfpathlineto{\pgfqpoint{4.660334in}{3.988244in}}%
\pgfpathlineto{\pgfqpoint{4.706345in}{3.994555in}}%
\pgfpathlineto{\pgfqpoint{4.752356in}{3.997856in}}%
\pgfpathlineto{\pgfqpoint{4.798367in}{3.988864in}}%
\pgfpathlineto{\pgfqpoint{4.844378in}{3.973090in}}%
\pgfpathlineto{\pgfqpoint{4.890390in}{3.973547in}}%
\pgfpathlineto{\pgfqpoint{4.936401in}{3.972108in}}%
\pgfpathlineto{\pgfqpoint{4.982412in}{3.982888in}}%
\pgfpathlineto{\pgfqpoint{5.028423in}{3.992226in}}%
\pgfpathlineto{\pgfqpoint{5.074434in}{3.989228in}}%
\pgfpathlineto{\pgfqpoint{5.120445in}{3.994678in}}%
\pgfpathlineto{\pgfqpoint{5.166456in}{4.000436in}}%
\pgfpathlineto{\pgfqpoint{5.212468in}{4.004734in}}%
\pgfpathlineto{\pgfqpoint{5.258479in}{4.016392in}}%
\pgfpathlineto{\pgfqpoint{5.304490in}{4.017961in}}%
\pgfpathlineto{\pgfqpoint{5.350501in}{4.027845in}}%
\pgfpathlineto{\pgfqpoint{5.396512in}{4.031386in}}%
\pgfpathlineto{\pgfqpoint{5.442523in}{4.038314in}}%
\pgfpathlineto{\pgfqpoint{5.488534in}{4.046640in}}%
\pgfpathlineto{\pgfqpoint{5.534545in}{4.056000in}}%
\pgfusepath{stroke}%
\end{pgfscope}%
\begin{pgfscope}%
\pgfpathrectangle{\pgfqpoint{0.800000in}{0.528000in}}{\pgfqpoint{4.960000in}{3.696000in}}%
\pgfusepath{clip}%
\pgfsetrectcap%
\pgfsetroundjoin%
\pgfsetlinewidth{1.505625pt}%
\definecolor{currentstroke}{rgb}{0.580392,0.403922,0.741176}%
\pgfsetstrokecolor{currentstroke}%
\pgfsetdash{}{0pt}%
\pgfpathmoveto{\pgfqpoint{1.025455in}{0.696000in}}%
\pgfpathlineto{\pgfqpoint{1.071466in}{0.978561in}}%
\pgfpathlineto{\pgfqpoint{1.117477in}{1.181064in}}%
\pgfpathlineto{\pgfqpoint{1.163488in}{1.282112in}}%
\pgfpathlineto{\pgfqpoint{1.209499in}{1.398551in}}%
\pgfpathlineto{\pgfqpoint{1.255510in}{1.471938in}}%
\pgfpathlineto{\pgfqpoint{1.301521in}{1.544542in}}%
\pgfpathlineto{\pgfqpoint{1.347532in}{1.584830in}}%
\pgfpathlineto{\pgfqpoint{1.393544in}{1.643965in}}%
\pgfpathlineto{\pgfqpoint{1.439555in}{1.699247in}}%
\pgfpathlineto{\pgfqpoint{1.485566in}{1.723019in}}%
\pgfpathlineto{\pgfqpoint{1.531577in}{1.777199in}}%
\pgfpathlineto{\pgfqpoint{1.577588in}{1.799260in}}%
\pgfpathlineto{\pgfqpoint{1.623599in}{1.833757in}}%
\pgfpathlineto{\pgfqpoint{1.669610in}{1.858052in}}%
\pgfpathlineto{\pgfqpoint{1.715622in}{1.894129in}}%
\pgfpathlineto{\pgfqpoint{1.761633in}{1.924661in}}%
\pgfpathlineto{\pgfqpoint{1.807644in}{1.953377in}}%
\pgfpathlineto{\pgfqpoint{1.853655in}{1.979361in}}%
\pgfpathlineto{\pgfqpoint{1.899666in}{1.999911in}}%
\pgfpathlineto{\pgfqpoint{1.945677in}{2.013342in}}%
\pgfpathlineto{\pgfqpoint{1.991688in}{2.026049in}}%
\pgfpathlineto{\pgfqpoint{2.037699in}{2.050385in}}%
\pgfpathlineto{\pgfqpoint{2.083711in}{2.075376in}}%
\pgfpathlineto{\pgfqpoint{2.129722in}{2.082391in}}%
\pgfpathlineto{\pgfqpoint{2.175733in}{2.110416in}}%
\pgfpathlineto{\pgfqpoint{2.221744in}{2.123083in}}%
\pgfpathlineto{\pgfqpoint{2.267755in}{2.144128in}}%
\pgfpathlineto{\pgfqpoint{2.313766in}{2.146576in}}%
\pgfpathlineto{\pgfqpoint{2.359777in}{2.145073in}}%
\pgfpathlineto{\pgfqpoint{2.405788in}{2.156958in}}%
\pgfpathlineto{\pgfqpoint{2.451800in}{2.183949in}}%
\pgfpathlineto{\pgfqpoint{2.497811in}{2.195439in}}%
\pgfpathlineto{\pgfqpoint{2.543822in}{2.207896in}}%
\pgfpathlineto{\pgfqpoint{2.589833in}{2.223414in}}%
\pgfpathlineto{\pgfqpoint{2.635844in}{2.232561in}}%
\pgfpathlineto{\pgfqpoint{2.681855in}{2.241542in}}%
\pgfpathlineto{\pgfqpoint{2.727866in}{2.244788in}}%
\pgfpathlineto{\pgfqpoint{2.773878in}{2.262572in}}%
\pgfpathlineto{\pgfqpoint{2.819889in}{2.278574in}}%
\pgfpathlineto{\pgfqpoint{2.865900in}{2.275983in}}%
\pgfpathlineto{\pgfqpoint{2.911911in}{2.298839in}}%
\pgfpathlineto{\pgfqpoint{2.957922in}{2.297125in}}%
\pgfpathlineto{\pgfqpoint{3.003933in}{2.310898in}}%
\pgfpathlineto{\pgfqpoint{3.049944in}{2.322191in}}%
\pgfpathlineto{\pgfqpoint{3.095955in}{2.331942in}}%
\pgfpathlineto{\pgfqpoint{3.141967in}{2.343750in}}%
\pgfpathlineto{\pgfqpoint{3.187978in}{2.354646in}}%
\pgfpathlineto{\pgfqpoint{3.233989in}{2.358040in}}%
\pgfpathlineto{\pgfqpoint{3.280000in}{2.367091in}}%
\pgfpathlineto{\pgfqpoint{3.326011in}{2.375057in}}%
\pgfpathlineto{\pgfqpoint{3.372022in}{2.382268in}}%
\pgfpathlineto{\pgfqpoint{3.418033in}{2.396338in}}%
\pgfpathlineto{\pgfqpoint{3.464045in}{2.402029in}}%
\pgfpathlineto{\pgfqpoint{3.510056in}{2.412568in}}%
\pgfpathlineto{\pgfqpoint{3.556067in}{2.418784in}}%
\pgfpathlineto{\pgfqpoint{3.602078in}{2.424214in}}%
\pgfpathlineto{\pgfqpoint{3.648089in}{2.428365in}}%
\pgfpathlineto{\pgfqpoint{3.694100in}{2.446524in}}%
\pgfpathlineto{\pgfqpoint{3.740111in}{2.447643in}}%
\pgfpathlineto{\pgfqpoint{3.786122in}{2.452334in}}%
\pgfpathlineto{\pgfqpoint{3.832134in}{2.462113in}}%
\pgfpathlineto{\pgfqpoint{3.878145in}{2.468708in}}%
\pgfpathlineto{\pgfqpoint{3.924156in}{2.473090in}}%
\pgfpathlineto{\pgfqpoint{3.970167in}{2.485290in}}%
\pgfpathlineto{\pgfqpoint{4.016178in}{2.488697in}}%
\pgfpathlineto{\pgfqpoint{4.062189in}{2.496769in}}%
\pgfpathlineto{\pgfqpoint{4.108200in}{2.501041in}}%
\pgfpathlineto{\pgfqpoint{4.154212in}{2.511133in}}%
\pgfpathlineto{\pgfqpoint{4.200223in}{2.515200in}}%
\pgfpathlineto{\pgfqpoint{4.246234in}{2.516513in}}%
\pgfpathlineto{\pgfqpoint{4.292245in}{2.524218in}}%
\pgfpathlineto{\pgfqpoint{4.338256in}{2.531680in}}%
\pgfpathlineto{\pgfqpoint{4.384267in}{2.534732in}}%
\pgfpathlineto{\pgfqpoint{4.430278in}{2.541913in}}%
\pgfpathlineto{\pgfqpoint{4.476289in}{2.549184in}}%
\pgfpathlineto{\pgfqpoint{4.522301in}{2.556481in}}%
\pgfpathlineto{\pgfqpoint{4.568312in}{2.560167in}}%
\pgfpathlineto{\pgfqpoint{4.614323in}{2.580079in}}%
\pgfpathlineto{\pgfqpoint{4.660334in}{2.575744in}}%
\pgfpathlineto{\pgfqpoint{4.706345in}{2.578466in}}%
\pgfpathlineto{\pgfqpoint{4.752356in}{2.580801in}}%
\pgfpathlineto{\pgfqpoint{4.798367in}{2.585020in}}%
\pgfpathlineto{\pgfqpoint{4.844378in}{2.592039in}}%
\pgfpathlineto{\pgfqpoint{4.890390in}{2.599422in}}%
\pgfpathlineto{\pgfqpoint{4.936401in}{2.607565in}}%
\pgfpathlineto{\pgfqpoint{4.982412in}{2.610790in}}%
\pgfpathlineto{\pgfqpoint{5.028423in}{2.618804in}}%
\pgfpathlineto{\pgfqpoint{5.074434in}{2.614812in}}%
\pgfpathlineto{\pgfqpoint{5.120445in}{2.621277in}}%
\pgfpathlineto{\pgfqpoint{5.166456in}{2.635661in}}%
\pgfpathlineto{\pgfqpoint{5.212468in}{2.635517in}}%
\pgfpathlineto{\pgfqpoint{5.258479in}{2.644674in}}%
\pgfpathlineto{\pgfqpoint{5.304490in}{2.643430in}}%
\pgfpathlineto{\pgfqpoint{5.350501in}{2.653820in}}%
\pgfpathlineto{\pgfqpoint{5.396512in}{2.651142in}}%
\pgfpathlineto{\pgfqpoint{5.442523in}{2.661435in}}%
\pgfpathlineto{\pgfqpoint{5.488534in}{2.666207in}}%
\pgfpathlineto{\pgfqpoint{5.534545in}{2.665483in}}%
\pgfusepath{stroke}%
\end{pgfscope}%
\begin{pgfscope}%
\pgfsetrectcap%
\pgfsetmiterjoin%
\pgfsetlinewidth{0.803000pt}%
\definecolor{currentstroke}{rgb}{0.000000,0.000000,0.000000}%
\pgfsetstrokecolor{currentstroke}%
\pgfsetdash{}{0pt}%
\pgfpathmoveto{\pgfqpoint{0.800000in}{0.528000in}}%
\pgfpathlineto{\pgfqpoint{0.800000in}{4.224000in}}%
\pgfusepath{stroke}%
\end{pgfscope}%
\begin{pgfscope}%
\pgfsetrectcap%
\pgfsetmiterjoin%
\pgfsetlinewidth{0.803000pt}%
\definecolor{currentstroke}{rgb}{0.000000,0.000000,0.000000}%
\pgfsetstrokecolor{currentstroke}%
\pgfsetdash{}{0pt}%
\pgfpathmoveto{\pgfqpoint{5.760000in}{0.528000in}}%
\pgfpathlineto{\pgfqpoint{5.760000in}{4.224000in}}%
\pgfusepath{stroke}%
\end{pgfscope}%
\begin{pgfscope}%
\pgfsetrectcap%
\pgfsetmiterjoin%
\pgfsetlinewidth{0.803000pt}%
\definecolor{currentstroke}{rgb}{0.000000,0.000000,0.000000}%
\pgfsetstrokecolor{currentstroke}%
\pgfsetdash{}{0pt}%
\pgfpathmoveto{\pgfqpoint{0.800000in}{0.528000in}}%
\pgfpathlineto{\pgfqpoint{5.760000in}{0.528000in}}%
\pgfusepath{stroke}%
\end{pgfscope}%
\begin{pgfscope}%
\pgfsetrectcap%
\pgfsetmiterjoin%
\pgfsetlinewidth{0.803000pt}%
\definecolor{currentstroke}{rgb}{0.000000,0.000000,0.000000}%
\pgfsetstrokecolor{currentstroke}%
\pgfsetdash{}{0pt}%
\pgfpathmoveto{\pgfqpoint{0.800000in}{4.224000in}}%
\pgfpathlineto{\pgfqpoint{5.760000in}{4.224000in}}%
\pgfusepath{stroke}%
\end{pgfscope}%
\begin{pgfscope}%
\pgfsetbuttcap%
\pgfsetmiterjoin%
\definecolor{currentfill}{rgb}{1.000000,1.000000,1.000000}%
\pgfsetfillcolor{currentfill}%
\pgfsetfillopacity{0.800000}%
\pgfsetlinewidth{1.003750pt}%
\definecolor{currentstroke}{rgb}{0.800000,0.800000,0.800000}%
\pgfsetstrokecolor{currentstroke}%
\pgfsetstrokeopacity{0.800000}%
\pgfsetdash{}{0pt}%
\pgfpathmoveto{\pgfqpoint{0.897222in}{3.144525in}}%
\pgfpathlineto{\pgfqpoint{1.739044in}{3.144525in}}%
\pgfpathquadraticcurveto{\pgfqpoint{1.766822in}{3.144525in}}{\pgfqpoint{1.766822in}{3.172303in}}%
\pgfpathlineto{\pgfqpoint{1.766822in}{4.126778in}}%
\pgfpathquadraticcurveto{\pgfqpoint{1.766822in}{4.154556in}}{\pgfqpoint{1.739044in}{4.154556in}}%
\pgfpathlineto{\pgfqpoint{0.897222in}{4.154556in}}%
\pgfpathquadraticcurveto{\pgfqpoint{0.869444in}{4.154556in}}{\pgfqpoint{0.869444in}{4.126778in}}%
\pgfpathlineto{\pgfqpoint{0.869444in}{3.172303in}}%
\pgfpathquadraticcurveto{\pgfqpoint{0.869444in}{3.144525in}}{\pgfqpoint{0.897222in}{3.144525in}}%
\pgfpathlineto{\pgfqpoint{0.897222in}{3.144525in}}%
\pgfpathclose%
\pgfusepath{stroke,fill}%
\end{pgfscope}%
\begin{pgfscope}%
\pgfsetrectcap%
\pgfsetroundjoin%
\pgfsetlinewidth{1.505625pt}%
\definecolor{currentstroke}{rgb}{0.121569,0.466667,0.705882}%
\pgfsetstrokecolor{currentstroke}%
\pgfsetdash{}{0pt}%
\pgfpathmoveto{\pgfqpoint{0.925000in}{4.050389in}}%
\pgfpathlineto{\pgfqpoint{1.063889in}{4.050389in}}%
\pgfpathlineto{\pgfqpoint{1.202778in}{4.050389in}}%
\pgfusepath{stroke}%
\end{pgfscope}%
\begin{pgfscope}%
\definecolor{textcolor}{rgb}{0.000000,0.000000,0.000000}%
\pgfsetstrokecolor{textcolor}%
\pgfsetfillcolor{textcolor}%
\pgftext[x=1.313889in,y=4.001778in,left,base]{\color{textcolor}\rmfamily\fontsize{10.000000}{12.000000}\selectfont quick}%
\end{pgfscope}%
\begin{pgfscope}%
\pgfsetrectcap%
\pgfsetroundjoin%
\pgfsetlinewidth{1.505625pt}%
\definecolor{currentstroke}{rgb}{1.000000,0.498039,0.054902}%
\pgfsetstrokecolor{currentstroke}%
\pgfsetdash{}{0pt}%
\pgfpathmoveto{\pgfqpoint{0.925000in}{3.856716in}}%
\pgfpathlineto{\pgfqpoint{1.063889in}{3.856716in}}%
\pgfpathlineto{\pgfqpoint{1.202778in}{3.856716in}}%
\pgfusepath{stroke}%
\end{pgfscope}%
\begin{pgfscope}%
\definecolor{textcolor}{rgb}{0.000000,0.000000,0.000000}%
\pgfsetstrokecolor{textcolor}%
\pgfsetfillcolor{textcolor}%
\pgftext[x=1.313889in,y=3.808105in,left,base]{\color{textcolor}\rmfamily\fontsize{10.000000}{12.000000}\selectfont merge}%
\end{pgfscope}%
\begin{pgfscope}%
\pgfsetrectcap%
\pgfsetroundjoin%
\pgfsetlinewidth{1.505625pt}%
\definecolor{currentstroke}{rgb}{0.172549,0.627451,0.172549}%
\pgfsetstrokecolor{currentstroke}%
\pgfsetdash{}{0pt}%
\pgfpathmoveto{\pgfqpoint{0.925000in}{3.663043in}}%
\pgfpathlineto{\pgfqpoint{1.063889in}{3.663043in}}%
\pgfpathlineto{\pgfqpoint{1.202778in}{3.663043in}}%
\pgfusepath{stroke}%
\end{pgfscope}%
\begin{pgfscope}%
\definecolor{textcolor}{rgb}{0.000000,0.000000,0.000000}%
\pgfsetstrokecolor{textcolor}%
\pgfsetfillcolor{textcolor}%
\pgftext[x=1.313889in,y=3.614432in,left,base]{\color{textcolor}\rmfamily\fontsize{10.000000}{12.000000}\selectfont heap}%
\end{pgfscope}%
\begin{pgfscope}%
\pgfsetrectcap%
\pgfsetroundjoin%
\pgfsetlinewidth{1.505625pt}%
\definecolor{currentstroke}{rgb}{0.839216,0.152941,0.156863}%
\pgfsetstrokecolor{currentstroke}%
\pgfsetdash{}{0pt}%
\pgfpathmoveto{\pgfqpoint{0.925000in}{3.469371in}}%
\pgfpathlineto{\pgfqpoint{1.063889in}{3.469371in}}%
\pgfpathlineto{\pgfqpoint{1.202778in}{3.469371in}}%
\pgfusepath{stroke}%
\end{pgfscope}%
\begin{pgfscope}%
\definecolor{textcolor}{rgb}{0.000000,0.000000,0.000000}%
\pgfsetstrokecolor{textcolor}%
\pgfsetfillcolor{textcolor}%
\pgftext[x=1.313889in,y=3.420759in,left,base]{\color{textcolor}\rmfamily\fontsize{10.000000}{12.000000}\selectfont insert}%
\end{pgfscope}%
\begin{pgfscope}%
\pgfsetrectcap%
\pgfsetroundjoin%
\pgfsetlinewidth{1.505625pt}%
\definecolor{currentstroke}{rgb}{0.580392,0.403922,0.741176}%
\pgfsetstrokecolor{currentstroke}%
\pgfsetdash{}{0pt}%
\pgfpathmoveto{\pgfqpoint{0.925000in}{3.275698in}}%
\pgfpathlineto{\pgfqpoint{1.063889in}{3.275698in}}%
\pgfpathlineto{\pgfqpoint{1.202778in}{3.275698in}}%
\pgfusepath{stroke}%
\end{pgfscope}%
\begin{pgfscope}%
\definecolor{textcolor}{rgb}{0.000000,0.000000,0.000000}%
\pgfsetstrokecolor{textcolor}%
\pgfsetfillcolor{textcolor}%
\pgftext[x=1.313889in,y=3.227087in,left,base]{\color{textcolor}\rmfamily\fontsize{10.000000}{12.000000}\selectfont bucket}%
\end{pgfscope}%
\end{pgfpicture}%
\makeatother%
\endgroup%

%% Creator: Matplotlib, PGF backend
%%
%% To include the figure in your LaTeX document, write
%%   \input{<filename>.pgf}
%%
%% Make sure the required packages are loaded in your preamble
%%   \usepackage{pgf}
%%
%% Also ensure that all the required font packages are loaded; for instance,
%% the lmodern package is sometimes necessary when using math font.
%%   \usepackage{lmodern}
%%
%% Figures using additional raster images can only be included by \input if
%% they are in the same directory as the main LaTeX file. For loading figures
%% from other directories you can use the `import` package
%%   \usepackage{import}
%%
%% and then include the figures with
%%   \import{<path to file>}{<filename>.pgf}
%%
%% Matplotlib used the following preamble
%%   
%%   \makeatletter\@ifpackageloaded{underscore}{}{\usepackage[strings]{underscore}}\makeatother
%%
\begingroup%
\makeatletter%
\begin{pgfpicture}%
\pgfpathrectangle{\pgfpointorigin}{\pgfqpoint{6.400000in}{4.800000in}}%
\pgfusepath{use as bounding box, clip}%
\begin{pgfscope}%
\pgfsetbuttcap%
\pgfsetmiterjoin%
\definecolor{currentfill}{rgb}{1.000000,1.000000,1.000000}%
\pgfsetfillcolor{currentfill}%
\pgfsetlinewidth{0.000000pt}%
\definecolor{currentstroke}{rgb}{1.000000,1.000000,1.000000}%
\pgfsetstrokecolor{currentstroke}%
\pgfsetdash{}{0pt}%
\pgfpathmoveto{\pgfqpoint{0.000000in}{0.000000in}}%
\pgfpathlineto{\pgfqpoint{6.400000in}{0.000000in}}%
\pgfpathlineto{\pgfqpoint{6.400000in}{4.800000in}}%
\pgfpathlineto{\pgfqpoint{0.000000in}{4.800000in}}%
\pgfpathlineto{\pgfqpoint{0.000000in}{0.000000in}}%
\pgfpathclose%
\pgfusepath{fill}%
\end{pgfscope}%
\begin{pgfscope}%
\pgfsetbuttcap%
\pgfsetmiterjoin%
\definecolor{currentfill}{rgb}{1.000000,1.000000,1.000000}%
\pgfsetfillcolor{currentfill}%
\pgfsetlinewidth{0.000000pt}%
\definecolor{currentstroke}{rgb}{0.000000,0.000000,0.000000}%
\pgfsetstrokecolor{currentstroke}%
\pgfsetstrokeopacity{0.000000}%
\pgfsetdash{}{0pt}%
\pgfpathmoveto{\pgfqpoint{0.800000in}{0.528000in}}%
\pgfpathlineto{\pgfqpoint{5.760000in}{0.528000in}}%
\pgfpathlineto{\pgfqpoint{5.760000in}{4.224000in}}%
\pgfpathlineto{\pgfqpoint{0.800000in}{4.224000in}}%
\pgfpathlineto{\pgfqpoint{0.800000in}{0.528000in}}%
\pgfpathclose%
\pgfusepath{fill}%
\end{pgfscope}%
\begin{pgfscope}%
\pgfsetbuttcap%
\pgfsetroundjoin%
\definecolor{currentfill}{rgb}{0.000000,0.000000,0.000000}%
\pgfsetfillcolor{currentfill}%
\pgfsetlinewidth{0.803000pt}%
\definecolor{currentstroke}{rgb}{0.000000,0.000000,0.000000}%
\pgfsetstrokecolor{currentstroke}%
\pgfsetdash{}{0pt}%
\pgfsys@defobject{currentmarker}{\pgfqpoint{0.000000in}{-0.048611in}}{\pgfqpoint{0.000000in}{0.000000in}}{%
\pgfpathmoveto{\pgfqpoint{0.000000in}{0.000000in}}%
\pgfpathlineto{\pgfqpoint{0.000000in}{-0.048611in}}%
\pgfusepath{stroke,fill}%
}%
\begin{pgfscope}%
\pgfsys@transformshift{0.979443in}{0.528000in}%
\pgfsys@useobject{currentmarker}{}%
\end{pgfscope}%
\end{pgfscope}%
\begin{pgfscope}%
\definecolor{textcolor}{rgb}{0.000000,0.000000,0.000000}%
\pgfsetstrokecolor{textcolor}%
\pgfsetfillcolor{textcolor}%
\pgftext[x=0.979443in,y=0.430778in,,top]{\color{textcolor}\rmfamily\fontsize{10.000000}{12.000000}\selectfont \(\displaystyle {0}\)}%
\end{pgfscope}%
\begin{pgfscope}%
\pgfsetbuttcap%
\pgfsetroundjoin%
\definecolor{currentfill}{rgb}{0.000000,0.000000,0.000000}%
\pgfsetfillcolor{currentfill}%
\pgfsetlinewidth{0.803000pt}%
\definecolor{currentstroke}{rgb}{0.000000,0.000000,0.000000}%
\pgfsetstrokecolor{currentstroke}%
\pgfsetdash{}{0pt}%
\pgfsys@defobject{currentmarker}{\pgfqpoint{0.000000in}{-0.048611in}}{\pgfqpoint{0.000000in}{0.000000in}}{%
\pgfpathmoveto{\pgfqpoint{0.000000in}{0.000000in}}%
\pgfpathlineto{\pgfqpoint{0.000000in}{-0.048611in}}%
\pgfusepath{stroke,fill}%
}%
\begin{pgfscope}%
\pgfsys@transformshift{1.899666in}{0.528000in}%
\pgfsys@useobject{currentmarker}{}%
\end{pgfscope}%
\end{pgfscope}%
\begin{pgfscope}%
\definecolor{textcolor}{rgb}{0.000000,0.000000,0.000000}%
\pgfsetstrokecolor{textcolor}%
\pgfsetfillcolor{textcolor}%
\pgftext[x=1.899666in,y=0.430778in,,top]{\color{textcolor}\rmfamily\fontsize{10.000000}{12.000000}\selectfont \(\displaystyle {2000}\)}%
\end{pgfscope}%
\begin{pgfscope}%
\pgfsetbuttcap%
\pgfsetroundjoin%
\definecolor{currentfill}{rgb}{0.000000,0.000000,0.000000}%
\pgfsetfillcolor{currentfill}%
\pgfsetlinewidth{0.803000pt}%
\definecolor{currentstroke}{rgb}{0.000000,0.000000,0.000000}%
\pgfsetstrokecolor{currentstroke}%
\pgfsetdash{}{0pt}%
\pgfsys@defobject{currentmarker}{\pgfqpoint{0.000000in}{-0.048611in}}{\pgfqpoint{0.000000in}{0.000000in}}{%
\pgfpathmoveto{\pgfqpoint{0.000000in}{0.000000in}}%
\pgfpathlineto{\pgfqpoint{0.000000in}{-0.048611in}}%
\pgfusepath{stroke,fill}%
}%
\begin{pgfscope}%
\pgfsys@transformshift{2.819889in}{0.528000in}%
\pgfsys@useobject{currentmarker}{}%
\end{pgfscope}%
\end{pgfscope}%
\begin{pgfscope}%
\definecolor{textcolor}{rgb}{0.000000,0.000000,0.000000}%
\pgfsetstrokecolor{textcolor}%
\pgfsetfillcolor{textcolor}%
\pgftext[x=2.819889in,y=0.430778in,,top]{\color{textcolor}\rmfamily\fontsize{10.000000}{12.000000}\selectfont \(\displaystyle {4000}\)}%
\end{pgfscope}%
\begin{pgfscope}%
\pgfsetbuttcap%
\pgfsetroundjoin%
\definecolor{currentfill}{rgb}{0.000000,0.000000,0.000000}%
\pgfsetfillcolor{currentfill}%
\pgfsetlinewidth{0.803000pt}%
\definecolor{currentstroke}{rgb}{0.000000,0.000000,0.000000}%
\pgfsetstrokecolor{currentstroke}%
\pgfsetdash{}{0pt}%
\pgfsys@defobject{currentmarker}{\pgfqpoint{0.000000in}{-0.048611in}}{\pgfqpoint{0.000000in}{0.000000in}}{%
\pgfpathmoveto{\pgfqpoint{0.000000in}{0.000000in}}%
\pgfpathlineto{\pgfqpoint{0.000000in}{-0.048611in}}%
\pgfusepath{stroke,fill}%
}%
\begin{pgfscope}%
\pgfsys@transformshift{3.740111in}{0.528000in}%
\pgfsys@useobject{currentmarker}{}%
\end{pgfscope}%
\end{pgfscope}%
\begin{pgfscope}%
\definecolor{textcolor}{rgb}{0.000000,0.000000,0.000000}%
\pgfsetstrokecolor{textcolor}%
\pgfsetfillcolor{textcolor}%
\pgftext[x=3.740111in,y=0.430778in,,top]{\color{textcolor}\rmfamily\fontsize{10.000000}{12.000000}\selectfont \(\displaystyle {6000}\)}%
\end{pgfscope}%
\begin{pgfscope}%
\pgfsetbuttcap%
\pgfsetroundjoin%
\definecolor{currentfill}{rgb}{0.000000,0.000000,0.000000}%
\pgfsetfillcolor{currentfill}%
\pgfsetlinewidth{0.803000pt}%
\definecolor{currentstroke}{rgb}{0.000000,0.000000,0.000000}%
\pgfsetstrokecolor{currentstroke}%
\pgfsetdash{}{0pt}%
\pgfsys@defobject{currentmarker}{\pgfqpoint{0.000000in}{-0.048611in}}{\pgfqpoint{0.000000in}{0.000000in}}{%
\pgfpathmoveto{\pgfqpoint{0.000000in}{0.000000in}}%
\pgfpathlineto{\pgfqpoint{0.000000in}{-0.048611in}}%
\pgfusepath{stroke,fill}%
}%
\begin{pgfscope}%
\pgfsys@transformshift{4.660334in}{0.528000in}%
\pgfsys@useobject{currentmarker}{}%
\end{pgfscope}%
\end{pgfscope}%
\begin{pgfscope}%
\definecolor{textcolor}{rgb}{0.000000,0.000000,0.000000}%
\pgfsetstrokecolor{textcolor}%
\pgfsetfillcolor{textcolor}%
\pgftext[x=4.660334in,y=0.430778in,,top]{\color{textcolor}\rmfamily\fontsize{10.000000}{12.000000}\selectfont \(\displaystyle {8000}\)}%
\end{pgfscope}%
\begin{pgfscope}%
\pgfsetbuttcap%
\pgfsetroundjoin%
\definecolor{currentfill}{rgb}{0.000000,0.000000,0.000000}%
\pgfsetfillcolor{currentfill}%
\pgfsetlinewidth{0.803000pt}%
\definecolor{currentstroke}{rgb}{0.000000,0.000000,0.000000}%
\pgfsetstrokecolor{currentstroke}%
\pgfsetdash{}{0pt}%
\pgfsys@defobject{currentmarker}{\pgfqpoint{0.000000in}{-0.048611in}}{\pgfqpoint{0.000000in}{0.000000in}}{%
\pgfpathmoveto{\pgfqpoint{0.000000in}{0.000000in}}%
\pgfpathlineto{\pgfqpoint{0.000000in}{-0.048611in}}%
\pgfusepath{stroke,fill}%
}%
\begin{pgfscope}%
\pgfsys@transformshift{5.580557in}{0.528000in}%
\pgfsys@useobject{currentmarker}{}%
\end{pgfscope}%
\end{pgfscope}%
\begin{pgfscope}%
\definecolor{textcolor}{rgb}{0.000000,0.000000,0.000000}%
\pgfsetstrokecolor{textcolor}%
\pgfsetfillcolor{textcolor}%
\pgftext[x=5.580557in,y=0.430778in,,top]{\color{textcolor}\rmfamily\fontsize{10.000000}{12.000000}\selectfont \(\displaystyle {10000}\)}%
\end{pgfscope}%
\begin{pgfscope}%
\definecolor{textcolor}{rgb}{0.000000,0.000000,0.000000}%
\pgfsetstrokecolor{textcolor}%
\pgfsetfillcolor{textcolor}%
\pgftext[x=3.280000in,y=0.251766in,,top]{\color{textcolor}\rmfamily\fontsize{10.000000}{12.000000}\selectfont Input Size}%
\end{pgfscope}%
\begin{pgfscope}%
\pgfsetbuttcap%
\pgfsetroundjoin%
\definecolor{currentfill}{rgb}{0.000000,0.000000,0.000000}%
\pgfsetfillcolor{currentfill}%
\pgfsetlinewidth{0.803000pt}%
\definecolor{currentstroke}{rgb}{0.000000,0.000000,0.000000}%
\pgfsetstrokecolor{currentstroke}%
\pgfsetdash{}{0pt}%
\pgfsys@defobject{currentmarker}{\pgfqpoint{-0.048611in}{0.000000in}}{\pgfqpoint{-0.000000in}{0.000000in}}{%
\pgfpathmoveto{\pgfqpoint{-0.000000in}{0.000000in}}%
\pgfpathlineto{\pgfqpoint{-0.048611in}{0.000000in}}%
\pgfusepath{stroke,fill}%
}%
\begin{pgfscope}%
\pgfsys@transformshift{0.800000in}{1.510745in}%
\pgfsys@useobject{currentmarker}{}%
\end{pgfscope}%
\end{pgfscope}%
\begin{pgfscope}%
\definecolor{textcolor}{rgb}{0.000000,0.000000,0.000000}%
\pgfsetstrokecolor{textcolor}%
\pgfsetfillcolor{textcolor}%
\pgftext[x=0.501581in, y=1.462520in, left, base]{\color{textcolor}\rmfamily\fontsize{10.000000}{12.000000}\selectfont \(\displaystyle {10^{3}}\)}%
\end{pgfscope}%
\begin{pgfscope}%
\pgfsetbuttcap%
\pgfsetroundjoin%
\definecolor{currentfill}{rgb}{0.000000,0.000000,0.000000}%
\pgfsetfillcolor{currentfill}%
\pgfsetlinewidth{0.803000pt}%
\definecolor{currentstroke}{rgb}{0.000000,0.000000,0.000000}%
\pgfsetstrokecolor{currentstroke}%
\pgfsetdash{}{0pt}%
\pgfsys@defobject{currentmarker}{\pgfqpoint{-0.048611in}{0.000000in}}{\pgfqpoint{-0.000000in}{0.000000in}}{%
\pgfpathmoveto{\pgfqpoint{-0.000000in}{0.000000in}}%
\pgfpathlineto{\pgfqpoint{-0.048611in}{0.000000in}}%
\pgfusepath{stroke,fill}%
}%
\begin{pgfscope}%
\pgfsys@transformshift{0.800000in}{2.647548in}%
\pgfsys@useobject{currentmarker}{}%
\end{pgfscope}%
\end{pgfscope}%
\begin{pgfscope}%
\definecolor{textcolor}{rgb}{0.000000,0.000000,0.000000}%
\pgfsetstrokecolor{textcolor}%
\pgfsetfillcolor{textcolor}%
\pgftext[x=0.501581in, y=2.599323in, left, base]{\color{textcolor}\rmfamily\fontsize{10.000000}{12.000000}\selectfont \(\displaystyle {10^{4}}\)}%
\end{pgfscope}%
\begin{pgfscope}%
\pgfsetbuttcap%
\pgfsetroundjoin%
\definecolor{currentfill}{rgb}{0.000000,0.000000,0.000000}%
\pgfsetfillcolor{currentfill}%
\pgfsetlinewidth{0.803000pt}%
\definecolor{currentstroke}{rgb}{0.000000,0.000000,0.000000}%
\pgfsetstrokecolor{currentstroke}%
\pgfsetdash{}{0pt}%
\pgfsys@defobject{currentmarker}{\pgfqpoint{-0.048611in}{0.000000in}}{\pgfqpoint{-0.000000in}{0.000000in}}{%
\pgfpathmoveto{\pgfqpoint{-0.000000in}{0.000000in}}%
\pgfpathlineto{\pgfqpoint{-0.048611in}{0.000000in}}%
\pgfusepath{stroke,fill}%
}%
\begin{pgfscope}%
\pgfsys@transformshift{0.800000in}{3.784351in}%
\pgfsys@useobject{currentmarker}{}%
\end{pgfscope}%
\end{pgfscope}%
\begin{pgfscope}%
\definecolor{textcolor}{rgb}{0.000000,0.000000,0.000000}%
\pgfsetstrokecolor{textcolor}%
\pgfsetfillcolor{textcolor}%
\pgftext[x=0.501581in, y=3.736126in, left, base]{\color{textcolor}\rmfamily\fontsize{10.000000}{12.000000}\selectfont \(\displaystyle {10^{5}}\)}%
\end{pgfscope}%
\begin{pgfscope}%
\pgfsetbuttcap%
\pgfsetroundjoin%
\definecolor{currentfill}{rgb}{0.000000,0.000000,0.000000}%
\pgfsetfillcolor{currentfill}%
\pgfsetlinewidth{0.602250pt}%
\definecolor{currentstroke}{rgb}{0.000000,0.000000,0.000000}%
\pgfsetstrokecolor{currentstroke}%
\pgfsetdash{}{0pt}%
\pgfsys@defobject{currentmarker}{\pgfqpoint{-0.027778in}{0.000000in}}{\pgfqpoint{-0.000000in}{0.000000in}}{%
\pgfpathmoveto{\pgfqpoint{-0.000000in}{0.000000in}}%
\pgfpathlineto{\pgfqpoint{-0.027778in}{0.000000in}}%
\pgfusepath{stroke,fill}%
}%
\begin{pgfscope}%
\pgfsys@transformshift{0.800000in}{0.716154in}%
\pgfsys@useobject{currentmarker}{}%
\end{pgfscope}%
\end{pgfscope}%
\begin{pgfscope}%
\pgfsetbuttcap%
\pgfsetroundjoin%
\definecolor{currentfill}{rgb}{0.000000,0.000000,0.000000}%
\pgfsetfillcolor{currentfill}%
\pgfsetlinewidth{0.602250pt}%
\definecolor{currentstroke}{rgb}{0.000000,0.000000,0.000000}%
\pgfsetstrokecolor{currentstroke}%
\pgfsetdash{}{0pt}%
\pgfsys@defobject{currentmarker}{\pgfqpoint{-0.027778in}{0.000000in}}{\pgfqpoint{-0.000000in}{0.000000in}}{%
\pgfpathmoveto{\pgfqpoint{-0.000000in}{0.000000in}}%
\pgfpathlineto{\pgfqpoint{-0.027778in}{0.000000in}}%
\pgfusepath{stroke,fill}%
}%
\begin{pgfscope}%
\pgfsys@transformshift{0.800000in}{0.916335in}%
\pgfsys@useobject{currentmarker}{}%
\end{pgfscope}%
\end{pgfscope}%
\begin{pgfscope}%
\pgfsetbuttcap%
\pgfsetroundjoin%
\definecolor{currentfill}{rgb}{0.000000,0.000000,0.000000}%
\pgfsetfillcolor{currentfill}%
\pgfsetlinewidth{0.602250pt}%
\definecolor{currentstroke}{rgb}{0.000000,0.000000,0.000000}%
\pgfsetstrokecolor{currentstroke}%
\pgfsetdash{}{0pt}%
\pgfsys@defobject{currentmarker}{\pgfqpoint{-0.027778in}{0.000000in}}{\pgfqpoint{-0.000000in}{0.000000in}}{%
\pgfpathmoveto{\pgfqpoint{-0.000000in}{0.000000in}}%
\pgfpathlineto{\pgfqpoint{-0.027778in}{0.000000in}}%
\pgfusepath{stroke,fill}%
}%
\begin{pgfscope}%
\pgfsys@transformshift{0.800000in}{1.058366in}%
\pgfsys@useobject{currentmarker}{}%
\end{pgfscope}%
\end{pgfscope}%
\begin{pgfscope}%
\pgfsetbuttcap%
\pgfsetroundjoin%
\definecolor{currentfill}{rgb}{0.000000,0.000000,0.000000}%
\pgfsetfillcolor{currentfill}%
\pgfsetlinewidth{0.602250pt}%
\definecolor{currentstroke}{rgb}{0.000000,0.000000,0.000000}%
\pgfsetstrokecolor{currentstroke}%
\pgfsetdash{}{0pt}%
\pgfsys@defobject{currentmarker}{\pgfqpoint{-0.027778in}{0.000000in}}{\pgfqpoint{-0.000000in}{0.000000in}}{%
\pgfpathmoveto{\pgfqpoint{-0.000000in}{0.000000in}}%
\pgfpathlineto{\pgfqpoint{-0.027778in}{0.000000in}}%
\pgfusepath{stroke,fill}%
}%
\begin{pgfscope}%
\pgfsys@transformshift{0.800000in}{1.168533in}%
\pgfsys@useobject{currentmarker}{}%
\end{pgfscope}%
\end{pgfscope}%
\begin{pgfscope}%
\pgfsetbuttcap%
\pgfsetroundjoin%
\definecolor{currentfill}{rgb}{0.000000,0.000000,0.000000}%
\pgfsetfillcolor{currentfill}%
\pgfsetlinewidth{0.602250pt}%
\definecolor{currentstroke}{rgb}{0.000000,0.000000,0.000000}%
\pgfsetstrokecolor{currentstroke}%
\pgfsetdash{}{0pt}%
\pgfsys@defobject{currentmarker}{\pgfqpoint{-0.027778in}{0.000000in}}{\pgfqpoint{-0.000000in}{0.000000in}}{%
\pgfpathmoveto{\pgfqpoint{-0.000000in}{0.000000in}}%
\pgfpathlineto{\pgfqpoint{-0.027778in}{0.000000in}}%
\pgfusepath{stroke,fill}%
}%
\begin{pgfscope}%
\pgfsys@transformshift{0.800000in}{1.258547in}%
\pgfsys@useobject{currentmarker}{}%
\end{pgfscope}%
\end{pgfscope}%
\begin{pgfscope}%
\pgfsetbuttcap%
\pgfsetroundjoin%
\definecolor{currentfill}{rgb}{0.000000,0.000000,0.000000}%
\pgfsetfillcolor{currentfill}%
\pgfsetlinewidth{0.602250pt}%
\definecolor{currentstroke}{rgb}{0.000000,0.000000,0.000000}%
\pgfsetstrokecolor{currentstroke}%
\pgfsetdash{}{0pt}%
\pgfsys@defobject{currentmarker}{\pgfqpoint{-0.027778in}{0.000000in}}{\pgfqpoint{-0.000000in}{0.000000in}}{%
\pgfpathmoveto{\pgfqpoint{-0.000000in}{0.000000in}}%
\pgfpathlineto{\pgfqpoint{-0.027778in}{0.000000in}}%
\pgfusepath{stroke,fill}%
}%
\begin{pgfscope}%
\pgfsys@transformshift{0.800000in}{1.334652in}%
\pgfsys@useobject{currentmarker}{}%
\end{pgfscope}%
\end{pgfscope}%
\begin{pgfscope}%
\pgfsetbuttcap%
\pgfsetroundjoin%
\definecolor{currentfill}{rgb}{0.000000,0.000000,0.000000}%
\pgfsetfillcolor{currentfill}%
\pgfsetlinewidth{0.602250pt}%
\definecolor{currentstroke}{rgb}{0.000000,0.000000,0.000000}%
\pgfsetstrokecolor{currentstroke}%
\pgfsetdash{}{0pt}%
\pgfsys@defobject{currentmarker}{\pgfqpoint{-0.027778in}{0.000000in}}{\pgfqpoint{-0.000000in}{0.000000in}}{%
\pgfpathmoveto{\pgfqpoint{-0.000000in}{0.000000in}}%
\pgfpathlineto{\pgfqpoint{-0.027778in}{0.000000in}}%
\pgfusepath{stroke,fill}%
}%
\begin{pgfscope}%
\pgfsys@transformshift{0.800000in}{1.400578in}%
\pgfsys@useobject{currentmarker}{}%
\end{pgfscope}%
\end{pgfscope}%
\begin{pgfscope}%
\pgfsetbuttcap%
\pgfsetroundjoin%
\definecolor{currentfill}{rgb}{0.000000,0.000000,0.000000}%
\pgfsetfillcolor{currentfill}%
\pgfsetlinewidth{0.602250pt}%
\definecolor{currentstroke}{rgb}{0.000000,0.000000,0.000000}%
\pgfsetstrokecolor{currentstroke}%
\pgfsetdash{}{0pt}%
\pgfsys@defobject{currentmarker}{\pgfqpoint{-0.027778in}{0.000000in}}{\pgfqpoint{-0.000000in}{0.000000in}}{%
\pgfpathmoveto{\pgfqpoint{-0.000000in}{0.000000in}}%
\pgfpathlineto{\pgfqpoint{-0.027778in}{0.000000in}}%
\pgfusepath{stroke,fill}%
}%
\begin{pgfscope}%
\pgfsys@transformshift{0.800000in}{1.458728in}%
\pgfsys@useobject{currentmarker}{}%
\end{pgfscope}%
\end{pgfscope}%
\begin{pgfscope}%
\pgfsetbuttcap%
\pgfsetroundjoin%
\definecolor{currentfill}{rgb}{0.000000,0.000000,0.000000}%
\pgfsetfillcolor{currentfill}%
\pgfsetlinewidth{0.602250pt}%
\definecolor{currentstroke}{rgb}{0.000000,0.000000,0.000000}%
\pgfsetstrokecolor{currentstroke}%
\pgfsetdash{}{0pt}%
\pgfsys@defobject{currentmarker}{\pgfqpoint{-0.027778in}{0.000000in}}{\pgfqpoint{-0.000000in}{0.000000in}}{%
\pgfpathmoveto{\pgfqpoint{-0.000000in}{0.000000in}}%
\pgfpathlineto{\pgfqpoint{-0.027778in}{0.000000in}}%
\pgfusepath{stroke,fill}%
}%
\begin{pgfscope}%
\pgfsys@transformshift{0.800000in}{1.852957in}%
\pgfsys@useobject{currentmarker}{}%
\end{pgfscope}%
\end{pgfscope}%
\begin{pgfscope}%
\pgfsetbuttcap%
\pgfsetroundjoin%
\definecolor{currentfill}{rgb}{0.000000,0.000000,0.000000}%
\pgfsetfillcolor{currentfill}%
\pgfsetlinewidth{0.602250pt}%
\definecolor{currentstroke}{rgb}{0.000000,0.000000,0.000000}%
\pgfsetstrokecolor{currentstroke}%
\pgfsetdash{}{0pt}%
\pgfsys@defobject{currentmarker}{\pgfqpoint{-0.027778in}{0.000000in}}{\pgfqpoint{-0.000000in}{0.000000in}}{%
\pgfpathmoveto{\pgfqpoint{-0.000000in}{0.000000in}}%
\pgfpathlineto{\pgfqpoint{-0.027778in}{0.000000in}}%
\pgfusepath{stroke,fill}%
}%
\begin{pgfscope}%
\pgfsys@transformshift{0.800000in}{2.053138in}%
\pgfsys@useobject{currentmarker}{}%
\end{pgfscope}%
\end{pgfscope}%
\begin{pgfscope}%
\pgfsetbuttcap%
\pgfsetroundjoin%
\definecolor{currentfill}{rgb}{0.000000,0.000000,0.000000}%
\pgfsetfillcolor{currentfill}%
\pgfsetlinewidth{0.602250pt}%
\definecolor{currentstroke}{rgb}{0.000000,0.000000,0.000000}%
\pgfsetstrokecolor{currentstroke}%
\pgfsetdash{}{0pt}%
\pgfsys@defobject{currentmarker}{\pgfqpoint{-0.027778in}{0.000000in}}{\pgfqpoint{-0.000000in}{0.000000in}}{%
\pgfpathmoveto{\pgfqpoint{-0.000000in}{0.000000in}}%
\pgfpathlineto{\pgfqpoint{-0.027778in}{0.000000in}}%
\pgfusepath{stroke,fill}%
}%
\begin{pgfscope}%
\pgfsys@transformshift{0.800000in}{2.195169in}%
\pgfsys@useobject{currentmarker}{}%
\end{pgfscope}%
\end{pgfscope}%
\begin{pgfscope}%
\pgfsetbuttcap%
\pgfsetroundjoin%
\definecolor{currentfill}{rgb}{0.000000,0.000000,0.000000}%
\pgfsetfillcolor{currentfill}%
\pgfsetlinewidth{0.602250pt}%
\definecolor{currentstroke}{rgb}{0.000000,0.000000,0.000000}%
\pgfsetstrokecolor{currentstroke}%
\pgfsetdash{}{0pt}%
\pgfsys@defobject{currentmarker}{\pgfqpoint{-0.027778in}{0.000000in}}{\pgfqpoint{-0.000000in}{0.000000in}}{%
\pgfpathmoveto{\pgfqpoint{-0.000000in}{0.000000in}}%
\pgfpathlineto{\pgfqpoint{-0.027778in}{0.000000in}}%
\pgfusepath{stroke,fill}%
}%
\begin{pgfscope}%
\pgfsys@transformshift{0.800000in}{2.305336in}%
\pgfsys@useobject{currentmarker}{}%
\end{pgfscope}%
\end{pgfscope}%
\begin{pgfscope}%
\pgfsetbuttcap%
\pgfsetroundjoin%
\definecolor{currentfill}{rgb}{0.000000,0.000000,0.000000}%
\pgfsetfillcolor{currentfill}%
\pgfsetlinewidth{0.602250pt}%
\definecolor{currentstroke}{rgb}{0.000000,0.000000,0.000000}%
\pgfsetstrokecolor{currentstroke}%
\pgfsetdash{}{0pt}%
\pgfsys@defobject{currentmarker}{\pgfqpoint{-0.027778in}{0.000000in}}{\pgfqpoint{-0.000000in}{0.000000in}}{%
\pgfpathmoveto{\pgfqpoint{-0.000000in}{0.000000in}}%
\pgfpathlineto{\pgfqpoint{-0.027778in}{0.000000in}}%
\pgfusepath{stroke,fill}%
}%
\begin{pgfscope}%
\pgfsys@transformshift{0.800000in}{2.395350in}%
\pgfsys@useobject{currentmarker}{}%
\end{pgfscope}%
\end{pgfscope}%
\begin{pgfscope}%
\pgfsetbuttcap%
\pgfsetroundjoin%
\definecolor{currentfill}{rgb}{0.000000,0.000000,0.000000}%
\pgfsetfillcolor{currentfill}%
\pgfsetlinewidth{0.602250pt}%
\definecolor{currentstroke}{rgb}{0.000000,0.000000,0.000000}%
\pgfsetstrokecolor{currentstroke}%
\pgfsetdash{}{0pt}%
\pgfsys@defobject{currentmarker}{\pgfqpoint{-0.027778in}{0.000000in}}{\pgfqpoint{-0.000000in}{0.000000in}}{%
\pgfpathmoveto{\pgfqpoint{-0.000000in}{0.000000in}}%
\pgfpathlineto{\pgfqpoint{-0.027778in}{0.000000in}}%
\pgfusepath{stroke,fill}%
}%
\begin{pgfscope}%
\pgfsys@transformshift{0.800000in}{2.471455in}%
\pgfsys@useobject{currentmarker}{}%
\end{pgfscope}%
\end{pgfscope}%
\begin{pgfscope}%
\pgfsetbuttcap%
\pgfsetroundjoin%
\definecolor{currentfill}{rgb}{0.000000,0.000000,0.000000}%
\pgfsetfillcolor{currentfill}%
\pgfsetlinewidth{0.602250pt}%
\definecolor{currentstroke}{rgb}{0.000000,0.000000,0.000000}%
\pgfsetstrokecolor{currentstroke}%
\pgfsetdash{}{0pt}%
\pgfsys@defobject{currentmarker}{\pgfqpoint{-0.027778in}{0.000000in}}{\pgfqpoint{-0.000000in}{0.000000in}}{%
\pgfpathmoveto{\pgfqpoint{-0.000000in}{0.000000in}}%
\pgfpathlineto{\pgfqpoint{-0.027778in}{0.000000in}}%
\pgfusepath{stroke,fill}%
}%
\begin{pgfscope}%
\pgfsys@transformshift{0.800000in}{2.537380in}%
\pgfsys@useobject{currentmarker}{}%
\end{pgfscope}%
\end{pgfscope}%
\begin{pgfscope}%
\pgfsetbuttcap%
\pgfsetroundjoin%
\definecolor{currentfill}{rgb}{0.000000,0.000000,0.000000}%
\pgfsetfillcolor{currentfill}%
\pgfsetlinewidth{0.602250pt}%
\definecolor{currentstroke}{rgb}{0.000000,0.000000,0.000000}%
\pgfsetstrokecolor{currentstroke}%
\pgfsetdash{}{0pt}%
\pgfsys@defobject{currentmarker}{\pgfqpoint{-0.027778in}{0.000000in}}{\pgfqpoint{-0.000000in}{0.000000in}}{%
\pgfpathmoveto{\pgfqpoint{-0.000000in}{0.000000in}}%
\pgfpathlineto{\pgfqpoint{-0.027778in}{0.000000in}}%
\pgfusepath{stroke,fill}%
}%
\begin{pgfscope}%
\pgfsys@transformshift{0.800000in}{2.595531in}%
\pgfsys@useobject{currentmarker}{}%
\end{pgfscope}%
\end{pgfscope}%
\begin{pgfscope}%
\pgfsetbuttcap%
\pgfsetroundjoin%
\definecolor{currentfill}{rgb}{0.000000,0.000000,0.000000}%
\pgfsetfillcolor{currentfill}%
\pgfsetlinewidth{0.602250pt}%
\definecolor{currentstroke}{rgb}{0.000000,0.000000,0.000000}%
\pgfsetstrokecolor{currentstroke}%
\pgfsetdash{}{0pt}%
\pgfsys@defobject{currentmarker}{\pgfqpoint{-0.027778in}{0.000000in}}{\pgfqpoint{-0.000000in}{0.000000in}}{%
\pgfpathmoveto{\pgfqpoint{-0.000000in}{0.000000in}}%
\pgfpathlineto{\pgfqpoint{-0.027778in}{0.000000in}}%
\pgfusepath{stroke,fill}%
}%
\begin{pgfscope}%
\pgfsys@transformshift{0.800000in}{2.989760in}%
\pgfsys@useobject{currentmarker}{}%
\end{pgfscope}%
\end{pgfscope}%
\begin{pgfscope}%
\pgfsetbuttcap%
\pgfsetroundjoin%
\definecolor{currentfill}{rgb}{0.000000,0.000000,0.000000}%
\pgfsetfillcolor{currentfill}%
\pgfsetlinewidth{0.602250pt}%
\definecolor{currentstroke}{rgb}{0.000000,0.000000,0.000000}%
\pgfsetstrokecolor{currentstroke}%
\pgfsetdash{}{0pt}%
\pgfsys@defobject{currentmarker}{\pgfqpoint{-0.027778in}{0.000000in}}{\pgfqpoint{-0.000000in}{0.000000in}}{%
\pgfpathmoveto{\pgfqpoint{-0.000000in}{0.000000in}}%
\pgfpathlineto{\pgfqpoint{-0.027778in}{0.000000in}}%
\pgfusepath{stroke,fill}%
}%
\begin{pgfscope}%
\pgfsys@transformshift{0.800000in}{3.189941in}%
\pgfsys@useobject{currentmarker}{}%
\end{pgfscope}%
\end{pgfscope}%
\begin{pgfscope}%
\pgfsetbuttcap%
\pgfsetroundjoin%
\definecolor{currentfill}{rgb}{0.000000,0.000000,0.000000}%
\pgfsetfillcolor{currentfill}%
\pgfsetlinewidth{0.602250pt}%
\definecolor{currentstroke}{rgb}{0.000000,0.000000,0.000000}%
\pgfsetstrokecolor{currentstroke}%
\pgfsetdash{}{0pt}%
\pgfsys@defobject{currentmarker}{\pgfqpoint{-0.027778in}{0.000000in}}{\pgfqpoint{-0.000000in}{0.000000in}}{%
\pgfpathmoveto{\pgfqpoint{-0.000000in}{0.000000in}}%
\pgfpathlineto{\pgfqpoint{-0.027778in}{0.000000in}}%
\pgfusepath{stroke,fill}%
}%
\begin{pgfscope}%
\pgfsys@transformshift{0.800000in}{3.331972in}%
\pgfsys@useobject{currentmarker}{}%
\end{pgfscope}%
\end{pgfscope}%
\begin{pgfscope}%
\pgfsetbuttcap%
\pgfsetroundjoin%
\definecolor{currentfill}{rgb}{0.000000,0.000000,0.000000}%
\pgfsetfillcolor{currentfill}%
\pgfsetlinewidth{0.602250pt}%
\definecolor{currentstroke}{rgb}{0.000000,0.000000,0.000000}%
\pgfsetstrokecolor{currentstroke}%
\pgfsetdash{}{0pt}%
\pgfsys@defobject{currentmarker}{\pgfqpoint{-0.027778in}{0.000000in}}{\pgfqpoint{-0.000000in}{0.000000in}}{%
\pgfpathmoveto{\pgfqpoint{-0.000000in}{0.000000in}}%
\pgfpathlineto{\pgfqpoint{-0.027778in}{0.000000in}}%
\pgfusepath{stroke,fill}%
}%
\begin{pgfscope}%
\pgfsys@transformshift{0.800000in}{3.442139in}%
\pgfsys@useobject{currentmarker}{}%
\end{pgfscope}%
\end{pgfscope}%
\begin{pgfscope}%
\pgfsetbuttcap%
\pgfsetroundjoin%
\definecolor{currentfill}{rgb}{0.000000,0.000000,0.000000}%
\pgfsetfillcolor{currentfill}%
\pgfsetlinewidth{0.602250pt}%
\definecolor{currentstroke}{rgb}{0.000000,0.000000,0.000000}%
\pgfsetstrokecolor{currentstroke}%
\pgfsetdash{}{0pt}%
\pgfsys@defobject{currentmarker}{\pgfqpoint{-0.027778in}{0.000000in}}{\pgfqpoint{-0.000000in}{0.000000in}}{%
\pgfpathmoveto{\pgfqpoint{-0.000000in}{0.000000in}}%
\pgfpathlineto{\pgfqpoint{-0.027778in}{0.000000in}}%
\pgfusepath{stroke,fill}%
}%
\begin{pgfscope}%
\pgfsys@transformshift{0.800000in}{3.532153in}%
\pgfsys@useobject{currentmarker}{}%
\end{pgfscope}%
\end{pgfscope}%
\begin{pgfscope}%
\pgfsetbuttcap%
\pgfsetroundjoin%
\definecolor{currentfill}{rgb}{0.000000,0.000000,0.000000}%
\pgfsetfillcolor{currentfill}%
\pgfsetlinewidth{0.602250pt}%
\definecolor{currentstroke}{rgb}{0.000000,0.000000,0.000000}%
\pgfsetstrokecolor{currentstroke}%
\pgfsetdash{}{0pt}%
\pgfsys@defobject{currentmarker}{\pgfqpoint{-0.027778in}{0.000000in}}{\pgfqpoint{-0.000000in}{0.000000in}}{%
\pgfpathmoveto{\pgfqpoint{-0.000000in}{0.000000in}}%
\pgfpathlineto{\pgfqpoint{-0.027778in}{0.000000in}}%
\pgfusepath{stroke,fill}%
}%
\begin{pgfscope}%
\pgfsys@transformshift{0.800000in}{3.608258in}%
\pgfsys@useobject{currentmarker}{}%
\end{pgfscope}%
\end{pgfscope}%
\begin{pgfscope}%
\pgfsetbuttcap%
\pgfsetroundjoin%
\definecolor{currentfill}{rgb}{0.000000,0.000000,0.000000}%
\pgfsetfillcolor{currentfill}%
\pgfsetlinewidth{0.602250pt}%
\definecolor{currentstroke}{rgb}{0.000000,0.000000,0.000000}%
\pgfsetstrokecolor{currentstroke}%
\pgfsetdash{}{0pt}%
\pgfsys@defobject{currentmarker}{\pgfqpoint{-0.027778in}{0.000000in}}{\pgfqpoint{-0.000000in}{0.000000in}}{%
\pgfpathmoveto{\pgfqpoint{-0.000000in}{0.000000in}}%
\pgfpathlineto{\pgfqpoint{-0.027778in}{0.000000in}}%
\pgfusepath{stroke,fill}%
}%
\begin{pgfscope}%
\pgfsys@transformshift{0.800000in}{3.674183in}%
\pgfsys@useobject{currentmarker}{}%
\end{pgfscope}%
\end{pgfscope}%
\begin{pgfscope}%
\pgfsetbuttcap%
\pgfsetroundjoin%
\definecolor{currentfill}{rgb}{0.000000,0.000000,0.000000}%
\pgfsetfillcolor{currentfill}%
\pgfsetlinewidth{0.602250pt}%
\definecolor{currentstroke}{rgb}{0.000000,0.000000,0.000000}%
\pgfsetstrokecolor{currentstroke}%
\pgfsetdash{}{0pt}%
\pgfsys@defobject{currentmarker}{\pgfqpoint{-0.027778in}{0.000000in}}{\pgfqpoint{-0.000000in}{0.000000in}}{%
\pgfpathmoveto{\pgfqpoint{-0.000000in}{0.000000in}}%
\pgfpathlineto{\pgfqpoint{-0.027778in}{0.000000in}}%
\pgfusepath{stroke,fill}%
}%
\begin{pgfscope}%
\pgfsys@transformshift{0.800000in}{3.732334in}%
\pgfsys@useobject{currentmarker}{}%
\end{pgfscope}%
\end{pgfscope}%
\begin{pgfscope}%
\pgfsetbuttcap%
\pgfsetroundjoin%
\definecolor{currentfill}{rgb}{0.000000,0.000000,0.000000}%
\pgfsetfillcolor{currentfill}%
\pgfsetlinewidth{0.602250pt}%
\definecolor{currentstroke}{rgb}{0.000000,0.000000,0.000000}%
\pgfsetstrokecolor{currentstroke}%
\pgfsetdash{}{0pt}%
\pgfsys@defobject{currentmarker}{\pgfqpoint{-0.027778in}{0.000000in}}{\pgfqpoint{-0.000000in}{0.000000in}}{%
\pgfpathmoveto{\pgfqpoint{-0.000000in}{0.000000in}}%
\pgfpathlineto{\pgfqpoint{-0.027778in}{0.000000in}}%
\pgfusepath{stroke,fill}%
}%
\begin{pgfscope}%
\pgfsys@transformshift{0.800000in}{4.126563in}%
\pgfsys@useobject{currentmarker}{}%
\end{pgfscope}%
\end{pgfscope}%
\begin{pgfscope}%
\definecolor{textcolor}{rgb}{0.000000,0.000000,0.000000}%
\pgfsetstrokecolor{textcolor}%
\pgfsetfillcolor{textcolor}%
\pgftext[x=0.446026in,y=2.376000in,,bottom,rotate=90.000000]{\color{textcolor}\rmfamily\fontsize{10.000000}{12.000000}\selectfont MEMORY (B) in log}%
\end{pgfscope}%
\begin{pgfscope}%
\pgfpathrectangle{\pgfqpoint{0.800000in}{0.528000in}}{\pgfqpoint{4.960000in}{3.696000in}}%
\pgfusepath{clip}%
\pgfsetrectcap%
\pgfsetroundjoin%
\pgfsetlinewidth{1.505625pt}%
\definecolor{currentstroke}{rgb}{0.121569,0.466667,0.705882}%
\pgfsetstrokecolor{currentstroke}%
\pgfsetdash{}{0pt}%
\pgfpathmoveto{\pgfqpoint{1.025455in}{2.393701in}}%
\pgfpathlineto{\pgfqpoint{1.071466in}{1.907142in}}%
\pgfpathlineto{\pgfqpoint{1.117477in}{1.716165in}}%
\pgfpathlineto{\pgfqpoint{1.163488in}{2.296570in}}%
\pgfpathlineto{\pgfqpoint{1.209499in}{1.744022in}}%
\pgfpathlineto{\pgfqpoint{1.255510in}{1.808560in}}%
\pgfpathlineto{\pgfqpoint{1.301521in}{1.800940in}}%
\pgfpathlineto{\pgfqpoint{1.347532in}{1.927932in}}%
\pgfpathlineto{\pgfqpoint{1.393544in}{1.851969in}}%
\pgfpathlineto{\pgfqpoint{1.439555in}{2.043164in}}%
\pgfpathlineto{\pgfqpoint{1.485566in}{1.872320in}}%
\pgfpathlineto{\pgfqpoint{1.531577in}{2.211676in}}%
\pgfpathlineto{\pgfqpoint{1.577588in}{1.844994in}}%
\pgfpathlineto{\pgfqpoint{1.623599in}{1.837919in}}%
\pgfpathlineto{\pgfqpoint{1.669610in}{1.844994in}}%
\pgfpathlineto{\pgfqpoint{1.715622in}{1.858846in}}%
\pgfpathlineto{\pgfqpoint{1.761633in}{1.844994in}}%
\pgfpathlineto{\pgfqpoint{1.807644in}{1.891867in}}%
\pgfpathlineto{\pgfqpoint{1.853655in}{1.968623in}}%
\pgfpathlineto{\pgfqpoint{1.899666in}{1.928780in}}%
\pgfpathlineto{\pgfqpoint{1.945677in}{1.898214in}}%
\pgfpathlineto{\pgfqpoint{1.991688in}{1.885437in}}%
\pgfpathlineto{\pgfqpoint{2.037699in}{2.208802in}}%
\pgfpathlineto{\pgfqpoint{2.083711in}{1.904480in}}%
\pgfpathlineto{\pgfqpoint{2.129722in}{1.885437in}}%
\pgfpathlineto{\pgfqpoint{2.175733in}{1.968623in}}%
\pgfpathlineto{\pgfqpoint{2.221744in}{1.904480in}}%
\pgfpathlineto{\pgfqpoint{2.267755in}{1.963125in}}%
\pgfpathlineto{\pgfqpoint{2.313766in}{1.995234in}}%
\pgfpathlineto{\pgfqpoint{2.359777in}{1.934673in}}%
\pgfpathlineto{\pgfqpoint{2.405788in}{1.904480in}}%
\pgfpathlineto{\pgfqpoint{2.451800in}{1.928780in}}%
\pgfpathlineto{\pgfqpoint{2.497811in}{2.015537in}}%
\pgfpathlineto{\pgfqpoint{2.543822in}{1.916780in}}%
\pgfpathlineto{\pgfqpoint{2.589833in}{2.010539in}}%
\pgfpathlineto{\pgfqpoint{2.635844in}{1.934673in}}%
\pgfpathlineto{\pgfqpoint{2.681855in}{1.957564in}}%
\pgfpathlineto{\pgfqpoint{2.727866in}{1.990026in}}%
\pgfpathlineto{\pgfqpoint{2.773878in}{1.995234in}}%
\pgfpathlineto{\pgfqpoint{2.819889in}{1.951940in}}%
\pgfpathlineto{\pgfqpoint{2.865900in}{1.940496in}}%
\pgfpathlineto{\pgfqpoint{2.911911in}{1.990026in}}%
\pgfpathlineto{\pgfqpoint{2.957922in}{1.940496in}}%
\pgfpathlineto{\pgfqpoint{3.003933in}{1.957564in}}%
\pgfpathlineto{\pgfqpoint{3.049944in}{1.984761in}}%
\pgfpathlineto{\pgfqpoint{3.095955in}{1.974062in}}%
\pgfpathlineto{\pgfqpoint{3.141967in}{1.928780in}}%
\pgfpathlineto{\pgfqpoint{3.187978in}{2.005490in}}%
\pgfpathlineto{\pgfqpoint{3.233989in}{2.010539in}}%
\pgfpathlineto{\pgfqpoint{3.280000in}{1.979440in}}%
\pgfpathlineto{\pgfqpoint{3.326011in}{1.957564in}}%
\pgfpathlineto{\pgfqpoint{3.372022in}{2.010539in}}%
\pgfpathlineto{\pgfqpoint{3.418033in}{1.951940in}}%
\pgfpathlineto{\pgfqpoint{3.464045in}{1.951940in}}%
\pgfpathlineto{\pgfqpoint{3.510056in}{2.105550in}}%
\pgfpathlineto{\pgfqpoint{3.556067in}{1.934673in}}%
\pgfpathlineto{\pgfqpoint{3.602078in}{1.957564in}}%
\pgfpathlineto{\pgfqpoint{3.648089in}{2.025383in}}%
\pgfpathlineto{\pgfqpoint{3.694100in}{2.300175in}}%
\pgfpathlineto{\pgfqpoint{3.740111in}{1.984761in}}%
\pgfpathlineto{\pgfqpoint{3.786122in}{1.984761in}}%
\pgfpathlineto{\pgfqpoint{3.832134in}{1.957564in}}%
\pgfpathlineto{\pgfqpoint{3.878145in}{2.005490in}}%
\pgfpathlineto{\pgfqpoint{3.924156in}{1.968623in}}%
\pgfpathlineto{\pgfqpoint{3.970167in}{2.030233in}}%
\pgfpathlineto{\pgfqpoint{4.016178in}{1.963125in}}%
\pgfpathlineto{\pgfqpoint{4.062189in}{2.071868in}}%
\pgfpathlineto{\pgfqpoint{4.108200in}{1.979440in}}%
\pgfpathlineto{\pgfqpoint{4.154212in}{1.951940in}}%
\pgfpathlineto{\pgfqpoint{4.200223in}{1.990026in}}%
\pgfpathlineto{\pgfqpoint{4.246234in}{2.005490in}}%
\pgfpathlineto{\pgfqpoint{4.292245in}{2.149149in}}%
\pgfpathlineto{\pgfqpoint{4.338256in}{2.030233in}}%
\pgfpathlineto{\pgfqpoint{4.384267in}{2.020484in}}%
\pgfpathlineto{\pgfqpoint{4.430278in}{1.995234in}}%
\pgfpathlineto{\pgfqpoint{4.476289in}{2.000389in}}%
\pgfpathlineto{\pgfqpoint{4.522301in}{2.005490in}}%
\pgfpathlineto{\pgfqpoint{4.568312in}{2.015537in}}%
\pgfpathlineto{\pgfqpoint{4.614323in}{2.044505in}}%
\pgfpathlineto{\pgfqpoint{4.660334in}{2.010539in}}%
\pgfpathlineto{\pgfqpoint{4.706345in}{2.005490in}}%
\pgfpathlineto{\pgfqpoint{4.752356in}{2.039794in}}%
\pgfpathlineto{\pgfqpoint{4.798367in}{2.062915in}}%
\pgfpathlineto{\pgfqpoint{4.844378in}{2.025383in}}%
\pgfpathlineto{\pgfqpoint{4.890390in}{1.968623in}}%
\pgfpathlineto{\pgfqpoint{4.936401in}{2.044505in}}%
\pgfpathlineto{\pgfqpoint{4.982412in}{2.080662in}}%
\pgfpathlineto{\pgfqpoint{5.028423in}{1.990026in}}%
\pgfpathlineto{\pgfqpoint{5.074434in}{1.984761in}}%
\pgfpathlineto{\pgfqpoint{5.120445in}{2.005490in}}%
\pgfpathlineto{\pgfqpoint{5.166456in}{2.000389in}}%
\pgfpathlineto{\pgfqpoint{5.212468in}{2.071868in}}%
\pgfpathlineto{\pgfqpoint{5.258479in}{2.076285in}}%
\pgfpathlineto{\pgfqpoint{5.304490in}{2.005490in}}%
\pgfpathlineto{\pgfqpoint{5.350501in}{2.058376in}}%
\pgfpathlineto{\pgfqpoint{5.396512in}{2.000389in}}%
\pgfpathlineto{\pgfqpoint{5.442523in}{2.015537in}}%
\pgfpathlineto{\pgfqpoint{5.488534in}{2.039794in}}%
\pgfpathlineto{\pgfqpoint{5.534545in}{2.053796in}}%
\pgfusepath{stroke}%
\end{pgfscope}%
\begin{pgfscope}%
\pgfpathrectangle{\pgfqpoint{0.800000in}{0.528000in}}{\pgfqpoint{4.960000in}{3.696000in}}%
\pgfusepath{clip}%
\pgfsetrectcap%
\pgfsetroundjoin%
\pgfsetlinewidth{1.505625pt}%
\definecolor{currentstroke}{rgb}{1.000000,0.498039,0.054902}%
\pgfsetstrokecolor{currentstroke}%
\pgfsetdash{}{0pt}%
\pgfpathmoveto{\pgfqpoint{1.025455in}{2.369332in}}%
\pgfpathlineto{\pgfqpoint{1.071466in}{2.235900in}}%
\pgfpathlineto{\pgfqpoint{1.117477in}{2.397321in}}%
\pgfpathlineto{\pgfqpoint{1.163488in}{2.438801in}}%
\pgfpathlineto{\pgfqpoint{1.209499in}{2.614419in}}%
\pgfpathlineto{\pgfqpoint{1.255510in}{2.664055in}}%
\pgfpathlineto{\pgfqpoint{1.301521in}{2.711371in}}%
\pgfpathlineto{\pgfqpoint{1.347532in}{2.776319in}}%
\pgfpathlineto{\pgfqpoint{1.393544in}{2.834250in}}%
\pgfpathlineto{\pgfqpoint{1.439555in}{2.884871in}}%
\pgfpathlineto{\pgfqpoint{1.485566in}{2.948164in}}%
\pgfpathlineto{\pgfqpoint{1.531577in}{2.975129in}}%
\pgfpathlineto{\pgfqpoint{1.577588in}{3.014600in}}%
\pgfpathlineto{\pgfqpoint{1.623599in}{3.050274in}}%
\pgfpathlineto{\pgfqpoint{1.669610in}{3.084034in}}%
\pgfpathlineto{\pgfqpoint{1.715622in}{3.115632in}}%
\pgfpathlineto{\pgfqpoint{1.761633in}{3.145616in}}%
\pgfpathlineto{\pgfqpoint{1.807644in}{3.173612in}}%
\pgfpathlineto{\pgfqpoint{1.853655in}{3.199847in}}%
\pgfpathlineto{\pgfqpoint{1.899666in}{3.224880in}}%
\pgfpathlineto{\pgfqpoint{1.945677in}{3.257796in}}%
\pgfpathlineto{\pgfqpoint{1.991688in}{3.272493in}}%
\pgfpathlineto{\pgfqpoint{2.037699in}{3.294068in}}%
\pgfpathlineto{\pgfqpoint{2.083711in}{3.314945in}}%
\pgfpathlineto{\pgfqpoint{2.129722in}{3.335170in}}%
\pgfpathlineto{\pgfqpoint{2.175733in}{3.354411in}}%
\pgfpathlineto{\pgfqpoint{2.221744in}{3.372930in}}%
\pgfpathlineto{\pgfqpoint{2.267755in}{3.390517in}}%
\pgfpathlineto{\pgfqpoint{2.313766in}{3.407922in}}%
\pgfpathlineto{\pgfqpoint{2.359777in}{3.424406in}}%
\pgfpathlineto{\pgfqpoint{2.405788in}{3.440755in}}%
\pgfpathlineto{\pgfqpoint{2.451800in}{3.456119in}}%
\pgfpathlineto{\pgfqpoint{2.497811in}{3.471391in}}%
\pgfpathlineto{\pgfqpoint{2.543822in}{3.486061in}}%
\pgfpathlineto{\pgfqpoint{2.589833in}{3.500237in}}%
\pgfpathlineto{\pgfqpoint{2.635844in}{3.514154in}}%
\pgfpathlineto{\pgfqpoint{2.681855in}{3.527556in}}%
\pgfpathlineto{\pgfqpoint{2.727866in}{3.540604in}}%
\pgfpathlineto{\pgfqpoint{2.773878in}{3.553380in}}%
\pgfpathlineto{\pgfqpoint{2.819889in}{3.565771in}}%
\pgfpathlineto{\pgfqpoint{2.865900in}{3.582518in}}%
\pgfpathlineto{\pgfqpoint{2.911911in}{3.590274in}}%
\pgfpathlineto{\pgfqpoint{2.957922in}{3.601840in}}%
\pgfpathlineto{\pgfqpoint{3.003933in}{3.613087in}}%
\pgfpathlineto{\pgfqpoint{3.049944in}{3.624192in}}%
\pgfpathlineto{\pgfqpoint{3.095955in}{3.634838in}}%
\pgfpathlineto{\pgfqpoint{3.141967in}{3.645522in}}%
\pgfpathlineto{\pgfqpoint{3.187978in}{3.655775in}}%
\pgfpathlineto{\pgfqpoint{3.233989in}{3.666070in}}%
\pgfpathlineto{\pgfqpoint{3.280000in}{3.675957in}}%
\pgfpathlineto{\pgfqpoint{3.326011in}{3.685603in}}%
\pgfpathlineto{\pgfqpoint{3.372022in}{3.695253in}}%
\pgfpathlineto{\pgfqpoint{3.418033in}{3.704625in}}%
\pgfpathlineto{\pgfqpoint{3.464045in}{3.713822in}}%
\pgfpathlineto{\pgfqpoint{3.510056in}{3.722762in}}%
\pgfpathlineto{\pgfqpoint{3.556067in}{3.731587in}}%
\pgfpathlineto{\pgfqpoint{3.602078in}{3.740473in}}%
\pgfpathlineto{\pgfqpoint{3.648089in}{3.748947in}}%
\pgfpathlineto{\pgfqpoint{3.694100in}{3.757403in}}%
\pgfpathlineto{\pgfqpoint{3.740111in}{3.765553in}}%
\pgfpathlineto{\pgfqpoint{3.786122in}{3.773731in}}%
\pgfpathlineto{\pgfqpoint{3.832134in}{3.781817in}}%
\pgfpathlineto{\pgfqpoint{3.878145in}{3.789693in}}%
\pgfpathlineto{\pgfqpoint{3.924156in}{3.797331in}}%
\pgfpathlineto{\pgfqpoint{3.970167in}{3.805080in}}%
\pgfpathlineto{\pgfqpoint{4.016178in}{3.812559in}}%
\pgfpathlineto{\pgfqpoint{4.062189in}{3.820001in}}%
\pgfpathlineto{\pgfqpoint{4.108200in}{3.827260in}}%
\pgfpathlineto{\pgfqpoint{4.154212in}{3.834413in}}%
\pgfpathlineto{\pgfqpoint{4.200223in}{3.841500in}}%
\pgfpathlineto{\pgfqpoint{4.246234in}{3.848451in}}%
\pgfpathlineto{\pgfqpoint{4.292245in}{3.855409in}}%
\pgfpathlineto{\pgfqpoint{4.338256in}{3.862270in}}%
\pgfpathlineto{\pgfqpoint{4.384267in}{3.868870in}}%
\pgfpathlineto{\pgfqpoint{4.430278in}{3.875515in}}%
\pgfpathlineto{\pgfqpoint{4.476289in}{3.881974in}}%
\pgfpathlineto{\pgfqpoint{4.522301in}{3.888510in}}%
\pgfpathlineto{\pgfqpoint{4.568312in}{3.894771in}}%
\pgfpathlineto{\pgfqpoint{4.614323in}{3.901047in}}%
\pgfpathlineto{\pgfqpoint{4.660334in}{3.907214in}}%
\pgfpathlineto{\pgfqpoint{4.706345in}{3.913396in}}%
\pgfpathlineto{\pgfqpoint{4.752356in}{3.921689in}}%
\pgfpathlineto{\pgfqpoint{4.798367in}{3.925635in}}%
\pgfpathlineto{\pgfqpoint{4.844378in}{3.931504in}}%
\pgfpathlineto{\pgfqpoint{4.890390in}{3.937362in}}%
\pgfpathlineto{\pgfqpoint{4.936401in}{3.943094in}}%
\pgfpathlineto{\pgfqpoint{4.982412in}{3.948732in}}%
\pgfpathlineto{\pgfqpoint{5.028423in}{3.954390in}}%
\pgfpathlineto{\pgfqpoint{5.074434in}{3.960011in}}%
\pgfpathlineto{\pgfqpoint{5.120445in}{3.965487in}}%
\pgfpathlineto{\pgfqpoint{5.166456in}{3.970931in}}%
\pgfpathlineto{\pgfqpoint{5.212468in}{3.976234in}}%
\pgfpathlineto{\pgfqpoint{5.258479in}{3.981667in}}%
\pgfpathlineto{\pgfqpoint{5.304490in}{3.986883in}}%
\pgfpathlineto{\pgfqpoint{5.350501in}{3.992122in}}%
\pgfpathlineto{\pgfqpoint{5.396512in}{3.997204in}}%
\pgfpathlineto{\pgfqpoint{5.442523in}{4.002361in}}%
\pgfpathlineto{\pgfqpoint{5.488534in}{4.007440in}}%
\pgfpathlineto{\pgfqpoint{5.534545in}{4.012417in}}%
\pgfusepath{stroke}%
\end{pgfscope}%
\begin{pgfscope}%
\pgfpathrectangle{\pgfqpoint{0.800000in}{0.528000in}}{\pgfqpoint{4.960000in}{3.696000in}}%
\pgfusepath{clip}%
\pgfsetrectcap%
\pgfsetroundjoin%
\pgfsetlinewidth{1.505625pt}%
\definecolor{currentstroke}{rgb}{0.172549,0.627451,0.172549}%
\pgfsetstrokecolor{currentstroke}%
\pgfsetdash{}{0pt}%
\pgfpathmoveto{\pgfqpoint{1.025455in}{2.145341in}}%
\pgfpathlineto{\pgfqpoint{1.071466in}{2.106733in}}%
\pgfpathlineto{\pgfqpoint{1.117477in}{2.302564in}}%
\pgfpathlineto{\pgfqpoint{1.163488in}{2.440306in}}%
\pgfpathlineto{\pgfqpoint{1.209499in}{2.547883in}}%
\pgfpathlineto{\pgfqpoint{1.255510in}{2.637574in}}%
\pgfpathlineto{\pgfqpoint{1.301521in}{2.712238in}}%
\pgfpathlineto{\pgfqpoint{1.347532in}{2.777079in}}%
\pgfpathlineto{\pgfqpoint{1.393544in}{2.834385in}}%
\pgfpathlineto{\pgfqpoint{1.439555in}{2.885725in}}%
\pgfpathlineto{\pgfqpoint{1.485566in}{2.932226in}}%
\pgfpathlineto{\pgfqpoint{1.531577in}{2.974722in}}%
\pgfpathlineto{\pgfqpoint{1.577588in}{3.013848in}}%
\pgfpathlineto{\pgfqpoint{1.623599in}{3.050099in}}%
\pgfpathlineto{\pgfqpoint{1.669610in}{3.083870in}}%
\pgfpathlineto{\pgfqpoint{1.715622in}{3.115478in}}%
\pgfpathlineto{\pgfqpoint{1.761633in}{3.145184in}}%
\pgfpathlineto{\pgfqpoint{1.807644in}{3.173203in}}%
\pgfpathlineto{\pgfqpoint{1.853655in}{3.199717in}}%
\pgfpathlineto{\pgfqpoint{1.899666in}{3.224880in}}%
\pgfpathlineto{\pgfqpoint{1.945677in}{3.248822in}}%
\pgfpathlineto{\pgfqpoint{1.991688in}{3.271656in}}%
\pgfpathlineto{\pgfqpoint{2.037699in}{3.293481in}}%
\pgfpathlineto{\pgfqpoint{2.083711in}{3.314382in}}%
\pgfpathlineto{\pgfqpoint{2.129722in}{3.334434in}}%
\pgfpathlineto{\pgfqpoint{2.175733in}{3.353703in}}%
\pgfpathlineto{\pgfqpoint{2.221744in}{3.372248in}}%
\pgfpathlineto{\pgfqpoint{2.267755in}{3.390122in}}%
\pgfpathlineto{\pgfqpoint{2.313766in}{3.407371in}}%
\pgfpathlineto{\pgfqpoint{2.359777in}{3.424038in}}%
\pgfpathlineto{\pgfqpoint{2.405788in}{3.440160in}}%
\pgfpathlineto{\pgfqpoint{2.451800in}{3.455773in}}%
\pgfpathlineto{\pgfqpoint{2.497811in}{3.470907in}}%
\pgfpathlineto{\pgfqpoint{2.543822in}{3.485591in}}%
\pgfpathlineto{\pgfqpoint{2.589833in}{3.499850in}}%
\pgfpathlineto{\pgfqpoint{2.635844in}{3.513710in}}%
\pgfpathlineto{\pgfqpoint{2.681855in}{3.527191in}}%
\pgfpathlineto{\pgfqpoint{2.727866in}{3.540313in}}%
\pgfpathlineto{\pgfqpoint{2.773878in}{3.553096in}}%
\pgfpathlineto{\pgfqpoint{2.819889in}{3.565556in}}%
\pgfpathlineto{\pgfqpoint{2.865900in}{3.577710in}}%
\pgfpathlineto{\pgfqpoint{2.911911in}{3.589571in}}%
\pgfpathlineto{\pgfqpoint{2.957922in}{3.601154in}}%
\pgfpathlineto{\pgfqpoint{3.003933in}{3.612472in}}%
\pgfpathlineto{\pgfqpoint{3.049944in}{3.623536in}}%
\pgfpathlineto{\pgfqpoint{3.095955in}{3.634357in}}%
\pgfpathlineto{\pgfqpoint{3.141967in}{3.644946in}}%
\pgfpathlineto{\pgfqpoint{3.187978in}{3.655313in}}%
\pgfpathlineto{\pgfqpoint{3.233989in}{3.665467in}}%
\pgfpathlineto{\pgfqpoint{3.280000in}{3.675416in}}%
\pgfpathlineto{\pgfqpoint{3.326011in}{3.685169in}}%
\pgfpathlineto{\pgfqpoint{3.372022in}{3.694732in}}%
\pgfpathlineto{\pgfqpoint{3.418033in}{3.704114in}}%
\pgfpathlineto{\pgfqpoint{3.464045in}{3.713321in}}%
\pgfpathlineto{\pgfqpoint{3.510056in}{3.722359in}}%
\pgfpathlineto{\pgfqpoint{3.556067in}{3.731235in}}%
\pgfpathlineto{\pgfqpoint{3.602078in}{3.739954in}}%
\pgfpathlineto{\pgfqpoint{3.648089in}{3.748522in}}%
\pgfpathlineto{\pgfqpoint{3.694100in}{3.756944in}}%
\pgfpathlineto{\pgfqpoint{3.740111in}{3.765224in}}%
\pgfpathlineto{\pgfqpoint{3.786122in}{3.773368in}}%
\pgfpathlineto{\pgfqpoint{3.832134in}{3.781380in}}%
\pgfpathlineto{\pgfqpoint{3.878145in}{3.789263in}}%
\pgfpathlineto{\pgfqpoint{3.924156in}{3.797023in}}%
\pgfpathlineto{\pgfqpoint{3.970167in}{3.804663in}}%
\pgfpathlineto{\pgfqpoint{4.016178in}{3.812186in}}%
\pgfpathlineto{\pgfqpoint{4.062189in}{3.819597in}}%
\pgfpathlineto{\pgfqpoint{4.108200in}{3.826897in}}%
\pgfpathlineto{\pgfqpoint{4.154212in}{3.834092in}}%
\pgfpathlineto{\pgfqpoint{4.200223in}{3.841183in}}%
\pgfpathlineto{\pgfqpoint{4.246234in}{3.848174in}}%
\pgfpathlineto{\pgfqpoint{4.292245in}{3.855067in}}%
\pgfpathlineto{\pgfqpoint{4.338256in}{3.861865in}}%
\pgfpathlineto{\pgfqpoint{4.384267in}{3.868571in}}%
\pgfpathlineto{\pgfqpoint{4.430278in}{3.875186in}}%
\pgfpathlineto{\pgfqpoint{4.476289in}{3.881715in}}%
\pgfpathlineto{\pgfqpoint{4.522301in}{3.888158in}}%
\pgfpathlineto{\pgfqpoint{4.568312in}{3.894518in}}%
\pgfpathlineto{\pgfqpoint{4.614323in}{3.900798in}}%
\pgfpathlineto{\pgfqpoint{4.660334in}{3.906998in}}%
\pgfpathlineto{\pgfqpoint{4.706345in}{3.913122in}}%
\pgfpathlineto{\pgfqpoint{4.752356in}{3.919170in}}%
\pgfpathlineto{\pgfqpoint{4.798367in}{3.925146in}}%
\pgfpathlineto{\pgfqpoint{4.844378in}{3.931050in}}%
\pgfpathlineto{\pgfqpoint{4.890390in}{3.936884in}}%
\pgfpathlineto{\pgfqpoint{4.936401in}{3.942650in}}%
\pgfpathlineto{\pgfqpoint{4.982412in}{3.948349in}}%
\pgfpathlineto{\pgfqpoint{5.028423in}{3.953984in}}%
\pgfpathlineto{\pgfqpoint{5.074434in}{3.959554in}}%
\pgfpathlineto{\pgfqpoint{5.120445in}{3.965063in}}%
\pgfpathlineto{\pgfqpoint{5.166456in}{3.970511in}}%
\pgfpathlineto{\pgfqpoint{5.212468in}{3.975899in}}%
\pgfpathlineto{\pgfqpoint{5.258479in}{3.981229in}}%
\pgfpathlineto{\pgfqpoint{5.304490in}{3.986503in}}%
\pgfpathlineto{\pgfqpoint{5.350501in}{3.991720in}}%
\pgfpathlineto{\pgfqpoint{5.396512in}{3.996883in}}%
\pgfpathlineto{\pgfqpoint{5.442523in}{4.001993in}}%
\pgfpathlineto{\pgfqpoint{5.488534in}{4.007050in}}%
\pgfpathlineto{\pgfqpoint{5.534545in}{4.012056in}}%
\pgfusepath{stroke}%
\end{pgfscope}%
\begin{pgfscope}%
\pgfpathrectangle{\pgfqpoint{0.800000in}{0.528000in}}{\pgfqpoint{4.960000in}{3.696000in}}%
\pgfusepath{clip}%
\pgfsetrectcap%
\pgfsetroundjoin%
\pgfsetlinewidth{1.505625pt}%
\definecolor{currentstroke}{rgb}{0.839216,0.152941,0.156863}%
\pgfsetstrokecolor{currentstroke}%
\pgfsetdash{}{0pt}%
\pgfpathmoveto{\pgfqpoint{1.025455in}{1.284200in}}%
\pgfpathlineto{\pgfqpoint{1.071466in}{0.696000in}}%
\pgfpathlineto{\pgfqpoint{1.117477in}{0.822356in}}%
\pgfpathlineto{\pgfqpoint{1.163488in}{0.822356in}}%
\pgfpathlineto{\pgfqpoint{1.209499in}{0.822356in}}%
\pgfpathlineto{\pgfqpoint{1.255510in}{0.822356in}}%
\pgfpathlineto{\pgfqpoint{1.301521in}{0.822356in}}%
\pgfpathlineto{\pgfqpoint{1.347532in}{0.822356in}}%
\pgfpathlineto{\pgfqpoint{1.393544in}{0.822356in}}%
\pgfpathlineto{\pgfqpoint{1.439555in}{0.822356in}}%
\pgfpathlineto{\pgfqpoint{1.485566in}{0.822356in}}%
\pgfpathlineto{\pgfqpoint{1.531577in}{0.822356in}}%
\pgfpathlineto{\pgfqpoint{1.577588in}{0.822356in}}%
\pgfpathlineto{\pgfqpoint{1.623599in}{0.822356in}}%
\pgfpathlineto{\pgfqpoint{1.669610in}{0.822356in}}%
\pgfpathlineto{\pgfqpoint{1.715622in}{0.822356in}}%
\pgfpathlineto{\pgfqpoint{1.761633in}{0.822356in}}%
\pgfpathlineto{\pgfqpoint{1.807644in}{0.822356in}}%
\pgfpathlineto{\pgfqpoint{1.853655in}{0.822356in}}%
\pgfpathlineto{\pgfqpoint{1.899666in}{0.822356in}}%
\pgfpathlineto{\pgfqpoint{1.945677in}{0.822356in}}%
\pgfpathlineto{\pgfqpoint{1.991688in}{0.822356in}}%
\pgfpathlineto{\pgfqpoint{2.037699in}{0.822356in}}%
\pgfpathlineto{\pgfqpoint{2.083711in}{0.822356in}}%
\pgfpathlineto{\pgfqpoint{2.129722in}{0.822356in}}%
\pgfpathlineto{\pgfqpoint{2.175733in}{0.822356in}}%
\pgfpathlineto{\pgfqpoint{2.221744in}{0.822356in}}%
\pgfpathlineto{\pgfqpoint{2.267755in}{0.822356in}}%
\pgfpathlineto{\pgfqpoint{2.313766in}{0.822356in}}%
\pgfpathlineto{\pgfqpoint{2.359777in}{0.822356in}}%
\pgfpathlineto{\pgfqpoint{2.405788in}{0.822356in}}%
\pgfpathlineto{\pgfqpoint{2.451800in}{0.822356in}}%
\pgfpathlineto{\pgfqpoint{2.497811in}{0.822356in}}%
\pgfpathlineto{\pgfqpoint{2.543822in}{0.822356in}}%
\pgfpathlineto{\pgfqpoint{2.589833in}{0.822356in}}%
\pgfpathlineto{\pgfqpoint{2.635844in}{0.822356in}}%
\pgfpathlineto{\pgfqpoint{2.681855in}{0.822356in}}%
\pgfpathlineto{\pgfqpoint{2.727866in}{0.822356in}}%
\pgfpathlineto{\pgfqpoint{2.773878in}{0.822356in}}%
\pgfpathlineto{\pgfqpoint{2.819889in}{0.822356in}}%
\pgfpathlineto{\pgfqpoint{2.865900in}{0.822356in}}%
\pgfpathlineto{\pgfqpoint{2.911911in}{0.822356in}}%
\pgfpathlineto{\pgfqpoint{2.957922in}{0.822356in}}%
\pgfpathlineto{\pgfqpoint{3.003933in}{0.822356in}}%
\pgfpathlineto{\pgfqpoint{3.049944in}{0.822356in}}%
\pgfpathlineto{\pgfqpoint{3.095955in}{0.822356in}}%
\pgfpathlineto{\pgfqpoint{3.141967in}{0.822356in}}%
\pgfpathlineto{\pgfqpoint{3.187978in}{0.822356in}}%
\pgfpathlineto{\pgfqpoint{3.233989in}{0.822356in}}%
\pgfpathlineto{\pgfqpoint{3.280000in}{0.822356in}}%
\pgfpathlineto{\pgfqpoint{3.326011in}{0.822356in}}%
\pgfpathlineto{\pgfqpoint{3.372022in}{0.822356in}}%
\pgfpathlineto{\pgfqpoint{3.418033in}{0.822356in}}%
\pgfpathlineto{\pgfqpoint{3.464045in}{0.822356in}}%
\pgfpathlineto{\pgfqpoint{3.510056in}{0.822356in}}%
\pgfpathlineto{\pgfqpoint{3.556067in}{0.822356in}}%
\pgfpathlineto{\pgfqpoint{3.602078in}{0.822356in}}%
\pgfpathlineto{\pgfqpoint{3.648089in}{0.822356in}}%
\pgfpathlineto{\pgfqpoint{3.694100in}{0.822356in}}%
\pgfpathlineto{\pgfqpoint{3.740111in}{0.822356in}}%
\pgfpathlineto{\pgfqpoint{3.786122in}{0.822356in}}%
\pgfpathlineto{\pgfqpoint{3.832134in}{0.822356in}}%
\pgfpathlineto{\pgfqpoint{3.878145in}{0.822356in}}%
\pgfpathlineto{\pgfqpoint{3.924156in}{0.822356in}}%
\pgfpathlineto{\pgfqpoint{3.970167in}{0.822356in}}%
\pgfpathlineto{\pgfqpoint{4.016178in}{0.822356in}}%
\pgfpathlineto{\pgfqpoint{4.062189in}{0.822356in}}%
\pgfpathlineto{\pgfqpoint{4.108200in}{0.822356in}}%
\pgfpathlineto{\pgfqpoint{4.108200in}{1.326115in}}%
\pgfpathlineto{\pgfqpoint{4.154212in}{0.822356in}}%
\pgfpathlineto{\pgfqpoint{4.200223in}{0.822356in}}%
\pgfpathlineto{\pgfqpoint{4.246234in}{0.822356in}}%
\pgfpathlineto{\pgfqpoint{4.292245in}{0.822356in}}%
\pgfpathlineto{\pgfqpoint{4.338256in}{0.822356in}}%
\pgfpathlineto{\pgfqpoint{4.384267in}{0.822356in}}%
\pgfpathlineto{\pgfqpoint{4.430278in}{0.822356in}}%
\pgfpathlineto{\pgfqpoint{4.476289in}{0.822356in}}%
\pgfpathlineto{\pgfqpoint{4.522301in}{0.822356in}}%
\pgfpathlineto{\pgfqpoint{4.568312in}{0.822356in}}%
\pgfpathlineto{\pgfqpoint{4.614323in}{0.822356in}}%
\pgfpathlineto{\pgfqpoint{4.660334in}{0.822356in}}%
\pgfpathlineto{\pgfqpoint{4.706345in}{0.822356in}}%
\pgfpathlineto{\pgfqpoint{4.752356in}{0.822356in}}%
\pgfpathlineto{\pgfqpoint{4.798367in}{0.822356in}}%
\pgfpathlineto{\pgfqpoint{4.844378in}{0.822356in}}%
\pgfpathlineto{\pgfqpoint{4.890390in}{0.822356in}}%
\pgfpathlineto{\pgfqpoint{4.936401in}{0.822356in}}%
\pgfpathlineto{\pgfqpoint{4.982412in}{0.822356in}}%
\pgfpathlineto{\pgfqpoint{5.028423in}{0.822356in}}%
\pgfpathlineto{\pgfqpoint{5.074434in}{0.822356in}}%
\pgfpathlineto{\pgfqpoint{5.120445in}{0.822356in}}%
\pgfpathlineto{\pgfqpoint{5.166456in}{0.822356in}}%
\pgfpathlineto{\pgfqpoint{5.212468in}{0.822356in}}%
\pgfpathlineto{\pgfqpoint{5.258479in}{0.822356in}}%
\pgfpathlineto{\pgfqpoint{5.304490in}{0.822356in}}%
\pgfpathlineto{\pgfqpoint{5.350501in}{0.822356in}}%
\pgfpathlineto{\pgfqpoint{5.396512in}{0.822356in}}%
\pgfpathlineto{\pgfqpoint{5.442523in}{0.822356in}}%
\pgfpathlineto{\pgfqpoint{5.488534in}{0.822356in}}%
\pgfpathlineto{\pgfqpoint{5.534545in}{0.822356in}}%
\pgfusepath{stroke}%
\end{pgfscope}%
\begin{pgfscope}%
\pgfpathrectangle{\pgfqpoint{0.800000in}{0.528000in}}{\pgfqpoint{4.960000in}{3.696000in}}%
\pgfusepath{clip}%
\pgfsetrectcap%
\pgfsetroundjoin%
\pgfsetlinewidth{1.505625pt}%
\definecolor{currentstroke}{rgb}{0.580392,0.403922,0.741176}%
\pgfsetstrokecolor{currentstroke}%
\pgfsetdash{}{0pt}%
\pgfpathmoveto{\pgfqpoint{1.025455in}{2.177067in}}%
\pgfpathlineto{\pgfqpoint{1.071466in}{2.202519in}}%
\pgfpathlineto{\pgfqpoint{1.117477in}{2.400262in}}%
\pgfpathlineto{\pgfqpoint{1.163488in}{2.522088in}}%
\pgfpathlineto{\pgfqpoint{1.209499in}{2.618468in}}%
\pgfpathlineto{\pgfqpoint{1.255510in}{2.715518in}}%
\pgfpathlineto{\pgfqpoint{1.301521in}{2.782368in}}%
\pgfpathlineto{\pgfqpoint{1.347532in}{2.838023in}}%
\pgfpathlineto{\pgfqpoint{1.393544in}{2.892866in}}%
\pgfpathlineto{\pgfqpoint{1.439555in}{2.951910in}}%
\pgfpathlineto{\pgfqpoint{1.485566in}{2.972579in}}%
\pgfpathlineto{\pgfqpoint{1.531577in}{3.025465in}}%
\pgfpathlineto{\pgfqpoint{1.577588in}{3.078867in}}%
\pgfpathlineto{\pgfqpoint{1.623599in}{3.093327in}}%
\pgfpathlineto{\pgfqpoint{1.669610in}{3.135063in}}%
\pgfpathlineto{\pgfqpoint{1.715622in}{3.179362in}}%
\pgfpathlineto{\pgfqpoint{1.761633in}{3.195895in}}%
\pgfpathlineto{\pgfqpoint{1.807644in}{3.239205in}}%
\pgfpathlineto{\pgfqpoint{1.853655in}{3.251444in}}%
\pgfpathlineto{\pgfqpoint{1.899666in}{3.259743in}}%
\pgfpathlineto{\pgfqpoint{1.945677in}{3.303049in}}%
\pgfpathlineto{\pgfqpoint{1.991688in}{3.311354in}}%
\pgfpathlineto{\pgfqpoint{2.037699in}{3.352426in}}%
\pgfpathlineto{\pgfqpoint{2.083711in}{3.361809in}}%
\pgfpathlineto{\pgfqpoint{2.129722in}{3.369465in}}%
\pgfpathlineto{\pgfqpoint{2.175733in}{3.410286in}}%
\pgfpathlineto{\pgfqpoint{2.221744in}{3.421615in}}%
\pgfpathlineto{\pgfqpoint{2.267755in}{3.430752in}}%
\pgfpathlineto{\pgfqpoint{2.313766in}{3.469890in}}%
\pgfpathlineto{\pgfqpoint{2.359777in}{3.474952in}}%
\pgfpathlineto{\pgfqpoint{2.405788in}{3.479074in}}%
\pgfpathlineto{\pgfqpoint{2.451800in}{3.488943in}}%
\pgfpathlineto{\pgfqpoint{2.497811in}{3.522816in}}%
\pgfpathlineto{\pgfqpoint{2.543822in}{3.530008in}}%
\pgfpathlineto{\pgfqpoint{2.589833in}{3.537098in}}%
\pgfpathlineto{\pgfqpoint{2.635844in}{3.546906in}}%
\pgfpathlineto{\pgfqpoint{2.681855in}{3.579537in}}%
\pgfpathlineto{\pgfqpoint{2.727866in}{3.588544in}}%
\pgfpathlineto{\pgfqpoint{2.773878in}{3.594614in}}%
\pgfpathlineto{\pgfqpoint{2.819889in}{3.597621in}}%
\pgfpathlineto{\pgfqpoint{2.865900in}{3.607439in}}%
\pgfpathlineto{\pgfqpoint{2.911911in}{3.640236in}}%
\pgfpathlineto{\pgfqpoint{2.957922in}{3.645287in}}%
\pgfpathlineto{\pgfqpoint{3.003933in}{3.653592in}}%
\pgfpathlineto{\pgfqpoint{3.049944in}{3.657080in}}%
\pgfpathlineto{\pgfqpoint{3.095955in}{3.661354in}}%
\pgfpathlineto{\pgfqpoint{3.141967in}{3.697118in}}%
\pgfpathlineto{\pgfqpoint{3.187978in}{3.702555in}}%
\pgfpathlineto{\pgfqpoint{3.233989in}{3.708854in}}%
\pgfpathlineto{\pgfqpoint{3.280000in}{3.710692in}}%
\pgfpathlineto{\pgfqpoint{3.326011in}{3.715619in}}%
\pgfpathlineto{\pgfqpoint{3.372022in}{3.722292in}}%
\pgfpathlineto{\pgfqpoint{3.418033in}{3.758715in}}%
\pgfpathlineto{\pgfqpoint{3.464045in}{3.761536in}}%
\pgfpathlineto{\pgfqpoint{3.510056in}{3.764176in}}%
\pgfpathlineto{\pgfqpoint{3.556067in}{3.768763in}}%
\pgfpathlineto{\pgfqpoint{3.602078in}{3.772499in}}%
\pgfpathlineto{\pgfqpoint{3.648089in}{3.775403in}}%
\pgfpathlineto{\pgfqpoint{3.694100in}{3.809885in}}%
\pgfpathlineto{\pgfqpoint{3.740111in}{3.817330in}}%
\pgfpathlineto{\pgfqpoint{3.786122in}{3.820864in}}%
\pgfpathlineto{\pgfqpoint{3.832134in}{3.825973in}}%
\pgfpathlineto{\pgfqpoint{3.878145in}{3.829698in}}%
\pgfpathlineto{\pgfqpoint{3.924156in}{3.828940in}}%
\pgfpathlineto{\pgfqpoint{3.970167in}{3.836390in}}%
\pgfpathlineto{\pgfqpoint{4.016178in}{3.840320in}}%
\pgfpathlineto{\pgfqpoint{4.062189in}{3.873194in}}%
\pgfpathlineto{\pgfqpoint{4.108200in}{3.877498in}}%
\pgfpathlineto{\pgfqpoint{4.154212in}{3.880595in}}%
\pgfpathlineto{\pgfqpoint{4.200223in}{3.883447in}}%
\pgfpathlineto{\pgfqpoint{4.246234in}{3.887277in}}%
\pgfpathlineto{\pgfqpoint{4.292245in}{3.887405in}}%
\pgfpathlineto{\pgfqpoint{4.338256in}{3.894882in}}%
\pgfpathlineto{\pgfqpoint{4.384267in}{3.896993in}}%
\pgfpathlineto{\pgfqpoint{4.430278in}{3.927590in}}%
\pgfpathlineto{\pgfqpoint{4.476289in}{3.930154in}}%
\pgfpathlineto{\pgfqpoint{4.522301in}{3.932968in}}%
\pgfpathlineto{\pgfqpoint{4.568312in}{3.940135in}}%
\pgfpathlineto{\pgfqpoint{4.614323in}{3.941516in}}%
\pgfpathlineto{\pgfqpoint{4.660334in}{3.943323in}}%
\pgfpathlineto{\pgfqpoint{4.706345in}{3.947740in}}%
\pgfpathlineto{\pgfqpoint{4.752356in}{3.951019in}}%
\pgfpathlineto{\pgfqpoint{4.798367in}{3.955033in}}%
\pgfpathlineto{\pgfqpoint{4.844378in}{3.984545in}}%
\pgfpathlineto{\pgfqpoint{4.890390in}{3.988505in}}%
\pgfpathlineto{\pgfqpoint{4.936401in}{3.990070in}}%
\pgfpathlineto{\pgfqpoint{4.982412in}{3.994399in}}%
\pgfpathlineto{\pgfqpoint{5.028423in}{3.997384in}}%
\pgfpathlineto{\pgfqpoint{5.074434in}{4.000325in}}%
\pgfpathlineto{\pgfqpoint{5.120445in}{4.002361in}}%
\pgfpathlineto{\pgfqpoint{5.166456in}{4.005980in}}%
\pgfpathlineto{\pgfqpoint{5.212468in}{4.008996in}}%
\pgfpathlineto{\pgfqpoint{5.258479in}{4.011869in}}%
\pgfpathlineto{\pgfqpoint{5.304490in}{4.012765in}}%
\pgfpathlineto{\pgfqpoint{5.350501in}{4.047034in}}%
\pgfpathlineto{\pgfqpoint{5.396512in}{4.047776in}}%
\pgfpathlineto{\pgfqpoint{5.442523in}{4.052823in}}%
\pgfpathlineto{\pgfqpoint{5.488534in}{4.053739in}}%
\pgfpathlineto{\pgfqpoint{5.534545in}{4.056000in}}%
\pgfusepath{stroke}%
\end{pgfscope}%
\begin{pgfscope}%
\pgfsetrectcap%
\pgfsetmiterjoin%
\pgfsetlinewidth{0.803000pt}%
\definecolor{currentstroke}{rgb}{0.000000,0.000000,0.000000}%
\pgfsetstrokecolor{currentstroke}%
\pgfsetdash{}{0pt}%
\pgfpathmoveto{\pgfqpoint{0.800000in}{0.528000in}}%
\pgfpathlineto{\pgfqpoint{0.800000in}{4.224000in}}%
\pgfusepath{stroke}%
\end{pgfscope}%
\begin{pgfscope}%
\pgfsetrectcap%
\pgfsetmiterjoin%
\pgfsetlinewidth{0.803000pt}%
\definecolor{currentstroke}{rgb}{0.000000,0.000000,0.000000}%
\pgfsetstrokecolor{currentstroke}%
\pgfsetdash{}{0pt}%
\pgfpathmoveto{\pgfqpoint{5.760000in}{0.528000in}}%
\pgfpathlineto{\pgfqpoint{5.760000in}{4.224000in}}%
\pgfusepath{stroke}%
\end{pgfscope}%
\begin{pgfscope}%
\pgfsetrectcap%
\pgfsetmiterjoin%
\pgfsetlinewidth{0.803000pt}%
\definecolor{currentstroke}{rgb}{0.000000,0.000000,0.000000}%
\pgfsetstrokecolor{currentstroke}%
\pgfsetdash{}{0pt}%
\pgfpathmoveto{\pgfqpoint{0.800000in}{0.528000in}}%
\pgfpathlineto{\pgfqpoint{5.760000in}{0.528000in}}%
\pgfusepath{stroke}%
\end{pgfscope}%
\begin{pgfscope}%
\pgfsetrectcap%
\pgfsetmiterjoin%
\pgfsetlinewidth{0.803000pt}%
\definecolor{currentstroke}{rgb}{0.000000,0.000000,0.000000}%
\pgfsetstrokecolor{currentstroke}%
\pgfsetdash{}{0pt}%
\pgfpathmoveto{\pgfqpoint{0.800000in}{4.224000in}}%
\pgfpathlineto{\pgfqpoint{5.760000in}{4.224000in}}%
\pgfusepath{stroke}%
\end{pgfscope}%
\begin{pgfscope}%
\pgfsetbuttcap%
\pgfsetmiterjoin%
\definecolor{currentfill}{rgb}{1.000000,1.000000,1.000000}%
\pgfsetfillcolor{currentfill}%
\pgfsetfillopacity{0.800000}%
\pgfsetlinewidth{1.003750pt}%
\definecolor{currentstroke}{rgb}{0.800000,0.800000,0.800000}%
\pgfsetstrokecolor{currentstroke}%
\pgfsetstrokeopacity{0.800000}%
\pgfsetdash{}{0pt}%
\pgfpathmoveto{\pgfqpoint{0.897222in}{3.144525in}}%
\pgfpathlineto{\pgfqpoint{1.739044in}{3.144525in}}%
\pgfpathquadraticcurveto{\pgfqpoint{1.766822in}{3.144525in}}{\pgfqpoint{1.766822in}{3.172303in}}%
\pgfpathlineto{\pgfqpoint{1.766822in}{4.126778in}}%
\pgfpathquadraticcurveto{\pgfqpoint{1.766822in}{4.154556in}}{\pgfqpoint{1.739044in}{4.154556in}}%
\pgfpathlineto{\pgfqpoint{0.897222in}{4.154556in}}%
\pgfpathquadraticcurveto{\pgfqpoint{0.869444in}{4.154556in}}{\pgfqpoint{0.869444in}{4.126778in}}%
\pgfpathlineto{\pgfqpoint{0.869444in}{3.172303in}}%
\pgfpathquadraticcurveto{\pgfqpoint{0.869444in}{3.144525in}}{\pgfqpoint{0.897222in}{3.144525in}}%
\pgfpathlineto{\pgfqpoint{0.897222in}{3.144525in}}%
\pgfpathclose%
\pgfusepath{stroke,fill}%
\end{pgfscope}%
\begin{pgfscope}%
\pgfsetrectcap%
\pgfsetroundjoin%
\pgfsetlinewidth{1.505625pt}%
\definecolor{currentstroke}{rgb}{0.121569,0.466667,0.705882}%
\pgfsetstrokecolor{currentstroke}%
\pgfsetdash{}{0pt}%
\pgfpathmoveto{\pgfqpoint{0.925000in}{4.050389in}}%
\pgfpathlineto{\pgfqpoint{1.063889in}{4.050389in}}%
\pgfpathlineto{\pgfqpoint{1.202778in}{4.050389in}}%
\pgfusepath{stroke}%
\end{pgfscope}%
\begin{pgfscope}%
\definecolor{textcolor}{rgb}{0.000000,0.000000,0.000000}%
\pgfsetstrokecolor{textcolor}%
\pgfsetfillcolor{textcolor}%
\pgftext[x=1.313889in,y=4.001778in,left,base]{\color{textcolor}\rmfamily\fontsize{10.000000}{12.000000}\selectfont quick}%
\end{pgfscope}%
\begin{pgfscope}%
\pgfsetrectcap%
\pgfsetroundjoin%
\pgfsetlinewidth{1.505625pt}%
\definecolor{currentstroke}{rgb}{1.000000,0.498039,0.054902}%
\pgfsetstrokecolor{currentstroke}%
\pgfsetdash{}{0pt}%
\pgfpathmoveto{\pgfqpoint{0.925000in}{3.856716in}}%
\pgfpathlineto{\pgfqpoint{1.063889in}{3.856716in}}%
\pgfpathlineto{\pgfqpoint{1.202778in}{3.856716in}}%
\pgfusepath{stroke}%
\end{pgfscope}%
\begin{pgfscope}%
\definecolor{textcolor}{rgb}{0.000000,0.000000,0.000000}%
\pgfsetstrokecolor{textcolor}%
\pgfsetfillcolor{textcolor}%
\pgftext[x=1.313889in,y=3.808105in,left,base]{\color{textcolor}\rmfamily\fontsize{10.000000}{12.000000}\selectfont merge}%
\end{pgfscope}%
\begin{pgfscope}%
\pgfsetrectcap%
\pgfsetroundjoin%
\pgfsetlinewidth{1.505625pt}%
\definecolor{currentstroke}{rgb}{0.172549,0.627451,0.172549}%
\pgfsetstrokecolor{currentstroke}%
\pgfsetdash{}{0pt}%
\pgfpathmoveto{\pgfqpoint{0.925000in}{3.663043in}}%
\pgfpathlineto{\pgfqpoint{1.063889in}{3.663043in}}%
\pgfpathlineto{\pgfqpoint{1.202778in}{3.663043in}}%
\pgfusepath{stroke}%
\end{pgfscope}%
\begin{pgfscope}%
\definecolor{textcolor}{rgb}{0.000000,0.000000,0.000000}%
\pgfsetstrokecolor{textcolor}%
\pgfsetfillcolor{textcolor}%
\pgftext[x=1.313889in,y=3.614432in,left,base]{\color{textcolor}\rmfamily\fontsize{10.000000}{12.000000}\selectfont heap}%
\end{pgfscope}%
\begin{pgfscope}%
\pgfsetrectcap%
\pgfsetroundjoin%
\pgfsetlinewidth{1.505625pt}%
\definecolor{currentstroke}{rgb}{0.839216,0.152941,0.156863}%
\pgfsetstrokecolor{currentstroke}%
\pgfsetdash{}{0pt}%
\pgfpathmoveto{\pgfqpoint{0.925000in}{3.469371in}}%
\pgfpathlineto{\pgfqpoint{1.063889in}{3.469371in}}%
\pgfpathlineto{\pgfqpoint{1.202778in}{3.469371in}}%
\pgfusepath{stroke}%
\end{pgfscope}%
\begin{pgfscope}%
\definecolor{textcolor}{rgb}{0.000000,0.000000,0.000000}%
\pgfsetstrokecolor{textcolor}%
\pgfsetfillcolor{textcolor}%
\pgftext[x=1.313889in,y=3.420759in,left,base]{\color{textcolor}\rmfamily\fontsize{10.000000}{12.000000}\selectfont insert}%
\end{pgfscope}%
\begin{pgfscope}%
\pgfsetrectcap%
\pgfsetroundjoin%
\pgfsetlinewidth{1.505625pt}%
\definecolor{currentstroke}{rgb}{0.580392,0.403922,0.741176}%
\pgfsetstrokecolor{currentstroke}%
\pgfsetdash{}{0pt}%
\pgfpathmoveto{\pgfqpoint{0.925000in}{3.275698in}}%
\pgfpathlineto{\pgfqpoint{1.063889in}{3.275698in}}%
\pgfpathlineto{\pgfqpoint{1.202778in}{3.275698in}}%
\pgfusepath{stroke}%
\end{pgfscope}%
\begin{pgfscope}%
\definecolor{textcolor}{rgb}{0.000000,0.000000,0.000000}%
\pgfsetstrokecolor{textcolor}%
\pgfsetfillcolor{textcolor}%
\pgftext[x=1.313889in,y=3.227087in,left,base]{\color{textcolor}\rmfamily\fontsize{10.000000}{12.000000}\selectfont bucket}%
\end{pgfscope}%
\end{pgfpicture}%
\makeatother%
\endgroup%

\subsection{Analysis}
As we can see the graphs match the theoretical complexities calculated. Let us
focus on two most important measurements - \textit{Time, Space}.
\subsubsection{Time}
\textit{Insertion sort} with its $O(n^2)$ performs the worst.
The other 4, which all have $O(nlogn)$, perform relatively the same.
The reason why heapsort doesn't perform nearly as good as the other three is
on the hardware level heapsort has \textbf{more instruction that the other three.}
\subsubsection{Space}
Here \textit{Inplace quicksort and insertion sort} are the best as the have
$O(1)$ space complexity. In-place heapsort also has $O(1)$ space complexity
but the ADT used created a new heap from the given array. That is the reason
it has a similar memory use of other $O(n)$ algorithms. But the heapsort
itself \textbf{doesn't} need any auxiliary space.
\section{Optimizing Quick Sort}
\subsection{Problem}
The worst case performance of Quick sort is $O(n^2)$. This occurs when the
\textit{pivot chosen is the greatest or the smallest} causing one of the
partitions to be of size $n-1$.
\subsection{Solution}
This has an easy fix by picking a better pivot. The solution implemented here
is called \texttt{Three median pivot}. Use \textit{Insertion sort} for
partitions \textbf{smaller than a certain threshold(Eg. 10)}
\begin{minted}{python}
    mid = (left + right) // 2
    if arr[mid] < arr[left]:
        arr[left], arr[mid] = arr[mid], arr[left]
    if arr[end] < arr[left]:
        arr[left], arr[end] = arr[end], arr[left]
    if arr[mid] < arr[end]:
        arr[end], arr[mid] = arr[mid], arr[end]
    pivot = arr[end]
    pivot_i = end
\end{minted}
This makes it so that the pivot is never the greatest or the smallest.
\subsection{Testing}
The data given to this particular comparison was sorted in the reverse order
which is the worst case of normal quick sort.
\subsection{Plots}
\subsubsection{Comparisons}
%% Creator: Matplotlib, PGF backend
%%
%% To include the figure in your LaTeX document, write
%%   \input{<filename>.pgf}
%%
%% Make sure the required packages are loaded in your preamble
%%   \usepackage{pgf}
%%
%% Also ensure that all the required font packages are loaded; for instance,
%% the lmodern package is sometimes necessary when using math font.
%%   \usepackage{lmodern}
%%
%% Figures using additional raster images can only be included by \input if
%% they are in the same directory as the main LaTeX file. For loading figures
%% from other directories you can use the `import` package
%%   \usepackage{import}
%%
%% and then include the figures with
%%   \import{<path to file>}{<filename>.pgf}
%%
%% Matplotlib used the following preamble
%%   
%%   \makeatletter\@ifpackageloaded{underscore}{}{\usepackage[strings]{underscore}}\makeatother
%%
\begingroup%
\makeatletter%
\begin{pgfpicture}%
\pgfpathrectangle{\pgfpointorigin}{\pgfqpoint{6.400000in}{4.800000in}}%
\pgfusepath{use as bounding box, clip}%
\begin{pgfscope}%
\pgfsetbuttcap%
\pgfsetmiterjoin%
\definecolor{currentfill}{rgb}{1.000000,1.000000,1.000000}%
\pgfsetfillcolor{currentfill}%
\pgfsetlinewidth{0.000000pt}%
\definecolor{currentstroke}{rgb}{1.000000,1.000000,1.000000}%
\pgfsetstrokecolor{currentstroke}%
\pgfsetdash{}{0pt}%
\pgfpathmoveto{\pgfqpoint{0.000000in}{0.000000in}}%
\pgfpathlineto{\pgfqpoint{6.400000in}{0.000000in}}%
\pgfpathlineto{\pgfqpoint{6.400000in}{4.800000in}}%
\pgfpathlineto{\pgfqpoint{0.000000in}{4.800000in}}%
\pgfpathlineto{\pgfqpoint{0.000000in}{0.000000in}}%
\pgfpathclose%
\pgfusepath{fill}%
\end{pgfscope}%
\begin{pgfscope}%
\pgfsetbuttcap%
\pgfsetmiterjoin%
\definecolor{currentfill}{rgb}{1.000000,1.000000,1.000000}%
\pgfsetfillcolor{currentfill}%
\pgfsetlinewidth{0.000000pt}%
\definecolor{currentstroke}{rgb}{0.000000,0.000000,0.000000}%
\pgfsetstrokecolor{currentstroke}%
\pgfsetstrokeopacity{0.000000}%
\pgfsetdash{}{0pt}%
\pgfpathmoveto{\pgfqpoint{0.800000in}{0.528000in}}%
\pgfpathlineto{\pgfqpoint{5.760000in}{0.528000in}}%
\pgfpathlineto{\pgfqpoint{5.760000in}{4.224000in}}%
\pgfpathlineto{\pgfqpoint{0.800000in}{4.224000in}}%
\pgfpathlineto{\pgfqpoint{0.800000in}{0.528000in}}%
\pgfpathclose%
\pgfusepath{fill}%
\end{pgfscope}%
\begin{pgfscope}%
\pgfsetbuttcap%
\pgfsetroundjoin%
\definecolor{currentfill}{rgb}{0.000000,0.000000,0.000000}%
\pgfsetfillcolor{currentfill}%
\pgfsetlinewidth{0.803000pt}%
\definecolor{currentstroke}{rgb}{0.000000,0.000000,0.000000}%
\pgfsetstrokecolor{currentstroke}%
\pgfsetdash{}{0pt}%
\pgfsys@defobject{currentmarker}{\pgfqpoint{0.000000in}{-0.048611in}}{\pgfqpoint{0.000000in}{0.000000in}}{%
\pgfpathmoveto{\pgfqpoint{0.000000in}{0.000000in}}%
\pgfpathlineto{\pgfqpoint{0.000000in}{-0.048611in}}%
\pgfusepath{stroke,fill}%
}%
\begin{pgfscope}%
\pgfsys@transformshift{0.979443in}{0.528000in}%
\pgfsys@useobject{currentmarker}{}%
\end{pgfscope}%
\end{pgfscope}%
\begin{pgfscope}%
\definecolor{textcolor}{rgb}{0.000000,0.000000,0.000000}%
\pgfsetstrokecolor{textcolor}%
\pgfsetfillcolor{textcolor}%
\pgftext[x=0.979443in,y=0.430778in,,top]{\color{textcolor}\rmfamily\fontsize{10.000000}{12.000000}\selectfont \(\displaystyle {0}\)}%
\end{pgfscope}%
\begin{pgfscope}%
\pgfsetbuttcap%
\pgfsetroundjoin%
\definecolor{currentfill}{rgb}{0.000000,0.000000,0.000000}%
\pgfsetfillcolor{currentfill}%
\pgfsetlinewidth{0.803000pt}%
\definecolor{currentstroke}{rgb}{0.000000,0.000000,0.000000}%
\pgfsetstrokecolor{currentstroke}%
\pgfsetdash{}{0pt}%
\pgfsys@defobject{currentmarker}{\pgfqpoint{0.000000in}{-0.048611in}}{\pgfqpoint{0.000000in}{0.000000in}}{%
\pgfpathmoveto{\pgfqpoint{0.000000in}{0.000000in}}%
\pgfpathlineto{\pgfqpoint{0.000000in}{-0.048611in}}%
\pgfusepath{stroke,fill}%
}%
\begin{pgfscope}%
\pgfsys@transformshift{1.899666in}{0.528000in}%
\pgfsys@useobject{currentmarker}{}%
\end{pgfscope}%
\end{pgfscope}%
\begin{pgfscope}%
\definecolor{textcolor}{rgb}{0.000000,0.000000,0.000000}%
\pgfsetstrokecolor{textcolor}%
\pgfsetfillcolor{textcolor}%
\pgftext[x=1.899666in,y=0.430778in,,top]{\color{textcolor}\rmfamily\fontsize{10.000000}{12.000000}\selectfont \(\displaystyle {200}\)}%
\end{pgfscope}%
\begin{pgfscope}%
\pgfsetbuttcap%
\pgfsetroundjoin%
\definecolor{currentfill}{rgb}{0.000000,0.000000,0.000000}%
\pgfsetfillcolor{currentfill}%
\pgfsetlinewidth{0.803000pt}%
\definecolor{currentstroke}{rgb}{0.000000,0.000000,0.000000}%
\pgfsetstrokecolor{currentstroke}%
\pgfsetdash{}{0pt}%
\pgfsys@defobject{currentmarker}{\pgfqpoint{0.000000in}{-0.048611in}}{\pgfqpoint{0.000000in}{0.000000in}}{%
\pgfpathmoveto{\pgfqpoint{0.000000in}{0.000000in}}%
\pgfpathlineto{\pgfqpoint{0.000000in}{-0.048611in}}%
\pgfusepath{stroke,fill}%
}%
\begin{pgfscope}%
\pgfsys@transformshift{2.819889in}{0.528000in}%
\pgfsys@useobject{currentmarker}{}%
\end{pgfscope}%
\end{pgfscope}%
\begin{pgfscope}%
\definecolor{textcolor}{rgb}{0.000000,0.000000,0.000000}%
\pgfsetstrokecolor{textcolor}%
\pgfsetfillcolor{textcolor}%
\pgftext[x=2.819889in,y=0.430778in,,top]{\color{textcolor}\rmfamily\fontsize{10.000000}{12.000000}\selectfont \(\displaystyle {400}\)}%
\end{pgfscope}%
\begin{pgfscope}%
\pgfsetbuttcap%
\pgfsetroundjoin%
\definecolor{currentfill}{rgb}{0.000000,0.000000,0.000000}%
\pgfsetfillcolor{currentfill}%
\pgfsetlinewidth{0.803000pt}%
\definecolor{currentstroke}{rgb}{0.000000,0.000000,0.000000}%
\pgfsetstrokecolor{currentstroke}%
\pgfsetdash{}{0pt}%
\pgfsys@defobject{currentmarker}{\pgfqpoint{0.000000in}{-0.048611in}}{\pgfqpoint{0.000000in}{0.000000in}}{%
\pgfpathmoveto{\pgfqpoint{0.000000in}{0.000000in}}%
\pgfpathlineto{\pgfqpoint{0.000000in}{-0.048611in}}%
\pgfusepath{stroke,fill}%
}%
\begin{pgfscope}%
\pgfsys@transformshift{3.740111in}{0.528000in}%
\pgfsys@useobject{currentmarker}{}%
\end{pgfscope}%
\end{pgfscope}%
\begin{pgfscope}%
\definecolor{textcolor}{rgb}{0.000000,0.000000,0.000000}%
\pgfsetstrokecolor{textcolor}%
\pgfsetfillcolor{textcolor}%
\pgftext[x=3.740111in,y=0.430778in,,top]{\color{textcolor}\rmfamily\fontsize{10.000000}{12.000000}\selectfont \(\displaystyle {600}\)}%
\end{pgfscope}%
\begin{pgfscope}%
\pgfsetbuttcap%
\pgfsetroundjoin%
\definecolor{currentfill}{rgb}{0.000000,0.000000,0.000000}%
\pgfsetfillcolor{currentfill}%
\pgfsetlinewidth{0.803000pt}%
\definecolor{currentstroke}{rgb}{0.000000,0.000000,0.000000}%
\pgfsetstrokecolor{currentstroke}%
\pgfsetdash{}{0pt}%
\pgfsys@defobject{currentmarker}{\pgfqpoint{0.000000in}{-0.048611in}}{\pgfqpoint{0.000000in}{0.000000in}}{%
\pgfpathmoveto{\pgfqpoint{0.000000in}{0.000000in}}%
\pgfpathlineto{\pgfqpoint{0.000000in}{-0.048611in}}%
\pgfusepath{stroke,fill}%
}%
\begin{pgfscope}%
\pgfsys@transformshift{4.660334in}{0.528000in}%
\pgfsys@useobject{currentmarker}{}%
\end{pgfscope}%
\end{pgfscope}%
\begin{pgfscope}%
\definecolor{textcolor}{rgb}{0.000000,0.000000,0.000000}%
\pgfsetstrokecolor{textcolor}%
\pgfsetfillcolor{textcolor}%
\pgftext[x=4.660334in,y=0.430778in,,top]{\color{textcolor}\rmfamily\fontsize{10.000000}{12.000000}\selectfont \(\displaystyle {800}\)}%
\end{pgfscope}%
\begin{pgfscope}%
\pgfsetbuttcap%
\pgfsetroundjoin%
\definecolor{currentfill}{rgb}{0.000000,0.000000,0.000000}%
\pgfsetfillcolor{currentfill}%
\pgfsetlinewidth{0.803000pt}%
\definecolor{currentstroke}{rgb}{0.000000,0.000000,0.000000}%
\pgfsetstrokecolor{currentstroke}%
\pgfsetdash{}{0pt}%
\pgfsys@defobject{currentmarker}{\pgfqpoint{0.000000in}{-0.048611in}}{\pgfqpoint{0.000000in}{0.000000in}}{%
\pgfpathmoveto{\pgfqpoint{0.000000in}{0.000000in}}%
\pgfpathlineto{\pgfqpoint{0.000000in}{-0.048611in}}%
\pgfusepath{stroke,fill}%
}%
\begin{pgfscope}%
\pgfsys@transformshift{5.580557in}{0.528000in}%
\pgfsys@useobject{currentmarker}{}%
\end{pgfscope}%
\end{pgfscope}%
\begin{pgfscope}%
\definecolor{textcolor}{rgb}{0.000000,0.000000,0.000000}%
\pgfsetstrokecolor{textcolor}%
\pgfsetfillcolor{textcolor}%
\pgftext[x=5.580557in,y=0.430778in,,top]{\color{textcolor}\rmfamily\fontsize{10.000000}{12.000000}\selectfont \(\displaystyle {1000}\)}%
\end{pgfscope}%
\begin{pgfscope}%
\definecolor{textcolor}{rgb}{0.000000,0.000000,0.000000}%
\pgfsetstrokecolor{textcolor}%
\pgfsetfillcolor{textcolor}%
\pgftext[x=3.280000in,y=0.251766in,,top]{\color{textcolor}\rmfamily\fontsize{10.000000}{12.000000}\selectfont Input Size}%
\end{pgfscope}%
\begin{pgfscope}%
\pgfsetbuttcap%
\pgfsetroundjoin%
\definecolor{currentfill}{rgb}{0.000000,0.000000,0.000000}%
\pgfsetfillcolor{currentfill}%
\pgfsetlinewidth{0.803000pt}%
\definecolor{currentstroke}{rgb}{0.000000,0.000000,0.000000}%
\pgfsetstrokecolor{currentstroke}%
\pgfsetdash{}{0pt}%
\pgfsys@defobject{currentmarker}{\pgfqpoint{-0.048611in}{0.000000in}}{\pgfqpoint{-0.000000in}{0.000000in}}{%
\pgfpathmoveto{\pgfqpoint{-0.000000in}{0.000000in}}%
\pgfpathlineto{\pgfqpoint{-0.048611in}{0.000000in}}%
\pgfusepath{stroke,fill}%
}%
\begin{pgfscope}%
\pgfsys@transformshift{0.800000in}{0.931597in}%
\pgfsys@useobject{currentmarker}{}%
\end{pgfscope}%
\end{pgfscope}%
\begin{pgfscope}%
\definecolor{textcolor}{rgb}{0.000000,0.000000,0.000000}%
\pgfsetstrokecolor{textcolor}%
\pgfsetfillcolor{textcolor}%
\pgftext[x=0.501581in, y=0.883372in, left, base]{\color{textcolor}\rmfamily\fontsize{10.000000}{12.000000}\selectfont \(\displaystyle {10^{2}}\)}%
\end{pgfscope}%
\begin{pgfscope}%
\pgfsetbuttcap%
\pgfsetroundjoin%
\definecolor{currentfill}{rgb}{0.000000,0.000000,0.000000}%
\pgfsetfillcolor{currentfill}%
\pgfsetlinewidth{0.803000pt}%
\definecolor{currentstroke}{rgb}{0.000000,0.000000,0.000000}%
\pgfsetstrokecolor{currentstroke}%
\pgfsetdash{}{0pt}%
\pgfsys@defobject{currentmarker}{\pgfqpoint{-0.048611in}{0.000000in}}{\pgfqpoint{-0.000000in}{0.000000in}}{%
\pgfpathmoveto{\pgfqpoint{-0.000000in}{0.000000in}}%
\pgfpathlineto{\pgfqpoint{-0.048611in}{0.000000in}}%
\pgfusepath{stroke,fill}%
}%
\begin{pgfscope}%
\pgfsys@transformshift{0.800000in}{1.714235in}%
\pgfsys@useobject{currentmarker}{}%
\end{pgfscope}%
\end{pgfscope}%
\begin{pgfscope}%
\definecolor{textcolor}{rgb}{0.000000,0.000000,0.000000}%
\pgfsetstrokecolor{textcolor}%
\pgfsetfillcolor{textcolor}%
\pgftext[x=0.501581in, y=1.666009in, left, base]{\color{textcolor}\rmfamily\fontsize{10.000000}{12.000000}\selectfont \(\displaystyle {10^{3}}\)}%
\end{pgfscope}%
\begin{pgfscope}%
\pgfsetbuttcap%
\pgfsetroundjoin%
\definecolor{currentfill}{rgb}{0.000000,0.000000,0.000000}%
\pgfsetfillcolor{currentfill}%
\pgfsetlinewidth{0.803000pt}%
\definecolor{currentstroke}{rgb}{0.000000,0.000000,0.000000}%
\pgfsetstrokecolor{currentstroke}%
\pgfsetdash{}{0pt}%
\pgfsys@defobject{currentmarker}{\pgfqpoint{-0.048611in}{0.000000in}}{\pgfqpoint{-0.000000in}{0.000000in}}{%
\pgfpathmoveto{\pgfqpoint{-0.000000in}{0.000000in}}%
\pgfpathlineto{\pgfqpoint{-0.048611in}{0.000000in}}%
\pgfusepath{stroke,fill}%
}%
\begin{pgfscope}%
\pgfsys@transformshift{0.800000in}{2.496872in}%
\pgfsys@useobject{currentmarker}{}%
\end{pgfscope}%
\end{pgfscope}%
\begin{pgfscope}%
\definecolor{textcolor}{rgb}{0.000000,0.000000,0.000000}%
\pgfsetstrokecolor{textcolor}%
\pgfsetfillcolor{textcolor}%
\pgftext[x=0.501581in, y=2.448647in, left, base]{\color{textcolor}\rmfamily\fontsize{10.000000}{12.000000}\selectfont \(\displaystyle {10^{4}}\)}%
\end{pgfscope}%
\begin{pgfscope}%
\pgfsetbuttcap%
\pgfsetroundjoin%
\definecolor{currentfill}{rgb}{0.000000,0.000000,0.000000}%
\pgfsetfillcolor{currentfill}%
\pgfsetlinewidth{0.803000pt}%
\definecolor{currentstroke}{rgb}{0.000000,0.000000,0.000000}%
\pgfsetstrokecolor{currentstroke}%
\pgfsetdash{}{0pt}%
\pgfsys@defobject{currentmarker}{\pgfqpoint{-0.048611in}{0.000000in}}{\pgfqpoint{-0.000000in}{0.000000in}}{%
\pgfpathmoveto{\pgfqpoint{-0.000000in}{0.000000in}}%
\pgfpathlineto{\pgfqpoint{-0.048611in}{0.000000in}}%
\pgfusepath{stroke,fill}%
}%
\begin{pgfscope}%
\pgfsys@transformshift{0.800000in}{3.279509in}%
\pgfsys@useobject{currentmarker}{}%
\end{pgfscope}%
\end{pgfscope}%
\begin{pgfscope}%
\definecolor{textcolor}{rgb}{0.000000,0.000000,0.000000}%
\pgfsetstrokecolor{textcolor}%
\pgfsetfillcolor{textcolor}%
\pgftext[x=0.501581in, y=3.231284in, left, base]{\color{textcolor}\rmfamily\fontsize{10.000000}{12.000000}\selectfont \(\displaystyle {10^{5}}\)}%
\end{pgfscope}%
\begin{pgfscope}%
\pgfsetbuttcap%
\pgfsetroundjoin%
\definecolor{currentfill}{rgb}{0.000000,0.000000,0.000000}%
\pgfsetfillcolor{currentfill}%
\pgfsetlinewidth{0.803000pt}%
\definecolor{currentstroke}{rgb}{0.000000,0.000000,0.000000}%
\pgfsetstrokecolor{currentstroke}%
\pgfsetdash{}{0pt}%
\pgfsys@defobject{currentmarker}{\pgfqpoint{-0.048611in}{0.000000in}}{\pgfqpoint{-0.000000in}{0.000000in}}{%
\pgfpathmoveto{\pgfqpoint{-0.000000in}{0.000000in}}%
\pgfpathlineto{\pgfqpoint{-0.048611in}{0.000000in}}%
\pgfusepath{stroke,fill}%
}%
\begin{pgfscope}%
\pgfsys@transformshift{0.800000in}{4.062147in}%
\pgfsys@useobject{currentmarker}{}%
\end{pgfscope}%
\end{pgfscope}%
\begin{pgfscope}%
\definecolor{textcolor}{rgb}{0.000000,0.000000,0.000000}%
\pgfsetstrokecolor{textcolor}%
\pgfsetfillcolor{textcolor}%
\pgftext[x=0.501581in, y=4.013922in, left, base]{\color{textcolor}\rmfamily\fontsize{10.000000}{12.000000}\selectfont \(\displaystyle {10^{6}}\)}%
\end{pgfscope}%
\begin{pgfscope}%
\pgfsetbuttcap%
\pgfsetroundjoin%
\definecolor{currentfill}{rgb}{0.000000,0.000000,0.000000}%
\pgfsetfillcolor{currentfill}%
\pgfsetlinewidth{0.602250pt}%
\definecolor{currentstroke}{rgb}{0.000000,0.000000,0.000000}%
\pgfsetstrokecolor{currentstroke}%
\pgfsetdash{}{0pt}%
\pgfsys@defobject{currentmarker}{\pgfqpoint{-0.027778in}{0.000000in}}{\pgfqpoint{-0.000000in}{0.000000in}}{%
\pgfpathmoveto{\pgfqpoint{-0.000000in}{0.000000in}}%
\pgfpathlineto{\pgfqpoint{-0.027778in}{0.000000in}}%
\pgfusepath{stroke,fill}%
}%
\begin{pgfscope}%
\pgfsys@transformshift{0.800000in}{0.620155in}%
\pgfsys@useobject{currentmarker}{}%
\end{pgfscope}%
\end{pgfscope}%
\begin{pgfscope}%
\pgfsetbuttcap%
\pgfsetroundjoin%
\definecolor{currentfill}{rgb}{0.000000,0.000000,0.000000}%
\pgfsetfillcolor{currentfill}%
\pgfsetlinewidth{0.602250pt}%
\definecolor{currentstroke}{rgb}{0.000000,0.000000,0.000000}%
\pgfsetstrokecolor{currentstroke}%
\pgfsetdash{}{0pt}%
\pgfsys@defobject{currentmarker}{\pgfqpoint{-0.027778in}{0.000000in}}{\pgfqpoint{-0.000000in}{0.000000in}}{%
\pgfpathmoveto{\pgfqpoint{-0.000000in}{0.000000in}}%
\pgfpathlineto{\pgfqpoint{-0.027778in}{0.000000in}}%
\pgfusepath{stroke,fill}%
}%
\begin{pgfscope}%
\pgfsys@transformshift{0.800000in}{0.696000in}%
\pgfsys@useobject{currentmarker}{}%
\end{pgfscope}%
\end{pgfscope}%
\begin{pgfscope}%
\pgfsetbuttcap%
\pgfsetroundjoin%
\definecolor{currentfill}{rgb}{0.000000,0.000000,0.000000}%
\pgfsetfillcolor{currentfill}%
\pgfsetlinewidth{0.602250pt}%
\definecolor{currentstroke}{rgb}{0.000000,0.000000,0.000000}%
\pgfsetstrokecolor{currentstroke}%
\pgfsetdash{}{0pt}%
\pgfsys@defobject{currentmarker}{\pgfqpoint{-0.027778in}{0.000000in}}{\pgfqpoint{-0.000000in}{0.000000in}}{%
\pgfpathmoveto{\pgfqpoint{-0.000000in}{0.000000in}}%
\pgfpathlineto{\pgfqpoint{-0.027778in}{0.000000in}}%
\pgfusepath{stroke,fill}%
}%
\begin{pgfscope}%
\pgfsys@transformshift{0.800000in}{0.757970in}%
\pgfsys@useobject{currentmarker}{}%
\end{pgfscope}%
\end{pgfscope}%
\begin{pgfscope}%
\pgfsetbuttcap%
\pgfsetroundjoin%
\definecolor{currentfill}{rgb}{0.000000,0.000000,0.000000}%
\pgfsetfillcolor{currentfill}%
\pgfsetlinewidth{0.602250pt}%
\definecolor{currentstroke}{rgb}{0.000000,0.000000,0.000000}%
\pgfsetstrokecolor{currentstroke}%
\pgfsetdash{}{0pt}%
\pgfsys@defobject{currentmarker}{\pgfqpoint{-0.027778in}{0.000000in}}{\pgfqpoint{-0.000000in}{0.000000in}}{%
\pgfpathmoveto{\pgfqpoint{-0.000000in}{0.000000in}}%
\pgfpathlineto{\pgfqpoint{-0.027778in}{0.000000in}}%
\pgfusepath{stroke,fill}%
}%
\begin{pgfscope}%
\pgfsys@transformshift{0.800000in}{0.810365in}%
\pgfsys@useobject{currentmarker}{}%
\end{pgfscope}%
\end{pgfscope}%
\begin{pgfscope}%
\pgfsetbuttcap%
\pgfsetroundjoin%
\definecolor{currentfill}{rgb}{0.000000,0.000000,0.000000}%
\pgfsetfillcolor{currentfill}%
\pgfsetlinewidth{0.602250pt}%
\definecolor{currentstroke}{rgb}{0.000000,0.000000,0.000000}%
\pgfsetstrokecolor{currentstroke}%
\pgfsetdash{}{0pt}%
\pgfsys@defobject{currentmarker}{\pgfqpoint{-0.027778in}{0.000000in}}{\pgfqpoint{-0.000000in}{0.000000in}}{%
\pgfpathmoveto{\pgfqpoint{-0.000000in}{0.000000in}}%
\pgfpathlineto{\pgfqpoint{-0.027778in}{0.000000in}}%
\pgfusepath{stroke,fill}%
}%
\begin{pgfscope}%
\pgfsys@transformshift{0.800000in}{0.855752in}%
\pgfsys@useobject{currentmarker}{}%
\end{pgfscope}%
\end{pgfscope}%
\begin{pgfscope}%
\pgfsetbuttcap%
\pgfsetroundjoin%
\definecolor{currentfill}{rgb}{0.000000,0.000000,0.000000}%
\pgfsetfillcolor{currentfill}%
\pgfsetlinewidth{0.602250pt}%
\definecolor{currentstroke}{rgb}{0.000000,0.000000,0.000000}%
\pgfsetstrokecolor{currentstroke}%
\pgfsetdash{}{0pt}%
\pgfsys@defobject{currentmarker}{\pgfqpoint{-0.027778in}{0.000000in}}{\pgfqpoint{-0.000000in}{0.000000in}}{%
\pgfpathmoveto{\pgfqpoint{-0.000000in}{0.000000in}}%
\pgfpathlineto{\pgfqpoint{-0.027778in}{0.000000in}}%
\pgfusepath{stroke,fill}%
}%
\begin{pgfscope}%
\pgfsys@transformshift{0.800000in}{0.895786in}%
\pgfsys@useobject{currentmarker}{}%
\end{pgfscope}%
\end{pgfscope}%
\begin{pgfscope}%
\pgfsetbuttcap%
\pgfsetroundjoin%
\definecolor{currentfill}{rgb}{0.000000,0.000000,0.000000}%
\pgfsetfillcolor{currentfill}%
\pgfsetlinewidth{0.602250pt}%
\definecolor{currentstroke}{rgb}{0.000000,0.000000,0.000000}%
\pgfsetstrokecolor{currentstroke}%
\pgfsetdash{}{0pt}%
\pgfsys@defobject{currentmarker}{\pgfqpoint{-0.027778in}{0.000000in}}{\pgfqpoint{-0.000000in}{0.000000in}}{%
\pgfpathmoveto{\pgfqpoint{-0.000000in}{0.000000in}}%
\pgfpathlineto{\pgfqpoint{-0.027778in}{0.000000in}}%
\pgfusepath{stroke,fill}%
}%
\begin{pgfscope}%
\pgfsys@transformshift{0.800000in}{1.167195in}%
\pgfsys@useobject{currentmarker}{}%
\end{pgfscope}%
\end{pgfscope}%
\begin{pgfscope}%
\pgfsetbuttcap%
\pgfsetroundjoin%
\definecolor{currentfill}{rgb}{0.000000,0.000000,0.000000}%
\pgfsetfillcolor{currentfill}%
\pgfsetlinewidth{0.602250pt}%
\definecolor{currentstroke}{rgb}{0.000000,0.000000,0.000000}%
\pgfsetstrokecolor{currentstroke}%
\pgfsetdash{}{0pt}%
\pgfsys@defobject{currentmarker}{\pgfqpoint{-0.027778in}{0.000000in}}{\pgfqpoint{-0.000000in}{0.000000in}}{%
\pgfpathmoveto{\pgfqpoint{-0.000000in}{0.000000in}}%
\pgfpathlineto{\pgfqpoint{-0.027778in}{0.000000in}}%
\pgfusepath{stroke,fill}%
}%
\begin{pgfscope}%
\pgfsys@transformshift{0.800000in}{1.305010in}%
\pgfsys@useobject{currentmarker}{}%
\end{pgfscope}%
\end{pgfscope}%
\begin{pgfscope}%
\pgfsetbuttcap%
\pgfsetroundjoin%
\definecolor{currentfill}{rgb}{0.000000,0.000000,0.000000}%
\pgfsetfillcolor{currentfill}%
\pgfsetlinewidth{0.602250pt}%
\definecolor{currentstroke}{rgb}{0.000000,0.000000,0.000000}%
\pgfsetstrokecolor{currentstroke}%
\pgfsetdash{}{0pt}%
\pgfsys@defobject{currentmarker}{\pgfqpoint{-0.027778in}{0.000000in}}{\pgfqpoint{-0.000000in}{0.000000in}}{%
\pgfpathmoveto{\pgfqpoint{-0.000000in}{0.000000in}}%
\pgfpathlineto{\pgfqpoint{-0.027778in}{0.000000in}}%
\pgfusepath{stroke,fill}%
}%
\begin{pgfscope}%
\pgfsys@transformshift{0.800000in}{1.402792in}%
\pgfsys@useobject{currentmarker}{}%
\end{pgfscope}%
\end{pgfscope}%
\begin{pgfscope}%
\pgfsetbuttcap%
\pgfsetroundjoin%
\definecolor{currentfill}{rgb}{0.000000,0.000000,0.000000}%
\pgfsetfillcolor{currentfill}%
\pgfsetlinewidth{0.602250pt}%
\definecolor{currentstroke}{rgb}{0.000000,0.000000,0.000000}%
\pgfsetstrokecolor{currentstroke}%
\pgfsetdash{}{0pt}%
\pgfsys@defobject{currentmarker}{\pgfqpoint{-0.027778in}{0.000000in}}{\pgfqpoint{-0.000000in}{0.000000in}}{%
\pgfpathmoveto{\pgfqpoint{-0.000000in}{0.000000in}}%
\pgfpathlineto{\pgfqpoint{-0.027778in}{0.000000in}}%
\pgfusepath{stroke,fill}%
}%
\begin{pgfscope}%
\pgfsys@transformshift{0.800000in}{1.478637in}%
\pgfsys@useobject{currentmarker}{}%
\end{pgfscope}%
\end{pgfscope}%
\begin{pgfscope}%
\pgfsetbuttcap%
\pgfsetroundjoin%
\definecolor{currentfill}{rgb}{0.000000,0.000000,0.000000}%
\pgfsetfillcolor{currentfill}%
\pgfsetlinewidth{0.602250pt}%
\definecolor{currentstroke}{rgb}{0.000000,0.000000,0.000000}%
\pgfsetstrokecolor{currentstroke}%
\pgfsetdash{}{0pt}%
\pgfsys@defobject{currentmarker}{\pgfqpoint{-0.027778in}{0.000000in}}{\pgfqpoint{-0.000000in}{0.000000in}}{%
\pgfpathmoveto{\pgfqpoint{-0.000000in}{0.000000in}}%
\pgfpathlineto{\pgfqpoint{-0.027778in}{0.000000in}}%
\pgfusepath{stroke,fill}%
}%
\begin{pgfscope}%
\pgfsys@transformshift{0.800000in}{1.540608in}%
\pgfsys@useobject{currentmarker}{}%
\end{pgfscope}%
\end{pgfscope}%
\begin{pgfscope}%
\pgfsetbuttcap%
\pgfsetroundjoin%
\definecolor{currentfill}{rgb}{0.000000,0.000000,0.000000}%
\pgfsetfillcolor{currentfill}%
\pgfsetlinewidth{0.602250pt}%
\definecolor{currentstroke}{rgb}{0.000000,0.000000,0.000000}%
\pgfsetstrokecolor{currentstroke}%
\pgfsetdash{}{0pt}%
\pgfsys@defobject{currentmarker}{\pgfqpoint{-0.027778in}{0.000000in}}{\pgfqpoint{-0.000000in}{0.000000in}}{%
\pgfpathmoveto{\pgfqpoint{-0.000000in}{0.000000in}}%
\pgfpathlineto{\pgfqpoint{-0.027778in}{0.000000in}}%
\pgfusepath{stroke,fill}%
}%
\begin{pgfscope}%
\pgfsys@transformshift{0.800000in}{1.593003in}%
\pgfsys@useobject{currentmarker}{}%
\end{pgfscope}%
\end{pgfscope}%
\begin{pgfscope}%
\pgfsetbuttcap%
\pgfsetroundjoin%
\definecolor{currentfill}{rgb}{0.000000,0.000000,0.000000}%
\pgfsetfillcolor{currentfill}%
\pgfsetlinewidth{0.602250pt}%
\definecolor{currentstroke}{rgb}{0.000000,0.000000,0.000000}%
\pgfsetstrokecolor{currentstroke}%
\pgfsetdash{}{0pt}%
\pgfsys@defobject{currentmarker}{\pgfqpoint{-0.027778in}{0.000000in}}{\pgfqpoint{-0.000000in}{0.000000in}}{%
\pgfpathmoveto{\pgfqpoint{-0.000000in}{0.000000in}}%
\pgfpathlineto{\pgfqpoint{-0.027778in}{0.000000in}}%
\pgfusepath{stroke,fill}%
}%
\begin{pgfscope}%
\pgfsys@transformshift{0.800000in}{1.638389in}%
\pgfsys@useobject{currentmarker}{}%
\end{pgfscope}%
\end{pgfscope}%
\begin{pgfscope}%
\pgfsetbuttcap%
\pgfsetroundjoin%
\definecolor{currentfill}{rgb}{0.000000,0.000000,0.000000}%
\pgfsetfillcolor{currentfill}%
\pgfsetlinewidth{0.602250pt}%
\definecolor{currentstroke}{rgb}{0.000000,0.000000,0.000000}%
\pgfsetstrokecolor{currentstroke}%
\pgfsetdash{}{0pt}%
\pgfsys@defobject{currentmarker}{\pgfqpoint{-0.027778in}{0.000000in}}{\pgfqpoint{-0.000000in}{0.000000in}}{%
\pgfpathmoveto{\pgfqpoint{-0.000000in}{0.000000in}}%
\pgfpathlineto{\pgfqpoint{-0.027778in}{0.000000in}}%
\pgfusepath{stroke,fill}%
}%
\begin{pgfscope}%
\pgfsys@transformshift{0.800000in}{1.678423in}%
\pgfsys@useobject{currentmarker}{}%
\end{pgfscope}%
\end{pgfscope}%
\begin{pgfscope}%
\pgfsetbuttcap%
\pgfsetroundjoin%
\definecolor{currentfill}{rgb}{0.000000,0.000000,0.000000}%
\pgfsetfillcolor{currentfill}%
\pgfsetlinewidth{0.602250pt}%
\definecolor{currentstroke}{rgb}{0.000000,0.000000,0.000000}%
\pgfsetstrokecolor{currentstroke}%
\pgfsetdash{}{0pt}%
\pgfsys@defobject{currentmarker}{\pgfqpoint{-0.027778in}{0.000000in}}{\pgfqpoint{-0.000000in}{0.000000in}}{%
\pgfpathmoveto{\pgfqpoint{-0.000000in}{0.000000in}}%
\pgfpathlineto{\pgfqpoint{-0.027778in}{0.000000in}}%
\pgfusepath{stroke,fill}%
}%
\begin{pgfscope}%
\pgfsys@transformshift{0.800000in}{1.949832in}%
\pgfsys@useobject{currentmarker}{}%
\end{pgfscope}%
\end{pgfscope}%
\begin{pgfscope}%
\pgfsetbuttcap%
\pgfsetroundjoin%
\definecolor{currentfill}{rgb}{0.000000,0.000000,0.000000}%
\pgfsetfillcolor{currentfill}%
\pgfsetlinewidth{0.602250pt}%
\definecolor{currentstroke}{rgb}{0.000000,0.000000,0.000000}%
\pgfsetstrokecolor{currentstroke}%
\pgfsetdash{}{0pt}%
\pgfsys@defobject{currentmarker}{\pgfqpoint{-0.027778in}{0.000000in}}{\pgfqpoint{-0.000000in}{0.000000in}}{%
\pgfpathmoveto{\pgfqpoint{-0.000000in}{0.000000in}}%
\pgfpathlineto{\pgfqpoint{-0.027778in}{0.000000in}}%
\pgfusepath{stroke,fill}%
}%
\begin{pgfscope}%
\pgfsys@transformshift{0.800000in}{2.087648in}%
\pgfsys@useobject{currentmarker}{}%
\end{pgfscope}%
\end{pgfscope}%
\begin{pgfscope}%
\pgfsetbuttcap%
\pgfsetroundjoin%
\definecolor{currentfill}{rgb}{0.000000,0.000000,0.000000}%
\pgfsetfillcolor{currentfill}%
\pgfsetlinewidth{0.602250pt}%
\definecolor{currentstroke}{rgb}{0.000000,0.000000,0.000000}%
\pgfsetstrokecolor{currentstroke}%
\pgfsetdash{}{0pt}%
\pgfsys@defobject{currentmarker}{\pgfqpoint{-0.027778in}{0.000000in}}{\pgfqpoint{-0.000000in}{0.000000in}}{%
\pgfpathmoveto{\pgfqpoint{-0.000000in}{0.000000in}}%
\pgfpathlineto{\pgfqpoint{-0.027778in}{0.000000in}}%
\pgfusepath{stroke,fill}%
}%
\begin{pgfscope}%
\pgfsys@transformshift{0.800000in}{2.185429in}%
\pgfsys@useobject{currentmarker}{}%
\end{pgfscope}%
\end{pgfscope}%
\begin{pgfscope}%
\pgfsetbuttcap%
\pgfsetroundjoin%
\definecolor{currentfill}{rgb}{0.000000,0.000000,0.000000}%
\pgfsetfillcolor{currentfill}%
\pgfsetlinewidth{0.602250pt}%
\definecolor{currentstroke}{rgb}{0.000000,0.000000,0.000000}%
\pgfsetstrokecolor{currentstroke}%
\pgfsetdash{}{0pt}%
\pgfsys@defobject{currentmarker}{\pgfqpoint{-0.027778in}{0.000000in}}{\pgfqpoint{-0.000000in}{0.000000in}}{%
\pgfpathmoveto{\pgfqpoint{-0.000000in}{0.000000in}}%
\pgfpathlineto{\pgfqpoint{-0.027778in}{0.000000in}}%
\pgfusepath{stroke,fill}%
}%
\begin{pgfscope}%
\pgfsys@transformshift{0.800000in}{2.261275in}%
\pgfsys@useobject{currentmarker}{}%
\end{pgfscope}%
\end{pgfscope}%
\begin{pgfscope}%
\pgfsetbuttcap%
\pgfsetroundjoin%
\definecolor{currentfill}{rgb}{0.000000,0.000000,0.000000}%
\pgfsetfillcolor{currentfill}%
\pgfsetlinewidth{0.602250pt}%
\definecolor{currentstroke}{rgb}{0.000000,0.000000,0.000000}%
\pgfsetstrokecolor{currentstroke}%
\pgfsetdash{}{0pt}%
\pgfsys@defobject{currentmarker}{\pgfqpoint{-0.027778in}{0.000000in}}{\pgfqpoint{-0.000000in}{0.000000in}}{%
\pgfpathmoveto{\pgfqpoint{-0.000000in}{0.000000in}}%
\pgfpathlineto{\pgfqpoint{-0.027778in}{0.000000in}}%
\pgfusepath{stroke,fill}%
}%
\begin{pgfscope}%
\pgfsys@transformshift{0.800000in}{2.323245in}%
\pgfsys@useobject{currentmarker}{}%
\end{pgfscope}%
\end{pgfscope}%
\begin{pgfscope}%
\pgfsetbuttcap%
\pgfsetroundjoin%
\definecolor{currentfill}{rgb}{0.000000,0.000000,0.000000}%
\pgfsetfillcolor{currentfill}%
\pgfsetlinewidth{0.602250pt}%
\definecolor{currentstroke}{rgb}{0.000000,0.000000,0.000000}%
\pgfsetstrokecolor{currentstroke}%
\pgfsetdash{}{0pt}%
\pgfsys@defobject{currentmarker}{\pgfqpoint{-0.027778in}{0.000000in}}{\pgfqpoint{-0.000000in}{0.000000in}}{%
\pgfpathmoveto{\pgfqpoint{-0.000000in}{0.000000in}}%
\pgfpathlineto{\pgfqpoint{-0.027778in}{0.000000in}}%
\pgfusepath{stroke,fill}%
}%
\begin{pgfscope}%
\pgfsys@transformshift{0.800000in}{2.375640in}%
\pgfsys@useobject{currentmarker}{}%
\end{pgfscope}%
\end{pgfscope}%
\begin{pgfscope}%
\pgfsetbuttcap%
\pgfsetroundjoin%
\definecolor{currentfill}{rgb}{0.000000,0.000000,0.000000}%
\pgfsetfillcolor{currentfill}%
\pgfsetlinewidth{0.602250pt}%
\definecolor{currentstroke}{rgb}{0.000000,0.000000,0.000000}%
\pgfsetstrokecolor{currentstroke}%
\pgfsetdash{}{0pt}%
\pgfsys@defobject{currentmarker}{\pgfqpoint{-0.027778in}{0.000000in}}{\pgfqpoint{-0.000000in}{0.000000in}}{%
\pgfpathmoveto{\pgfqpoint{-0.000000in}{0.000000in}}%
\pgfpathlineto{\pgfqpoint{-0.027778in}{0.000000in}}%
\pgfusepath{stroke,fill}%
}%
\begin{pgfscope}%
\pgfsys@transformshift{0.800000in}{2.421027in}%
\pgfsys@useobject{currentmarker}{}%
\end{pgfscope}%
\end{pgfscope}%
\begin{pgfscope}%
\pgfsetbuttcap%
\pgfsetroundjoin%
\definecolor{currentfill}{rgb}{0.000000,0.000000,0.000000}%
\pgfsetfillcolor{currentfill}%
\pgfsetlinewidth{0.602250pt}%
\definecolor{currentstroke}{rgb}{0.000000,0.000000,0.000000}%
\pgfsetstrokecolor{currentstroke}%
\pgfsetdash{}{0pt}%
\pgfsys@defobject{currentmarker}{\pgfqpoint{-0.027778in}{0.000000in}}{\pgfqpoint{-0.000000in}{0.000000in}}{%
\pgfpathmoveto{\pgfqpoint{-0.000000in}{0.000000in}}%
\pgfpathlineto{\pgfqpoint{-0.027778in}{0.000000in}}%
\pgfusepath{stroke,fill}%
}%
\begin{pgfscope}%
\pgfsys@transformshift{0.800000in}{2.461061in}%
\pgfsys@useobject{currentmarker}{}%
\end{pgfscope}%
\end{pgfscope}%
\begin{pgfscope}%
\pgfsetbuttcap%
\pgfsetroundjoin%
\definecolor{currentfill}{rgb}{0.000000,0.000000,0.000000}%
\pgfsetfillcolor{currentfill}%
\pgfsetlinewidth{0.602250pt}%
\definecolor{currentstroke}{rgb}{0.000000,0.000000,0.000000}%
\pgfsetstrokecolor{currentstroke}%
\pgfsetdash{}{0pt}%
\pgfsys@defobject{currentmarker}{\pgfqpoint{-0.027778in}{0.000000in}}{\pgfqpoint{-0.000000in}{0.000000in}}{%
\pgfpathmoveto{\pgfqpoint{-0.000000in}{0.000000in}}%
\pgfpathlineto{\pgfqpoint{-0.027778in}{0.000000in}}%
\pgfusepath{stroke,fill}%
}%
\begin{pgfscope}%
\pgfsys@transformshift{0.800000in}{2.732469in}%
\pgfsys@useobject{currentmarker}{}%
\end{pgfscope}%
\end{pgfscope}%
\begin{pgfscope}%
\pgfsetbuttcap%
\pgfsetroundjoin%
\definecolor{currentfill}{rgb}{0.000000,0.000000,0.000000}%
\pgfsetfillcolor{currentfill}%
\pgfsetlinewidth{0.602250pt}%
\definecolor{currentstroke}{rgb}{0.000000,0.000000,0.000000}%
\pgfsetstrokecolor{currentstroke}%
\pgfsetdash{}{0pt}%
\pgfsys@defobject{currentmarker}{\pgfqpoint{-0.027778in}{0.000000in}}{\pgfqpoint{-0.000000in}{0.000000in}}{%
\pgfpathmoveto{\pgfqpoint{-0.000000in}{0.000000in}}%
\pgfpathlineto{\pgfqpoint{-0.027778in}{0.000000in}}%
\pgfusepath{stroke,fill}%
}%
\begin{pgfscope}%
\pgfsys@transformshift{0.800000in}{2.870285in}%
\pgfsys@useobject{currentmarker}{}%
\end{pgfscope}%
\end{pgfscope}%
\begin{pgfscope}%
\pgfsetbuttcap%
\pgfsetroundjoin%
\definecolor{currentfill}{rgb}{0.000000,0.000000,0.000000}%
\pgfsetfillcolor{currentfill}%
\pgfsetlinewidth{0.602250pt}%
\definecolor{currentstroke}{rgb}{0.000000,0.000000,0.000000}%
\pgfsetstrokecolor{currentstroke}%
\pgfsetdash{}{0pt}%
\pgfsys@defobject{currentmarker}{\pgfqpoint{-0.027778in}{0.000000in}}{\pgfqpoint{-0.000000in}{0.000000in}}{%
\pgfpathmoveto{\pgfqpoint{-0.000000in}{0.000000in}}%
\pgfpathlineto{\pgfqpoint{-0.027778in}{0.000000in}}%
\pgfusepath{stroke,fill}%
}%
\begin{pgfscope}%
\pgfsys@transformshift{0.800000in}{2.968067in}%
\pgfsys@useobject{currentmarker}{}%
\end{pgfscope}%
\end{pgfscope}%
\begin{pgfscope}%
\pgfsetbuttcap%
\pgfsetroundjoin%
\definecolor{currentfill}{rgb}{0.000000,0.000000,0.000000}%
\pgfsetfillcolor{currentfill}%
\pgfsetlinewidth{0.602250pt}%
\definecolor{currentstroke}{rgb}{0.000000,0.000000,0.000000}%
\pgfsetstrokecolor{currentstroke}%
\pgfsetdash{}{0pt}%
\pgfsys@defobject{currentmarker}{\pgfqpoint{-0.027778in}{0.000000in}}{\pgfqpoint{-0.000000in}{0.000000in}}{%
\pgfpathmoveto{\pgfqpoint{-0.000000in}{0.000000in}}%
\pgfpathlineto{\pgfqpoint{-0.027778in}{0.000000in}}%
\pgfusepath{stroke,fill}%
}%
\begin{pgfscope}%
\pgfsys@transformshift{0.800000in}{3.043912in}%
\pgfsys@useobject{currentmarker}{}%
\end{pgfscope}%
\end{pgfscope}%
\begin{pgfscope}%
\pgfsetbuttcap%
\pgfsetroundjoin%
\definecolor{currentfill}{rgb}{0.000000,0.000000,0.000000}%
\pgfsetfillcolor{currentfill}%
\pgfsetlinewidth{0.602250pt}%
\definecolor{currentstroke}{rgb}{0.000000,0.000000,0.000000}%
\pgfsetstrokecolor{currentstroke}%
\pgfsetdash{}{0pt}%
\pgfsys@defobject{currentmarker}{\pgfqpoint{-0.027778in}{0.000000in}}{\pgfqpoint{-0.000000in}{0.000000in}}{%
\pgfpathmoveto{\pgfqpoint{-0.000000in}{0.000000in}}%
\pgfpathlineto{\pgfqpoint{-0.027778in}{0.000000in}}%
\pgfusepath{stroke,fill}%
}%
\begin{pgfscope}%
\pgfsys@transformshift{0.800000in}{3.105882in}%
\pgfsys@useobject{currentmarker}{}%
\end{pgfscope}%
\end{pgfscope}%
\begin{pgfscope}%
\pgfsetbuttcap%
\pgfsetroundjoin%
\definecolor{currentfill}{rgb}{0.000000,0.000000,0.000000}%
\pgfsetfillcolor{currentfill}%
\pgfsetlinewidth{0.602250pt}%
\definecolor{currentstroke}{rgb}{0.000000,0.000000,0.000000}%
\pgfsetstrokecolor{currentstroke}%
\pgfsetdash{}{0pt}%
\pgfsys@defobject{currentmarker}{\pgfqpoint{-0.027778in}{0.000000in}}{\pgfqpoint{-0.000000in}{0.000000in}}{%
\pgfpathmoveto{\pgfqpoint{-0.000000in}{0.000000in}}%
\pgfpathlineto{\pgfqpoint{-0.027778in}{0.000000in}}%
\pgfusepath{stroke,fill}%
}%
\begin{pgfscope}%
\pgfsys@transformshift{0.800000in}{3.158277in}%
\pgfsys@useobject{currentmarker}{}%
\end{pgfscope}%
\end{pgfscope}%
\begin{pgfscope}%
\pgfsetbuttcap%
\pgfsetroundjoin%
\definecolor{currentfill}{rgb}{0.000000,0.000000,0.000000}%
\pgfsetfillcolor{currentfill}%
\pgfsetlinewidth{0.602250pt}%
\definecolor{currentstroke}{rgb}{0.000000,0.000000,0.000000}%
\pgfsetstrokecolor{currentstroke}%
\pgfsetdash{}{0pt}%
\pgfsys@defobject{currentmarker}{\pgfqpoint{-0.027778in}{0.000000in}}{\pgfqpoint{-0.000000in}{0.000000in}}{%
\pgfpathmoveto{\pgfqpoint{-0.000000in}{0.000000in}}%
\pgfpathlineto{\pgfqpoint{-0.027778in}{0.000000in}}%
\pgfusepath{stroke,fill}%
}%
\begin{pgfscope}%
\pgfsys@transformshift{0.800000in}{3.203664in}%
\pgfsys@useobject{currentmarker}{}%
\end{pgfscope}%
\end{pgfscope}%
\begin{pgfscope}%
\pgfsetbuttcap%
\pgfsetroundjoin%
\definecolor{currentfill}{rgb}{0.000000,0.000000,0.000000}%
\pgfsetfillcolor{currentfill}%
\pgfsetlinewidth{0.602250pt}%
\definecolor{currentstroke}{rgb}{0.000000,0.000000,0.000000}%
\pgfsetstrokecolor{currentstroke}%
\pgfsetdash{}{0pt}%
\pgfsys@defobject{currentmarker}{\pgfqpoint{-0.027778in}{0.000000in}}{\pgfqpoint{-0.000000in}{0.000000in}}{%
\pgfpathmoveto{\pgfqpoint{-0.000000in}{0.000000in}}%
\pgfpathlineto{\pgfqpoint{-0.027778in}{0.000000in}}%
\pgfusepath{stroke,fill}%
}%
\begin{pgfscope}%
\pgfsys@transformshift{0.800000in}{3.243698in}%
\pgfsys@useobject{currentmarker}{}%
\end{pgfscope}%
\end{pgfscope}%
\begin{pgfscope}%
\pgfsetbuttcap%
\pgfsetroundjoin%
\definecolor{currentfill}{rgb}{0.000000,0.000000,0.000000}%
\pgfsetfillcolor{currentfill}%
\pgfsetlinewidth{0.602250pt}%
\definecolor{currentstroke}{rgb}{0.000000,0.000000,0.000000}%
\pgfsetstrokecolor{currentstroke}%
\pgfsetdash{}{0pt}%
\pgfsys@defobject{currentmarker}{\pgfqpoint{-0.027778in}{0.000000in}}{\pgfqpoint{-0.000000in}{0.000000in}}{%
\pgfpathmoveto{\pgfqpoint{-0.000000in}{0.000000in}}%
\pgfpathlineto{\pgfqpoint{-0.027778in}{0.000000in}}%
\pgfusepath{stroke,fill}%
}%
\begin{pgfscope}%
\pgfsys@transformshift{0.800000in}{3.515107in}%
\pgfsys@useobject{currentmarker}{}%
\end{pgfscope}%
\end{pgfscope}%
\begin{pgfscope}%
\pgfsetbuttcap%
\pgfsetroundjoin%
\definecolor{currentfill}{rgb}{0.000000,0.000000,0.000000}%
\pgfsetfillcolor{currentfill}%
\pgfsetlinewidth{0.602250pt}%
\definecolor{currentstroke}{rgb}{0.000000,0.000000,0.000000}%
\pgfsetstrokecolor{currentstroke}%
\pgfsetdash{}{0pt}%
\pgfsys@defobject{currentmarker}{\pgfqpoint{-0.027778in}{0.000000in}}{\pgfqpoint{-0.000000in}{0.000000in}}{%
\pgfpathmoveto{\pgfqpoint{-0.000000in}{0.000000in}}%
\pgfpathlineto{\pgfqpoint{-0.027778in}{0.000000in}}%
\pgfusepath{stroke,fill}%
}%
\begin{pgfscope}%
\pgfsys@transformshift{0.800000in}{3.652922in}%
\pgfsys@useobject{currentmarker}{}%
\end{pgfscope}%
\end{pgfscope}%
\begin{pgfscope}%
\pgfsetbuttcap%
\pgfsetroundjoin%
\definecolor{currentfill}{rgb}{0.000000,0.000000,0.000000}%
\pgfsetfillcolor{currentfill}%
\pgfsetlinewidth{0.602250pt}%
\definecolor{currentstroke}{rgb}{0.000000,0.000000,0.000000}%
\pgfsetstrokecolor{currentstroke}%
\pgfsetdash{}{0pt}%
\pgfsys@defobject{currentmarker}{\pgfqpoint{-0.027778in}{0.000000in}}{\pgfqpoint{-0.000000in}{0.000000in}}{%
\pgfpathmoveto{\pgfqpoint{-0.000000in}{0.000000in}}%
\pgfpathlineto{\pgfqpoint{-0.027778in}{0.000000in}}%
\pgfusepath{stroke,fill}%
}%
\begin{pgfscope}%
\pgfsys@transformshift{0.800000in}{3.750704in}%
\pgfsys@useobject{currentmarker}{}%
\end{pgfscope}%
\end{pgfscope}%
\begin{pgfscope}%
\pgfsetbuttcap%
\pgfsetroundjoin%
\definecolor{currentfill}{rgb}{0.000000,0.000000,0.000000}%
\pgfsetfillcolor{currentfill}%
\pgfsetlinewidth{0.602250pt}%
\definecolor{currentstroke}{rgb}{0.000000,0.000000,0.000000}%
\pgfsetstrokecolor{currentstroke}%
\pgfsetdash{}{0pt}%
\pgfsys@defobject{currentmarker}{\pgfqpoint{-0.027778in}{0.000000in}}{\pgfqpoint{-0.000000in}{0.000000in}}{%
\pgfpathmoveto{\pgfqpoint{-0.000000in}{0.000000in}}%
\pgfpathlineto{\pgfqpoint{-0.027778in}{0.000000in}}%
\pgfusepath{stroke,fill}%
}%
\begin{pgfscope}%
\pgfsys@transformshift{0.800000in}{3.826550in}%
\pgfsys@useobject{currentmarker}{}%
\end{pgfscope}%
\end{pgfscope}%
\begin{pgfscope}%
\pgfsetbuttcap%
\pgfsetroundjoin%
\definecolor{currentfill}{rgb}{0.000000,0.000000,0.000000}%
\pgfsetfillcolor{currentfill}%
\pgfsetlinewidth{0.602250pt}%
\definecolor{currentstroke}{rgb}{0.000000,0.000000,0.000000}%
\pgfsetstrokecolor{currentstroke}%
\pgfsetdash{}{0pt}%
\pgfsys@defobject{currentmarker}{\pgfqpoint{-0.027778in}{0.000000in}}{\pgfqpoint{-0.000000in}{0.000000in}}{%
\pgfpathmoveto{\pgfqpoint{-0.000000in}{0.000000in}}%
\pgfpathlineto{\pgfqpoint{-0.027778in}{0.000000in}}%
\pgfusepath{stroke,fill}%
}%
\begin{pgfscope}%
\pgfsys@transformshift{0.800000in}{3.888520in}%
\pgfsys@useobject{currentmarker}{}%
\end{pgfscope}%
\end{pgfscope}%
\begin{pgfscope}%
\pgfsetbuttcap%
\pgfsetroundjoin%
\definecolor{currentfill}{rgb}{0.000000,0.000000,0.000000}%
\pgfsetfillcolor{currentfill}%
\pgfsetlinewidth{0.602250pt}%
\definecolor{currentstroke}{rgb}{0.000000,0.000000,0.000000}%
\pgfsetstrokecolor{currentstroke}%
\pgfsetdash{}{0pt}%
\pgfsys@defobject{currentmarker}{\pgfqpoint{-0.027778in}{0.000000in}}{\pgfqpoint{-0.000000in}{0.000000in}}{%
\pgfpathmoveto{\pgfqpoint{-0.000000in}{0.000000in}}%
\pgfpathlineto{\pgfqpoint{-0.027778in}{0.000000in}}%
\pgfusepath{stroke,fill}%
}%
\begin{pgfscope}%
\pgfsys@transformshift{0.800000in}{3.940915in}%
\pgfsys@useobject{currentmarker}{}%
\end{pgfscope}%
\end{pgfscope}%
\begin{pgfscope}%
\pgfsetbuttcap%
\pgfsetroundjoin%
\definecolor{currentfill}{rgb}{0.000000,0.000000,0.000000}%
\pgfsetfillcolor{currentfill}%
\pgfsetlinewidth{0.602250pt}%
\definecolor{currentstroke}{rgb}{0.000000,0.000000,0.000000}%
\pgfsetstrokecolor{currentstroke}%
\pgfsetdash{}{0pt}%
\pgfsys@defobject{currentmarker}{\pgfqpoint{-0.027778in}{0.000000in}}{\pgfqpoint{-0.000000in}{0.000000in}}{%
\pgfpathmoveto{\pgfqpoint{-0.000000in}{0.000000in}}%
\pgfpathlineto{\pgfqpoint{-0.027778in}{0.000000in}}%
\pgfusepath{stroke,fill}%
}%
\begin{pgfscope}%
\pgfsys@transformshift{0.800000in}{3.986301in}%
\pgfsys@useobject{currentmarker}{}%
\end{pgfscope}%
\end{pgfscope}%
\begin{pgfscope}%
\pgfsetbuttcap%
\pgfsetroundjoin%
\definecolor{currentfill}{rgb}{0.000000,0.000000,0.000000}%
\pgfsetfillcolor{currentfill}%
\pgfsetlinewidth{0.602250pt}%
\definecolor{currentstroke}{rgb}{0.000000,0.000000,0.000000}%
\pgfsetstrokecolor{currentstroke}%
\pgfsetdash{}{0pt}%
\pgfsys@defobject{currentmarker}{\pgfqpoint{-0.027778in}{0.000000in}}{\pgfqpoint{-0.000000in}{0.000000in}}{%
\pgfpathmoveto{\pgfqpoint{-0.000000in}{0.000000in}}%
\pgfpathlineto{\pgfqpoint{-0.027778in}{0.000000in}}%
\pgfusepath{stroke,fill}%
}%
\begin{pgfscope}%
\pgfsys@transformshift{0.800000in}{4.026335in}%
\pgfsys@useobject{currentmarker}{}%
\end{pgfscope}%
\end{pgfscope}%
\begin{pgfscope}%
\definecolor{textcolor}{rgb}{0.000000,0.000000,0.000000}%
\pgfsetstrokecolor{textcolor}%
\pgfsetfillcolor{textcolor}%
\pgftext[x=0.446026in,y=2.376000in,,bottom,rotate=90.000000]{\color{textcolor}\rmfamily\fontsize{10.000000}{12.000000}\selectfont Comparisons}%
\end{pgfscope}%
\begin{pgfscope}%
\pgfpathrectangle{\pgfqpoint{0.800000in}{0.528000in}}{\pgfqpoint{4.960000in}{3.696000in}}%
\pgfusepath{clip}%
\pgfsetrectcap%
\pgfsetroundjoin%
\pgfsetlinewidth{1.505625pt}%
\definecolor{currentstroke}{rgb}{0.121569,0.466667,0.705882}%
\pgfsetstrokecolor{currentstroke}%
\pgfsetdash{}{0pt}%
\pgfpathmoveto{\pgfqpoint{1.025455in}{0.987855in}}%
\pgfpathlineto{\pgfqpoint{1.071466in}{1.433639in}}%
\pgfpathlineto{\pgfqpoint{1.117477in}{1.699651in}}%
\pgfpathlineto{\pgfqpoint{1.163488in}{1.890165in}}%
\pgfpathlineto{\pgfqpoint{1.209499in}{2.038747in}}%
\pgfpathlineto{\pgfqpoint{1.255510in}{2.160580in}}%
\pgfpathlineto{\pgfqpoint{1.301521in}{2.263848in}}%
\pgfpathlineto{\pgfqpoint{1.347532in}{2.353471in}}%
\pgfpathlineto{\pgfqpoint{1.393544in}{2.432637in}}%
\pgfpathlineto{\pgfqpoint{1.439555in}{2.503536in}}%
\pgfpathlineto{\pgfqpoint{1.485566in}{2.567732in}}%
\pgfpathlineto{\pgfqpoint{1.531577in}{2.626384in}}%
\pgfpathlineto{\pgfqpoint{1.577588in}{2.680374in}}%
\pgfpathlineto{\pgfqpoint{1.623599in}{2.730390in}}%
\pgfpathlineto{\pgfqpoint{1.669610in}{2.776975in}}%
\pgfpathlineto{\pgfqpoint{1.715622in}{2.820572in}}%
\pgfpathlineto{\pgfqpoint{1.761633in}{2.861540in}}%
\pgfpathlineto{\pgfqpoint{1.807644in}{2.900179in}}%
\pgfpathlineto{\pgfqpoint{1.853655in}{2.936739in}}%
\pgfpathlineto{\pgfqpoint{1.899666in}{2.971432in}}%
\pgfpathlineto{\pgfqpoint{1.945677in}{3.004440in}}%
\pgfpathlineto{\pgfqpoint{1.991688in}{3.035920in}}%
\pgfpathlineto{\pgfqpoint{2.037699in}{3.066006in}}%
\pgfpathlineto{\pgfqpoint{2.083711in}{3.094816in}}%
\pgfpathlineto{\pgfqpoint{2.129722in}{3.122455in}}%
\pgfpathlineto{\pgfqpoint{2.175733in}{3.149014in}}%
\pgfpathlineto{\pgfqpoint{2.221744in}{3.174574in}}%
\pgfpathlineto{\pgfqpoint{2.267755in}{3.199208in}}%
\pgfpathlineto{\pgfqpoint{2.313766in}{3.222980in}}%
\pgfpathlineto{\pgfqpoint{2.359777in}{3.245949in}}%
\pgfpathlineto{\pgfqpoint{2.405788in}{3.268167in}}%
\pgfpathlineto{\pgfqpoint{2.451800in}{3.289682in}}%
\pgfpathlineto{\pgfqpoint{2.497811in}{3.310536in}}%
\pgfpathlineto{\pgfqpoint{2.543822in}{3.330770in}}%
\pgfpathlineto{\pgfqpoint{2.589833in}{3.350419in}}%
\pgfpathlineto{\pgfqpoint{2.635844in}{3.369516in}}%
\pgfpathlineto{\pgfqpoint{2.681855in}{3.388091in}}%
\pgfpathlineto{\pgfqpoint{2.727866in}{3.406172in}}%
\pgfpathlineto{\pgfqpoint{2.773878in}{3.423785in}}%
\pgfpathlineto{\pgfqpoint{2.819889in}{3.440952in}}%
\pgfpathlineto{\pgfqpoint{2.865900in}{3.457697in}}%
\pgfpathlineto{\pgfqpoint{2.911911in}{3.474039in}}%
\pgfpathlineto{\pgfqpoint{2.957922in}{3.489998in}}%
\pgfpathlineto{\pgfqpoint{3.003933in}{3.505590in}}%
\pgfpathlineto{\pgfqpoint{3.049944in}{3.520833in}}%
\pgfpathlineto{\pgfqpoint{3.095955in}{3.535742in}}%
\pgfpathlineto{\pgfqpoint{3.141967in}{3.550330in}}%
\pgfpathlineto{\pgfqpoint{3.187978in}{3.564612in}}%
\pgfpathlineto{\pgfqpoint{3.233989in}{3.578600in}}%
\pgfpathlineto{\pgfqpoint{3.280000in}{3.592306in}}%
\pgfpathlineto{\pgfqpoint{3.326011in}{3.605742in}}%
\pgfpathlineto{\pgfqpoint{3.372022in}{3.618916in}}%
\pgfpathlineto{\pgfqpoint{3.418033in}{3.631841in}}%
\pgfpathlineto{\pgfqpoint{3.464045in}{3.644524in}}%
\pgfpathlineto{\pgfqpoint{3.510056in}{3.656975in}}%
\pgfpathlineto{\pgfqpoint{3.556067in}{3.669202in}}%
\pgfpathlineto{\pgfqpoint{3.602078in}{3.681212in}}%
\pgfpathlineto{\pgfqpoint{3.648089in}{3.693015in}}%
\pgfpathlineto{\pgfqpoint{3.694100in}{3.704616in}}%
\pgfpathlineto{\pgfqpoint{3.740111in}{3.716022in}}%
\pgfpathlineto{\pgfqpoint{3.786122in}{3.727240in}}%
\pgfpathlineto{\pgfqpoint{3.832134in}{3.738276in}}%
\pgfpathlineto{\pgfqpoint{3.878145in}{3.749135in}}%
\pgfpathlineto{\pgfqpoint{3.924156in}{3.759824in}}%
\pgfpathlineto{\pgfqpoint{3.970167in}{3.770347in}}%
\pgfpathlineto{\pgfqpoint{4.016178in}{3.780710in}}%
\pgfpathlineto{\pgfqpoint{4.062189in}{3.790918in}}%
\pgfpathlineto{\pgfqpoint{4.108200in}{3.800974in}}%
\pgfpathlineto{\pgfqpoint{4.154212in}{3.810884in}}%
\pgfpathlineto{\pgfqpoint{4.200223in}{3.820651in}}%
\pgfpathlineto{\pgfqpoint{4.246234in}{3.830280in}}%
\pgfpathlineto{\pgfqpoint{4.292245in}{3.839775in}}%
\pgfpathlineto{\pgfqpoint{4.338256in}{3.849138in}}%
\pgfpathlineto{\pgfqpoint{4.384267in}{3.858375in}}%
\pgfpathlineto{\pgfqpoint{4.430278in}{3.867487in}}%
\pgfpathlineto{\pgfqpoint{4.476289in}{3.876479in}}%
\pgfpathlineto{\pgfqpoint{4.522301in}{3.885354in}}%
\pgfpathlineto{\pgfqpoint{4.568312in}{3.894115in}}%
\pgfpathlineto{\pgfqpoint{4.614323in}{3.902763in}}%
\pgfpathlineto{\pgfqpoint{4.660334in}{3.911304in}}%
\pgfpathlineto{\pgfqpoint{4.706345in}{3.919738in}}%
\pgfpathlineto{\pgfqpoint{4.752356in}{3.928069in}}%
\pgfpathlineto{\pgfqpoint{4.798367in}{3.936299in}}%
\pgfpathlineto{\pgfqpoint{4.844378in}{3.944430in}}%
\pgfpathlineto{\pgfqpoint{4.890390in}{3.952466in}}%
\pgfpathlineto{\pgfqpoint{4.936401in}{3.960408in}}%
\pgfpathlineto{\pgfqpoint{4.982412in}{3.968257in}}%
\pgfpathlineto{\pgfqpoint{5.028423in}{3.976018in}}%
\pgfpathlineto{\pgfqpoint{5.074434in}{3.983690in}}%
\pgfpathlineto{\pgfqpoint{5.120445in}{3.991277in}}%
\pgfpathlineto{\pgfqpoint{5.166456in}{3.998781in}}%
\pgfpathlineto{\pgfqpoint{5.212468in}{4.006202in}}%
\pgfpathlineto{\pgfqpoint{5.258479in}{4.013543in}}%
\pgfpathlineto{\pgfqpoint{5.304490in}{4.020806in}}%
\pgfpathlineto{\pgfqpoint{5.350501in}{4.027992in}}%
\pgfpathlineto{\pgfqpoint{5.396512in}{4.035103in}}%
\pgfpathlineto{\pgfqpoint{5.442523in}{4.042140in}}%
\pgfpathlineto{\pgfqpoint{5.488534in}{4.049105in}}%
\pgfpathlineto{\pgfqpoint{5.534545in}{4.056000in}}%
\pgfusepath{stroke}%
\end{pgfscope}%
\begin{pgfscope}%
\pgfpathrectangle{\pgfqpoint{0.800000in}{0.528000in}}{\pgfqpoint{4.960000in}{3.696000in}}%
\pgfusepath{clip}%
\pgfsetrectcap%
\pgfsetroundjoin%
\pgfsetlinewidth{1.505625pt}%
\definecolor{currentstroke}{rgb}{1.000000,0.498039,0.054902}%
\pgfsetstrokecolor{currentstroke}%
\pgfsetdash{}{0pt}%
\pgfpathmoveto{\pgfqpoint{1.025455in}{0.696000in}}%
\pgfpathlineto{\pgfqpoint{1.071466in}{1.033601in}}%
\pgfpathlineto{\pgfqpoint{1.117477in}{1.214699in}}%
\pgfpathlineto{\pgfqpoint{1.163488in}{1.343530in}}%
\pgfpathlineto{\pgfqpoint{1.209499in}{1.442070in}}%
\pgfpathlineto{\pgfqpoint{1.255510in}{1.521979in}}%
\pgfpathlineto{\pgfqpoint{1.301521in}{1.588112in}}%
\pgfpathlineto{\pgfqpoint{1.347532in}{1.648023in}}%
\pgfpathlineto{\pgfqpoint{1.393544in}{1.698229in}}%
\pgfpathlineto{\pgfqpoint{1.439555in}{1.741650in}}%
\pgfpathlineto{\pgfqpoint{1.485566in}{1.782658in}}%
\pgfpathlineto{\pgfqpoint{1.531577in}{1.818747in}}%
\pgfpathlineto{\pgfqpoint{1.577588in}{1.851597in}}%
\pgfpathlineto{\pgfqpoint{1.623599in}{1.884240in}}%
\pgfpathlineto{\pgfqpoint{1.669610in}{1.914398in}}%
\pgfpathlineto{\pgfqpoint{1.715622in}{1.940878in}}%
\pgfpathlineto{\pgfqpoint{1.761633in}{1.966254in}}%
\pgfpathlineto{\pgfqpoint{1.807644in}{1.990168in}}%
\pgfpathlineto{\pgfqpoint{1.853655in}{2.011519in}}%
\pgfpathlineto{\pgfqpoint{1.899666in}{2.033074in}}%
\pgfpathlineto{\pgfqpoint{1.945677in}{2.053594in}}%
\pgfpathlineto{\pgfqpoint{1.991688in}{2.072353in}}%
\pgfpathlineto{\pgfqpoint{2.037699in}{2.090468in}}%
\pgfpathlineto{\pgfqpoint{2.083711in}{2.107880in}}%
\pgfpathlineto{\pgfqpoint{2.129722in}{2.123935in}}%
\pgfpathlineto{\pgfqpoint{2.175733in}{2.140334in}}%
\pgfpathlineto{\pgfqpoint{2.221744in}{2.156534in}}%
\pgfpathlineto{\pgfqpoint{2.267755in}{2.171554in}}%
\pgfpathlineto{\pgfqpoint{2.313766in}{2.186532in}}%
\pgfpathlineto{\pgfqpoint{2.359777in}{2.200716in}}%
\pgfpathlineto{\pgfqpoint{2.405788in}{2.213940in}}%
\pgfpathlineto{\pgfqpoint{2.451800in}{2.227196in}}%
\pgfpathlineto{\pgfqpoint{2.497811in}{2.239809in}}%
\pgfpathlineto{\pgfqpoint{2.543822in}{2.251622in}}%
\pgfpathlineto{\pgfqpoint{2.589833in}{2.263511in}}%
\pgfpathlineto{\pgfqpoint{2.635844in}{2.274867in}}%
\pgfpathlineto{\pgfqpoint{2.681855in}{2.285540in}}%
\pgfpathlineto{\pgfqpoint{2.727866in}{2.296317in}}%
\pgfpathlineto{\pgfqpoint{2.773878in}{2.306882in}}%
\pgfpathlineto{\pgfqpoint{2.819889in}{2.317071in}}%
\pgfpathlineto{\pgfqpoint{2.865900in}{2.327243in}}%
\pgfpathlineto{\pgfqpoint{2.911911in}{2.337011in}}%
\pgfpathlineto{\pgfqpoint{2.957922in}{2.346348in}}%
\pgfpathlineto{\pgfqpoint{3.003933in}{2.355692in}}%
\pgfpathlineto{\pgfqpoint{3.049944in}{2.364686in}}%
\pgfpathlineto{\pgfqpoint{3.095955in}{2.373301in}}%
\pgfpathlineto{\pgfqpoint{3.141967in}{2.381942in}}%
\pgfpathlineto{\pgfqpoint{3.187978in}{2.390276in}}%
\pgfpathlineto{\pgfqpoint{3.233989in}{2.398274in}}%
\pgfpathlineto{\pgfqpoint{3.280000in}{2.406310in}}%
\pgfpathlineto{\pgfqpoint{3.326011in}{2.414073in}}%
\pgfpathlineto{\pgfqpoint{3.372022in}{2.422045in}}%
\pgfpathlineto{\pgfqpoint{3.418033in}{2.430041in}}%
\pgfpathlineto{\pgfqpoint{3.464045in}{2.437934in}}%
\pgfpathlineto{\pgfqpoint{3.510056in}{2.445371in}}%
\pgfpathlineto{\pgfqpoint{3.556067in}{2.452842in}}%
\pgfpathlineto{\pgfqpoint{3.602078in}{2.460229in}}%
\pgfpathlineto{\pgfqpoint{3.648089in}{2.467198in}}%
\pgfpathlineto{\pgfqpoint{3.694100in}{2.474210in}}%
\pgfpathlineto{\pgfqpoint{3.740111in}{2.481151in}}%
\pgfpathlineto{\pgfqpoint{3.786122in}{2.487708in}}%
\pgfpathlineto{\pgfqpoint{3.832134in}{2.494313in}}%
\pgfpathlineto{\pgfqpoint{3.878145in}{2.500859in}}%
\pgfpathlineto{\pgfqpoint{3.924156in}{2.507051in}}%
\pgfpathlineto{\pgfqpoint{3.970167in}{2.513294in}}%
\pgfpathlineto{\pgfqpoint{4.016178in}{2.519488in}}%
\pgfpathlineto{\pgfqpoint{4.062189in}{2.525352in}}%
\pgfpathlineto{\pgfqpoint{4.108200in}{2.531270in}}%
\pgfpathlineto{\pgfqpoint{4.154212in}{2.537148in}}%
\pgfpathlineto{\pgfqpoint{4.200223in}{2.542717in}}%
\pgfpathlineto{\pgfqpoint{4.246234in}{2.548343in}}%
\pgfpathlineto{\pgfqpoint{4.292245in}{2.553935in}}%
\pgfpathlineto{\pgfqpoint{4.338256in}{2.559239in}}%
\pgfpathlineto{\pgfqpoint{4.384267in}{2.564600in}}%
\pgfpathlineto{\pgfqpoint{4.430278in}{2.569933in}}%
\pgfpathlineto{\pgfqpoint{4.476289in}{2.574994in}}%
\pgfpathlineto{\pgfqpoint{4.522301in}{2.580167in}}%
\pgfpathlineto{\pgfqpoint{4.568312in}{2.585420in}}%
\pgfpathlineto{\pgfqpoint{4.614323in}{2.590464in}}%
\pgfpathlineto{\pgfqpoint{4.660334in}{2.595511in}}%
\pgfpathlineto{\pgfqpoint{4.706345in}{2.600534in}}%
\pgfpathlineto{\pgfqpoint{4.752356in}{2.605360in}}%
\pgfpathlineto{\pgfqpoint{4.798367in}{2.610192in}}%
\pgfpathlineto{\pgfqpoint{4.844378in}{2.615004in}}%
\pgfpathlineto{\pgfqpoint{4.890390in}{2.619630in}}%
\pgfpathlineto{\pgfqpoint{4.936401in}{2.624265in}}%
\pgfpathlineto{\pgfqpoint{4.982412in}{2.628883in}}%
\pgfpathlineto{\pgfqpoint{5.028423in}{2.633325in}}%
\pgfpathlineto{\pgfqpoint{5.074434in}{2.637778in}}%
\pgfpathlineto{\pgfqpoint{5.120445in}{2.642217in}}%
\pgfpathlineto{\pgfqpoint{5.166456in}{2.646490in}}%
\pgfpathlineto{\pgfqpoint{5.212468in}{2.650775in}}%
\pgfpathlineto{\pgfqpoint{5.258479in}{2.655048in}}%
\pgfpathlineto{\pgfqpoint{5.304490in}{2.659164in}}%
\pgfpathlineto{\pgfqpoint{5.350501in}{2.663292in}}%
\pgfpathlineto{\pgfqpoint{5.396512in}{2.667413in}}%
\pgfpathlineto{\pgfqpoint{5.442523in}{2.671382in}}%
\pgfpathlineto{\pgfqpoint{5.488534in}{2.675365in}}%
\pgfpathlineto{\pgfqpoint{5.534545in}{2.679343in}}%
\pgfusepath{stroke}%
\end{pgfscope}%
\begin{pgfscope}%
\pgfsetrectcap%
\pgfsetmiterjoin%
\pgfsetlinewidth{0.803000pt}%
\definecolor{currentstroke}{rgb}{0.000000,0.000000,0.000000}%
\pgfsetstrokecolor{currentstroke}%
\pgfsetdash{}{0pt}%
\pgfpathmoveto{\pgfqpoint{0.800000in}{0.528000in}}%
\pgfpathlineto{\pgfqpoint{0.800000in}{4.224000in}}%
\pgfusepath{stroke}%
\end{pgfscope}%
\begin{pgfscope}%
\pgfsetrectcap%
\pgfsetmiterjoin%
\pgfsetlinewidth{0.803000pt}%
\definecolor{currentstroke}{rgb}{0.000000,0.000000,0.000000}%
\pgfsetstrokecolor{currentstroke}%
\pgfsetdash{}{0pt}%
\pgfpathmoveto{\pgfqpoint{5.760000in}{0.528000in}}%
\pgfpathlineto{\pgfqpoint{5.760000in}{4.224000in}}%
\pgfusepath{stroke}%
\end{pgfscope}%
\begin{pgfscope}%
\pgfsetrectcap%
\pgfsetmiterjoin%
\pgfsetlinewidth{0.803000pt}%
\definecolor{currentstroke}{rgb}{0.000000,0.000000,0.000000}%
\pgfsetstrokecolor{currentstroke}%
\pgfsetdash{}{0pt}%
\pgfpathmoveto{\pgfqpoint{0.800000in}{0.528000in}}%
\pgfpathlineto{\pgfqpoint{5.760000in}{0.528000in}}%
\pgfusepath{stroke}%
\end{pgfscope}%
\begin{pgfscope}%
\pgfsetrectcap%
\pgfsetmiterjoin%
\pgfsetlinewidth{0.803000pt}%
\definecolor{currentstroke}{rgb}{0.000000,0.000000,0.000000}%
\pgfsetstrokecolor{currentstroke}%
\pgfsetdash{}{0pt}%
\pgfpathmoveto{\pgfqpoint{0.800000in}{4.224000in}}%
\pgfpathlineto{\pgfqpoint{5.760000in}{4.224000in}}%
\pgfusepath{stroke}%
\end{pgfscope}%
\begin{pgfscope}%
\pgfsetbuttcap%
\pgfsetmiterjoin%
\definecolor{currentfill}{rgb}{1.000000,1.000000,1.000000}%
\pgfsetfillcolor{currentfill}%
\pgfsetfillopacity{0.800000}%
\pgfsetlinewidth{1.003750pt}%
\definecolor{currentstroke}{rgb}{0.800000,0.800000,0.800000}%
\pgfsetstrokecolor{currentstroke}%
\pgfsetstrokeopacity{0.800000}%
\pgfsetdash{}{0pt}%
\pgfpathmoveto{\pgfqpoint{0.897222in}{3.725543in}}%
\pgfpathlineto{\pgfqpoint{1.975542in}{3.725543in}}%
\pgfpathquadraticcurveto{\pgfqpoint{2.003319in}{3.725543in}}{\pgfqpoint{2.003319in}{3.753321in}}%
\pgfpathlineto{\pgfqpoint{2.003319in}{4.126778in}}%
\pgfpathquadraticcurveto{\pgfqpoint{2.003319in}{4.154556in}}{\pgfqpoint{1.975542in}{4.154556in}}%
\pgfpathlineto{\pgfqpoint{0.897222in}{4.154556in}}%
\pgfpathquadraticcurveto{\pgfqpoint{0.869444in}{4.154556in}}{\pgfqpoint{0.869444in}{4.126778in}}%
\pgfpathlineto{\pgfqpoint{0.869444in}{3.753321in}}%
\pgfpathquadraticcurveto{\pgfqpoint{0.869444in}{3.725543in}}{\pgfqpoint{0.897222in}{3.725543in}}%
\pgfpathlineto{\pgfqpoint{0.897222in}{3.725543in}}%
\pgfpathclose%
\pgfusepath{stroke,fill}%
\end{pgfscope}%
\begin{pgfscope}%
\pgfsetrectcap%
\pgfsetroundjoin%
\pgfsetlinewidth{1.505625pt}%
\definecolor{currentstroke}{rgb}{0.121569,0.466667,0.705882}%
\pgfsetstrokecolor{currentstroke}%
\pgfsetdash{}{0pt}%
\pgfpathmoveto{\pgfqpoint{0.925000in}{4.050389in}}%
\pgfpathlineto{\pgfqpoint{1.063889in}{4.050389in}}%
\pgfpathlineto{\pgfqpoint{1.202778in}{4.050389in}}%
\pgfusepath{stroke}%
\end{pgfscope}%
\begin{pgfscope}%
\definecolor{textcolor}{rgb}{0.000000,0.000000,0.000000}%
\pgfsetstrokecolor{textcolor}%
\pgfsetfillcolor{textcolor}%
\pgftext[x=1.313889in,y=4.001778in,left,base]{\color{textcolor}\rmfamily\fontsize{10.000000}{12.000000}\selectfont quicksort}%
\end{pgfscope}%
\begin{pgfscope}%
\pgfsetrectcap%
\pgfsetroundjoin%
\pgfsetlinewidth{1.505625pt}%
\definecolor{currentstroke}{rgb}{1.000000,0.498039,0.054902}%
\pgfsetstrokecolor{currentstroke}%
\pgfsetdash{}{0pt}%
\pgfpathmoveto{\pgfqpoint{0.925000in}{3.856716in}}%
\pgfpathlineto{\pgfqpoint{1.063889in}{3.856716in}}%
\pgfpathlineto{\pgfqpoint{1.202778in}{3.856716in}}%
\pgfusepath{stroke}%
\end{pgfscope}%
\begin{pgfscope}%
\definecolor{textcolor}{rgb}{0.000000,0.000000,0.000000}%
\pgfsetstrokecolor{textcolor}%
\pgfsetfillcolor{textcolor}%
\pgftext[x=1.313889in,y=3.808105in,left,base]{\color{textcolor}\rmfamily\fontsize{10.000000}{12.000000}\selectfont bquicksort}%
\end{pgfscope}%
\end{pgfpicture}%
\makeatother%
\endgroup%

\subsubsection{Swaps}
%% Creator: Matplotlib, PGF backend
%%
%% To include the figure in your LaTeX document, write
%%   \input{<filename>.pgf}
%%
%% Make sure the required packages are loaded in your preamble
%%   \usepackage{pgf}
%%
%% Also ensure that all the required font packages are loaded; for instance,
%% the lmodern package is sometimes necessary when using math font.
%%   \usepackage{lmodern}
%%
%% Figures using additional raster images can only be included by \input if
%% they are in the same directory as the main LaTeX file. For loading figures
%% from other directories you can use the `import` package
%%   \usepackage{import}
%%
%% and then include the figures with
%%   \import{<path to file>}{<filename>.pgf}
%%
%% Matplotlib used the following preamble
%%   
%%   \makeatletter\@ifpackageloaded{underscore}{}{\usepackage[strings]{underscore}}\makeatother
%%
\begingroup%
\makeatletter%
\begin{pgfpicture}%
\pgfpathrectangle{\pgfpointorigin}{\pgfqpoint{6.400000in}{4.800000in}}%
\pgfusepath{use as bounding box, clip}%
\begin{pgfscope}%
\pgfsetbuttcap%
\pgfsetmiterjoin%
\definecolor{currentfill}{rgb}{1.000000,1.000000,1.000000}%
\pgfsetfillcolor{currentfill}%
\pgfsetlinewidth{0.000000pt}%
\definecolor{currentstroke}{rgb}{1.000000,1.000000,1.000000}%
\pgfsetstrokecolor{currentstroke}%
\pgfsetdash{}{0pt}%
\pgfpathmoveto{\pgfqpoint{0.000000in}{0.000000in}}%
\pgfpathlineto{\pgfqpoint{6.400000in}{0.000000in}}%
\pgfpathlineto{\pgfqpoint{6.400000in}{4.800000in}}%
\pgfpathlineto{\pgfqpoint{0.000000in}{4.800000in}}%
\pgfpathlineto{\pgfqpoint{0.000000in}{0.000000in}}%
\pgfpathclose%
\pgfusepath{fill}%
\end{pgfscope}%
\begin{pgfscope}%
\pgfsetbuttcap%
\pgfsetmiterjoin%
\definecolor{currentfill}{rgb}{1.000000,1.000000,1.000000}%
\pgfsetfillcolor{currentfill}%
\pgfsetlinewidth{0.000000pt}%
\definecolor{currentstroke}{rgb}{0.000000,0.000000,0.000000}%
\pgfsetstrokecolor{currentstroke}%
\pgfsetstrokeopacity{0.000000}%
\pgfsetdash{}{0pt}%
\pgfpathmoveto{\pgfqpoint{0.800000in}{0.528000in}}%
\pgfpathlineto{\pgfqpoint{5.760000in}{0.528000in}}%
\pgfpathlineto{\pgfqpoint{5.760000in}{4.224000in}}%
\pgfpathlineto{\pgfqpoint{0.800000in}{4.224000in}}%
\pgfpathlineto{\pgfqpoint{0.800000in}{0.528000in}}%
\pgfpathclose%
\pgfusepath{fill}%
\end{pgfscope}%
\begin{pgfscope}%
\pgfsetbuttcap%
\pgfsetroundjoin%
\definecolor{currentfill}{rgb}{0.000000,0.000000,0.000000}%
\pgfsetfillcolor{currentfill}%
\pgfsetlinewidth{0.803000pt}%
\definecolor{currentstroke}{rgb}{0.000000,0.000000,0.000000}%
\pgfsetstrokecolor{currentstroke}%
\pgfsetdash{}{0pt}%
\pgfsys@defobject{currentmarker}{\pgfqpoint{0.000000in}{-0.048611in}}{\pgfqpoint{0.000000in}{0.000000in}}{%
\pgfpathmoveto{\pgfqpoint{0.000000in}{0.000000in}}%
\pgfpathlineto{\pgfqpoint{0.000000in}{-0.048611in}}%
\pgfusepath{stroke,fill}%
}%
\begin{pgfscope}%
\pgfsys@transformshift{0.979443in}{0.528000in}%
\pgfsys@useobject{currentmarker}{}%
\end{pgfscope}%
\end{pgfscope}%
\begin{pgfscope}%
\definecolor{textcolor}{rgb}{0.000000,0.000000,0.000000}%
\pgfsetstrokecolor{textcolor}%
\pgfsetfillcolor{textcolor}%
\pgftext[x=0.979443in,y=0.430778in,,top]{\color{textcolor}\rmfamily\fontsize{10.000000}{12.000000}\selectfont \(\displaystyle {0}\)}%
\end{pgfscope}%
\begin{pgfscope}%
\pgfsetbuttcap%
\pgfsetroundjoin%
\definecolor{currentfill}{rgb}{0.000000,0.000000,0.000000}%
\pgfsetfillcolor{currentfill}%
\pgfsetlinewidth{0.803000pt}%
\definecolor{currentstroke}{rgb}{0.000000,0.000000,0.000000}%
\pgfsetstrokecolor{currentstroke}%
\pgfsetdash{}{0pt}%
\pgfsys@defobject{currentmarker}{\pgfqpoint{0.000000in}{-0.048611in}}{\pgfqpoint{0.000000in}{0.000000in}}{%
\pgfpathmoveto{\pgfqpoint{0.000000in}{0.000000in}}%
\pgfpathlineto{\pgfqpoint{0.000000in}{-0.048611in}}%
\pgfusepath{stroke,fill}%
}%
\begin{pgfscope}%
\pgfsys@transformshift{1.899666in}{0.528000in}%
\pgfsys@useobject{currentmarker}{}%
\end{pgfscope}%
\end{pgfscope}%
\begin{pgfscope}%
\definecolor{textcolor}{rgb}{0.000000,0.000000,0.000000}%
\pgfsetstrokecolor{textcolor}%
\pgfsetfillcolor{textcolor}%
\pgftext[x=1.899666in,y=0.430778in,,top]{\color{textcolor}\rmfamily\fontsize{10.000000}{12.000000}\selectfont \(\displaystyle {200}\)}%
\end{pgfscope}%
\begin{pgfscope}%
\pgfsetbuttcap%
\pgfsetroundjoin%
\definecolor{currentfill}{rgb}{0.000000,0.000000,0.000000}%
\pgfsetfillcolor{currentfill}%
\pgfsetlinewidth{0.803000pt}%
\definecolor{currentstroke}{rgb}{0.000000,0.000000,0.000000}%
\pgfsetstrokecolor{currentstroke}%
\pgfsetdash{}{0pt}%
\pgfsys@defobject{currentmarker}{\pgfqpoint{0.000000in}{-0.048611in}}{\pgfqpoint{0.000000in}{0.000000in}}{%
\pgfpathmoveto{\pgfqpoint{0.000000in}{0.000000in}}%
\pgfpathlineto{\pgfqpoint{0.000000in}{-0.048611in}}%
\pgfusepath{stroke,fill}%
}%
\begin{pgfscope}%
\pgfsys@transformshift{2.819889in}{0.528000in}%
\pgfsys@useobject{currentmarker}{}%
\end{pgfscope}%
\end{pgfscope}%
\begin{pgfscope}%
\definecolor{textcolor}{rgb}{0.000000,0.000000,0.000000}%
\pgfsetstrokecolor{textcolor}%
\pgfsetfillcolor{textcolor}%
\pgftext[x=2.819889in,y=0.430778in,,top]{\color{textcolor}\rmfamily\fontsize{10.000000}{12.000000}\selectfont \(\displaystyle {400}\)}%
\end{pgfscope}%
\begin{pgfscope}%
\pgfsetbuttcap%
\pgfsetroundjoin%
\definecolor{currentfill}{rgb}{0.000000,0.000000,0.000000}%
\pgfsetfillcolor{currentfill}%
\pgfsetlinewidth{0.803000pt}%
\definecolor{currentstroke}{rgb}{0.000000,0.000000,0.000000}%
\pgfsetstrokecolor{currentstroke}%
\pgfsetdash{}{0pt}%
\pgfsys@defobject{currentmarker}{\pgfqpoint{0.000000in}{-0.048611in}}{\pgfqpoint{0.000000in}{0.000000in}}{%
\pgfpathmoveto{\pgfqpoint{0.000000in}{0.000000in}}%
\pgfpathlineto{\pgfqpoint{0.000000in}{-0.048611in}}%
\pgfusepath{stroke,fill}%
}%
\begin{pgfscope}%
\pgfsys@transformshift{3.740111in}{0.528000in}%
\pgfsys@useobject{currentmarker}{}%
\end{pgfscope}%
\end{pgfscope}%
\begin{pgfscope}%
\definecolor{textcolor}{rgb}{0.000000,0.000000,0.000000}%
\pgfsetstrokecolor{textcolor}%
\pgfsetfillcolor{textcolor}%
\pgftext[x=3.740111in,y=0.430778in,,top]{\color{textcolor}\rmfamily\fontsize{10.000000}{12.000000}\selectfont \(\displaystyle {600}\)}%
\end{pgfscope}%
\begin{pgfscope}%
\pgfsetbuttcap%
\pgfsetroundjoin%
\definecolor{currentfill}{rgb}{0.000000,0.000000,0.000000}%
\pgfsetfillcolor{currentfill}%
\pgfsetlinewidth{0.803000pt}%
\definecolor{currentstroke}{rgb}{0.000000,0.000000,0.000000}%
\pgfsetstrokecolor{currentstroke}%
\pgfsetdash{}{0pt}%
\pgfsys@defobject{currentmarker}{\pgfqpoint{0.000000in}{-0.048611in}}{\pgfqpoint{0.000000in}{0.000000in}}{%
\pgfpathmoveto{\pgfqpoint{0.000000in}{0.000000in}}%
\pgfpathlineto{\pgfqpoint{0.000000in}{-0.048611in}}%
\pgfusepath{stroke,fill}%
}%
\begin{pgfscope}%
\pgfsys@transformshift{4.660334in}{0.528000in}%
\pgfsys@useobject{currentmarker}{}%
\end{pgfscope}%
\end{pgfscope}%
\begin{pgfscope}%
\definecolor{textcolor}{rgb}{0.000000,0.000000,0.000000}%
\pgfsetstrokecolor{textcolor}%
\pgfsetfillcolor{textcolor}%
\pgftext[x=4.660334in,y=0.430778in,,top]{\color{textcolor}\rmfamily\fontsize{10.000000}{12.000000}\selectfont \(\displaystyle {800}\)}%
\end{pgfscope}%
\begin{pgfscope}%
\pgfsetbuttcap%
\pgfsetroundjoin%
\definecolor{currentfill}{rgb}{0.000000,0.000000,0.000000}%
\pgfsetfillcolor{currentfill}%
\pgfsetlinewidth{0.803000pt}%
\definecolor{currentstroke}{rgb}{0.000000,0.000000,0.000000}%
\pgfsetstrokecolor{currentstroke}%
\pgfsetdash{}{0pt}%
\pgfsys@defobject{currentmarker}{\pgfqpoint{0.000000in}{-0.048611in}}{\pgfqpoint{0.000000in}{0.000000in}}{%
\pgfpathmoveto{\pgfqpoint{0.000000in}{0.000000in}}%
\pgfpathlineto{\pgfqpoint{0.000000in}{-0.048611in}}%
\pgfusepath{stroke,fill}%
}%
\begin{pgfscope}%
\pgfsys@transformshift{5.580557in}{0.528000in}%
\pgfsys@useobject{currentmarker}{}%
\end{pgfscope}%
\end{pgfscope}%
\begin{pgfscope}%
\definecolor{textcolor}{rgb}{0.000000,0.000000,0.000000}%
\pgfsetstrokecolor{textcolor}%
\pgfsetfillcolor{textcolor}%
\pgftext[x=5.580557in,y=0.430778in,,top]{\color{textcolor}\rmfamily\fontsize{10.000000}{12.000000}\selectfont \(\displaystyle {1000}\)}%
\end{pgfscope}%
\begin{pgfscope}%
\definecolor{textcolor}{rgb}{0.000000,0.000000,0.000000}%
\pgfsetstrokecolor{textcolor}%
\pgfsetfillcolor{textcolor}%
\pgftext[x=3.280000in,y=0.251766in,,top]{\color{textcolor}\rmfamily\fontsize{10.000000}{12.000000}\selectfont Input Size}%
\end{pgfscope}%
\begin{pgfscope}%
\pgfsetbuttcap%
\pgfsetroundjoin%
\definecolor{currentfill}{rgb}{0.000000,0.000000,0.000000}%
\pgfsetfillcolor{currentfill}%
\pgfsetlinewidth{0.803000pt}%
\definecolor{currentstroke}{rgb}{0.000000,0.000000,0.000000}%
\pgfsetstrokecolor{currentstroke}%
\pgfsetdash{}{0pt}%
\pgfsys@defobject{currentmarker}{\pgfqpoint{-0.048611in}{0.000000in}}{\pgfqpoint{-0.000000in}{0.000000in}}{%
\pgfpathmoveto{\pgfqpoint{-0.000000in}{0.000000in}}%
\pgfpathlineto{\pgfqpoint{-0.048611in}{0.000000in}}%
\pgfusepath{stroke,fill}%
}%
\begin{pgfscope}%
\pgfsys@transformshift{0.800000in}{0.768935in}%
\pgfsys@useobject{currentmarker}{}%
\end{pgfscope}%
\end{pgfscope}%
\begin{pgfscope}%
\definecolor{textcolor}{rgb}{0.000000,0.000000,0.000000}%
\pgfsetstrokecolor{textcolor}%
\pgfsetfillcolor{textcolor}%
\pgftext[x=0.501581in, y=0.720710in, left, base]{\color{textcolor}\rmfamily\fontsize{10.000000}{12.000000}\selectfont \(\displaystyle {10^{1}}\)}%
\end{pgfscope}%
\begin{pgfscope}%
\pgfsetbuttcap%
\pgfsetroundjoin%
\definecolor{currentfill}{rgb}{0.000000,0.000000,0.000000}%
\pgfsetfillcolor{currentfill}%
\pgfsetlinewidth{0.803000pt}%
\definecolor{currentstroke}{rgb}{0.000000,0.000000,0.000000}%
\pgfsetstrokecolor{currentstroke}%
\pgfsetdash{}{0pt}%
\pgfsys@defobject{currentmarker}{\pgfqpoint{-0.048611in}{0.000000in}}{\pgfqpoint{-0.000000in}{0.000000in}}{%
\pgfpathmoveto{\pgfqpoint{-0.000000in}{0.000000in}}%
\pgfpathlineto{\pgfqpoint{-0.048611in}{0.000000in}}%
\pgfusepath{stroke,fill}%
}%
\begin{pgfscope}%
\pgfsys@transformshift{0.800000in}{2.362891in}%
\pgfsys@useobject{currentmarker}{}%
\end{pgfscope}%
\end{pgfscope}%
\begin{pgfscope}%
\definecolor{textcolor}{rgb}{0.000000,0.000000,0.000000}%
\pgfsetstrokecolor{textcolor}%
\pgfsetfillcolor{textcolor}%
\pgftext[x=0.501581in, y=2.314666in, left, base]{\color{textcolor}\rmfamily\fontsize{10.000000}{12.000000}\selectfont \(\displaystyle {10^{2}}\)}%
\end{pgfscope}%
\begin{pgfscope}%
\pgfsetbuttcap%
\pgfsetroundjoin%
\definecolor{currentfill}{rgb}{0.000000,0.000000,0.000000}%
\pgfsetfillcolor{currentfill}%
\pgfsetlinewidth{0.803000pt}%
\definecolor{currentstroke}{rgb}{0.000000,0.000000,0.000000}%
\pgfsetstrokecolor{currentstroke}%
\pgfsetdash{}{0pt}%
\pgfsys@defobject{currentmarker}{\pgfqpoint{-0.048611in}{0.000000in}}{\pgfqpoint{-0.000000in}{0.000000in}}{%
\pgfpathmoveto{\pgfqpoint{-0.000000in}{0.000000in}}%
\pgfpathlineto{\pgfqpoint{-0.048611in}{0.000000in}}%
\pgfusepath{stroke,fill}%
}%
\begin{pgfscope}%
\pgfsys@transformshift{0.800000in}{3.956847in}%
\pgfsys@useobject{currentmarker}{}%
\end{pgfscope}%
\end{pgfscope}%
\begin{pgfscope}%
\definecolor{textcolor}{rgb}{0.000000,0.000000,0.000000}%
\pgfsetstrokecolor{textcolor}%
\pgfsetfillcolor{textcolor}%
\pgftext[x=0.501581in, y=3.908621in, left, base]{\color{textcolor}\rmfamily\fontsize{10.000000}{12.000000}\selectfont \(\displaystyle {10^{3}}\)}%
\end{pgfscope}%
\begin{pgfscope}%
\pgfsetbuttcap%
\pgfsetroundjoin%
\definecolor{currentfill}{rgb}{0.000000,0.000000,0.000000}%
\pgfsetfillcolor{currentfill}%
\pgfsetlinewidth{0.602250pt}%
\definecolor{currentstroke}{rgb}{0.000000,0.000000,0.000000}%
\pgfsetstrokecolor{currentstroke}%
\pgfsetdash{}{0pt}%
\pgfsys@defobject{currentmarker}{\pgfqpoint{-0.027778in}{0.000000in}}{\pgfqpoint{-0.000000in}{0.000000in}}{%
\pgfpathmoveto{\pgfqpoint{-0.000000in}{0.000000in}}%
\pgfpathlineto{\pgfqpoint{-0.027778in}{0.000000in}}%
\pgfusepath{stroke,fill}%
}%
\begin{pgfscope}%
\pgfsys@transformshift{0.800000in}{0.614465in}%
\pgfsys@useobject{currentmarker}{}%
\end{pgfscope}%
\end{pgfscope}%
\begin{pgfscope}%
\pgfsetbuttcap%
\pgfsetroundjoin%
\definecolor{currentfill}{rgb}{0.000000,0.000000,0.000000}%
\pgfsetfillcolor{currentfill}%
\pgfsetlinewidth{0.602250pt}%
\definecolor{currentstroke}{rgb}{0.000000,0.000000,0.000000}%
\pgfsetstrokecolor{currentstroke}%
\pgfsetdash{}{0pt}%
\pgfsys@defobject{currentmarker}{\pgfqpoint{-0.027778in}{0.000000in}}{\pgfqpoint{-0.000000in}{0.000000in}}{%
\pgfpathmoveto{\pgfqpoint{-0.000000in}{0.000000in}}%
\pgfpathlineto{\pgfqpoint{-0.027778in}{0.000000in}}%
\pgfusepath{stroke,fill}%
}%
\begin{pgfscope}%
\pgfsys@transformshift{0.800000in}{0.696000in}%
\pgfsys@useobject{currentmarker}{}%
\end{pgfscope}%
\end{pgfscope}%
\begin{pgfscope}%
\pgfsetbuttcap%
\pgfsetroundjoin%
\definecolor{currentfill}{rgb}{0.000000,0.000000,0.000000}%
\pgfsetfillcolor{currentfill}%
\pgfsetlinewidth{0.602250pt}%
\definecolor{currentstroke}{rgb}{0.000000,0.000000,0.000000}%
\pgfsetstrokecolor{currentstroke}%
\pgfsetdash{}{0pt}%
\pgfsys@defobject{currentmarker}{\pgfqpoint{-0.027778in}{0.000000in}}{\pgfqpoint{-0.000000in}{0.000000in}}{%
\pgfpathmoveto{\pgfqpoint{-0.000000in}{0.000000in}}%
\pgfpathlineto{\pgfqpoint{-0.027778in}{0.000000in}}%
\pgfusepath{stroke,fill}%
}%
\begin{pgfscope}%
\pgfsys@transformshift{0.800000in}{1.248764in}%
\pgfsys@useobject{currentmarker}{}%
\end{pgfscope}%
\end{pgfscope}%
\begin{pgfscope}%
\pgfsetbuttcap%
\pgfsetroundjoin%
\definecolor{currentfill}{rgb}{0.000000,0.000000,0.000000}%
\pgfsetfillcolor{currentfill}%
\pgfsetlinewidth{0.602250pt}%
\definecolor{currentstroke}{rgb}{0.000000,0.000000,0.000000}%
\pgfsetstrokecolor{currentstroke}%
\pgfsetdash{}{0pt}%
\pgfsys@defobject{currentmarker}{\pgfqpoint{-0.027778in}{0.000000in}}{\pgfqpoint{-0.000000in}{0.000000in}}{%
\pgfpathmoveto{\pgfqpoint{-0.000000in}{0.000000in}}%
\pgfpathlineto{\pgfqpoint{-0.027778in}{0.000000in}}%
\pgfusepath{stroke,fill}%
}%
\begin{pgfscope}%
\pgfsys@transformshift{0.800000in}{1.529446in}%
\pgfsys@useobject{currentmarker}{}%
\end{pgfscope}%
\end{pgfscope}%
\begin{pgfscope}%
\pgfsetbuttcap%
\pgfsetroundjoin%
\definecolor{currentfill}{rgb}{0.000000,0.000000,0.000000}%
\pgfsetfillcolor{currentfill}%
\pgfsetlinewidth{0.602250pt}%
\definecolor{currentstroke}{rgb}{0.000000,0.000000,0.000000}%
\pgfsetstrokecolor{currentstroke}%
\pgfsetdash{}{0pt}%
\pgfsys@defobject{currentmarker}{\pgfqpoint{-0.027778in}{0.000000in}}{\pgfqpoint{-0.000000in}{0.000000in}}{%
\pgfpathmoveto{\pgfqpoint{-0.000000in}{0.000000in}}%
\pgfpathlineto{\pgfqpoint{-0.027778in}{0.000000in}}%
\pgfusepath{stroke,fill}%
}%
\begin{pgfscope}%
\pgfsys@transformshift{0.800000in}{1.728592in}%
\pgfsys@useobject{currentmarker}{}%
\end{pgfscope}%
\end{pgfscope}%
\begin{pgfscope}%
\pgfsetbuttcap%
\pgfsetroundjoin%
\definecolor{currentfill}{rgb}{0.000000,0.000000,0.000000}%
\pgfsetfillcolor{currentfill}%
\pgfsetlinewidth{0.602250pt}%
\definecolor{currentstroke}{rgb}{0.000000,0.000000,0.000000}%
\pgfsetstrokecolor{currentstroke}%
\pgfsetdash{}{0pt}%
\pgfsys@defobject{currentmarker}{\pgfqpoint{-0.027778in}{0.000000in}}{\pgfqpoint{-0.000000in}{0.000000in}}{%
\pgfpathmoveto{\pgfqpoint{-0.000000in}{0.000000in}}%
\pgfpathlineto{\pgfqpoint{-0.027778in}{0.000000in}}%
\pgfusepath{stroke,fill}%
}%
\begin{pgfscope}%
\pgfsys@transformshift{0.800000in}{1.883063in}%
\pgfsys@useobject{currentmarker}{}%
\end{pgfscope}%
\end{pgfscope}%
\begin{pgfscope}%
\pgfsetbuttcap%
\pgfsetroundjoin%
\definecolor{currentfill}{rgb}{0.000000,0.000000,0.000000}%
\pgfsetfillcolor{currentfill}%
\pgfsetlinewidth{0.602250pt}%
\definecolor{currentstroke}{rgb}{0.000000,0.000000,0.000000}%
\pgfsetstrokecolor{currentstroke}%
\pgfsetdash{}{0pt}%
\pgfsys@defobject{currentmarker}{\pgfqpoint{-0.027778in}{0.000000in}}{\pgfqpoint{-0.000000in}{0.000000in}}{%
\pgfpathmoveto{\pgfqpoint{-0.000000in}{0.000000in}}%
\pgfpathlineto{\pgfqpoint{-0.027778in}{0.000000in}}%
\pgfusepath{stroke,fill}%
}%
\begin{pgfscope}%
\pgfsys@transformshift{0.800000in}{2.009274in}%
\pgfsys@useobject{currentmarker}{}%
\end{pgfscope}%
\end{pgfscope}%
\begin{pgfscope}%
\pgfsetbuttcap%
\pgfsetroundjoin%
\definecolor{currentfill}{rgb}{0.000000,0.000000,0.000000}%
\pgfsetfillcolor{currentfill}%
\pgfsetlinewidth{0.602250pt}%
\definecolor{currentstroke}{rgb}{0.000000,0.000000,0.000000}%
\pgfsetstrokecolor{currentstroke}%
\pgfsetdash{}{0pt}%
\pgfsys@defobject{currentmarker}{\pgfqpoint{-0.027778in}{0.000000in}}{\pgfqpoint{-0.000000in}{0.000000in}}{%
\pgfpathmoveto{\pgfqpoint{-0.000000in}{0.000000in}}%
\pgfpathlineto{\pgfqpoint{-0.027778in}{0.000000in}}%
\pgfusepath{stroke,fill}%
}%
\begin{pgfscope}%
\pgfsys@transformshift{0.800000in}{2.115984in}%
\pgfsys@useobject{currentmarker}{}%
\end{pgfscope}%
\end{pgfscope}%
\begin{pgfscope}%
\pgfsetbuttcap%
\pgfsetroundjoin%
\definecolor{currentfill}{rgb}{0.000000,0.000000,0.000000}%
\pgfsetfillcolor{currentfill}%
\pgfsetlinewidth{0.602250pt}%
\definecolor{currentstroke}{rgb}{0.000000,0.000000,0.000000}%
\pgfsetstrokecolor{currentstroke}%
\pgfsetdash{}{0pt}%
\pgfsys@defobject{currentmarker}{\pgfqpoint{-0.027778in}{0.000000in}}{\pgfqpoint{-0.000000in}{0.000000in}}{%
\pgfpathmoveto{\pgfqpoint{-0.000000in}{0.000000in}}%
\pgfpathlineto{\pgfqpoint{-0.027778in}{0.000000in}}%
\pgfusepath{stroke,fill}%
}%
\begin{pgfscope}%
\pgfsys@transformshift{0.800000in}{2.208421in}%
\pgfsys@useobject{currentmarker}{}%
\end{pgfscope}%
\end{pgfscope}%
\begin{pgfscope}%
\pgfsetbuttcap%
\pgfsetroundjoin%
\definecolor{currentfill}{rgb}{0.000000,0.000000,0.000000}%
\pgfsetfillcolor{currentfill}%
\pgfsetlinewidth{0.602250pt}%
\definecolor{currentstroke}{rgb}{0.000000,0.000000,0.000000}%
\pgfsetstrokecolor{currentstroke}%
\pgfsetdash{}{0pt}%
\pgfsys@defobject{currentmarker}{\pgfqpoint{-0.027778in}{0.000000in}}{\pgfqpoint{-0.000000in}{0.000000in}}{%
\pgfpathmoveto{\pgfqpoint{-0.000000in}{0.000000in}}%
\pgfpathlineto{\pgfqpoint{-0.027778in}{0.000000in}}%
\pgfusepath{stroke,fill}%
}%
\begin{pgfscope}%
\pgfsys@transformshift{0.800000in}{2.289956in}%
\pgfsys@useobject{currentmarker}{}%
\end{pgfscope}%
\end{pgfscope}%
\begin{pgfscope}%
\pgfsetbuttcap%
\pgfsetroundjoin%
\definecolor{currentfill}{rgb}{0.000000,0.000000,0.000000}%
\pgfsetfillcolor{currentfill}%
\pgfsetlinewidth{0.602250pt}%
\definecolor{currentstroke}{rgb}{0.000000,0.000000,0.000000}%
\pgfsetstrokecolor{currentstroke}%
\pgfsetdash{}{0pt}%
\pgfsys@defobject{currentmarker}{\pgfqpoint{-0.027778in}{0.000000in}}{\pgfqpoint{-0.000000in}{0.000000in}}{%
\pgfpathmoveto{\pgfqpoint{-0.000000in}{0.000000in}}%
\pgfpathlineto{\pgfqpoint{-0.027778in}{0.000000in}}%
\pgfusepath{stroke,fill}%
}%
\begin{pgfscope}%
\pgfsys@transformshift{0.800000in}{2.842720in}%
\pgfsys@useobject{currentmarker}{}%
\end{pgfscope}%
\end{pgfscope}%
\begin{pgfscope}%
\pgfsetbuttcap%
\pgfsetroundjoin%
\definecolor{currentfill}{rgb}{0.000000,0.000000,0.000000}%
\pgfsetfillcolor{currentfill}%
\pgfsetlinewidth{0.602250pt}%
\definecolor{currentstroke}{rgb}{0.000000,0.000000,0.000000}%
\pgfsetstrokecolor{currentstroke}%
\pgfsetdash{}{0pt}%
\pgfsys@defobject{currentmarker}{\pgfqpoint{-0.027778in}{0.000000in}}{\pgfqpoint{-0.000000in}{0.000000in}}{%
\pgfpathmoveto{\pgfqpoint{-0.000000in}{0.000000in}}%
\pgfpathlineto{\pgfqpoint{-0.027778in}{0.000000in}}%
\pgfusepath{stroke,fill}%
}%
\begin{pgfscope}%
\pgfsys@transformshift{0.800000in}{3.123401in}%
\pgfsys@useobject{currentmarker}{}%
\end{pgfscope}%
\end{pgfscope}%
\begin{pgfscope}%
\pgfsetbuttcap%
\pgfsetroundjoin%
\definecolor{currentfill}{rgb}{0.000000,0.000000,0.000000}%
\pgfsetfillcolor{currentfill}%
\pgfsetlinewidth{0.602250pt}%
\definecolor{currentstroke}{rgb}{0.000000,0.000000,0.000000}%
\pgfsetstrokecolor{currentstroke}%
\pgfsetdash{}{0pt}%
\pgfsys@defobject{currentmarker}{\pgfqpoint{-0.027778in}{0.000000in}}{\pgfqpoint{-0.000000in}{0.000000in}}{%
\pgfpathmoveto{\pgfqpoint{-0.000000in}{0.000000in}}%
\pgfpathlineto{\pgfqpoint{-0.027778in}{0.000000in}}%
\pgfusepath{stroke,fill}%
}%
\begin{pgfscope}%
\pgfsys@transformshift{0.800000in}{3.322548in}%
\pgfsys@useobject{currentmarker}{}%
\end{pgfscope}%
\end{pgfscope}%
\begin{pgfscope}%
\pgfsetbuttcap%
\pgfsetroundjoin%
\definecolor{currentfill}{rgb}{0.000000,0.000000,0.000000}%
\pgfsetfillcolor{currentfill}%
\pgfsetlinewidth{0.602250pt}%
\definecolor{currentstroke}{rgb}{0.000000,0.000000,0.000000}%
\pgfsetstrokecolor{currentstroke}%
\pgfsetdash{}{0pt}%
\pgfsys@defobject{currentmarker}{\pgfqpoint{-0.027778in}{0.000000in}}{\pgfqpoint{-0.000000in}{0.000000in}}{%
\pgfpathmoveto{\pgfqpoint{-0.000000in}{0.000000in}}%
\pgfpathlineto{\pgfqpoint{-0.027778in}{0.000000in}}%
\pgfusepath{stroke,fill}%
}%
\begin{pgfscope}%
\pgfsys@transformshift{0.800000in}{3.477018in}%
\pgfsys@useobject{currentmarker}{}%
\end{pgfscope}%
\end{pgfscope}%
\begin{pgfscope}%
\pgfsetbuttcap%
\pgfsetroundjoin%
\definecolor{currentfill}{rgb}{0.000000,0.000000,0.000000}%
\pgfsetfillcolor{currentfill}%
\pgfsetlinewidth{0.602250pt}%
\definecolor{currentstroke}{rgb}{0.000000,0.000000,0.000000}%
\pgfsetstrokecolor{currentstroke}%
\pgfsetdash{}{0pt}%
\pgfsys@defobject{currentmarker}{\pgfqpoint{-0.027778in}{0.000000in}}{\pgfqpoint{-0.000000in}{0.000000in}}{%
\pgfpathmoveto{\pgfqpoint{-0.000000in}{0.000000in}}%
\pgfpathlineto{\pgfqpoint{-0.027778in}{0.000000in}}%
\pgfusepath{stroke,fill}%
}%
\begin{pgfscope}%
\pgfsys@transformshift{0.800000in}{3.603230in}%
\pgfsys@useobject{currentmarker}{}%
\end{pgfscope}%
\end{pgfscope}%
\begin{pgfscope}%
\pgfsetbuttcap%
\pgfsetroundjoin%
\definecolor{currentfill}{rgb}{0.000000,0.000000,0.000000}%
\pgfsetfillcolor{currentfill}%
\pgfsetlinewidth{0.602250pt}%
\definecolor{currentstroke}{rgb}{0.000000,0.000000,0.000000}%
\pgfsetstrokecolor{currentstroke}%
\pgfsetdash{}{0pt}%
\pgfsys@defobject{currentmarker}{\pgfqpoint{-0.027778in}{0.000000in}}{\pgfqpoint{-0.000000in}{0.000000in}}{%
\pgfpathmoveto{\pgfqpoint{-0.000000in}{0.000000in}}%
\pgfpathlineto{\pgfqpoint{-0.027778in}{0.000000in}}%
\pgfusepath{stroke,fill}%
}%
\begin{pgfscope}%
\pgfsys@transformshift{0.800000in}{3.709940in}%
\pgfsys@useobject{currentmarker}{}%
\end{pgfscope}%
\end{pgfscope}%
\begin{pgfscope}%
\pgfsetbuttcap%
\pgfsetroundjoin%
\definecolor{currentfill}{rgb}{0.000000,0.000000,0.000000}%
\pgfsetfillcolor{currentfill}%
\pgfsetlinewidth{0.602250pt}%
\definecolor{currentstroke}{rgb}{0.000000,0.000000,0.000000}%
\pgfsetstrokecolor{currentstroke}%
\pgfsetdash{}{0pt}%
\pgfsys@defobject{currentmarker}{\pgfqpoint{-0.027778in}{0.000000in}}{\pgfqpoint{-0.000000in}{0.000000in}}{%
\pgfpathmoveto{\pgfqpoint{-0.000000in}{0.000000in}}%
\pgfpathlineto{\pgfqpoint{-0.027778in}{0.000000in}}%
\pgfusepath{stroke,fill}%
}%
\begin{pgfscope}%
\pgfsys@transformshift{0.800000in}{3.802376in}%
\pgfsys@useobject{currentmarker}{}%
\end{pgfscope}%
\end{pgfscope}%
\begin{pgfscope}%
\pgfsetbuttcap%
\pgfsetroundjoin%
\definecolor{currentfill}{rgb}{0.000000,0.000000,0.000000}%
\pgfsetfillcolor{currentfill}%
\pgfsetlinewidth{0.602250pt}%
\definecolor{currentstroke}{rgb}{0.000000,0.000000,0.000000}%
\pgfsetstrokecolor{currentstroke}%
\pgfsetdash{}{0pt}%
\pgfsys@defobject{currentmarker}{\pgfqpoint{-0.027778in}{0.000000in}}{\pgfqpoint{-0.000000in}{0.000000in}}{%
\pgfpathmoveto{\pgfqpoint{-0.000000in}{0.000000in}}%
\pgfpathlineto{\pgfqpoint{-0.027778in}{0.000000in}}%
\pgfusepath{stroke,fill}%
}%
\begin{pgfscope}%
\pgfsys@transformshift{0.800000in}{3.883911in}%
\pgfsys@useobject{currentmarker}{}%
\end{pgfscope}%
\end{pgfscope}%
\begin{pgfscope}%
\definecolor{textcolor}{rgb}{0.000000,0.000000,0.000000}%
\pgfsetstrokecolor{textcolor}%
\pgfsetfillcolor{textcolor}%
\pgftext[x=0.446026in,y=2.376000in,,bottom,rotate=90.000000]{\color{textcolor}\rmfamily\fontsize{10.000000}{12.000000}\selectfont Swaps}%
\end{pgfscope}%
\begin{pgfscope}%
\pgfpathrectangle{\pgfqpoint{0.800000in}{0.528000in}}{\pgfqpoint{4.960000in}{3.696000in}}%
\pgfusepath{clip}%
\pgfsetrectcap%
\pgfsetroundjoin%
\pgfsetlinewidth{1.505625pt}%
\definecolor{currentstroke}{rgb}{0.121569,0.466667,0.705882}%
\pgfsetstrokecolor{currentstroke}%
\pgfsetdash{}{0pt}%
\pgfpathmoveto{\pgfqpoint{1.025455in}{0.696000in}}%
\pgfpathlineto{\pgfqpoint{1.071466in}{1.213256in}}%
\pgfpathlineto{\pgfqpoint{1.117477in}{1.505977in}}%
\pgfpathlineto{\pgfqpoint{1.163488in}{1.711066in}}%
\pgfpathlineto{\pgfqpoint{1.209499in}{1.869077in}}%
\pgfpathlineto{\pgfqpoint{1.255510in}{1.997639in}}%
\pgfpathlineto{\pgfqpoint{1.301521in}{2.106024in}}%
\pgfpathlineto{\pgfqpoint{1.347532in}{2.199713in}}%
\pgfpathlineto{\pgfqpoint{1.393544in}{2.282221in}}%
\pgfpathlineto{\pgfqpoint{1.439555in}{2.355934in}}%
\pgfpathlineto{\pgfqpoint{1.485566in}{2.422547in}}%
\pgfpathlineto{\pgfqpoint{1.531577in}{2.483310in}}%
\pgfpathlineto{\pgfqpoint{1.577588in}{2.539166in}}%
\pgfpathlineto{\pgfqpoint{1.623599in}{2.590850in}}%
\pgfpathlineto{\pgfqpoint{1.669610in}{2.638942in}}%
\pgfpathlineto{\pgfqpoint{1.715622in}{2.683909in}}%
\pgfpathlineto{\pgfqpoint{1.761633in}{2.726132in}}%
\pgfpathlineto{\pgfqpoint{1.807644in}{2.765928in}}%
\pgfpathlineto{\pgfqpoint{1.853655in}{2.803559in}}%
\pgfpathlineto{\pgfqpoint{1.899666in}{2.839250in}}%
\pgfpathlineto{\pgfqpoint{1.945677in}{2.873190in}}%
\pgfpathlineto{\pgfqpoint{1.991688in}{2.905544in}}%
\pgfpathlineto{\pgfqpoint{2.037699in}{2.936453in}}%
\pgfpathlineto{\pgfqpoint{2.083711in}{2.966041in}}%
\pgfpathlineto{\pgfqpoint{2.129722in}{2.994415in}}%
\pgfpathlineto{\pgfqpoint{2.175733in}{3.021673in}}%
\pgfpathlineto{\pgfqpoint{2.221744in}{3.047897in}}%
\pgfpathlineto{\pgfqpoint{2.267755in}{3.073164in}}%
\pgfpathlineto{\pgfqpoint{2.313766in}{3.097542in}}%
\pgfpathlineto{\pgfqpoint{2.359777in}{3.121090in}}%
\pgfpathlineto{\pgfqpoint{2.405788in}{3.143863in}}%
\pgfpathlineto{\pgfqpoint{2.451800in}{3.165911in}}%
\pgfpathlineto{\pgfqpoint{2.497811in}{3.187278in}}%
\pgfpathlineto{\pgfqpoint{2.543822in}{3.208006in}}%
\pgfpathlineto{\pgfqpoint{2.589833in}{3.228131in}}%
\pgfpathlineto{\pgfqpoint{2.635844in}{3.247687in}}%
\pgfpathlineto{\pgfqpoint{2.681855in}{3.266706in}}%
\pgfpathlineto{\pgfqpoint{2.727866in}{3.285216in}}%
\pgfpathlineto{\pgfqpoint{2.773878in}{3.303245in}}%
\pgfpathlineto{\pgfqpoint{2.819889in}{3.320815in}}%
\pgfpathlineto{\pgfqpoint{2.865900in}{3.337951in}}%
\pgfpathlineto{\pgfqpoint{2.911911in}{3.354673in}}%
\pgfpathlineto{\pgfqpoint{2.957922in}{3.371000in}}%
\pgfpathlineto{\pgfqpoint{3.003933in}{3.386951in}}%
\pgfpathlineto{\pgfqpoint{3.049944in}{3.402543in}}%
\pgfpathlineto{\pgfqpoint{3.095955in}{3.417791in}}%
\pgfpathlineto{\pgfqpoint{3.141967in}{3.432711in}}%
\pgfpathlineto{\pgfqpoint{3.187978in}{3.447316in}}%
\pgfpathlineto{\pgfqpoint{3.233989in}{3.461619in}}%
\pgfpathlineto{\pgfqpoint{3.280000in}{3.475632in}}%
\pgfpathlineto{\pgfqpoint{3.326011in}{3.489368in}}%
\pgfpathlineto{\pgfqpoint{3.372022in}{3.502836in}}%
\pgfpathlineto{\pgfqpoint{3.418033in}{3.516047in}}%
\pgfpathlineto{\pgfqpoint{3.464045in}{3.529011in}}%
\pgfpathlineto{\pgfqpoint{3.510056in}{3.541737in}}%
\pgfpathlineto{\pgfqpoint{3.556067in}{3.554232in}}%
\pgfpathlineto{\pgfqpoint{3.602078in}{3.566507in}}%
\pgfpathlineto{\pgfqpoint{3.648089in}{3.578567in}}%
\pgfpathlineto{\pgfqpoint{3.694100in}{3.590421in}}%
\pgfpathlineto{\pgfqpoint{3.740111in}{3.602075in}}%
\pgfpathlineto{\pgfqpoint{3.786122in}{3.613536in}}%
\pgfpathlineto{\pgfqpoint{3.832134in}{3.624811in}}%
\pgfpathlineto{\pgfqpoint{3.878145in}{3.635905in}}%
\pgfpathlineto{\pgfqpoint{3.924156in}{3.646824in}}%
\pgfpathlineto{\pgfqpoint{3.970167in}{3.657573in}}%
\pgfpathlineto{\pgfqpoint{4.016178in}{3.668158in}}%
\pgfpathlineto{\pgfqpoint{4.062189in}{3.678584in}}%
\pgfpathlineto{\pgfqpoint{4.108200in}{3.688855in}}%
\pgfpathlineto{\pgfqpoint{4.154212in}{3.698975in}}%
\pgfpathlineto{\pgfqpoint{4.200223in}{3.708950in}}%
\pgfpathlineto{\pgfqpoint{4.246234in}{3.718783in}}%
\pgfpathlineto{\pgfqpoint{4.292245in}{3.728479in}}%
\pgfpathlineto{\pgfqpoint{4.338256in}{3.738040in}}%
\pgfpathlineto{\pgfqpoint{4.384267in}{3.747472in}}%
\pgfpathlineto{\pgfqpoint{4.430278in}{3.756776in}}%
\pgfpathlineto{\pgfqpoint{4.476289in}{3.765957in}}%
\pgfpathlineto{\pgfqpoint{4.522301in}{3.775018in}}%
\pgfpathlineto{\pgfqpoint{4.568312in}{3.783962in}}%
\pgfpathlineto{\pgfqpoint{4.614323in}{3.792792in}}%
\pgfpathlineto{\pgfqpoint{4.660334in}{3.801511in}}%
\pgfpathlineto{\pgfqpoint{4.706345in}{3.810121in}}%
\pgfpathlineto{\pgfqpoint{4.752356in}{3.818625in}}%
\pgfpathlineto{\pgfqpoint{4.798367in}{3.827026in}}%
\pgfpathlineto{\pgfqpoint{4.844378in}{3.835327in}}%
\pgfpathlineto{\pgfqpoint{4.890390in}{3.843529in}}%
\pgfpathlineto{\pgfqpoint{4.936401in}{3.851635in}}%
\pgfpathlineto{\pgfqpoint{4.982412in}{3.859647in}}%
\pgfpathlineto{\pgfqpoint{5.028423in}{3.867567in}}%
\pgfpathlineto{\pgfqpoint{5.074434in}{3.875398in}}%
\pgfpathlineto{\pgfqpoint{5.120445in}{3.883142in}}%
\pgfpathlineto{\pgfqpoint{5.166456in}{3.890799in}}%
\pgfpathlineto{\pgfqpoint{5.212468in}{3.898373in}}%
\pgfpathlineto{\pgfqpoint{5.258479in}{3.905865in}}%
\pgfpathlineto{\pgfqpoint{5.304490in}{3.913277in}}%
\pgfpathlineto{\pgfqpoint{5.350501in}{3.920610in}}%
\pgfpathlineto{\pgfqpoint{5.396512in}{3.927866in}}%
\pgfpathlineto{\pgfqpoint{5.442523in}{3.935047in}}%
\pgfpathlineto{\pgfqpoint{5.488534in}{3.942155in}}%
\pgfpathlineto{\pgfqpoint{5.534545in}{3.949190in}}%
\pgfusepath{stroke}%
\end{pgfscope}%
\begin{pgfscope}%
\pgfpathrectangle{\pgfqpoint{0.800000in}{0.528000in}}{\pgfqpoint{4.960000in}{3.696000in}}%
\pgfusepath{clip}%
\pgfsetrectcap%
\pgfsetroundjoin%
\pgfsetlinewidth{1.505625pt}%
\definecolor{currentstroke}{rgb}{1.000000,0.498039,0.054902}%
\pgfsetstrokecolor{currentstroke}%
\pgfsetdash{}{0pt}%
\pgfpathmoveto{\pgfqpoint{1.025455in}{0.768935in}}%
\pgfpathlineto{\pgfqpoint{1.071466in}{1.314742in}}%
\pgfpathlineto{\pgfqpoint{1.117477in}{1.616089in}}%
\pgfpathlineto{\pgfqpoint{1.163488in}{1.810127in}}%
\pgfpathlineto{\pgfqpoint{1.209499in}{1.973766in}}%
\pgfpathlineto{\pgfqpoint{1.255510in}{2.106024in}}%
\pgfpathlineto{\pgfqpoint{1.301521in}{2.208421in}}%
\pgfpathlineto{\pgfqpoint{1.347532in}{2.305170in}}%
\pgfpathlineto{\pgfqpoint{1.393544in}{2.390041in}}%
\pgfpathlineto{\pgfqpoint{1.439555in}{2.459641in}}%
\pgfpathlineto{\pgfqpoint{1.485566in}{2.528350in}}%
\pgfpathlineto{\pgfqpoint{1.531577in}{2.590850in}}%
\pgfpathlineto{\pgfqpoint{1.577588in}{2.643573in}}%
\pgfpathlineto{\pgfqpoint{1.623599in}{2.696849in}}%
\pgfpathlineto{\pgfqpoint{1.669610in}{2.746316in}}%
\pgfpathlineto{\pgfqpoint{1.715622in}{2.788751in}}%
\pgfpathlineto{\pgfqpoint{1.761633in}{2.832257in}}%
\pgfpathlineto{\pgfqpoint{1.807644in}{2.873190in}}%
\pgfpathlineto{\pgfqpoint{1.853655in}{2.908698in}}%
\pgfpathlineto{\pgfqpoint{1.899666in}{2.945463in}}%
\pgfpathlineto{\pgfqpoint{1.945677in}{2.980373in}}%
\pgfpathlineto{\pgfqpoint{1.991688in}{3.010898in}}%
\pgfpathlineto{\pgfqpoint{2.037699in}{3.042731in}}%
\pgfpathlineto{\pgfqpoint{2.083711in}{3.073164in}}%
\pgfpathlineto{\pgfqpoint{2.129722in}{3.099933in}}%
\pgfpathlineto{\pgfqpoint{2.175733in}{3.128001in}}%
\pgfpathlineto{\pgfqpoint{2.221744in}{3.154975in}}%
\pgfpathlineto{\pgfqpoint{2.267755in}{3.178810in}}%
\pgfpathlineto{\pgfqpoint{2.313766in}{3.203910in}}%
\pgfpathlineto{\pgfqpoint{2.359777in}{3.228131in}}%
\pgfpathlineto{\pgfqpoint{2.405788in}{3.249613in}}%
\pgfpathlineto{\pgfqpoint{2.451800in}{3.272311in}}%
\pgfpathlineto{\pgfqpoint{2.497811in}{3.294289in}}%
\pgfpathlineto{\pgfqpoint{2.543822in}{3.313840in}}%
\pgfpathlineto{\pgfqpoint{2.589833in}{3.334558in}}%
\pgfpathlineto{\pgfqpoint{2.635844in}{3.354673in}}%
\pgfpathlineto{\pgfqpoint{2.681855in}{3.372612in}}%
\pgfpathlineto{\pgfqpoint{2.727866in}{3.391666in}}%
\pgfpathlineto{\pgfqpoint{2.773878in}{3.410209in}}%
\pgfpathlineto{\pgfqpoint{2.819889in}{3.426781in}}%
\pgfpathlineto{\pgfqpoint{2.865900in}{3.444419in}}%
\pgfpathlineto{\pgfqpoint{2.911911in}{3.461619in}}%
\pgfpathlineto{\pgfqpoint{2.957922in}{3.477018in}}%
\pgfpathlineto{\pgfqpoint{3.003933in}{3.493436in}}%
\pgfpathlineto{\pgfqpoint{3.049944in}{3.509473in}}%
\pgfpathlineto{\pgfqpoint{3.095955in}{3.523855in}}%
\pgfpathlineto{\pgfqpoint{3.141967in}{3.539210in}}%
\pgfpathlineto{\pgfqpoint{3.187978in}{3.554232in}}%
\pgfpathlineto{\pgfqpoint{3.233989in}{3.567722in}}%
\pgfpathlineto{\pgfqpoint{3.280000in}{3.582144in}}%
\pgfpathlineto{\pgfqpoint{3.326011in}{3.596272in}}%
\pgfpathlineto{\pgfqpoint{3.372022in}{3.608974in}}%
\pgfpathlineto{\pgfqpoint{3.418033in}{3.622571in}}%
\pgfpathlineto{\pgfqpoint{3.464045in}{3.635905in}}%
\pgfpathlineto{\pgfqpoint{3.510056in}{3.647906in}}%
\pgfpathlineto{\pgfqpoint{3.556067in}{3.660766in}}%
\pgfpathlineto{\pgfqpoint{3.602078in}{3.673391in}}%
\pgfpathlineto{\pgfqpoint{3.648089in}{3.684764in}}%
\pgfpathlineto{\pgfqpoint{3.694100in}{3.696963in}}%
\pgfpathlineto{\pgfqpoint{3.740111in}{3.708950in}}%
\pgfpathlineto{\pgfqpoint{3.786122in}{3.719759in}}%
\pgfpathlineto{\pgfqpoint{3.832134in}{3.731361in}}%
\pgfpathlineto{\pgfqpoint{3.878145in}{3.742772in}}%
\pgfpathlineto{\pgfqpoint{3.924156in}{3.753069in}}%
\pgfpathlineto{\pgfqpoint{3.970167in}{3.764131in}}%
\pgfpathlineto{\pgfqpoint{4.016178in}{3.775018in}}%
\pgfpathlineto{\pgfqpoint{4.062189in}{3.784850in}}%
\pgfpathlineto{\pgfqpoint{4.108200in}{3.795419in}}%
\pgfpathlineto{\pgfqpoint{4.154212in}{3.805829in}}%
\pgfpathlineto{\pgfqpoint{4.200223in}{3.815236in}}%
\pgfpathlineto{\pgfqpoint{4.246234in}{3.825354in}}%
\pgfpathlineto{\pgfqpoint{4.292245in}{3.835327in}}%
\pgfpathlineto{\pgfqpoint{4.338256in}{3.844344in}}%
\pgfpathlineto{\pgfqpoint{4.384267in}{3.854048in}}%
\pgfpathlineto{\pgfqpoint{4.430278in}{3.863619in}}%
\pgfpathlineto{\pgfqpoint{4.476289in}{3.872277in}}%
\pgfpathlineto{\pgfqpoint{4.522301in}{3.881600in}}%
\pgfpathlineto{\pgfqpoint{4.568312in}{3.890799in}}%
\pgfpathlineto{\pgfqpoint{4.614323in}{3.899126in}}%
\pgfpathlineto{\pgfqpoint{4.660334in}{3.908097in}}%
\pgfpathlineto{\pgfqpoint{4.706345in}{3.916953in}}%
\pgfpathlineto{\pgfqpoint{4.752356in}{3.924973in}}%
\pgfpathlineto{\pgfqpoint{4.798367in}{3.933617in}}%
\pgfpathlineto{\pgfqpoint{4.844378in}{3.942155in}}%
\pgfpathlineto{\pgfqpoint{4.890390in}{3.949889in}}%
\pgfpathlineto{\pgfqpoint{4.936401in}{3.958230in}}%
\pgfpathlineto{\pgfqpoint{4.982412in}{3.966471in}}%
\pgfpathlineto{\pgfqpoint{5.028423in}{3.973940in}}%
\pgfpathlineto{\pgfqpoint{5.074434in}{3.981997in}}%
\pgfpathlineto{\pgfqpoint{5.120445in}{3.989962in}}%
\pgfpathlineto{\pgfqpoint{5.166456in}{3.997183in}}%
\pgfpathlineto{\pgfqpoint{5.212468in}{4.004976in}}%
\pgfpathlineto{\pgfqpoint{5.258479in}{4.012682in}}%
\pgfpathlineto{\pgfqpoint{5.304490in}{4.019671in}}%
\pgfpathlineto{\pgfqpoint{5.350501in}{4.027216in}}%
\pgfpathlineto{\pgfqpoint{5.396512in}{4.034680in}}%
\pgfpathlineto{\pgfqpoint{5.442523in}{4.041451in}}%
\pgfpathlineto{\pgfqpoint{5.488534in}{4.048764in}}%
\pgfpathlineto{\pgfqpoint{5.534545in}{4.056000in}}%
\pgfusepath{stroke}%
\end{pgfscope}%
\begin{pgfscope}%
\pgfsetrectcap%
\pgfsetmiterjoin%
\pgfsetlinewidth{0.803000pt}%
\definecolor{currentstroke}{rgb}{0.000000,0.000000,0.000000}%
\pgfsetstrokecolor{currentstroke}%
\pgfsetdash{}{0pt}%
\pgfpathmoveto{\pgfqpoint{0.800000in}{0.528000in}}%
\pgfpathlineto{\pgfqpoint{0.800000in}{4.224000in}}%
\pgfusepath{stroke}%
\end{pgfscope}%
\begin{pgfscope}%
\pgfsetrectcap%
\pgfsetmiterjoin%
\pgfsetlinewidth{0.803000pt}%
\definecolor{currentstroke}{rgb}{0.000000,0.000000,0.000000}%
\pgfsetstrokecolor{currentstroke}%
\pgfsetdash{}{0pt}%
\pgfpathmoveto{\pgfqpoint{5.760000in}{0.528000in}}%
\pgfpathlineto{\pgfqpoint{5.760000in}{4.224000in}}%
\pgfusepath{stroke}%
\end{pgfscope}%
\begin{pgfscope}%
\pgfsetrectcap%
\pgfsetmiterjoin%
\pgfsetlinewidth{0.803000pt}%
\definecolor{currentstroke}{rgb}{0.000000,0.000000,0.000000}%
\pgfsetstrokecolor{currentstroke}%
\pgfsetdash{}{0pt}%
\pgfpathmoveto{\pgfqpoint{0.800000in}{0.528000in}}%
\pgfpathlineto{\pgfqpoint{5.760000in}{0.528000in}}%
\pgfusepath{stroke}%
\end{pgfscope}%
\begin{pgfscope}%
\pgfsetrectcap%
\pgfsetmiterjoin%
\pgfsetlinewidth{0.803000pt}%
\definecolor{currentstroke}{rgb}{0.000000,0.000000,0.000000}%
\pgfsetstrokecolor{currentstroke}%
\pgfsetdash{}{0pt}%
\pgfpathmoveto{\pgfqpoint{0.800000in}{4.224000in}}%
\pgfpathlineto{\pgfqpoint{5.760000in}{4.224000in}}%
\pgfusepath{stroke}%
\end{pgfscope}%
\begin{pgfscope}%
\pgfsetbuttcap%
\pgfsetmiterjoin%
\definecolor{currentfill}{rgb}{1.000000,1.000000,1.000000}%
\pgfsetfillcolor{currentfill}%
\pgfsetfillopacity{0.800000}%
\pgfsetlinewidth{1.003750pt}%
\definecolor{currentstroke}{rgb}{0.800000,0.800000,0.800000}%
\pgfsetstrokecolor{currentstroke}%
\pgfsetstrokeopacity{0.800000}%
\pgfsetdash{}{0pt}%
\pgfpathmoveto{\pgfqpoint{0.897222in}{3.725543in}}%
\pgfpathlineto{\pgfqpoint{1.975542in}{3.725543in}}%
\pgfpathquadraticcurveto{\pgfqpoint{2.003319in}{3.725543in}}{\pgfqpoint{2.003319in}{3.753321in}}%
\pgfpathlineto{\pgfqpoint{2.003319in}{4.126778in}}%
\pgfpathquadraticcurveto{\pgfqpoint{2.003319in}{4.154556in}}{\pgfqpoint{1.975542in}{4.154556in}}%
\pgfpathlineto{\pgfqpoint{0.897222in}{4.154556in}}%
\pgfpathquadraticcurveto{\pgfqpoint{0.869444in}{4.154556in}}{\pgfqpoint{0.869444in}{4.126778in}}%
\pgfpathlineto{\pgfqpoint{0.869444in}{3.753321in}}%
\pgfpathquadraticcurveto{\pgfqpoint{0.869444in}{3.725543in}}{\pgfqpoint{0.897222in}{3.725543in}}%
\pgfpathlineto{\pgfqpoint{0.897222in}{3.725543in}}%
\pgfpathclose%
\pgfusepath{stroke,fill}%
\end{pgfscope}%
\begin{pgfscope}%
\pgfsetrectcap%
\pgfsetroundjoin%
\pgfsetlinewidth{1.505625pt}%
\definecolor{currentstroke}{rgb}{0.121569,0.466667,0.705882}%
\pgfsetstrokecolor{currentstroke}%
\pgfsetdash{}{0pt}%
\pgfpathmoveto{\pgfqpoint{0.925000in}{4.050389in}}%
\pgfpathlineto{\pgfqpoint{1.063889in}{4.050389in}}%
\pgfpathlineto{\pgfqpoint{1.202778in}{4.050389in}}%
\pgfusepath{stroke}%
\end{pgfscope}%
\begin{pgfscope}%
\definecolor{textcolor}{rgb}{0.000000,0.000000,0.000000}%
\pgfsetstrokecolor{textcolor}%
\pgfsetfillcolor{textcolor}%
\pgftext[x=1.313889in,y=4.001778in,left,base]{\color{textcolor}\rmfamily\fontsize{10.000000}{12.000000}\selectfont quicksort}%
\end{pgfscope}%
\begin{pgfscope}%
\pgfsetrectcap%
\pgfsetroundjoin%
\pgfsetlinewidth{1.505625pt}%
\definecolor{currentstroke}{rgb}{1.000000,0.498039,0.054902}%
\pgfsetstrokecolor{currentstroke}%
\pgfsetdash{}{0pt}%
\pgfpathmoveto{\pgfqpoint{0.925000in}{3.856716in}}%
\pgfpathlineto{\pgfqpoint{1.063889in}{3.856716in}}%
\pgfpathlineto{\pgfqpoint{1.202778in}{3.856716in}}%
\pgfusepath{stroke}%
\end{pgfscope}%
\begin{pgfscope}%
\definecolor{textcolor}{rgb}{0.000000,0.000000,0.000000}%
\pgfsetstrokecolor{textcolor}%
\pgfsetfillcolor{textcolor}%
\pgftext[x=1.313889in,y=3.808105in,left,base]{\color{textcolor}\rmfamily\fontsize{10.000000}{12.000000}\selectfont bquicksort}%
\end{pgfscope}%
\end{pgfpicture}%
\makeatother%
\endgroup%

\subsubsection{Basic Operations}
%% Creator: Matplotlib, PGF backend
%%
%% To include the figure in your LaTeX document, write
%%   \input{<filename>.pgf}
%%
%% Make sure the required packages are loaded in your preamble
%%   \usepackage{pgf}
%%
%% Also ensure that all the required font packages are loaded; for instance,
%% the lmodern package is sometimes necessary when using math font.
%%   \usepackage{lmodern}
%%
%% Figures using additional raster images can only be included by \input if
%% they are in the same directory as the main LaTeX file. For loading figures
%% from other directories you can use the `import` package
%%   \usepackage{import}
%%
%% and then include the figures with
%%   \import{<path to file>}{<filename>.pgf}
%%
%% Matplotlib used the following preamble
%%   
%%   \makeatletter\@ifpackageloaded{underscore}{}{\usepackage[strings]{underscore}}\makeatother
%%
\begingroup%
\makeatletter%
\begin{pgfpicture}%
\pgfpathrectangle{\pgfpointorigin}{\pgfqpoint{6.400000in}{4.800000in}}%
\pgfusepath{use as bounding box, clip}%
\begin{pgfscope}%
\pgfsetbuttcap%
\pgfsetmiterjoin%
\definecolor{currentfill}{rgb}{1.000000,1.000000,1.000000}%
\pgfsetfillcolor{currentfill}%
\pgfsetlinewidth{0.000000pt}%
\definecolor{currentstroke}{rgb}{1.000000,1.000000,1.000000}%
\pgfsetstrokecolor{currentstroke}%
\pgfsetdash{}{0pt}%
\pgfpathmoveto{\pgfqpoint{0.000000in}{0.000000in}}%
\pgfpathlineto{\pgfqpoint{6.400000in}{0.000000in}}%
\pgfpathlineto{\pgfqpoint{6.400000in}{4.800000in}}%
\pgfpathlineto{\pgfqpoint{0.000000in}{4.800000in}}%
\pgfpathlineto{\pgfqpoint{0.000000in}{0.000000in}}%
\pgfpathclose%
\pgfusepath{fill}%
\end{pgfscope}%
\begin{pgfscope}%
\pgfsetbuttcap%
\pgfsetmiterjoin%
\definecolor{currentfill}{rgb}{1.000000,1.000000,1.000000}%
\pgfsetfillcolor{currentfill}%
\pgfsetlinewidth{0.000000pt}%
\definecolor{currentstroke}{rgb}{0.000000,0.000000,0.000000}%
\pgfsetstrokecolor{currentstroke}%
\pgfsetstrokeopacity{0.000000}%
\pgfsetdash{}{0pt}%
\pgfpathmoveto{\pgfqpoint{0.800000in}{0.528000in}}%
\pgfpathlineto{\pgfqpoint{5.760000in}{0.528000in}}%
\pgfpathlineto{\pgfqpoint{5.760000in}{4.224000in}}%
\pgfpathlineto{\pgfqpoint{0.800000in}{4.224000in}}%
\pgfpathlineto{\pgfqpoint{0.800000in}{0.528000in}}%
\pgfpathclose%
\pgfusepath{fill}%
\end{pgfscope}%
\begin{pgfscope}%
\pgfsetbuttcap%
\pgfsetroundjoin%
\definecolor{currentfill}{rgb}{0.000000,0.000000,0.000000}%
\pgfsetfillcolor{currentfill}%
\pgfsetlinewidth{0.803000pt}%
\definecolor{currentstroke}{rgb}{0.000000,0.000000,0.000000}%
\pgfsetstrokecolor{currentstroke}%
\pgfsetdash{}{0pt}%
\pgfsys@defobject{currentmarker}{\pgfqpoint{0.000000in}{-0.048611in}}{\pgfqpoint{0.000000in}{0.000000in}}{%
\pgfpathmoveto{\pgfqpoint{0.000000in}{0.000000in}}%
\pgfpathlineto{\pgfqpoint{0.000000in}{-0.048611in}}%
\pgfusepath{stroke,fill}%
}%
\begin{pgfscope}%
\pgfsys@transformshift{1.532094in}{0.528000in}%
\pgfsys@useobject{currentmarker}{}%
\end{pgfscope}%
\end{pgfscope}%
\begin{pgfscope}%
\definecolor{textcolor}{rgb}{0.000000,0.000000,0.000000}%
\pgfsetstrokecolor{textcolor}%
\pgfsetfillcolor{textcolor}%
\pgftext[x=1.532094in,y=0.430778in,,top]{\color{textcolor}\rmfamily\fontsize{10.000000}{12.000000}\selectfont \(\displaystyle {200}\)}%
\end{pgfscope}%
\begin{pgfscope}%
\pgfsetbuttcap%
\pgfsetroundjoin%
\definecolor{currentfill}{rgb}{0.000000,0.000000,0.000000}%
\pgfsetfillcolor{currentfill}%
\pgfsetlinewidth{0.803000pt}%
\definecolor{currentstroke}{rgb}{0.000000,0.000000,0.000000}%
\pgfsetstrokecolor{currentstroke}%
\pgfsetdash{}{0pt}%
\pgfsys@defobject{currentmarker}{\pgfqpoint{0.000000in}{-0.048611in}}{\pgfqpoint{0.000000in}{0.000000in}}{%
\pgfpathmoveto{\pgfqpoint{0.000000in}{0.000000in}}%
\pgfpathlineto{\pgfqpoint{0.000000in}{-0.048611in}}%
\pgfusepath{stroke,fill}%
}%
\begin{pgfscope}%
\pgfsys@transformshift{2.545373in}{0.528000in}%
\pgfsys@useobject{currentmarker}{}%
\end{pgfscope}%
\end{pgfscope}%
\begin{pgfscope}%
\definecolor{textcolor}{rgb}{0.000000,0.000000,0.000000}%
\pgfsetstrokecolor{textcolor}%
\pgfsetfillcolor{textcolor}%
\pgftext[x=2.545373in,y=0.430778in,,top]{\color{textcolor}\rmfamily\fontsize{10.000000}{12.000000}\selectfont \(\displaystyle {400}\)}%
\end{pgfscope}%
\begin{pgfscope}%
\pgfsetbuttcap%
\pgfsetroundjoin%
\definecolor{currentfill}{rgb}{0.000000,0.000000,0.000000}%
\pgfsetfillcolor{currentfill}%
\pgfsetlinewidth{0.803000pt}%
\definecolor{currentstroke}{rgb}{0.000000,0.000000,0.000000}%
\pgfsetstrokecolor{currentstroke}%
\pgfsetdash{}{0pt}%
\pgfsys@defobject{currentmarker}{\pgfqpoint{0.000000in}{-0.048611in}}{\pgfqpoint{0.000000in}{0.000000in}}{%
\pgfpathmoveto{\pgfqpoint{0.000000in}{0.000000in}}%
\pgfpathlineto{\pgfqpoint{0.000000in}{-0.048611in}}%
\pgfusepath{stroke,fill}%
}%
\begin{pgfscope}%
\pgfsys@transformshift{3.558652in}{0.528000in}%
\pgfsys@useobject{currentmarker}{}%
\end{pgfscope}%
\end{pgfscope}%
\begin{pgfscope}%
\definecolor{textcolor}{rgb}{0.000000,0.000000,0.000000}%
\pgfsetstrokecolor{textcolor}%
\pgfsetfillcolor{textcolor}%
\pgftext[x=3.558652in,y=0.430778in,,top]{\color{textcolor}\rmfamily\fontsize{10.000000}{12.000000}\selectfont \(\displaystyle {600}\)}%
\end{pgfscope}%
\begin{pgfscope}%
\pgfsetbuttcap%
\pgfsetroundjoin%
\definecolor{currentfill}{rgb}{0.000000,0.000000,0.000000}%
\pgfsetfillcolor{currentfill}%
\pgfsetlinewidth{0.803000pt}%
\definecolor{currentstroke}{rgb}{0.000000,0.000000,0.000000}%
\pgfsetstrokecolor{currentstroke}%
\pgfsetdash{}{0pt}%
\pgfsys@defobject{currentmarker}{\pgfqpoint{0.000000in}{-0.048611in}}{\pgfqpoint{0.000000in}{0.000000in}}{%
\pgfpathmoveto{\pgfqpoint{0.000000in}{0.000000in}}%
\pgfpathlineto{\pgfqpoint{0.000000in}{-0.048611in}}%
\pgfusepath{stroke,fill}%
}%
\begin{pgfscope}%
\pgfsys@transformshift{4.571931in}{0.528000in}%
\pgfsys@useobject{currentmarker}{}%
\end{pgfscope}%
\end{pgfscope}%
\begin{pgfscope}%
\definecolor{textcolor}{rgb}{0.000000,0.000000,0.000000}%
\pgfsetstrokecolor{textcolor}%
\pgfsetfillcolor{textcolor}%
\pgftext[x=4.571931in,y=0.430778in,,top]{\color{textcolor}\rmfamily\fontsize{10.000000}{12.000000}\selectfont \(\displaystyle {800}\)}%
\end{pgfscope}%
\begin{pgfscope}%
\pgfsetbuttcap%
\pgfsetroundjoin%
\definecolor{currentfill}{rgb}{0.000000,0.000000,0.000000}%
\pgfsetfillcolor{currentfill}%
\pgfsetlinewidth{0.803000pt}%
\definecolor{currentstroke}{rgb}{0.000000,0.000000,0.000000}%
\pgfsetstrokecolor{currentstroke}%
\pgfsetdash{}{0pt}%
\pgfsys@defobject{currentmarker}{\pgfqpoint{0.000000in}{-0.048611in}}{\pgfqpoint{0.000000in}{0.000000in}}{%
\pgfpathmoveto{\pgfqpoint{0.000000in}{0.000000in}}%
\pgfpathlineto{\pgfqpoint{0.000000in}{-0.048611in}}%
\pgfusepath{stroke,fill}%
}%
\begin{pgfscope}%
\pgfsys@transformshift{5.585209in}{0.528000in}%
\pgfsys@useobject{currentmarker}{}%
\end{pgfscope}%
\end{pgfscope}%
\begin{pgfscope}%
\definecolor{textcolor}{rgb}{0.000000,0.000000,0.000000}%
\pgfsetstrokecolor{textcolor}%
\pgfsetfillcolor{textcolor}%
\pgftext[x=5.585209in,y=0.430778in,,top]{\color{textcolor}\rmfamily\fontsize{10.000000}{12.000000}\selectfont \(\displaystyle {1000}\)}%
\end{pgfscope}%
\begin{pgfscope}%
\definecolor{textcolor}{rgb}{0.000000,0.000000,0.000000}%
\pgfsetstrokecolor{textcolor}%
\pgfsetfillcolor{textcolor}%
\pgftext[x=3.280000in,y=0.251766in,,top]{\color{textcolor}\rmfamily\fontsize{10.000000}{12.000000}\selectfont Input Size}%
\end{pgfscope}%
\begin{pgfscope}%
\pgfsetbuttcap%
\pgfsetroundjoin%
\definecolor{currentfill}{rgb}{0.000000,0.000000,0.000000}%
\pgfsetfillcolor{currentfill}%
\pgfsetlinewidth{0.803000pt}%
\definecolor{currentstroke}{rgb}{0.000000,0.000000,0.000000}%
\pgfsetstrokecolor{currentstroke}%
\pgfsetdash{}{0pt}%
\pgfsys@defobject{currentmarker}{\pgfqpoint{-0.048611in}{0.000000in}}{\pgfqpoint{-0.000000in}{0.000000in}}{%
\pgfpathmoveto{\pgfqpoint{-0.000000in}{0.000000in}}%
\pgfpathlineto{\pgfqpoint{-0.048611in}{0.000000in}}%
\pgfusepath{stroke,fill}%
}%
\begin{pgfscope}%
\pgfsys@transformshift{0.800000in}{0.581093in}%
\pgfsys@useobject{currentmarker}{}%
\end{pgfscope}%
\end{pgfscope}%
\begin{pgfscope}%
\definecolor{textcolor}{rgb}{0.000000,0.000000,0.000000}%
\pgfsetstrokecolor{textcolor}%
\pgfsetfillcolor{textcolor}%
\pgftext[x=0.501581in, y=0.532868in, left, base]{\color{textcolor}\rmfamily\fontsize{10.000000}{12.000000}\selectfont \(\displaystyle {10^{3}}\)}%
\end{pgfscope}%
\begin{pgfscope}%
\pgfsetbuttcap%
\pgfsetroundjoin%
\definecolor{currentfill}{rgb}{0.000000,0.000000,0.000000}%
\pgfsetfillcolor{currentfill}%
\pgfsetlinewidth{0.803000pt}%
\definecolor{currentstroke}{rgb}{0.000000,0.000000,0.000000}%
\pgfsetstrokecolor{currentstroke}%
\pgfsetdash{}{0pt}%
\pgfsys@defobject{currentmarker}{\pgfqpoint{-0.048611in}{0.000000in}}{\pgfqpoint{-0.000000in}{0.000000in}}{%
\pgfpathmoveto{\pgfqpoint{-0.000000in}{0.000000in}}%
\pgfpathlineto{\pgfqpoint{-0.048611in}{0.000000in}}%
\pgfusepath{stroke,fill}%
}%
\begin{pgfscope}%
\pgfsys@transformshift{0.800000in}{1.677234in}%
\pgfsys@useobject{currentmarker}{}%
\end{pgfscope}%
\end{pgfscope}%
\begin{pgfscope}%
\definecolor{textcolor}{rgb}{0.000000,0.000000,0.000000}%
\pgfsetstrokecolor{textcolor}%
\pgfsetfillcolor{textcolor}%
\pgftext[x=0.501581in, y=1.629009in, left, base]{\color{textcolor}\rmfamily\fontsize{10.000000}{12.000000}\selectfont \(\displaystyle {10^{4}}\)}%
\end{pgfscope}%
\begin{pgfscope}%
\pgfsetbuttcap%
\pgfsetroundjoin%
\definecolor{currentfill}{rgb}{0.000000,0.000000,0.000000}%
\pgfsetfillcolor{currentfill}%
\pgfsetlinewidth{0.803000pt}%
\definecolor{currentstroke}{rgb}{0.000000,0.000000,0.000000}%
\pgfsetstrokecolor{currentstroke}%
\pgfsetdash{}{0pt}%
\pgfsys@defobject{currentmarker}{\pgfqpoint{-0.048611in}{0.000000in}}{\pgfqpoint{-0.000000in}{0.000000in}}{%
\pgfpathmoveto{\pgfqpoint{-0.000000in}{0.000000in}}%
\pgfpathlineto{\pgfqpoint{-0.048611in}{0.000000in}}%
\pgfusepath{stroke,fill}%
}%
\begin{pgfscope}%
\pgfsys@transformshift{0.800000in}{2.773375in}%
\pgfsys@useobject{currentmarker}{}%
\end{pgfscope}%
\end{pgfscope}%
\begin{pgfscope}%
\definecolor{textcolor}{rgb}{0.000000,0.000000,0.000000}%
\pgfsetstrokecolor{textcolor}%
\pgfsetfillcolor{textcolor}%
\pgftext[x=0.501581in, y=2.725150in, left, base]{\color{textcolor}\rmfamily\fontsize{10.000000}{12.000000}\selectfont \(\displaystyle {10^{5}}\)}%
\end{pgfscope}%
\begin{pgfscope}%
\pgfsetbuttcap%
\pgfsetroundjoin%
\definecolor{currentfill}{rgb}{0.000000,0.000000,0.000000}%
\pgfsetfillcolor{currentfill}%
\pgfsetlinewidth{0.803000pt}%
\definecolor{currentstroke}{rgb}{0.000000,0.000000,0.000000}%
\pgfsetstrokecolor{currentstroke}%
\pgfsetdash{}{0pt}%
\pgfsys@defobject{currentmarker}{\pgfqpoint{-0.048611in}{0.000000in}}{\pgfqpoint{-0.000000in}{0.000000in}}{%
\pgfpathmoveto{\pgfqpoint{-0.000000in}{0.000000in}}%
\pgfpathlineto{\pgfqpoint{-0.048611in}{0.000000in}}%
\pgfusepath{stroke,fill}%
}%
\begin{pgfscope}%
\pgfsys@transformshift{0.800000in}{3.869516in}%
\pgfsys@useobject{currentmarker}{}%
\end{pgfscope}%
\end{pgfscope}%
\begin{pgfscope}%
\definecolor{textcolor}{rgb}{0.000000,0.000000,0.000000}%
\pgfsetstrokecolor{textcolor}%
\pgfsetfillcolor{textcolor}%
\pgftext[x=0.501581in, y=3.821291in, left, base]{\color{textcolor}\rmfamily\fontsize{10.000000}{12.000000}\selectfont \(\displaystyle {10^{6}}\)}%
\end{pgfscope}%
\begin{pgfscope}%
\pgfsetbuttcap%
\pgfsetroundjoin%
\definecolor{currentfill}{rgb}{0.000000,0.000000,0.000000}%
\pgfsetfillcolor{currentfill}%
\pgfsetlinewidth{0.602250pt}%
\definecolor{currentstroke}{rgb}{0.000000,0.000000,0.000000}%
\pgfsetstrokecolor{currentstroke}%
\pgfsetdash{}{0pt}%
\pgfsys@defobject{currentmarker}{\pgfqpoint{-0.027778in}{0.000000in}}{\pgfqpoint{-0.000000in}{0.000000in}}{%
\pgfpathmoveto{\pgfqpoint{-0.000000in}{0.000000in}}%
\pgfpathlineto{\pgfqpoint{-0.027778in}{0.000000in}}%
\pgfusepath{stroke,fill}%
}%
\begin{pgfscope}%
\pgfsys@transformshift{0.800000in}{0.530937in}%
\pgfsys@useobject{currentmarker}{}%
\end{pgfscope}%
\end{pgfscope}%
\begin{pgfscope}%
\pgfsetbuttcap%
\pgfsetroundjoin%
\definecolor{currentfill}{rgb}{0.000000,0.000000,0.000000}%
\pgfsetfillcolor{currentfill}%
\pgfsetlinewidth{0.602250pt}%
\definecolor{currentstroke}{rgb}{0.000000,0.000000,0.000000}%
\pgfsetstrokecolor{currentstroke}%
\pgfsetdash{}{0pt}%
\pgfsys@defobject{currentmarker}{\pgfqpoint{-0.027778in}{0.000000in}}{\pgfqpoint{-0.000000in}{0.000000in}}{%
\pgfpathmoveto{\pgfqpoint{-0.000000in}{0.000000in}}%
\pgfpathlineto{\pgfqpoint{-0.027778in}{0.000000in}}%
\pgfusepath{stroke,fill}%
}%
\begin{pgfscope}%
\pgfsys@transformshift{0.800000in}{0.911065in}%
\pgfsys@useobject{currentmarker}{}%
\end{pgfscope}%
\end{pgfscope}%
\begin{pgfscope}%
\pgfsetbuttcap%
\pgfsetroundjoin%
\definecolor{currentfill}{rgb}{0.000000,0.000000,0.000000}%
\pgfsetfillcolor{currentfill}%
\pgfsetlinewidth{0.602250pt}%
\definecolor{currentstroke}{rgb}{0.000000,0.000000,0.000000}%
\pgfsetstrokecolor{currentstroke}%
\pgfsetdash{}{0pt}%
\pgfsys@defobject{currentmarker}{\pgfqpoint{-0.027778in}{0.000000in}}{\pgfqpoint{-0.000000in}{0.000000in}}{%
\pgfpathmoveto{\pgfqpoint{-0.000000in}{0.000000in}}%
\pgfpathlineto{\pgfqpoint{-0.027778in}{0.000000in}}%
\pgfusepath{stroke,fill}%
}%
\begin{pgfscope}%
\pgfsys@transformshift{0.800000in}{1.104085in}%
\pgfsys@useobject{currentmarker}{}%
\end{pgfscope}%
\end{pgfscope}%
\begin{pgfscope}%
\pgfsetbuttcap%
\pgfsetroundjoin%
\definecolor{currentfill}{rgb}{0.000000,0.000000,0.000000}%
\pgfsetfillcolor{currentfill}%
\pgfsetlinewidth{0.602250pt}%
\definecolor{currentstroke}{rgb}{0.000000,0.000000,0.000000}%
\pgfsetstrokecolor{currentstroke}%
\pgfsetdash{}{0pt}%
\pgfsys@defobject{currentmarker}{\pgfqpoint{-0.027778in}{0.000000in}}{\pgfqpoint{-0.000000in}{0.000000in}}{%
\pgfpathmoveto{\pgfqpoint{-0.000000in}{0.000000in}}%
\pgfpathlineto{\pgfqpoint{-0.027778in}{0.000000in}}%
\pgfusepath{stroke,fill}%
}%
\begin{pgfscope}%
\pgfsys@transformshift{0.800000in}{1.241036in}%
\pgfsys@useobject{currentmarker}{}%
\end{pgfscope}%
\end{pgfscope}%
\begin{pgfscope}%
\pgfsetbuttcap%
\pgfsetroundjoin%
\definecolor{currentfill}{rgb}{0.000000,0.000000,0.000000}%
\pgfsetfillcolor{currentfill}%
\pgfsetlinewidth{0.602250pt}%
\definecolor{currentstroke}{rgb}{0.000000,0.000000,0.000000}%
\pgfsetstrokecolor{currentstroke}%
\pgfsetdash{}{0pt}%
\pgfsys@defobject{currentmarker}{\pgfqpoint{-0.027778in}{0.000000in}}{\pgfqpoint{-0.000000in}{0.000000in}}{%
\pgfpathmoveto{\pgfqpoint{-0.000000in}{0.000000in}}%
\pgfpathlineto{\pgfqpoint{-0.027778in}{0.000000in}}%
\pgfusepath{stroke,fill}%
}%
\begin{pgfscope}%
\pgfsys@transformshift{0.800000in}{1.347263in}%
\pgfsys@useobject{currentmarker}{}%
\end{pgfscope}%
\end{pgfscope}%
\begin{pgfscope}%
\pgfsetbuttcap%
\pgfsetroundjoin%
\definecolor{currentfill}{rgb}{0.000000,0.000000,0.000000}%
\pgfsetfillcolor{currentfill}%
\pgfsetlinewidth{0.602250pt}%
\definecolor{currentstroke}{rgb}{0.000000,0.000000,0.000000}%
\pgfsetstrokecolor{currentstroke}%
\pgfsetdash{}{0pt}%
\pgfsys@defobject{currentmarker}{\pgfqpoint{-0.027778in}{0.000000in}}{\pgfqpoint{-0.000000in}{0.000000in}}{%
\pgfpathmoveto{\pgfqpoint{-0.000000in}{0.000000in}}%
\pgfpathlineto{\pgfqpoint{-0.027778in}{0.000000in}}%
\pgfusepath{stroke,fill}%
}%
\begin{pgfscope}%
\pgfsys@transformshift{0.800000in}{1.434057in}%
\pgfsys@useobject{currentmarker}{}%
\end{pgfscope}%
\end{pgfscope}%
\begin{pgfscope}%
\pgfsetbuttcap%
\pgfsetroundjoin%
\definecolor{currentfill}{rgb}{0.000000,0.000000,0.000000}%
\pgfsetfillcolor{currentfill}%
\pgfsetlinewidth{0.602250pt}%
\definecolor{currentstroke}{rgb}{0.000000,0.000000,0.000000}%
\pgfsetstrokecolor{currentstroke}%
\pgfsetdash{}{0pt}%
\pgfsys@defobject{currentmarker}{\pgfqpoint{-0.027778in}{0.000000in}}{\pgfqpoint{-0.000000in}{0.000000in}}{%
\pgfpathmoveto{\pgfqpoint{-0.000000in}{0.000000in}}%
\pgfpathlineto{\pgfqpoint{-0.027778in}{0.000000in}}%
\pgfusepath{stroke,fill}%
}%
\begin{pgfscope}%
\pgfsys@transformshift{0.800000in}{1.507440in}%
\pgfsys@useobject{currentmarker}{}%
\end{pgfscope}%
\end{pgfscope}%
\begin{pgfscope}%
\pgfsetbuttcap%
\pgfsetroundjoin%
\definecolor{currentfill}{rgb}{0.000000,0.000000,0.000000}%
\pgfsetfillcolor{currentfill}%
\pgfsetlinewidth{0.602250pt}%
\definecolor{currentstroke}{rgb}{0.000000,0.000000,0.000000}%
\pgfsetstrokecolor{currentstroke}%
\pgfsetdash{}{0pt}%
\pgfsys@defobject{currentmarker}{\pgfqpoint{-0.027778in}{0.000000in}}{\pgfqpoint{-0.000000in}{0.000000in}}{%
\pgfpathmoveto{\pgfqpoint{-0.000000in}{0.000000in}}%
\pgfpathlineto{\pgfqpoint{-0.027778in}{0.000000in}}%
\pgfusepath{stroke,fill}%
}%
\begin{pgfscope}%
\pgfsys@transformshift{0.800000in}{1.571007in}%
\pgfsys@useobject{currentmarker}{}%
\end{pgfscope}%
\end{pgfscope}%
\begin{pgfscope}%
\pgfsetbuttcap%
\pgfsetroundjoin%
\definecolor{currentfill}{rgb}{0.000000,0.000000,0.000000}%
\pgfsetfillcolor{currentfill}%
\pgfsetlinewidth{0.602250pt}%
\definecolor{currentstroke}{rgb}{0.000000,0.000000,0.000000}%
\pgfsetstrokecolor{currentstroke}%
\pgfsetdash{}{0pt}%
\pgfsys@defobject{currentmarker}{\pgfqpoint{-0.027778in}{0.000000in}}{\pgfqpoint{-0.000000in}{0.000000in}}{%
\pgfpathmoveto{\pgfqpoint{-0.000000in}{0.000000in}}%
\pgfpathlineto{\pgfqpoint{-0.027778in}{0.000000in}}%
\pgfusepath{stroke,fill}%
}%
\begin{pgfscope}%
\pgfsys@transformshift{0.800000in}{1.627078in}%
\pgfsys@useobject{currentmarker}{}%
\end{pgfscope}%
\end{pgfscope}%
\begin{pgfscope}%
\pgfsetbuttcap%
\pgfsetroundjoin%
\definecolor{currentfill}{rgb}{0.000000,0.000000,0.000000}%
\pgfsetfillcolor{currentfill}%
\pgfsetlinewidth{0.602250pt}%
\definecolor{currentstroke}{rgb}{0.000000,0.000000,0.000000}%
\pgfsetstrokecolor{currentstroke}%
\pgfsetdash{}{0pt}%
\pgfsys@defobject{currentmarker}{\pgfqpoint{-0.027778in}{0.000000in}}{\pgfqpoint{-0.000000in}{0.000000in}}{%
\pgfpathmoveto{\pgfqpoint{-0.000000in}{0.000000in}}%
\pgfpathlineto{\pgfqpoint{-0.027778in}{0.000000in}}%
\pgfusepath{stroke,fill}%
}%
\begin{pgfscope}%
\pgfsys@transformshift{0.800000in}{2.007205in}%
\pgfsys@useobject{currentmarker}{}%
\end{pgfscope}%
\end{pgfscope}%
\begin{pgfscope}%
\pgfsetbuttcap%
\pgfsetroundjoin%
\definecolor{currentfill}{rgb}{0.000000,0.000000,0.000000}%
\pgfsetfillcolor{currentfill}%
\pgfsetlinewidth{0.602250pt}%
\definecolor{currentstroke}{rgb}{0.000000,0.000000,0.000000}%
\pgfsetstrokecolor{currentstroke}%
\pgfsetdash{}{0pt}%
\pgfsys@defobject{currentmarker}{\pgfqpoint{-0.027778in}{0.000000in}}{\pgfqpoint{-0.000000in}{0.000000in}}{%
\pgfpathmoveto{\pgfqpoint{-0.000000in}{0.000000in}}%
\pgfpathlineto{\pgfqpoint{-0.027778in}{0.000000in}}%
\pgfusepath{stroke,fill}%
}%
\begin{pgfscope}%
\pgfsys@transformshift{0.800000in}{2.200226in}%
\pgfsys@useobject{currentmarker}{}%
\end{pgfscope}%
\end{pgfscope}%
\begin{pgfscope}%
\pgfsetbuttcap%
\pgfsetroundjoin%
\definecolor{currentfill}{rgb}{0.000000,0.000000,0.000000}%
\pgfsetfillcolor{currentfill}%
\pgfsetlinewidth{0.602250pt}%
\definecolor{currentstroke}{rgb}{0.000000,0.000000,0.000000}%
\pgfsetstrokecolor{currentstroke}%
\pgfsetdash{}{0pt}%
\pgfsys@defobject{currentmarker}{\pgfqpoint{-0.027778in}{0.000000in}}{\pgfqpoint{-0.000000in}{0.000000in}}{%
\pgfpathmoveto{\pgfqpoint{-0.000000in}{0.000000in}}%
\pgfpathlineto{\pgfqpoint{-0.027778in}{0.000000in}}%
\pgfusepath{stroke,fill}%
}%
\begin{pgfscope}%
\pgfsys@transformshift{0.800000in}{2.337177in}%
\pgfsys@useobject{currentmarker}{}%
\end{pgfscope}%
\end{pgfscope}%
\begin{pgfscope}%
\pgfsetbuttcap%
\pgfsetroundjoin%
\definecolor{currentfill}{rgb}{0.000000,0.000000,0.000000}%
\pgfsetfillcolor{currentfill}%
\pgfsetlinewidth{0.602250pt}%
\definecolor{currentstroke}{rgb}{0.000000,0.000000,0.000000}%
\pgfsetstrokecolor{currentstroke}%
\pgfsetdash{}{0pt}%
\pgfsys@defobject{currentmarker}{\pgfqpoint{-0.027778in}{0.000000in}}{\pgfqpoint{-0.000000in}{0.000000in}}{%
\pgfpathmoveto{\pgfqpoint{-0.000000in}{0.000000in}}%
\pgfpathlineto{\pgfqpoint{-0.027778in}{0.000000in}}%
\pgfusepath{stroke,fill}%
}%
\begin{pgfscope}%
\pgfsys@transformshift{0.800000in}{2.443404in}%
\pgfsys@useobject{currentmarker}{}%
\end{pgfscope}%
\end{pgfscope}%
\begin{pgfscope}%
\pgfsetbuttcap%
\pgfsetroundjoin%
\definecolor{currentfill}{rgb}{0.000000,0.000000,0.000000}%
\pgfsetfillcolor{currentfill}%
\pgfsetlinewidth{0.602250pt}%
\definecolor{currentstroke}{rgb}{0.000000,0.000000,0.000000}%
\pgfsetstrokecolor{currentstroke}%
\pgfsetdash{}{0pt}%
\pgfsys@defobject{currentmarker}{\pgfqpoint{-0.027778in}{0.000000in}}{\pgfqpoint{-0.000000in}{0.000000in}}{%
\pgfpathmoveto{\pgfqpoint{-0.000000in}{0.000000in}}%
\pgfpathlineto{\pgfqpoint{-0.027778in}{0.000000in}}%
\pgfusepath{stroke,fill}%
}%
\begin{pgfscope}%
\pgfsys@transformshift{0.800000in}{2.530198in}%
\pgfsys@useobject{currentmarker}{}%
\end{pgfscope}%
\end{pgfscope}%
\begin{pgfscope}%
\pgfsetbuttcap%
\pgfsetroundjoin%
\definecolor{currentfill}{rgb}{0.000000,0.000000,0.000000}%
\pgfsetfillcolor{currentfill}%
\pgfsetlinewidth{0.602250pt}%
\definecolor{currentstroke}{rgb}{0.000000,0.000000,0.000000}%
\pgfsetstrokecolor{currentstroke}%
\pgfsetdash{}{0pt}%
\pgfsys@defobject{currentmarker}{\pgfqpoint{-0.027778in}{0.000000in}}{\pgfqpoint{-0.000000in}{0.000000in}}{%
\pgfpathmoveto{\pgfqpoint{-0.000000in}{0.000000in}}%
\pgfpathlineto{\pgfqpoint{-0.027778in}{0.000000in}}%
\pgfusepath{stroke,fill}%
}%
\begin{pgfscope}%
\pgfsys@transformshift{0.800000in}{2.603581in}%
\pgfsys@useobject{currentmarker}{}%
\end{pgfscope}%
\end{pgfscope}%
\begin{pgfscope}%
\pgfsetbuttcap%
\pgfsetroundjoin%
\definecolor{currentfill}{rgb}{0.000000,0.000000,0.000000}%
\pgfsetfillcolor{currentfill}%
\pgfsetlinewidth{0.602250pt}%
\definecolor{currentstroke}{rgb}{0.000000,0.000000,0.000000}%
\pgfsetstrokecolor{currentstroke}%
\pgfsetdash{}{0pt}%
\pgfsys@defobject{currentmarker}{\pgfqpoint{-0.027778in}{0.000000in}}{\pgfqpoint{-0.000000in}{0.000000in}}{%
\pgfpathmoveto{\pgfqpoint{-0.000000in}{0.000000in}}%
\pgfpathlineto{\pgfqpoint{-0.027778in}{0.000000in}}%
\pgfusepath{stroke,fill}%
}%
\begin{pgfscope}%
\pgfsys@transformshift{0.800000in}{2.667148in}%
\pgfsys@useobject{currentmarker}{}%
\end{pgfscope}%
\end{pgfscope}%
\begin{pgfscope}%
\pgfsetbuttcap%
\pgfsetroundjoin%
\definecolor{currentfill}{rgb}{0.000000,0.000000,0.000000}%
\pgfsetfillcolor{currentfill}%
\pgfsetlinewidth{0.602250pt}%
\definecolor{currentstroke}{rgb}{0.000000,0.000000,0.000000}%
\pgfsetstrokecolor{currentstroke}%
\pgfsetdash{}{0pt}%
\pgfsys@defobject{currentmarker}{\pgfqpoint{-0.027778in}{0.000000in}}{\pgfqpoint{-0.000000in}{0.000000in}}{%
\pgfpathmoveto{\pgfqpoint{-0.000000in}{0.000000in}}%
\pgfpathlineto{\pgfqpoint{-0.027778in}{0.000000in}}%
\pgfusepath{stroke,fill}%
}%
\begin{pgfscope}%
\pgfsys@transformshift{0.800000in}{2.723218in}%
\pgfsys@useobject{currentmarker}{}%
\end{pgfscope}%
\end{pgfscope}%
\begin{pgfscope}%
\pgfsetbuttcap%
\pgfsetroundjoin%
\definecolor{currentfill}{rgb}{0.000000,0.000000,0.000000}%
\pgfsetfillcolor{currentfill}%
\pgfsetlinewidth{0.602250pt}%
\definecolor{currentstroke}{rgb}{0.000000,0.000000,0.000000}%
\pgfsetstrokecolor{currentstroke}%
\pgfsetdash{}{0pt}%
\pgfsys@defobject{currentmarker}{\pgfqpoint{-0.027778in}{0.000000in}}{\pgfqpoint{-0.000000in}{0.000000in}}{%
\pgfpathmoveto{\pgfqpoint{-0.000000in}{0.000000in}}%
\pgfpathlineto{\pgfqpoint{-0.027778in}{0.000000in}}%
\pgfusepath{stroke,fill}%
}%
\begin{pgfscope}%
\pgfsys@transformshift{0.800000in}{3.103346in}%
\pgfsys@useobject{currentmarker}{}%
\end{pgfscope}%
\end{pgfscope}%
\begin{pgfscope}%
\pgfsetbuttcap%
\pgfsetroundjoin%
\definecolor{currentfill}{rgb}{0.000000,0.000000,0.000000}%
\pgfsetfillcolor{currentfill}%
\pgfsetlinewidth{0.602250pt}%
\definecolor{currentstroke}{rgb}{0.000000,0.000000,0.000000}%
\pgfsetstrokecolor{currentstroke}%
\pgfsetdash{}{0pt}%
\pgfsys@defobject{currentmarker}{\pgfqpoint{-0.027778in}{0.000000in}}{\pgfqpoint{-0.000000in}{0.000000in}}{%
\pgfpathmoveto{\pgfqpoint{-0.000000in}{0.000000in}}%
\pgfpathlineto{\pgfqpoint{-0.027778in}{0.000000in}}%
\pgfusepath{stroke,fill}%
}%
\begin{pgfscope}%
\pgfsys@transformshift{0.800000in}{3.296367in}%
\pgfsys@useobject{currentmarker}{}%
\end{pgfscope}%
\end{pgfscope}%
\begin{pgfscope}%
\pgfsetbuttcap%
\pgfsetroundjoin%
\definecolor{currentfill}{rgb}{0.000000,0.000000,0.000000}%
\pgfsetfillcolor{currentfill}%
\pgfsetlinewidth{0.602250pt}%
\definecolor{currentstroke}{rgb}{0.000000,0.000000,0.000000}%
\pgfsetstrokecolor{currentstroke}%
\pgfsetdash{}{0pt}%
\pgfsys@defobject{currentmarker}{\pgfqpoint{-0.027778in}{0.000000in}}{\pgfqpoint{-0.000000in}{0.000000in}}{%
\pgfpathmoveto{\pgfqpoint{-0.000000in}{0.000000in}}%
\pgfpathlineto{\pgfqpoint{-0.027778in}{0.000000in}}%
\pgfusepath{stroke,fill}%
}%
\begin{pgfscope}%
\pgfsys@transformshift{0.800000in}{3.433318in}%
\pgfsys@useobject{currentmarker}{}%
\end{pgfscope}%
\end{pgfscope}%
\begin{pgfscope}%
\pgfsetbuttcap%
\pgfsetroundjoin%
\definecolor{currentfill}{rgb}{0.000000,0.000000,0.000000}%
\pgfsetfillcolor{currentfill}%
\pgfsetlinewidth{0.602250pt}%
\definecolor{currentstroke}{rgb}{0.000000,0.000000,0.000000}%
\pgfsetstrokecolor{currentstroke}%
\pgfsetdash{}{0pt}%
\pgfsys@defobject{currentmarker}{\pgfqpoint{-0.027778in}{0.000000in}}{\pgfqpoint{-0.000000in}{0.000000in}}{%
\pgfpathmoveto{\pgfqpoint{-0.000000in}{0.000000in}}%
\pgfpathlineto{\pgfqpoint{-0.027778in}{0.000000in}}%
\pgfusepath{stroke,fill}%
}%
\begin{pgfscope}%
\pgfsys@transformshift{0.800000in}{3.539545in}%
\pgfsys@useobject{currentmarker}{}%
\end{pgfscope}%
\end{pgfscope}%
\begin{pgfscope}%
\pgfsetbuttcap%
\pgfsetroundjoin%
\definecolor{currentfill}{rgb}{0.000000,0.000000,0.000000}%
\pgfsetfillcolor{currentfill}%
\pgfsetlinewidth{0.602250pt}%
\definecolor{currentstroke}{rgb}{0.000000,0.000000,0.000000}%
\pgfsetstrokecolor{currentstroke}%
\pgfsetdash{}{0pt}%
\pgfsys@defobject{currentmarker}{\pgfqpoint{-0.027778in}{0.000000in}}{\pgfqpoint{-0.000000in}{0.000000in}}{%
\pgfpathmoveto{\pgfqpoint{-0.000000in}{0.000000in}}%
\pgfpathlineto{\pgfqpoint{-0.027778in}{0.000000in}}%
\pgfusepath{stroke,fill}%
}%
\begin{pgfscope}%
\pgfsys@transformshift{0.800000in}{3.626338in}%
\pgfsys@useobject{currentmarker}{}%
\end{pgfscope}%
\end{pgfscope}%
\begin{pgfscope}%
\pgfsetbuttcap%
\pgfsetroundjoin%
\definecolor{currentfill}{rgb}{0.000000,0.000000,0.000000}%
\pgfsetfillcolor{currentfill}%
\pgfsetlinewidth{0.602250pt}%
\definecolor{currentstroke}{rgb}{0.000000,0.000000,0.000000}%
\pgfsetstrokecolor{currentstroke}%
\pgfsetdash{}{0pt}%
\pgfsys@defobject{currentmarker}{\pgfqpoint{-0.027778in}{0.000000in}}{\pgfqpoint{-0.000000in}{0.000000in}}{%
\pgfpathmoveto{\pgfqpoint{-0.000000in}{0.000000in}}%
\pgfpathlineto{\pgfqpoint{-0.027778in}{0.000000in}}%
\pgfusepath{stroke,fill}%
}%
\begin{pgfscope}%
\pgfsys@transformshift{0.800000in}{3.699722in}%
\pgfsys@useobject{currentmarker}{}%
\end{pgfscope}%
\end{pgfscope}%
\begin{pgfscope}%
\pgfsetbuttcap%
\pgfsetroundjoin%
\definecolor{currentfill}{rgb}{0.000000,0.000000,0.000000}%
\pgfsetfillcolor{currentfill}%
\pgfsetlinewidth{0.602250pt}%
\definecolor{currentstroke}{rgb}{0.000000,0.000000,0.000000}%
\pgfsetstrokecolor{currentstroke}%
\pgfsetdash{}{0pt}%
\pgfsys@defobject{currentmarker}{\pgfqpoint{-0.027778in}{0.000000in}}{\pgfqpoint{-0.000000in}{0.000000in}}{%
\pgfpathmoveto{\pgfqpoint{-0.000000in}{0.000000in}}%
\pgfpathlineto{\pgfqpoint{-0.027778in}{0.000000in}}%
\pgfusepath{stroke,fill}%
}%
\begin{pgfscope}%
\pgfsys@transformshift{0.800000in}{3.763289in}%
\pgfsys@useobject{currentmarker}{}%
\end{pgfscope}%
\end{pgfscope}%
\begin{pgfscope}%
\pgfsetbuttcap%
\pgfsetroundjoin%
\definecolor{currentfill}{rgb}{0.000000,0.000000,0.000000}%
\pgfsetfillcolor{currentfill}%
\pgfsetlinewidth{0.602250pt}%
\definecolor{currentstroke}{rgb}{0.000000,0.000000,0.000000}%
\pgfsetstrokecolor{currentstroke}%
\pgfsetdash{}{0pt}%
\pgfsys@defobject{currentmarker}{\pgfqpoint{-0.027778in}{0.000000in}}{\pgfqpoint{-0.000000in}{0.000000in}}{%
\pgfpathmoveto{\pgfqpoint{-0.000000in}{0.000000in}}%
\pgfpathlineto{\pgfqpoint{-0.027778in}{0.000000in}}%
\pgfusepath{stroke,fill}%
}%
\begin{pgfscope}%
\pgfsys@transformshift{0.800000in}{3.819359in}%
\pgfsys@useobject{currentmarker}{}%
\end{pgfscope}%
\end{pgfscope}%
\begin{pgfscope}%
\pgfsetbuttcap%
\pgfsetroundjoin%
\definecolor{currentfill}{rgb}{0.000000,0.000000,0.000000}%
\pgfsetfillcolor{currentfill}%
\pgfsetlinewidth{0.602250pt}%
\definecolor{currentstroke}{rgb}{0.000000,0.000000,0.000000}%
\pgfsetstrokecolor{currentstroke}%
\pgfsetdash{}{0pt}%
\pgfsys@defobject{currentmarker}{\pgfqpoint{-0.027778in}{0.000000in}}{\pgfqpoint{-0.000000in}{0.000000in}}{%
\pgfpathmoveto{\pgfqpoint{-0.000000in}{0.000000in}}%
\pgfpathlineto{\pgfqpoint{-0.027778in}{0.000000in}}%
\pgfusepath{stroke,fill}%
}%
\begin{pgfscope}%
\pgfsys@transformshift{0.800000in}{4.199487in}%
\pgfsys@useobject{currentmarker}{}%
\end{pgfscope}%
\end{pgfscope}%
\begin{pgfscope}%
\definecolor{textcolor}{rgb}{0.000000,0.000000,0.000000}%
\pgfsetstrokecolor{textcolor}%
\pgfsetfillcolor{textcolor}%
\pgftext[x=0.446026in,y=2.376000in,,bottom,rotate=90.000000]{\color{textcolor}\rmfamily\fontsize{10.000000}{12.000000}\selectfont Basic Operations}%
\end{pgfscope}%
\begin{pgfscope}%
\pgfpathrectangle{\pgfqpoint{0.800000in}{0.528000in}}{\pgfqpoint{4.960000in}{3.696000in}}%
\pgfusepath{clip}%
\pgfsetrectcap%
\pgfsetroundjoin%
\pgfsetlinewidth{1.505625pt}%
\definecolor{currentstroke}{rgb}{0.121569,0.466667,0.705882}%
\pgfsetstrokecolor{currentstroke}%
\pgfsetdash{}{0pt}%
\pgfpathmoveto{\pgfqpoint{1.025455in}{1.899160in}}%
\pgfpathlineto{\pgfqpoint{1.076118in}{1.987375in}}%
\pgfpathlineto{\pgfqpoint{1.126782in}{2.068096in}}%
\pgfpathlineto{\pgfqpoint{1.177446in}{2.142499in}}%
\pgfpathlineto{\pgfqpoint{1.228110in}{2.211501in}}%
\pgfpathlineto{\pgfqpoint{1.278774in}{2.275835in}}%
\pgfpathlineto{\pgfqpoint{1.329438in}{2.336092in}}%
\pgfpathlineto{\pgfqpoint{1.380102in}{2.392760in}}%
\pgfpathlineto{\pgfqpoint{1.430766in}{2.446242in}}%
\pgfpathlineto{\pgfqpoint{1.481430in}{2.496877in}}%
\pgfpathlineto{\pgfqpoint{1.532094in}{2.544954in}}%
\pgfpathlineto{\pgfqpoint{1.582758in}{2.590718in}}%
\pgfpathlineto{\pgfqpoint{1.633422in}{2.634383in}}%
\pgfpathlineto{\pgfqpoint{1.684086in}{2.676131in}}%
\pgfpathlineto{\pgfqpoint{1.734750in}{2.716126in}}%
\pgfpathlineto{\pgfqpoint{1.785414in}{2.754507in}}%
\pgfpathlineto{\pgfqpoint{1.836078in}{2.791400in}}%
\pgfpathlineto{\pgfqpoint{1.886742in}{2.826917in}}%
\pgfpathlineto{\pgfqpoint{1.937406in}{2.861156in}}%
\pgfpathlineto{\pgfqpoint{1.988069in}{2.894206in}}%
\pgfpathlineto{\pgfqpoint{2.038733in}{2.926146in}}%
\pgfpathlineto{\pgfqpoint{2.089397in}{2.957050in}}%
\pgfpathlineto{\pgfqpoint{2.140061in}{2.986982in}}%
\pgfpathlineto{\pgfqpoint{2.190725in}{3.016002in}}%
\pgfpathlineto{\pgfqpoint{2.241389in}{3.044163in}}%
\pgfpathlineto{\pgfqpoint{2.292053in}{3.071515in}}%
\pgfpathlineto{\pgfqpoint{2.342717in}{3.098102in}}%
\pgfpathlineto{\pgfqpoint{2.393381in}{3.123968in}}%
\pgfpathlineto{\pgfqpoint{2.444045in}{3.149149in}}%
\pgfpathlineto{\pgfqpoint{2.494709in}{3.173681in}}%
\pgfpathlineto{\pgfqpoint{2.545373in}{3.197597in}}%
\pgfpathlineto{\pgfqpoint{2.596037in}{3.220927in}}%
\pgfpathlineto{\pgfqpoint{2.646701in}{3.243698in}}%
\pgfpathlineto{\pgfqpoint{2.697365in}{3.265938in}}%
\pgfpathlineto{\pgfqpoint{2.748029in}{3.287670in}}%
\pgfpathlineto{\pgfqpoint{2.798693in}{3.308917in}}%
\pgfpathlineto{\pgfqpoint{2.849356in}{3.329700in}}%
\pgfpathlineto{\pgfqpoint{2.900020in}{3.350039in}}%
\pgfpathlineto{\pgfqpoint{2.950684in}{3.369953in}}%
\pgfpathlineto{\pgfqpoint{3.001348in}{3.389458in}}%
\pgfpathlineto{\pgfqpoint{3.052012in}{3.408572in}}%
\pgfpathlineto{\pgfqpoint{3.102676in}{3.427310in}}%
\pgfpathlineto{\pgfqpoint{3.153340in}{3.445686in}}%
\pgfpathlineto{\pgfqpoint{3.204004in}{3.463714in}}%
\pgfpathlineto{\pgfqpoint{3.254668in}{3.481407in}}%
\pgfpathlineto{\pgfqpoint{3.305332in}{3.498778in}}%
\pgfpathlineto{\pgfqpoint{3.355996in}{3.515837in}}%
\pgfpathlineto{\pgfqpoint{3.406660in}{3.532595in}}%
\pgfpathlineto{\pgfqpoint{3.457324in}{3.549064in}}%
\pgfpathlineto{\pgfqpoint{3.507988in}{3.565253in}}%
\pgfpathlineto{\pgfqpoint{3.558652in}{3.581171in}}%
\pgfpathlineto{\pgfqpoint{3.609316in}{3.596827in}}%
\pgfpathlineto{\pgfqpoint{3.659980in}{3.612230in}}%
\pgfpathlineto{\pgfqpoint{3.710644in}{3.627388in}}%
\pgfpathlineto{\pgfqpoint{3.761307in}{3.642308in}}%
\pgfpathlineto{\pgfqpoint{3.811971in}{3.656998in}}%
\pgfpathlineto{\pgfqpoint{3.862635in}{3.671465in}}%
\pgfpathlineto{\pgfqpoint{3.913299in}{3.685715in}}%
\pgfpathlineto{\pgfqpoint{3.963963in}{3.699755in}}%
\pgfpathlineto{\pgfqpoint{4.014627in}{3.713591in}}%
\pgfpathlineto{\pgfqpoint{4.065291in}{3.727229in}}%
\pgfpathlineto{\pgfqpoint{4.115955in}{3.740674in}}%
\pgfpathlineto{\pgfqpoint{4.166619in}{3.753932in}}%
\pgfpathlineto{\pgfqpoint{4.217283in}{3.767008in}}%
\pgfpathlineto{\pgfqpoint{4.267947in}{3.779906in}}%
\pgfpathlineto{\pgfqpoint{4.318611in}{3.792633in}}%
\pgfpathlineto{\pgfqpoint{4.369275in}{3.805191in}}%
\pgfpathlineto{\pgfqpoint{4.419939in}{3.817586in}}%
\pgfpathlineto{\pgfqpoint{4.470603in}{3.829822in}}%
\pgfpathlineto{\pgfqpoint{4.521267in}{3.841902in}}%
\pgfpathlineto{\pgfqpoint{4.571931in}{3.853831in}}%
\pgfpathlineto{\pgfqpoint{4.622594in}{3.865613in}}%
\pgfpathlineto{\pgfqpoint{4.673258in}{3.877250in}}%
\pgfpathlineto{\pgfqpoint{4.723922in}{3.888747in}}%
\pgfpathlineto{\pgfqpoint{4.774586in}{3.900106in}}%
\pgfpathlineto{\pgfqpoint{4.825250in}{3.911332in}}%
\pgfpathlineto{\pgfqpoint{4.875914in}{3.922427in}}%
\pgfpathlineto{\pgfqpoint{4.926578in}{3.933394in}}%
\pgfpathlineto{\pgfqpoint{4.977242in}{3.944236in}}%
\pgfpathlineto{\pgfqpoint{5.027906in}{3.954957in}}%
\pgfpathlineto{\pgfqpoint{5.078570in}{3.965557in}}%
\pgfpathlineto{\pgfqpoint{5.129234in}{3.976041in}}%
\pgfpathlineto{\pgfqpoint{5.179898in}{3.986411in}}%
\pgfpathlineto{\pgfqpoint{5.230562in}{3.996669in}}%
\pgfpathlineto{\pgfqpoint{5.281226in}{4.006818in}}%
\pgfpathlineto{\pgfqpoint{5.331890in}{4.016860in}}%
\pgfpathlineto{\pgfqpoint{5.382554in}{4.026797in}}%
\pgfpathlineto{\pgfqpoint{5.433218in}{4.036631in}}%
\pgfpathlineto{\pgfqpoint{5.483882in}{4.046365in}}%
\pgfpathlineto{\pgfqpoint{5.534545in}{4.056000in}}%
\pgfusepath{stroke}%
\end{pgfscope}%
\begin{pgfscope}%
\pgfpathrectangle{\pgfqpoint{0.800000in}{0.528000in}}{\pgfqpoint{4.960000in}{3.696000in}}%
\pgfusepath{clip}%
\pgfsetrectcap%
\pgfsetroundjoin%
\pgfsetlinewidth{1.505625pt}%
\definecolor{currentstroke}{rgb}{1.000000,0.498039,0.054902}%
\pgfsetstrokecolor{currentstroke}%
\pgfsetdash{}{0pt}%
\pgfpathmoveto{\pgfqpoint{1.025455in}{0.696000in}}%
\pgfpathlineto{\pgfqpoint{1.076118in}{0.795524in}}%
\pgfpathlineto{\pgfqpoint{1.126782in}{0.834817in}}%
\pgfpathlineto{\pgfqpoint{1.177446in}{0.871112in}}%
\pgfpathlineto{\pgfqpoint{1.228110in}{0.904835in}}%
\pgfpathlineto{\pgfqpoint{1.278774in}{0.953180in}}%
\pgfpathlineto{\pgfqpoint{1.329438in}{1.011930in}}%
\pgfpathlineto{\pgfqpoint{1.380102in}{1.053046in}}%
\pgfpathlineto{\pgfqpoint{1.430766in}{1.078831in}}%
\pgfpathlineto{\pgfqpoint{1.481430in}{1.102814in}}%
\pgfpathlineto{\pgfqpoint{1.532094in}{1.126556in}}%
\pgfpathlineto{\pgfqpoint{1.582758in}{1.173096in}}%
\pgfpathlineto{\pgfqpoint{1.633422in}{1.211073in}}%
\pgfpathlineto{\pgfqpoint{1.684086in}{1.236492in}}%
\pgfpathlineto{\pgfqpoint{1.734750in}{1.255915in}}%
\pgfpathlineto{\pgfqpoint{1.785414in}{1.274577in}}%
\pgfpathlineto{\pgfqpoint{1.836078in}{1.292535in}}%
\pgfpathlineto{\pgfqpoint{1.886742in}{1.309841in}}%
\pgfpathlineto{\pgfqpoint{1.937406in}{1.326539in}}%
\pgfpathlineto{\pgfqpoint{1.988069in}{1.346310in}}%
\pgfpathlineto{\pgfqpoint{2.038733in}{1.372209in}}%
\pgfpathlineto{\pgfqpoint{2.089397in}{1.400021in}}%
\pgfpathlineto{\pgfqpoint{2.140061in}{1.423223in}}%
\pgfpathlineto{\pgfqpoint{2.190725in}{1.446276in}}%
\pgfpathlineto{\pgfqpoint{2.241389in}{1.465226in}}%
\pgfpathlineto{\pgfqpoint{2.292053in}{1.477984in}}%
\pgfpathlineto{\pgfqpoint{2.342717in}{1.491254in}}%
\pgfpathlineto{\pgfqpoint{2.393381in}{1.503753in}}%
\pgfpathlineto{\pgfqpoint{2.444045in}{1.515732in}}%
\pgfpathlineto{\pgfqpoint{2.494709in}{1.527612in}}%
\pgfpathlineto{\pgfqpoint{2.545373in}{1.539585in}}%
\pgfpathlineto{\pgfqpoint{2.596037in}{1.561329in}}%
\pgfpathlineto{\pgfqpoint{2.646701in}{1.579909in}}%
\pgfpathlineto{\pgfqpoint{2.697365in}{1.598128in}}%
\pgfpathlineto{\pgfqpoint{2.748029in}{1.617892in}}%
\pgfpathlineto{\pgfqpoint{2.798693in}{1.632599in}}%
\pgfpathlineto{\pgfqpoint{2.849356in}{1.642687in}}%
\pgfpathlineto{\pgfqpoint{2.900020in}{1.652565in}}%
\pgfpathlineto{\pgfqpoint{2.950684in}{1.662243in}}%
\pgfpathlineto{\pgfqpoint{3.001348in}{1.671728in}}%
\pgfpathlineto{\pgfqpoint{3.052012in}{1.681027in}}%
\pgfpathlineto{\pgfqpoint{3.102676in}{1.690149in}}%
\pgfpathlineto{\pgfqpoint{3.153340in}{1.699099in}}%
\pgfpathlineto{\pgfqpoint{3.204004in}{1.707883in}}%
\pgfpathlineto{\pgfqpoint{3.254668in}{1.716509in}}%
\pgfpathlineto{\pgfqpoint{3.305332in}{1.724981in}}%
\pgfpathlineto{\pgfqpoint{3.355996in}{1.733305in}}%
\pgfpathlineto{\pgfqpoint{3.406660in}{1.741485in}}%
\pgfpathlineto{\pgfqpoint{3.457324in}{1.752626in}}%
\pgfpathlineto{\pgfqpoint{3.507988in}{1.765019in}}%
\pgfpathlineto{\pgfqpoint{3.558652in}{1.777097in}}%
\pgfpathlineto{\pgfqpoint{3.609316in}{1.790305in}}%
\pgfpathlineto{\pgfqpoint{3.659980in}{1.801766in}}%
\pgfpathlineto{\pgfqpoint{3.710644in}{1.812957in}}%
\pgfpathlineto{\pgfqpoint{3.761307in}{1.825219in}}%
\pgfpathlineto{\pgfqpoint{3.811971in}{1.836288in}}%
\pgfpathlineto{\pgfqpoint{3.862635in}{1.847005in}}%
\pgfpathlineto{\pgfqpoint{3.913299in}{1.858723in}}%
\pgfpathlineto{\pgfqpoint{3.963963in}{1.865406in}}%
\pgfpathlineto{\pgfqpoint{4.014627in}{1.871902in}}%
\pgfpathlineto{\pgfqpoint{4.065291in}{1.878311in}}%
\pgfpathlineto{\pgfqpoint{4.115955in}{1.884911in}}%
\pgfpathlineto{\pgfqpoint{4.166619in}{1.891513in}}%
\pgfpathlineto{\pgfqpoint{4.217283in}{1.897934in}}%
\pgfpathlineto{\pgfqpoint{4.267947in}{1.904181in}}%
\pgfpathlineto{\pgfqpoint{4.318611in}{1.910260in}}%
\pgfpathlineto{\pgfqpoint{4.369275in}{1.916175in}}%
\pgfpathlineto{\pgfqpoint{4.419939in}{1.922103in}}%
\pgfpathlineto{\pgfqpoint{4.470603in}{1.928212in}}%
\pgfpathlineto{\pgfqpoint{4.521267in}{1.934243in}}%
\pgfpathlineto{\pgfqpoint{4.571931in}{1.940116in}}%
\pgfpathlineto{\pgfqpoint{4.622594in}{1.950417in}}%
\pgfpathlineto{\pgfqpoint{4.673258in}{1.959423in}}%
\pgfpathlineto{\pgfqpoint{4.723922in}{1.968184in}}%
\pgfpathlineto{\pgfqpoint{4.774586in}{1.978129in}}%
\pgfpathlineto{\pgfqpoint{4.825250in}{1.986854in}}%
\pgfpathlineto{\pgfqpoint{4.875914in}{1.995421in}}%
\pgfpathlineto{\pgfqpoint{4.926578in}{2.004819in}}%
\pgfpathlineto{\pgfqpoint{4.977242in}{2.013072in}}%
\pgfpathlineto{\pgfqpoint{5.027906in}{2.021184in}}%
\pgfpathlineto{\pgfqpoint{5.078570in}{2.028228in}}%
\pgfpathlineto{\pgfqpoint{5.129234in}{2.033325in}}%
\pgfpathlineto{\pgfqpoint{5.179898in}{2.038368in}}%
\pgfpathlineto{\pgfqpoint{5.230562in}{2.043357in}}%
\pgfpathlineto{\pgfqpoint{5.281226in}{2.048296in}}%
\pgfpathlineto{\pgfqpoint{5.331890in}{2.053183in}}%
\pgfpathlineto{\pgfqpoint{5.382554in}{2.058021in}}%
\pgfpathlineto{\pgfqpoint{5.433218in}{2.062810in}}%
\pgfpathlineto{\pgfqpoint{5.483882in}{2.067552in}}%
\pgfpathlineto{\pgfqpoint{5.534545in}{2.072246in}}%
\pgfusepath{stroke}%
\end{pgfscope}%
\begin{pgfscope}%
\pgfsetrectcap%
\pgfsetmiterjoin%
\pgfsetlinewidth{0.803000pt}%
\definecolor{currentstroke}{rgb}{0.000000,0.000000,0.000000}%
\pgfsetstrokecolor{currentstroke}%
\pgfsetdash{}{0pt}%
\pgfpathmoveto{\pgfqpoint{0.800000in}{0.528000in}}%
\pgfpathlineto{\pgfqpoint{0.800000in}{4.224000in}}%
\pgfusepath{stroke}%
\end{pgfscope}%
\begin{pgfscope}%
\pgfsetrectcap%
\pgfsetmiterjoin%
\pgfsetlinewidth{0.803000pt}%
\definecolor{currentstroke}{rgb}{0.000000,0.000000,0.000000}%
\pgfsetstrokecolor{currentstroke}%
\pgfsetdash{}{0pt}%
\pgfpathmoveto{\pgfqpoint{5.760000in}{0.528000in}}%
\pgfpathlineto{\pgfqpoint{5.760000in}{4.224000in}}%
\pgfusepath{stroke}%
\end{pgfscope}%
\begin{pgfscope}%
\pgfsetrectcap%
\pgfsetmiterjoin%
\pgfsetlinewidth{0.803000pt}%
\definecolor{currentstroke}{rgb}{0.000000,0.000000,0.000000}%
\pgfsetstrokecolor{currentstroke}%
\pgfsetdash{}{0pt}%
\pgfpathmoveto{\pgfqpoint{0.800000in}{0.528000in}}%
\pgfpathlineto{\pgfqpoint{5.760000in}{0.528000in}}%
\pgfusepath{stroke}%
\end{pgfscope}%
\begin{pgfscope}%
\pgfsetrectcap%
\pgfsetmiterjoin%
\pgfsetlinewidth{0.803000pt}%
\definecolor{currentstroke}{rgb}{0.000000,0.000000,0.000000}%
\pgfsetstrokecolor{currentstroke}%
\pgfsetdash{}{0pt}%
\pgfpathmoveto{\pgfqpoint{0.800000in}{4.224000in}}%
\pgfpathlineto{\pgfqpoint{5.760000in}{4.224000in}}%
\pgfusepath{stroke}%
\end{pgfscope}%
\begin{pgfscope}%
\pgfsetbuttcap%
\pgfsetmiterjoin%
\definecolor{currentfill}{rgb}{1.000000,1.000000,1.000000}%
\pgfsetfillcolor{currentfill}%
\pgfsetfillopacity{0.800000}%
\pgfsetlinewidth{1.003750pt}%
\definecolor{currentstroke}{rgb}{0.800000,0.800000,0.800000}%
\pgfsetstrokecolor{currentstroke}%
\pgfsetstrokeopacity{0.800000}%
\pgfsetdash{}{0pt}%
\pgfpathmoveto{\pgfqpoint{0.897222in}{3.725543in}}%
\pgfpathlineto{\pgfqpoint{1.975542in}{3.725543in}}%
\pgfpathquadraticcurveto{\pgfqpoint{2.003319in}{3.725543in}}{\pgfqpoint{2.003319in}{3.753321in}}%
\pgfpathlineto{\pgfqpoint{2.003319in}{4.126778in}}%
\pgfpathquadraticcurveto{\pgfqpoint{2.003319in}{4.154556in}}{\pgfqpoint{1.975542in}{4.154556in}}%
\pgfpathlineto{\pgfqpoint{0.897222in}{4.154556in}}%
\pgfpathquadraticcurveto{\pgfqpoint{0.869444in}{4.154556in}}{\pgfqpoint{0.869444in}{4.126778in}}%
\pgfpathlineto{\pgfqpoint{0.869444in}{3.753321in}}%
\pgfpathquadraticcurveto{\pgfqpoint{0.869444in}{3.725543in}}{\pgfqpoint{0.897222in}{3.725543in}}%
\pgfpathlineto{\pgfqpoint{0.897222in}{3.725543in}}%
\pgfpathclose%
\pgfusepath{stroke,fill}%
\end{pgfscope}%
\begin{pgfscope}%
\pgfsetrectcap%
\pgfsetroundjoin%
\pgfsetlinewidth{1.505625pt}%
\definecolor{currentstroke}{rgb}{0.121569,0.466667,0.705882}%
\pgfsetstrokecolor{currentstroke}%
\pgfsetdash{}{0pt}%
\pgfpathmoveto{\pgfqpoint{0.925000in}{4.050389in}}%
\pgfpathlineto{\pgfqpoint{1.063889in}{4.050389in}}%
\pgfpathlineto{\pgfqpoint{1.202778in}{4.050389in}}%
\pgfusepath{stroke}%
\end{pgfscope}%
\begin{pgfscope}%
\definecolor{textcolor}{rgb}{0.000000,0.000000,0.000000}%
\pgfsetstrokecolor{textcolor}%
\pgfsetfillcolor{textcolor}%
\pgftext[x=1.313889in,y=4.001778in,left,base]{\color{textcolor}\rmfamily\fontsize{10.000000}{12.000000}\selectfont quicksort}%
\end{pgfscope}%
\begin{pgfscope}%
\pgfsetrectcap%
\pgfsetroundjoin%
\pgfsetlinewidth{1.505625pt}%
\definecolor{currentstroke}{rgb}{1.000000,0.498039,0.054902}%
\pgfsetstrokecolor{currentstroke}%
\pgfsetdash{}{0pt}%
\pgfpathmoveto{\pgfqpoint{0.925000in}{3.856716in}}%
\pgfpathlineto{\pgfqpoint{1.063889in}{3.856716in}}%
\pgfpathlineto{\pgfqpoint{1.202778in}{3.856716in}}%
\pgfusepath{stroke}%
\end{pgfscope}%
\begin{pgfscope}%
\definecolor{textcolor}{rgb}{0.000000,0.000000,0.000000}%
\pgfsetstrokecolor{textcolor}%
\pgfsetfillcolor{textcolor}%
\pgftext[x=1.313889in,y=3.808105in,left,base]{\color{textcolor}\rmfamily\fontsize{10.000000}{12.000000}\selectfont bquicksort}%
\end{pgfscope}%
\end{pgfpicture}%
\makeatother%
\endgroup%

\subsubsection{Time}
%% Creator: Matplotlib, PGF backend
%%
%% To include the figure in your LaTeX document, write
%%   \input{<filename>.pgf}
%%
%% Make sure the required packages are loaded in your preamble
%%   \usepackage{pgf}
%%
%% Also ensure that all the required font packages are loaded; for instance,
%% the lmodern package is sometimes necessary when using math font.
%%   \usepackage{lmodern}
%%
%% Figures using additional raster images can only be included by \input if
%% they are in the same directory as the main LaTeX file. For loading figures
%% from other directories you can use the `import` package
%%   \usepackage{import}
%%
%% and then include the figures with
%%   \import{<path to file>}{<filename>.pgf}
%%
%% Matplotlib used the following preamble
%%   
%%   \makeatletter\@ifpackageloaded{underscore}{}{\usepackage[strings]{underscore}}\makeatother
%%
\begingroup%
\makeatletter%
\begin{pgfpicture}%
\pgfpathrectangle{\pgfpointorigin}{\pgfqpoint{6.400000in}{4.800000in}}%
\pgfusepath{use as bounding box, clip}%
\begin{pgfscope}%
\pgfsetbuttcap%
\pgfsetmiterjoin%
\definecolor{currentfill}{rgb}{1.000000,1.000000,1.000000}%
\pgfsetfillcolor{currentfill}%
\pgfsetlinewidth{0.000000pt}%
\definecolor{currentstroke}{rgb}{1.000000,1.000000,1.000000}%
\pgfsetstrokecolor{currentstroke}%
\pgfsetdash{}{0pt}%
\pgfpathmoveto{\pgfqpoint{0.000000in}{0.000000in}}%
\pgfpathlineto{\pgfqpoint{6.400000in}{0.000000in}}%
\pgfpathlineto{\pgfqpoint{6.400000in}{4.800000in}}%
\pgfpathlineto{\pgfqpoint{0.000000in}{4.800000in}}%
\pgfpathlineto{\pgfqpoint{0.000000in}{0.000000in}}%
\pgfpathclose%
\pgfusepath{fill}%
\end{pgfscope}%
\begin{pgfscope}%
\pgfsetbuttcap%
\pgfsetmiterjoin%
\definecolor{currentfill}{rgb}{1.000000,1.000000,1.000000}%
\pgfsetfillcolor{currentfill}%
\pgfsetlinewidth{0.000000pt}%
\definecolor{currentstroke}{rgb}{0.000000,0.000000,0.000000}%
\pgfsetstrokecolor{currentstroke}%
\pgfsetstrokeopacity{0.000000}%
\pgfsetdash{}{0pt}%
\pgfpathmoveto{\pgfqpoint{0.800000in}{0.528000in}}%
\pgfpathlineto{\pgfqpoint{5.760000in}{0.528000in}}%
\pgfpathlineto{\pgfqpoint{5.760000in}{4.224000in}}%
\pgfpathlineto{\pgfqpoint{0.800000in}{4.224000in}}%
\pgfpathlineto{\pgfqpoint{0.800000in}{0.528000in}}%
\pgfpathclose%
\pgfusepath{fill}%
\end{pgfscope}%
\begin{pgfscope}%
\pgfsetbuttcap%
\pgfsetroundjoin%
\definecolor{currentfill}{rgb}{0.000000,0.000000,0.000000}%
\pgfsetfillcolor{currentfill}%
\pgfsetlinewidth{0.803000pt}%
\definecolor{currentstroke}{rgb}{0.000000,0.000000,0.000000}%
\pgfsetstrokecolor{currentstroke}%
\pgfsetdash{}{0pt}%
\pgfsys@defobject{currentmarker}{\pgfqpoint{0.000000in}{-0.048611in}}{\pgfqpoint{0.000000in}{0.000000in}}{%
\pgfpathmoveto{\pgfqpoint{0.000000in}{0.000000in}}%
\pgfpathlineto{\pgfqpoint{0.000000in}{-0.048611in}}%
\pgfusepath{stroke,fill}%
}%
\begin{pgfscope}%
\pgfsys@transformshift{0.979443in}{0.528000in}%
\pgfsys@useobject{currentmarker}{}%
\end{pgfscope}%
\end{pgfscope}%
\begin{pgfscope}%
\definecolor{textcolor}{rgb}{0.000000,0.000000,0.000000}%
\pgfsetstrokecolor{textcolor}%
\pgfsetfillcolor{textcolor}%
\pgftext[x=0.979443in,y=0.430778in,,top]{\color{textcolor}\rmfamily\fontsize{10.000000}{12.000000}\selectfont \(\displaystyle {0}\)}%
\end{pgfscope}%
\begin{pgfscope}%
\pgfsetbuttcap%
\pgfsetroundjoin%
\definecolor{currentfill}{rgb}{0.000000,0.000000,0.000000}%
\pgfsetfillcolor{currentfill}%
\pgfsetlinewidth{0.803000pt}%
\definecolor{currentstroke}{rgb}{0.000000,0.000000,0.000000}%
\pgfsetstrokecolor{currentstroke}%
\pgfsetdash{}{0pt}%
\pgfsys@defobject{currentmarker}{\pgfqpoint{0.000000in}{-0.048611in}}{\pgfqpoint{0.000000in}{0.000000in}}{%
\pgfpathmoveto{\pgfqpoint{0.000000in}{0.000000in}}%
\pgfpathlineto{\pgfqpoint{0.000000in}{-0.048611in}}%
\pgfusepath{stroke,fill}%
}%
\begin{pgfscope}%
\pgfsys@transformshift{1.899666in}{0.528000in}%
\pgfsys@useobject{currentmarker}{}%
\end{pgfscope}%
\end{pgfscope}%
\begin{pgfscope}%
\definecolor{textcolor}{rgb}{0.000000,0.000000,0.000000}%
\pgfsetstrokecolor{textcolor}%
\pgfsetfillcolor{textcolor}%
\pgftext[x=1.899666in,y=0.430778in,,top]{\color{textcolor}\rmfamily\fontsize{10.000000}{12.000000}\selectfont \(\displaystyle {200}\)}%
\end{pgfscope}%
\begin{pgfscope}%
\pgfsetbuttcap%
\pgfsetroundjoin%
\definecolor{currentfill}{rgb}{0.000000,0.000000,0.000000}%
\pgfsetfillcolor{currentfill}%
\pgfsetlinewidth{0.803000pt}%
\definecolor{currentstroke}{rgb}{0.000000,0.000000,0.000000}%
\pgfsetstrokecolor{currentstroke}%
\pgfsetdash{}{0pt}%
\pgfsys@defobject{currentmarker}{\pgfqpoint{0.000000in}{-0.048611in}}{\pgfqpoint{0.000000in}{0.000000in}}{%
\pgfpathmoveto{\pgfqpoint{0.000000in}{0.000000in}}%
\pgfpathlineto{\pgfqpoint{0.000000in}{-0.048611in}}%
\pgfusepath{stroke,fill}%
}%
\begin{pgfscope}%
\pgfsys@transformshift{2.819889in}{0.528000in}%
\pgfsys@useobject{currentmarker}{}%
\end{pgfscope}%
\end{pgfscope}%
\begin{pgfscope}%
\definecolor{textcolor}{rgb}{0.000000,0.000000,0.000000}%
\pgfsetstrokecolor{textcolor}%
\pgfsetfillcolor{textcolor}%
\pgftext[x=2.819889in,y=0.430778in,,top]{\color{textcolor}\rmfamily\fontsize{10.000000}{12.000000}\selectfont \(\displaystyle {400}\)}%
\end{pgfscope}%
\begin{pgfscope}%
\pgfsetbuttcap%
\pgfsetroundjoin%
\definecolor{currentfill}{rgb}{0.000000,0.000000,0.000000}%
\pgfsetfillcolor{currentfill}%
\pgfsetlinewidth{0.803000pt}%
\definecolor{currentstroke}{rgb}{0.000000,0.000000,0.000000}%
\pgfsetstrokecolor{currentstroke}%
\pgfsetdash{}{0pt}%
\pgfsys@defobject{currentmarker}{\pgfqpoint{0.000000in}{-0.048611in}}{\pgfqpoint{0.000000in}{0.000000in}}{%
\pgfpathmoveto{\pgfqpoint{0.000000in}{0.000000in}}%
\pgfpathlineto{\pgfqpoint{0.000000in}{-0.048611in}}%
\pgfusepath{stroke,fill}%
}%
\begin{pgfscope}%
\pgfsys@transformshift{3.740111in}{0.528000in}%
\pgfsys@useobject{currentmarker}{}%
\end{pgfscope}%
\end{pgfscope}%
\begin{pgfscope}%
\definecolor{textcolor}{rgb}{0.000000,0.000000,0.000000}%
\pgfsetstrokecolor{textcolor}%
\pgfsetfillcolor{textcolor}%
\pgftext[x=3.740111in,y=0.430778in,,top]{\color{textcolor}\rmfamily\fontsize{10.000000}{12.000000}\selectfont \(\displaystyle {600}\)}%
\end{pgfscope}%
\begin{pgfscope}%
\pgfsetbuttcap%
\pgfsetroundjoin%
\definecolor{currentfill}{rgb}{0.000000,0.000000,0.000000}%
\pgfsetfillcolor{currentfill}%
\pgfsetlinewidth{0.803000pt}%
\definecolor{currentstroke}{rgb}{0.000000,0.000000,0.000000}%
\pgfsetstrokecolor{currentstroke}%
\pgfsetdash{}{0pt}%
\pgfsys@defobject{currentmarker}{\pgfqpoint{0.000000in}{-0.048611in}}{\pgfqpoint{0.000000in}{0.000000in}}{%
\pgfpathmoveto{\pgfqpoint{0.000000in}{0.000000in}}%
\pgfpathlineto{\pgfqpoint{0.000000in}{-0.048611in}}%
\pgfusepath{stroke,fill}%
}%
\begin{pgfscope}%
\pgfsys@transformshift{4.660334in}{0.528000in}%
\pgfsys@useobject{currentmarker}{}%
\end{pgfscope}%
\end{pgfscope}%
\begin{pgfscope}%
\definecolor{textcolor}{rgb}{0.000000,0.000000,0.000000}%
\pgfsetstrokecolor{textcolor}%
\pgfsetfillcolor{textcolor}%
\pgftext[x=4.660334in,y=0.430778in,,top]{\color{textcolor}\rmfamily\fontsize{10.000000}{12.000000}\selectfont \(\displaystyle {800}\)}%
\end{pgfscope}%
\begin{pgfscope}%
\pgfsetbuttcap%
\pgfsetroundjoin%
\definecolor{currentfill}{rgb}{0.000000,0.000000,0.000000}%
\pgfsetfillcolor{currentfill}%
\pgfsetlinewidth{0.803000pt}%
\definecolor{currentstroke}{rgb}{0.000000,0.000000,0.000000}%
\pgfsetstrokecolor{currentstroke}%
\pgfsetdash{}{0pt}%
\pgfsys@defobject{currentmarker}{\pgfqpoint{0.000000in}{-0.048611in}}{\pgfqpoint{0.000000in}{0.000000in}}{%
\pgfpathmoveto{\pgfqpoint{0.000000in}{0.000000in}}%
\pgfpathlineto{\pgfqpoint{0.000000in}{-0.048611in}}%
\pgfusepath{stroke,fill}%
}%
\begin{pgfscope}%
\pgfsys@transformshift{5.580557in}{0.528000in}%
\pgfsys@useobject{currentmarker}{}%
\end{pgfscope}%
\end{pgfscope}%
\begin{pgfscope}%
\definecolor{textcolor}{rgb}{0.000000,0.000000,0.000000}%
\pgfsetstrokecolor{textcolor}%
\pgfsetfillcolor{textcolor}%
\pgftext[x=5.580557in,y=0.430778in,,top]{\color{textcolor}\rmfamily\fontsize{10.000000}{12.000000}\selectfont \(\displaystyle {1000}\)}%
\end{pgfscope}%
\begin{pgfscope}%
\definecolor{textcolor}{rgb}{0.000000,0.000000,0.000000}%
\pgfsetstrokecolor{textcolor}%
\pgfsetfillcolor{textcolor}%
\pgftext[x=3.280000in,y=0.251766in,,top]{\color{textcolor}\rmfamily\fontsize{10.000000}{12.000000}\selectfont Input Size}%
\end{pgfscope}%
\begin{pgfscope}%
\pgfsetbuttcap%
\pgfsetroundjoin%
\definecolor{currentfill}{rgb}{0.000000,0.000000,0.000000}%
\pgfsetfillcolor{currentfill}%
\pgfsetlinewidth{0.803000pt}%
\definecolor{currentstroke}{rgb}{0.000000,0.000000,0.000000}%
\pgfsetstrokecolor{currentstroke}%
\pgfsetdash{}{0pt}%
\pgfsys@defobject{currentmarker}{\pgfqpoint{-0.048611in}{0.000000in}}{\pgfqpoint{-0.000000in}{0.000000in}}{%
\pgfpathmoveto{\pgfqpoint{-0.000000in}{0.000000in}}%
\pgfpathlineto{\pgfqpoint{-0.048611in}{0.000000in}}%
\pgfusepath{stroke,fill}%
}%
\begin{pgfscope}%
\pgfsys@transformshift{0.800000in}{1.043472in}%
\pgfsys@useobject{currentmarker}{}%
\end{pgfscope}%
\end{pgfscope}%
\begin{pgfscope}%
\definecolor{textcolor}{rgb}{0.000000,0.000000,0.000000}%
\pgfsetstrokecolor{textcolor}%
\pgfsetfillcolor{textcolor}%
\pgftext[x=0.501581in, y=0.995247in, left, base]{\color{textcolor}\rmfamily\fontsize{10.000000}{12.000000}\selectfont \(\displaystyle {10^{5}}\)}%
\end{pgfscope}%
\begin{pgfscope}%
\pgfsetbuttcap%
\pgfsetroundjoin%
\definecolor{currentfill}{rgb}{0.000000,0.000000,0.000000}%
\pgfsetfillcolor{currentfill}%
\pgfsetlinewidth{0.803000pt}%
\definecolor{currentstroke}{rgb}{0.000000,0.000000,0.000000}%
\pgfsetstrokecolor{currentstroke}%
\pgfsetdash{}{0pt}%
\pgfsys@defobject{currentmarker}{\pgfqpoint{-0.048611in}{0.000000in}}{\pgfqpoint{-0.000000in}{0.000000in}}{%
\pgfpathmoveto{\pgfqpoint{-0.000000in}{0.000000in}}%
\pgfpathlineto{\pgfqpoint{-0.048611in}{0.000000in}}%
\pgfusepath{stroke,fill}%
}%
\begin{pgfscope}%
\pgfsys@transformshift{0.800000in}{1.686538in}%
\pgfsys@useobject{currentmarker}{}%
\end{pgfscope}%
\end{pgfscope}%
\begin{pgfscope}%
\definecolor{textcolor}{rgb}{0.000000,0.000000,0.000000}%
\pgfsetstrokecolor{textcolor}%
\pgfsetfillcolor{textcolor}%
\pgftext[x=0.501581in, y=1.638313in, left, base]{\color{textcolor}\rmfamily\fontsize{10.000000}{12.000000}\selectfont \(\displaystyle {10^{6}}\)}%
\end{pgfscope}%
\begin{pgfscope}%
\pgfsetbuttcap%
\pgfsetroundjoin%
\definecolor{currentfill}{rgb}{0.000000,0.000000,0.000000}%
\pgfsetfillcolor{currentfill}%
\pgfsetlinewidth{0.803000pt}%
\definecolor{currentstroke}{rgb}{0.000000,0.000000,0.000000}%
\pgfsetstrokecolor{currentstroke}%
\pgfsetdash{}{0pt}%
\pgfsys@defobject{currentmarker}{\pgfqpoint{-0.048611in}{0.000000in}}{\pgfqpoint{-0.000000in}{0.000000in}}{%
\pgfpathmoveto{\pgfqpoint{-0.000000in}{0.000000in}}%
\pgfpathlineto{\pgfqpoint{-0.048611in}{0.000000in}}%
\pgfusepath{stroke,fill}%
}%
\begin{pgfscope}%
\pgfsys@transformshift{0.800000in}{2.329604in}%
\pgfsys@useobject{currentmarker}{}%
\end{pgfscope}%
\end{pgfscope}%
\begin{pgfscope}%
\definecolor{textcolor}{rgb}{0.000000,0.000000,0.000000}%
\pgfsetstrokecolor{textcolor}%
\pgfsetfillcolor{textcolor}%
\pgftext[x=0.501581in, y=2.281379in, left, base]{\color{textcolor}\rmfamily\fontsize{10.000000}{12.000000}\selectfont \(\displaystyle {10^{7}}\)}%
\end{pgfscope}%
\begin{pgfscope}%
\pgfsetbuttcap%
\pgfsetroundjoin%
\definecolor{currentfill}{rgb}{0.000000,0.000000,0.000000}%
\pgfsetfillcolor{currentfill}%
\pgfsetlinewidth{0.803000pt}%
\definecolor{currentstroke}{rgb}{0.000000,0.000000,0.000000}%
\pgfsetstrokecolor{currentstroke}%
\pgfsetdash{}{0pt}%
\pgfsys@defobject{currentmarker}{\pgfqpoint{-0.048611in}{0.000000in}}{\pgfqpoint{-0.000000in}{0.000000in}}{%
\pgfpathmoveto{\pgfqpoint{-0.000000in}{0.000000in}}%
\pgfpathlineto{\pgfqpoint{-0.048611in}{0.000000in}}%
\pgfusepath{stroke,fill}%
}%
\begin{pgfscope}%
\pgfsys@transformshift{0.800000in}{2.972671in}%
\pgfsys@useobject{currentmarker}{}%
\end{pgfscope}%
\end{pgfscope}%
\begin{pgfscope}%
\definecolor{textcolor}{rgb}{0.000000,0.000000,0.000000}%
\pgfsetstrokecolor{textcolor}%
\pgfsetfillcolor{textcolor}%
\pgftext[x=0.501581in, y=2.924445in, left, base]{\color{textcolor}\rmfamily\fontsize{10.000000}{12.000000}\selectfont \(\displaystyle {10^{8}}\)}%
\end{pgfscope}%
\begin{pgfscope}%
\pgfsetbuttcap%
\pgfsetroundjoin%
\definecolor{currentfill}{rgb}{0.000000,0.000000,0.000000}%
\pgfsetfillcolor{currentfill}%
\pgfsetlinewidth{0.803000pt}%
\definecolor{currentstroke}{rgb}{0.000000,0.000000,0.000000}%
\pgfsetstrokecolor{currentstroke}%
\pgfsetdash{}{0pt}%
\pgfsys@defobject{currentmarker}{\pgfqpoint{-0.048611in}{0.000000in}}{\pgfqpoint{-0.000000in}{0.000000in}}{%
\pgfpathmoveto{\pgfqpoint{-0.000000in}{0.000000in}}%
\pgfpathlineto{\pgfqpoint{-0.048611in}{0.000000in}}%
\pgfusepath{stroke,fill}%
}%
\begin{pgfscope}%
\pgfsys@transformshift{0.800000in}{3.615737in}%
\pgfsys@useobject{currentmarker}{}%
\end{pgfscope}%
\end{pgfscope}%
\begin{pgfscope}%
\definecolor{textcolor}{rgb}{0.000000,0.000000,0.000000}%
\pgfsetstrokecolor{textcolor}%
\pgfsetfillcolor{textcolor}%
\pgftext[x=0.501581in, y=3.567511in, left, base]{\color{textcolor}\rmfamily\fontsize{10.000000}{12.000000}\selectfont \(\displaystyle {10^{9}}\)}%
\end{pgfscope}%
\begin{pgfscope}%
\pgfsetbuttcap%
\pgfsetroundjoin%
\definecolor{currentfill}{rgb}{0.000000,0.000000,0.000000}%
\pgfsetfillcolor{currentfill}%
\pgfsetlinewidth{0.602250pt}%
\definecolor{currentstroke}{rgb}{0.000000,0.000000,0.000000}%
\pgfsetstrokecolor{currentstroke}%
\pgfsetdash{}{0pt}%
\pgfsys@defobject{currentmarker}{\pgfqpoint{-0.027778in}{0.000000in}}{\pgfqpoint{-0.000000in}{0.000000in}}{%
\pgfpathmoveto{\pgfqpoint{-0.000000in}{0.000000in}}%
\pgfpathlineto{\pgfqpoint{-0.027778in}{0.000000in}}%
\pgfusepath{stroke,fill}%
}%
\begin{pgfscope}%
\pgfsys@transformshift{0.800000in}{0.593988in}%
\pgfsys@useobject{currentmarker}{}%
\end{pgfscope}%
\end{pgfscope}%
\begin{pgfscope}%
\pgfsetbuttcap%
\pgfsetroundjoin%
\definecolor{currentfill}{rgb}{0.000000,0.000000,0.000000}%
\pgfsetfillcolor{currentfill}%
\pgfsetlinewidth{0.602250pt}%
\definecolor{currentstroke}{rgb}{0.000000,0.000000,0.000000}%
\pgfsetstrokecolor{currentstroke}%
\pgfsetdash{}{0pt}%
\pgfsys@defobject{currentmarker}{\pgfqpoint{-0.027778in}{0.000000in}}{\pgfqpoint{-0.000000in}{0.000000in}}{%
\pgfpathmoveto{\pgfqpoint{-0.000000in}{0.000000in}}%
\pgfpathlineto{\pgfqpoint{-0.027778in}{0.000000in}}%
\pgfusepath{stroke,fill}%
}%
\begin{pgfscope}%
\pgfsys@transformshift{0.800000in}{0.707226in}%
\pgfsys@useobject{currentmarker}{}%
\end{pgfscope}%
\end{pgfscope}%
\begin{pgfscope}%
\pgfsetbuttcap%
\pgfsetroundjoin%
\definecolor{currentfill}{rgb}{0.000000,0.000000,0.000000}%
\pgfsetfillcolor{currentfill}%
\pgfsetlinewidth{0.602250pt}%
\definecolor{currentstroke}{rgb}{0.000000,0.000000,0.000000}%
\pgfsetstrokecolor{currentstroke}%
\pgfsetdash{}{0pt}%
\pgfsys@defobject{currentmarker}{\pgfqpoint{-0.027778in}{0.000000in}}{\pgfqpoint{-0.000000in}{0.000000in}}{%
\pgfpathmoveto{\pgfqpoint{-0.000000in}{0.000000in}}%
\pgfpathlineto{\pgfqpoint{-0.027778in}{0.000000in}}%
\pgfusepath{stroke,fill}%
}%
\begin{pgfscope}%
\pgfsys@transformshift{0.800000in}{0.787570in}%
\pgfsys@useobject{currentmarker}{}%
\end{pgfscope}%
\end{pgfscope}%
\begin{pgfscope}%
\pgfsetbuttcap%
\pgfsetroundjoin%
\definecolor{currentfill}{rgb}{0.000000,0.000000,0.000000}%
\pgfsetfillcolor{currentfill}%
\pgfsetlinewidth{0.602250pt}%
\definecolor{currentstroke}{rgb}{0.000000,0.000000,0.000000}%
\pgfsetstrokecolor{currentstroke}%
\pgfsetdash{}{0pt}%
\pgfsys@defobject{currentmarker}{\pgfqpoint{-0.027778in}{0.000000in}}{\pgfqpoint{-0.000000in}{0.000000in}}{%
\pgfpathmoveto{\pgfqpoint{-0.000000in}{0.000000in}}%
\pgfpathlineto{\pgfqpoint{-0.027778in}{0.000000in}}%
\pgfusepath{stroke,fill}%
}%
\begin{pgfscope}%
\pgfsys@transformshift{0.800000in}{0.849890in}%
\pgfsys@useobject{currentmarker}{}%
\end{pgfscope}%
\end{pgfscope}%
\begin{pgfscope}%
\pgfsetbuttcap%
\pgfsetroundjoin%
\definecolor{currentfill}{rgb}{0.000000,0.000000,0.000000}%
\pgfsetfillcolor{currentfill}%
\pgfsetlinewidth{0.602250pt}%
\definecolor{currentstroke}{rgb}{0.000000,0.000000,0.000000}%
\pgfsetstrokecolor{currentstroke}%
\pgfsetdash{}{0pt}%
\pgfsys@defobject{currentmarker}{\pgfqpoint{-0.027778in}{0.000000in}}{\pgfqpoint{-0.000000in}{0.000000in}}{%
\pgfpathmoveto{\pgfqpoint{-0.000000in}{0.000000in}}%
\pgfpathlineto{\pgfqpoint{-0.027778in}{0.000000in}}%
\pgfusepath{stroke,fill}%
}%
\begin{pgfscope}%
\pgfsys@transformshift{0.800000in}{0.900808in}%
\pgfsys@useobject{currentmarker}{}%
\end{pgfscope}%
\end{pgfscope}%
\begin{pgfscope}%
\pgfsetbuttcap%
\pgfsetroundjoin%
\definecolor{currentfill}{rgb}{0.000000,0.000000,0.000000}%
\pgfsetfillcolor{currentfill}%
\pgfsetlinewidth{0.602250pt}%
\definecolor{currentstroke}{rgb}{0.000000,0.000000,0.000000}%
\pgfsetstrokecolor{currentstroke}%
\pgfsetdash{}{0pt}%
\pgfsys@defobject{currentmarker}{\pgfqpoint{-0.027778in}{0.000000in}}{\pgfqpoint{-0.000000in}{0.000000in}}{%
\pgfpathmoveto{\pgfqpoint{-0.000000in}{0.000000in}}%
\pgfpathlineto{\pgfqpoint{-0.027778in}{0.000000in}}%
\pgfusepath{stroke,fill}%
}%
\begin{pgfscope}%
\pgfsys@transformshift{0.800000in}{0.943860in}%
\pgfsys@useobject{currentmarker}{}%
\end{pgfscope}%
\end{pgfscope}%
\begin{pgfscope}%
\pgfsetbuttcap%
\pgfsetroundjoin%
\definecolor{currentfill}{rgb}{0.000000,0.000000,0.000000}%
\pgfsetfillcolor{currentfill}%
\pgfsetlinewidth{0.602250pt}%
\definecolor{currentstroke}{rgb}{0.000000,0.000000,0.000000}%
\pgfsetstrokecolor{currentstroke}%
\pgfsetdash{}{0pt}%
\pgfsys@defobject{currentmarker}{\pgfqpoint{-0.027778in}{0.000000in}}{\pgfqpoint{-0.000000in}{0.000000in}}{%
\pgfpathmoveto{\pgfqpoint{-0.000000in}{0.000000in}}%
\pgfpathlineto{\pgfqpoint{-0.027778in}{0.000000in}}%
\pgfusepath{stroke,fill}%
}%
\begin{pgfscope}%
\pgfsys@transformshift{0.800000in}{0.981152in}%
\pgfsys@useobject{currentmarker}{}%
\end{pgfscope}%
\end{pgfscope}%
\begin{pgfscope}%
\pgfsetbuttcap%
\pgfsetroundjoin%
\definecolor{currentfill}{rgb}{0.000000,0.000000,0.000000}%
\pgfsetfillcolor{currentfill}%
\pgfsetlinewidth{0.602250pt}%
\definecolor{currentstroke}{rgb}{0.000000,0.000000,0.000000}%
\pgfsetstrokecolor{currentstroke}%
\pgfsetdash{}{0pt}%
\pgfsys@defobject{currentmarker}{\pgfqpoint{-0.027778in}{0.000000in}}{\pgfqpoint{-0.000000in}{0.000000in}}{%
\pgfpathmoveto{\pgfqpoint{-0.000000in}{0.000000in}}%
\pgfpathlineto{\pgfqpoint{-0.027778in}{0.000000in}}%
\pgfusepath{stroke,fill}%
}%
\begin{pgfscope}%
\pgfsys@transformshift{0.800000in}{1.014047in}%
\pgfsys@useobject{currentmarker}{}%
\end{pgfscope}%
\end{pgfscope}%
\begin{pgfscope}%
\pgfsetbuttcap%
\pgfsetroundjoin%
\definecolor{currentfill}{rgb}{0.000000,0.000000,0.000000}%
\pgfsetfillcolor{currentfill}%
\pgfsetlinewidth{0.602250pt}%
\definecolor{currentstroke}{rgb}{0.000000,0.000000,0.000000}%
\pgfsetstrokecolor{currentstroke}%
\pgfsetdash{}{0pt}%
\pgfsys@defobject{currentmarker}{\pgfqpoint{-0.027778in}{0.000000in}}{\pgfqpoint{-0.000000in}{0.000000in}}{%
\pgfpathmoveto{\pgfqpoint{-0.000000in}{0.000000in}}%
\pgfpathlineto{\pgfqpoint{-0.027778in}{0.000000in}}%
\pgfusepath{stroke,fill}%
}%
\begin{pgfscope}%
\pgfsys@transformshift{0.800000in}{1.237054in}%
\pgfsys@useobject{currentmarker}{}%
\end{pgfscope}%
\end{pgfscope}%
\begin{pgfscope}%
\pgfsetbuttcap%
\pgfsetroundjoin%
\definecolor{currentfill}{rgb}{0.000000,0.000000,0.000000}%
\pgfsetfillcolor{currentfill}%
\pgfsetlinewidth{0.602250pt}%
\definecolor{currentstroke}{rgb}{0.000000,0.000000,0.000000}%
\pgfsetstrokecolor{currentstroke}%
\pgfsetdash{}{0pt}%
\pgfsys@defobject{currentmarker}{\pgfqpoint{-0.027778in}{0.000000in}}{\pgfqpoint{-0.000000in}{0.000000in}}{%
\pgfpathmoveto{\pgfqpoint{-0.000000in}{0.000000in}}%
\pgfpathlineto{\pgfqpoint{-0.027778in}{0.000000in}}%
\pgfusepath{stroke,fill}%
}%
\begin{pgfscope}%
\pgfsys@transformshift{0.800000in}{1.350292in}%
\pgfsys@useobject{currentmarker}{}%
\end{pgfscope}%
\end{pgfscope}%
\begin{pgfscope}%
\pgfsetbuttcap%
\pgfsetroundjoin%
\definecolor{currentfill}{rgb}{0.000000,0.000000,0.000000}%
\pgfsetfillcolor{currentfill}%
\pgfsetlinewidth{0.602250pt}%
\definecolor{currentstroke}{rgb}{0.000000,0.000000,0.000000}%
\pgfsetstrokecolor{currentstroke}%
\pgfsetdash{}{0pt}%
\pgfsys@defobject{currentmarker}{\pgfqpoint{-0.027778in}{0.000000in}}{\pgfqpoint{-0.000000in}{0.000000in}}{%
\pgfpathmoveto{\pgfqpoint{-0.000000in}{0.000000in}}%
\pgfpathlineto{\pgfqpoint{-0.027778in}{0.000000in}}%
\pgfusepath{stroke,fill}%
}%
\begin{pgfscope}%
\pgfsys@transformshift{0.800000in}{1.430636in}%
\pgfsys@useobject{currentmarker}{}%
\end{pgfscope}%
\end{pgfscope}%
\begin{pgfscope}%
\pgfsetbuttcap%
\pgfsetroundjoin%
\definecolor{currentfill}{rgb}{0.000000,0.000000,0.000000}%
\pgfsetfillcolor{currentfill}%
\pgfsetlinewidth{0.602250pt}%
\definecolor{currentstroke}{rgb}{0.000000,0.000000,0.000000}%
\pgfsetstrokecolor{currentstroke}%
\pgfsetdash{}{0pt}%
\pgfsys@defobject{currentmarker}{\pgfqpoint{-0.027778in}{0.000000in}}{\pgfqpoint{-0.000000in}{0.000000in}}{%
\pgfpathmoveto{\pgfqpoint{-0.000000in}{0.000000in}}%
\pgfpathlineto{\pgfqpoint{-0.027778in}{0.000000in}}%
\pgfusepath{stroke,fill}%
}%
\begin{pgfscope}%
\pgfsys@transformshift{0.800000in}{1.492956in}%
\pgfsys@useobject{currentmarker}{}%
\end{pgfscope}%
\end{pgfscope}%
\begin{pgfscope}%
\pgfsetbuttcap%
\pgfsetroundjoin%
\definecolor{currentfill}{rgb}{0.000000,0.000000,0.000000}%
\pgfsetfillcolor{currentfill}%
\pgfsetlinewidth{0.602250pt}%
\definecolor{currentstroke}{rgb}{0.000000,0.000000,0.000000}%
\pgfsetstrokecolor{currentstroke}%
\pgfsetdash{}{0pt}%
\pgfsys@defobject{currentmarker}{\pgfqpoint{-0.027778in}{0.000000in}}{\pgfqpoint{-0.000000in}{0.000000in}}{%
\pgfpathmoveto{\pgfqpoint{-0.000000in}{0.000000in}}%
\pgfpathlineto{\pgfqpoint{-0.027778in}{0.000000in}}%
\pgfusepath{stroke,fill}%
}%
\begin{pgfscope}%
\pgfsys@transformshift{0.800000in}{1.543875in}%
\pgfsys@useobject{currentmarker}{}%
\end{pgfscope}%
\end{pgfscope}%
\begin{pgfscope}%
\pgfsetbuttcap%
\pgfsetroundjoin%
\definecolor{currentfill}{rgb}{0.000000,0.000000,0.000000}%
\pgfsetfillcolor{currentfill}%
\pgfsetlinewidth{0.602250pt}%
\definecolor{currentstroke}{rgb}{0.000000,0.000000,0.000000}%
\pgfsetstrokecolor{currentstroke}%
\pgfsetdash{}{0pt}%
\pgfsys@defobject{currentmarker}{\pgfqpoint{-0.027778in}{0.000000in}}{\pgfqpoint{-0.000000in}{0.000000in}}{%
\pgfpathmoveto{\pgfqpoint{-0.000000in}{0.000000in}}%
\pgfpathlineto{\pgfqpoint{-0.027778in}{0.000000in}}%
\pgfusepath{stroke,fill}%
}%
\begin{pgfscope}%
\pgfsys@transformshift{0.800000in}{1.586926in}%
\pgfsys@useobject{currentmarker}{}%
\end{pgfscope}%
\end{pgfscope}%
\begin{pgfscope}%
\pgfsetbuttcap%
\pgfsetroundjoin%
\definecolor{currentfill}{rgb}{0.000000,0.000000,0.000000}%
\pgfsetfillcolor{currentfill}%
\pgfsetlinewidth{0.602250pt}%
\definecolor{currentstroke}{rgb}{0.000000,0.000000,0.000000}%
\pgfsetstrokecolor{currentstroke}%
\pgfsetdash{}{0pt}%
\pgfsys@defobject{currentmarker}{\pgfqpoint{-0.027778in}{0.000000in}}{\pgfqpoint{-0.000000in}{0.000000in}}{%
\pgfpathmoveto{\pgfqpoint{-0.000000in}{0.000000in}}%
\pgfpathlineto{\pgfqpoint{-0.027778in}{0.000000in}}%
\pgfusepath{stroke,fill}%
}%
\begin{pgfscope}%
\pgfsys@transformshift{0.800000in}{1.624219in}%
\pgfsys@useobject{currentmarker}{}%
\end{pgfscope}%
\end{pgfscope}%
\begin{pgfscope}%
\pgfsetbuttcap%
\pgfsetroundjoin%
\definecolor{currentfill}{rgb}{0.000000,0.000000,0.000000}%
\pgfsetfillcolor{currentfill}%
\pgfsetlinewidth{0.602250pt}%
\definecolor{currentstroke}{rgb}{0.000000,0.000000,0.000000}%
\pgfsetstrokecolor{currentstroke}%
\pgfsetdash{}{0pt}%
\pgfsys@defobject{currentmarker}{\pgfqpoint{-0.027778in}{0.000000in}}{\pgfqpoint{-0.000000in}{0.000000in}}{%
\pgfpathmoveto{\pgfqpoint{-0.000000in}{0.000000in}}%
\pgfpathlineto{\pgfqpoint{-0.027778in}{0.000000in}}%
\pgfusepath{stroke,fill}%
}%
\begin{pgfscope}%
\pgfsys@transformshift{0.800000in}{1.657113in}%
\pgfsys@useobject{currentmarker}{}%
\end{pgfscope}%
\end{pgfscope}%
\begin{pgfscope}%
\pgfsetbuttcap%
\pgfsetroundjoin%
\definecolor{currentfill}{rgb}{0.000000,0.000000,0.000000}%
\pgfsetfillcolor{currentfill}%
\pgfsetlinewidth{0.602250pt}%
\definecolor{currentstroke}{rgb}{0.000000,0.000000,0.000000}%
\pgfsetstrokecolor{currentstroke}%
\pgfsetdash{}{0pt}%
\pgfsys@defobject{currentmarker}{\pgfqpoint{-0.027778in}{0.000000in}}{\pgfqpoint{-0.000000in}{0.000000in}}{%
\pgfpathmoveto{\pgfqpoint{-0.000000in}{0.000000in}}%
\pgfpathlineto{\pgfqpoint{-0.027778in}{0.000000in}}%
\pgfusepath{stroke,fill}%
}%
\begin{pgfscope}%
\pgfsys@transformshift{0.800000in}{1.880120in}%
\pgfsys@useobject{currentmarker}{}%
\end{pgfscope}%
\end{pgfscope}%
\begin{pgfscope}%
\pgfsetbuttcap%
\pgfsetroundjoin%
\definecolor{currentfill}{rgb}{0.000000,0.000000,0.000000}%
\pgfsetfillcolor{currentfill}%
\pgfsetlinewidth{0.602250pt}%
\definecolor{currentstroke}{rgb}{0.000000,0.000000,0.000000}%
\pgfsetstrokecolor{currentstroke}%
\pgfsetdash{}{0pt}%
\pgfsys@defobject{currentmarker}{\pgfqpoint{-0.027778in}{0.000000in}}{\pgfqpoint{-0.000000in}{0.000000in}}{%
\pgfpathmoveto{\pgfqpoint{-0.000000in}{0.000000in}}%
\pgfpathlineto{\pgfqpoint{-0.027778in}{0.000000in}}%
\pgfusepath{stroke,fill}%
}%
\begin{pgfscope}%
\pgfsys@transformshift{0.800000in}{1.993359in}%
\pgfsys@useobject{currentmarker}{}%
\end{pgfscope}%
\end{pgfscope}%
\begin{pgfscope}%
\pgfsetbuttcap%
\pgfsetroundjoin%
\definecolor{currentfill}{rgb}{0.000000,0.000000,0.000000}%
\pgfsetfillcolor{currentfill}%
\pgfsetlinewidth{0.602250pt}%
\definecolor{currentstroke}{rgb}{0.000000,0.000000,0.000000}%
\pgfsetstrokecolor{currentstroke}%
\pgfsetdash{}{0pt}%
\pgfsys@defobject{currentmarker}{\pgfqpoint{-0.027778in}{0.000000in}}{\pgfqpoint{-0.000000in}{0.000000in}}{%
\pgfpathmoveto{\pgfqpoint{-0.000000in}{0.000000in}}%
\pgfpathlineto{\pgfqpoint{-0.027778in}{0.000000in}}%
\pgfusepath{stroke,fill}%
}%
\begin{pgfscope}%
\pgfsys@transformshift{0.800000in}{2.073703in}%
\pgfsys@useobject{currentmarker}{}%
\end{pgfscope}%
\end{pgfscope}%
\begin{pgfscope}%
\pgfsetbuttcap%
\pgfsetroundjoin%
\definecolor{currentfill}{rgb}{0.000000,0.000000,0.000000}%
\pgfsetfillcolor{currentfill}%
\pgfsetlinewidth{0.602250pt}%
\definecolor{currentstroke}{rgb}{0.000000,0.000000,0.000000}%
\pgfsetstrokecolor{currentstroke}%
\pgfsetdash{}{0pt}%
\pgfsys@defobject{currentmarker}{\pgfqpoint{-0.027778in}{0.000000in}}{\pgfqpoint{-0.000000in}{0.000000in}}{%
\pgfpathmoveto{\pgfqpoint{-0.000000in}{0.000000in}}%
\pgfpathlineto{\pgfqpoint{-0.027778in}{0.000000in}}%
\pgfusepath{stroke,fill}%
}%
\begin{pgfscope}%
\pgfsys@transformshift{0.800000in}{2.136022in}%
\pgfsys@useobject{currentmarker}{}%
\end{pgfscope}%
\end{pgfscope}%
\begin{pgfscope}%
\pgfsetbuttcap%
\pgfsetroundjoin%
\definecolor{currentfill}{rgb}{0.000000,0.000000,0.000000}%
\pgfsetfillcolor{currentfill}%
\pgfsetlinewidth{0.602250pt}%
\definecolor{currentstroke}{rgb}{0.000000,0.000000,0.000000}%
\pgfsetstrokecolor{currentstroke}%
\pgfsetdash{}{0pt}%
\pgfsys@defobject{currentmarker}{\pgfqpoint{-0.027778in}{0.000000in}}{\pgfqpoint{-0.000000in}{0.000000in}}{%
\pgfpathmoveto{\pgfqpoint{-0.000000in}{0.000000in}}%
\pgfpathlineto{\pgfqpoint{-0.027778in}{0.000000in}}%
\pgfusepath{stroke,fill}%
}%
\begin{pgfscope}%
\pgfsys@transformshift{0.800000in}{2.186941in}%
\pgfsys@useobject{currentmarker}{}%
\end{pgfscope}%
\end{pgfscope}%
\begin{pgfscope}%
\pgfsetbuttcap%
\pgfsetroundjoin%
\definecolor{currentfill}{rgb}{0.000000,0.000000,0.000000}%
\pgfsetfillcolor{currentfill}%
\pgfsetlinewidth{0.602250pt}%
\definecolor{currentstroke}{rgb}{0.000000,0.000000,0.000000}%
\pgfsetstrokecolor{currentstroke}%
\pgfsetdash{}{0pt}%
\pgfsys@defobject{currentmarker}{\pgfqpoint{-0.027778in}{0.000000in}}{\pgfqpoint{-0.000000in}{0.000000in}}{%
\pgfpathmoveto{\pgfqpoint{-0.000000in}{0.000000in}}%
\pgfpathlineto{\pgfqpoint{-0.027778in}{0.000000in}}%
\pgfusepath{stroke,fill}%
}%
\begin{pgfscope}%
\pgfsys@transformshift{0.800000in}{2.229992in}%
\pgfsys@useobject{currentmarker}{}%
\end{pgfscope}%
\end{pgfscope}%
\begin{pgfscope}%
\pgfsetbuttcap%
\pgfsetroundjoin%
\definecolor{currentfill}{rgb}{0.000000,0.000000,0.000000}%
\pgfsetfillcolor{currentfill}%
\pgfsetlinewidth{0.602250pt}%
\definecolor{currentstroke}{rgb}{0.000000,0.000000,0.000000}%
\pgfsetstrokecolor{currentstroke}%
\pgfsetdash{}{0pt}%
\pgfsys@defobject{currentmarker}{\pgfqpoint{-0.027778in}{0.000000in}}{\pgfqpoint{-0.000000in}{0.000000in}}{%
\pgfpathmoveto{\pgfqpoint{-0.000000in}{0.000000in}}%
\pgfpathlineto{\pgfqpoint{-0.027778in}{0.000000in}}%
\pgfusepath{stroke,fill}%
}%
\begin{pgfscope}%
\pgfsys@transformshift{0.800000in}{2.267285in}%
\pgfsys@useobject{currentmarker}{}%
\end{pgfscope}%
\end{pgfscope}%
\begin{pgfscope}%
\pgfsetbuttcap%
\pgfsetroundjoin%
\definecolor{currentfill}{rgb}{0.000000,0.000000,0.000000}%
\pgfsetfillcolor{currentfill}%
\pgfsetlinewidth{0.602250pt}%
\definecolor{currentstroke}{rgb}{0.000000,0.000000,0.000000}%
\pgfsetstrokecolor{currentstroke}%
\pgfsetdash{}{0pt}%
\pgfsys@defobject{currentmarker}{\pgfqpoint{-0.027778in}{0.000000in}}{\pgfqpoint{-0.000000in}{0.000000in}}{%
\pgfpathmoveto{\pgfqpoint{-0.000000in}{0.000000in}}%
\pgfpathlineto{\pgfqpoint{-0.027778in}{0.000000in}}%
\pgfusepath{stroke,fill}%
}%
\begin{pgfscope}%
\pgfsys@transformshift{0.800000in}{2.300179in}%
\pgfsys@useobject{currentmarker}{}%
\end{pgfscope}%
\end{pgfscope}%
\begin{pgfscope}%
\pgfsetbuttcap%
\pgfsetroundjoin%
\definecolor{currentfill}{rgb}{0.000000,0.000000,0.000000}%
\pgfsetfillcolor{currentfill}%
\pgfsetlinewidth{0.602250pt}%
\definecolor{currentstroke}{rgb}{0.000000,0.000000,0.000000}%
\pgfsetstrokecolor{currentstroke}%
\pgfsetdash{}{0pt}%
\pgfsys@defobject{currentmarker}{\pgfqpoint{-0.027778in}{0.000000in}}{\pgfqpoint{-0.000000in}{0.000000in}}{%
\pgfpathmoveto{\pgfqpoint{-0.000000in}{0.000000in}}%
\pgfpathlineto{\pgfqpoint{-0.027778in}{0.000000in}}%
\pgfusepath{stroke,fill}%
}%
\begin{pgfscope}%
\pgfsys@transformshift{0.800000in}{2.523187in}%
\pgfsys@useobject{currentmarker}{}%
\end{pgfscope}%
\end{pgfscope}%
\begin{pgfscope}%
\pgfsetbuttcap%
\pgfsetroundjoin%
\definecolor{currentfill}{rgb}{0.000000,0.000000,0.000000}%
\pgfsetfillcolor{currentfill}%
\pgfsetlinewidth{0.602250pt}%
\definecolor{currentstroke}{rgb}{0.000000,0.000000,0.000000}%
\pgfsetstrokecolor{currentstroke}%
\pgfsetdash{}{0pt}%
\pgfsys@defobject{currentmarker}{\pgfqpoint{-0.027778in}{0.000000in}}{\pgfqpoint{-0.000000in}{0.000000in}}{%
\pgfpathmoveto{\pgfqpoint{-0.000000in}{0.000000in}}%
\pgfpathlineto{\pgfqpoint{-0.027778in}{0.000000in}}%
\pgfusepath{stroke,fill}%
}%
\begin{pgfscope}%
\pgfsys@transformshift{0.800000in}{2.636425in}%
\pgfsys@useobject{currentmarker}{}%
\end{pgfscope}%
\end{pgfscope}%
\begin{pgfscope}%
\pgfsetbuttcap%
\pgfsetroundjoin%
\definecolor{currentfill}{rgb}{0.000000,0.000000,0.000000}%
\pgfsetfillcolor{currentfill}%
\pgfsetlinewidth{0.602250pt}%
\definecolor{currentstroke}{rgb}{0.000000,0.000000,0.000000}%
\pgfsetstrokecolor{currentstroke}%
\pgfsetdash{}{0pt}%
\pgfsys@defobject{currentmarker}{\pgfqpoint{-0.027778in}{0.000000in}}{\pgfqpoint{-0.000000in}{0.000000in}}{%
\pgfpathmoveto{\pgfqpoint{-0.000000in}{0.000000in}}%
\pgfpathlineto{\pgfqpoint{-0.027778in}{0.000000in}}%
\pgfusepath{stroke,fill}%
}%
\begin{pgfscope}%
\pgfsys@transformshift{0.800000in}{2.716769in}%
\pgfsys@useobject{currentmarker}{}%
\end{pgfscope}%
\end{pgfscope}%
\begin{pgfscope}%
\pgfsetbuttcap%
\pgfsetroundjoin%
\definecolor{currentfill}{rgb}{0.000000,0.000000,0.000000}%
\pgfsetfillcolor{currentfill}%
\pgfsetlinewidth{0.602250pt}%
\definecolor{currentstroke}{rgb}{0.000000,0.000000,0.000000}%
\pgfsetstrokecolor{currentstroke}%
\pgfsetdash{}{0pt}%
\pgfsys@defobject{currentmarker}{\pgfqpoint{-0.027778in}{0.000000in}}{\pgfqpoint{-0.000000in}{0.000000in}}{%
\pgfpathmoveto{\pgfqpoint{-0.000000in}{0.000000in}}%
\pgfpathlineto{\pgfqpoint{-0.027778in}{0.000000in}}%
\pgfusepath{stroke,fill}%
}%
\begin{pgfscope}%
\pgfsys@transformshift{0.800000in}{2.779088in}%
\pgfsys@useobject{currentmarker}{}%
\end{pgfscope}%
\end{pgfscope}%
\begin{pgfscope}%
\pgfsetbuttcap%
\pgfsetroundjoin%
\definecolor{currentfill}{rgb}{0.000000,0.000000,0.000000}%
\pgfsetfillcolor{currentfill}%
\pgfsetlinewidth{0.602250pt}%
\definecolor{currentstroke}{rgb}{0.000000,0.000000,0.000000}%
\pgfsetstrokecolor{currentstroke}%
\pgfsetdash{}{0pt}%
\pgfsys@defobject{currentmarker}{\pgfqpoint{-0.027778in}{0.000000in}}{\pgfqpoint{-0.000000in}{0.000000in}}{%
\pgfpathmoveto{\pgfqpoint{-0.000000in}{0.000000in}}%
\pgfpathlineto{\pgfqpoint{-0.027778in}{0.000000in}}%
\pgfusepath{stroke,fill}%
}%
\begin{pgfscope}%
\pgfsys@transformshift{0.800000in}{2.830007in}%
\pgfsys@useobject{currentmarker}{}%
\end{pgfscope}%
\end{pgfscope}%
\begin{pgfscope}%
\pgfsetbuttcap%
\pgfsetroundjoin%
\definecolor{currentfill}{rgb}{0.000000,0.000000,0.000000}%
\pgfsetfillcolor{currentfill}%
\pgfsetlinewidth{0.602250pt}%
\definecolor{currentstroke}{rgb}{0.000000,0.000000,0.000000}%
\pgfsetstrokecolor{currentstroke}%
\pgfsetdash{}{0pt}%
\pgfsys@defobject{currentmarker}{\pgfqpoint{-0.027778in}{0.000000in}}{\pgfqpoint{-0.000000in}{0.000000in}}{%
\pgfpathmoveto{\pgfqpoint{-0.000000in}{0.000000in}}%
\pgfpathlineto{\pgfqpoint{-0.027778in}{0.000000in}}%
\pgfusepath{stroke,fill}%
}%
\begin{pgfscope}%
\pgfsys@transformshift{0.800000in}{2.873058in}%
\pgfsys@useobject{currentmarker}{}%
\end{pgfscope}%
\end{pgfscope}%
\begin{pgfscope}%
\pgfsetbuttcap%
\pgfsetroundjoin%
\definecolor{currentfill}{rgb}{0.000000,0.000000,0.000000}%
\pgfsetfillcolor{currentfill}%
\pgfsetlinewidth{0.602250pt}%
\definecolor{currentstroke}{rgb}{0.000000,0.000000,0.000000}%
\pgfsetstrokecolor{currentstroke}%
\pgfsetdash{}{0pt}%
\pgfsys@defobject{currentmarker}{\pgfqpoint{-0.027778in}{0.000000in}}{\pgfqpoint{-0.000000in}{0.000000in}}{%
\pgfpathmoveto{\pgfqpoint{-0.000000in}{0.000000in}}%
\pgfpathlineto{\pgfqpoint{-0.027778in}{0.000000in}}%
\pgfusepath{stroke,fill}%
}%
\begin{pgfscope}%
\pgfsys@transformshift{0.800000in}{2.910351in}%
\pgfsys@useobject{currentmarker}{}%
\end{pgfscope}%
\end{pgfscope}%
\begin{pgfscope}%
\pgfsetbuttcap%
\pgfsetroundjoin%
\definecolor{currentfill}{rgb}{0.000000,0.000000,0.000000}%
\pgfsetfillcolor{currentfill}%
\pgfsetlinewidth{0.602250pt}%
\definecolor{currentstroke}{rgb}{0.000000,0.000000,0.000000}%
\pgfsetstrokecolor{currentstroke}%
\pgfsetdash{}{0pt}%
\pgfsys@defobject{currentmarker}{\pgfqpoint{-0.027778in}{0.000000in}}{\pgfqpoint{-0.000000in}{0.000000in}}{%
\pgfpathmoveto{\pgfqpoint{-0.000000in}{0.000000in}}%
\pgfpathlineto{\pgfqpoint{-0.027778in}{0.000000in}}%
\pgfusepath{stroke,fill}%
}%
\begin{pgfscope}%
\pgfsys@transformshift{0.800000in}{2.943245in}%
\pgfsys@useobject{currentmarker}{}%
\end{pgfscope}%
\end{pgfscope}%
\begin{pgfscope}%
\pgfsetbuttcap%
\pgfsetroundjoin%
\definecolor{currentfill}{rgb}{0.000000,0.000000,0.000000}%
\pgfsetfillcolor{currentfill}%
\pgfsetlinewidth{0.602250pt}%
\definecolor{currentstroke}{rgb}{0.000000,0.000000,0.000000}%
\pgfsetstrokecolor{currentstroke}%
\pgfsetdash{}{0pt}%
\pgfsys@defobject{currentmarker}{\pgfqpoint{-0.027778in}{0.000000in}}{\pgfqpoint{-0.000000in}{0.000000in}}{%
\pgfpathmoveto{\pgfqpoint{-0.000000in}{0.000000in}}%
\pgfpathlineto{\pgfqpoint{-0.027778in}{0.000000in}}%
\pgfusepath{stroke,fill}%
}%
\begin{pgfscope}%
\pgfsys@transformshift{0.800000in}{3.166253in}%
\pgfsys@useobject{currentmarker}{}%
\end{pgfscope}%
\end{pgfscope}%
\begin{pgfscope}%
\pgfsetbuttcap%
\pgfsetroundjoin%
\definecolor{currentfill}{rgb}{0.000000,0.000000,0.000000}%
\pgfsetfillcolor{currentfill}%
\pgfsetlinewidth{0.602250pt}%
\definecolor{currentstroke}{rgb}{0.000000,0.000000,0.000000}%
\pgfsetstrokecolor{currentstroke}%
\pgfsetdash{}{0pt}%
\pgfsys@defobject{currentmarker}{\pgfqpoint{-0.027778in}{0.000000in}}{\pgfqpoint{-0.000000in}{0.000000in}}{%
\pgfpathmoveto{\pgfqpoint{-0.000000in}{0.000000in}}%
\pgfpathlineto{\pgfqpoint{-0.027778in}{0.000000in}}%
\pgfusepath{stroke,fill}%
}%
\begin{pgfscope}%
\pgfsys@transformshift{0.800000in}{3.279491in}%
\pgfsys@useobject{currentmarker}{}%
\end{pgfscope}%
\end{pgfscope}%
\begin{pgfscope}%
\pgfsetbuttcap%
\pgfsetroundjoin%
\definecolor{currentfill}{rgb}{0.000000,0.000000,0.000000}%
\pgfsetfillcolor{currentfill}%
\pgfsetlinewidth{0.602250pt}%
\definecolor{currentstroke}{rgb}{0.000000,0.000000,0.000000}%
\pgfsetstrokecolor{currentstroke}%
\pgfsetdash{}{0pt}%
\pgfsys@defobject{currentmarker}{\pgfqpoint{-0.027778in}{0.000000in}}{\pgfqpoint{-0.000000in}{0.000000in}}{%
\pgfpathmoveto{\pgfqpoint{-0.000000in}{0.000000in}}%
\pgfpathlineto{\pgfqpoint{-0.027778in}{0.000000in}}%
\pgfusepath{stroke,fill}%
}%
\begin{pgfscope}%
\pgfsys@transformshift{0.800000in}{3.359835in}%
\pgfsys@useobject{currentmarker}{}%
\end{pgfscope}%
\end{pgfscope}%
\begin{pgfscope}%
\pgfsetbuttcap%
\pgfsetroundjoin%
\definecolor{currentfill}{rgb}{0.000000,0.000000,0.000000}%
\pgfsetfillcolor{currentfill}%
\pgfsetlinewidth{0.602250pt}%
\definecolor{currentstroke}{rgb}{0.000000,0.000000,0.000000}%
\pgfsetstrokecolor{currentstroke}%
\pgfsetdash{}{0pt}%
\pgfsys@defobject{currentmarker}{\pgfqpoint{-0.027778in}{0.000000in}}{\pgfqpoint{-0.000000in}{0.000000in}}{%
\pgfpathmoveto{\pgfqpoint{-0.000000in}{0.000000in}}%
\pgfpathlineto{\pgfqpoint{-0.027778in}{0.000000in}}%
\pgfusepath{stroke,fill}%
}%
\begin{pgfscope}%
\pgfsys@transformshift{0.800000in}{3.422155in}%
\pgfsys@useobject{currentmarker}{}%
\end{pgfscope}%
\end{pgfscope}%
\begin{pgfscope}%
\pgfsetbuttcap%
\pgfsetroundjoin%
\definecolor{currentfill}{rgb}{0.000000,0.000000,0.000000}%
\pgfsetfillcolor{currentfill}%
\pgfsetlinewidth{0.602250pt}%
\definecolor{currentstroke}{rgb}{0.000000,0.000000,0.000000}%
\pgfsetstrokecolor{currentstroke}%
\pgfsetdash{}{0pt}%
\pgfsys@defobject{currentmarker}{\pgfqpoint{-0.027778in}{0.000000in}}{\pgfqpoint{-0.000000in}{0.000000in}}{%
\pgfpathmoveto{\pgfqpoint{-0.000000in}{0.000000in}}%
\pgfpathlineto{\pgfqpoint{-0.027778in}{0.000000in}}%
\pgfusepath{stroke,fill}%
}%
\begin{pgfscope}%
\pgfsys@transformshift{0.800000in}{3.473073in}%
\pgfsys@useobject{currentmarker}{}%
\end{pgfscope}%
\end{pgfscope}%
\begin{pgfscope}%
\pgfsetbuttcap%
\pgfsetroundjoin%
\definecolor{currentfill}{rgb}{0.000000,0.000000,0.000000}%
\pgfsetfillcolor{currentfill}%
\pgfsetlinewidth{0.602250pt}%
\definecolor{currentstroke}{rgb}{0.000000,0.000000,0.000000}%
\pgfsetstrokecolor{currentstroke}%
\pgfsetdash{}{0pt}%
\pgfsys@defobject{currentmarker}{\pgfqpoint{-0.027778in}{0.000000in}}{\pgfqpoint{-0.000000in}{0.000000in}}{%
\pgfpathmoveto{\pgfqpoint{-0.000000in}{0.000000in}}%
\pgfpathlineto{\pgfqpoint{-0.027778in}{0.000000in}}%
\pgfusepath{stroke,fill}%
}%
\begin{pgfscope}%
\pgfsys@transformshift{0.800000in}{3.516125in}%
\pgfsys@useobject{currentmarker}{}%
\end{pgfscope}%
\end{pgfscope}%
\begin{pgfscope}%
\pgfsetbuttcap%
\pgfsetroundjoin%
\definecolor{currentfill}{rgb}{0.000000,0.000000,0.000000}%
\pgfsetfillcolor{currentfill}%
\pgfsetlinewidth{0.602250pt}%
\definecolor{currentstroke}{rgb}{0.000000,0.000000,0.000000}%
\pgfsetstrokecolor{currentstroke}%
\pgfsetdash{}{0pt}%
\pgfsys@defobject{currentmarker}{\pgfqpoint{-0.027778in}{0.000000in}}{\pgfqpoint{-0.000000in}{0.000000in}}{%
\pgfpathmoveto{\pgfqpoint{-0.000000in}{0.000000in}}%
\pgfpathlineto{\pgfqpoint{-0.027778in}{0.000000in}}%
\pgfusepath{stroke,fill}%
}%
\begin{pgfscope}%
\pgfsys@transformshift{0.800000in}{3.553417in}%
\pgfsys@useobject{currentmarker}{}%
\end{pgfscope}%
\end{pgfscope}%
\begin{pgfscope}%
\pgfsetbuttcap%
\pgfsetroundjoin%
\definecolor{currentfill}{rgb}{0.000000,0.000000,0.000000}%
\pgfsetfillcolor{currentfill}%
\pgfsetlinewidth{0.602250pt}%
\definecolor{currentstroke}{rgb}{0.000000,0.000000,0.000000}%
\pgfsetstrokecolor{currentstroke}%
\pgfsetdash{}{0pt}%
\pgfsys@defobject{currentmarker}{\pgfqpoint{-0.027778in}{0.000000in}}{\pgfqpoint{-0.000000in}{0.000000in}}{%
\pgfpathmoveto{\pgfqpoint{-0.000000in}{0.000000in}}%
\pgfpathlineto{\pgfqpoint{-0.027778in}{0.000000in}}%
\pgfusepath{stroke,fill}%
}%
\begin{pgfscope}%
\pgfsys@transformshift{0.800000in}{3.586312in}%
\pgfsys@useobject{currentmarker}{}%
\end{pgfscope}%
\end{pgfscope}%
\begin{pgfscope}%
\pgfsetbuttcap%
\pgfsetroundjoin%
\definecolor{currentfill}{rgb}{0.000000,0.000000,0.000000}%
\pgfsetfillcolor{currentfill}%
\pgfsetlinewidth{0.602250pt}%
\definecolor{currentstroke}{rgb}{0.000000,0.000000,0.000000}%
\pgfsetstrokecolor{currentstroke}%
\pgfsetdash{}{0pt}%
\pgfsys@defobject{currentmarker}{\pgfqpoint{-0.027778in}{0.000000in}}{\pgfqpoint{-0.000000in}{0.000000in}}{%
\pgfpathmoveto{\pgfqpoint{-0.000000in}{0.000000in}}%
\pgfpathlineto{\pgfqpoint{-0.027778in}{0.000000in}}%
\pgfusepath{stroke,fill}%
}%
\begin{pgfscope}%
\pgfsys@transformshift{0.800000in}{3.809319in}%
\pgfsys@useobject{currentmarker}{}%
\end{pgfscope}%
\end{pgfscope}%
\begin{pgfscope}%
\pgfsetbuttcap%
\pgfsetroundjoin%
\definecolor{currentfill}{rgb}{0.000000,0.000000,0.000000}%
\pgfsetfillcolor{currentfill}%
\pgfsetlinewidth{0.602250pt}%
\definecolor{currentstroke}{rgb}{0.000000,0.000000,0.000000}%
\pgfsetstrokecolor{currentstroke}%
\pgfsetdash{}{0pt}%
\pgfsys@defobject{currentmarker}{\pgfqpoint{-0.027778in}{0.000000in}}{\pgfqpoint{-0.000000in}{0.000000in}}{%
\pgfpathmoveto{\pgfqpoint{-0.000000in}{0.000000in}}%
\pgfpathlineto{\pgfqpoint{-0.027778in}{0.000000in}}%
\pgfusepath{stroke,fill}%
}%
\begin{pgfscope}%
\pgfsys@transformshift{0.800000in}{3.922557in}%
\pgfsys@useobject{currentmarker}{}%
\end{pgfscope}%
\end{pgfscope}%
\begin{pgfscope}%
\pgfsetbuttcap%
\pgfsetroundjoin%
\definecolor{currentfill}{rgb}{0.000000,0.000000,0.000000}%
\pgfsetfillcolor{currentfill}%
\pgfsetlinewidth{0.602250pt}%
\definecolor{currentstroke}{rgb}{0.000000,0.000000,0.000000}%
\pgfsetstrokecolor{currentstroke}%
\pgfsetdash{}{0pt}%
\pgfsys@defobject{currentmarker}{\pgfqpoint{-0.027778in}{0.000000in}}{\pgfqpoint{-0.000000in}{0.000000in}}{%
\pgfpathmoveto{\pgfqpoint{-0.000000in}{0.000000in}}%
\pgfpathlineto{\pgfqpoint{-0.027778in}{0.000000in}}%
\pgfusepath{stroke,fill}%
}%
\begin{pgfscope}%
\pgfsys@transformshift{0.800000in}{4.002901in}%
\pgfsys@useobject{currentmarker}{}%
\end{pgfscope}%
\end{pgfscope}%
\begin{pgfscope}%
\pgfsetbuttcap%
\pgfsetroundjoin%
\definecolor{currentfill}{rgb}{0.000000,0.000000,0.000000}%
\pgfsetfillcolor{currentfill}%
\pgfsetlinewidth{0.602250pt}%
\definecolor{currentstroke}{rgb}{0.000000,0.000000,0.000000}%
\pgfsetstrokecolor{currentstroke}%
\pgfsetdash{}{0pt}%
\pgfsys@defobject{currentmarker}{\pgfqpoint{-0.027778in}{0.000000in}}{\pgfqpoint{-0.000000in}{0.000000in}}{%
\pgfpathmoveto{\pgfqpoint{-0.000000in}{0.000000in}}%
\pgfpathlineto{\pgfqpoint{-0.027778in}{0.000000in}}%
\pgfusepath{stroke,fill}%
}%
\begin{pgfscope}%
\pgfsys@transformshift{0.800000in}{4.065221in}%
\pgfsys@useobject{currentmarker}{}%
\end{pgfscope}%
\end{pgfscope}%
\begin{pgfscope}%
\pgfsetbuttcap%
\pgfsetroundjoin%
\definecolor{currentfill}{rgb}{0.000000,0.000000,0.000000}%
\pgfsetfillcolor{currentfill}%
\pgfsetlinewidth{0.602250pt}%
\definecolor{currentstroke}{rgb}{0.000000,0.000000,0.000000}%
\pgfsetstrokecolor{currentstroke}%
\pgfsetdash{}{0pt}%
\pgfsys@defobject{currentmarker}{\pgfqpoint{-0.027778in}{0.000000in}}{\pgfqpoint{-0.000000in}{0.000000in}}{%
\pgfpathmoveto{\pgfqpoint{-0.000000in}{0.000000in}}%
\pgfpathlineto{\pgfqpoint{-0.027778in}{0.000000in}}%
\pgfusepath{stroke,fill}%
}%
\begin{pgfscope}%
\pgfsys@transformshift{0.800000in}{4.116139in}%
\pgfsys@useobject{currentmarker}{}%
\end{pgfscope}%
\end{pgfscope}%
\begin{pgfscope}%
\pgfsetbuttcap%
\pgfsetroundjoin%
\definecolor{currentfill}{rgb}{0.000000,0.000000,0.000000}%
\pgfsetfillcolor{currentfill}%
\pgfsetlinewidth{0.602250pt}%
\definecolor{currentstroke}{rgb}{0.000000,0.000000,0.000000}%
\pgfsetstrokecolor{currentstroke}%
\pgfsetdash{}{0pt}%
\pgfsys@defobject{currentmarker}{\pgfqpoint{-0.027778in}{0.000000in}}{\pgfqpoint{-0.000000in}{0.000000in}}{%
\pgfpathmoveto{\pgfqpoint{-0.000000in}{0.000000in}}%
\pgfpathlineto{\pgfqpoint{-0.027778in}{0.000000in}}%
\pgfusepath{stroke,fill}%
}%
\begin{pgfscope}%
\pgfsys@transformshift{0.800000in}{4.159191in}%
\pgfsys@useobject{currentmarker}{}%
\end{pgfscope}%
\end{pgfscope}%
\begin{pgfscope}%
\pgfsetbuttcap%
\pgfsetroundjoin%
\definecolor{currentfill}{rgb}{0.000000,0.000000,0.000000}%
\pgfsetfillcolor{currentfill}%
\pgfsetlinewidth{0.602250pt}%
\definecolor{currentstroke}{rgb}{0.000000,0.000000,0.000000}%
\pgfsetstrokecolor{currentstroke}%
\pgfsetdash{}{0pt}%
\pgfsys@defobject{currentmarker}{\pgfqpoint{-0.027778in}{0.000000in}}{\pgfqpoint{-0.000000in}{0.000000in}}{%
\pgfpathmoveto{\pgfqpoint{-0.000000in}{0.000000in}}%
\pgfpathlineto{\pgfqpoint{-0.027778in}{0.000000in}}%
\pgfusepath{stroke,fill}%
}%
\begin{pgfscope}%
\pgfsys@transformshift{0.800000in}{4.196483in}%
\pgfsys@useobject{currentmarker}{}%
\end{pgfscope}%
\end{pgfscope}%
\begin{pgfscope}%
\definecolor{textcolor}{rgb}{0.000000,0.000000,0.000000}%
\pgfsetstrokecolor{textcolor}%
\pgfsetfillcolor{textcolor}%
\pgftext[x=0.446026in,y=2.376000in,,bottom,rotate=90.000000]{\color{textcolor}\rmfamily\fontsize{10.000000}{12.000000}\selectfont Time (ns)}%
\end{pgfscope}%
\begin{pgfscope}%
\pgfpathrectangle{\pgfqpoint{0.800000in}{0.528000in}}{\pgfqpoint{4.960000in}{3.696000in}}%
\pgfusepath{clip}%
\pgfsetrectcap%
\pgfsetroundjoin%
\pgfsetlinewidth{1.505625pt}%
\definecolor{currentstroke}{rgb}{0.121569,0.466667,0.705882}%
\pgfsetstrokecolor{currentstroke}%
\pgfsetdash{}{0pt}%
\pgfpathmoveto{\pgfqpoint{1.025455in}{0.840495in}}%
\pgfpathlineto{\pgfqpoint{1.071466in}{1.186506in}}%
\pgfpathlineto{\pgfqpoint{1.117477in}{1.413828in}}%
\pgfpathlineto{\pgfqpoint{1.163488in}{1.587903in}}%
\pgfpathlineto{\pgfqpoint{1.209499in}{1.711742in}}%
\pgfpathlineto{\pgfqpoint{1.255510in}{1.821825in}}%
\pgfpathlineto{\pgfqpoint{1.301521in}{1.912899in}}%
\pgfpathlineto{\pgfqpoint{1.347532in}{1.993169in}}%
\pgfpathlineto{\pgfqpoint{1.393544in}{2.064635in}}%
\pgfpathlineto{\pgfqpoint{1.439555in}{2.139054in}}%
\pgfpathlineto{\pgfqpoint{1.485566in}{2.195150in}}%
\pgfpathlineto{\pgfqpoint{1.531577in}{2.253176in}}%
\pgfpathlineto{\pgfqpoint{1.577588in}{2.308884in}}%
\pgfpathlineto{\pgfqpoint{1.623599in}{2.362827in}}%
\pgfpathlineto{\pgfqpoint{1.669610in}{2.442662in}}%
\pgfpathlineto{\pgfqpoint{1.715622in}{2.454860in}}%
\pgfpathlineto{\pgfqpoint{1.761633in}{2.492645in}}%
\pgfpathlineto{\pgfqpoint{1.807644in}{2.564814in}}%
\pgfpathlineto{\pgfqpoint{1.853655in}{2.685054in}}%
\pgfpathlineto{\pgfqpoint{1.899666in}{2.610805in}}%
\pgfpathlineto{\pgfqpoint{1.945677in}{2.643548in}}%
\pgfpathlineto{\pgfqpoint{1.991688in}{2.679207in}}%
\pgfpathlineto{\pgfqpoint{2.037699in}{2.716402in}}%
\pgfpathlineto{\pgfqpoint{2.083711in}{2.745072in}}%
\pgfpathlineto{\pgfqpoint{2.129722in}{2.777442in}}%
\pgfpathlineto{\pgfqpoint{2.175733in}{2.808501in}}%
\pgfpathlineto{\pgfqpoint{2.221744in}{2.852725in}}%
\pgfpathlineto{\pgfqpoint{2.267755in}{2.875752in}}%
\pgfpathlineto{\pgfqpoint{2.313766in}{2.899186in}}%
\pgfpathlineto{\pgfqpoint{2.359777in}{2.930526in}}%
\pgfpathlineto{\pgfqpoint{2.405788in}{2.951260in}}%
\pgfpathlineto{\pgfqpoint{2.451800in}{2.979613in}}%
\pgfpathlineto{\pgfqpoint{2.497811in}{3.021133in}}%
\pgfpathlineto{\pgfqpoint{2.543822in}{3.040158in}}%
\pgfpathlineto{\pgfqpoint{2.589833in}{3.084466in}}%
\pgfpathlineto{\pgfqpoint{2.635844in}{3.095515in}}%
\pgfpathlineto{\pgfqpoint{2.681855in}{3.122559in}}%
\pgfpathlineto{\pgfqpoint{2.727866in}{3.155584in}}%
\pgfpathlineto{\pgfqpoint{2.773878in}{3.187907in}}%
\pgfpathlineto{\pgfqpoint{2.819889in}{3.200321in}}%
\pgfpathlineto{\pgfqpoint{2.865900in}{3.228745in}}%
\pgfpathlineto{\pgfqpoint{2.911911in}{3.244795in}}%
\pgfpathlineto{\pgfqpoint{2.957922in}{3.272415in}}%
\pgfpathlineto{\pgfqpoint{3.003933in}{3.291138in}}%
\pgfpathlineto{\pgfqpoint{3.049944in}{3.310493in}}%
\pgfpathlineto{\pgfqpoint{3.095955in}{3.320874in}}%
\pgfpathlineto{\pgfqpoint{3.141967in}{3.343738in}}%
\pgfpathlineto{\pgfqpoint{3.187978in}{3.373861in}}%
\pgfpathlineto{\pgfqpoint{3.233989in}{3.378510in}}%
\pgfpathlineto{\pgfqpoint{3.280000in}{3.411248in}}%
\pgfpathlineto{\pgfqpoint{3.326011in}{3.414188in}}%
\pgfpathlineto{\pgfqpoint{3.372022in}{3.440988in}}%
\pgfpathlineto{\pgfqpoint{3.418033in}{3.461031in}}%
\pgfpathlineto{\pgfqpoint{3.464045in}{3.475000in}}%
\pgfpathlineto{\pgfqpoint{3.510056in}{3.500982in}}%
\pgfpathlineto{\pgfqpoint{3.556067in}{3.501630in}}%
\pgfpathlineto{\pgfqpoint{3.602078in}{3.527066in}}%
\pgfpathlineto{\pgfqpoint{3.648089in}{3.546836in}}%
\pgfpathlineto{\pgfqpoint{3.694100in}{3.569430in}}%
\pgfpathlineto{\pgfqpoint{3.740111in}{3.579415in}}%
\pgfpathlineto{\pgfqpoint{3.786122in}{3.588911in}}%
\pgfpathlineto{\pgfqpoint{3.832134in}{3.606734in}}%
\pgfpathlineto{\pgfqpoint{3.878145in}{3.618875in}}%
\pgfpathlineto{\pgfqpoint{3.924156in}{3.641637in}}%
\pgfpathlineto{\pgfqpoint{3.970167in}{3.712666in}}%
\pgfpathlineto{\pgfqpoint{4.016178in}{3.678993in}}%
\pgfpathlineto{\pgfqpoint{4.062189in}{3.690343in}}%
\pgfpathlineto{\pgfqpoint{4.108200in}{3.718237in}}%
\pgfpathlineto{\pgfqpoint{4.154212in}{3.763732in}}%
\pgfpathlineto{\pgfqpoint{4.200223in}{3.777999in}}%
\pgfpathlineto{\pgfqpoint{4.246234in}{3.786144in}}%
\pgfpathlineto{\pgfqpoint{4.292245in}{3.869425in}}%
\pgfpathlineto{\pgfqpoint{4.338256in}{3.767099in}}%
\pgfpathlineto{\pgfqpoint{4.384267in}{3.786082in}}%
\pgfpathlineto{\pgfqpoint{4.430278in}{3.787049in}}%
\pgfpathlineto{\pgfqpoint{4.476289in}{3.792710in}}%
\pgfpathlineto{\pgfqpoint{4.522301in}{3.810946in}}%
\pgfpathlineto{\pgfqpoint{4.568312in}{3.822215in}}%
\pgfpathlineto{\pgfqpoint{4.614323in}{3.824722in}}%
\pgfpathlineto{\pgfqpoint{4.660334in}{3.846824in}}%
\pgfpathlineto{\pgfqpoint{4.706345in}{3.851458in}}%
\pgfpathlineto{\pgfqpoint{4.752356in}{3.857753in}}%
\pgfpathlineto{\pgfqpoint{4.798367in}{3.890656in}}%
\pgfpathlineto{\pgfqpoint{4.844378in}{3.896026in}}%
\pgfpathlineto{\pgfqpoint{4.890390in}{3.899995in}}%
\pgfpathlineto{\pgfqpoint{4.936401in}{3.896447in}}%
\pgfpathlineto{\pgfqpoint{4.982412in}{3.916331in}}%
\pgfpathlineto{\pgfqpoint{5.028423in}{3.927110in}}%
\pgfpathlineto{\pgfqpoint{5.074434in}{3.920147in}}%
\pgfpathlineto{\pgfqpoint{5.120445in}{3.949678in}}%
\pgfpathlineto{\pgfqpoint{5.166456in}{3.944727in}}%
\pgfpathlineto{\pgfqpoint{5.212468in}{3.950763in}}%
\pgfpathlineto{\pgfqpoint{5.258479in}{3.963114in}}%
\pgfpathlineto{\pgfqpoint{5.304490in}{4.006397in}}%
\pgfpathlineto{\pgfqpoint{5.350501in}{3.989107in}}%
\pgfpathlineto{\pgfqpoint{5.396512in}{3.994754in}}%
\pgfpathlineto{\pgfqpoint{5.442523in}{4.008333in}}%
\pgfpathlineto{\pgfqpoint{5.488534in}{4.026390in}}%
\pgfpathlineto{\pgfqpoint{5.534545in}{4.056000in}}%
\pgfusepath{stroke}%
\end{pgfscope}%
\begin{pgfscope}%
\pgfpathrectangle{\pgfqpoint{0.800000in}{0.528000in}}{\pgfqpoint{4.960000in}{3.696000in}}%
\pgfusepath{clip}%
\pgfsetrectcap%
\pgfsetroundjoin%
\pgfsetlinewidth{1.505625pt}%
\definecolor{currentstroke}{rgb}{1.000000,0.498039,0.054902}%
\pgfsetstrokecolor{currentstroke}%
\pgfsetdash{}{0pt}%
\pgfpathmoveto{\pgfqpoint{1.025455in}{0.696000in}}%
\pgfpathlineto{\pgfqpoint{1.071466in}{0.858979in}}%
\pgfpathlineto{\pgfqpoint{1.117477in}{0.987721in}}%
\pgfpathlineto{\pgfqpoint{1.163488in}{1.132406in}}%
\pgfpathlineto{\pgfqpoint{1.209499in}{1.228845in}}%
\pgfpathlineto{\pgfqpoint{1.255510in}{1.299128in}}%
\pgfpathlineto{\pgfqpoint{1.301521in}{1.356663in}}%
\pgfpathlineto{\pgfqpoint{1.347532in}{1.409200in}}%
\pgfpathlineto{\pgfqpoint{1.393544in}{1.451170in}}%
\pgfpathlineto{\pgfqpoint{1.439555in}{1.491895in}}%
\pgfpathlineto{\pgfqpoint{1.485566in}{1.521501in}}%
\pgfpathlineto{\pgfqpoint{1.531577in}{1.557309in}}%
\pgfpathlineto{\pgfqpoint{1.577588in}{1.580410in}}%
\pgfpathlineto{\pgfqpoint{1.623599in}{1.610633in}}%
\pgfpathlineto{\pgfqpoint{1.669610in}{1.632596in}}%
\pgfpathlineto{\pgfqpoint{1.715622in}{1.653971in}}%
\pgfpathlineto{\pgfqpoint{1.761633in}{1.674034in}}%
\pgfpathlineto{\pgfqpoint{1.807644in}{1.753826in}}%
\pgfpathlineto{\pgfqpoint{1.853655in}{1.712817in}}%
\pgfpathlineto{\pgfqpoint{1.899666in}{1.732707in}}%
\pgfpathlineto{\pgfqpoint{1.945677in}{1.746892in}}%
\pgfpathlineto{\pgfqpoint{1.991688in}{1.762253in}}%
\pgfpathlineto{\pgfqpoint{2.037699in}{1.778120in}}%
\pgfpathlineto{\pgfqpoint{2.083711in}{1.794459in}}%
\pgfpathlineto{\pgfqpoint{2.129722in}{1.814534in}}%
\pgfpathlineto{\pgfqpoint{2.175733in}{1.823367in}}%
\pgfpathlineto{\pgfqpoint{2.221744in}{1.848732in}}%
\pgfpathlineto{\pgfqpoint{2.267755in}{1.859187in}}%
\pgfpathlineto{\pgfqpoint{2.313766in}{1.875965in}}%
\pgfpathlineto{\pgfqpoint{2.359777in}{1.892631in}}%
\pgfpathlineto{\pgfqpoint{2.405788in}{1.904686in}}%
\pgfpathlineto{\pgfqpoint{2.451800in}{1.921080in}}%
\pgfpathlineto{\pgfqpoint{2.497811in}{1.935501in}}%
\pgfpathlineto{\pgfqpoint{2.543822in}{1.944512in}}%
\pgfpathlineto{\pgfqpoint{2.589833in}{1.959618in}}%
\pgfpathlineto{\pgfqpoint{2.635844in}{1.964181in}}%
\pgfpathlineto{\pgfqpoint{2.681855in}{1.979566in}}%
\pgfpathlineto{\pgfqpoint{2.727866in}{1.990013in}}%
\pgfpathlineto{\pgfqpoint{2.773878in}{1.995421in}}%
\pgfpathlineto{\pgfqpoint{2.819889in}{2.008622in}}%
\pgfpathlineto{\pgfqpoint{2.865900in}{2.018530in}}%
\pgfpathlineto{\pgfqpoint{2.911911in}{2.025915in}}%
\pgfpathlineto{\pgfqpoint{2.957922in}{2.038405in}}%
\pgfpathlineto{\pgfqpoint{3.003933in}{2.045568in}}%
\pgfpathlineto{\pgfqpoint{3.049944in}{2.053024in}}%
\pgfpathlineto{\pgfqpoint{3.095955in}{2.063828in}}%
\pgfpathlineto{\pgfqpoint{3.141967in}{2.072504in}}%
\pgfpathlineto{\pgfqpoint{3.187978in}{2.075813in}}%
\pgfpathlineto{\pgfqpoint{3.233989in}{2.083828in}}%
\pgfpathlineto{\pgfqpoint{3.280000in}{2.094160in}}%
\pgfpathlineto{\pgfqpoint{3.326011in}{2.098817in}}%
\pgfpathlineto{\pgfqpoint{3.372022in}{2.110077in}}%
\pgfpathlineto{\pgfqpoint{3.418033in}{2.114011in}}%
\pgfpathlineto{\pgfqpoint{3.464045in}{2.126144in}}%
\pgfpathlineto{\pgfqpoint{3.510056in}{2.206788in}}%
\pgfpathlineto{\pgfqpoint{3.556067in}{2.142371in}}%
\pgfpathlineto{\pgfqpoint{3.602078in}{2.142160in}}%
\pgfpathlineto{\pgfqpoint{3.648089in}{2.153358in}}%
\pgfpathlineto{\pgfqpoint{3.694100in}{2.158068in}}%
\pgfpathlineto{\pgfqpoint{3.740111in}{2.171032in}}%
\pgfpathlineto{\pgfqpoint{3.786122in}{2.175465in}}%
\pgfpathlineto{\pgfqpoint{3.832134in}{2.173486in}}%
\pgfpathlineto{\pgfqpoint{3.878145in}{2.176688in}}%
\pgfpathlineto{\pgfqpoint{3.924156in}{2.182930in}}%
\pgfpathlineto{\pgfqpoint{3.970167in}{2.188862in}}%
\pgfpathlineto{\pgfqpoint{4.016178in}{2.193646in}}%
\pgfpathlineto{\pgfqpoint{4.062189in}{2.199032in}}%
\pgfpathlineto{\pgfqpoint{4.108200in}{2.203770in}}%
\pgfpathlineto{\pgfqpoint{4.154212in}{2.209507in}}%
\pgfpathlineto{\pgfqpoint{4.200223in}{2.212869in}}%
\pgfpathlineto{\pgfqpoint{4.246234in}{2.218968in}}%
\pgfpathlineto{\pgfqpoint{4.292245in}{2.223823in}}%
\pgfpathlineto{\pgfqpoint{4.338256in}{2.227287in}}%
\pgfpathlineto{\pgfqpoint{4.384267in}{2.238714in}}%
\pgfpathlineto{\pgfqpoint{4.430278in}{2.262547in}}%
\pgfpathlineto{\pgfqpoint{4.476289in}{2.413095in}}%
\pgfpathlineto{\pgfqpoint{4.522301in}{2.252441in}}%
\pgfpathlineto{\pgfqpoint{4.568312in}{2.258615in}}%
\pgfpathlineto{\pgfqpoint{4.614323in}{2.254849in}}%
\pgfpathlineto{\pgfqpoint{4.660334in}{2.394590in}}%
\pgfpathlineto{\pgfqpoint{4.706345in}{2.313481in}}%
\pgfpathlineto{\pgfqpoint{4.752356in}{2.269183in}}%
\pgfpathlineto{\pgfqpoint{4.798367in}{2.274171in}}%
\pgfpathlineto{\pgfqpoint{4.844378in}{2.277830in}}%
\pgfpathlineto{\pgfqpoint{4.890390in}{2.283535in}}%
\pgfpathlineto{\pgfqpoint{4.936401in}{2.285586in}}%
\pgfpathlineto{\pgfqpoint{4.982412in}{2.423676in}}%
\pgfpathlineto{\pgfqpoint{5.028423in}{2.295244in}}%
\pgfpathlineto{\pgfqpoint{5.074434in}{2.296934in}}%
\pgfpathlineto{\pgfqpoint{5.120445in}{2.302145in}}%
\pgfpathlineto{\pgfqpoint{5.166456in}{2.303314in}}%
\pgfpathlineto{\pgfqpoint{5.212468in}{2.308859in}}%
\pgfpathlineto{\pgfqpoint{5.258479in}{2.310326in}}%
\pgfpathlineto{\pgfqpoint{5.304490in}{2.316100in}}%
\pgfpathlineto{\pgfqpoint{5.350501in}{2.317370in}}%
\pgfpathlineto{\pgfqpoint{5.396512in}{2.322097in}}%
\pgfpathlineto{\pgfqpoint{5.442523in}{2.327008in}}%
\pgfpathlineto{\pgfqpoint{5.488534in}{2.328264in}}%
\pgfpathlineto{\pgfqpoint{5.534545in}{2.331517in}}%
\pgfusepath{stroke}%
\end{pgfscope}%
\begin{pgfscope}%
\pgfsetrectcap%
\pgfsetmiterjoin%
\pgfsetlinewidth{0.803000pt}%
\definecolor{currentstroke}{rgb}{0.000000,0.000000,0.000000}%
\pgfsetstrokecolor{currentstroke}%
\pgfsetdash{}{0pt}%
\pgfpathmoveto{\pgfqpoint{0.800000in}{0.528000in}}%
\pgfpathlineto{\pgfqpoint{0.800000in}{4.224000in}}%
\pgfusepath{stroke}%
\end{pgfscope}%
\begin{pgfscope}%
\pgfsetrectcap%
\pgfsetmiterjoin%
\pgfsetlinewidth{0.803000pt}%
\definecolor{currentstroke}{rgb}{0.000000,0.000000,0.000000}%
\pgfsetstrokecolor{currentstroke}%
\pgfsetdash{}{0pt}%
\pgfpathmoveto{\pgfqpoint{5.760000in}{0.528000in}}%
\pgfpathlineto{\pgfqpoint{5.760000in}{4.224000in}}%
\pgfusepath{stroke}%
\end{pgfscope}%
\begin{pgfscope}%
\pgfsetrectcap%
\pgfsetmiterjoin%
\pgfsetlinewidth{0.803000pt}%
\definecolor{currentstroke}{rgb}{0.000000,0.000000,0.000000}%
\pgfsetstrokecolor{currentstroke}%
\pgfsetdash{}{0pt}%
\pgfpathmoveto{\pgfqpoint{0.800000in}{0.528000in}}%
\pgfpathlineto{\pgfqpoint{5.760000in}{0.528000in}}%
\pgfusepath{stroke}%
\end{pgfscope}%
\begin{pgfscope}%
\pgfsetrectcap%
\pgfsetmiterjoin%
\pgfsetlinewidth{0.803000pt}%
\definecolor{currentstroke}{rgb}{0.000000,0.000000,0.000000}%
\pgfsetstrokecolor{currentstroke}%
\pgfsetdash{}{0pt}%
\pgfpathmoveto{\pgfqpoint{0.800000in}{4.224000in}}%
\pgfpathlineto{\pgfqpoint{5.760000in}{4.224000in}}%
\pgfusepath{stroke}%
\end{pgfscope}%
\begin{pgfscope}%
\pgfsetbuttcap%
\pgfsetmiterjoin%
\definecolor{currentfill}{rgb}{1.000000,1.000000,1.000000}%
\pgfsetfillcolor{currentfill}%
\pgfsetfillopacity{0.800000}%
\pgfsetlinewidth{1.003750pt}%
\definecolor{currentstroke}{rgb}{0.800000,0.800000,0.800000}%
\pgfsetstrokecolor{currentstroke}%
\pgfsetstrokeopacity{0.800000}%
\pgfsetdash{}{0pt}%
\pgfpathmoveto{\pgfqpoint{0.897222in}{3.725543in}}%
\pgfpathlineto{\pgfqpoint{1.975542in}{3.725543in}}%
\pgfpathquadraticcurveto{\pgfqpoint{2.003319in}{3.725543in}}{\pgfqpoint{2.003319in}{3.753321in}}%
\pgfpathlineto{\pgfqpoint{2.003319in}{4.126778in}}%
\pgfpathquadraticcurveto{\pgfqpoint{2.003319in}{4.154556in}}{\pgfqpoint{1.975542in}{4.154556in}}%
\pgfpathlineto{\pgfqpoint{0.897222in}{4.154556in}}%
\pgfpathquadraticcurveto{\pgfqpoint{0.869444in}{4.154556in}}{\pgfqpoint{0.869444in}{4.126778in}}%
\pgfpathlineto{\pgfqpoint{0.869444in}{3.753321in}}%
\pgfpathquadraticcurveto{\pgfqpoint{0.869444in}{3.725543in}}{\pgfqpoint{0.897222in}{3.725543in}}%
\pgfpathlineto{\pgfqpoint{0.897222in}{3.725543in}}%
\pgfpathclose%
\pgfusepath{stroke,fill}%
\end{pgfscope}%
\begin{pgfscope}%
\pgfsetrectcap%
\pgfsetroundjoin%
\pgfsetlinewidth{1.505625pt}%
\definecolor{currentstroke}{rgb}{0.121569,0.466667,0.705882}%
\pgfsetstrokecolor{currentstroke}%
\pgfsetdash{}{0pt}%
\pgfpathmoveto{\pgfqpoint{0.925000in}{4.050389in}}%
\pgfpathlineto{\pgfqpoint{1.063889in}{4.050389in}}%
\pgfpathlineto{\pgfqpoint{1.202778in}{4.050389in}}%
\pgfusepath{stroke}%
\end{pgfscope}%
\begin{pgfscope}%
\definecolor{textcolor}{rgb}{0.000000,0.000000,0.000000}%
\pgfsetstrokecolor{textcolor}%
\pgfsetfillcolor{textcolor}%
\pgftext[x=1.313889in,y=4.001778in,left,base]{\color{textcolor}\rmfamily\fontsize{10.000000}{12.000000}\selectfont quicksort}%
\end{pgfscope}%
\begin{pgfscope}%
\pgfsetrectcap%
\pgfsetroundjoin%
\pgfsetlinewidth{1.505625pt}%
\definecolor{currentstroke}{rgb}{1.000000,0.498039,0.054902}%
\pgfsetstrokecolor{currentstroke}%
\pgfsetdash{}{0pt}%
\pgfpathmoveto{\pgfqpoint{0.925000in}{3.856716in}}%
\pgfpathlineto{\pgfqpoint{1.063889in}{3.856716in}}%
\pgfpathlineto{\pgfqpoint{1.202778in}{3.856716in}}%
\pgfusepath{stroke}%
\end{pgfscope}%
\begin{pgfscope}%
\definecolor{textcolor}{rgb}{0.000000,0.000000,0.000000}%
\pgfsetstrokecolor{textcolor}%
\pgfsetfillcolor{textcolor}%
\pgftext[x=1.313889in,y=3.808105in,left,base]{\color{textcolor}\rmfamily\fontsize{10.000000}{12.000000}\selectfont bquicksort}%
\end{pgfscope}%
\end{pgfpicture}%
\makeatother%
\endgroup%

\subsubsection{Memory}
%% Creator: Matplotlib, PGF backend
%%
%% To include the figure in your LaTeX document, write
%%   \input{<filename>.pgf}
%%
%% Make sure the required packages are loaded in your preamble
%%   \usepackage{pgf}
%%
%% Also ensure that all the required font packages are loaded; for instance,
%% the lmodern package is sometimes necessary when using math font.
%%   \usepackage{lmodern}
%%
%% Figures using additional raster images can only be included by \input if
%% they are in the same directory as the main LaTeX file. For loading figures
%% from other directories you can use the `import` package
%%   \usepackage{import}
%%
%% and then include the figures with
%%   \import{<path to file>}{<filename>.pgf}
%%
%% Matplotlib used the following preamble
%%   
%%   \makeatletter\@ifpackageloaded{underscore}{}{\usepackage[strings]{underscore}}\makeatother
%%
\begingroup%
\makeatletter%
\begin{pgfpicture}%
\pgfpathrectangle{\pgfpointorigin}{\pgfqpoint{6.400000in}{4.800000in}}%
\pgfusepath{use as bounding box, clip}%
\begin{pgfscope}%
\pgfsetbuttcap%
\pgfsetmiterjoin%
\definecolor{currentfill}{rgb}{1.000000,1.000000,1.000000}%
\pgfsetfillcolor{currentfill}%
\pgfsetlinewidth{0.000000pt}%
\definecolor{currentstroke}{rgb}{1.000000,1.000000,1.000000}%
\pgfsetstrokecolor{currentstroke}%
\pgfsetdash{}{0pt}%
\pgfpathmoveto{\pgfqpoint{0.000000in}{0.000000in}}%
\pgfpathlineto{\pgfqpoint{6.400000in}{0.000000in}}%
\pgfpathlineto{\pgfqpoint{6.400000in}{4.800000in}}%
\pgfpathlineto{\pgfqpoint{0.000000in}{4.800000in}}%
\pgfpathlineto{\pgfqpoint{0.000000in}{0.000000in}}%
\pgfpathclose%
\pgfusepath{fill}%
\end{pgfscope}%
\begin{pgfscope}%
\pgfsetbuttcap%
\pgfsetmiterjoin%
\definecolor{currentfill}{rgb}{1.000000,1.000000,1.000000}%
\pgfsetfillcolor{currentfill}%
\pgfsetlinewidth{0.000000pt}%
\definecolor{currentstroke}{rgb}{0.000000,0.000000,0.000000}%
\pgfsetstrokecolor{currentstroke}%
\pgfsetstrokeopacity{0.000000}%
\pgfsetdash{}{0pt}%
\pgfpathmoveto{\pgfqpoint{0.800000in}{0.528000in}}%
\pgfpathlineto{\pgfqpoint{5.760000in}{0.528000in}}%
\pgfpathlineto{\pgfqpoint{5.760000in}{4.224000in}}%
\pgfpathlineto{\pgfqpoint{0.800000in}{4.224000in}}%
\pgfpathlineto{\pgfqpoint{0.800000in}{0.528000in}}%
\pgfpathclose%
\pgfusepath{fill}%
\end{pgfscope}%
\begin{pgfscope}%
\pgfsetbuttcap%
\pgfsetroundjoin%
\definecolor{currentfill}{rgb}{0.000000,0.000000,0.000000}%
\pgfsetfillcolor{currentfill}%
\pgfsetlinewidth{0.803000pt}%
\definecolor{currentstroke}{rgb}{0.000000,0.000000,0.000000}%
\pgfsetstrokecolor{currentstroke}%
\pgfsetdash{}{0pt}%
\pgfsys@defobject{currentmarker}{\pgfqpoint{0.000000in}{-0.048611in}}{\pgfqpoint{0.000000in}{0.000000in}}{%
\pgfpathmoveto{\pgfqpoint{0.000000in}{0.000000in}}%
\pgfpathlineto{\pgfqpoint{0.000000in}{-0.048611in}}%
\pgfusepath{stroke,fill}%
}%
\begin{pgfscope}%
\pgfsys@transformshift{0.979443in}{0.528000in}%
\pgfsys@useobject{currentmarker}{}%
\end{pgfscope}%
\end{pgfscope}%
\begin{pgfscope}%
\definecolor{textcolor}{rgb}{0.000000,0.000000,0.000000}%
\pgfsetstrokecolor{textcolor}%
\pgfsetfillcolor{textcolor}%
\pgftext[x=0.979443in,y=0.430778in,,top]{\color{textcolor}\rmfamily\fontsize{10.000000}{12.000000}\selectfont \(\displaystyle {0}\)}%
\end{pgfscope}%
\begin{pgfscope}%
\pgfsetbuttcap%
\pgfsetroundjoin%
\definecolor{currentfill}{rgb}{0.000000,0.000000,0.000000}%
\pgfsetfillcolor{currentfill}%
\pgfsetlinewidth{0.803000pt}%
\definecolor{currentstroke}{rgb}{0.000000,0.000000,0.000000}%
\pgfsetstrokecolor{currentstroke}%
\pgfsetdash{}{0pt}%
\pgfsys@defobject{currentmarker}{\pgfqpoint{0.000000in}{-0.048611in}}{\pgfqpoint{0.000000in}{0.000000in}}{%
\pgfpathmoveto{\pgfqpoint{0.000000in}{0.000000in}}%
\pgfpathlineto{\pgfqpoint{0.000000in}{-0.048611in}}%
\pgfusepath{stroke,fill}%
}%
\begin{pgfscope}%
\pgfsys@transformshift{1.899666in}{0.528000in}%
\pgfsys@useobject{currentmarker}{}%
\end{pgfscope}%
\end{pgfscope}%
\begin{pgfscope}%
\definecolor{textcolor}{rgb}{0.000000,0.000000,0.000000}%
\pgfsetstrokecolor{textcolor}%
\pgfsetfillcolor{textcolor}%
\pgftext[x=1.899666in,y=0.430778in,,top]{\color{textcolor}\rmfamily\fontsize{10.000000}{12.000000}\selectfont \(\displaystyle {200}\)}%
\end{pgfscope}%
\begin{pgfscope}%
\pgfsetbuttcap%
\pgfsetroundjoin%
\definecolor{currentfill}{rgb}{0.000000,0.000000,0.000000}%
\pgfsetfillcolor{currentfill}%
\pgfsetlinewidth{0.803000pt}%
\definecolor{currentstroke}{rgb}{0.000000,0.000000,0.000000}%
\pgfsetstrokecolor{currentstroke}%
\pgfsetdash{}{0pt}%
\pgfsys@defobject{currentmarker}{\pgfqpoint{0.000000in}{-0.048611in}}{\pgfqpoint{0.000000in}{0.000000in}}{%
\pgfpathmoveto{\pgfqpoint{0.000000in}{0.000000in}}%
\pgfpathlineto{\pgfqpoint{0.000000in}{-0.048611in}}%
\pgfusepath{stroke,fill}%
}%
\begin{pgfscope}%
\pgfsys@transformshift{2.819889in}{0.528000in}%
\pgfsys@useobject{currentmarker}{}%
\end{pgfscope}%
\end{pgfscope}%
\begin{pgfscope}%
\definecolor{textcolor}{rgb}{0.000000,0.000000,0.000000}%
\pgfsetstrokecolor{textcolor}%
\pgfsetfillcolor{textcolor}%
\pgftext[x=2.819889in,y=0.430778in,,top]{\color{textcolor}\rmfamily\fontsize{10.000000}{12.000000}\selectfont \(\displaystyle {400}\)}%
\end{pgfscope}%
\begin{pgfscope}%
\pgfsetbuttcap%
\pgfsetroundjoin%
\definecolor{currentfill}{rgb}{0.000000,0.000000,0.000000}%
\pgfsetfillcolor{currentfill}%
\pgfsetlinewidth{0.803000pt}%
\definecolor{currentstroke}{rgb}{0.000000,0.000000,0.000000}%
\pgfsetstrokecolor{currentstroke}%
\pgfsetdash{}{0pt}%
\pgfsys@defobject{currentmarker}{\pgfqpoint{0.000000in}{-0.048611in}}{\pgfqpoint{0.000000in}{0.000000in}}{%
\pgfpathmoveto{\pgfqpoint{0.000000in}{0.000000in}}%
\pgfpathlineto{\pgfqpoint{0.000000in}{-0.048611in}}%
\pgfusepath{stroke,fill}%
}%
\begin{pgfscope}%
\pgfsys@transformshift{3.740111in}{0.528000in}%
\pgfsys@useobject{currentmarker}{}%
\end{pgfscope}%
\end{pgfscope}%
\begin{pgfscope}%
\definecolor{textcolor}{rgb}{0.000000,0.000000,0.000000}%
\pgfsetstrokecolor{textcolor}%
\pgfsetfillcolor{textcolor}%
\pgftext[x=3.740111in,y=0.430778in,,top]{\color{textcolor}\rmfamily\fontsize{10.000000}{12.000000}\selectfont \(\displaystyle {600}\)}%
\end{pgfscope}%
\begin{pgfscope}%
\pgfsetbuttcap%
\pgfsetroundjoin%
\definecolor{currentfill}{rgb}{0.000000,0.000000,0.000000}%
\pgfsetfillcolor{currentfill}%
\pgfsetlinewidth{0.803000pt}%
\definecolor{currentstroke}{rgb}{0.000000,0.000000,0.000000}%
\pgfsetstrokecolor{currentstroke}%
\pgfsetdash{}{0pt}%
\pgfsys@defobject{currentmarker}{\pgfqpoint{0.000000in}{-0.048611in}}{\pgfqpoint{0.000000in}{0.000000in}}{%
\pgfpathmoveto{\pgfqpoint{0.000000in}{0.000000in}}%
\pgfpathlineto{\pgfqpoint{0.000000in}{-0.048611in}}%
\pgfusepath{stroke,fill}%
}%
\begin{pgfscope}%
\pgfsys@transformshift{4.660334in}{0.528000in}%
\pgfsys@useobject{currentmarker}{}%
\end{pgfscope}%
\end{pgfscope}%
\begin{pgfscope}%
\definecolor{textcolor}{rgb}{0.000000,0.000000,0.000000}%
\pgfsetstrokecolor{textcolor}%
\pgfsetfillcolor{textcolor}%
\pgftext[x=4.660334in,y=0.430778in,,top]{\color{textcolor}\rmfamily\fontsize{10.000000}{12.000000}\selectfont \(\displaystyle {800}\)}%
\end{pgfscope}%
\begin{pgfscope}%
\pgfsetbuttcap%
\pgfsetroundjoin%
\definecolor{currentfill}{rgb}{0.000000,0.000000,0.000000}%
\pgfsetfillcolor{currentfill}%
\pgfsetlinewidth{0.803000pt}%
\definecolor{currentstroke}{rgb}{0.000000,0.000000,0.000000}%
\pgfsetstrokecolor{currentstroke}%
\pgfsetdash{}{0pt}%
\pgfsys@defobject{currentmarker}{\pgfqpoint{0.000000in}{-0.048611in}}{\pgfqpoint{0.000000in}{0.000000in}}{%
\pgfpathmoveto{\pgfqpoint{0.000000in}{0.000000in}}%
\pgfpathlineto{\pgfqpoint{0.000000in}{-0.048611in}}%
\pgfusepath{stroke,fill}%
}%
\begin{pgfscope}%
\pgfsys@transformshift{5.580557in}{0.528000in}%
\pgfsys@useobject{currentmarker}{}%
\end{pgfscope}%
\end{pgfscope}%
\begin{pgfscope}%
\definecolor{textcolor}{rgb}{0.000000,0.000000,0.000000}%
\pgfsetstrokecolor{textcolor}%
\pgfsetfillcolor{textcolor}%
\pgftext[x=5.580557in,y=0.430778in,,top]{\color{textcolor}\rmfamily\fontsize{10.000000}{12.000000}\selectfont \(\displaystyle {1000}\)}%
\end{pgfscope}%
\begin{pgfscope}%
\definecolor{textcolor}{rgb}{0.000000,0.000000,0.000000}%
\pgfsetstrokecolor{textcolor}%
\pgfsetfillcolor{textcolor}%
\pgftext[x=3.280000in,y=0.251766in,,top]{\color{textcolor}\rmfamily\fontsize{10.000000}{12.000000}\selectfont Input Size}%
\end{pgfscope}%
\begin{pgfscope}%
\pgfsetbuttcap%
\pgfsetroundjoin%
\definecolor{currentfill}{rgb}{0.000000,0.000000,0.000000}%
\pgfsetfillcolor{currentfill}%
\pgfsetlinewidth{0.803000pt}%
\definecolor{currentstroke}{rgb}{0.000000,0.000000,0.000000}%
\pgfsetstrokecolor{currentstroke}%
\pgfsetdash{}{0pt}%
\pgfsys@defobject{currentmarker}{\pgfqpoint{-0.048611in}{0.000000in}}{\pgfqpoint{-0.000000in}{0.000000in}}{%
\pgfpathmoveto{\pgfqpoint{-0.000000in}{0.000000in}}%
\pgfpathlineto{\pgfqpoint{-0.048611in}{0.000000in}}%
\pgfusepath{stroke,fill}%
}%
\begin{pgfscope}%
\pgfsys@transformshift{0.800000in}{1.268712in}%
\pgfsys@useobject{currentmarker}{}%
\end{pgfscope}%
\end{pgfscope}%
\begin{pgfscope}%
\definecolor{textcolor}{rgb}{0.000000,0.000000,0.000000}%
\pgfsetstrokecolor{textcolor}%
\pgfsetfillcolor{textcolor}%
\pgftext[x=0.501581in, y=1.220487in, left, base]{\color{textcolor}\rmfamily\fontsize{10.000000}{12.000000}\selectfont \(\displaystyle {10^{3}}\)}%
\end{pgfscope}%
\begin{pgfscope}%
\pgfsetbuttcap%
\pgfsetroundjoin%
\definecolor{currentfill}{rgb}{0.000000,0.000000,0.000000}%
\pgfsetfillcolor{currentfill}%
\pgfsetlinewidth{0.803000pt}%
\definecolor{currentstroke}{rgb}{0.000000,0.000000,0.000000}%
\pgfsetstrokecolor{currentstroke}%
\pgfsetdash{}{0pt}%
\pgfsys@defobject{currentmarker}{\pgfqpoint{-0.048611in}{0.000000in}}{\pgfqpoint{-0.000000in}{0.000000in}}{%
\pgfpathmoveto{\pgfqpoint{-0.000000in}{0.000000in}}%
\pgfpathlineto{\pgfqpoint{-0.048611in}{0.000000in}}%
\pgfusepath{stroke,fill}%
}%
\begin{pgfscope}%
\pgfsys@transformshift{0.800000in}{2.328098in}%
\pgfsys@useobject{currentmarker}{}%
\end{pgfscope}%
\end{pgfscope}%
\begin{pgfscope}%
\definecolor{textcolor}{rgb}{0.000000,0.000000,0.000000}%
\pgfsetstrokecolor{textcolor}%
\pgfsetfillcolor{textcolor}%
\pgftext[x=0.501581in, y=2.279872in, left, base]{\color{textcolor}\rmfamily\fontsize{10.000000}{12.000000}\selectfont \(\displaystyle {10^{4}}\)}%
\end{pgfscope}%
\begin{pgfscope}%
\pgfsetbuttcap%
\pgfsetroundjoin%
\definecolor{currentfill}{rgb}{0.000000,0.000000,0.000000}%
\pgfsetfillcolor{currentfill}%
\pgfsetlinewidth{0.803000pt}%
\definecolor{currentstroke}{rgb}{0.000000,0.000000,0.000000}%
\pgfsetstrokecolor{currentstroke}%
\pgfsetdash{}{0pt}%
\pgfsys@defobject{currentmarker}{\pgfqpoint{-0.048611in}{0.000000in}}{\pgfqpoint{-0.000000in}{0.000000in}}{%
\pgfpathmoveto{\pgfqpoint{-0.000000in}{0.000000in}}%
\pgfpathlineto{\pgfqpoint{-0.048611in}{0.000000in}}%
\pgfusepath{stroke,fill}%
}%
\begin{pgfscope}%
\pgfsys@transformshift{0.800000in}{3.387483in}%
\pgfsys@useobject{currentmarker}{}%
\end{pgfscope}%
\end{pgfscope}%
\begin{pgfscope}%
\definecolor{textcolor}{rgb}{0.000000,0.000000,0.000000}%
\pgfsetstrokecolor{textcolor}%
\pgfsetfillcolor{textcolor}%
\pgftext[x=0.501581in, y=3.339258in, left, base]{\color{textcolor}\rmfamily\fontsize{10.000000}{12.000000}\selectfont \(\displaystyle {10^{5}}\)}%
\end{pgfscope}%
\begin{pgfscope}%
\pgfsetbuttcap%
\pgfsetroundjoin%
\definecolor{currentfill}{rgb}{0.000000,0.000000,0.000000}%
\pgfsetfillcolor{currentfill}%
\pgfsetlinewidth{0.602250pt}%
\definecolor{currentstroke}{rgb}{0.000000,0.000000,0.000000}%
\pgfsetstrokecolor{currentstroke}%
\pgfsetdash{}{0pt}%
\pgfsys@defobject{currentmarker}{\pgfqpoint{-0.027778in}{0.000000in}}{\pgfqpoint{-0.000000in}{0.000000in}}{%
\pgfpathmoveto{\pgfqpoint{-0.000000in}{0.000000in}}%
\pgfpathlineto{\pgfqpoint{-0.027778in}{0.000000in}}%
\pgfusepath{stroke,fill}%
}%
\begin{pgfscope}%
\pgfsys@transformshift{0.800000in}{0.528233in}%
\pgfsys@useobject{currentmarker}{}%
\end{pgfscope}%
\end{pgfscope}%
\begin{pgfscope}%
\pgfsetbuttcap%
\pgfsetroundjoin%
\definecolor{currentfill}{rgb}{0.000000,0.000000,0.000000}%
\pgfsetfillcolor{currentfill}%
\pgfsetlinewidth{0.602250pt}%
\definecolor{currentstroke}{rgb}{0.000000,0.000000,0.000000}%
\pgfsetstrokecolor{currentstroke}%
\pgfsetdash{}{0pt}%
\pgfsys@defobject{currentmarker}{\pgfqpoint{-0.027778in}{0.000000in}}{\pgfqpoint{-0.000000in}{0.000000in}}{%
\pgfpathmoveto{\pgfqpoint{-0.000000in}{0.000000in}}%
\pgfpathlineto{\pgfqpoint{-0.027778in}{0.000000in}}%
\pgfusepath{stroke,fill}%
}%
\begin{pgfscope}%
\pgfsys@transformshift{0.800000in}{0.714782in}%
\pgfsys@useobject{currentmarker}{}%
\end{pgfscope}%
\end{pgfscope}%
\begin{pgfscope}%
\pgfsetbuttcap%
\pgfsetroundjoin%
\definecolor{currentfill}{rgb}{0.000000,0.000000,0.000000}%
\pgfsetfillcolor{currentfill}%
\pgfsetlinewidth{0.602250pt}%
\definecolor{currentstroke}{rgb}{0.000000,0.000000,0.000000}%
\pgfsetstrokecolor{currentstroke}%
\pgfsetdash{}{0pt}%
\pgfsys@defobject{currentmarker}{\pgfqpoint{-0.027778in}{0.000000in}}{\pgfqpoint{-0.000000in}{0.000000in}}{%
\pgfpathmoveto{\pgfqpoint{-0.000000in}{0.000000in}}%
\pgfpathlineto{\pgfqpoint{-0.027778in}{0.000000in}}%
\pgfusepath{stroke,fill}%
}%
\begin{pgfscope}%
\pgfsys@transformshift{0.800000in}{0.847140in}%
\pgfsys@useobject{currentmarker}{}%
\end{pgfscope}%
\end{pgfscope}%
\begin{pgfscope}%
\pgfsetbuttcap%
\pgfsetroundjoin%
\definecolor{currentfill}{rgb}{0.000000,0.000000,0.000000}%
\pgfsetfillcolor{currentfill}%
\pgfsetlinewidth{0.602250pt}%
\definecolor{currentstroke}{rgb}{0.000000,0.000000,0.000000}%
\pgfsetstrokecolor{currentstroke}%
\pgfsetdash{}{0pt}%
\pgfsys@defobject{currentmarker}{\pgfqpoint{-0.027778in}{0.000000in}}{\pgfqpoint{-0.000000in}{0.000000in}}{%
\pgfpathmoveto{\pgfqpoint{-0.000000in}{0.000000in}}%
\pgfpathlineto{\pgfqpoint{-0.027778in}{0.000000in}}%
\pgfusepath{stroke,fill}%
}%
\begin{pgfscope}%
\pgfsys@transformshift{0.800000in}{0.949805in}%
\pgfsys@useobject{currentmarker}{}%
\end{pgfscope}%
\end{pgfscope}%
\begin{pgfscope}%
\pgfsetbuttcap%
\pgfsetroundjoin%
\definecolor{currentfill}{rgb}{0.000000,0.000000,0.000000}%
\pgfsetfillcolor{currentfill}%
\pgfsetlinewidth{0.602250pt}%
\definecolor{currentstroke}{rgb}{0.000000,0.000000,0.000000}%
\pgfsetstrokecolor{currentstroke}%
\pgfsetdash{}{0pt}%
\pgfsys@defobject{currentmarker}{\pgfqpoint{-0.027778in}{0.000000in}}{\pgfqpoint{-0.000000in}{0.000000in}}{%
\pgfpathmoveto{\pgfqpoint{-0.000000in}{0.000000in}}%
\pgfpathlineto{\pgfqpoint{-0.027778in}{0.000000in}}%
\pgfusepath{stroke,fill}%
}%
\begin{pgfscope}%
\pgfsys@transformshift{0.800000in}{1.033688in}%
\pgfsys@useobject{currentmarker}{}%
\end{pgfscope}%
\end{pgfscope}%
\begin{pgfscope}%
\pgfsetbuttcap%
\pgfsetroundjoin%
\definecolor{currentfill}{rgb}{0.000000,0.000000,0.000000}%
\pgfsetfillcolor{currentfill}%
\pgfsetlinewidth{0.602250pt}%
\definecolor{currentstroke}{rgb}{0.000000,0.000000,0.000000}%
\pgfsetstrokecolor{currentstroke}%
\pgfsetdash{}{0pt}%
\pgfsys@defobject{currentmarker}{\pgfqpoint{-0.027778in}{0.000000in}}{\pgfqpoint{-0.000000in}{0.000000in}}{%
\pgfpathmoveto{\pgfqpoint{-0.000000in}{0.000000in}}%
\pgfpathlineto{\pgfqpoint{-0.027778in}{0.000000in}}%
\pgfusepath{stroke,fill}%
}%
\begin{pgfscope}%
\pgfsys@transformshift{0.800000in}{1.104611in}%
\pgfsys@useobject{currentmarker}{}%
\end{pgfscope}%
\end{pgfscope}%
\begin{pgfscope}%
\pgfsetbuttcap%
\pgfsetroundjoin%
\definecolor{currentfill}{rgb}{0.000000,0.000000,0.000000}%
\pgfsetfillcolor{currentfill}%
\pgfsetlinewidth{0.602250pt}%
\definecolor{currentstroke}{rgb}{0.000000,0.000000,0.000000}%
\pgfsetstrokecolor{currentstroke}%
\pgfsetdash{}{0pt}%
\pgfsys@defobject{currentmarker}{\pgfqpoint{-0.027778in}{0.000000in}}{\pgfqpoint{-0.000000in}{0.000000in}}{%
\pgfpathmoveto{\pgfqpoint{-0.000000in}{0.000000in}}%
\pgfpathlineto{\pgfqpoint{-0.027778in}{0.000000in}}%
\pgfusepath{stroke,fill}%
}%
\begin{pgfscope}%
\pgfsys@transformshift{0.800000in}{1.166047in}%
\pgfsys@useobject{currentmarker}{}%
\end{pgfscope}%
\end{pgfscope}%
\begin{pgfscope}%
\pgfsetbuttcap%
\pgfsetroundjoin%
\definecolor{currentfill}{rgb}{0.000000,0.000000,0.000000}%
\pgfsetfillcolor{currentfill}%
\pgfsetlinewidth{0.602250pt}%
\definecolor{currentstroke}{rgb}{0.000000,0.000000,0.000000}%
\pgfsetstrokecolor{currentstroke}%
\pgfsetdash{}{0pt}%
\pgfsys@defobject{currentmarker}{\pgfqpoint{-0.027778in}{0.000000in}}{\pgfqpoint{-0.000000in}{0.000000in}}{%
\pgfpathmoveto{\pgfqpoint{-0.000000in}{0.000000in}}%
\pgfpathlineto{\pgfqpoint{-0.027778in}{0.000000in}}%
\pgfusepath{stroke,fill}%
}%
\begin{pgfscope}%
\pgfsys@transformshift{0.800000in}{1.220237in}%
\pgfsys@useobject{currentmarker}{}%
\end{pgfscope}%
\end{pgfscope}%
\begin{pgfscope}%
\pgfsetbuttcap%
\pgfsetroundjoin%
\definecolor{currentfill}{rgb}{0.000000,0.000000,0.000000}%
\pgfsetfillcolor{currentfill}%
\pgfsetlinewidth{0.602250pt}%
\definecolor{currentstroke}{rgb}{0.000000,0.000000,0.000000}%
\pgfsetstrokecolor{currentstroke}%
\pgfsetdash{}{0pt}%
\pgfsys@defobject{currentmarker}{\pgfqpoint{-0.027778in}{0.000000in}}{\pgfqpoint{-0.000000in}{0.000000in}}{%
\pgfpathmoveto{\pgfqpoint{-0.000000in}{0.000000in}}%
\pgfpathlineto{\pgfqpoint{-0.027778in}{0.000000in}}%
\pgfusepath{stroke,fill}%
}%
\begin{pgfscope}%
\pgfsys@transformshift{0.800000in}{1.587619in}%
\pgfsys@useobject{currentmarker}{}%
\end{pgfscope}%
\end{pgfscope}%
\begin{pgfscope}%
\pgfsetbuttcap%
\pgfsetroundjoin%
\definecolor{currentfill}{rgb}{0.000000,0.000000,0.000000}%
\pgfsetfillcolor{currentfill}%
\pgfsetlinewidth{0.602250pt}%
\definecolor{currentstroke}{rgb}{0.000000,0.000000,0.000000}%
\pgfsetstrokecolor{currentstroke}%
\pgfsetdash{}{0pt}%
\pgfsys@defobject{currentmarker}{\pgfqpoint{-0.027778in}{0.000000in}}{\pgfqpoint{-0.000000in}{0.000000in}}{%
\pgfpathmoveto{\pgfqpoint{-0.000000in}{0.000000in}}%
\pgfpathlineto{\pgfqpoint{-0.027778in}{0.000000in}}%
\pgfusepath{stroke,fill}%
}%
\begin{pgfscope}%
\pgfsys@transformshift{0.800000in}{1.774167in}%
\pgfsys@useobject{currentmarker}{}%
\end{pgfscope}%
\end{pgfscope}%
\begin{pgfscope}%
\pgfsetbuttcap%
\pgfsetroundjoin%
\definecolor{currentfill}{rgb}{0.000000,0.000000,0.000000}%
\pgfsetfillcolor{currentfill}%
\pgfsetlinewidth{0.602250pt}%
\definecolor{currentstroke}{rgb}{0.000000,0.000000,0.000000}%
\pgfsetstrokecolor{currentstroke}%
\pgfsetdash{}{0pt}%
\pgfsys@defobject{currentmarker}{\pgfqpoint{-0.027778in}{0.000000in}}{\pgfqpoint{-0.000000in}{0.000000in}}{%
\pgfpathmoveto{\pgfqpoint{-0.000000in}{0.000000in}}%
\pgfpathlineto{\pgfqpoint{-0.027778in}{0.000000in}}%
\pgfusepath{stroke,fill}%
}%
\begin{pgfscope}%
\pgfsys@transformshift{0.800000in}{1.906526in}%
\pgfsys@useobject{currentmarker}{}%
\end{pgfscope}%
\end{pgfscope}%
\begin{pgfscope}%
\pgfsetbuttcap%
\pgfsetroundjoin%
\definecolor{currentfill}{rgb}{0.000000,0.000000,0.000000}%
\pgfsetfillcolor{currentfill}%
\pgfsetlinewidth{0.602250pt}%
\definecolor{currentstroke}{rgb}{0.000000,0.000000,0.000000}%
\pgfsetstrokecolor{currentstroke}%
\pgfsetdash{}{0pt}%
\pgfsys@defobject{currentmarker}{\pgfqpoint{-0.027778in}{0.000000in}}{\pgfqpoint{-0.000000in}{0.000000in}}{%
\pgfpathmoveto{\pgfqpoint{-0.000000in}{0.000000in}}%
\pgfpathlineto{\pgfqpoint{-0.027778in}{0.000000in}}%
\pgfusepath{stroke,fill}%
}%
\begin{pgfscope}%
\pgfsys@transformshift{0.800000in}{2.009191in}%
\pgfsys@useobject{currentmarker}{}%
\end{pgfscope}%
\end{pgfscope}%
\begin{pgfscope}%
\pgfsetbuttcap%
\pgfsetroundjoin%
\definecolor{currentfill}{rgb}{0.000000,0.000000,0.000000}%
\pgfsetfillcolor{currentfill}%
\pgfsetlinewidth{0.602250pt}%
\definecolor{currentstroke}{rgb}{0.000000,0.000000,0.000000}%
\pgfsetstrokecolor{currentstroke}%
\pgfsetdash{}{0pt}%
\pgfsys@defobject{currentmarker}{\pgfqpoint{-0.027778in}{0.000000in}}{\pgfqpoint{-0.000000in}{0.000000in}}{%
\pgfpathmoveto{\pgfqpoint{-0.000000in}{0.000000in}}%
\pgfpathlineto{\pgfqpoint{-0.027778in}{0.000000in}}%
\pgfusepath{stroke,fill}%
}%
\begin{pgfscope}%
\pgfsys@transformshift{0.800000in}{2.093074in}%
\pgfsys@useobject{currentmarker}{}%
\end{pgfscope}%
\end{pgfscope}%
\begin{pgfscope}%
\pgfsetbuttcap%
\pgfsetroundjoin%
\definecolor{currentfill}{rgb}{0.000000,0.000000,0.000000}%
\pgfsetfillcolor{currentfill}%
\pgfsetlinewidth{0.602250pt}%
\definecolor{currentstroke}{rgb}{0.000000,0.000000,0.000000}%
\pgfsetstrokecolor{currentstroke}%
\pgfsetdash{}{0pt}%
\pgfsys@defobject{currentmarker}{\pgfqpoint{-0.027778in}{0.000000in}}{\pgfqpoint{-0.000000in}{0.000000in}}{%
\pgfpathmoveto{\pgfqpoint{-0.000000in}{0.000000in}}%
\pgfpathlineto{\pgfqpoint{-0.027778in}{0.000000in}}%
\pgfusepath{stroke,fill}%
}%
\begin{pgfscope}%
\pgfsys@transformshift{0.800000in}{2.163997in}%
\pgfsys@useobject{currentmarker}{}%
\end{pgfscope}%
\end{pgfscope}%
\begin{pgfscope}%
\pgfsetbuttcap%
\pgfsetroundjoin%
\definecolor{currentfill}{rgb}{0.000000,0.000000,0.000000}%
\pgfsetfillcolor{currentfill}%
\pgfsetlinewidth{0.602250pt}%
\definecolor{currentstroke}{rgb}{0.000000,0.000000,0.000000}%
\pgfsetstrokecolor{currentstroke}%
\pgfsetdash{}{0pt}%
\pgfsys@defobject{currentmarker}{\pgfqpoint{-0.027778in}{0.000000in}}{\pgfqpoint{-0.000000in}{0.000000in}}{%
\pgfpathmoveto{\pgfqpoint{-0.000000in}{0.000000in}}%
\pgfpathlineto{\pgfqpoint{-0.027778in}{0.000000in}}%
\pgfusepath{stroke,fill}%
}%
\begin{pgfscope}%
\pgfsys@transformshift{0.800000in}{2.225433in}%
\pgfsys@useobject{currentmarker}{}%
\end{pgfscope}%
\end{pgfscope}%
\begin{pgfscope}%
\pgfsetbuttcap%
\pgfsetroundjoin%
\definecolor{currentfill}{rgb}{0.000000,0.000000,0.000000}%
\pgfsetfillcolor{currentfill}%
\pgfsetlinewidth{0.602250pt}%
\definecolor{currentstroke}{rgb}{0.000000,0.000000,0.000000}%
\pgfsetstrokecolor{currentstroke}%
\pgfsetdash{}{0pt}%
\pgfsys@defobject{currentmarker}{\pgfqpoint{-0.027778in}{0.000000in}}{\pgfqpoint{-0.000000in}{0.000000in}}{%
\pgfpathmoveto{\pgfqpoint{-0.000000in}{0.000000in}}%
\pgfpathlineto{\pgfqpoint{-0.027778in}{0.000000in}}%
\pgfusepath{stroke,fill}%
}%
\begin{pgfscope}%
\pgfsys@transformshift{0.800000in}{2.279623in}%
\pgfsys@useobject{currentmarker}{}%
\end{pgfscope}%
\end{pgfscope}%
\begin{pgfscope}%
\pgfsetbuttcap%
\pgfsetroundjoin%
\definecolor{currentfill}{rgb}{0.000000,0.000000,0.000000}%
\pgfsetfillcolor{currentfill}%
\pgfsetlinewidth{0.602250pt}%
\definecolor{currentstroke}{rgb}{0.000000,0.000000,0.000000}%
\pgfsetstrokecolor{currentstroke}%
\pgfsetdash{}{0pt}%
\pgfsys@defobject{currentmarker}{\pgfqpoint{-0.027778in}{0.000000in}}{\pgfqpoint{-0.000000in}{0.000000in}}{%
\pgfpathmoveto{\pgfqpoint{-0.000000in}{0.000000in}}%
\pgfpathlineto{\pgfqpoint{-0.027778in}{0.000000in}}%
\pgfusepath{stroke,fill}%
}%
\begin{pgfscope}%
\pgfsys@transformshift{0.800000in}{2.647005in}%
\pgfsys@useobject{currentmarker}{}%
\end{pgfscope}%
\end{pgfscope}%
\begin{pgfscope}%
\pgfsetbuttcap%
\pgfsetroundjoin%
\definecolor{currentfill}{rgb}{0.000000,0.000000,0.000000}%
\pgfsetfillcolor{currentfill}%
\pgfsetlinewidth{0.602250pt}%
\definecolor{currentstroke}{rgb}{0.000000,0.000000,0.000000}%
\pgfsetstrokecolor{currentstroke}%
\pgfsetdash{}{0pt}%
\pgfsys@defobject{currentmarker}{\pgfqpoint{-0.027778in}{0.000000in}}{\pgfqpoint{-0.000000in}{0.000000in}}{%
\pgfpathmoveto{\pgfqpoint{-0.000000in}{0.000000in}}%
\pgfpathlineto{\pgfqpoint{-0.027778in}{0.000000in}}%
\pgfusepath{stroke,fill}%
}%
\begin{pgfscope}%
\pgfsys@transformshift{0.800000in}{2.833553in}%
\pgfsys@useobject{currentmarker}{}%
\end{pgfscope}%
\end{pgfscope}%
\begin{pgfscope}%
\pgfsetbuttcap%
\pgfsetroundjoin%
\definecolor{currentfill}{rgb}{0.000000,0.000000,0.000000}%
\pgfsetfillcolor{currentfill}%
\pgfsetlinewidth{0.602250pt}%
\definecolor{currentstroke}{rgb}{0.000000,0.000000,0.000000}%
\pgfsetstrokecolor{currentstroke}%
\pgfsetdash{}{0pt}%
\pgfsys@defobject{currentmarker}{\pgfqpoint{-0.027778in}{0.000000in}}{\pgfqpoint{-0.000000in}{0.000000in}}{%
\pgfpathmoveto{\pgfqpoint{-0.000000in}{0.000000in}}%
\pgfpathlineto{\pgfqpoint{-0.027778in}{0.000000in}}%
\pgfusepath{stroke,fill}%
}%
\begin{pgfscope}%
\pgfsys@transformshift{0.800000in}{2.965911in}%
\pgfsys@useobject{currentmarker}{}%
\end{pgfscope}%
\end{pgfscope}%
\begin{pgfscope}%
\pgfsetbuttcap%
\pgfsetroundjoin%
\definecolor{currentfill}{rgb}{0.000000,0.000000,0.000000}%
\pgfsetfillcolor{currentfill}%
\pgfsetlinewidth{0.602250pt}%
\definecolor{currentstroke}{rgb}{0.000000,0.000000,0.000000}%
\pgfsetstrokecolor{currentstroke}%
\pgfsetdash{}{0pt}%
\pgfsys@defobject{currentmarker}{\pgfqpoint{-0.027778in}{0.000000in}}{\pgfqpoint{-0.000000in}{0.000000in}}{%
\pgfpathmoveto{\pgfqpoint{-0.000000in}{0.000000in}}%
\pgfpathlineto{\pgfqpoint{-0.027778in}{0.000000in}}%
\pgfusepath{stroke,fill}%
}%
\begin{pgfscope}%
\pgfsys@transformshift{0.800000in}{3.068576in}%
\pgfsys@useobject{currentmarker}{}%
\end{pgfscope}%
\end{pgfscope}%
\begin{pgfscope}%
\pgfsetbuttcap%
\pgfsetroundjoin%
\definecolor{currentfill}{rgb}{0.000000,0.000000,0.000000}%
\pgfsetfillcolor{currentfill}%
\pgfsetlinewidth{0.602250pt}%
\definecolor{currentstroke}{rgb}{0.000000,0.000000,0.000000}%
\pgfsetstrokecolor{currentstroke}%
\pgfsetdash{}{0pt}%
\pgfsys@defobject{currentmarker}{\pgfqpoint{-0.027778in}{0.000000in}}{\pgfqpoint{-0.000000in}{0.000000in}}{%
\pgfpathmoveto{\pgfqpoint{-0.000000in}{0.000000in}}%
\pgfpathlineto{\pgfqpoint{-0.027778in}{0.000000in}}%
\pgfusepath{stroke,fill}%
}%
\begin{pgfscope}%
\pgfsys@transformshift{0.800000in}{3.152460in}%
\pgfsys@useobject{currentmarker}{}%
\end{pgfscope}%
\end{pgfscope}%
\begin{pgfscope}%
\pgfsetbuttcap%
\pgfsetroundjoin%
\definecolor{currentfill}{rgb}{0.000000,0.000000,0.000000}%
\pgfsetfillcolor{currentfill}%
\pgfsetlinewidth{0.602250pt}%
\definecolor{currentstroke}{rgb}{0.000000,0.000000,0.000000}%
\pgfsetstrokecolor{currentstroke}%
\pgfsetdash{}{0pt}%
\pgfsys@defobject{currentmarker}{\pgfqpoint{-0.027778in}{0.000000in}}{\pgfqpoint{-0.000000in}{0.000000in}}{%
\pgfpathmoveto{\pgfqpoint{-0.000000in}{0.000000in}}%
\pgfpathlineto{\pgfqpoint{-0.027778in}{0.000000in}}%
\pgfusepath{stroke,fill}%
}%
\begin{pgfscope}%
\pgfsys@transformshift{0.800000in}{3.223382in}%
\pgfsys@useobject{currentmarker}{}%
\end{pgfscope}%
\end{pgfscope}%
\begin{pgfscope}%
\pgfsetbuttcap%
\pgfsetroundjoin%
\definecolor{currentfill}{rgb}{0.000000,0.000000,0.000000}%
\pgfsetfillcolor{currentfill}%
\pgfsetlinewidth{0.602250pt}%
\definecolor{currentstroke}{rgb}{0.000000,0.000000,0.000000}%
\pgfsetstrokecolor{currentstroke}%
\pgfsetdash{}{0pt}%
\pgfsys@defobject{currentmarker}{\pgfqpoint{-0.027778in}{0.000000in}}{\pgfqpoint{-0.000000in}{0.000000in}}{%
\pgfpathmoveto{\pgfqpoint{-0.000000in}{0.000000in}}%
\pgfpathlineto{\pgfqpoint{-0.027778in}{0.000000in}}%
\pgfusepath{stroke,fill}%
}%
\begin{pgfscope}%
\pgfsys@transformshift{0.800000in}{3.284818in}%
\pgfsys@useobject{currentmarker}{}%
\end{pgfscope}%
\end{pgfscope}%
\begin{pgfscope}%
\pgfsetbuttcap%
\pgfsetroundjoin%
\definecolor{currentfill}{rgb}{0.000000,0.000000,0.000000}%
\pgfsetfillcolor{currentfill}%
\pgfsetlinewidth{0.602250pt}%
\definecolor{currentstroke}{rgb}{0.000000,0.000000,0.000000}%
\pgfsetstrokecolor{currentstroke}%
\pgfsetdash{}{0pt}%
\pgfsys@defobject{currentmarker}{\pgfqpoint{-0.027778in}{0.000000in}}{\pgfqpoint{-0.000000in}{0.000000in}}{%
\pgfpathmoveto{\pgfqpoint{-0.000000in}{0.000000in}}%
\pgfpathlineto{\pgfqpoint{-0.027778in}{0.000000in}}%
\pgfusepath{stroke,fill}%
}%
\begin{pgfscope}%
\pgfsys@transformshift{0.800000in}{3.339009in}%
\pgfsys@useobject{currentmarker}{}%
\end{pgfscope}%
\end{pgfscope}%
\begin{pgfscope}%
\pgfsetbuttcap%
\pgfsetroundjoin%
\definecolor{currentfill}{rgb}{0.000000,0.000000,0.000000}%
\pgfsetfillcolor{currentfill}%
\pgfsetlinewidth{0.602250pt}%
\definecolor{currentstroke}{rgb}{0.000000,0.000000,0.000000}%
\pgfsetstrokecolor{currentstroke}%
\pgfsetdash{}{0pt}%
\pgfsys@defobject{currentmarker}{\pgfqpoint{-0.027778in}{0.000000in}}{\pgfqpoint{-0.000000in}{0.000000in}}{%
\pgfpathmoveto{\pgfqpoint{-0.000000in}{0.000000in}}%
\pgfpathlineto{\pgfqpoint{-0.027778in}{0.000000in}}%
\pgfusepath{stroke,fill}%
}%
\begin{pgfscope}%
\pgfsys@transformshift{0.800000in}{3.706390in}%
\pgfsys@useobject{currentmarker}{}%
\end{pgfscope}%
\end{pgfscope}%
\begin{pgfscope}%
\pgfsetbuttcap%
\pgfsetroundjoin%
\definecolor{currentfill}{rgb}{0.000000,0.000000,0.000000}%
\pgfsetfillcolor{currentfill}%
\pgfsetlinewidth{0.602250pt}%
\definecolor{currentstroke}{rgb}{0.000000,0.000000,0.000000}%
\pgfsetstrokecolor{currentstroke}%
\pgfsetdash{}{0pt}%
\pgfsys@defobject{currentmarker}{\pgfqpoint{-0.027778in}{0.000000in}}{\pgfqpoint{-0.000000in}{0.000000in}}{%
\pgfpathmoveto{\pgfqpoint{-0.000000in}{0.000000in}}%
\pgfpathlineto{\pgfqpoint{-0.027778in}{0.000000in}}%
\pgfusepath{stroke,fill}%
}%
\begin{pgfscope}%
\pgfsys@transformshift{0.800000in}{3.892939in}%
\pgfsys@useobject{currentmarker}{}%
\end{pgfscope}%
\end{pgfscope}%
\begin{pgfscope}%
\pgfsetbuttcap%
\pgfsetroundjoin%
\definecolor{currentfill}{rgb}{0.000000,0.000000,0.000000}%
\pgfsetfillcolor{currentfill}%
\pgfsetlinewidth{0.602250pt}%
\definecolor{currentstroke}{rgb}{0.000000,0.000000,0.000000}%
\pgfsetstrokecolor{currentstroke}%
\pgfsetdash{}{0pt}%
\pgfsys@defobject{currentmarker}{\pgfqpoint{-0.027778in}{0.000000in}}{\pgfqpoint{-0.000000in}{0.000000in}}{%
\pgfpathmoveto{\pgfqpoint{-0.000000in}{0.000000in}}%
\pgfpathlineto{\pgfqpoint{-0.027778in}{0.000000in}}%
\pgfusepath{stroke,fill}%
}%
\begin{pgfscope}%
\pgfsys@transformshift{0.800000in}{4.025297in}%
\pgfsys@useobject{currentmarker}{}%
\end{pgfscope}%
\end{pgfscope}%
\begin{pgfscope}%
\pgfsetbuttcap%
\pgfsetroundjoin%
\definecolor{currentfill}{rgb}{0.000000,0.000000,0.000000}%
\pgfsetfillcolor{currentfill}%
\pgfsetlinewidth{0.602250pt}%
\definecolor{currentstroke}{rgb}{0.000000,0.000000,0.000000}%
\pgfsetstrokecolor{currentstroke}%
\pgfsetdash{}{0pt}%
\pgfsys@defobject{currentmarker}{\pgfqpoint{-0.027778in}{0.000000in}}{\pgfqpoint{-0.000000in}{0.000000in}}{%
\pgfpathmoveto{\pgfqpoint{-0.000000in}{0.000000in}}%
\pgfpathlineto{\pgfqpoint{-0.027778in}{0.000000in}}%
\pgfusepath{stroke,fill}%
}%
\begin{pgfscope}%
\pgfsys@transformshift{0.800000in}{4.127962in}%
\pgfsys@useobject{currentmarker}{}%
\end{pgfscope}%
\end{pgfscope}%
\begin{pgfscope}%
\pgfsetbuttcap%
\pgfsetroundjoin%
\definecolor{currentfill}{rgb}{0.000000,0.000000,0.000000}%
\pgfsetfillcolor{currentfill}%
\pgfsetlinewidth{0.602250pt}%
\definecolor{currentstroke}{rgb}{0.000000,0.000000,0.000000}%
\pgfsetstrokecolor{currentstroke}%
\pgfsetdash{}{0pt}%
\pgfsys@defobject{currentmarker}{\pgfqpoint{-0.027778in}{0.000000in}}{\pgfqpoint{-0.000000in}{0.000000in}}{%
\pgfpathmoveto{\pgfqpoint{-0.000000in}{0.000000in}}%
\pgfpathlineto{\pgfqpoint{-0.027778in}{0.000000in}}%
\pgfusepath{stroke,fill}%
}%
\begin{pgfscope}%
\pgfsys@transformshift{0.800000in}{4.211846in}%
\pgfsys@useobject{currentmarker}{}%
\end{pgfscope}%
\end{pgfscope}%
\begin{pgfscope}%
\definecolor{textcolor}{rgb}{0.000000,0.000000,0.000000}%
\pgfsetstrokecolor{textcolor}%
\pgfsetfillcolor{textcolor}%
\pgftext[x=0.446026in,y=2.376000in,,bottom,rotate=90.000000]{\color{textcolor}\rmfamily\fontsize{10.000000}{12.000000}\selectfont Memory (B)}%
\end{pgfscope}%
\begin{pgfscope}%
\pgfpathrectangle{\pgfqpoint{0.800000in}{0.528000in}}{\pgfqpoint{4.960000in}{3.696000in}}%
\pgfusepath{clip}%
\pgfsetrectcap%
\pgfsetroundjoin%
\pgfsetlinewidth{1.505625pt}%
\definecolor{currentstroke}{rgb}{0.121569,0.466667,0.705882}%
\pgfsetstrokecolor{currentstroke}%
\pgfsetdash{}{0pt}%
\pgfpathmoveto{\pgfqpoint{1.025455in}{2.013222in}}%
\pgfpathlineto{\pgfqpoint{1.071466in}{2.053710in}}%
\pgfpathlineto{\pgfqpoint{1.117477in}{2.012491in}}%
\pgfpathlineto{\pgfqpoint{1.163488in}{2.012491in}}%
\pgfpathlineto{\pgfqpoint{1.209499in}{2.012491in}}%
\pgfpathlineto{\pgfqpoint{1.255510in}{2.012491in}}%
\pgfpathlineto{\pgfqpoint{1.301521in}{2.012491in}}%
\pgfpathlineto{\pgfqpoint{1.347532in}{2.012491in}}%
\pgfpathlineto{\pgfqpoint{1.393544in}{2.012491in}}%
\pgfpathlineto{\pgfqpoint{1.439555in}{2.138595in}}%
\pgfpathlineto{\pgfqpoint{1.485566in}{2.012491in}}%
\pgfpathlineto{\pgfqpoint{1.531577in}{2.012491in}}%
\pgfpathlineto{\pgfqpoint{1.577588in}{2.012491in}}%
\pgfpathlineto{\pgfqpoint{1.623599in}{2.012491in}}%
\pgfpathlineto{\pgfqpoint{1.669610in}{2.012491in}}%
\pgfpathlineto{\pgfqpoint{1.715622in}{2.012491in}}%
\pgfpathlineto{\pgfqpoint{1.761633in}{2.012491in}}%
\pgfpathlineto{\pgfqpoint{1.807644in}{2.012491in}}%
\pgfpathlineto{\pgfqpoint{1.853655in}{2.012491in}}%
\pgfpathlineto{\pgfqpoint{1.899666in}{2.012491in}}%
\pgfpathlineto{\pgfqpoint{1.945677in}{2.012491in}}%
\pgfpathlineto{\pgfqpoint{1.991688in}{2.293917in}}%
\pgfpathlineto{\pgfqpoint{2.037699in}{2.483035in}}%
\pgfpathlineto{\pgfqpoint{2.083711in}{2.616673in}}%
\pgfpathlineto{\pgfqpoint{2.129722in}{2.720104in}}%
\pgfpathlineto{\pgfqpoint{2.175733in}{2.809049in}}%
\pgfpathlineto{\pgfqpoint{2.221744in}{2.891193in}}%
\pgfpathlineto{\pgfqpoint{2.267755in}{2.960869in}}%
\pgfpathlineto{\pgfqpoint{2.313766in}{3.021368in}}%
\pgfpathlineto{\pgfqpoint{2.359777in}{3.074828in}}%
\pgfpathlineto{\pgfqpoint{2.405788in}{3.122718in}}%
\pgfpathlineto{\pgfqpoint{2.451800in}{3.166089in}}%
\pgfpathlineto{\pgfqpoint{2.497811in}{3.205722in}}%
\pgfpathlineto{\pgfqpoint{2.543822in}{3.242210in}}%
\pgfpathlineto{\pgfqpoint{2.589833in}{3.276016in}}%
\pgfpathlineto{\pgfqpoint{2.635844in}{3.307507in}}%
\pgfpathlineto{\pgfqpoint{2.681855in}{3.336980in}}%
\pgfpathlineto{\pgfqpoint{2.727866in}{3.364678in}}%
\pgfpathlineto{\pgfqpoint{2.773878in}{3.390802in}}%
\pgfpathlineto{\pgfqpoint{2.819889in}{3.415523in}}%
\pgfpathlineto{\pgfqpoint{2.865900in}{3.438983in}}%
\pgfpathlineto{\pgfqpoint{2.911911in}{3.461304in}}%
\pgfpathlineto{\pgfqpoint{2.957922in}{3.482593in}}%
\pgfpathlineto{\pgfqpoint{3.003933in}{3.502940in}}%
\pgfpathlineto{\pgfqpoint{3.049944in}{3.522425in}}%
\pgfpathlineto{\pgfqpoint{3.095955in}{3.541118in}}%
\pgfpathlineto{\pgfqpoint{3.141967in}{3.559081in}}%
\pgfpathlineto{\pgfqpoint{3.187978in}{3.576369in}}%
\pgfpathlineto{\pgfqpoint{3.233989in}{3.593031in}}%
\pgfpathlineto{\pgfqpoint{3.280000in}{3.609111in}}%
\pgfpathlineto{\pgfqpoint{3.326011in}{3.624648in}}%
\pgfpathlineto{\pgfqpoint{3.372022in}{3.639304in}}%
\pgfpathlineto{\pgfqpoint{3.418033in}{3.653508in}}%
\pgfpathlineto{\pgfqpoint{3.464045in}{3.667287in}}%
\pgfpathlineto{\pgfqpoint{3.510056in}{3.680665in}}%
\pgfpathlineto{\pgfqpoint{3.556067in}{3.693665in}}%
\pgfpathlineto{\pgfqpoint{3.602078in}{3.706307in}}%
\pgfpathlineto{\pgfqpoint{3.648089in}{3.718612in}}%
\pgfpathlineto{\pgfqpoint{3.694100in}{3.730596in}}%
\pgfpathlineto{\pgfqpoint{3.740111in}{3.742276in}}%
\pgfpathlineto{\pgfqpoint{3.786122in}{3.753666in}}%
\pgfpathlineto{\pgfqpoint{3.832134in}{3.764782in}}%
\pgfpathlineto{\pgfqpoint{3.878145in}{3.775635in}}%
\pgfpathlineto{\pgfqpoint{3.924156in}{3.786238in}}%
\pgfpathlineto{\pgfqpoint{3.970167in}{3.796602in}}%
\pgfpathlineto{\pgfqpoint{4.016178in}{3.806738in}}%
\pgfpathlineto{\pgfqpoint{4.062189in}{3.816655in}}%
\pgfpathlineto{\pgfqpoint{4.108200in}{3.826363in}}%
\pgfpathlineto{\pgfqpoint{4.154212in}{3.835871in}}%
\pgfpathlineto{\pgfqpoint{4.200223in}{3.845186in}}%
\pgfpathlineto{\pgfqpoint{4.246234in}{3.854316in}}%
\pgfpathlineto{\pgfqpoint{4.292245in}{3.863268in}}%
\pgfpathlineto{\pgfqpoint{4.338256in}{3.872050in}}%
\pgfpathlineto{\pgfqpoint{4.384267in}{3.880667in}}%
\pgfpathlineto{\pgfqpoint{4.430278in}{3.889126in}}%
\pgfpathlineto{\pgfqpoint{4.476289in}{3.897432in}}%
\pgfpathlineto{\pgfqpoint{4.522301in}{3.905590in}}%
\pgfpathlineto{\pgfqpoint{4.568312in}{3.913607in}}%
\pgfpathlineto{\pgfqpoint{4.614323in}{3.921486in}}%
\pgfpathlineto{\pgfqpoint{4.660334in}{3.929233in}}%
\pgfpathlineto{\pgfqpoint{4.706345in}{3.936851in}}%
\pgfpathlineto{\pgfqpoint{4.752356in}{3.944345in}}%
\pgfpathlineto{\pgfqpoint{4.798367in}{3.951719in}}%
\pgfpathlineto{\pgfqpoint{4.844378in}{3.958977in}}%
\pgfpathlineto{\pgfqpoint{4.890390in}{3.966122in}}%
\pgfpathlineto{\pgfqpoint{4.936401in}{3.973158in}}%
\pgfpathlineto{\pgfqpoint{4.982412in}{3.980088in}}%
\pgfpathlineto{\pgfqpoint{5.028423in}{3.986914in}}%
\pgfpathlineto{\pgfqpoint{5.074434in}{3.993642in}}%
\pgfpathlineto{\pgfqpoint{5.120445in}{4.000272in}}%
\pgfpathlineto{\pgfqpoint{5.166456in}{4.006808in}}%
\pgfpathlineto{\pgfqpoint{5.212468in}{4.013252in}}%
\pgfpathlineto{\pgfqpoint{5.258479in}{4.019608in}}%
\pgfpathlineto{\pgfqpoint{5.304490in}{4.025876in}}%
\pgfpathlineto{\pgfqpoint{5.350501in}{4.032061in}}%
\pgfpathlineto{\pgfqpoint{5.396512in}{4.038164in}}%
\pgfpathlineto{\pgfqpoint{5.442523in}{4.044186in}}%
\pgfpathlineto{\pgfqpoint{5.488534in}{4.050131in}}%
\pgfpathlineto{\pgfqpoint{5.534545in}{4.056000in}}%
\pgfusepath{stroke}%
\end{pgfscope}%
\begin{pgfscope}%
\pgfpathrectangle{\pgfqpoint{0.800000in}{0.528000in}}{\pgfqpoint{4.960000in}{3.696000in}}%
\pgfusepath{clip}%
\pgfsetrectcap%
\pgfsetroundjoin%
\pgfsetlinewidth{1.505625pt}%
\definecolor{currentstroke}{rgb}{1.000000,0.498039,0.054902}%
\pgfsetstrokecolor{currentstroke}%
\pgfsetdash{}{0pt}%
\pgfpathmoveto{\pgfqpoint{1.025455in}{1.729764in}}%
\pgfpathlineto{\pgfqpoint{1.071466in}{1.370640in}}%
\pgfpathlineto{\pgfqpoint{1.117477in}{0.696000in}}%
\pgfpathlineto{\pgfqpoint{1.163488in}{1.163741in}}%
\pgfpathlineto{\pgfqpoint{1.209499in}{0.738688in}}%
\pgfpathlineto{\pgfqpoint{1.255510in}{0.738688in}}%
\pgfpathlineto{\pgfqpoint{1.301521in}{1.163741in}}%
\pgfpathlineto{\pgfqpoint{1.347532in}{0.738688in}}%
\pgfpathlineto{\pgfqpoint{1.393544in}{0.738688in}}%
\pgfpathlineto{\pgfqpoint{1.439555in}{0.738688in}}%
\pgfpathlineto{\pgfqpoint{1.485566in}{0.738688in}}%
\pgfpathlineto{\pgfqpoint{1.531577in}{1.581831in}}%
\pgfpathlineto{\pgfqpoint{1.577588in}{1.163741in}}%
\pgfpathlineto{\pgfqpoint{1.623599in}{0.738688in}}%
\pgfpathlineto{\pgfqpoint{1.669610in}{0.738688in}}%
\pgfpathlineto{\pgfqpoint{1.715622in}{0.738688in}}%
\pgfpathlineto{\pgfqpoint{1.761633in}{0.738688in}}%
\pgfpathlineto{\pgfqpoint{1.807644in}{0.738688in}}%
\pgfpathlineto{\pgfqpoint{1.853655in}{0.738688in}}%
\pgfpathlineto{\pgfqpoint{1.899666in}{0.738688in}}%
\pgfpathlineto{\pgfqpoint{1.945677in}{0.738688in}}%
\pgfpathlineto{\pgfqpoint{1.991688in}{0.738688in}}%
\pgfpathlineto{\pgfqpoint{2.037699in}{0.777749in}}%
\pgfpathlineto{\pgfqpoint{2.083711in}{0.777749in}}%
\pgfpathlineto{\pgfqpoint{2.129722in}{0.777749in}}%
\pgfpathlineto{\pgfqpoint{2.175733in}{1.277823in}}%
\pgfpathlineto{\pgfqpoint{2.221744in}{1.207802in}}%
\pgfpathlineto{\pgfqpoint{2.267755in}{1.263157in}}%
\pgfpathlineto{\pgfqpoint{2.313766in}{1.263157in}}%
\pgfpathlineto{\pgfqpoint{2.359777in}{1.312563in}}%
\pgfpathlineto{\pgfqpoint{2.405788in}{1.312563in}}%
\pgfpathlineto{\pgfqpoint{2.451800in}{1.312563in}}%
\pgfpathlineto{\pgfqpoint{2.497811in}{1.312563in}}%
\pgfpathlineto{\pgfqpoint{2.543822in}{1.312563in}}%
\pgfpathlineto{\pgfqpoint{2.589833in}{1.357173in}}%
\pgfpathlineto{\pgfqpoint{2.635844in}{1.357173in}}%
\pgfpathlineto{\pgfqpoint{2.681855in}{1.357173in}}%
\pgfpathlineto{\pgfqpoint{2.727866in}{1.357173in}}%
\pgfpathlineto{\pgfqpoint{2.773878in}{1.367681in}}%
\pgfpathlineto{\pgfqpoint{2.819889in}{1.367681in}}%
\pgfpathlineto{\pgfqpoint{2.865900in}{1.367681in}}%
\pgfpathlineto{\pgfqpoint{2.911911in}{1.367681in}}%
\pgfpathlineto{\pgfqpoint{2.957922in}{1.367681in}}%
\pgfpathlineto{\pgfqpoint{3.003933in}{1.367681in}}%
\pgfpathlineto{\pgfqpoint{3.049944in}{1.367681in}}%
\pgfpathlineto{\pgfqpoint{3.095955in}{1.367681in}}%
\pgfpathlineto{\pgfqpoint{3.141967in}{1.367681in}}%
\pgfpathlineto{\pgfqpoint{3.187978in}{1.367681in}}%
\pgfpathlineto{\pgfqpoint{3.233989in}{1.367681in}}%
\pgfpathlineto{\pgfqpoint{3.280000in}{1.367681in}}%
\pgfpathlineto{\pgfqpoint{3.326011in}{1.367681in}}%
\pgfpathlineto{\pgfqpoint{3.372022in}{1.567878in}}%
\pgfpathlineto{\pgfqpoint{3.418033in}{1.435199in}}%
\pgfpathlineto{\pgfqpoint{3.464045in}{1.435199in}}%
\pgfpathlineto{\pgfqpoint{3.510056in}{1.435199in}}%
\pgfpathlineto{\pgfqpoint{3.556067in}{1.435199in}}%
\pgfpathlineto{\pgfqpoint{3.602078in}{1.435199in}}%
\pgfpathlineto{\pgfqpoint{3.648089in}{1.435199in}}%
\pgfpathlineto{\pgfqpoint{3.694100in}{1.435199in}}%
\pgfpathlineto{\pgfqpoint{3.740111in}{1.435199in}}%
\pgfpathlineto{\pgfqpoint{3.786122in}{1.435199in}}%
\pgfpathlineto{\pgfqpoint{3.832134in}{1.435199in}}%
\pgfpathlineto{\pgfqpoint{3.878145in}{1.435199in}}%
\pgfpathlineto{\pgfqpoint{3.924156in}{1.435199in}}%
\pgfpathlineto{\pgfqpoint{3.970167in}{1.435199in}}%
\pgfpathlineto{\pgfqpoint{4.016178in}{1.435199in}}%
\pgfpathlineto{\pgfqpoint{4.062189in}{1.435199in}}%
\pgfpathlineto{\pgfqpoint{4.108200in}{1.435199in}}%
\pgfpathlineto{\pgfqpoint{4.154212in}{1.435199in}}%
\pgfpathlineto{\pgfqpoint{4.200223in}{1.435199in}}%
\pgfpathlineto{\pgfqpoint{4.246234in}{1.435199in}}%
\pgfpathlineto{\pgfqpoint{4.292245in}{1.435199in}}%
\pgfpathlineto{\pgfqpoint{4.338256in}{1.435199in}}%
\pgfpathlineto{\pgfqpoint{4.384267in}{1.435199in}}%
\pgfpathlineto{\pgfqpoint{4.430278in}{1.435199in}}%
\pgfpathlineto{\pgfqpoint{4.476289in}{1.435199in}}%
\pgfpathlineto{\pgfqpoint{4.522301in}{1.435199in}}%
\pgfpathlineto{\pgfqpoint{4.568312in}{1.444084in}}%
\pgfpathlineto{\pgfqpoint{4.614323in}{1.444084in}}%
\pgfpathlineto{\pgfqpoint{4.660334in}{1.444084in}}%
\pgfpathlineto{\pgfqpoint{4.706345in}{1.444084in}}%
\pgfpathlineto{\pgfqpoint{4.752356in}{1.444084in}}%
\pgfpathlineto{\pgfqpoint{4.798367in}{1.444084in}}%
\pgfpathlineto{\pgfqpoint{4.844378in}{1.444084in}}%
\pgfpathlineto{\pgfqpoint{4.890390in}{1.444084in}}%
\pgfpathlineto{\pgfqpoint{4.936401in}{1.444084in}}%
\pgfpathlineto{\pgfqpoint{4.982412in}{1.444084in}}%
\pgfpathlineto{\pgfqpoint{5.028423in}{1.444084in}}%
\pgfpathlineto{\pgfqpoint{5.074434in}{1.444084in}}%
\pgfpathlineto{\pgfqpoint{5.120445in}{1.444084in}}%
\pgfpathlineto{\pgfqpoint{5.166456in}{1.444084in}}%
\pgfpathlineto{\pgfqpoint{5.212468in}{1.444084in}}%
\pgfpathlineto{\pgfqpoint{5.258479in}{1.444084in}}%
\pgfpathlineto{\pgfqpoint{5.304490in}{1.444084in}}%
\pgfpathlineto{\pgfqpoint{5.350501in}{1.444084in}}%
\pgfpathlineto{\pgfqpoint{5.396512in}{1.444084in}}%
\pgfpathlineto{\pgfqpoint{5.442523in}{1.444084in}}%
\pgfpathlineto{\pgfqpoint{5.488534in}{1.444084in}}%
\pgfpathlineto{\pgfqpoint{5.534545in}{1.444084in}}%
\pgfusepath{stroke}%
\end{pgfscope}%
\begin{pgfscope}%
\pgfsetrectcap%
\pgfsetmiterjoin%
\pgfsetlinewidth{0.803000pt}%
\definecolor{currentstroke}{rgb}{0.000000,0.000000,0.000000}%
\pgfsetstrokecolor{currentstroke}%
\pgfsetdash{}{0pt}%
\pgfpathmoveto{\pgfqpoint{0.800000in}{0.528000in}}%
\pgfpathlineto{\pgfqpoint{0.800000in}{4.224000in}}%
\pgfusepath{stroke}%
\end{pgfscope}%
\begin{pgfscope}%
\pgfsetrectcap%
\pgfsetmiterjoin%
\pgfsetlinewidth{0.803000pt}%
\definecolor{currentstroke}{rgb}{0.000000,0.000000,0.000000}%
\pgfsetstrokecolor{currentstroke}%
\pgfsetdash{}{0pt}%
\pgfpathmoveto{\pgfqpoint{5.760000in}{0.528000in}}%
\pgfpathlineto{\pgfqpoint{5.760000in}{4.224000in}}%
\pgfusepath{stroke}%
\end{pgfscope}%
\begin{pgfscope}%
\pgfsetrectcap%
\pgfsetmiterjoin%
\pgfsetlinewidth{0.803000pt}%
\definecolor{currentstroke}{rgb}{0.000000,0.000000,0.000000}%
\pgfsetstrokecolor{currentstroke}%
\pgfsetdash{}{0pt}%
\pgfpathmoveto{\pgfqpoint{0.800000in}{0.528000in}}%
\pgfpathlineto{\pgfqpoint{5.760000in}{0.528000in}}%
\pgfusepath{stroke}%
\end{pgfscope}%
\begin{pgfscope}%
\pgfsetrectcap%
\pgfsetmiterjoin%
\pgfsetlinewidth{0.803000pt}%
\definecolor{currentstroke}{rgb}{0.000000,0.000000,0.000000}%
\pgfsetstrokecolor{currentstroke}%
\pgfsetdash{}{0pt}%
\pgfpathmoveto{\pgfqpoint{0.800000in}{4.224000in}}%
\pgfpathlineto{\pgfqpoint{5.760000in}{4.224000in}}%
\pgfusepath{stroke}%
\end{pgfscope}%
\begin{pgfscope}%
\pgfsetbuttcap%
\pgfsetmiterjoin%
\definecolor{currentfill}{rgb}{1.000000,1.000000,1.000000}%
\pgfsetfillcolor{currentfill}%
\pgfsetfillopacity{0.800000}%
\pgfsetlinewidth{1.003750pt}%
\definecolor{currentstroke}{rgb}{0.800000,0.800000,0.800000}%
\pgfsetstrokecolor{currentstroke}%
\pgfsetstrokeopacity{0.800000}%
\pgfsetdash{}{0pt}%
\pgfpathmoveto{\pgfqpoint{0.897222in}{3.725543in}}%
\pgfpathlineto{\pgfqpoint{1.975542in}{3.725543in}}%
\pgfpathquadraticcurveto{\pgfqpoint{2.003319in}{3.725543in}}{\pgfqpoint{2.003319in}{3.753321in}}%
\pgfpathlineto{\pgfqpoint{2.003319in}{4.126778in}}%
\pgfpathquadraticcurveto{\pgfqpoint{2.003319in}{4.154556in}}{\pgfqpoint{1.975542in}{4.154556in}}%
\pgfpathlineto{\pgfqpoint{0.897222in}{4.154556in}}%
\pgfpathquadraticcurveto{\pgfqpoint{0.869444in}{4.154556in}}{\pgfqpoint{0.869444in}{4.126778in}}%
\pgfpathlineto{\pgfqpoint{0.869444in}{3.753321in}}%
\pgfpathquadraticcurveto{\pgfqpoint{0.869444in}{3.725543in}}{\pgfqpoint{0.897222in}{3.725543in}}%
\pgfpathlineto{\pgfqpoint{0.897222in}{3.725543in}}%
\pgfpathclose%
\pgfusepath{stroke,fill}%
\end{pgfscope}%
\begin{pgfscope}%
\pgfsetrectcap%
\pgfsetroundjoin%
\pgfsetlinewidth{1.505625pt}%
\definecolor{currentstroke}{rgb}{0.121569,0.466667,0.705882}%
\pgfsetstrokecolor{currentstroke}%
\pgfsetdash{}{0pt}%
\pgfpathmoveto{\pgfqpoint{0.925000in}{4.050389in}}%
\pgfpathlineto{\pgfqpoint{1.063889in}{4.050389in}}%
\pgfpathlineto{\pgfqpoint{1.202778in}{4.050389in}}%
\pgfusepath{stroke}%
\end{pgfscope}%
\begin{pgfscope}%
\definecolor{textcolor}{rgb}{0.000000,0.000000,0.000000}%
\pgfsetstrokecolor{textcolor}%
\pgfsetfillcolor{textcolor}%
\pgftext[x=1.313889in,y=4.001778in,left,base]{\color{textcolor}\rmfamily\fontsize{10.000000}{12.000000}\selectfont quicksort}%
\end{pgfscope}%
\begin{pgfscope}%
\pgfsetrectcap%
\pgfsetroundjoin%
\pgfsetlinewidth{1.505625pt}%
\definecolor{currentstroke}{rgb}{1.000000,0.498039,0.054902}%
\pgfsetstrokecolor{currentstroke}%
\pgfsetdash{}{0pt}%
\pgfpathmoveto{\pgfqpoint{0.925000in}{3.856716in}}%
\pgfpathlineto{\pgfqpoint{1.063889in}{3.856716in}}%
\pgfpathlineto{\pgfqpoint{1.202778in}{3.856716in}}%
\pgfusepath{stroke}%
\end{pgfscope}%
\begin{pgfscope}%
\definecolor{textcolor}{rgb}{0.000000,0.000000,0.000000}%
\pgfsetstrokecolor{textcolor}%
\pgfsetfillcolor{textcolor}%
\pgftext[x=1.313889in,y=3.808105in,left,base]{\color{textcolor}\rmfamily\fontsize{10.000000}{12.000000}\selectfont bquicksort}%
\end{pgfscope}%
\end{pgfpicture}%
\makeatother%
\endgroup%

\subsection{Analysis}
As we can see the time taken has \textit{decreased significantly}.
The number of comparison, swaps and basic operations has also shown a
\textit{slight decrease}. The decrease in memory is probably due to
\textit{lesser number of recursion calls}.
\section{Optimizing Merge Sort}
\subsection{Problem}
The main problem with merge sort is not the time complexity as it is one of the
fastest but the \textit{space overhead}. Implemented in its most basic form
it requires $O(n)$ space to create \textit{subarrays} each recursion.
\subsection{Solution}
The solution implemented here is from
\textit{Katajainen, Jyrki; Pasanen, Tomi; Teuhola, Jukka (1996)}.
Instead of creating subarray we pass the original array with index ranges.
The only time subarrays are created is in the \texttt{merge()} routine.
Here \texttt{merge()} creates a copy of the smaller subarray.
\subsection{Plots}
\subsubsection{Comparisons}
%% Creator: Matplotlib, PGF backend
%%
%% To include the figure in your LaTeX document, write
%%   \input{<filename>.pgf}
%%
%% Make sure the required packages are loaded in your preamble
%%   \usepackage{pgf}
%%
%% Also ensure that all the required font packages are loaded; for instance,
%% the lmodern package is sometimes necessary when using math font.
%%   \usepackage{lmodern}
%%
%% Figures using additional raster images can only be included by \input if
%% they are in the same directory as the main LaTeX file. For loading figures
%% from other directories you can use the `import` package
%%   \usepackage{import}
%%
%% and then include the figures with
%%   \import{<path to file>}{<filename>.pgf}
%%
%% Matplotlib used the following preamble
%%   
%%   \makeatletter\@ifpackageloaded{underscore}{}{\usepackage[strings]{underscore}}\makeatother
%%
\begingroup%
\makeatletter%
\begin{pgfpicture}%
\pgfpathrectangle{\pgfpointorigin}{\pgfqpoint{6.400000in}{4.800000in}}%
\pgfusepath{use as bounding box, clip}%
\begin{pgfscope}%
\pgfsetbuttcap%
\pgfsetmiterjoin%
\definecolor{currentfill}{rgb}{1.000000,1.000000,1.000000}%
\pgfsetfillcolor{currentfill}%
\pgfsetlinewidth{0.000000pt}%
\definecolor{currentstroke}{rgb}{1.000000,1.000000,1.000000}%
\pgfsetstrokecolor{currentstroke}%
\pgfsetdash{}{0pt}%
\pgfpathmoveto{\pgfqpoint{0.000000in}{0.000000in}}%
\pgfpathlineto{\pgfqpoint{6.400000in}{0.000000in}}%
\pgfpathlineto{\pgfqpoint{6.400000in}{4.800000in}}%
\pgfpathlineto{\pgfqpoint{0.000000in}{4.800000in}}%
\pgfpathlineto{\pgfqpoint{0.000000in}{0.000000in}}%
\pgfpathclose%
\pgfusepath{fill}%
\end{pgfscope}%
\begin{pgfscope}%
\pgfsetbuttcap%
\pgfsetmiterjoin%
\definecolor{currentfill}{rgb}{1.000000,1.000000,1.000000}%
\pgfsetfillcolor{currentfill}%
\pgfsetlinewidth{0.000000pt}%
\definecolor{currentstroke}{rgb}{0.000000,0.000000,0.000000}%
\pgfsetstrokecolor{currentstroke}%
\pgfsetstrokeopacity{0.000000}%
\pgfsetdash{}{0pt}%
\pgfpathmoveto{\pgfqpoint{0.800000in}{0.528000in}}%
\pgfpathlineto{\pgfqpoint{5.760000in}{0.528000in}}%
\pgfpathlineto{\pgfqpoint{5.760000in}{4.224000in}}%
\pgfpathlineto{\pgfqpoint{0.800000in}{4.224000in}}%
\pgfpathlineto{\pgfqpoint{0.800000in}{0.528000in}}%
\pgfpathclose%
\pgfusepath{fill}%
\end{pgfscope}%
\begin{pgfscope}%
\pgfsetbuttcap%
\pgfsetroundjoin%
\definecolor{currentfill}{rgb}{0.000000,0.000000,0.000000}%
\pgfsetfillcolor{currentfill}%
\pgfsetlinewidth{0.803000pt}%
\definecolor{currentstroke}{rgb}{0.000000,0.000000,0.000000}%
\pgfsetstrokecolor{currentstroke}%
\pgfsetdash{}{0pt}%
\pgfsys@defobject{currentmarker}{\pgfqpoint{0.000000in}{-0.048611in}}{\pgfqpoint{0.000000in}{0.000000in}}{%
\pgfpathmoveto{\pgfqpoint{0.000000in}{0.000000in}}%
\pgfpathlineto{\pgfqpoint{0.000000in}{-0.048611in}}%
\pgfusepath{stroke,fill}%
}%
\begin{pgfscope}%
\pgfsys@transformshift{0.979443in}{0.528000in}%
\pgfsys@useobject{currentmarker}{}%
\end{pgfscope}%
\end{pgfscope}%
\begin{pgfscope}%
\definecolor{textcolor}{rgb}{0.000000,0.000000,0.000000}%
\pgfsetstrokecolor{textcolor}%
\pgfsetfillcolor{textcolor}%
\pgftext[x=0.979443in,y=0.430778in,,top]{\color{textcolor}\rmfamily\fontsize{10.000000}{12.000000}\selectfont \(\displaystyle {0}\)}%
\end{pgfscope}%
\begin{pgfscope}%
\pgfsetbuttcap%
\pgfsetroundjoin%
\definecolor{currentfill}{rgb}{0.000000,0.000000,0.000000}%
\pgfsetfillcolor{currentfill}%
\pgfsetlinewidth{0.803000pt}%
\definecolor{currentstroke}{rgb}{0.000000,0.000000,0.000000}%
\pgfsetstrokecolor{currentstroke}%
\pgfsetdash{}{0pt}%
\pgfsys@defobject{currentmarker}{\pgfqpoint{0.000000in}{-0.048611in}}{\pgfqpoint{0.000000in}{0.000000in}}{%
\pgfpathmoveto{\pgfqpoint{0.000000in}{0.000000in}}%
\pgfpathlineto{\pgfqpoint{0.000000in}{-0.048611in}}%
\pgfusepath{stroke,fill}%
}%
\begin{pgfscope}%
\pgfsys@transformshift{1.899666in}{0.528000in}%
\pgfsys@useobject{currentmarker}{}%
\end{pgfscope}%
\end{pgfscope}%
\begin{pgfscope}%
\definecolor{textcolor}{rgb}{0.000000,0.000000,0.000000}%
\pgfsetstrokecolor{textcolor}%
\pgfsetfillcolor{textcolor}%
\pgftext[x=1.899666in,y=0.430778in,,top]{\color{textcolor}\rmfamily\fontsize{10.000000}{12.000000}\selectfont \(\displaystyle {2000}\)}%
\end{pgfscope}%
\begin{pgfscope}%
\pgfsetbuttcap%
\pgfsetroundjoin%
\definecolor{currentfill}{rgb}{0.000000,0.000000,0.000000}%
\pgfsetfillcolor{currentfill}%
\pgfsetlinewidth{0.803000pt}%
\definecolor{currentstroke}{rgb}{0.000000,0.000000,0.000000}%
\pgfsetstrokecolor{currentstroke}%
\pgfsetdash{}{0pt}%
\pgfsys@defobject{currentmarker}{\pgfqpoint{0.000000in}{-0.048611in}}{\pgfqpoint{0.000000in}{0.000000in}}{%
\pgfpathmoveto{\pgfqpoint{0.000000in}{0.000000in}}%
\pgfpathlineto{\pgfqpoint{0.000000in}{-0.048611in}}%
\pgfusepath{stroke,fill}%
}%
\begin{pgfscope}%
\pgfsys@transformshift{2.819889in}{0.528000in}%
\pgfsys@useobject{currentmarker}{}%
\end{pgfscope}%
\end{pgfscope}%
\begin{pgfscope}%
\definecolor{textcolor}{rgb}{0.000000,0.000000,0.000000}%
\pgfsetstrokecolor{textcolor}%
\pgfsetfillcolor{textcolor}%
\pgftext[x=2.819889in,y=0.430778in,,top]{\color{textcolor}\rmfamily\fontsize{10.000000}{12.000000}\selectfont \(\displaystyle {4000}\)}%
\end{pgfscope}%
\begin{pgfscope}%
\pgfsetbuttcap%
\pgfsetroundjoin%
\definecolor{currentfill}{rgb}{0.000000,0.000000,0.000000}%
\pgfsetfillcolor{currentfill}%
\pgfsetlinewidth{0.803000pt}%
\definecolor{currentstroke}{rgb}{0.000000,0.000000,0.000000}%
\pgfsetstrokecolor{currentstroke}%
\pgfsetdash{}{0pt}%
\pgfsys@defobject{currentmarker}{\pgfqpoint{0.000000in}{-0.048611in}}{\pgfqpoint{0.000000in}{0.000000in}}{%
\pgfpathmoveto{\pgfqpoint{0.000000in}{0.000000in}}%
\pgfpathlineto{\pgfqpoint{0.000000in}{-0.048611in}}%
\pgfusepath{stroke,fill}%
}%
\begin{pgfscope}%
\pgfsys@transformshift{3.740111in}{0.528000in}%
\pgfsys@useobject{currentmarker}{}%
\end{pgfscope}%
\end{pgfscope}%
\begin{pgfscope}%
\definecolor{textcolor}{rgb}{0.000000,0.000000,0.000000}%
\pgfsetstrokecolor{textcolor}%
\pgfsetfillcolor{textcolor}%
\pgftext[x=3.740111in,y=0.430778in,,top]{\color{textcolor}\rmfamily\fontsize{10.000000}{12.000000}\selectfont \(\displaystyle {6000}\)}%
\end{pgfscope}%
\begin{pgfscope}%
\pgfsetbuttcap%
\pgfsetroundjoin%
\definecolor{currentfill}{rgb}{0.000000,0.000000,0.000000}%
\pgfsetfillcolor{currentfill}%
\pgfsetlinewidth{0.803000pt}%
\definecolor{currentstroke}{rgb}{0.000000,0.000000,0.000000}%
\pgfsetstrokecolor{currentstroke}%
\pgfsetdash{}{0pt}%
\pgfsys@defobject{currentmarker}{\pgfqpoint{0.000000in}{-0.048611in}}{\pgfqpoint{0.000000in}{0.000000in}}{%
\pgfpathmoveto{\pgfqpoint{0.000000in}{0.000000in}}%
\pgfpathlineto{\pgfqpoint{0.000000in}{-0.048611in}}%
\pgfusepath{stroke,fill}%
}%
\begin{pgfscope}%
\pgfsys@transformshift{4.660334in}{0.528000in}%
\pgfsys@useobject{currentmarker}{}%
\end{pgfscope}%
\end{pgfscope}%
\begin{pgfscope}%
\definecolor{textcolor}{rgb}{0.000000,0.000000,0.000000}%
\pgfsetstrokecolor{textcolor}%
\pgfsetfillcolor{textcolor}%
\pgftext[x=4.660334in,y=0.430778in,,top]{\color{textcolor}\rmfamily\fontsize{10.000000}{12.000000}\selectfont \(\displaystyle {8000}\)}%
\end{pgfscope}%
\begin{pgfscope}%
\pgfsetbuttcap%
\pgfsetroundjoin%
\definecolor{currentfill}{rgb}{0.000000,0.000000,0.000000}%
\pgfsetfillcolor{currentfill}%
\pgfsetlinewidth{0.803000pt}%
\definecolor{currentstroke}{rgb}{0.000000,0.000000,0.000000}%
\pgfsetstrokecolor{currentstroke}%
\pgfsetdash{}{0pt}%
\pgfsys@defobject{currentmarker}{\pgfqpoint{0.000000in}{-0.048611in}}{\pgfqpoint{0.000000in}{0.000000in}}{%
\pgfpathmoveto{\pgfqpoint{0.000000in}{0.000000in}}%
\pgfpathlineto{\pgfqpoint{0.000000in}{-0.048611in}}%
\pgfusepath{stroke,fill}%
}%
\begin{pgfscope}%
\pgfsys@transformshift{5.580557in}{0.528000in}%
\pgfsys@useobject{currentmarker}{}%
\end{pgfscope}%
\end{pgfscope}%
\begin{pgfscope}%
\definecolor{textcolor}{rgb}{0.000000,0.000000,0.000000}%
\pgfsetstrokecolor{textcolor}%
\pgfsetfillcolor{textcolor}%
\pgftext[x=5.580557in,y=0.430778in,,top]{\color{textcolor}\rmfamily\fontsize{10.000000}{12.000000}\selectfont \(\displaystyle {10000}\)}%
\end{pgfscope}%
\begin{pgfscope}%
\definecolor{textcolor}{rgb}{0.000000,0.000000,0.000000}%
\pgfsetstrokecolor{textcolor}%
\pgfsetfillcolor{textcolor}%
\pgftext[x=3.280000in,y=0.251766in,,top]{\color{textcolor}\rmfamily\fontsize{10.000000}{12.000000}\selectfont Input Size}%
\end{pgfscope}%
\begin{pgfscope}%
\pgfsetbuttcap%
\pgfsetroundjoin%
\definecolor{currentfill}{rgb}{0.000000,0.000000,0.000000}%
\pgfsetfillcolor{currentfill}%
\pgfsetlinewidth{0.803000pt}%
\definecolor{currentstroke}{rgb}{0.000000,0.000000,0.000000}%
\pgfsetstrokecolor{currentstroke}%
\pgfsetdash{}{0pt}%
\pgfsys@defobject{currentmarker}{\pgfqpoint{-0.048611in}{0.000000in}}{\pgfqpoint{-0.000000in}{0.000000in}}{%
\pgfpathmoveto{\pgfqpoint{-0.000000in}{0.000000in}}%
\pgfpathlineto{\pgfqpoint{-0.048611in}{0.000000in}}%
\pgfusepath{stroke,fill}%
}%
\begin{pgfscope}%
\pgfsys@transformshift{0.800000in}{1.733236in}%
\pgfsys@useobject{currentmarker}{}%
\end{pgfscope}%
\end{pgfscope}%
\begin{pgfscope}%
\definecolor{textcolor}{rgb}{0.000000,0.000000,0.000000}%
\pgfsetstrokecolor{textcolor}%
\pgfsetfillcolor{textcolor}%
\pgftext[x=0.501581in, y=1.685010in, left, base]{\color{textcolor}\rmfamily\fontsize{10.000000}{12.000000}\selectfont \(\displaystyle {10^{4}}\)}%
\end{pgfscope}%
\begin{pgfscope}%
\pgfsetbuttcap%
\pgfsetroundjoin%
\definecolor{currentfill}{rgb}{0.000000,0.000000,0.000000}%
\pgfsetfillcolor{currentfill}%
\pgfsetlinewidth{0.803000pt}%
\definecolor{currentstroke}{rgb}{0.000000,0.000000,0.000000}%
\pgfsetstrokecolor{currentstroke}%
\pgfsetdash{}{0pt}%
\pgfsys@defobject{currentmarker}{\pgfqpoint{-0.048611in}{0.000000in}}{\pgfqpoint{-0.000000in}{0.000000in}}{%
\pgfpathmoveto{\pgfqpoint{-0.000000in}{0.000000in}}%
\pgfpathlineto{\pgfqpoint{-0.048611in}{0.000000in}}%
\pgfusepath{stroke,fill}%
}%
\begin{pgfscope}%
\pgfsys@transformshift{0.800000in}{3.192827in}%
\pgfsys@useobject{currentmarker}{}%
\end{pgfscope}%
\end{pgfscope}%
\begin{pgfscope}%
\definecolor{textcolor}{rgb}{0.000000,0.000000,0.000000}%
\pgfsetstrokecolor{textcolor}%
\pgfsetfillcolor{textcolor}%
\pgftext[x=0.501581in, y=3.144602in, left, base]{\color{textcolor}\rmfamily\fontsize{10.000000}{12.000000}\selectfont \(\displaystyle {10^{5}}\)}%
\end{pgfscope}%
\begin{pgfscope}%
\pgfsetbuttcap%
\pgfsetroundjoin%
\definecolor{currentfill}{rgb}{0.000000,0.000000,0.000000}%
\pgfsetfillcolor{currentfill}%
\pgfsetlinewidth{0.602250pt}%
\definecolor{currentstroke}{rgb}{0.000000,0.000000,0.000000}%
\pgfsetstrokecolor{currentstroke}%
\pgfsetdash{}{0pt}%
\pgfsys@defobject{currentmarker}{\pgfqpoint{-0.027778in}{0.000000in}}{\pgfqpoint{-0.000000in}{0.000000in}}{%
\pgfpathmoveto{\pgfqpoint{-0.000000in}{0.000000in}}%
\pgfpathlineto{\pgfqpoint{-0.027778in}{0.000000in}}%
\pgfusepath{stroke,fill}%
}%
\begin{pgfscope}%
\pgfsys@transformshift{0.800000in}{0.713025in}%
\pgfsys@useobject{currentmarker}{}%
\end{pgfscope}%
\end{pgfscope}%
\begin{pgfscope}%
\pgfsetbuttcap%
\pgfsetroundjoin%
\definecolor{currentfill}{rgb}{0.000000,0.000000,0.000000}%
\pgfsetfillcolor{currentfill}%
\pgfsetlinewidth{0.602250pt}%
\definecolor{currentstroke}{rgb}{0.000000,0.000000,0.000000}%
\pgfsetstrokecolor{currentstroke}%
\pgfsetdash{}{0pt}%
\pgfsys@defobject{currentmarker}{\pgfqpoint{-0.027778in}{0.000000in}}{\pgfqpoint{-0.000000in}{0.000000in}}{%
\pgfpathmoveto{\pgfqpoint{-0.000000in}{0.000000in}}%
\pgfpathlineto{\pgfqpoint{-0.027778in}{0.000000in}}%
\pgfusepath{stroke,fill}%
}%
\begin{pgfscope}%
\pgfsys@transformshift{0.800000in}{0.970046in}%
\pgfsys@useobject{currentmarker}{}%
\end{pgfscope}%
\end{pgfscope}%
\begin{pgfscope}%
\pgfsetbuttcap%
\pgfsetroundjoin%
\definecolor{currentfill}{rgb}{0.000000,0.000000,0.000000}%
\pgfsetfillcolor{currentfill}%
\pgfsetlinewidth{0.602250pt}%
\definecolor{currentstroke}{rgb}{0.000000,0.000000,0.000000}%
\pgfsetstrokecolor{currentstroke}%
\pgfsetdash{}{0pt}%
\pgfsys@defobject{currentmarker}{\pgfqpoint{-0.027778in}{0.000000in}}{\pgfqpoint{-0.000000in}{0.000000in}}{%
\pgfpathmoveto{\pgfqpoint{-0.000000in}{0.000000in}}%
\pgfpathlineto{\pgfqpoint{-0.027778in}{0.000000in}}%
\pgfusepath{stroke,fill}%
}%
\begin{pgfscope}%
\pgfsys@transformshift{0.800000in}{1.152406in}%
\pgfsys@useobject{currentmarker}{}%
\end{pgfscope}%
\end{pgfscope}%
\begin{pgfscope}%
\pgfsetbuttcap%
\pgfsetroundjoin%
\definecolor{currentfill}{rgb}{0.000000,0.000000,0.000000}%
\pgfsetfillcolor{currentfill}%
\pgfsetlinewidth{0.602250pt}%
\definecolor{currentstroke}{rgb}{0.000000,0.000000,0.000000}%
\pgfsetstrokecolor{currentstroke}%
\pgfsetdash{}{0pt}%
\pgfsys@defobject{currentmarker}{\pgfqpoint{-0.027778in}{0.000000in}}{\pgfqpoint{-0.000000in}{0.000000in}}{%
\pgfpathmoveto{\pgfqpoint{-0.000000in}{0.000000in}}%
\pgfpathlineto{\pgfqpoint{-0.027778in}{0.000000in}}%
\pgfusepath{stroke,fill}%
}%
\begin{pgfscope}%
\pgfsys@transformshift{0.800000in}{1.293855in}%
\pgfsys@useobject{currentmarker}{}%
\end{pgfscope}%
\end{pgfscope}%
\begin{pgfscope}%
\pgfsetbuttcap%
\pgfsetroundjoin%
\definecolor{currentfill}{rgb}{0.000000,0.000000,0.000000}%
\pgfsetfillcolor{currentfill}%
\pgfsetlinewidth{0.602250pt}%
\definecolor{currentstroke}{rgb}{0.000000,0.000000,0.000000}%
\pgfsetstrokecolor{currentstroke}%
\pgfsetdash{}{0pt}%
\pgfsys@defobject{currentmarker}{\pgfqpoint{-0.027778in}{0.000000in}}{\pgfqpoint{-0.000000in}{0.000000in}}{%
\pgfpathmoveto{\pgfqpoint{-0.000000in}{0.000000in}}%
\pgfpathlineto{\pgfqpoint{-0.027778in}{0.000000in}}%
\pgfusepath{stroke,fill}%
}%
\begin{pgfscope}%
\pgfsys@transformshift{0.800000in}{1.409427in}%
\pgfsys@useobject{currentmarker}{}%
\end{pgfscope}%
\end{pgfscope}%
\begin{pgfscope}%
\pgfsetbuttcap%
\pgfsetroundjoin%
\definecolor{currentfill}{rgb}{0.000000,0.000000,0.000000}%
\pgfsetfillcolor{currentfill}%
\pgfsetlinewidth{0.602250pt}%
\definecolor{currentstroke}{rgb}{0.000000,0.000000,0.000000}%
\pgfsetstrokecolor{currentstroke}%
\pgfsetdash{}{0pt}%
\pgfsys@defobject{currentmarker}{\pgfqpoint{-0.027778in}{0.000000in}}{\pgfqpoint{-0.000000in}{0.000000in}}{%
\pgfpathmoveto{\pgfqpoint{-0.000000in}{0.000000in}}%
\pgfpathlineto{\pgfqpoint{-0.027778in}{0.000000in}}%
\pgfusepath{stroke,fill}%
}%
\begin{pgfscope}%
\pgfsys@transformshift{0.800000in}{1.507142in}%
\pgfsys@useobject{currentmarker}{}%
\end{pgfscope}%
\end{pgfscope}%
\begin{pgfscope}%
\pgfsetbuttcap%
\pgfsetroundjoin%
\definecolor{currentfill}{rgb}{0.000000,0.000000,0.000000}%
\pgfsetfillcolor{currentfill}%
\pgfsetlinewidth{0.602250pt}%
\definecolor{currentstroke}{rgb}{0.000000,0.000000,0.000000}%
\pgfsetstrokecolor{currentstroke}%
\pgfsetdash{}{0pt}%
\pgfsys@defobject{currentmarker}{\pgfqpoint{-0.027778in}{0.000000in}}{\pgfqpoint{-0.000000in}{0.000000in}}{%
\pgfpathmoveto{\pgfqpoint{-0.000000in}{0.000000in}}%
\pgfpathlineto{\pgfqpoint{-0.027778in}{0.000000in}}%
\pgfusepath{stroke,fill}%
}%
\begin{pgfscope}%
\pgfsys@transformshift{0.800000in}{1.591786in}%
\pgfsys@useobject{currentmarker}{}%
\end{pgfscope}%
\end{pgfscope}%
\begin{pgfscope}%
\pgfsetbuttcap%
\pgfsetroundjoin%
\definecolor{currentfill}{rgb}{0.000000,0.000000,0.000000}%
\pgfsetfillcolor{currentfill}%
\pgfsetlinewidth{0.602250pt}%
\definecolor{currentstroke}{rgb}{0.000000,0.000000,0.000000}%
\pgfsetstrokecolor{currentstroke}%
\pgfsetdash{}{0pt}%
\pgfsys@defobject{currentmarker}{\pgfqpoint{-0.027778in}{0.000000in}}{\pgfqpoint{-0.000000in}{0.000000in}}{%
\pgfpathmoveto{\pgfqpoint{-0.000000in}{0.000000in}}%
\pgfpathlineto{\pgfqpoint{-0.027778in}{0.000000in}}%
\pgfusepath{stroke,fill}%
}%
\begin{pgfscope}%
\pgfsys@transformshift{0.800000in}{1.666448in}%
\pgfsys@useobject{currentmarker}{}%
\end{pgfscope}%
\end{pgfscope}%
\begin{pgfscope}%
\pgfsetbuttcap%
\pgfsetroundjoin%
\definecolor{currentfill}{rgb}{0.000000,0.000000,0.000000}%
\pgfsetfillcolor{currentfill}%
\pgfsetlinewidth{0.602250pt}%
\definecolor{currentstroke}{rgb}{0.000000,0.000000,0.000000}%
\pgfsetstrokecolor{currentstroke}%
\pgfsetdash{}{0pt}%
\pgfsys@defobject{currentmarker}{\pgfqpoint{-0.027778in}{0.000000in}}{\pgfqpoint{-0.000000in}{0.000000in}}{%
\pgfpathmoveto{\pgfqpoint{-0.000000in}{0.000000in}}%
\pgfpathlineto{\pgfqpoint{-0.027778in}{0.000000in}}%
\pgfusepath{stroke,fill}%
}%
\begin{pgfscope}%
\pgfsys@transformshift{0.800000in}{2.172616in}%
\pgfsys@useobject{currentmarker}{}%
\end{pgfscope}%
\end{pgfscope}%
\begin{pgfscope}%
\pgfsetbuttcap%
\pgfsetroundjoin%
\definecolor{currentfill}{rgb}{0.000000,0.000000,0.000000}%
\pgfsetfillcolor{currentfill}%
\pgfsetlinewidth{0.602250pt}%
\definecolor{currentstroke}{rgb}{0.000000,0.000000,0.000000}%
\pgfsetstrokecolor{currentstroke}%
\pgfsetdash{}{0pt}%
\pgfsys@defobject{currentmarker}{\pgfqpoint{-0.027778in}{0.000000in}}{\pgfqpoint{-0.000000in}{0.000000in}}{%
\pgfpathmoveto{\pgfqpoint{-0.000000in}{0.000000in}}%
\pgfpathlineto{\pgfqpoint{-0.027778in}{0.000000in}}%
\pgfusepath{stroke,fill}%
}%
\begin{pgfscope}%
\pgfsys@transformshift{0.800000in}{2.429638in}%
\pgfsys@useobject{currentmarker}{}%
\end{pgfscope}%
\end{pgfscope}%
\begin{pgfscope}%
\pgfsetbuttcap%
\pgfsetroundjoin%
\definecolor{currentfill}{rgb}{0.000000,0.000000,0.000000}%
\pgfsetfillcolor{currentfill}%
\pgfsetlinewidth{0.602250pt}%
\definecolor{currentstroke}{rgb}{0.000000,0.000000,0.000000}%
\pgfsetstrokecolor{currentstroke}%
\pgfsetdash{}{0pt}%
\pgfsys@defobject{currentmarker}{\pgfqpoint{-0.027778in}{0.000000in}}{\pgfqpoint{-0.000000in}{0.000000in}}{%
\pgfpathmoveto{\pgfqpoint{-0.000000in}{0.000000in}}%
\pgfpathlineto{\pgfqpoint{-0.027778in}{0.000000in}}%
\pgfusepath{stroke,fill}%
}%
\begin{pgfscope}%
\pgfsys@transformshift{0.800000in}{2.611997in}%
\pgfsys@useobject{currentmarker}{}%
\end{pgfscope}%
\end{pgfscope}%
\begin{pgfscope}%
\pgfsetbuttcap%
\pgfsetroundjoin%
\definecolor{currentfill}{rgb}{0.000000,0.000000,0.000000}%
\pgfsetfillcolor{currentfill}%
\pgfsetlinewidth{0.602250pt}%
\definecolor{currentstroke}{rgb}{0.000000,0.000000,0.000000}%
\pgfsetstrokecolor{currentstroke}%
\pgfsetdash{}{0pt}%
\pgfsys@defobject{currentmarker}{\pgfqpoint{-0.027778in}{0.000000in}}{\pgfqpoint{-0.000000in}{0.000000in}}{%
\pgfpathmoveto{\pgfqpoint{-0.000000in}{0.000000in}}%
\pgfpathlineto{\pgfqpoint{-0.027778in}{0.000000in}}%
\pgfusepath{stroke,fill}%
}%
\begin{pgfscope}%
\pgfsys@transformshift{0.800000in}{2.753446in}%
\pgfsys@useobject{currentmarker}{}%
\end{pgfscope}%
\end{pgfscope}%
\begin{pgfscope}%
\pgfsetbuttcap%
\pgfsetroundjoin%
\definecolor{currentfill}{rgb}{0.000000,0.000000,0.000000}%
\pgfsetfillcolor{currentfill}%
\pgfsetlinewidth{0.602250pt}%
\definecolor{currentstroke}{rgb}{0.000000,0.000000,0.000000}%
\pgfsetstrokecolor{currentstroke}%
\pgfsetdash{}{0pt}%
\pgfsys@defobject{currentmarker}{\pgfqpoint{-0.027778in}{0.000000in}}{\pgfqpoint{-0.000000in}{0.000000in}}{%
\pgfpathmoveto{\pgfqpoint{-0.000000in}{0.000000in}}%
\pgfpathlineto{\pgfqpoint{-0.027778in}{0.000000in}}%
\pgfusepath{stroke,fill}%
}%
\begin{pgfscope}%
\pgfsys@transformshift{0.800000in}{2.869019in}%
\pgfsys@useobject{currentmarker}{}%
\end{pgfscope}%
\end{pgfscope}%
\begin{pgfscope}%
\pgfsetbuttcap%
\pgfsetroundjoin%
\definecolor{currentfill}{rgb}{0.000000,0.000000,0.000000}%
\pgfsetfillcolor{currentfill}%
\pgfsetlinewidth{0.602250pt}%
\definecolor{currentstroke}{rgb}{0.000000,0.000000,0.000000}%
\pgfsetstrokecolor{currentstroke}%
\pgfsetdash{}{0pt}%
\pgfsys@defobject{currentmarker}{\pgfqpoint{-0.027778in}{0.000000in}}{\pgfqpoint{-0.000000in}{0.000000in}}{%
\pgfpathmoveto{\pgfqpoint{-0.000000in}{0.000000in}}%
\pgfpathlineto{\pgfqpoint{-0.027778in}{0.000000in}}%
\pgfusepath{stroke,fill}%
}%
\begin{pgfscope}%
\pgfsys@transformshift{0.800000in}{2.966734in}%
\pgfsys@useobject{currentmarker}{}%
\end{pgfscope}%
\end{pgfscope}%
\begin{pgfscope}%
\pgfsetbuttcap%
\pgfsetroundjoin%
\definecolor{currentfill}{rgb}{0.000000,0.000000,0.000000}%
\pgfsetfillcolor{currentfill}%
\pgfsetlinewidth{0.602250pt}%
\definecolor{currentstroke}{rgb}{0.000000,0.000000,0.000000}%
\pgfsetstrokecolor{currentstroke}%
\pgfsetdash{}{0pt}%
\pgfsys@defobject{currentmarker}{\pgfqpoint{-0.027778in}{0.000000in}}{\pgfqpoint{-0.000000in}{0.000000in}}{%
\pgfpathmoveto{\pgfqpoint{-0.000000in}{0.000000in}}%
\pgfpathlineto{\pgfqpoint{-0.027778in}{0.000000in}}%
\pgfusepath{stroke,fill}%
}%
\begin{pgfscope}%
\pgfsys@transformshift{0.800000in}{3.051378in}%
\pgfsys@useobject{currentmarker}{}%
\end{pgfscope}%
\end{pgfscope}%
\begin{pgfscope}%
\pgfsetbuttcap%
\pgfsetroundjoin%
\definecolor{currentfill}{rgb}{0.000000,0.000000,0.000000}%
\pgfsetfillcolor{currentfill}%
\pgfsetlinewidth{0.602250pt}%
\definecolor{currentstroke}{rgb}{0.000000,0.000000,0.000000}%
\pgfsetstrokecolor{currentstroke}%
\pgfsetdash{}{0pt}%
\pgfsys@defobject{currentmarker}{\pgfqpoint{-0.027778in}{0.000000in}}{\pgfqpoint{-0.000000in}{0.000000in}}{%
\pgfpathmoveto{\pgfqpoint{-0.000000in}{0.000000in}}%
\pgfpathlineto{\pgfqpoint{-0.027778in}{0.000000in}}%
\pgfusepath{stroke,fill}%
}%
\begin{pgfscope}%
\pgfsys@transformshift{0.800000in}{3.126040in}%
\pgfsys@useobject{currentmarker}{}%
\end{pgfscope}%
\end{pgfscope}%
\begin{pgfscope}%
\pgfsetbuttcap%
\pgfsetroundjoin%
\definecolor{currentfill}{rgb}{0.000000,0.000000,0.000000}%
\pgfsetfillcolor{currentfill}%
\pgfsetlinewidth{0.602250pt}%
\definecolor{currentstroke}{rgb}{0.000000,0.000000,0.000000}%
\pgfsetstrokecolor{currentstroke}%
\pgfsetdash{}{0pt}%
\pgfsys@defobject{currentmarker}{\pgfqpoint{-0.027778in}{0.000000in}}{\pgfqpoint{-0.000000in}{0.000000in}}{%
\pgfpathmoveto{\pgfqpoint{-0.000000in}{0.000000in}}%
\pgfpathlineto{\pgfqpoint{-0.027778in}{0.000000in}}%
\pgfusepath{stroke,fill}%
}%
\begin{pgfscope}%
\pgfsys@transformshift{0.800000in}{3.632208in}%
\pgfsys@useobject{currentmarker}{}%
\end{pgfscope}%
\end{pgfscope}%
\begin{pgfscope}%
\pgfsetbuttcap%
\pgfsetroundjoin%
\definecolor{currentfill}{rgb}{0.000000,0.000000,0.000000}%
\pgfsetfillcolor{currentfill}%
\pgfsetlinewidth{0.602250pt}%
\definecolor{currentstroke}{rgb}{0.000000,0.000000,0.000000}%
\pgfsetstrokecolor{currentstroke}%
\pgfsetdash{}{0pt}%
\pgfsys@defobject{currentmarker}{\pgfqpoint{-0.027778in}{0.000000in}}{\pgfqpoint{-0.000000in}{0.000000in}}{%
\pgfpathmoveto{\pgfqpoint{-0.000000in}{0.000000in}}%
\pgfpathlineto{\pgfqpoint{-0.027778in}{0.000000in}}%
\pgfusepath{stroke,fill}%
}%
\begin{pgfscope}%
\pgfsys@transformshift{0.800000in}{3.889229in}%
\pgfsys@useobject{currentmarker}{}%
\end{pgfscope}%
\end{pgfscope}%
\begin{pgfscope}%
\pgfsetbuttcap%
\pgfsetroundjoin%
\definecolor{currentfill}{rgb}{0.000000,0.000000,0.000000}%
\pgfsetfillcolor{currentfill}%
\pgfsetlinewidth{0.602250pt}%
\definecolor{currentstroke}{rgb}{0.000000,0.000000,0.000000}%
\pgfsetstrokecolor{currentstroke}%
\pgfsetdash{}{0pt}%
\pgfsys@defobject{currentmarker}{\pgfqpoint{-0.027778in}{0.000000in}}{\pgfqpoint{-0.000000in}{0.000000in}}{%
\pgfpathmoveto{\pgfqpoint{-0.000000in}{0.000000in}}%
\pgfpathlineto{\pgfqpoint{-0.027778in}{0.000000in}}%
\pgfusepath{stroke,fill}%
}%
\begin{pgfscope}%
\pgfsys@transformshift{0.800000in}{4.071589in}%
\pgfsys@useobject{currentmarker}{}%
\end{pgfscope}%
\end{pgfscope}%
\begin{pgfscope}%
\pgfsetbuttcap%
\pgfsetroundjoin%
\definecolor{currentfill}{rgb}{0.000000,0.000000,0.000000}%
\pgfsetfillcolor{currentfill}%
\pgfsetlinewidth{0.602250pt}%
\definecolor{currentstroke}{rgb}{0.000000,0.000000,0.000000}%
\pgfsetstrokecolor{currentstroke}%
\pgfsetdash{}{0pt}%
\pgfsys@defobject{currentmarker}{\pgfqpoint{-0.027778in}{0.000000in}}{\pgfqpoint{-0.000000in}{0.000000in}}{%
\pgfpathmoveto{\pgfqpoint{-0.000000in}{0.000000in}}%
\pgfpathlineto{\pgfqpoint{-0.027778in}{0.000000in}}%
\pgfusepath{stroke,fill}%
}%
\begin{pgfscope}%
\pgfsys@transformshift{0.800000in}{4.213038in}%
\pgfsys@useobject{currentmarker}{}%
\end{pgfscope}%
\end{pgfscope}%
\begin{pgfscope}%
\definecolor{textcolor}{rgb}{0.000000,0.000000,0.000000}%
\pgfsetstrokecolor{textcolor}%
\pgfsetfillcolor{textcolor}%
\pgftext[x=0.446026in,y=2.376000in,,bottom,rotate=90.000000]{\color{textcolor}\rmfamily\fontsize{10.000000}{12.000000}\selectfont Comparisons}%
\end{pgfscope}%
\begin{pgfscope}%
\pgfpathrectangle{\pgfqpoint{0.800000in}{0.528000in}}{\pgfqpoint{4.960000in}{3.696000in}}%
\pgfusepath{clip}%
\pgfsetrectcap%
\pgfsetroundjoin%
\pgfsetlinewidth{1.505625pt}%
\definecolor{currentstroke}{rgb}{0.121569,0.466667,0.705882}%
\pgfsetstrokecolor{currentstroke}%
\pgfsetdash{}{0pt}%
\pgfpathmoveto{\pgfqpoint{1.025455in}{0.701188in}}%
\pgfpathlineto{\pgfqpoint{1.071466in}{1.232258in}}%
\pgfpathlineto{\pgfqpoint{1.117477in}{1.533135in}}%
\pgfpathlineto{\pgfqpoint{1.163488in}{1.745477in}}%
\pgfpathlineto{\pgfqpoint{1.209499in}{1.909080in}}%
\pgfpathlineto{\pgfqpoint{1.255510in}{2.046008in}}%
\pgfpathlineto{\pgfqpoint{1.301521in}{2.160166in}}%
\pgfpathlineto{\pgfqpoint{1.347532in}{2.255869in}}%
\pgfpathlineto{\pgfqpoint{1.393544in}{2.340948in}}%
\pgfpathlineto{\pgfqpoint{1.439555in}{2.417930in}}%
\pgfpathlineto{\pgfqpoint{1.485566in}{2.485795in}}%
\pgfpathlineto{\pgfqpoint{1.531577in}{2.551395in}}%
\pgfpathlineto{\pgfqpoint{1.577588in}{2.609091in}}%
\pgfpathlineto{\pgfqpoint{1.623599in}{2.664601in}}%
\pgfpathlineto{\pgfqpoint{1.669610in}{2.713428in}}%
\pgfpathlineto{\pgfqpoint{1.715622in}{2.758510in}}%
\pgfpathlineto{\pgfqpoint{1.761633in}{2.801374in}}%
\pgfpathlineto{\pgfqpoint{1.807644in}{2.842757in}}%
\pgfpathlineto{\pgfqpoint{1.853655in}{2.880130in}}%
\pgfpathlineto{\pgfqpoint{1.899666in}{2.917246in}}%
\pgfpathlineto{\pgfqpoint{1.945677in}{2.953123in}}%
\pgfpathlineto{\pgfqpoint{1.991688in}{2.986744in}}%
\pgfpathlineto{\pgfqpoint{2.037699in}{3.020039in}}%
\pgfpathlineto{\pgfqpoint{2.083711in}{3.050006in}}%
\pgfpathlineto{\pgfqpoint{2.129722in}{3.079697in}}%
\pgfpathlineto{\pgfqpoint{2.175733in}{3.107726in}}%
\pgfpathlineto{\pgfqpoint{2.221744in}{3.135249in}}%
\pgfpathlineto{\pgfqpoint{2.267755in}{3.160586in}}%
\pgfpathlineto{\pgfqpoint{2.313766in}{3.185579in}}%
\pgfpathlineto{\pgfqpoint{2.359777in}{3.209388in}}%
\pgfpathlineto{\pgfqpoint{2.405788in}{3.232931in}}%
\pgfpathlineto{\pgfqpoint{2.451800in}{3.256113in}}%
\pgfpathlineto{\pgfqpoint{2.497811in}{3.277701in}}%
\pgfpathlineto{\pgfqpoint{2.543822in}{3.298867in}}%
\pgfpathlineto{\pgfqpoint{2.589833in}{3.319412in}}%
\pgfpathlineto{\pgfqpoint{2.635844in}{3.339020in}}%
\pgfpathlineto{\pgfqpoint{2.681855in}{3.359162in}}%
\pgfpathlineto{\pgfqpoint{2.727866in}{3.377103in}}%
\pgfpathlineto{\pgfqpoint{2.773878in}{3.395398in}}%
\pgfpathlineto{\pgfqpoint{2.819889in}{3.413269in}}%
\pgfpathlineto{\pgfqpoint{2.865900in}{3.430179in}}%
\pgfpathlineto{\pgfqpoint{2.911911in}{3.447787in}}%
\pgfpathlineto{\pgfqpoint{2.957922in}{3.464796in}}%
\pgfpathlineto{\pgfqpoint{3.003933in}{3.481585in}}%
\pgfpathlineto{\pgfqpoint{3.049944in}{3.497431in}}%
\pgfpathlineto{\pgfqpoint{3.095955in}{3.513449in}}%
\pgfpathlineto{\pgfqpoint{3.141967in}{3.529326in}}%
\pgfpathlineto{\pgfqpoint{3.187978in}{3.544219in}}%
\pgfpathlineto{\pgfqpoint{3.233989in}{3.559310in}}%
\pgfpathlineto{\pgfqpoint{3.280000in}{3.573480in}}%
\pgfpathlineto{\pgfqpoint{3.326011in}{3.587422in}}%
\pgfpathlineto{\pgfqpoint{3.372022in}{3.600650in}}%
\pgfpathlineto{\pgfqpoint{3.418033in}{3.614698in}}%
\pgfpathlineto{\pgfqpoint{3.464045in}{3.627822in}}%
\pgfpathlineto{\pgfqpoint{3.510056in}{3.640849in}}%
\pgfpathlineto{\pgfqpoint{3.556067in}{3.654055in}}%
\pgfpathlineto{\pgfqpoint{3.602078in}{3.666108in}}%
\pgfpathlineto{\pgfqpoint{3.648089in}{3.678467in}}%
\pgfpathlineto{\pgfqpoint{3.694100in}{3.690810in}}%
\pgfpathlineto{\pgfqpoint{3.740111in}{3.702462in}}%
\pgfpathlineto{\pgfqpoint{3.786122in}{3.714139in}}%
\pgfpathlineto{\pgfqpoint{3.832134in}{3.726042in}}%
\pgfpathlineto{\pgfqpoint{3.878145in}{3.737059in}}%
\pgfpathlineto{\pgfqpoint{3.924156in}{3.747894in}}%
\pgfpathlineto{\pgfqpoint{3.970167in}{3.759118in}}%
\pgfpathlineto{\pgfqpoint{4.016178in}{3.769615in}}%
\pgfpathlineto{\pgfqpoint{4.062189in}{3.779982in}}%
\pgfpathlineto{\pgfqpoint{4.108200in}{3.790385in}}%
\pgfpathlineto{\pgfqpoint{4.154212in}{3.800984in}}%
\pgfpathlineto{\pgfqpoint{4.200223in}{3.811179in}}%
\pgfpathlineto{\pgfqpoint{4.246234in}{3.821359in}}%
\pgfpathlineto{\pgfqpoint{4.292245in}{3.830363in}}%
\pgfpathlineto{\pgfqpoint{4.338256in}{3.840402in}}%
\pgfpathlineto{\pgfqpoint{4.384267in}{3.849749in}}%
\pgfpathlineto{\pgfqpoint{4.430278in}{3.859053in}}%
\pgfpathlineto{\pgfqpoint{4.476289in}{3.868196in}}%
\pgfpathlineto{\pgfqpoint{4.522301in}{3.877481in}}%
\pgfpathlineto{\pgfqpoint{4.568312in}{3.886559in}}%
\pgfpathlineto{\pgfqpoint{4.614323in}{3.895288in}}%
\pgfpathlineto{\pgfqpoint{4.660334in}{3.904319in}}%
\pgfpathlineto{\pgfqpoint{4.706345in}{3.912343in}}%
\pgfpathlineto{\pgfqpoint{4.752356in}{3.920907in}}%
\pgfpathlineto{\pgfqpoint{4.798367in}{3.929528in}}%
\pgfpathlineto{\pgfqpoint{4.844378in}{3.938474in}}%
\pgfpathlineto{\pgfqpoint{4.890390in}{3.947026in}}%
\pgfpathlineto{\pgfqpoint{4.936401in}{3.955419in}}%
\pgfpathlineto{\pgfqpoint{4.982412in}{3.963822in}}%
\pgfpathlineto{\pgfqpoint{5.028423in}{3.972067in}}%
\pgfpathlineto{\pgfqpoint{5.074434in}{3.979986in}}%
\pgfpathlineto{\pgfqpoint{5.120445in}{3.987833in}}%
\pgfpathlineto{\pgfqpoint{5.166456in}{3.995813in}}%
\pgfpathlineto{\pgfqpoint{5.212468in}{4.003697in}}%
\pgfpathlineto{\pgfqpoint{5.258479in}{4.011142in}}%
\pgfpathlineto{\pgfqpoint{5.304490in}{4.018832in}}%
\pgfpathlineto{\pgfqpoint{5.350501in}{4.026563in}}%
\pgfpathlineto{\pgfqpoint{5.396512in}{4.033790in}}%
\pgfpathlineto{\pgfqpoint{5.442523in}{4.041268in}}%
\pgfpathlineto{\pgfqpoint{5.488534in}{4.048790in}}%
\pgfpathlineto{\pgfqpoint{5.534545in}{4.055756in}}%
\pgfusepath{stroke}%
\end{pgfscope}%
\begin{pgfscope}%
\pgfpathrectangle{\pgfqpoint{0.800000in}{0.528000in}}{\pgfqpoint{4.960000in}{3.696000in}}%
\pgfusepath{clip}%
\pgfsetrectcap%
\pgfsetroundjoin%
\pgfsetlinewidth{1.505625pt}%
\definecolor{currentstroke}{rgb}{1.000000,0.498039,0.054902}%
\pgfsetstrokecolor{currentstroke}%
\pgfsetdash{}{0pt}%
\pgfpathmoveto{\pgfqpoint{1.025455in}{0.696000in}}%
\pgfpathlineto{\pgfqpoint{1.071466in}{1.227208in}}%
\pgfpathlineto{\pgfqpoint{1.117477in}{1.529473in}}%
\pgfpathlineto{\pgfqpoint{1.163488in}{1.745850in}}%
\pgfpathlineto{\pgfqpoint{1.209499in}{1.909464in}}%
\pgfpathlineto{\pgfqpoint{1.255510in}{2.044225in}}%
\pgfpathlineto{\pgfqpoint{1.301521in}{2.159260in}}%
\pgfpathlineto{\pgfqpoint{1.347532in}{2.255313in}}%
\pgfpathlineto{\pgfqpoint{1.393544in}{2.340753in}}%
\pgfpathlineto{\pgfqpoint{1.439555in}{2.415990in}}%
\pgfpathlineto{\pgfqpoint{1.485566in}{2.485369in}}%
\pgfpathlineto{\pgfqpoint{1.531577in}{2.551256in}}%
\pgfpathlineto{\pgfqpoint{1.577588in}{2.608422in}}%
\pgfpathlineto{\pgfqpoint{1.623599in}{2.662321in}}%
\pgfpathlineto{\pgfqpoint{1.669610in}{2.713698in}}%
\pgfpathlineto{\pgfqpoint{1.715622in}{2.758158in}}%
\pgfpathlineto{\pgfqpoint{1.761633in}{2.802431in}}%
\pgfpathlineto{\pgfqpoint{1.807644in}{2.843175in}}%
\pgfpathlineto{\pgfqpoint{1.853655in}{2.881437in}}%
\pgfpathlineto{\pgfqpoint{1.899666in}{2.917618in}}%
\pgfpathlineto{\pgfqpoint{1.945677in}{2.952438in}}%
\pgfpathlineto{\pgfqpoint{1.991688in}{2.986499in}}%
\pgfpathlineto{\pgfqpoint{2.037699in}{3.018922in}}%
\pgfpathlineto{\pgfqpoint{2.083711in}{3.050133in}}%
\pgfpathlineto{\pgfqpoint{2.129722in}{3.079470in}}%
\pgfpathlineto{\pgfqpoint{2.175733in}{3.107596in}}%
\pgfpathlineto{\pgfqpoint{2.221744in}{3.134290in}}%
\pgfpathlineto{\pgfqpoint{2.267755in}{3.161346in}}%
\pgfpathlineto{\pgfqpoint{2.313766in}{3.185873in}}%
\pgfpathlineto{\pgfqpoint{2.359777in}{3.209968in}}%
\pgfpathlineto{\pgfqpoint{2.405788in}{3.233882in}}%
\pgfpathlineto{\pgfqpoint{2.451800in}{3.255309in}}%
\pgfpathlineto{\pgfqpoint{2.497811in}{3.277723in}}%
\pgfpathlineto{\pgfqpoint{2.543822in}{3.299168in}}%
\pgfpathlineto{\pgfqpoint{2.589833in}{3.319599in}}%
\pgfpathlineto{\pgfqpoint{2.635844in}{3.338648in}}%
\pgfpathlineto{\pgfqpoint{2.681855in}{3.358001in}}%
\pgfpathlineto{\pgfqpoint{2.727866in}{3.377065in}}%
\pgfpathlineto{\pgfqpoint{2.773878in}{3.395453in}}%
\pgfpathlineto{\pgfqpoint{2.819889in}{3.413090in}}%
\pgfpathlineto{\pgfqpoint{2.865900in}{3.430371in}}%
\pgfpathlineto{\pgfqpoint{2.911911in}{3.447762in}}%
\pgfpathlineto{\pgfqpoint{2.957922in}{3.465340in}}%
\pgfpathlineto{\pgfqpoint{3.003933in}{3.481408in}}%
\pgfpathlineto{\pgfqpoint{3.049944in}{3.497431in}}%
\pgfpathlineto{\pgfqpoint{3.095955in}{3.513931in}}%
\pgfpathlineto{\pgfqpoint{3.141967in}{3.528506in}}%
\pgfpathlineto{\pgfqpoint{3.187978in}{3.544262in}}%
\pgfpathlineto{\pgfqpoint{3.233989in}{3.558456in}}%
\pgfpathlineto{\pgfqpoint{3.280000in}{3.573195in}}%
\pgfpathlineto{\pgfqpoint{3.326011in}{3.587265in}}%
\pgfpathlineto{\pgfqpoint{3.372022in}{3.601017in}}%
\pgfpathlineto{\pgfqpoint{3.418033in}{3.614985in}}%
\pgfpathlineto{\pgfqpoint{3.464045in}{3.628581in}}%
\pgfpathlineto{\pgfqpoint{3.510056in}{3.641212in}}%
\pgfpathlineto{\pgfqpoint{3.556067in}{3.653607in}}%
\pgfpathlineto{\pgfqpoint{3.602078in}{3.666096in}}%
\pgfpathlineto{\pgfqpoint{3.648089in}{3.678809in}}%
\pgfpathlineto{\pgfqpoint{3.694100in}{3.690601in}}%
\pgfpathlineto{\pgfqpoint{3.740111in}{3.702740in}}%
\pgfpathlineto{\pgfqpoint{3.786122in}{3.713771in}}%
\pgfpathlineto{\pgfqpoint{3.832134in}{3.725960in}}%
\pgfpathlineto{\pgfqpoint{3.878145in}{3.736860in}}%
\pgfpathlineto{\pgfqpoint{3.924156in}{3.748211in}}%
\pgfpathlineto{\pgfqpoint{3.970167in}{3.759040in}}%
\pgfpathlineto{\pgfqpoint{4.016178in}{3.769906in}}%
\pgfpathlineto{\pgfqpoint{4.062189in}{3.779912in}}%
\pgfpathlineto{\pgfqpoint{4.108200in}{3.790503in}}%
\pgfpathlineto{\pgfqpoint{4.154212in}{3.800678in}}%
\pgfpathlineto{\pgfqpoint{4.200223in}{3.811007in}}%
\pgfpathlineto{\pgfqpoint{4.246234in}{3.820559in}}%
\pgfpathlineto{\pgfqpoint{4.292245in}{3.830771in}}%
\pgfpathlineto{\pgfqpoint{4.338256in}{3.840456in}}%
\pgfpathlineto{\pgfqpoint{4.384267in}{3.850180in}}%
\pgfpathlineto{\pgfqpoint{4.430278in}{3.859412in}}%
\pgfpathlineto{\pgfqpoint{4.476289in}{3.868729in}}%
\pgfpathlineto{\pgfqpoint{4.522301in}{3.877563in}}%
\pgfpathlineto{\pgfqpoint{4.568312in}{3.886636in}}%
\pgfpathlineto{\pgfqpoint{4.614323in}{3.895062in}}%
\pgfpathlineto{\pgfqpoint{4.660334in}{3.904096in}}%
\pgfpathlineto{\pgfqpoint{4.706345in}{3.912938in}}%
\pgfpathlineto{\pgfqpoint{4.752356in}{3.920614in}}%
\pgfpathlineto{\pgfqpoint{4.798367in}{3.930292in}}%
\pgfpathlineto{\pgfqpoint{4.844378in}{3.938349in}}%
\pgfpathlineto{\pgfqpoint{4.890390in}{3.947057in}}%
\pgfpathlineto{\pgfqpoint{4.936401in}{3.955106in}}%
\pgfpathlineto{\pgfqpoint{4.982412in}{3.964077in}}%
\pgfpathlineto{\pgfqpoint{5.028423in}{3.972089in}}%
\pgfpathlineto{\pgfqpoint{5.074434in}{3.979894in}}%
\pgfpathlineto{\pgfqpoint{5.120445in}{3.988581in}}%
\pgfpathlineto{\pgfqpoint{5.166456in}{3.996088in}}%
\pgfpathlineto{\pgfqpoint{5.212468in}{4.003873in}}%
\pgfpathlineto{\pgfqpoint{5.258479in}{4.011271in}}%
\pgfpathlineto{\pgfqpoint{5.304490in}{4.019311in}}%
\pgfpathlineto{\pgfqpoint{5.350501in}{4.026522in}}%
\pgfpathlineto{\pgfqpoint{5.396512in}{4.034307in}}%
\pgfpathlineto{\pgfqpoint{5.442523in}{4.041022in}}%
\pgfpathlineto{\pgfqpoint{5.488534in}{4.048869in}}%
\pgfpathlineto{\pgfqpoint{5.534545in}{4.056000in}}%
\pgfusepath{stroke}%
\end{pgfscope}%
\begin{pgfscope}%
\pgfsetrectcap%
\pgfsetmiterjoin%
\pgfsetlinewidth{0.803000pt}%
\definecolor{currentstroke}{rgb}{0.000000,0.000000,0.000000}%
\pgfsetstrokecolor{currentstroke}%
\pgfsetdash{}{0pt}%
\pgfpathmoveto{\pgfqpoint{0.800000in}{0.528000in}}%
\pgfpathlineto{\pgfqpoint{0.800000in}{4.224000in}}%
\pgfusepath{stroke}%
\end{pgfscope}%
\begin{pgfscope}%
\pgfsetrectcap%
\pgfsetmiterjoin%
\pgfsetlinewidth{0.803000pt}%
\definecolor{currentstroke}{rgb}{0.000000,0.000000,0.000000}%
\pgfsetstrokecolor{currentstroke}%
\pgfsetdash{}{0pt}%
\pgfpathmoveto{\pgfqpoint{5.760000in}{0.528000in}}%
\pgfpathlineto{\pgfqpoint{5.760000in}{4.224000in}}%
\pgfusepath{stroke}%
\end{pgfscope}%
\begin{pgfscope}%
\pgfsetrectcap%
\pgfsetmiterjoin%
\pgfsetlinewidth{0.803000pt}%
\definecolor{currentstroke}{rgb}{0.000000,0.000000,0.000000}%
\pgfsetstrokecolor{currentstroke}%
\pgfsetdash{}{0pt}%
\pgfpathmoveto{\pgfqpoint{0.800000in}{0.528000in}}%
\pgfpathlineto{\pgfqpoint{5.760000in}{0.528000in}}%
\pgfusepath{stroke}%
\end{pgfscope}%
\begin{pgfscope}%
\pgfsetrectcap%
\pgfsetmiterjoin%
\pgfsetlinewidth{0.803000pt}%
\definecolor{currentstroke}{rgb}{0.000000,0.000000,0.000000}%
\pgfsetstrokecolor{currentstroke}%
\pgfsetdash{}{0pt}%
\pgfpathmoveto{\pgfqpoint{0.800000in}{4.224000in}}%
\pgfpathlineto{\pgfqpoint{5.760000in}{4.224000in}}%
\pgfusepath{stroke}%
\end{pgfscope}%
\begin{pgfscope}%
\pgfsetbuttcap%
\pgfsetmiterjoin%
\definecolor{currentfill}{rgb}{1.000000,1.000000,1.000000}%
\pgfsetfillcolor{currentfill}%
\pgfsetfillopacity{0.800000}%
\pgfsetlinewidth{1.003750pt}%
\definecolor{currentstroke}{rgb}{0.800000,0.800000,0.800000}%
\pgfsetstrokecolor{currentstroke}%
\pgfsetstrokeopacity{0.800000}%
\pgfsetdash{}{0pt}%
\pgfpathmoveto{\pgfqpoint{0.897222in}{3.725543in}}%
\pgfpathlineto{\pgfqpoint{2.014507in}{3.725543in}}%
\pgfpathquadraticcurveto{\pgfqpoint{2.042285in}{3.725543in}}{\pgfqpoint{2.042285in}{3.753321in}}%
\pgfpathlineto{\pgfqpoint{2.042285in}{4.126778in}}%
\pgfpathquadraticcurveto{\pgfqpoint{2.042285in}{4.154556in}}{\pgfqpoint{2.014507in}{4.154556in}}%
\pgfpathlineto{\pgfqpoint{0.897222in}{4.154556in}}%
\pgfpathquadraticcurveto{\pgfqpoint{0.869444in}{4.154556in}}{\pgfqpoint{0.869444in}{4.126778in}}%
\pgfpathlineto{\pgfqpoint{0.869444in}{3.753321in}}%
\pgfpathquadraticcurveto{\pgfqpoint{0.869444in}{3.725543in}}{\pgfqpoint{0.897222in}{3.725543in}}%
\pgfpathlineto{\pgfqpoint{0.897222in}{3.725543in}}%
\pgfpathclose%
\pgfusepath{stroke,fill}%
\end{pgfscope}%
\begin{pgfscope}%
\pgfsetrectcap%
\pgfsetroundjoin%
\pgfsetlinewidth{1.505625pt}%
\definecolor{currentstroke}{rgb}{0.121569,0.466667,0.705882}%
\pgfsetstrokecolor{currentstroke}%
\pgfsetdash{}{0pt}%
\pgfpathmoveto{\pgfqpoint{0.925000in}{4.050389in}}%
\pgfpathlineto{\pgfqpoint{1.063889in}{4.050389in}}%
\pgfpathlineto{\pgfqpoint{1.202778in}{4.050389in}}%
\pgfusepath{stroke}%
\end{pgfscope}%
\begin{pgfscope}%
\definecolor{textcolor}{rgb}{0.000000,0.000000,0.000000}%
\pgfsetstrokecolor{textcolor}%
\pgfsetfillcolor{textcolor}%
\pgftext[x=1.313889in,y=4.001778in,left,base]{\color{textcolor}\rmfamily\fontsize{10.000000}{12.000000}\selectfont mergesort}%
\end{pgfscope}%
\begin{pgfscope}%
\pgfsetrectcap%
\pgfsetroundjoin%
\pgfsetlinewidth{1.505625pt}%
\definecolor{currentstroke}{rgb}{1.000000,0.498039,0.054902}%
\pgfsetstrokecolor{currentstroke}%
\pgfsetdash{}{0pt}%
\pgfpathmoveto{\pgfqpoint{0.925000in}{3.856716in}}%
\pgfpathlineto{\pgfqpoint{1.063889in}{3.856716in}}%
\pgfpathlineto{\pgfqpoint{1.202778in}{3.856716in}}%
\pgfusepath{stroke}%
\end{pgfscope}%
\begin{pgfscope}%
\definecolor{textcolor}{rgb}{0.000000,0.000000,0.000000}%
\pgfsetstrokecolor{textcolor}%
\pgfsetfillcolor{textcolor}%
\pgftext[x=1.313889in,y=3.808105in,left,base]{\color{textcolor}\rmfamily\fontsize{10.000000}{12.000000}\selectfont bmergesort}%
\end{pgfscope}%
\end{pgfpicture}%
\makeatother%
\endgroup%

\subsubsection{Swaps}
%% Creator: Matplotlib, PGF backend
%%
%% To include the figure in your LaTeX document, write
%%   \input{<filename>.pgf}
%%
%% Make sure the required packages are loaded in your preamble
%%   \usepackage{pgf}
%%
%% Also ensure that all the required font packages are loaded; for instance,
%% the lmodern package is sometimes necessary when using math font.
%%   \usepackage{lmodern}
%%
%% Figures using additional raster images can only be included by \input if
%% they are in the same directory as the main LaTeX file. For loading figures
%% from other directories you can use the `import` package
%%   \usepackage{import}
%%
%% and then include the figures with
%%   \import{<path to file>}{<filename>.pgf}
%%
%% Matplotlib used the following preamble
%%   
%%   \makeatletter\@ifpackageloaded{underscore}{}{\usepackage[strings]{underscore}}\makeatother
%%
\begingroup%
\makeatletter%
\begin{pgfpicture}%
\pgfpathrectangle{\pgfpointorigin}{\pgfqpoint{6.400000in}{4.800000in}}%
\pgfusepath{use as bounding box, clip}%
\begin{pgfscope}%
\pgfsetbuttcap%
\pgfsetmiterjoin%
\definecolor{currentfill}{rgb}{1.000000,1.000000,1.000000}%
\pgfsetfillcolor{currentfill}%
\pgfsetlinewidth{0.000000pt}%
\definecolor{currentstroke}{rgb}{1.000000,1.000000,1.000000}%
\pgfsetstrokecolor{currentstroke}%
\pgfsetdash{}{0pt}%
\pgfpathmoveto{\pgfqpoint{0.000000in}{0.000000in}}%
\pgfpathlineto{\pgfqpoint{6.400000in}{0.000000in}}%
\pgfpathlineto{\pgfqpoint{6.400000in}{4.800000in}}%
\pgfpathlineto{\pgfqpoint{0.000000in}{4.800000in}}%
\pgfpathlineto{\pgfqpoint{0.000000in}{0.000000in}}%
\pgfpathclose%
\pgfusepath{fill}%
\end{pgfscope}%
\begin{pgfscope}%
\pgfsetbuttcap%
\pgfsetmiterjoin%
\definecolor{currentfill}{rgb}{1.000000,1.000000,1.000000}%
\pgfsetfillcolor{currentfill}%
\pgfsetlinewidth{0.000000pt}%
\definecolor{currentstroke}{rgb}{0.000000,0.000000,0.000000}%
\pgfsetstrokecolor{currentstroke}%
\pgfsetstrokeopacity{0.000000}%
\pgfsetdash{}{0pt}%
\pgfpathmoveto{\pgfqpoint{0.800000in}{0.528000in}}%
\pgfpathlineto{\pgfqpoint{5.760000in}{0.528000in}}%
\pgfpathlineto{\pgfqpoint{5.760000in}{4.224000in}}%
\pgfpathlineto{\pgfqpoint{0.800000in}{4.224000in}}%
\pgfpathlineto{\pgfqpoint{0.800000in}{0.528000in}}%
\pgfpathclose%
\pgfusepath{fill}%
\end{pgfscope}%
\begin{pgfscope}%
\pgfsetbuttcap%
\pgfsetroundjoin%
\definecolor{currentfill}{rgb}{0.000000,0.000000,0.000000}%
\pgfsetfillcolor{currentfill}%
\pgfsetlinewidth{0.803000pt}%
\definecolor{currentstroke}{rgb}{0.000000,0.000000,0.000000}%
\pgfsetstrokecolor{currentstroke}%
\pgfsetdash{}{0pt}%
\pgfsys@defobject{currentmarker}{\pgfqpoint{0.000000in}{-0.048611in}}{\pgfqpoint{0.000000in}{0.000000in}}{%
\pgfpathmoveto{\pgfqpoint{0.000000in}{0.000000in}}%
\pgfpathlineto{\pgfqpoint{0.000000in}{-0.048611in}}%
\pgfusepath{stroke,fill}%
}%
\begin{pgfscope}%
\pgfsys@transformshift{0.979443in}{0.528000in}%
\pgfsys@useobject{currentmarker}{}%
\end{pgfscope}%
\end{pgfscope}%
\begin{pgfscope}%
\definecolor{textcolor}{rgb}{0.000000,0.000000,0.000000}%
\pgfsetstrokecolor{textcolor}%
\pgfsetfillcolor{textcolor}%
\pgftext[x=0.979443in,y=0.430778in,,top]{\color{textcolor}\rmfamily\fontsize{10.000000}{12.000000}\selectfont \(\displaystyle {0}\)}%
\end{pgfscope}%
\begin{pgfscope}%
\pgfsetbuttcap%
\pgfsetroundjoin%
\definecolor{currentfill}{rgb}{0.000000,0.000000,0.000000}%
\pgfsetfillcolor{currentfill}%
\pgfsetlinewidth{0.803000pt}%
\definecolor{currentstroke}{rgb}{0.000000,0.000000,0.000000}%
\pgfsetstrokecolor{currentstroke}%
\pgfsetdash{}{0pt}%
\pgfsys@defobject{currentmarker}{\pgfqpoint{0.000000in}{-0.048611in}}{\pgfqpoint{0.000000in}{0.000000in}}{%
\pgfpathmoveto{\pgfqpoint{0.000000in}{0.000000in}}%
\pgfpathlineto{\pgfqpoint{0.000000in}{-0.048611in}}%
\pgfusepath{stroke,fill}%
}%
\begin{pgfscope}%
\pgfsys@transformshift{1.899666in}{0.528000in}%
\pgfsys@useobject{currentmarker}{}%
\end{pgfscope}%
\end{pgfscope}%
\begin{pgfscope}%
\definecolor{textcolor}{rgb}{0.000000,0.000000,0.000000}%
\pgfsetstrokecolor{textcolor}%
\pgfsetfillcolor{textcolor}%
\pgftext[x=1.899666in,y=0.430778in,,top]{\color{textcolor}\rmfamily\fontsize{10.000000}{12.000000}\selectfont \(\displaystyle {2000}\)}%
\end{pgfscope}%
\begin{pgfscope}%
\pgfsetbuttcap%
\pgfsetroundjoin%
\definecolor{currentfill}{rgb}{0.000000,0.000000,0.000000}%
\pgfsetfillcolor{currentfill}%
\pgfsetlinewidth{0.803000pt}%
\definecolor{currentstroke}{rgb}{0.000000,0.000000,0.000000}%
\pgfsetstrokecolor{currentstroke}%
\pgfsetdash{}{0pt}%
\pgfsys@defobject{currentmarker}{\pgfqpoint{0.000000in}{-0.048611in}}{\pgfqpoint{0.000000in}{0.000000in}}{%
\pgfpathmoveto{\pgfqpoint{0.000000in}{0.000000in}}%
\pgfpathlineto{\pgfqpoint{0.000000in}{-0.048611in}}%
\pgfusepath{stroke,fill}%
}%
\begin{pgfscope}%
\pgfsys@transformshift{2.819889in}{0.528000in}%
\pgfsys@useobject{currentmarker}{}%
\end{pgfscope}%
\end{pgfscope}%
\begin{pgfscope}%
\definecolor{textcolor}{rgb}{0.000000,0.000000,0.000000}%
\pgfsetstrokecolor{textcolor}%
\pgfsetfillcolor{textcolor}%
\pgftext[x=2.819889in,y=0.430778in,,top]{\color{textcolor}\rmfamily\fontsize{10.000000}{12.000000}\selectfont \(\displaystyle {4000}\)}%
\end{pgfscope}%
\begin{pgfscope}%
\pgfsetbuttcap%
\pgfsetroundjoin%
\definecolor{currentfill}{rgb}{0.000000,0.000000,0.000000}%
\pgfsetfillcolor{currentfill}%
\pgfsetlinewidth{0.803000pt}%
\definecolor{currentstroke}{rgb}{0.000000,0.000000,0.000000}%
\pgfsetstrokecolor{currentstroke}%
\pgfsetdash{}{0pt}%
\pgfsys@defobject{currentmarker}{\pgfqpoint{0.000000in}{-0.048611in}}{\pgfqpoint{0.000000in}{0.000000in}}{%
\pgfpathmoveto{\pgfqpoint{0.000000in}{0.000000in}}%
\pgfpathlineto{\pgfqpoint{0.000000in}{-0.048611in}}%
\pgfusepath{stroke,fill}%
}%
\begin{pgfscope}%
\pgfsys@transformshift{3.740111in}{0.528000in}%
\pgfsys@useobject{currentmarker}{}%
\end{pgfscope}%
\end{pgfscope}%
\begin{pgfscope}%
\definecolor{textcolor}{rgb}{0.000000,0.000000,0.000000}%
\pgfsetstrokecolor{textcolor}%
\pgfsetfillcolor{textcolor}%
\pgftext[x=3.740111in,y=0.430778in,,top]{\color{textcolor}\rmfamily\fontsize{10.000000}{12.000000}\selectfont \(\displaystyle {6000}\)}%
\end{pgfscope}%
\begin{pgfscope}%
\pgfsetbuttcap%
\pgfsetroundjoin%
\definecolor{currentfill}{rgb}{0.000000,0.000000,0.000000}%
\pgfsetfillcolor{currentfill}%
\pgfsetlinewidth{0.803000pt}%
\definecolor{currentstroke}{rgb}{0.000000,0.000000,0.000000}%
\pgfsetstrokecolor{currentstroke}%
\pgfsetdash{}{0pt}%
\pgfsys@defobject{currentmarker}{\pgfqpoint{0.000000in}{-0.048611in}}{\pgfqpoint{0.000000in}{0.000000in}}{%
\pgfpathmoveto{\pgfqpoint{0.000000in}{0.000000in}}%
\pgfpathlineto{\pgfqpoint{0.000000in}{-0.048611in}}%
\pgfusepath{stroke,fill}%
}%
\begin{pgfscope}%
\pgfsys@transformshift{4.660334in}{0.528000in}%
\pgfsys@useobject{currentmarker}{}%
\end{pgfscope}%
\end{pgfscope}%
\begin{pgfscope}%
\definecolor{textcolor}{rgb}{0.000000,0.000000,0.000000}%
\pgfsetstrokecolor{textcolor}%
\pgfsetfillcolor{textcolor}%
\pgftext[x=4.660334in,y=0.430778in,,top]{\color{textcolor}\rmfamily\fontsize{10.000000}{12.000000}\selectfont \(\displaystyle {8000}\)}%
\end{pgfscope}%
\begin{pgfscope}%
\pgfsetbuttcap%
\pgfsetroundjoin%
\definecolor{currentfill}{rgb}{0.000000,0.000000,0.000000}%
\pgfsetfillcolor{currentfill}%
\pgfsetlinewidth{0.803000pt}%
\definecolor{currentstroke}{rgb}{0.000000,0.000000,0.000000}%
\pgfsetstrokecolor{currentstroke}%
\pgfsetdash{}{0pt}%
\pgfsys@defobject{currentmarker}{\pgfqpoint{0.000000in}{-0.048611in}}{\pgfqpoint{0.000000in}{0.000000in}}{%
\pgfpathmoveto{\pgfqpoint{0.000000in}{0.000000in}}%
\pgfpathlineto{\pgfqpoint{0.000000in}{-0.048611in}}%
\pgfusepath{stroke,fill}%
}%
\begin{pgfscope}%
\pgfsys@transformshift{5.580557in}{0.528000in}%
\pgfsys@useobject{currentmarker}{}%
\end{pgfscope}%
\end{pgfscope}%
\begin{pgfscope}%
\definecolor{textcolor}{rgb}{0.000000,0.000000,0.000000}%
\pgfsetstrokecolor{textcolor}%
\pgfsetfillcolor{textcolor}%
\pgftext[x=5.580557in,y=0.430778in,,top]{\color{textcolor}\rmfamily\fontsize{10.000000}{12.000000}\selectfont \(\displaystyle {10000}\)}%
\end{pgfscope}%
\begin{pgfscope}%
\definecolor{textcolor}{rgb}{0.000000,0.000000,0.000000}%
\pgfsetstrokecolor{textcolor}%
\pgfsetfillcolor{textcolor}%
\pgftext[x=3.280000in,y=0.251766in,,top]{\color{textcolor}\rmfamily\fontsize{10.000000}{12.000000}\selectfont Input Size}%
\end{pgfscope}%
\begin{pgfscope}%
\pgfsetbuttcap%
\pgfsetroundjoin%
\definecolor{currentfill}{rgb}{0.000000,0.000000,0.000000}%
\pgfsetfillcolor{currentfill}%
\pgfsetlinewidth{0.803000pt}%
\definecolor{currentstroke}{rgb}{0.000000,0.000000,0.000000}%
\pgfsetstrokecolor{currentstroke}%
\pgfsetdash{}{0pt}%
\pgfsys@defobject{currentmarker}{\pgfqpoint{-0.048611in}{0.000000in}}{\pgfqpoint{-0.000000in}{0.000000in}}{%
\pgfpathmoveto{\pgfqpoint{-0.000000in}{0.000000in}}%
\pgfpathlineto{\pgfqpoint{-0.048611in}{0.000000in}}%
\pgfusepath{stroke,fill}%
}%
\begin{pgfscope}%
\pgfsys@transformshift{0.800000in}{1.494648in}%
\pgfsys@useobject{currentmarker}{}%
\end{pgfscope}%
\end{pgfscope}%
\begin{pgfscope}%
\definecolor{textcolor}{rgb}{0.000000,0.000000,0.000000}%
\pgfsetstrokecolor{textcolor}%
\pgfsetfillcolor{textcolor}%
\pgftext[x=0.501581in, y=1.446423in, left, base]{\color{textcolor}\rmfamily\fontsize{10.000000}{12.000000}\selectfont \(\displaystyle {10^{3}}\)}%
\end{pgfscope}%
\begin{pgfscope}%
\pgfsetbuttcap%
\pgfsetroundjoin%
\definecolor{currentfill}{rgb}{0.000000,0.000000,0.000000}%
\pgfsetfillcolor{currentfill}%
\pgfsetlinewidth{0.803000pt}%
\definecolor{currentstroke}{rgb}{0.000000,0.000000,0.000000}%
\pgfsetstrokecolor{currentstroke}%
\pgfsetdash{}{0pt}%
\pgfsys@defobject{currentmarker}{\pgfqpoint{-0.048611in}{0.000000in}}{\pgfqpoint{-0.000000in}{0.000000in}}{%
\pgfpathmoveto{\pgfqpoint{-0.000000in}{0.000000in}}%
\pgfpathlineto{\pgfqpoint{-0.048611in}{0.000000in}}%
\pgfusepath{stroke,fill}%
}%
\begin{pgfscope}%
\pgfsys@transformshift{0.800000in}{2.931181in}%
\pgfsys@useobject{currentmarker}{}%
\end{pgfscope}%
\end{pgfscope}%
\begin{pgfscope}%
\definecolor{textcolor}{rgb}{0.000000,0.000000,0.000000}%
\pgfsetstrokecolor{textcolor}%
\pgfsetfillcolor{textcolor}%
\pgftext[x=0.501581in, y=2.882955in, left, base]{\color{textcolor}\rmfamily\fontsize{10.000000}{12.000000}\selectfont \(\displaystyle {10^{4}}\)}%
\end{pgfscope}%
\begin{pgfscope}%
\pgfsetbuttcap%
\pgfsetroundjoin%
\definecolor{currentfill}{rgb}{0.000000,0.000000,0.000000}%
\pgfsetfillcolor{currentfill}%
\pgfsetlinewidth{0.602250pt}%
\definecolor{currentstroke}{rgb}{0.000000,0.000000,0.000000}%
\pgfsetstrokecolor{currentstroke}%
\pgfsetdash{}{0pt}%
\pgfsys@defobject{currentmarker}{\pgfqpoint{-0.027778in}{0.000000in}}{\pgfqpoint{-0.000000in}{0.000000in}}{%
\pgfpathmoveto{\pgfqpoint{-0.000000in}{0.000000in}}%
\pgfpathlineto{\pgfqpoint{-0.027778in}{0.000000in}}%
\pgfusepath{stroke,fill}%
}%
\begin{pgfscope}%
\pgfsys@transformshift{0.800000in}{0.743515in}%
\pgfsys@useobject{currentmarker}{}%
\end{pgfscope}%
\end{pgfscope}%
\begin{pgfscope}%
\pgfsetbuttcap%
\pgfsetroundjoin%
\definecolor{currentfill}{rgb}{0.000000,0.000000,0.000000}%
\pgfsetfillcolor{currentfill}%
\pgfsetlinewidth{0.602250pt}%
\definecolor{currentstroke}{rgb}{0.000000,0.000000,0.000000}%
\pgfsetstrokecolor{currentstroke}%
\pgfsetdash{}{0pt}%
\pgfsys@defobject{currentmarker}{\pgfqpoint{-0.027778in}{0.000000in}}{\pgfqpoint{-0.000000in}{0.000000in}}{%
\pgfpathmoveto{\pgfqpoint{-0.000000in}{0.000000in}}%
\pgfpathlineto{\pgfqpoint{-0.027778in}{0.000000in}}%
\pgfusepath{stroke,fill}%
}%
\begin{pgfscope}%
\pgfsys@transformshift{0.800000in}{0.922994in}%
\pgfsys@useobject{currentmarker}{}%
\end{pgfscope}%
\end{pgfscope}%
\begin{pgfscope}%
\pgfsetbuttcap%
\pgfsetroundjoin%
\definecolor{currentfill}{rgb}{0.000000,0.000000,0.000000}%
\pgfsetfillcolor{currentfill}%
\pgfsetlinewidth{0.602250pt}%
\definecolor{currentstroke}{rgb}{0.000000,0.000000,0.000000}%
\pgfsetstrokecolor{currentstroke}%
\pgfsetdash{}{0pt}%
\pgfsys@defobject{currentmarker}{\pgfqpoint{-0.027778in}{0.000000in}}{\pgfqpoint{-0.000000in}{0.000000in}}{%
\pgfpathmoveto{\pgfqpoint{-0.000000in}{0.000000in}}%
\pgfpathlineto{\pgfqpoint{-0.027778in}{0.000000in}}%
\pgfusepath{stroke,fill}%
}%
\begin{pgfscope}%
\pgfsys@transformshift{0.800000in}{1.062208in}%
\pgfsys@useobject{currentmarker}{}%
\end{pgfscope}%
\end{pgfscope}%
\begin{pgfscope}%
\pgfsetbuttcap%
\pgfsetroundjoin%
\definecolor{currentfill}{rgb}{0.000000,0.000000,0.000000}%
\pgfsetfillcolor{currentfill}%
\pgfsetlinewidth{0.602250pt}%
\definecolor{currentstroke}{rgb}{0.000000,0.000000,0.000000}%
\pgfsetstrokecolor{currentstroke}%
\pgfsetdash{}{0pt}%
\pgfsys@defobject{currentmarker}{\pgfqpoint{-0.027778in}{0.000000in}}{\pgfqpoint{-0.000000in}{0.000000in}}{%
\pgfpathmoveto{\pgfqpoint{-0.000000in}{0.000000in}}%
\pgfpathlineto{\pgfqpoint{-0.027778in}{0.000000in}}%
\pgfusepath{stroke,fill}%
}%
\begin{pgfscope}%
\pgfsys@transformshift{0.800000in}{1.175955in}%
\pgfsys@useobject{currentmarker}{}%
\end{pgfscope}%
\end{pgfscope}%
\begin{pgfscope}%
\pgfsetbuttcap%
\pgfsetroundjoin%
\definecolor{currentfill}{rgb}{0.000000,0.000000,0.000000}%
\pgfsetfillcolor{currentfill}%
\pgfsetlinewidth{0.602250pt}%
\definecolor{currentstroke}{rgb}{0.000000,0.000000,0.000000}%
\pgfsetstrokecolor{currentstroke}%
\pgfsetdash{}{0pt}%
\pgfsys@defobject{currentmarker}{\pgfqpoint{-0.027778in}{0.000000in}}{\pgfqpoint{-0.000000in}{0.000000in}}{%
\pgfpathmoveto{\pgfqpoint{-0.000000in}{0.000000in}}%
\pgfpathlineto{\pgfqpoint{-0.027778in}{0.000000in}}%
\pgfusepath{stroke,fill}%
}%
\begin{pgfscope}%
\pgfsys@transformshift{0.800000in}{1.272126in}%
\pgfsys@useobject{currentmarker}{}%
\end{pgfscope}%
\end{pgfscope}%
\begin{pgfscope}%
\pgfsetbuttcap%
\pgfsetroundjoin%
\definecolor{currentfill}{rgb}{0.000000,0.000000,0.000000}%
\pgfsetfillcolor{currentfill}%
\pgfsetlinewidth{0.602250pt}%
\definecolor{currentstroke}{rgb}{0.000000,0.000000,0.000000}%
\pgfsetstrokecolor{currentstroke}%
\pgfsetdash{}{0pt}%
\pgfsys@defobject{currentmarker}{\pgfqpoint{-0.027778in}{0.000000in}}{\pgfqpoint{-0.000000in}{0.000000in}}{%
\pgfpathmoveto{\pgfqpoint{-0.000000in}{0.000000in}}%
\pgfpathlineto{\pgfqpoint{-0.027778in}{0.000000in}}%
\pgfusepath{stroke,fill}%
}%
\begin{pgfscope}%
\pgfsys@transformshift{0.800000in}{1.355433in}%
\pgfsys@useobject{currentmarker}{}%
\end{pgfscope}%
\end{pgfscope}%
\begin{pgfscope}%
\pgfsetbuttcap%
\pgfsetroundjoin%
\definecolor{currentfill}{rgb}{0.000000,0.000000,0.000000}%
\pgfsetfillcolor{currentfill}%
\pgfsetlinewidth{0.602250pt}%
\definecolor{currentstroke}{rgb}{0.000000,0.000000,0.000000}%
\pgfsetstrokecolor{currentstroke}%
\pgfsetdash{}{0pt}%
\pgfsys@defobject{currentmarker}{\pgfqpoint{-0.027778in}{0.000000in}}{\pgfqpoint{-0.000000in}{0.000000in}}{%
\pgfpathmoveto{\pgfqpoint{-0.000000in}{0.000000in}}%
\pgfpathlineto{\pgfqpoint{-0.027778in}{0.000000in}}%
\pgfusepath{stroke,fill}%
}%
\begin{pgfscope}%
\pgfsys@transformshift{0.800000in}{1.428916in}%
\pgfsys@useobject{currentmarker}{}%
\end{pgfscope}%
\end{pgfscope}%
\begin{pgfscope}%
\pgfsetbuttcap%
\pgfsetroundjoin%
\definecolor{currentfill}{rgb}{0.000000,0.000000,0.000000}%
\pgfsetfillcolor{currentfill}%
\pgfsetlinewidth{0.602250pt}%
\definecolor{currentstroke}{rgb}{0.000000,0.000000,0.000000}%
\pgfsetstrokecolor{currentstroke}%
\pgfsetdash{}{0pt}%
\pgfsys@defobject{currentmarker}{\pgfqpoint{-0.027778in}{0.000000in}}{\pgfqpoint{-0.000000in}{0.000000in}}{%
\pgfpathmoveto{\pgfqpoint{-0.000000in}{0.000000in}}%
\pgfpathlineto{\pgfqpoint{-0.027778in}{0.000000in}}%
\pgfusepath{stroke,fill}%
}%
\begin{pgfscope}%
\pgfsys@transformshift{0.800000in}{1.927087in}%
\pgfsys@useobject{currentmarker}{}%
\end{pgfscope}%
\end{pgfscope}%
\begin{pgfscope}%
\pgfsetbuttcap%
\pgfsetroundjoin%
\definecolor{currentfill}{rgb}{0.000000,0.000000,0.000000}%
\pgfsetfillcolor{currentfill}%
\pgfsetlinewidth{0.602250pt}%
\definecolor{currentstroke}{rgb}{0.000000,0.000000,0.000000}%
\pgfsetstrokecolor{currentstroke}%
\pgfsetdash{}{0pt}%
\pgfsys@defobject{currentmarker}{\pgfqpoint{-0.027778in}{0.000000in}}{\pgfqpoint{-0.000000in}{0.000000in}}{%
\pgfpathmoveto{\pgfqpoint{-0.000000in}{0.000000in}}%
\pgfpathlineto{\pgfqpoint{-0.027778in}{0.000000in}}%
\pgfusepath{stroke,fill}%
}%
\begin{pgfscope}%
\pgfsys@transformshift{0.800000in}{2.180048in}%
\pgfsys@useobject{currentmarker}{}%
\end{pgfscope}%
\end{pgfscope}%
\begin{pgfscope}%
\pgfsetbuttcap%
\pgfsetroundjoin%
\definecolor{currentfill}{rgb}{0.000000,0.000000,0.000000}%
\pgfsetfillcolor{currentfill}%
\pgfsetlinewidth{0.602250pt}%
\definecolor{currentstroke}{rgb}{0.000000,0.000000,0.000000}%
\pgfsetstrokecolor{currentstroke}%
\pgfsetdash{}{0pt}%
\pgfsys@defobject{currentmarker}{\pgfqpoint{-0.027778in}{0.000000in}}{\pgfqpoint{-0.000000in}{0.000000in}}{%
\pgfpathmoveto{\pgfqpoint{-0.000000in}{0.000000in}}%
\pgfpathlineto{\pgfqpoint{-0.027778in}{0.000000in}}%
\pgfusepath{stroke,fill}%
}%
\begin{pgfscope}%
\pgfsys@transformshift{0.800000in}{2.359527in}%
\pgfsys@useobject{currentmarker}{}%
\end{pgfscope}%
\end{pgfscope}%
\begin{pgfscope}%
\pgfsetbuttcap%
\pgfsetroundjoin%
\definecolor{currentfill}{rgb}{0.000000,0.000000,0.000000}%
\pgfsetfillcolor{currentfill}%
\pgfsetlinewidth{0.602250pt}%
\definecolor{currentstroke}{rgb}{0.000000,0.000000,0.000000}%
\pgfsetstrokecolor{currentstroke}%
\pgfsetdash{}{0pt}%
\pgfsys@defobject{currentmarker}{\pgfqpoint{-0.027778in}{0.000000in}}{\pgfqpoint{-0.000000in}{0.000000in}}{%
\pgfpathmoveto{\pgfqpoint{-0.000000in}{0.000000in}}%
\pgfpathlineto{\pgfqpoint{-0.027778in}{0.000000in}}%
\pgfusepath{stroke,fill}%
}%
\begin{pgfscope}%
\pgfsys@transformshift{0.800000in}{2.498741in}%
\pgfsys@useobject{currentmarker}{}%
\end{pgfscope}%
\end{pgfscope}%
\begin{pgfscope}%
\pgfsetbuttcap%
\pgfsetroundjoin%
\definecolor{currentfill}{rgb}{0.000000,0.000000,0.000000}%
\pgfsetfillcolor{currentfill}%
\pgfsetlinewidth{0.602250pt}%
\definecolor{currentstroke}{rgb}{0.000000,0.000000,0.000000}%
\pgfsetstrokecolor{currentstroke}%
\pgfsetdash{}{0pt}%
\pgfsys@defobject{currentmarker}{\pgfqpoint{-0.027778in}{0.000000in}}{\pgfqpoint{-0.000000in}{0.000000in}}{%
\pgfpathmoveto{\pgfqpoint{-0.000000in}{0.000000in}}%
\pgfpathlineto{\pgfqpoint{-0.027778in}{0.000000in}}%
\pgfusepath{stroke,fill}%
}%
\begin{pgfscope}%
\pgfsys@transformshift{0.800000in}{2.612488in}%
\pgfsys@useobject{currentmarker}{}%
\end{pgfscope}%
\end{pgfscope}%
\begin{pgfscope}%
\pgfsetbuttcap%
\pgfsetroundjoin%
\definecolor{currentfill}{rgb}{0.000000,0.000000,0.000000}%
\pgfsetfillcolor{currentfill}%
\pgfsetlinewidth{0.602250pt}%
\definecolor{currentstroke}{rgb}{0.000000,0.000000,0.000000}%
\pgfsetstrokecolor{currentstroke}%
\pgfsetdash{}{0pt}%
\pgfsys@defobject{currentmarker}{\pgfqpoint{-0.027778in}{0.000000in}}{\pgfqpoint{-0.000000in}{0.000000in}}{%
\pgfpathmoveto{\pgfqpoint{-0.000000in}{0.000000in}}%
\pgfpathlineto{\pgfqpoint{-0.027778in}{0.000000in}}%
\pgfusepath{stroke,fill}%
}%
\begin{pgfscope}%
\pgfsys@transformshift{0.800000in}{2.708659in}%
\pgfsys@useobject{currentmarker}{}%
\end{pgfscope}%
\end{pgfscope}%
\begin{pgfscope}%
\pgfsetbuttcap%
\pgfsetroundjoin%
\definecolor{currentfill}{rgb}{0.000000,0.000000,0.000000}%
\pgfsetfillcolor{currentfill}%
\pgfsetlinewidth{0.602250pt}%
\definecolor{currentstroke}{rgb}{0.000000,0.000000,0.000000}%
\pgfsetstrokecolor{currentstroke}%
\pgfsetdash{}{0pt}%
\pgfsys@defobject{currentmarker}{\pgfqpoint{-0.027778in}{0.000000in}}{\pgfqpoint{-0.000000in}{0.000000in}}{%
\pgfpathmoveto{\pgfqpoint{-0.000000in}{0.000000in}}%
\pgfpathlineto{\pgfqpoint{-0.027778in}{0.000000in}}%
\pgfusepath{stroke,fill}%
}%
\begin{pgfscope}%
\pgfsys@transformshift{0.800000in}{2.791966in}%
\pgfsys@useobject{currentmarker}{}%
\end{pgfscope}%
\end{pgfscope}%
\begin{pgfscope}%
\pgfsetbuttcap%
\pgfsetroundjoin%
\definecolor{currentfill}{rgb}{0.000000,0.000000,0.000000}%
\pgfsetfillcolor{currentfill}%
\pgfsetlinewidth{0.602250pt}%
\definecolor{currentstroke}{rgb}{0.000000,0.000000,0.000000}%
\pgfsetstrokecolor{currentstroke}%
\pgfsetdash{}{0pt}%
\pgfsys@defobject{currentmarker}{\pgfqpoint{-0.027778in}{0.000000in}}{\pgfqpoint{-0.000000in}{0.000000in}}{%
\pgfpathmoveto{\pgfqpoint{-0.000000in}{0.000000in}}%
\pgfpathlineto{\pgfqpoint{-0.027778in}{0.000000in}}%
\pgfusepath{stroke,fill}%
}%
\begin{pgfscope}%
\pgfsys@transformshift{0.800000in}{2.865449in}%
\pgfsys@useobject{currentmarker}{}%
\end{pgfscope}%
\end{pgfscope}%
\begin{pgfscope}%
\pgfsetbuttcap%
\pgfsetroundjoin%
\definecolor{currentfill}{rgb}{0.000000,0.000000,0.000000}%
\pgfsetfillcolor{currentfill}%
\pgfsetlinewidth{0.602250pt}%
\definecolor{currentstroke}{rgb}{0.000000,0.000000,0.000000}%
\pgfsetstrokecolor{currentstroke}%
\pgfsetdash{}{0pt}%
\pgfsys@defobject{currentmarker}{\pgfqpoint{-0.027778in}{0.000000in}}{\pgfqpoint{-0.000000in}{0.000000in}}{%
\pgfpathmoveto{\pgfqpoint{-0.000000in}{0.000000in}}%
\pgfpathlineto{\pgfqpoint{-0.027778in}{0.000000in}}%
\pgfusepath{stroke,fill}%
}%
\begin{pgfscope}%
\pgfsys@transformshift{0.800000in}{3.363620in}%
\pgfsys@useobject{currentmarker}{}%
\end{pgfscope}%
\end{pgfscope}%
\begin{pgfscope}%
\pgfsetbuttcap%
\pgfsetroundjoin%
\definecolor{currentfill}{rgb}{0.000000,0.000000,0.000000}%
\pgfsetfillcolor{currentfill}%
\pgfsetlinewidth{0.602250pt}%
\definecolor{currentstroke}{rgb}{0.000000,0.000000,0.000000}%
\pgfsetstrokecolor{currentstroke}%
\pgfsetdash{}{0pt}%
\pgfsys@defobject{currentmarker}{\pgfqpoint{-0.027778in}{0.000000in}}{\pgfqpoint{-0.000000in}{0.000000in}}{%
\pgfpathmoveto{\pgfqpoint{-0.000000in}{0.000000in}}%
\pgfpathlineto{\pgfqpoint{-0.027778in}{0.000000in}}%
\pgfusepath{stroke,fill}%
}%
\begin{pgfscope}%
\pgfsys@transformshift{0.800000in}{3.616581in}%
\pgfsys@useobject{currentmarker}{}%
\end{pgfscope}%
\end{pgfscope}%
\begin{pgfscope}%
\pgfsetbuttcap%
\pgfsetroundjoin%
\definecolor{currentfill}{rgb}{0.000000,0.000000,0.000000}%
\pgfsetfillcolor{currentfill}%
\pgfsetlinewidth{0.602250pt}%
\definecolor{currentstroke}{rgb}{0.000000,0.000000,0.000000}%
\pgfsetstrokecolor{currentstroke}%
\pgfsetdash{}{0pt}%
\pgfsys@defobject{currentmarker}{\pgfqpoint{-0.027778in}{0.000000in}}{\pgfqpoint{-0.000000in}{0.000000in}}{%
\pgfpathmoveto{\pgfqpoint{-0.000000in}{0.000000in}}%
\pgfpathlineto{\pgfqpoint{-0.027778in}{0.000000in}}%
\pgfusepath{stroke,fill}%
}%
\begin{pgfscope}%
\pgfsys@transformshift{0.800000in}{3.796060in}%
\pgfsys@useobject{currentmarker}{}%
\end{pgfscope}%
\end{pgfscope}%
\begin{pgfscope}%
\pgfsetbuttcap%
\pgfsetroundjoin%
\definecolor{currentfill}{rgb}{0.000000,0.000000,0.000000}%
\pgfsetfillcolor{currentfill}%
\pgfsetlinewidth{0.602250pt}%
\definecolor{currentstroke}{rgb}{0.000000,0.000000,0.000000}%
\pgfsetstrokecolor{currentstroke}%
\pgfsetdash{}{0pt}%
\pgfsys@defobject{currentmarker}{\pgfqpoint{-0.027778in}{0.000000in}}{\pgfqpoint{-0.000000in}{0.000000in}}{%
\pgfpathmoveto{\pgfqpoint{-0.000000in}{0.000000in}}%
\pgfpathlineto{\pgfqpoint{-0.027778in}{0.000000in}}%
\pgfusepath{stroke,fill}%
}%
\begin{pgfscope}%
\pgfsys@transformshift{0.800000in}{3.935274in}%
\pgfsys@useobject{currentmarker}{}%
\end{pgfscope}%
\end{pgfscope}%
\begin{pgfscope}%
\pgfsetbuttcap%
\pgfsetroundjoin%
\definecolor{currentfill}{rgb}{0.000000,0.000000,0.000000}%
\pgfsetfillcolor{currentfill}%
\pgfsetlinewidth{0.602250pt}%
\definecolor{currentstroke}{rgb}{0.000000,0.000000,0.000000}%
\pgfsetstrokecolor{currentstroke}%
\pgfsetdash{}{0pt}%
\pgfsys@defobject{currentmarker}{\pgfqpoint{-0.027778in}{0.000000in}}{\pgfqpoint{-0.000000in}{0.000000in}}{%
\pgfpathmoveto{\pgfqpoint{-0.000000in}{0.000000in}}%
\pgfpathlineto{\pgfqpoint{-0.027778in}{0.000000in}}%
\pgfusepath{stroke,fill}%
}%
\begin{pgfscope}%
\pgfsys@transformshift{0.800000in}{4.049021in}%
\pgfsys@useobject{currentmarker}{}%
\end{pgfscope}%
\end{pgfscope}%
\begin{pgfscope}%
\pgfsetbuttcap%
\pgfsetroundjoin%
\definecolor{currentfill}{rgb}{0.000000,0.000000,0.000000}%
\pgfsetfillcolor{currentfill}%
\pgfsetlinewidth{0.602250pt}%
\definecolor{currentstroke}{rgb}{0.000000,0.000000,0.000000}%
\pgfsetstrokecolor{currentstroke}%
\pgfsetdash{}{0pt}%
\pgfsys@defobject{currentmarker}{\pgfqpoint{-0.027778in}{0.000000in}}{\pgfqpoint{-0.000000in}{0.000000in}}{%
\pgfpathmoveto{\pgfqpoint{-0.000000in}{0.000000in}}%
\pgfpathlineto{\pgfqpoint{-0.027778in}{0.000000in}}%
\pgfusepath{stroke,fill}%
}%
\begin{pgfscope}%
\pgfsys@transformshift{0.800000in}{4.145192in}%
\pgfsys@useobject{currentmarker}{}%
\end{pgfscope}%
\end{pgfscope}%
\begin{pgfscope}%
\definecolor{textcolor}{rgb}{0.000000,0.000000,0.000000}%
\pgfsetstrokecolor{textcolor}%
\pgfsetfillcolor{textcolor}%
\pgftext[x=0.446026in,y=2.376000in,,bottom,rotate=90.000000]{\color{textcolor}\rmfamily\fontsize{10.000000}{12.000000}\selectfont Swaps}%
\end{pgfscope}%
\begin{pgfscope}%
\pgfpathrectangle{\pgfqpoint{0.800000in}{0.528000in}}{\pgfqpoint{4.960000in}{3.696000in}}%
\pgfusepath{clip}%
\pgfsetrectcap%
\pgfsetroundjoin%
\pgfsetlinewidth{1.505625pt}%
\definecolor{currentstroke}{rgb}{0.121569,0.466667,0.705882}%
\pgfsetstrokecolor{currentstroke}%
\pgfsetdash{}{0pt}%
\pgfpathmoveto{\pgfqpoint{1.025455in}{0.696000in}}%
\pgfpathlineto{\pgfqpoint{1.071466in}{1.255874in}}%
\pgfpathlineto{\pgfqpoint{1.117477in}{1.544393in}}%
\pgfpathlineto{\pgfqpoint{1.163488in}{1.751341in}}%
\pgfpathlineto{\pgfqpoint{1.209499in}{1.905507in}}%
\pgfpathlineto{\pgfqpoint{1.255510in}{2.063300in}}%
\pgfpathlineto{\pgfqpoint{1.301521in}{2.181917in}}%
\pgfpathlineto{\pgfqpoint{1.347532in}{2.263615in}}%
\pgfpathlineto{\pgfqpoint{1.393544in}{2.345329in}}%
\pgfpathlineto{\pgfqpoint{1.439555in}{2.421395in}}%
\pgfpathlineto{\pgfqpoint{1.485566in}{2.487409in}}%
\pgfpathlineto{\pgfqpoint{1.531577in}{2.564186in}}%
\pgfpathlineto{\pgfqpoint{1.577588in}{2.621469in}}%
\pgfpathlineto{\pgfqpoint{1.623599in}{2.680213in}}%
\pgfpathlineto{\pgfqpoint{1.669610in}{2.719876in}}%
\pgfpathlineto{\pgfqpoint{1.715622in}{2.765848in}}%
\pgfpathlineto{\pgfqpoint{1.761633in}{2.808360in}}%
\pgfpathlineto{\pgfqpoint{1.807644in}{2.845516in}}%
\pgfpathlineto{\pgfqpoint{1.853655in}{2.886844in}}%
\pgfpathlineto{\pgfqpoint{1.899666in}{2.920801in}}%
\pgfpathlineto{\pgfqpoint{1.945677in}{2.955050in}}%
\pgfpathlineto{\pgfqpoint{1.991688in}{2.990246in}}%
\pgfpathlineto{\pgfqpoint{2.037699in}{3.027102in}}%
\pgfpathlineto{\pgfqpoint{2.083711in}{3.054728in}}%
\pgfpathlineto{\pgfqpoint{2.129722in}{3.086652in}}%
\pgfpathlineto{\pgfqpoint{2.175733in}{3.113864in}}%
\pgfpathlineto{\pgfqpoint{2.221744in}{3.141722in}}%
\pgfpathlineto{\pgfqpoint{2.267755in}{3.166509in}}%
\pgfpathlineto{\pgfqpoint{2.313766in}{3.189608in}}%
\pgfpathlineto{\pgfqpoint{2.359777in}{3.210248in}}%
\pgfpathlineto{\pgfqpoint{2.405788in}{3.235268in}}%
\pgfpathlineto{\pgfqpoint{2.451800in}{3.259285in}}%
\pgfpathlineto{\pgfqpoint{2.497811in}{3.279171in}}%
\pgfpathlineto{\pgfqpoint{2.543822in}{3.302653in}}%
\pgfpathlineto{\pgfqpoint{2.589833in}{3.324453in}}%
\pgfpathlineto{\pgfqpoint{2.635844in}{3.340261in}}%
\pgfpathlineto{\pgfqpoint{2.681855in}{3.363994in}}%
\pgfpathlineto{\pgfqpoint{2.727866in}{3.377563in}}%
\pgfpathlineto{\pgfqpoint{2.773878in}{3.395454in}}%
\pgfpathlineto{\pgfqpoint{2.819889in}{3.410304in}}%
\pgfpathlineto{\pgfqpoint{2.865900in}{3.424414in}}%
\pgfpathlineto{\pgfqpoint{2.911911in}{3.446596in}}%
\pgfpathlineto{\pgfqpoint{2.957922in}{3.464152in}}%
\pgfpathlineto{\pgfqpoint{3.003933in}{3.483034in}}%
\pgfpathlineto{\pgfqpoint{3.049944in}{3.498578in}}%
\pgfpathlineto{\pgfqpoint{3.095955in}{3.514920in}}%
\pgfpathlineto{\pgfqpoint{3.141967in}{3.533393in}}%
\pgfpathlineto{\pgfqpoint{3.187978in}{3.547094in}}%
\pgfpathlineto{\pgfqpoint{3.233989in}{3.562977in}}%
\pgfpathlineto{\pgfqpoint{3.280000in}{3.577624in}}%
\pgfpathlineto{\pgfqpoint{3.326011in}{3.590593in}}%
\pgfpathlineto{\pgfqpoint{3.372022in}{3.604359in}}%
\pgfpathlineto{\pgfqpoint{3.418033in}{3.618885in}}%
\pgfpathlineto{\pgfqpoint{3.464045in}{3.631418in}}%
\pgfpathlineto{\pgfqpoint{3.510056in}{3.644659in}}%
\pgfpathlineto{\pgfqpoint{3.556067in}{3.656748in}}%
\pgfpathlineto{\pgfqpoint{3.602078in}{3.669525in}}%
\pgfpathlineto{\pgfqpoint{3.648089in}{3.680451in}}%
\pgfpathlineto{\pgfqpoint{3.694100in}{3.692628in}}%
\pgfpathlineto{\pgfqpoint{3.740111in}{3.702635in}}%
\pgfpathlineto{\pgfqpoint{3.786122in}{3.711968in}}%
\pgfpathlineto{\pgfqpoint{3.832134in}{3.726329in}}%
\pgfpathlineto{\pgfqpoint{3.878145in}{3.736878in}}%
\pgfpathlineto{\pgfqpoint{3.924156in}{3.747961in}}%
\pgfpathlineto{\pgfqpoint{3.970167in}{3.761377in}}%
\pgfpathlineto{\pgfqpoint{4.016178in}{3.770153in}}%
\pgfpathlineto{\pgfqpoint{4.062189in}{3.780936in}}%
\pgfpathlineto{\pgfqpoint{4.108200in}{3.791866in}}%
\pgfpathlineto{\pgfqpoint{4.154212in}{3.801758in}}%
\pgfpathlineto{\pgfqpoint{4.200223in}{3.811526in}}%
\pgfpathlineto{\pgfqpoint{4.246234in}{3.820694in}}%
\pgfpathlineto{\pgfqpoint{4.292245in}{3.828708in}}%
\pgfpathlineto{\pgfqpoint{4.338256in}{3.839625in}}%
\pgfpathlineto{\pgfqpoint{4.384267in}{3.848650in}}%
\pgfpathlineto{\pgfqpoint{4.430278in}{3.857560in}}%
\pgfpathlineto{\pgfqpoint{4.476289in}{3.865676in}}%
\pgfpathlineto{\pgfqpoint{4.522301in}{3.874417in}}%
\pgfpathlineto{\pgfqpoint{4.568312in}{3.884568in}}%
\pgfpathlineto{\pgfqpoint{4.614323in}{3.891874in}}%
\pgfpathlineto{\pgfqpoint{4.660334in}{3.900007in}}%
\pgfpathlineto{\pgfqpoint{4.706345in}{3.904494in}}%
\pgfpathlineto{\pgfqpoint{4.752356in}{3.911468in}}%
\pgfpathlineto{\pgfqpoint{4.798367in}{3.922097in}}%
\pgfpathlineto{\pgfqpoint{4.844378in}{3.932999in}}%
\pgfpathlineto{\pgfqpoint{4.890390in}{3.942778in}}%
\pgfpathlineto{\pgfqpoint{4.936401in}{3.951944in}}%
\pgfpathlineto{\pgfqpoint{4.982412in}{3.960486in}}%
\pgfpathlineto{\pgfqpoint{5.028423in}{3.969280in}}%
\pgfpathlineto{\pgfqpoint{5.074434in}{3.978708in}}%
\pgfpathlineto{\pgfqpoint{5.120445in}{3.987319in}}%
\pgfpathlineto{\pgfqpoint{5.166456in}{3.993794in}}%
\pgfpathlineto{\pgfqpoint{5.212468in}{4.001236in}}%
\pgfpathlineto{\pgfqpoint{5.258479in}{4.010130in}}%
\pgfpathlineto{\pgfqpoint{5.304490in}{4.018419in}}%
\pgfpathlineto{\pgfqpoint{5.350501in}{4.028408in}}%
\pgfpathlineto{\pgfqpoint{5.396512in}{4.033630in}}%
\pgfpathlineto{\pgfqpoint{5.442523in}{4.042078in}}%
\pgfpathlineto{\pgfqpoint{5.488534in}{4.049997in}}%
\pgfpathlineto{\pgfqpoint{5.534545in}{4.056000in}}%
\pgfusepath{stroke}%
\end{pgfscope}%
\begin{pgfscope}%
\pgfpathrectangle{\pgfqpoint{0.800000in}{0.528000in}}{\pgfqpoint{4.960000in}{3.696000in}}%
\pgfusepath{clip}%
\pgfsetrectcap%
\pgfsetroundjoin%
\pgfsetlinewidth{1.505625pt}%
\definecolor{currentstroke}{rgb}{1.000000,0.498039,0.054902}%
\pgfsetstrokecolor{currentstroke}%
\pgfsetdash{}{0pt}%
\pgfusepath{stroke}%
\end{pgfscope}%
\begin{pgfscope}%
\pgfsetrectcap%
\pgfsetmiterjoin%
\pgfsetlinewidth{0.803000pt}%
\definecolor{currentstroke}{rgb}{0.000000,0.000000,0.000000}%
\pgfsetstrokecolor{currentstroke}%
\pgfsetdash{}{0pt}%
\pgfpathmoveto{\pgfqpoint{0.800000in}{0.528000in}}%
\pgfpathlineto{\pgfqpoint{0.800000in}{4.224000in}}%
\pgfusepath{stroke}%
\end{pgfscope}%
\begin{pgfscope}%
\pgfsetrectcap%
\pgfsetmiterjoin%
\pgfsetlinewidth{0.803000pt}%
\definecolor{currentstroke}{rgb}{0.000000,0.000000,0.000000}%
\pgfsetstrokecolor{currentstroke}%
\pgfsetdash{}{0pt}%
\pgfpathmoveto{\pgfqpoint{5.760000in}{0.528000in}}%
\pgfpathlineto{\pgfqpoint{5.760000in}{4.224000in}}%
\pgfusepath{stroke}%
\end{pgfscope}%
\begin{pgfscope}%
\pgfsetrectcap%
\pgfsetmiterjoin%
\pgfsetlinewidth{0.803000pt}%
\definecolor{currentstroke}{rgb}{0.000000,0.000000,0.000000}%
\pgfsetstrokecolor{currentstroke}%
\pgfsetdash{}{0pt}%
\pgfpathmoveto{\pgfqpoint{0.800000in}{0.528000in}}%
\pgfpathlineto{\pgfqpoint{5.760000in}{0.528000in}}%
\pgfusepath{stroke}%
\end{pgfscope}%
\begin{pgfscope}%
\pgfsetrectcap%
\pgfsetmiterjoin%
\pgfsetlinewidth{0.803000pt}%
\definecolor{currentstroke}{rgb}{0.000000,0.000000,0.000000}%
\pgfsetstrokecolor{currentstroke}%
\pgfsetdash{}{0pt}%
\pgfpathmoveto{\pgfqpoint{0.800000in}{4.224000in}}%
\pgfpathlineto{\pgfqpoint{5.760000in}{4.224000in}}%
\pgfusepath{stroke}%
\end{pgfscope}%
\begin{pgfscope}%
\pgfsetbuttcap%
\pgfsetmiterjoin%
\definecolor{currentfill}{rgb}{1.000000,1.000000,1.000000}%
\pgfsetfillcolor{currentfill}%
\pgfsetfillopacity{0.800000}%
\pgfsetlinewidth{1.003750pt}%
\definecolor{currentstroke}{rgb}{0.800000,0.800000,0.800000}%
\pgfsetstrokecolor{currentstroke}%
\pgfsetstrokeopacity{0.800000}%
\pgfsetdash{}{0pt}%
\pgfpathmoveto{\pgfqpoint{0.897222in}{3.725543in}}%
\pgfpathlineto{\pgfqpoint{2.014507in}{3.725543in}}%
\pgfpathquadraticcurveto{\pgfqpoint{2.042285in}{3.725543in}}{\pgfqpoint{2.042285in}{3.753321in}}%
\pgfpathlineto{\pgfqpoint{2.042285in}{4.126778in}}%
\pgfpathquadraticcurveto{\pgfqpoint{2.042285in}{4.154556in}}{\pgfqpoint{2.014507in}{4.154556in}}%
\pgfpathlineto{\pgfqpoint{0.897222in}{4.154556in}}%
\pgfpathquadraticcurveto{\pgfqpoint{0.869444in}{4.154556in}}{\pgfqpoint{0.869444in}{4.126778in}}%
\pgfpathlineto{\pgfqpoint{0.869444in}{3.753321in}}%
\pgfpathquadraticcurveto{\pgfqpoint{0.869444in}{3.725543in}}{\pgfqpoint{0.897222in}{3.725543in}}%
\pgfpathlineto{\pgfqpoint{0.897222in}{3.725543in}}%
\pgfpathclose%
\pgfusepath{stroke,fill}%
\end{pgfscope}%
\begin{pgfscope}%
\pgfsetrectcap%
\pgfsetroundjoin%
\pgfsetlinewidth{1.505625pt}%
\definecolor{currentstroke}{rgb}{0.121569,0.466667,0.705882}%
\pgfsetstrokecolor{currentstroke}%
\pgfsetdash{}{0pt}%
\pgfpathmoveto{\pgfqpoint{0.925000in}{4.050389in}}%
\pgfpathlineto{\pgfqpoint{1.063889in}{4.050389in}}%
\pgfpathlineto{\pgfqpoint{1.202778in}{4.050389in}}%
\pgfusepath{stroke}%
\end{pgfscope}%
\begin{pgfscope}%
\definecolor{textcolor}{rgb}{0.000000,0.000000,0.000000}%
\pgfsetstrokecolor{textcolor}%
\pgfsetfillcolor{textcolor}%
\pgftext[x=1.313889in,y=4.001778in,left,base]{\color{textcolor}\rmfamily\fontsize{10.000000}{12.000000}\selectfont mergesort}%
\end{pgfscope}%
\begin{pgfscope}%
\pgfsetrectcap%
\pgfsetroundjoin%
\pgfsetlinewidth{1.505625pt}%
\definecolor{currentstroke}{rgb}{1.000000,0.498039,0.054902}%
\pgfsetstrokecolor{currentstroke}%
\pgfsetdash{}{0pt}%
\pgfpathmoveto{\pgfqpoint{0.925000in}{3.856716in}}%
\pgfpathlineto{\pgfqpoint{1.063889in}{3.856716in}}%
\pgfpathlineto{\pgfqpoint{1.202778in}{3.856716in}}%
\pgfusepath{stroke}%
\end{pgfscope}%
\begin{pgfscope}%
\definecolor{textcolor}{rgb}{0.000000,0.000000,0.000000}%
\pgfsetstrokecolor{textcolor}%
\pgfsetfillcolor{textcolor}%
\pgftext[x=1.313889in,y=3.808105in,left,base]{\color{textcolor}\rmfamily\fontsize{10.000000}{12.000000}\selectfont bmergesort}%
\end{pgfscope}%
\end{pgfpicture}%
\makeatother%
\endgroup%

\subsubsection{Basic Operations}
%% Creator: Matplotlib, PGF backend
%%
%% To include the figure in your LaTeX document, write
%%   \input{<filename>.pgf}
%%
%% Make sure the required packages are loaded in your preamble
%%   \usepackage{pgf}
%%
%% Also ensure that all the required font packages are loaded; for instance,
%% the lmodern package is sometimes necessary when using math font.
%%   \usepackage{lmodern}
%%
%% Figures using additional raster images can only be included by \input if
%% they are in the same directory as the main LaTeX file. For loading figures
%% from other directories you can use the `import` package
%%   \usepackage{import}
%%
%% and then include the figures with
%%   \import{<path to file>}{<filename>.pgf}
%%
%% Matplotlib used the following preamble
%%   
%%   \makeatletter\@ifpackageloaded{underscore}{}{\usepackage[strings]{underscore}}\makeatother
%%
\begingroup%
\makeatletter%
\begin{pgfpicture}%
\pgfpathrectangle{\pgfpointorigin}{\pgfqpoint{6.400000in}{4.800000in}}%
\pgfusepath{use as bounding box, clip}%
\begin{pgfscope}%
\pgfsetbuttcap%
\pgfsetmiterjoin%
\definecolor{currentfill}{rgb}{1.000000,1.000000,1.000000}%
\pgfsetfillcolor{currentfill}%
\pgfsetlinewidth{0.000000pt}%
\definecolor{currentstroke}{rgb}{1.000000,1.000000,1.000000}%
\pgfsetstrokecolor{currentstroke}%
\pgfsetdash{}{0pt}%
\pgfpathmoveto{\pgfqpoint{0.000000in}{0.000000in}}%
\pgfpathlineto{\pgfqpoint{6.400000in}{0.000000in}}%
\pgfpathlineto{\pgfqpoint{6.400000in}{4.800000in}}%
\pgfpathlineto{\pgfqpoint{0.000000in}{4.800000in}}%
\pgfpathlineto{\pgfqpoint{0.000000in}{0.000000in}}%
\pgfpathclose%
\pgfusepath{fill}%
\end{pgfscope}%
\begin{pgfscope}%
\pgfsetbuttcap%
\pgfsetmiterjoin%
\definecolor{currentfill}{rgb}{1.000000,1.000000,1.000000}%
\pgfsetfillcolor{currentfill}%
\pgfsetlinewidth{0.000000pt}%
\definecolor{currentstroke}{rgb}{0.000000,0.000000,0.000000}%
\pgfsetstrokecolor{currentstroke}%
\pgfsetstrokeopacity{0.000000}%
\pgfsetdash{}{0pt}%
\pgfpathmoveto{\pgfqpoint{0.800000in}{0.528000in}}%
\pgfpathlineto{\pgfqpoint{5.760000in}{0.528000in}}%
\pgfpathlineto{\pgfqpoint{5.760000in}{4.224000in}}%
\pgfpathlineto{\pgfqpoint{0.800000in}{4.224000in}}%
\pgfpathlineto{\pgfqpoint{0.800000in}{0.528000in}}%
\pgfpathclose%
\pgfusepath{fill}%
\end{pgfscope}%
\begin{pgfscope}%
\pgfsetbuttcap%
\pgfsetroundjoin%
\definecolor{currentfill}{rgb}{0.000000,0.000000,0.000000}%
\pgfsetfillcolor{currentfill}%
\pgfsetlinewidth{0.803000pt}%
\definecolor{currentstroke}{rgb}{0.000000,0.000000,0.000000}%
\pgfsetstrokecolor{currentstroke}%
\pgfsetdash{}{0pt}%
\pgfsys@defobject{currentmarker}{\pgfqpoint{0.000000in}{-0.048611in}}{\pgfqpoint{0.000000in}{0.000000in}}{%
\pgfpathmoveto{\pgfqpoint{0.000000in}{0.000000in}}%
\pgfpathlineto{\pgfqpoint{0.000000in}{-0.048611in}}%
\pgfusepath{stroke,fill}%
}%
\begin{pgfscope}%
\pgfsys@transformshift{0.979443in}{0.528000in}%
\pgfsys@useobject{currentmarker}{}%
\end{pgfscope}%
\end{pgfscope}%
\begin{pgfscope}%
\definecolor{textcolor}{rgb}{0.000000,0.000000,0.000000}%
\pgfsetstrokecolor{textcolor}%
\pgfsetfillcolor{textcolor}%
\pgftext[x=0.979443in,y=0.430778in,,top]{\color{textcolor}\rmfamily\fontsize{10.000000}{12.000000}\selectfont \(\displaystyle {0}\)}%
\end{pgfscope}%
\begin{pgfscope}%
\pgfsetbuttcap%
\pgfsetroundjoin%
\definecolor{currentfill}{rgb}{0.000000,0.000000,0.000000}%
\pgfsetfillcolor{currentfill}%
\pgfsetlinewidth{0.803000pt}%
\definecolor{currentstroke}{rgb}{0.000000,0.000000,0.000000}%
\pgfsetstrokecolor{currentstroke}%
\pgfsetdash{}{0pt}%
\pgfsys@defobject{currentmarker}{\pgfqpoint{0.000000in}{-0.048611in}}{\pgfqpoint{0.000000in}{0.000000in}}{%
\pgfpathmoveto{\pgfqpoint{0.000000in}{0.000000in}}%
\pgfpathlineto{\pgfqpoint{0.000000in}{-0.048611in}}%
\pgfusepath{stroke,fill}%
}%
\begin{pgfscope}%
\pgfsys@transformshift{1.899666in}{0.528000in}%
\pgfsys@useobject{currentmarker}{}%
\end{pgfscope}%
\end{pgfscope}%
\begin{pgfscope}%
\definecolor{textcolor}{rgb}{0.000000,0.000000,0.000000}%
\pgfsetstrokecolor{textcolor}%
\pgfsetfillcolor{textcolor}%
\pgftext[x=1.899666in,y=0.430778in,,top]{\color{textcolor}\rmfamily\fontsize{10.000000}{12.000000}\selectfont \(\displaystyle {2000}\)}%
\end{pgfscope}%
\begin{pgfscope}%
\pgfsetbuttcap%
\pgfsetroundjoin%
\definecolor{currentfill}{rgb}{0.000000,0.000000,0.000000}%
\pgfsetfillcolor{currentfill}%
\pgfsetlinewidth{0.803000pt}%
\definecolor{currentstroke}{rgb}{0.000000,0.000000,0.000000}%
\pgfsetstrokecolor{currentstroke}%
\pgfsetdash{}{0pt}%
\pgfsys@defobject{currentmarker}{\pgfqpoint{0.000000in}{-0.048611in}}{\pgfqpoint{0.000000in}{0.000000in}}{%
\pgfpathmoveto{\pgfqpoint{0.000000in}{0.000000in}}%
\pgfpathlineto{\pgfqpoint{0.000000in}{-0.048611in}}%
\pgfusepath{stroke,fill}%
}%
\begin{pgfscope}%
\pgfsys@transformshift{2.819889in}{0.528000in}%
\pgfsys@useobject{currentmarker}{}%
\end{pgfscope}%
\end{pgfscope}%
\begin{pgfscope}%
\definecolor{textcolor}{rgb}{0.000000,0.000000,0.000000}%
\pgfsetstrokecolor{textcolor}%
\pgfsetfillcolor{textcolor}%
\pgftext[x=2.819889in,y=0.430778in,,top]{\color{textcolor}\rmfamily\fontsize{10.000000}{12.000000}\selectfont \(\displaystyle {4000}\)}%
\end{pgfscope}%
\begin{pgfscope}%
\pgfsetbuttcap%
\pgfsetroundjoin%
\definecolor{currentfill}{rgb}{0.000000,0.000000,0.000000}%
\pgfsetfillcolor{currentfill}%
\pgfsetlinewidth{0.803000pt}%
\definecolor{currentstroke}{rgb}{0.000000,0.000000,0.000000}%
\pgfsetstrokecolor{currentstroke}%
\pgfsetdash{}{0pt}%
\pgfsys@defobject{currentmarker}{\pgfqpoint{0.000000in}{-0.048611in}}{\pgfqpoint{0.000000in}{0.000000in}}{%
\pgfpathmoveto{\pgfqpoint{0.000000in}{0.000000in}}%
\pgfpathlineto{\pgfqpoint{0.000000in}{-0.048611in}}%
\pgfusepath{stroke,fill}%
}%
\begin{pgfscope}%
\pgfsys@transformshift{3.740111in}{0.528000in}%
\pgfsys@useobject{currentmarker}{}%
\end{pgfscope}%
\end{pgfscope}%
\begin{pgfscope}%
\definecolor{textcolor}{rgb}{0.000000,0.000000,0.000000}%
\pgfsetstrokecolor{textcolor}%
\pgfsetfillcolor{textcolor}%
\pgftext[x=3.740111in,y=0.430778in,,top]{\color{textcolor}\rmfamily\fontsize{10.000000}{12.000000}\selectfont \(\displaystyle {6000}\)}%
\end{pgfscope}%
\begin{pgfscope}%
\pgfsetbuttcap%
\pgfsetroundjoin%
\definecolor{currentfill}{rgb}{0.000000,0.000000,0.000000}%
\pgfsetfillcolor{currentfill}%
\pgfsetlinewidth{0.803000pt}%
\definecolor{currentstroke}{rgb}{0.000000,0.000000,0.000000}%
\pgfsetstrokecolor{currentstroke}%
\pgfsetdash{}{0pt}%
\pgfsys@defobject{currentmarker}{\pgfqpoint{0.000000in}{-0.048611in}}{\pgfqpoint{0.000000in}{0.000000in}}{%
\pgfpathmoveto{\pgfqpoint{0.000000in}{0.000000in}}%
\pgfpathlineto{\pgfqpoint{0.000000in}{-0.048611in}}%
\pgfusepath{stroke,fill}%
}%
\begin{pgfscope}%
\pgfsys@transformshift{4.660334in}{0.528000in}%
\pgfsys@useobject{currentmarker}{}%
\end{pgfscope}%
\end{pgfscope}%
\begin{pgfscope}%
\definecolor{textcolor}{rgb}{0.000000,0.000000,0.000000}%
\pgfsetstrokecolor{textcolor}%
\pgfsetfillcolor{textcolor}%
\pgftext[x=4.660334in,y=0.430778in,,top]{\color{textcolor}\rmfamily\fontsize{10.000000}{12.000000}\selectfont \(\displaystyle {8000}\)}%
\end{pgfscope}%
\begin{pgfscope}%
\pgfsetbuttcap%
\pgfsetroundjoin%
\definecolor{currentfill}{rgb}{0.000000,0.000000,0.000000}%
\pgfsetfillcolor{currentfill}%
\pgfsetlinewidth{0.803000pt}%
\definecolor{currentstroke}{rgb}{0.000000,0.000000,0.000000}%
\pgfsetstrokecolor{currentstroke}%
\pgfsetdash{}{0pt}%
\pgfsys@defobject{currentmarker}{\pgfqpoint{0.000000in}{-0.048611in}}{\pgfqpoint{0.000000in}{0.000000in}}{%
\pgfpathmoveto{\pgfqpoint{0.000000in}{0.000000in}}%
\pgfpathlineto{\pgfqpoint{0.000000in}{-0.048611in}}%
\pgfusepath{stroke,fill}%
}%
\begin{pgfscope}%
\pgfsys@transformshift{5.580557in}{0.528000in}%
\pgfsys@useobject{currentmarker}{}%
\end{pgfscope}%
\end{pgfscope}%
\begin{pgfscope}%
\definecolor{textcolor}{rgb}{0.000000,0.000000,0.000000}%
\pgfsetstrokecolor{textcolor}%
\pgfsetfillcolor{textcolor}%
\pgftext[x=5.580557in,y=0.430778in,,top]{\color{textcolor}\rmfamily\fontsize{10.000000}{12.000000}\selectfont \(\displaystyle {10000}\)}%
\end{pgfscope}%
\begin{pgfscope}%
\definecolor{textcolor}{rgb}{0.000000,0.000000,0.000000}%
\pgfsetstrokecolor{textcolor}%
\pgfsetfillcolor{textcolor}%
\pgftext[x=3.280000in,y=0.251766in,,top]{\color{textcolor}\rmfamily\fontsize{10.000000}{12.000000}\selectfont Input Size}%
\end{pgfscope}%
\begin{pgfscope}%
\pgfsetbuttcap%
\pgfsetroundjoin%
\definecolor{currentfill}{rgb}{0.000000,0.000000,0.000000}%
\pgfsetfillcolor{currentfill}%
\pgfsetlinewidth{0.803000pt}%
\definecolor{currentstroke}{rgb}{0.000000,0.000000,0.000000}%
\pgfsetstrokecolor{currentstroke}%
\pgfsetdash{}{0pt}%
\pgfsys@defobject{currentmarker}{\pgfqpoint{-0.048611in}{0.000000in}}{\pgfqpoint{-0.000000in}{0.000000in}}{%
\pgfpathmoveto{\pgfqpoint{-0.000000in}{0.000000in}}%
\pgfpathlineto{\pgfqpoint{-0.048611in}{0.000000in}}%
\pgfusepath{stroke,fill}%
}%
\begin{pgfscope}%
\pgfsys@transformshift{0.800000in}{0.950704in}%
\pgfsys@useobject{currentmarker}{}%
\end{pgfscope}%
\end{pgfscope}%
\begin{pgfscope}%
\definecolor{textcolor}{rgb}{0.000000,0.000000,0.000000}%
\pgfsetstrokecolor{textcolor}%
\pgfsetfillcolor{textcolor}%
\pgftext[x=0.501581in, y=0.902479in, left, base]{\color{textcolor}\rmfamily\fontsize{10.000000}{12.000000}\selectfont \(\displaystyle {10^{4}}\)}%
\end{pgfscope}%
\begin{pgfscope}%
\pgfsetbuttcap%
\pgfsetroundjoin%
\definecolor{currentfill}{rgb}{0.000000,0.000000,0.000000}%
\pgfsetfillcolor{currentfill}%
\pgfsetlinewidth{0.803000pt}%
\definecolor{currentstroke}{rgb}{0.000000,0.000000,0.000000}%
\pgfsetstrokecolor{currentstroke}%
\pgfsetdash{}{0pt}%
\pgfsys@defobject{currentmarker}{\pgfqpoint{-0.048611in}{0.000000in}}{\pgfqpoint{-0.000000in}{0.000000in}}{%
\pgfpathmoveto{\pgfqpoint{-0.000000in}{0.000000in}}%
\pgfpathlineto{\pgfqpoint{-0.048611in}{0.000000in}}%
\pgfusepath{stroke,fill}%
}%
\begin{pgfscope}%
\pgfsys@transformshift{0.800000in}{2.408084in}%
\pgfsys@useobject{currentmarker}{}%
\end{pgfscope}%
\end{pgfscope}%
\begin{pgfscope}%
\definecolor{textcolor}{rgb}{0.000000,0.000000,0.000000}%
\pgfsetstrokecolor{textcolor}%
\pgfsetfillcolor{textcolor}%
\pgftext[x=0.501581in, y=2.359858in, left, base]{\color{textcolor}\rmfamily\fontsize{10.000000}{12.000000}\selectfont \(\displaystyle {10^{5}}\)}%
\end{pgfscope}%
\begin{pgfscope}%
\pgfsetbuttcap%
\pgfsetroundjoin%
\definecolor{currentfill}{rgb}{0.000000,0.000000,0.000000}%
\pgfsetfillcolor{currentfill}%
\pgfsetlinewidth{0.803000pt}%
\definecolor{currentstroke}{rgb}{0.000000,0.000000,0.000000}%
\pgfsetstrokecolor{currentstroke}%
\pgfsetdash{}{0pt}%
\pgfsys@defobject{currentmarker}{\pgfqpoint{-0.048611in}{0.000000in}}{\pgfqpoint{-0.000000in}{0.000000in}}{%
\pgfpathmoveto{\pgfqpoint{-0.000000in}{0.000000in}}%
\pgfpathlineto{\pgfqpoint{-0.048611in}{0.000000in}}%
\pgfusepath{stroke,fill}%
}%
\begin{pgfscope}%
\pgfsys@transformshift{0.800000in}{3.865463in}%
\pgfsys@useobject{currentmarker}{}%
\end{pgfscope}%
\end{pgfscope}%
\begin{pgfscope}%
\definecolor{textcolor}{rgb}{0.000000,0.000000,0.000000}%
\pgfsetstrokecolor{textcolor}%
\pgfsetfillcolor{textcolor}%
\pgftext[x=0.501581in, y=3.817238in, left, base]{\color{textcolor}\rmfamily\fontsize{10.000000}{12.000000}\selectfont \(\displaystyle {10^{6}}\)}%
\end{pgfscope}%
\begin{pgfscope}%
\pgfsetbuttcap%
\pgfsetroundjoin%
\definecolor{currentfill}{rgb}{0.000000,0.000000,0.000000}%
\pgfsetfillcolor{currentfill}%
\pgfsetlinewidth{0.602250pt}%
\definecolor{currentstroke}{rgb}{0.000000,0.000000,0.000000}%
\pgfsetstrokecolor{currentstroke}%
\pgfsetdash{}{0pt}%
\pgfsys@defobject{currentmarker}{\pgfqpoint{-0.027778in}{0.000000in}}{\pgfqpoint{-0.000000in}{0.000000in}}{%
\pgfpathmoveto{\pgfqpoint{-0.000000in}{0.000000in}}%
\pgfpathlineto{\pgfqpoint{-0.027778in}{0.000000in}}%
\pgfusepath{stroke,fill}%
}%
\begin{pgfscope}%
\pgfsys@transformshift{0.800000in}{0.627386in}%
\pgfsys@useobject{currentmarker}{}%
\end{pgfscope}%
\end{pgfscope}%
\begin{pgfscope}%
\pgfsetbuttcap%
\pgfsetroundjoin%
\definecolor{currentfill}{rgb}{0.000000,0.000000,0.000000}%
\pgfsetfillcolor{currentfill}%
\pgfsetlinewidth{0.602250pt}%
\definecolor{currentstroke}{rgb}{0.000000,0.000000,0.000000}%
\pgfsetstrokecolor{currentstroke}%
\pgfsetdash{}{0pt}%
\pgfsys@defobject{currentmarker}{\pgfqpoint{-0.027778in}{0.000000in}}{\pgfqpoint{-0.000000in}{0.000000in}}{%
\pgfpathmoveto{\pgfqpoint{-0.000000in}{0.000000in}}%
\pgfpathlineto{\pgfqpoint{-0.027778in}{0.000000in}}%
\pgfusepath{stroke,fill}%
}%
\begin{pgfscope}%
\pgfsys@transformshift{0.800000in}{0.724953in}%
\pgfsys@useobject{currentmarker}{}%
\end{pgfscope}%
\end{pgfscope}%
\begin{pgfscope}%
\pgfsetbuttcap%
\pgfsetroundjoin%
\definecolor{currentfill}{rgb}{0.000000,0.000000,0.000000}%
\pgfsetfillcolor{currentfill}%
\pgfsetlinewidth{0.602250pt}%
\definecolor{currentstroke}{rgb}{0.000000,0.000000,0.000000}%
\pgfsetstrokecolor{currentstroke}%
\pgfsetdash{}{0pt}%
\pgfsys@defobject{currentmarker}{\pgfqpoint{-0.027778in}{0.000000in}}{\pgfqpoint{-0.000000in}{0.000000in}}{%
\pgfpathmoveto{\pgfqpoint{-0.000000in}{0.000000in}}%
\pgfpathlineto{\pgfqpoint{-0.027778in}{0.000000in}}%
\pgfusepath{stroke,fill}%
}%
\begin{pgfscope}%
\pgfsys@transformshift{0.800000in}{0.809470in}%
\pgfsys@useobject{currentmarker}{}%
\end{pgfscope}%
\end{pgfscope}%
\begin{pgfscope}%
\pgfsetbuttcap%
\pgfsetroundjoin%
\definecolor{currentfill}{rgb}{0.000000,0.000000,0.000000}%
\pgfsetfillcolor{currentfill}%
\pgfsetlinewidth{0.602250pt}%
\definecolor{currentstroke}{rgb}{0.000000,0.000000,0.000000}%
\pgfsetstrokecolor{currentstroke}%
\pgfsetdash{}{0pt}%
\pgfsys@defobject{currentmarker}{\pgfqpoint{-0.027778in}{0.000000in}}{\pgfqpoint{-0.000000in}{0.000000in}}{%
\pgfpathmoveto{\pgfqpoint{-0.000000in}{0.000000in}}%
\pgfpathlineto{\pgfqpoint{-0.027778in}{0.000000in}}%
\pgfusepath{stroke,fill}%
}%
\begin{pgfscope}%
\pgfsys@transformshift{0.800000in}{0.884018in}%
\pgfsys@useobject{currentmarker}{}%
\end{pgfscope}%
\end{pgfscope}%
\begin{pgfscope}%
\pgfsetbuttcap%
\pgfsetroundjoin%
\definecolor{currentfill}{rgb}{0.000000,0.000000,0.000000}%
\pgfsetfillcolor{currentfill}%
\pgfsetlinewidth{0.602250pt}%
\definecolor{currentstroke}{rgb}{0.000000,0.000000,0.000000}%
\pgfsetstrokecolor{currentstroke}%
\pgfsetdash{}{0pt}%
\pgfsys@defobject{currentmarker}{\pgfqpoint{-0.027778in}{0.000000in}}{\pgfqpoint{-0.000000in}{0.000000in}}{%
\pgfpathmoveto{\pgfqpoint{-0.000000in}{0.000000in}}%
\pgfpathlineto{\pgfqpoint{-0.027778in}{0.000000in}}%
\pgfusepath{stroke,fill}%
}%
\begin{pgfscope}%
\pgfsys@transformshift{0.800000in}{1.389419in}%
\pgfsys@useobject{currentmarker}{}%
\end{pgfscope}%
\end{pgfscope}%
\begin{pgfscope}%
\pgfsetbuttcap%
\pgfsetroundjoin%
\definecolor{currentfill}{rgb}{0.000000,0.000000,0.000000}%
\pgfsetfillcolor{currentfill}%
\pgfsetlinewidth{0.602250pt}%
\definecolor{currentstroke}{rgb}{0.000000,0.000000,0.000000}%
\pgfsetstrokecolor{currentstroke}%
\pgfsetdash{}{0pt}%
\pgfsys@defobject{currentmarker}{\pgfqpoint{-0.027778in}{0.000000in}}{\pgfqpoint{-0.000000in}{0.000000in}}{%
\pgfpathmoveto{\pgfqpoint{-0.000000in}{0.000000in}}%
\pgfpathlineto{\pgfqpoint{-0.027778in}{0.000000in}}%
\pgfusepath{stroke,fill}%
}%
\begin{pgfscope}%
\pgfsys@transformshift{0.800000in}{1.646051in}%
\pgfsys@useobject{currentmarker}{}%
\end{pgfscope}%
\end{pgfscope}%
\begin{pgfscope}%
\pgfsetbuttcap%
\pgfsetroundjoin%
\definecolor{currentfill}{rgb}{0.000000,0.000000,0.000000}%
\pgfsetfillcolor{currentfill}%
\pgfsetlinewidth{0.602250pt}%
\definecolor{currentstroke}{rgb}{0.000000,0.000000,0.000000}%
\pgfsetstrokecolor{currentstroke}%
\pgfsetdash{}{0pt}%
\pgfsys@defobject{currentmarker}{\pgfqpoint{-0.027778in}{0.000000in}}{\pgfqpoint{-0.000000in}{0.000000in}}{%
\pgfpathmoveto{\pgfqpoint{-0.000000in}{0.000000in}}%
\pgfpathlineto{\pgfqpoint{-0.027778in}{0.000000in}}%
\pgfusepath{stroke,fill}%
}%
\begin{pgfscope}%
\pgfsys@transformshift{0.800000in}{1.828134in}%
\pgfsys@useobject{currentmarker}{}%
\end{pgfscope}%
\end{pgfscope}%
\begin{pgfscope}%
\pgfsetbuttcap%
\pgfsetroundjoin%
\definecolor{currentfill}{rgb}{0.000000,0.000000,0.000000}%
\pgfsetfillcolor{currentfill}%
\pgfsetlinewidth{0.602250pt}%
\definecolor{currentstroke}{rgb}{0.000000,0.000000,0.000000}%
\pgfsetstrokecolor{currentstroke}%
\pgfsetdash{}{0pt}%
\pgfsys@defobject{currentmarker}{\pgfqpoint{-0.027778in}{0.000000in}}{\pgfqpoint{-0.000000in}{0.000000in}}{%
\pgfpathmoveto{\pgfqpoint{-0.000000in}{0.000000in}}%
\pgfpathlineto{\pgfqpoint{-0.027778in}{0.000000in}}%
\pgfusepath{stroke,fill}%
}%
\begin{pgfscope}%
\pgfsys@transformshift{0.800000in}{1.969369in}%
\pgfsys@useobject{currentmarker}{}%
\end{pgfscope}%
\end{pgfscope}%
\begin{pgfscope}%
\pgfsetbuttcap%
\pgfsetroundjoin%
\definecolor{currentfill}{rgb}{0.000000,0.000000,0.000000}%
\pgfsetfillcolor{currentfill}%
\pgfsetlinewidth{0.602250pt}%
\definecolor{currentstroke}{rgb}{0.000000,0.000000,0.000000}%
\pgfsetstrokecolor{currentstroke}%
\pgfsetdash{}{0pt}%
\pgfsys@defobject{currentmarker}{\pgfqpoint{-0.027778in}{0.000000in}}{\pgfqpoint{-0.000000in}{0.000000in}}{%
\pgfpathmoveto{\pgfqpoint{-0.000000in}{0.000000in}}%
\pgfpathlineto{\pgfqpoint{-0.027778in}{0.000000in}}%
\pgfusepath{stroke,fill}%
}%
\begin{pgfscope}%
\pgfsys@transformshift{0.800000in}{2.084766in}%
\pgfsys@useobject{currentmarker}{}%
\end{pgfscope}%
\end{pgfscope}%
\begin{pgfscope}%
\pgfsetbuttcap%
\pgfsetroundjoin%
\definecolor{currentfill}{rgb}{0.000000,0.000000,0.000000}%
\pgfsetfillcolor{currentfill}%
\pgfsetlinewidth{0.602250pt}%
\definecolor{currentstroke}{rgb}{0.000000,0.000000,0.000000}%
\pgfsetstrokecolor{currentstroke}%
\pgfsetdash{}{0pt}%
\pgfsys@defobject{currentmarker}{\pgfqpoint{-0.027778in}{0.000000in}}{\pgfqpoint{-0.000000in}{0.000000in}}{%
\pgfpathmoveto{\pgfqpoint{-0.000000in}{0.000000in}}%
\pgfpathlineto{\pgfqpoint{-0.027778in}{0.000000in}}%
\pgfusepath{stroke,fill}%
}%
\begin{pgfscope}%
\pgfsys@transformshift{0.800000in}{2.182333in}%
\pgfsys@useobject{currentmarker}{}%
\end{pgfscope}%
\end{pgfscope}%
\begin{pgfscope}%
\pgfsetbuttcap%
\pgfsetroundjoin%
\definecolor{currentfill}{rgb}{0.000000,0.000000,0.000000}%
\pgfsetfillcolor{currentfill}%
\pgfsetlinewidth{0.602250pt}%
\definecolor{currentstroke}{rgb}{0.000000,0.000000,0.000000}%
\pgfsetstrokecolor{currentstroke}%
\pgfsetdash{}{0pt}%
\pgfsys@defobject{currentmarker}{\pgfqpoint{-0.027778in}{0.000000in}}{\pgfqpoint{-0.000000in}{0.000000in}}{%
\pgfpathmoveto{\pgfqpoint{-0.000000in}{0.000000in}}%
\pgfpathlineto{\pgfqpoint{-0.027778in}{0.000000in}}%
\pgfusepath{stroke,fill}%
}%
\begin{pgfscope}%
\pgfsys@transformshift{0.800000in}{2.266849in}%
\pgfsys@useobject{currentmarker}{}%
\end{pgfscope}%
\end{pgfscope}%
\begin{pgfscope}%
\pgfsetbuttcap%
\pgfsetroundjoin%
\definecolor{currentfill}{rgb}{0.000000,0.000000,0.000000}%
\pgfsetfillcolor{currentfill}%
\pgfsetlinewidth{0.602250pt}%
\definecolor{currentstroke}{rgb}{0.000000,0.000000,0.000000}%
\pgfsetstrokecolor{currentstroke}%
\pgfsetdash{}{0pt}%
\pgfsys@defobject{currentmarker}{\pgfqpoint{-0.027778in}{0.000000in}}{\pgfqpoint{-0.000000in}{0.000000in}}{%
\pgfpathmoveto{\pgfqpoint{-0.000000in}{0.000000in}}%
\pgfpathlineto{\pgfqpoint{-0.027778in}{0.000000in}}%
\pgfusepath{stroke,fill}%
}%
\begin{pgfscope}%
\pgfsys@transformshift{0.800000in}{2.341398in}%
\pgfsys@useobject{currentmarker}{}%
\end{pgfscope}%
\end{pgfscope}%
\begin{pgfscope}%
\pgfsetbuttcap%
\pgfsetroundjoin%
\definecolor{currentfill}{rgb}{0.000000,0.000000,0.000000}%
\pgfsetfillcolor{currentfill}%
\pgfsetlinewidth{0.602250pt}%
\definecolor{currentstroke}{rgb}{0.000000,0.000000,0.000000}%
\pgfsetstrokecolor{currentstroke}%
\pgfsetdash{}{0pt}%
\pgfsys@defobject{currentmarker}{\pgfqpoint{-0.027778in}{0.000000in}}{\pgfqpoint{-0.000000in}{0.000000in}}{%
\pgfpathmoveto{\pgfqpoint{-0.000000in}{0.000000in}}%
\pgfpathlineto{\pgfqpoint{-0.027778in}{0.000000in}}%
\pgfusepath{stroke,fill}%
}%
\begin{pgfscope}%
\pgfsys@transformshift{0.800000in}{2.846798in}%
\pgfsys@useobject{currentmarker}{}%
\end{pgfscope}%
\end{pgfscope}%
\begin{pgfscope}%
\pgfsetbuttcap%
\pgfsetroundjoin%
\definecolor{currentfill}{rgb}{0.000000,0.000000,0.000000}%
\pgfsetfillcolor{currentfill}%
\pgfsetlinewidth{0.602250pt}%
\definecolor{currentstroke}{rgb}{0.000000,0.000000,0.000000}%
\pgfsetstrokecolor{currentstroke}%
\pgfsetdash{}{0pt}%
\pgfsys@defobject{currentmarker}{\pgfqpoint{-0.027778in}{0.000000in}}{\pgfqpoint{-0.000000in}{0.000000in}}{%
\pgfpathmoveto{\pgfqpoint{-0.000000in}{0.000000in}}%
\pgfpathlineto{\pgfqpoint{-0.027778in}{0.000000in}}%
\pgfusepath{stroke,fill}%
}%
\begin{pgfscope}%
\pgfsys@transformshift{0.800000in}{3.103430in}%
\pgfsys@useobject{currentmarker}{}%
\end{pgfscope}%
\end{pgfscope}%
\begin{pgfscope}%
\pgfsetbuttcap%
\pgfsetroundjoin%
\definecolor{currentfill}{rgb}{0.000000,0.000000,0.000000}%
\pgfsetfillcolor{currentfill}%
\pgfsetlinewidth{0.602250pt}%
\definecolor{currentstroke}{rgb}{0.000000,0.000000,0.000000}%
\pgfsetstrokecolor{currentstroke}%
\pgfsetdash{}{0pt}%
\pgfsys@defobject{currentmarker}{\pgfqpoint{-0.027778in}{0.000000in}}{\pgfqpoint{-0.000000in}{0.000000in}}{%
\pgfpathmoveto{\pgfqpoint{-0.000000in}{0.000000in}}%
\pgfpathlineto{\pgfqpoint{-0.027778in}{0.000000in}}%
\pgfusepath{stroke,fill}%
}%
\begin{pgfscope}%
\pgfsys@transformshift{0.800000in}{3.285513in}%
\pgfsys@useobject{currentmarker}{}%
\end{pgfscope}%
\end{pgfscope}%
\begin{pgfscope}%
\pgfsetbuttcap%
\pgfsetroundjoin%
\definecolor{currentfill}{rgb}{0.000000,0.000000,0.000000}%
\pgfsetfillcolor{currentfill}%
\pgfsetlinewidth{0.602250pt}%
\definecolor{currentstroke}{rgb}{0.000000,0.000000,0.000000}%
\pgfsetstrokecolor{currentstroke}%
\pgfsetdash{}{0pt}%
\pgfsys@defobject{currentmarker}{\pgfqpoint{-0.027778in}{0.000000in}}{\pgfqpoint{-0.000000in}{0.000000in}}{%
\pgfpathmoveto{\pgfqpoint{-0.000000in}{0.000000in}}%
\pgfpathlineto{\pgfqpoint{-0.027778in}{0.000000in}}%
\pgfusepath{stroke,fill}%
}%
\begin{pgfscope}%
\pgfsys@transformshift{0.800000in}{3.426748in}%
\pgfsys@useobject{currentmarker}{}%
\end{pgfscope}%
\end{pgfscope}%
\begin{pgfscope}%
\pgfsetbuttcap%
\pgfsetroundjoin%
\definecolor{currentfill}{rgb}{0.000000,0.000000,0.000000}%
\pgfsetfillcolor{currentfill}%
\pgfsetlinewidth{0.602250pt}%
\definecolor{currentstroke}{rgb}{0.000000,0.000000,0.000000}%
\pgfsetstrokecolor{currentstroke}%
\pgfsetdash{}{0pt}%
\pgfsys@defobject{currentmarker}{\pgfqpoint{-0.027778in}{0.000000in}}{\pgfqpoint{-0.000000in}{0.000000in}}{%
\pgfpathmoveto{\pgfqpoint{-0.000000in}{0.000000in}}%
\pgfpathlineto{\pgfqpoint{-0.027778in}{0.000000in}}%
\pgfusepath{stroke,fill}%
}%
\begin{pgfscope}%
\pgfsys@transformshift{0.800000in}{3.542145in}%
\pgfsys@useobject{currentmarker}{}%
\end{pgfscope}%
\end{pgfscope}%
\begin{pgfscope}%
\pgfsetbuttcap%
\pgfsetroundjoin%
\definecolor{currentfill}{rgb}{0.000000,0.000000,0.000000}%
\pgfsetfillcolor{currentfill}%
\pgfsetlinewidth{0.602250pt}%
\definecolor{currentstroke}{rgb}{0.000000,0.000000,0.000000}%
\pgfsetstrokecolor{currentstroke}%
\pgfsetdash{}{0pt}%
\pgfsys@defobject{currentmarker}{\pgfqpoint{-0.027778in}{0.000000in}}{\pgfqpoint{-0.000000in}{0.000000in}}{%
\pgfpathmoveto{\pgfqpoint{-0.000000in}{0.000000in}}%
\pgfpathlineto{\pgfqpoint{-0.027778in}{0.000000in}}%
\pgfusepath{stroke,fill}%
}%
\begin{pgfscope}%
\pgfsys@transformshift{0.800000in}{3.639712in}%
\pgfsys@useobject{currentmarker}{}%
\end{pgfscope}%
\end{pgfscope}%
\begin{pgfscope}%
\pgfsetbuttcap%
\pgfsetroundjoin%
\definecolor{currentfill}{rgb}{0.000000,0.000000,0.000000}%
\pgfsetfillcolor{currentfill}%
\pgfsetlinewidth{0.602250pt}%
\definecolor{currentstroke}{rgb}{0.000000,0.000000,0.000000}%
\pgfsetstrokecolor{currentstroke}%
\pgfsetdash{}{0pt}%
\pgfsys@defobject{currentmarker}{\pgfqpoint{-0.027778in}{0.000000in}}{\pgfqpoint{-0.000000in}{0.000000in}}{%
\pgfpathmoveto{\pgfqpoint{-0.000000in}{0.000000in}}%
\pgfpathlineto{\pgfqpoint{-0.027778in}{0.000000in}}%
\pgfusepath{stroke,fill}%
}%
\begin{pgfscope}%
\pgfsys@transformshift{0.800000in}{3.724228in}%
\pgfsys@useobject{currentmarker}{}%
\end{pgfscope}%
\end{pgfscope}%
\begin{pgfscope}%
\pgfsetbuttcap%
\pgfsetroundjoin%
\definecolor{currentfill}{rgb}{0.000000,0.000000,0.000000}%
\pgfsetfillcolor{currentfill}%
\pgfsetlinewidth{0.602250pt}%
\definecolor{currentstroke}{rgb}{0.000000,0.000000,0.000000}%
\pgfsetstrokecolor{currentstroke}%
\pgfsetdash{}{0pt}%
\pgfsys@defobject{currentmarker}{\pgfqpoint{-0.027778in}{0.000000in}}{\pgfqpoint{-0.000000in}{0.000000in}}{%
\pgfpathmoveto{\pgfqpoint{-0.000000in}{0.000000in}}%
\pgfpathlineto{\pgfqpoint{-0.027778in}{0.000000in}}%
\pgfusepath{stroke,fill}%
}%
\begin{pgfscope}%
\pgfsys@transformshift{0.800000in}{3.798777in}%
\pgfsys@useobject{currentmarker}{}%
\end{pgfscope}%
\end{pgfscope}%
\begin{pgfscope}%
\definecolor{textcolor}{rgb}{0.000000,0.000000,0.000000}%
\pgfsetstrokecolor{textcolor}%
\pgfsetfillcolor{textcolor}%
\pgftext[x=0.446026in,y=2.376000in,,bottom,rotate=90.000000]{\color{textcolor}\rmfamily\fontsize{10.000000}{12.000000}\selectfont Basic Operations}%
\end{pgfscope}%
\begin{pgfscope}%
\pgfpathrectangle{\pgfqpoint{0.800000in}{0.528000in}}{\pgfqpoint{4.960000in}{3.696000in}}%
\pgfusepath{clip}%
\pgfsetrectcap%
\pgfsetroundjoin%
\pgfsetlinewidth{1.505625pt}%
\definecolor{currentstroke}{rgb}{0.121569,0.466667,0.705882}%
\pgfsetstrokecolor{currentstroke}%
\pgfsetdash{}{0pt}%
\pgfpathmoveto{\pgfqpoint{1.025455in}{0.696000in}}%
\pgfpathlineto{\pgfqpoint{1.071466in}{1.215428in}}%
\pgfpathlineto{\pgfqpoint{1.117477in}{1.513017in}}%
\pgfpathlineto{\pgfqpoint{1.163488in}{1.723686in}}%
\pgfpathlineto{\pgfqpoint{1.209499in}{1.882925in}}%
\pgfpathlineto{\pgfqpoint{1.255510in}{2.018396in}}%
\pgfpathlineto{\pgfqpoint{1.301521in}{2.131119in}}%
\pgfpathlineto{\pgfqpoint{1.347532in}{2.226464in}}%
\pgfpathlineto{\pgfqpoint{1.393544in}{2.309901in}}%
\pgfpathlineto{\pgfqpoint{1.439555in}{2.384214in}}%
\pgfpathlineto{\pgfqpoint{1.485566in}{2.453168in}}%
\pgfpathlineto{\pgfqpoint{1.531577in}{2.517423in}}%
\pgfpathlineto{\pgfqpoint{1.577588in}{2.575217in}}%
\pgfpathlineto{\pgfqpoint{1.623599in}{2.628977in}}%
\pgfpathlineto{\pgfqpoint{1.669610in}{2.677849in}}%
\pgfpathlineto{\pgfqpoint{1.715622in}{2.723139in}}%
\pgfpathlineto{\pgfqpoint{1.761633in}{2.765640in}}%
\pgfpathlineto{\pgfqpoint{1.807644in}{2.805845in}}%
\pgfpathlineto{\pgfqpoint{1.853655in}{2.843196in}}%
\pgfpathlineto{\pgfqpoint{1.899666in}{2.879025in}}%
\pgfpathlineto{\pgfqpoint{1.945677in}{2.914073in}}%
\pgfpathlineto{\pgfqpoint{1.991688in}{2.947956in}}%
\pgfpathlineto{\pgfqpoint{2.037699in}{2.980538in}}%
\pgfpathlineto{\pgfqpoint{2.083711in}{3.011012in}}%
\pgfpathlineto{\pgfqpoint{2.129722in}{3.040416in}}%
\pgfpathlineto{\pgfqpoint{2.175733in}{3.068413in}}%
\pgfpathlineto{\pgfqpoint{2.221744in}{3.095432in}}%
\pgfpathlineto{\pgfqpoint{2.267755in}{3.121026in}}%
\pgfpathlineto{\pgfqpoint{2.313766in}{3.145818in}}%
\pgfpathlineto{\pgfqpoint{2.359777in}{3.169603in}}%
\pgfpathlineto{\pgfqpoint{2.405788in}{3.192710in}}%
\pgfpathlineto{\pgfqpoint{2.451800in}{3.215151in}}%
\pgfpathlineto{\pgfqpoint{2.497811in}{3.236585in}}%
\pgfpathlineto{\pgfqpoint{2.543822in}{3.257405in}}%
\pgfpathlineto{\pgfqpoint{2.589833in}{3.277582in}}%
\pgfpathlineto{\pgfqpoint{2.635844in}{3.297046in}}%
\pgfpathlineto{\pgfqpoint{2.681855in}{3.316275in}}%
\pgfpathlineto{\pgfqpoint{2.727866in}{3.334449in}}%
\pgfpathlineto{\pgfqpoint{2.773878in}{3.352377in}}%
\pgfpathlineto{\pgfqpoint{2.819889in}{3.369839in}}%
\pgfpathlineto{\pgfqpoint{2.865900in}{3.386719in}}%
\pgfpathlineto{\pgfqpoint{2.911911in}{3.404268in}}%
\pgfpathlineto{\pgfqpoint{2.957922in}{3.421305in}}%
\pgfpathlineto{\pgfqpoint{3.003933in}{3.437966in}}%
\pgfpathlineto{\pgfqpoint{3.049944in}{3.454042in}}%
\pgfpathlineto{\pgfqpoint{3.095955in}{3.469892in}}%
\pgfpathlineto{\pgfqpoint{3.141967in}{3.485433in}}%
\pgfpathlineto{\pgfqpoint{3.187978in}{3.500417in}}%
\pgfpathlineto{\pgfqpoint{3.233989in}{3.515222in}}%
\pgfpathlineto{\pgfqpoint{3.280000in}{3.529512in}}%
\pgfpathlineto{\pgfqpoint{3.326011in}{3.543512in}}%
\pgfpathlineto{\pgfqpoint{3.372022in}{3.557081in}}%
\pgfpathlineto{\pgfqpoint{3.418033in}{3.570702in}}%
\pgfpathlineto{\pgfqpoint{3.464045in}{3.583844in}}%
\pgfpathlineto{\pgfqpoint{3.510056in}{3.596772in}}%
\pgfpathlineto{\pgfqpoint{3.556067in}{3.609577in}}%
\pgfpathlineto{\pgfqpoint{3.602078in}{3.621855in}}%
\pgfpathlineto{\pgfqpoint{3.648089in}{3.634064in}}%
\pgfpathlineto{\pgfqpoint{3.694100in}{3.646109in}}%
\pgfpathlineto{\pgfqpoint{3.740111in}{3.657789in}}%
\pgfpathlineto{\pgfqpoint{3.786122in}{3.669330in}}%
\pgfpathlineto{\pgfqpoint{3.832134in}{3.680800in}}%
\pgfpathlineto{\pgfqpoint{3.878145in}{3.691859in}}%
\pgfpathlineto{\pgfqpoint{3.924156in}{3.702730in}}%
\pgfpathlineto{\pgfqpoint{3.970167in}{3.713594in}}%
\pgfpathlineto{\pgfqpoint{4.016178in}{3.724110in}}%
\pgfpathlineto{\pgfqpoint{4.062189in}{3.734467in}}%
\pgfpathlineto{\pgfqpoint{4.108200in}{3.744720in}}%
\pgfpathlineto{\pgfqpoint{4.154212in}{3.754923in}}%
\pgfpathlineto{\pgfqpoint{4.200223in}{3.764892in}}%
\pgfpathlineto{\pgfqpoint{4.246234in}{3.774753in}}%
\pgfpathlineto{\pgfqpoint{4.292245in}{3.784147in}}%
\pgfpathlineto{\pgfqpoint{4.338256in}{3.793763in}}%
\pgfpathlineto{\pgfqpoint{4.384267in}{3.803070in}}%
\pgfpathlineto{\pgfqpoint{4.430278in}{3.812270in}}%
\pgfpathlineto{\pgfqpoint{4.476289in}{3.821331in}}%
\pgfpathlineto{\pgfqpoint{4.522301in}{3.830348in}}%
\pgfpathlineto{\pgfqpoint{4.568312in}{3.839216in}}%
\pgfpathlineto{\pgfqpoint{4.614323in}{3.847892in}}%
\pgfpathlineto{\pgfqpoint{4.660334in}{3.856582in}}%
\pgfpathlineto{\pgfqpoint{4.706345in}{3.864881in}}%
\pgfpathlineto{\pgfqpoint{4.752356in}{3.873301in}}%
\pgfpathlineto{\pgfqpoint{4.798367in}{3.882004in}}%
\pgfpathlineto{\pgfqpoint{4.844378in}{3.890727in}}%
\pgfpathlineto{\pgfqpoint{4.890390in}{3.899247in}}%
\pgfpathlineto{\pgfqpoint{4.936401in}{3.907639in}}%
\pgfpathlineto{\pgfqpoint{4.982412in}{3.915959in}}%
\pgfpathlineto{\pgfqpoint{5.028423in}{3.924157in}}%
\pgfpathlineto{\pgfqpoint{5.074434in}{3.932181in}}%
\pgfpathlineto{\pgfqpoint{5.120445in}{3.940112in}}%
\pgfpathlineto{\pgfqpoint{5.166456in}{3.948017in}}%
\pgfpathlineto{\pgfqpoint{5.212468in}{3.955825in}}%
\pgfpathlineto{\pgfqpoint{5.258479in}{3.963431in}}%
\pgfpathlineto{\pgfqpoint{5.304490in}{3.971050in}}%
\pgfpathlineto{\pgfqpoint{5.350501in}{3.978620in}}%
\pgfpathlineto{\pgfqpoint{5.396512in}{3.985973in}}%
\pgfpathlineto{\pgfqpoint{5.442523in}{3.993345in}}%
\pgfpathlineto{\pgfqpoint{5.488534in}{4.000672in}}%
\pgfpathlineto{\pgfqpoint{5.534545in}{4.007771in}}%
\pgfusepath{stroke}%
\end{pgfscope}%
\begin{pgfscope}%
\pgfpathrectangle{\pgfqpoint{0.800000in}{0.528000in}}{\pgfqpoint{4.960000in}{3.696000in}}%
\pgfusepath{clip}%
\pgfsetrectcap%
\pgfsetroundjoin%
\pgfsetlinewidth{1.505625pt}%
\definecolor{currentstroke}{rgb}{1.000000,0.498039,0.054902}%
\pgfsetstrokecolor{currentstroke}%
\pgfsetdash{}{0pt}%
\pgfpathmoveto{\pgfqpoint{1.025455in}{0.782147in}}%
\pgfpathlineto{\pgfqpoint{1.071466in}{1.292042in}}%
\pgfpathlineto{\pgfqpoint{1.117477in}{1.585477in}}%
\pgfpathlineto{\pgfqpoint{1.163488in}{1.794201in}}%
\pgfpathlineto{\pgfqpoint{1.209499in}{1.951602in}}%
\pgfpathlineto{\pgfqpoint{1.255510in}{2.084470in}}%
\pgfpathlineto{\pgfqpoint{1.301521in}{2.195947in}}%
\pgfpathlineto{\pgfqpoint{1.347532in}{2.290348in}}%
\pgfpathlineto{\pgfqpoint{1.393544in}{2.373041in}}%
\pgfpathlineto{\pgfqpoint{1.439555in}{2.446157in}}%
\pgfpathlineto{\pgfqpoint{1.485566in}{2.514728in}}%
\pgfpathlineto{\pgfqpoint{1.531577in}{2.578219in}}%
\pgfpathlineto{\pgfqpoint{1.577588in}{2.635219in}}%
\pgfpathlineto{\pgfqpoint{1.623599in}{2.687914in}}%
\pgfpathlineto{\pgfqpoint{1.669610in}{2.737032in}}%
\pgfpathlineto{\pgfqpoint{1.715622in}{2.781756in}}%
\pgfpathlineto{\pgfqpoint{1.761633in}{2.824288in}}%
\pgfpathlineto{\pgfqpoint{1.807644in}{2.863963in}}%
\pgfpathlineto{\pgfqpoint{1.853655in}{2.901291in}}%
\pgfpathlineto{\pgfqpoint{1.899666in}{2.936567in}}%
\pgfpathlineto{\pgfqpoint{1.945677in}{2.970959in}}%
\pgfpathlineto{\pgfqpoint{1.991688in}{3.004585in}}%
\pgfpathlineto{\pgfqpoint{2.037699in}{3.036543in}}%
\pgfpathlineto{\pgfqpoint{2.083711in}{3.067067in}}%
\pgfpathlineto{\pgfqpoint{2.129722in}{3.096071in}}%
\pgfpathlineto{\pgfqpoint{2.175733in}{3.123829in}}%
\pgfpathlineto{\pgfqpoint{2.221744in}{3.150353in}}%
\pgfpathlineto{\pgfqpoint{2.267755in}{3.176217in}}%
\pgfpathlineto{\pgfqpoint{2.313766in}{3.200664in}}%
\pgfpathlineto{\pgfqpoint{2.359777in}{3.224338in}}%
\pgfpathlineto{\pgfqpoint{2.405788in}{3.247356in}}%
\pgfpathlineto{\pgfqpoint{2.451800in}{3.269109in}}%
\pgfpathlineto{\pgfqpoint{2.497811in}{3.290616in}}%
\pgfpathlineto{\pgfqpoint{2.543822in}{3.311359in}}%
\pgfpathlineto{\pgfqpoint{2.589833in}{3.331355in}}%
\pgfpathlineto{\pgfqpoint{2.635844in}{3.350527in}}%
\pgfpathlineto{\pgfqpoint{2.681855in}{3.369380in}}%
\pgfpathlineto{\pgfqpoint{2.727866in}{3.387768in}}%
\pgfpathlineto{\pgfqpoint{2.773878in}{3.405603in}}%
\pgfpathlineto{\pgfqpoint{2.819889in}{3.422883in}}%
\pgfpathlineto{\pgfqpoint{2.865900in}{3.439768in}}%
\pgfpathlineto{\pgfqpoint{2.911911in}{3.457072in}}%
\pgfpathlineto{\pgfqpoint{2.957922in}{3.474100in}}%
\pgfpathlineto{\pgfqpoint{3.003933in}{3.490388in}}%
\pgfpathlineto{\pgfqpoint{3.049944in}{3.506367in}}%
\pgfpathlineto{\pgfqpoint{3.095955in}{3.522201in}}%
\pgfpathlineto{\pgfqpoint{3.141967in}{3.537219in}}%
\pgfpathlineto{\pgfqpoint{3.187978in}{3.552318in}}%
\pgfpathlineto{\pgfqpoint{3.233989in}{3.566730in}}%
\pgfpathlineto{\pgfqpoint{3.280000in}{3.581064in}}%
\pgfpathlineto{\pgfqpoint{3.326011in}{3.594986in}}%
\pgfpathlineto{\pgfqpoint{3.372022in}{3.608604in}}%
\pgfpathlineto{\pgfqpoint{3.418033in}{3.622081in}}%
\pgfpathlineto{\pgfqpoint{3.464045in}{3.635257in}}%
\pgfpathlineto{\pgfqpoint{3.510056in}{3.647969in}}%
\pgfpathlineto{\pgfqpoint{3.556067in}{3.660435in}}%
\pgfpathlineto{\pgfqpoint{3.602078in}{3.672754in}}%
\pgfpathlineto{\pgfqpoint{3.648089in}{3.684971in}}%
\pgfpathlineto{\pgfqpoint{3.694100in}{3.696765in}}%
\pgfpathlineto{\pgfqpoint{3.740111in}{3.708504in}}%
\pgfpathlineto{\pgfqpoint{3.786122in}{3.719777in}}%
\pgfpathlineto{\pgfqpoint{3.832134in}{3.731239in}}%
\pgfpathlineto{\pgfqpoint{3.878145in}{3.742193in}}%
\pgfpathlineto{\pgfqpoint{3.924156in}{3.753142in}}%
\pgfpathlineto{\pgfqpoint{3.970167in}{3.763814in}}%
\pgfpathlineto{\pgfqpoint{4.016178in}{3.774371in}}%
\pgfpathlineto{\pgfqpoint{4.062189in}{3.784560in}}%
\pgfpathlineto{\pgfqpoint{4.108200in}{3.794801in}}%
\pgfpathlineto{\pgfqpoint{4.154212in}{3.804809in}}%
\pgfpathlineto{\pgfqpoint{4.200223in}{3.814751in}}%
\pgfpathlineto{\pgfqpoint{4.246234in}{3.824365in}}%
\pgfpathlineto{\pgfqpoint{4.292245in}{3.834064in}}%
\pgfpathlineto{\pgfqpoint{4.338256in}{3.843512in}}%
\pgfpathlineto{\pgfqpoint{4.384267in}{3.852875in}}%
\pgfpathlineto{\pgfqpoint{4.430278in}{3.862001in}}%
\pgfpathlineto{\pgfqpoint{4.476289in}{3.871061in}}%
\pgfpathlineto{\pgfqpoint{4.522301in}{3.879893in}}%
\pgfpathlineto{\pgfqpoint{4.568312in}{3.888706in}}%
\pgfpathlineto{\pgfqpoint{4.614323in}{3.897250in}}%
\pgfpathlineto{\pgfqpoint{4.660334in}{3.905886in}}%
\pgfpathlineto{\pgfqpoint{4.706345in}{3.914387in}}%
\pgfpathlineto{\pgfqpoint{4.752356in}{3.922503in}}%
\pgfpathlineto{\pgfqpoint{4.798367in}{3.931434in}}%
\pgfpathlineto{\pgfqpoint{4.844378in}{3.939815in}}%
\pgfpathlineto{\pgfqpoint{4.890390in}{3.948303in}}%
\pgfpathlineto{\pgfqpoint{4.936401in}{3.956523in}}%
\pgfpathlineto{\pgfqpoint{4.982412in}{3.964932in}}%
\pgfpathlineto{\pgfqpoint{5.028423in}{3.972990in}}%
\pgfpathlineto{\pgfqpoint{5.074434in}{3.980917in}}%
\pgfpathlineto{\pgfqpoint{5.120445in}{3.989027in}}%
\pgfpathlineto{\pgfqpoint{5.166456in}{3.996727in}}%
\pgfpathlineto{\pgfqpoint{5.212468in}{4.004441in}}%
\pgfpathlineto{\pgfqpoint{5.258479in}{4.011978in}}%
\pgfpathlineto{\pgfqpoint{5.304490in}{4.019635in}}%
\pgfpathlineto{\pgfqpoint{5.350501in}{4.026991in}}%
\pgfpathlineto{\pgfqpoint{5.396512in}{4.034452in}}%
\pgfpathlineto{\pgfqpoint{5.442523in}{4.041543in}}%
\pgfpathlineto{\pgfqpoint{5.488534in}{4.048903in}}%
\pgfpathlineto{\pgfqpoint{5.534545in}{4.056000in}}%
\pgfusepath{stroke}%
\end{pgfscope}%
\begin{pgfscope}%
\pgfsetrectcap%
\pgfsetmiterjoin%
\pgfsetlinewidth{0.803000pt}%
\definecolor{currentstroke}{rgb}{0.000000,0.000000,0.000000}%
\pgfsetstrokecolor{currentstroke}%
\pgfsetdash{}{0pt}%
\pgfpathmoveto{\pgfqpoint{0.800000in}{0.528000in}}%
\pgfpathlineto{\pgfqpoint{0.800000in}{4.224000in}}%
\pgfusepath{stroke}%
\end{pgfscope}%
\begin{pgfscope}%
\pgfsetrectcap%
\pgfsetmiterjoin%
\pgfsetlinewidth{0.803000pt}%
\definecolor{currentstroke}{rgb}{0.000000,0.000000,0.000000}%
\pgfsetstrokecolor{currentstroke}%
\pgfsetdash{}{0pt}%
\pgfpathmoveto{\pgfqpoint{5.760000in}{0.528000in}}%
\pgfpathlineto{\pgfqpoint{5.760000in}{4.224000in}}%
\pgfusepath{stroke}%
\end{pgfscope}%
\begin{pgfscope}%
\pgfsetrectcap%
\pgfsetmiterjoin%
\pgfsetlinewidth{0.803000pt}%
\definecolor{currentstroke}{rgb}{0.000000,0.000000,0.000000}%
\pgfsetstrokecolor{currentstroke}%
\pgfsetdash{}{0pt}%
\pgfpathmoveto{\pgfqpoint{0.800000in}{0.528000in}}%
\pgfpathlineto{\pgfqpoint{5.760000in}{0.528000in}}%
\pgfusepath{stroke}%
\end{pgfscope}%
\begin{pgfscope}%
\pgfsetrectcap%
\pgfsetmiterjoin%
\pgfsetlinewidth{0.803000pt}%
\definecolor{currentstroke}{rgb}{0.000000,0.000000,0.000000}%
\pgfsetstrokecolor{currentstroke}%
\pgfsetdash{}{0pt}%
\pgfpathmoveto{\pgfqpoint{0.800000in}{4.224000in}}%
\pgfpathlineto{\pgfqpoint{5.760000in}{4.224000in}}%
\pgfusepath{stroke}%
\end{pgfscope}%
\begin{pgfscope}%
\pgfsetbuttcap%
\pgfsetmiterjoin%
\definecolor{currentfill}{rgb}{1.000000,1.000000,1.000000}%
\pgfsetfillcolor{currentfill}%
\pgfsetfillopacity{0.800000}%
\pgfsetlinewidth{1.003750pt}%
\definecolor{currentstroke}{rgb}{0.800000,0.800000,0.800000}%
\pgfsetstrokecolor{currentstroke}%
\pgfsetstrokeopacity{0.800000}%
\pgfsetdash{}{0pt}%
\pgfpathmoveto{\pgfqpoint{0.897222in}{3.725543in}}%
\pgfpathlineto{\pgfqpoint{2.014507in}{3.725543in}}%
\pgfpathquadraticcurveto{\pgfqpoint{2.042285in}{3.725543in}}{\pgfqpoint{2.042285in}{3.753321in}}%
\pgfpathlineto{\pgfqpoint{2.042285in}{4.126778in}}%
\pgfpathquadraticcurveto{\pgfqpoint{2.042285in}{4.154556in}}{\pgfqpoint{2.014507in}{4.154556in}}%
\pgfpathlineto{\pgfqpoint{0.897222in}{4.154556in}}%
\pgfpathquadraticcurveto{\pgfqpoint{0.869444in}{4.154556in}}{\pgfqpoint{0.869444in}{4.126778in}}%
\pgfpathlineto{\pgfqpoint{0.869444in}{3.753321in}}%
\pgfpathquadraticcurveto{\pgfqpoint{0.869444in}{3.725543in}}{\pgfqpoint{0.897222in}{3.725543in}}%
\pgfpathlineto{\pgfqpoint{0.897222in}{3.725543in}}%
\pgfpathclose%
\pgfusepath{stroke,fill}%
\end{pgfscope}%
\begin{pgfscope}%
\pgfsetrectcap%
\pgfsetroundjoin%
\pgfsetlinewidth{1.505625pt}%
\definecolor{currentstroke}{rgb}{0.121569,0.466667,0.705882}%
\pgfsetstrokecolor{currentstroke}%
\pgfsetdash{}{0pt}%
\pgfpathmoveto{\pgfqpoint{0.925000in}{4.050389in}}%
\pgfpathlineto{\pgfqpoint{1.063889in}{4.050389in}}%
\pgfpathlineto{\pgfqpoint{1.202778in}{4.050389in}}%
\pgfusepath{stroke}%
\end{pgfscope}%
\begin{pgfscope}%
\definecolor{textcolor}{rgb}{0.000000,0.000000,0.000000}%
\pgfsetstrokecolor{textcolor}%
\pgfsetfillcolor{textcolor}%
\pgftext[x=1.313889in,y=4.001778in,left,base]{\color{textcolor}\rmfamily\fontsize{10.000000}{12.000000}\selectfont mergesort}%
\end{pgfscope}%
\begin{pgfscope}%
\pgfsetrectcap%
\pgfsetroundjoin%
\pgfsetlinewidth{1.505625pt}%
\definecolor{currentstroke}{rgb}{1.000000,0.498039,0.054902}%
\pgfsetstrokecolor{currentstroke}%
\pgfsetdash{}{0pt}%
\pgfpathmoveto{\pgfqpoint{0.925000in}{3.856716in}}%
\pgfpathlineto{\pgfqpoint{1.063889in}{3.856716in}}%
\pgfpathlineto{\pgfqpoint{1.202778in}{3.856716in}}%
\pgfusepath{stroke}%
\end{pgfscope}%
\begin{pgfscope}%
\definecolor{textcolor}{rgb}{0.000000,0.000000,0.000000}%
\pgfsetstrokecolor{textcolor}%
\pgfsetfillcolor{textcolor}%
\pgftext[x=1.313889in,y=3.808105in,left,base]{\color{textcolor}\rmfamily\fontsize{10.000000}{12.000000}\selectfont bmergesort}%
\end{pgfscope}%
\end{pgfpicture}%
\makeatother%
\endgroup%

\subsubsection{Time}
%% Creator: Matplotlib, PGF backend
%%
%% To include the figure in your LaTeX document, write
%%   \input{<filename>.pgf}
%%
%% Make sure the required packages are loaded in your preamble
%%   \usepackage{pgf}
%%
%% Also ensure that all the required font packages are loaded; for instance,
%% the lmodern package is sometimes necessary when using math font.
%%   \usepackage{lmodern}
%%
%% Figures using additional raster images can only be included by \input if
%% they are in the same directory as the main LaTeX file. For loading figures
%% from other directories you can use the `import` package
%%   \usepackage{import}
%%
%% and then include the figures with
%%   \import{<path to file>}{<filename>.pgf}
%%
%% Matplotlib used the following preamble
%%   
%%   \makeatletter\@ifpackageloaded{underscore}{}{\usepackage[strings]{underscore}}\makeatother
%%
\begingroup%
\makeatletter%
\begin{pgfpicture}%
\pgfpathrectangle{\pgfpointorigin}{\pgfqpoint{6.400000in}{4.800000in}}%
\pgfusepath{use as bounding box, clip}%
\begin{pgfscope}%
\pgfsetbuttcap%
\pgfsetmiterjoin%
\definecolor{currentfill}{rgb}{1.000000,1.000000,1.000000}%
\pgfsetfillcolor{currentfill}%
\pgfsetlinewidth{0.000000pt}%
\definecolor{currentstroke}{rgb}{1.000000,1.000000,1.000000}%
\pgfsetstrokecolor{currentstroke}%
\pgfsetdash{}{0pt}%
\pgfpathmoveto{\pgfqpoint{0.000000in}{0.000000in}}%
\pgfpathlineto{\pgfqpoint{6.400000in}{0.000000in}}%
\pgfpathlineto{\pgfqpoint{6.400000in}{4.800000in}}%
\pgfpathlineto{\pgfqpoint{0.000000in}{4.800000in}}%
\pgfpathlineto{\pgfqpoint{0.000000in}{0.000000in}}%
\pgfpathclose%
\pgfusepath{fill}%
\end{pgfscope}%
\begin{pgfscope}%
\pgfsetbuttcap%
\pgfsetmiterjoin%
\definecolor{currentfill}{rgb}{1.000000,1.000000,1.000000}%
\pgfsetfillcolor{currentfill}%
\pgfsetlinewidth{0.000000pt}%
\definecolor{currentstroke}{rgb}{0.000000,0.000000,0.000000}%
\pgfsetstrokecolor{currentstroke}%
\pgfsetstrokeopacity{0.000000}%
\pgfsetdash{}{0pt}%
\pgfpathmoveto{\pgfqpoint{0.800000in}{0.528000in}}%
\pgfpathlineto{\pgfqpoint{5.760000in}{0.528000in}}%
\pgfpathlineto{\pgfqpoint{5.760000in}{4.224000in}}%
\pgfpathlineto{\pgfqpoint{0.800000in}{4.224000in}}%
\pgfpathlineto{\pgfqpoint{0.800000in}{0.528000in}}%
\pgfpathclose%
\pgfusepath{fill}%
\end{pgfscope}%
\begin{pgfscope}%
\pgfsetbuttcap%
\pgfsetroundjoin%
\definecolor{currentfill}{rgb}{0.000000,0.000000,0.000000}%
\pgfsetfillcolor{currentfill}%
\pgfsetlinewidth{0.803000pt}%
\definecolor{currentstroke}{rgb}{0.000000,0.000000,0.000000}%
\pgfsetstrokecolor{currentstroke}%
\pgfsetdash{}{0pt}%
\pgfsys@defobject{currentmarker}{\pgfqpoint{0.000000in}{-0.048611in}}{\pgfqpoint{0.000000in}{0.000000in}}{%
\pgfpathmoveto{\pgfqpoint{0.000000in}{0.000000in}}%
\pgfpathlineto{\pgfqpoint{0.000000in}{-0.048611in}}%
\pgfusepath{stroke,fill}%
}%
\begin{pgfscope}%
\pgfsys@transformshift{0.979443in}{0.528000in}%
\pgfsys@useobject{currentmarker}{}%
\end{pgfscope}%
\end{pgfscope}%
\begin{pgfscope}%
\definecolor{textcolor}{rgb}{0.000000,0.000000,0.000000}%
\pgfsetstrokecolor{textcolor}%
\pgfsetfillcolor{textcolor}%
\pgftext[x=0.979443in,y=0.430778in,,top]{\color{textcolor}\rmfamily\fontsize{10.000000}{12.000000}\selectfont \(\displaystyle {0}\)}%
\end{pgfscope}%
\begin{pgfscope}%
\pgfsetbuttcap%
\pgfsetroundjoin%
\definecolor{currentfill}{rgb}{0.000000,0.000000,0.000000}%
\pgfsetfillcolor{currentfill}%
\pgfsetlinewidth{0.803000pt}%
\definecolor{currentstroke}{rgb}{0.000000,0.000000,0.000000}%
\pgfsetstrokecolor{currentstroke}%
\pgfsetdash{}{0pt}%
\pgfsys@defobject{currentmarker}{\pgfqpoint{0.000000in}{-0.048611in}}{\pgfqpoint{0.000000in}{0.000000in}}{%
\pgfpathmoveto{\pgfqpoint{0.000000in}{0.000000in}}%
\pgfpathlineto{\pgfqpoint{0.000000in}{-0.048611in}}%
\pgfusepath{stroke,fill}%
}%
\begin{pgfscope}%
\pgfsys@transformshift{1.899666in}{0.528000in}%
\pgfsys@useobject{currentmarker}{}%
\end{pgfscope}%
\end{pgfscope}%
\begin{pgfscope}%
\definecolor{textcolor}{rgb}{0.000000,0.000000,0.000000}%
\pgfsetstrokecolor{textcolor}%
\pgfsetfillcolor{textcolor}%
\pgftext[x=1.899666in,y=0.430778in,,top]{\color{textcolor}\rmfamily\fontsize{10.000000}{12.000000}\selectfont \(\displaystyle {2000}\)}%
\end{pgfscope}%
\begin{pgfscope}%
\pgfsetbuttcap%
\pgfsetroundjoin%
\definecolor{currentfill}{rgb}{0.000000,0.000000,0.000000}%
\pgfsetfillcolor{currentfill}%
\pgfsetlinewidth{0.803000pt}%
\definecolor{currentstroke}{rgb}{0.000000,0.000000,0.000000}%
\pgfsetstrokecolor{currentstroke}%
\pgfsetdash{}{0pt}%
\pgfsys@defobject{currentmarker}{\pgfqpoint{0.000000in}{-0.048611in}}{\pgfqpoint{0.000000in}{0.000000in}}{%
\pgfpathmoveto{\pgfqpoint{0.000000in}{0.000000in}}%
\pgfpathlineto{\pgfqpoint{0.000000in}{-0.048611in}}%
\pgfusepath{stroke,fill}%
}%
\begin{pgfscope}%
\pgfsys@transformshift{2.819889in}{0.528000in}%
\pgfsys@useobject{currentmarker}{}%
\end{pgfscope}%
\end{pgfscope}%
\begin{pgfscope}%
\definecolor{textcolor}{rgb}{0.000000,0.000000,0.000000}%
\pgfsetstrokecolor{textcolor}%
\pgfsetfillcolor{textcolor}%
\pgftext[x=2.819889in,y=0.430778in,,top]{\color{textcolor}\rmfamily\fontsize{10.000000}{12.000000}\selectfont \(\displaystyle {4000}\)}%
\end{pgfscope}%
\begin{pgfscope}%
\pgfsetbuttcap%
\pgfsetroundjoin%
\definecolor{currentfill}{rgb}{0.000000,0.000000,0.000000}%
\pgfsetfillcolor{currentfill}%
\pgfsetlinewidth{0.803000pt}%
\definecolor{currentstroke}{rgb}{0.000000,0.000000,0.000000}%
\pgfsetstrokecolor{currentstroke}%
\pgfsetdash{}{0pt}%
\pgfsys@defobject{currentmarker}{\pgfqpoint{0.000000in}{-0.048611in}}{\pgfqpoint{0.000000in}{0.000000in}}{%
\pgfpathmoveto{\pgfqpoint{0.000000in}{0.000000in}}%
\pgfpathlineto{\pgfqpoint{0.000000in}{-0.048611in}}%
\pgfusepath{stroke,fill}%
}%
\begin{pgfscope}%
\pgfsys@transformshift{3.740111in}{0.528000in}%
\pgfsys@useobject{currentmarker}{}%
\end{pgfscope}%
\end{pgfscope}%
\begin{pgfscope}%
\definecolor{textcolor}{rgb}{0.000000,0.000000,0.000000}%
\pgfsetstrokecolor{textcolor}%
\pgfsetfillcolor{textcolor}%
\pgftext[x=3.740111in,y=0.430778in,,top]{\color{textcolor}\rmfamily\fontsize{10.000000}{12.000000}\selectfont \(\displaystyle {6000}\)}%
\end{pgfscope}%
\begin{pgfscope}%
\pgfsetbuttcap%
\pgfsetroundjoin%
\definecolor{currentfill}{rgb}{0.000000,0.000000,0.000000}%
\pgfsetfillcolor{currentfill}%
\pgfsetlinewidth{0.803000pt}%
\definecolor{currentstroke}{rgb}{0.000000,0.000000,0.000000}%
\pgfsetstrokecolor{currentstroke}%
\pgfsetdash{}{0pt}%
\pgfsys@defobject{currentmarker}{\pgfqpoint{0.000000in}{-0.048611in}}{\pgfqpoint{0.000000in}{0.000000in}}{%
\pgfpathmoveto{\pgfqpoint{0.000000in}{0.000000in}}%
\pgfpathlineto{\pgfqpoint{0.000000in}{-0.048611in}}%
\pgfusepath{stroke,fill}%
}%
\begin{pgfscope}%
\pgfsys@transformshift{4.660334in}{0.528000in}%
\pgfsys@useobject{currentmarker}{}%
\end{pgfscope}%
\end{pgfscope}%
\begin{pgfscope}%
\definecolor{textcolor}{rgb}{0.000000,0.000000,0.000000}%
\pgfsetstrokecolor{textcolor}%
\pgfsetfillcolor{textcolor}%
\pgftext[x=4.660334in,y=0.430778in,,top]{\color{textcolor}\rmfamily\fontsize{10.000000}{12.000000}\selectfont \(\displaystyle {8000}\)}%
\end{pgfscope}%
\begin{pgfscope}%
\pgfsetbuttcap%
\pgfsetroundjoin%
\definecolor{currentfill}{rgb}{0.000000,0.000000,0.000000}%
\pgfsetfillcolor{currentfill}%
\pgfsetlinewidth{0.803000pt}%
\definecolor{currentstroke}{rgb}{0.000000,0.000000,0.000000}%
\pgfsetstrokecolor{currentstroke}%
\pgfsetdash{}{0pt}%
\pgfsys@defobject{currentmarker}{\pgfqpoint{0.000000in}{-0.048611in}}{\pgfqpoint{0.000000in}{0.000000in}}{%
\pgfpathmoveto{\pgfqpoint{0.000000in}{0.000000in}}%
\pgfpathlineto{\pgfqpoint{0.000000in}{-0.048611in}}%
\pgfusepath{stroke,fill}%
}%
\begin{pgfscope}%
\pgfsys@transformshift{5.580557in}{0.528000in}%
\pgfsys@useobject{currentmarker}{}%
\end{pgfscope}%
\end{pgfscope}%
\begin{pgfscope}%
\definecolor{textcolor}{rgb}{0.000000,0.000000,0.000000}%
\pgfsetstrokecolor{textcolor}%
\pgfsetfillcolor{textcolor}%
\pgftext[x=5.580557in,y=0.430778in,,top]{\color{textcolor}\rmfamily\fontsize{10.000000}{12.000000}\selectfont \(\displaystyle {10000}\)}%
\end{pgfscope}%
\begin{pgfscope}%
\definecolor{textcolor}{rgb}{0.000000,0.000000,0.000000}%
\pgfsetstrokecolor{textcolor}%
\pgfsetfillcolor{textcolor}%
\pgftext[x=3.280000in,y=0.251766in,,top]{\color{textcolor}\rmfamily\fontsize{10.000000}{12.000000}\selectfont Input Size}%
\end{pgfscope}%
\begin{pgfscope}%
\pgfsetbuttcap%
\pgfsetroundjoin%
\definecolor{currentfill}{rgb}{0.000000,0.000000,0.000000}%
\pgfsetfillcolor{currentfill}%
\pgfsetlinewidth{0.803000pt}%
\definecolor{currentstroke}{rgb}{0.000000,0.000000,0.000000}%
\pgfsetstrokecolor{currentstroke}%
\pgfsetdash{}{0pt}%
\pgfsys@defobject{currentmarker}{\pgfqpoint{-0.048611in}{0.000000in}}{\pgfqpoint{-0.000000in}{0.000000in}}{%
\pgfpathmoveto{\pgfqpoint{-0.000000in}{0.000000in}}%
\pgfpathlineto{\pgfqpoint{-0.048611in}{0.000000in}}%
\pgfusepath{stroke,fill}%
}%
\begin{pgfscope}%
\pgfsys@transformshift{0.800000in}{0.720570in}%
\pgfsys@useobject{currentmarker}{}%
\end{pgfscope}%
\end{pgfscope}%
\begin{pgfscope}%
\definecolor{textcolor}{rgb}{0.000000,0.000000,0.000000}%
\pgfsetstrokecolor{textcolor}%
\pgfsetfillcolor{textcolor}%
\pgftext[x=0.501581in, y=0.672345in, left, base]{\color{textcolor}\rmfamily\fontsize{10.000000}{12.000000}\selectfont \(\displaystyle {10^{6}}\)}%
\end{pgfscope}%
\begin{pgfscope}%
\pgfsetbuttcap%
\pgfsetroundjoin%
\definecolor{currentfill}{rgb}{0.000000,0.000000,0.000000}%
\pgfsetfillcolor{currentfill}%
\pgfsetlinewidth{0.803000pt}%
\definecolor{currentstroke}{rgb}{0.000000,0.000000,0.000000}%
\pgfsetstrokecolor{currentstroke}%
\pgfsetdash{}{0pt}%
\pgfsys@defobject{currentmarker}{\pgfqpoint{-0.048611in}{0.000000in}}{\pgfqpoint{-0.000000in}{0.000000in}}{%
\pgfpathmoveto{\pgfqpoint{-0.000000in}{0.000000in}}%
\pgfpathlineto{\pgfqpoint{-0.048611in}{0.000000in}}%
\pgfusepath{stroke,fill}%
}%
\begin{pgfscope}%
\pgfsys@transformshift{0.800000in}{2.130858in}%
\pgfsys@useobject{currentmarker}{}%
\end{pgfscope}%
\end{pgfscope}%
\begin{pgfscope}%
\definecolor{textcolor}{rgb}{0.000000,0.000000,0.000000}%
\pgfsetstrokecolor{textcolor}%
\pgfsetfillcolor{textcolor}%
\pgftext[x=0.501581in, y=2.082633in, left, base]{\color{textcolor}\rmfamily\fontsize{10.000000}{12.000000}\selectfont \(\displaystyle {10^{7}}\)}%
\end{pgfscope}%
\begin{pgfscope}%
\pgfsetbuttcap%
\pgfsetroundjoin%
\definecolor{currentfill}{rgb}{0.000000,0.000000,0.000000}%
\pgfsetfillcolor{currentfill}%
\pgfsetlinewidth{0.803000pt}%
\definecolor{currentstroke}{rgb}{0.000000,0.000000,0.000000}%
\pgfsetstrokecolor{currentstroke}%
\pgfsetdash{}{0pt}%
\pgfsys@defobject{currentmarker}{\pgfqpoint{-0.048611in}{0.000000in}}{\pgfqpoint{-0.000000in}{0.000000in}}{%
\pgfpathmoveto{\pgfqpoint{-0.000000in}{0.000000in}}%
\pgfpathlineto{\pgfqpoint{-0.048611in}{0.000000in}}%
\pgfusepath{stroke,fill}%
}%
\begin{pgfscope}%
\pgfsys@transformshift{0.800000in}{3.541146in}%
\pgfsys@useobject{currentmarker}{}%
\end{pgfscope}%
\end{pgfscope}%
\begin{pgfscope}%
\definecolor{textcolor}{rgb}{0.000000,0.000000,0.000000}%
\pgfsetstrokecolor{textcolor}%
\pgfsetfillcolor{textcolor}%
\pgftext[x=0.501581in, y=3.492921in, left, base]{\color{textcolor}\rmfamily\fontsize{10.000000}{12.000000}\selectfont \(\displaystyle {10^{8}}\)}%
\end{pgfscope}%
\begin{pgfscope}%
\pgfsetbuttcap%
\pgfsetroundjoin%
\definecolor{currentfill}{rgb}{0.000000,0.000000,0.000000}%
\pgfsetfillcolor{currentfill}%
\pgfsetlinewidth{0.602250pt}%
\definecolor{currentstroke}{rgb}{0.000000,0.000000,0.000000}%
\pgfsetstrokecolor{currentstroke}%
\pgfsetdash{}{0pt}%
\pgfsys@defobject{currentmarker}{\pgfqpoint{-0.027778in}{0.000000in}}{\pgfqpoint{-0.000000in}{0.000000in}}{%
\pgfpathmoveto{\pgfqpoint{-0.000000in}{0.000000in}}%
\pgfpathlineto{\pgfqpoint{-0.027778in}{0.000000in}}%
\pgfusepath{stroke,fill}%
}%
\begin{pgfscope}%
\pgfsys@transformshift{0.800000in}{0.583899in}%
\pgfsys@useobject{currentmarker}{}%
\end{pgfscope}%
\end{pgfscope}%
\begin{pgfscope}%
\pgfsetbuttcap%
\pgfsetroundjoin%
\definecolor{currentfill}{rgb}{0.000000,0.000000,0.000000}%
\pgfsetfillcolor{currentfill}%
\pgfsetlinewidth{0.602250pt}%
\definecolor{currentstroke}{rgb}{0.000000,0.000000,0.000000}%
\pgfsetstrokecolor{currentstroke}%
\pgfsetdash{}{0pt}%
\pgfsys@defobject{currentmarker}{\pgfqpoint{-0.027778in}{0.000000in}}{\pgfqpoint{-0.000000in}{0.000000in}}{%
\pgfpathmoveto{\pgfqpoint{-0.000000in}{0.000000in}}%
\pgfpathlineto{\pgfqpoint{-0.027778in}{0.000000in}}%
\pgfusepath{stroke,fill}%
}%
\begin{pgfscope}%
\pgfsys@transformshift{0.800000in}{0.656039in}%
\pgfsys@useobject{currentmarker}{}%
\end{pgfscope}%
\end{pgfscope}%
\begin{pgfscope}%
\pgfsetbuttcap%
\pgfsetroundjoin%
\definecolor{currentfill}{rgb}{0.000000,0.000000,0.000000}%
\pgfsetfillcolor{currentfill}%
\pgfsetlinewidth{0.602250pt}%
\definecolor{currentstroke}{rgb}{0.000000,0.000000,0.000000}%
\pgfsetstrokecolor{currentstroke}%
\pgfsetdash{}{0pt}%
\pgfsys@defobject{currentmarker}{\pgfqpoint{-0.027778in}{0.000000in}}{\pgfqpoint{-0.000000in}{0.000000in}}{%
\pgfpathmoveto{\pgfqpoint{-0.000000in}{0.000000in}}%
\pgfpathlineto{\pgfqpoint{-0.027778in}{0.000000in}}%
\pgfusepath{stroke,fill}%
}%
\begin{pgfscope}%
\pgfsys@transformshift{0.800000in}{1.145109in}%
\pgfsys@useobject{currentmarker}{}%
\end{pgfscope}%
\end{pgfscope}%
\begin{pgfscope}%
\pgfsetbuttcap%
\pgfsetroundjoin%
\definecolor{currentfill}{rgb}{0.000000,0.000000,0.000000}%
\pgfsetfillcolor{currentfill}%
\pgfsetlinewidth{0.602250pt}%
\definecolor{currentstroke}{rgb}{0.000000,0.000000,0.000000}%
\pgfsetstrokecolor{currentstroke}%
\pgfsetdash{}{0pt}%
\pgfsys@defobject{currentmarker}{\pgfqpoint{-0.027778in}{0.000000in}}{\pgfqpoint{-0.000000in}{0.000000in}}{%
\pgfpathmoveto{\pgfqpoint{-0.000000in}{0.000000in}}%
\pgfpathlineto{\pgfqpoint{-0.027778in}{0.000000in}}%
\pgfusepath{stroke,fill}%
}%
\begin{pgfscope}%
\pgfsys@transformshift{0.800000in}{1.393449in}%
\pgfsys@useobject{currentmarker}{}%
\end{pgfscope}%
\end{pgfscope}%
\begin{pgfscope}%
\pgfsetbuttcap%
\pgfsetroundjoin%
\definecolor{currentfill}{rgb}{0.000000,0.000000,0.000000}%
\pgfsetfillcolor{currentfill}%
\pgfsetlinewidth{0.602250pt}%
\definecolor{currentstroke}{rgb}{0.000000,0.000000,0.000000}%
\pgfsetstrokecolor{currentstroke}%
\pgfsetdash{}{0pt}%
\pgfsys@defobject{currentmarker}{\pgfqpoint{-0.027778in}{0.000000in}}{\pgfqpoint{-0.000000in}{0.000000in}}{%
\pgfpathmoveto{\pgfqpoint{-0.000000in}{0.000000in}}%
\pgfpathlineto{\pgfqpoint{-0.027778in}{0.000000in}}%
\pgfusepath{stroke,fill}%
}%
\begin{pgfscope}%
\pgfsys@transformshift{0.800000in}{1.569648in}%
\pgfsys@useobject{currentmarker}{}%
\end{pgfscope}%
\end{pgfscope}%
\begin{pgfscope}%
\pgfsetbuttcap%
\pgfsetroundjoin%
\definecolor{currentfill}{rgb}{0.000000,0.000000,0.000000}%
\pgfsetfillcolor{currentfill}%
\pgfsetlinewidth{0.602250pt}%
\definecolor{currentstroke}{rgb}{0.000000,0.000000,0.000000}%
\pgfsetstrokecolor{currentstroke}%
\pgfsetdash{}{0pt}%
\pgfsys@defobject{currentmarker}{\pgfqpoint{-0.027778in}{0.000000in}}{\pgfqpoint{-0.000000in}{0.000000in}}{%
\pgfpathmoveto{\pgfqpoint{-0.000000in}{0.000000in}}%
\pgfpathlineto{\pgfqpoint{-0.027778in}{0.000000in}}%
\pgfusepath{stroke,fill}%
}%
\begin{pgfscope}%
\pgfsys@transformshift{0.800000in}{1.706319in}%
\pgfsys@useobject{currentmarker}{}%
\end{pgfscope}%
\end{pgfscope}%
\begin{pgfscope}%
\pgfsetbuttcap%
\pgfsetroundjoin%
\definecolor{currentfill}{rgb}{0.000000,0.000000,0.000000}%
\pgfsetfillcolor{currentfill}%
\pgfsetlinewidth{0.602250pt}%
\definecolor{currentstroke}{rgb}{0.000000,0.000000,0.000000}%
\pgfsetstrokecolor{currentstroke}%
\pgfsetdash{}{0pt}%
\pgfsys@defobject{currentmarker}{\pgfqpoint{-0.027778in}{0.000000in}}{\pgfqpoint{-0.000000in}{0.000000in}}{%
\pgfpathmoveto{\pgfqpoint{-0.000000in}{0.000000in}}%
\pgfpathlineto{\pgfqpoint{-0.027778in}{0.000000in}}%
\pgfusepath{stroke,fill}%
}%
\begin{pgfscope}%
\pgfsys@transformshift{0.800000in}{1.817988in}%
\pgfsys@useobject{currentmarker}{}%
\end{pgfscope}%
\end{pgfscope}%
\begin{pgfscope}%
\pgfsetbuttcap%
\pgfsetroundjoin%
\definecolor{currentfill}{rgb}{0.000000,0.000000,0.000000}%
\pgfsetfillcolor{currentfill}%
\pgfsetlinewidth{0.602250pt}%
\definecolor{currentstroke}{rgb}{0.000000,0.000000,0.000000}%
\pgfsetstrokecolor{currentstroke}%
\pgfsetdash{}{0pt}%
\pgfsys@defobject{currentmarker}{\pgfqpoint{-0.027778in}{0.000000in}}{\pgfqpoint{-0.000000in}{0.000000in}}{%
\pgfpathmoveto{\pgfqpoint{-0.000000in}{0.000000in}}%
\pgfpathlineto{\pgfqpoint{-0.027778in}{0.000000in}}%
\pgfusepath{stroke,fill}%
}%
\begin{pgfscope}%
\pgfsys@transformshift{0.800000in}{1.912402in}%
\pgfsys@useobject{currentmarker}{}%
\end{pgfscope}%
\end{pgfscope}%
\begin{pgfscope}%
\pgfsetbuttcap%
\pgfsetroundjoin%
\definecolor{currentfill}{rgb}{0.000000,0.000000,0.000000}%
\pgfsetfillcolor{currentfill}%
\pgfsetlinewidth{0.602250pt}%
\definecolor{currentstroke}{rgb}{0.000000,0.000000,0.000000}%
\pgfsetstrokecolor{currentstroke}%
\pgfsetdash{}{0pt}%
\pgfsys@defobject{currentmarker}{\pgfqpoint{-0.027778in}{0.000000in}}{\pgfqpoint{-0.000000in}{0.000000in}}{%
\pgfpathmoveto{\pgfqpoint{-0.000000in}{0.000000in}}%
\pgfpathlineto{\pgfqpoint{-0.027778in}{0.000000in}}%
\pgfusepath{stroke,fill}%
}%
\begin{pgfscope}%
\pgfsys@transformshift{0.800000in}{1.994187in}%
\pgfsys@useobject{currentmarker}{}%
\end{pgfscope}%
\end{pgfscope}%
\begin{pgfscope}%
\pgfsetbuttcap%
\pgfsetroundjoin%
\definecolor{currentfill}{rgb}{0.000000,0.000000,0.000000}%
\pgfsetfillcolor{currentfill}%
\pgfsetlinewidth{0.602250pt}%
\definecolor{currentstroke}{rgb}{0.000000,0.000000,0.000000}%
\pgfsetstrokecolor{currentstroke}%
\pgfsetdash{}{0pt}%
\pgfsys@defobject{currentmarker}{\pgfqpoint{-0.027778in}{0.000000in}}{\pgfqpoint{-0.000000in}{0.000000in}}{%
\pgfpathmoveto{\pgfqpoint{-0.000000in}{0.000000in}}%
\pgfpathlineto{\pgfqpoint{-0.027778in}{0.000000in}}%
\pgfusepath{stroke,fill}%
}%
\begin{pgfscope}%
\pgfsys@transformshift{0.800000in}{2.066327in}%
\pgfsys@useobject{currentmarker}{}%
\end{pgfscope}%
\end{pgfscope}%
\begin{pgfscope}%
\pgfsetbuttcap%
\pgfsetroundjoin%
\definecolor{currentfill}{rgb}{0.000000,0.000000,0.000000}%
\pgfsetfillcolor{currentfill}%
\pgfsetlinewidth{0.602250pt}%
\definecolor{currentstroke}{rgb}{0.000000,0.000000,0.000000}%
\pgfsetstrokecolor{currentstroke}%
\pgfsetdash{}{0pt}%
\pgfsys@defobject{currentmarker}{\pgfqpoint{-0.027778in}{0.000000in}}{\pgfqpoint{-0.000000in}{0.000000in}}{%
\pgfpathmoveto{\pgfqpoint{-0.000000in}{0.000000in}}%
\pgfpathlineto{\pgfqpoint{-0.027778in}{0.000000in}}%
\pgfusepath{stroke,fill}%
}%
\begin{pgfscope}%
\pgfsys@transformshift{0.800000in}{2.555397in}%
\pgfsys@useobject{currentmarker}{}%
\end{pgfscope}%
\end{pgfscope}%
\begin{pgfscope}%
\pgfsetbuttcap%
\pgfsetroundjoin%
\definecolor{currentfill}{rgb}{0.000000,0.000000,0.000000}%
\pgfsetfillcolor{currentfill}%
\pgfsetlinewidth{0.602250pt}%
\definecolor{currentstroke}{rgb}{0.000000,0.000000,0.000000}%
\pgfsetstrokecolor{currentstroke}%
\pgfsetdash{}{0pt}%
\pgfsys@defobject{currentmarker}{\pgfqpoint{-0.027778in}{0.000000in}}{\pgfqpoint{-0.000000in}{0.000000in}}{%
\pgfpathmoveto{\pgfqpoint{-0.000000in}{0.000000in}}%
\pgfpathlineto{\pgfqpoint{-0.027778in}{0.000000in}}%
\pgfusepath{stroke,fill}%
}%
\begin{pgfscope}%
\pgfsys@transformshift{0.800000in}{2.803737in}%
\pgfsys@useobject{currentmarker}{}%
\end{pgfscope}%
\end{pgfscope}%
\begin{pgfscope}%
\pgfsetbuttcap%
\pgfsetroundjoin%
\definecolor{currentfill}{rgb}{0.000000,0.000000,0.000000}%
\pgfsetfillcolor{currentfill}%
\pgfsetlinewidth{0.602250pt}%
\definecolor{currentstroke}{rgb}{0.000000,0.000000,0.000000}%
\pgfsetstrokecolor{currentstroke}%
\pgfsetdash{}{0pt}%
\pgfsys@defobject{currentmarker}{\pgfqpoint{-0.027778in}{0.000000in}}{\pgfqpoint{-0.000000in}{0.000000in}}{%
\pgfpathmoveto{\pgfqpoint{-0.000000in}{0.000000in}}%
\pgfpathlineto{\pgfqpoint{-0.027778in}{0.000000in}}%
\pgfusepath{stroke,fill}%
}%
\begin{pgfscope}%
\pgfsys@transformshift{0.800000in}{2.979936in}%
\pgfsys@useobject{currentmarker}{}%
\end{pgfscope}%
\end{pgfscope}%
\begin{pgfscope}%
\pgfsetbuttcap%
\pgfsetroundjoin%
\definecolor{currentfill}{rgb}{0.000000,0.000000,0.000000}%
\pgfsetfillcolor{currentfill}%
\pgfsetlinewidth{0.602250pt}%
\definecolor{currentstroke}{rgb}{0.000000,0.000000,0.000000}%
\pgfsetstrokecolor{currentstroke}%
\pgfsetdash{}{0pt}%
\pgfsys@defobject{currentmarker}{\pgfqpoint{-0.027778in}{0.000000in}}{\pgfqpoint{-0.000000in}{0.000000in}}{%
\pgfpathmoveto{\pgfqpoint{-0.000000in}{0.000000in}}%
\pgfpathlineto{\pgfqpoint{-0.027778in}{0.000000in}}%
\pgfusepath{stroke,fill}%
}%
\begin{pgfscope}%
\pgfsys@transformshift{0.800000in}{3.116607in}%
\pgfsys@useobject{currentmarker}{}%
\end{pgfscope}%
\end{pgfscope}%
\begin{pgfscope}%
\pgfsetbuttcap%
\pgfsetroundjoin%
\definecolor{currentfill}{rgb}{0.000000,0.000000,0.000000}%
\pgfsetfillcolor{currentfill}%
\pgfsetlinewidth{0.602250pt}%
\definecolor{currentstroke}{rgb}{0.000000,0.000000,0.000000}%
\pgfsetstrokecolor{currentstroke}%
\pgfsetdash{}{0pt}%
\pgfsys@defobject{currentmarker}{\pgfqpoint{-0.027778in}{0.000000in}}{\pgfqpoint{-0.000000in}{0.000000in}}{%
\pgfpathmoveto{\pgfqpoint{-0.000000in}{0.000000in}}%
\pgfpathlineto{\pgfqpoint{-0.027778in}{0.000000in}}%
\pgfusepath{stroke,fill}%
}%
\begin{pgfscope}%
\pgfsys@transformshift{0.800000in}{3.228276in}%
\pgfsys@useobject{currentmarker}{}%
\end{pgfscope}%
\end{pgfscope}%
\begin{pgfscope}%
\pgfsetbuttcap%
\pgfsetroundjoin%
\definecolor{currentfill}{rgb}{0.000000,0.000000,0.000000}%
\pgfsetfillcolor{currentfill}%
\pgfsetlinewidth{0.602250pt}%
\definecolor{currentstroke}{rgb}{0.000000,0.000000,0.000000}%
\pgfsetstrokecolor{currentstroke}%
\pgfsetdash{}{0pt}%
\pgfsys@defobject{currentmarker}{\pgfqpoint{-0.027778in}{0.000000in}}{\pgfqpoint{-0.000000in}{0.000000in}}{%
\pgfpathmoveto{\pgfqpoint{-0.000000in}{0.000000in}}%
\pgfpathlineto{\pgfqpoint{-0.027778in}{0.000000in}}%
\pgfusepath{stroke,fill}%
}%
\begin{pgfscope}%
\pgfsys@transformshift{0.800000in}{3.322690in}%
\pgfsys@useobject{currentmarker}{}%
\end{pgfscope}%
\end{pgfscope}%
\begin{pgfscope}%
\pgfsetbuttcap%
\pgfsetroundjoin%
\definecolor{currentfill}{rgb}{0.000000,0.000000,0.000000}%
\pgfsetfillcolor{currentfill}%
\pgfsetlinewidth{0.602250pt}%
\definecolor{currentstroke}{rgb}{0.000000,0.000000,0.000000}%
\pgfsetstrokecolor{currentstroke}%
\pgfsetdash{}{0pt}%
\pgfsys@defobject{currentmarker}{\pgfqpoint{-0.027778in}{0.000000in}}{\pgfqpoint{-0.000000in}{0.000000in}}{%
\pgfpathmoveto{\pgfqpoint{-0.000000in}{0.000000in}}%
\pgfpathlineto{\pgfqpoint{-0.027778in}{0.000000in}}%
\pgfusepath{stroke,fill}%
}%
\begin{pgfscope}%
\pgfsys@transformshift{0.800000in}{3.404475in}%
\pgfsys@useobject{currentmarker}{}%
\end{pgfscope}%
\end{pgfscope}%
\begin{pgfscope}%
\pgfsetbuttcap%
\pgfsetroundjoin%
\definecolor{currentfill}{rgb}{0.000000,0.000000,0.000000}%
\pgfsetfillcolor{currentfill}%
\pgfsetlinewidth{0.602250pt}%
\definecolor{currentstroke}{rgb}{0.000000,0.000000,0.000000}%
\pgfsetstrokecolor{currentstroke}%
\pgfsetdash{}{0pt}%
\pgfsys@defobject{currentmarker}{\pgfqpoint{-0.027778in}{0.000000in}}{\pgfqpoint{-0.000000in}{0.000000in}}{%
\pgfpathmoveto{\pgfqpoint{-0.000000in}{0.000000in}}%
\pgfpathlineto{\pgfqpoint{-0.027778in}{0.000000in}}%
\pgfusepath{stroke,fill}%
}%
\begin{pgfscope}%
\pgfsys@transformshift{0.800000in}{3.476615in}%
\pgfsys@useobject{currentmarker}{}%
\end{pgfscope}%
\end{pgfscope}%
\begin{pgfscope}%
\pgfsetbuttcap%
\pgfsetroundjoin%
\definecolor{currentfill}{rgb}{0.000000,0.000000,0.000000}%
\pgfsetfillcolor{currentfill}%
\pgfsetlinewidth{0.602250pt}%
\definecolor{currentstroke}{rgb}{0.000000,0.000000,0.000000}%
\pgfsetstrokecolor{currentstroke}%
\pgfsetdash{}{0pt}%
\pgfsys@defobject{currentmarker}{\pgfqpoint{-0.027778in}{0.000000in}}{\pgfqpoint{-0.000000in}{0.000000in}}{%
\pgfpathmoveto{\pgfqpoint{-0.000000in}{0.000000in}}%
\pgfpathlineto{\pgfqpoint{-0.027778in}{0.000000in}}%
\pgfusepath{stroke,fill}%
}%
\begin{pgfscope}%
\pgfsys@transformshift{0.800000in}{3.965685in}%
\pgfsys@useobject{currentmarker}{}%
\end{pgfscope}%
\end{pgfscope}%
\begin{pgfscope}%
\pgfsetbuttcap%
\pgfsetroundjoin%
\definecolor{currentfill}{rgb}{0.000000,0.000000,0.000000}%
\pgfsetfillcolor{currentfill}%
\pgfsetlinewidth{0.602250pt}%
\definecolor{currentstroke}{rgb}{0.000000,0.000000,0.000000}%
\pgfsetstrokecolor{currentstroke}%
\pgfsetdash{}{0pt}%
\pgfsys@defobject{currentmarker}{\pgfqpoint{-0.027778in}{0.000000in}}{\pgfqpoint{-0.000000in}{0.000000in}}{%
\pgfpathmoveto{\pgfqpoint{-0.000000in}{0.000000in}}%
\pgfpathlineto{\pgfqpoint{-0.027778in}{0.000000in}}%
\pgfusepath{stroke,fill}%
}%
\begin{pgfscope}%
\pgfsys@transformshift{0.800000in}{4.214024in}%
\pgfsys@useobject{currentmarker}{}%
\end{pgfscope}%
\end{pgfscope}%
\begin{pgfscope}%
\definecolor{textcolor}{rgb}{0.000000,0.000000,0.000000}%
\pgfsetstrokecolor{textcolor}%
\pgfsetfillcolor{textcolor}%
\pgftext[x=0.446026in,y=2.376000in,,bottom,rotate=90.000000]{\color{textcolor}\rmfamily\fontsize{10.000000}{12.000000}\selectfont Time (ns)}%
\end{pgfscope}%
\begin{pgfscope}%
\pgfpathrectangle{\pgfqpoint{0.800000in}{0.528000in}}{\pgfqpoint{4.960000in}{3.696000in}}%
\pgfusepath{clip}%
\pgfsetrectcap%
\pgfsetroundjoin%
\pgfsetlinewidth{1.505625pt}%
\definecolor{currentstroke}{rgb}{0.121569,0.466667,0.705882}%
\pgfsetstrokecolor{currentstroke}%
\pgfsetdash{}{0pt}%
\pgfpathmoveto{\pgfqpoint{1.025455in}{0.739319in}}%
\pgfpathlineto{\pgfqpoint{1.071466in}{1.252510in}}%
\pgfpathlineto{\pgfqpoint{1.117477in}{1.541026in}}%
\pgfpathlineto{\pgfqpoint{1.163488in}{1.736280in}}%
\pgfpathlineto{\pgfqpoint{1.209499in}{1.884666in}}%
\pgfpathlineto{\pgfqpoint{1.255510in}{2.046000in}}%
\pgfpathlineto{\pgfqpoint{1.301521in}{2.148670in}}%
\pgfpathlineto{\pgfqpoint{1.347532in}{2.239709in}}%
\pgfpathlineto{\pgfqpoint{1.393544in}{2.322583in}}%
\pgfpathlineto{\pgfqpoint{1.439555in}{2.392271in}}%
\pgfpathlineto{\pgfqpoint{1.485566in}{2.484202in}}%
\pgfpathlineto{\pgfqpoint{1.531577in}{2.549312in}}%
\pgfpathlineto{\pgfqpoint{1.577588in}{2.603050in}}%
\pgfpathlineto{\pgfqpoint{1.623599in}{2.659271in}}%
\pgfpathlineto{\pgfqpoint{1.669610in}{2.723261in}}%
\pgfpathlineto{\pgfqpoint{1.715622in}{2.747071in}}%
\pgfpathlineto{\pgfqpoint{1.761633in}{2.789978in}}%
\pgfpathlineto{\pgfqpoint{1.807644in}{2.830170in}}%
\pgfpathlineto{\pgfqpoint{1.853655in}{2.883546in}}%
\pgfpathlineto{\pgfqpoint{1.899666in}{2.901996in}}%
\pgfpathlineto{\pgfqpoint{1.945677in}{2.959910in}}%
\pgfpathlineto{\pgfqpoint{1.991688in}{2.993773in}}%
\pgfpathlineto{\pgfqpoint{2.037699in}{3.023526in}}%
\pgfpathlineto{\pgfqpoint{2.083711in}{3.055045in}}%
\pgfpathlineto{\pgfqpoint{2.129722in}{3.079903in}}%
\pgfpathlineto{\pgfqpoint{2.175733in}{3.123356in}}%
\pgfpathlineto{\pgfqpoint{2.221744in}{3.133790in}}%
\pgfpathlineto{\pgfqpoint{2.267755in}{3.164091in}}%
\pgfpathlineto{\pgfqpoint{2.313766in}{3.199526in}}%
\pgfpathlineto{\pgfqpoint{2.359777in}{3.207524in}}%
\pgfpathlineto{\pgfqpoint{2.405788in}{3.230770in}}%
\pgfpathlineto{\pgfqpoint{2.451800in}{3.253720in}}%
\pgfpathlineto{\pgfqpoint{2.497811in}{3.273668in}}%
\pgfpathlineto{\pgfqpoint{2.543822in}{3.340186in}}%
\pgfpathlineto{\pgfqpoint{2.589833in}{3.313332in}}%
\pgfpathlineto{\pgfqpoint{2.635844in}{3.331177in}}%
\pgfpathlineto{\pgfqpoint{2.681855in}{3.356510in}}%
\pgfpathlineto{\pgfqpoint{2.727866in}{3.365153in}}%
\pgfpathlineto{\pgfqpoint{2.773878in}{3.384762in}}%
\pgfpathlineto{\pgfqpoint{2.819889in}{3.403074in}}%
\pgfpathlineto{\pgfqpoint{2.865900in}{3.419007in}}%
\pgfpathlineto{\pgfqpoint{2.911911in}{3.451953in}}%
\pgfpathlineto{\pgfqpoint{2.957922in}{3.467708in}}%
\pgfpathlineto{\pgfqpoint{3.003933in}{3.496474in}}%
\pgfpathlineto{\pgfqpoint{3.049944in}{3.497938in}}%
\pgfpathlineto{\pgfqpoint{3.095955in}{3.513123in}}%
\pgfpathlineto{\pgfqpoint{3.141967in}{3.531356in}}%
\pgfpathlineto{\pgfqpoint{3.187978in}{3.545454in}}%
\pgfpathlineto{\pgfqpoint{3.233989in}{3.558178in}}%
\pgfpathlineto{\pgfqpoint{3.280000in}{3.590782in}}%
\pgfpathlineto{\pgfqpoint{3.326011in}{3.588698in}}%
\pgfpathlineto{\pgfqpoint{3.372022in}{3.609170in}}%
\pgfpathlineto{\pgfqpoint{3.418033in}{3.615055in}}%
\pgfpathlineto{\pgfqpoint{3.464045in}{3.626086in}}%
\pgfpathlineto{\pgfqpoint{3.510056in}{3.638664in}}%
\pgfpathlineto{\pgfqpoint{3.556067in}{3.652511in}}%
\pgfpathlineto{\pgfqpoint{3.602078in}{3.673086in}}%
\pgfpathlineto{\pgfqpoint{3.648089in}{3.675950in}}%
\pgfpathlineto{\pgfqpoint{3.694100in}{3.686087in}}%
\pgfpathlineto{\pgfqpoint{3.740111in}{3.701079in}}%
\pgfpathlineto{\pgfqpoint{3.786122in}{3.711650in}}%
\pgfpathlineto{\pgfqpoint{3.832134in}{3.725585in}}%
\pgfpathlineto{\pgfqpoint{3.878145in}{3.729399in}}%
\pgfpathlineto{\pgfqpoint{3.924156in}{3.758430in}}%
\pgfpathlineto{\pgfqpoint{3.970167in}{3.754385in}}%
\pgfpathlineto{\pgfqpoint{4.016178in}{3.764984in}}%
\pgfpathlineto{\pgfqpoint{4.062189in}{3.781972in}}%
\pgfpathlineto{\pgfqpoint{4.108200in}{3.786141in}}%
\pgfpathlineto{\pgfqpoint{4.154212in}{3.813324in}}%
\pgfpathlineto{\pgfqpoint{4.200223in}{3.819929in}}%
\pgfpathlineto{\pgfqpoint{4.246234in}{3.854044in}}%
\pgfpathlineto{\pgfqpoint{4.292245in}{3.826816in}}%
\pgfpathlineto{\pgfqpoint{4.338256in}{3.832906in}}%
\pgfpathlineto{\pgfqpoint{4.384267in}{3.845319in}}%
\pgfpathlineto{\pgfqpoint{4.430278in}{3.851790in}}%
\pgfpathlineto{\pgfqpoint{4.476289in}{3.875919in}}%
\pgfpathlineto{\pgfqpoint{4.522301in}{3.868730in}}%
\pgfpathlineto{\pgfqpoint{4.568312in}{3.890417in}}%
\pgfpathlineto{\pgfqpoint{4.614323in}{3.887643in}}%
\pgfpathlineto{\pgfqpoint{4.660334in}{3.893481in}}%
\pgfpathlineto{\pgfqpoint{4.706345in}{3.910067in}}%
\pgfpathlineto{\pgfqpoint{4.752356in}{3.912611in}}%
\pgfpathlineto{\pgfqpoint{4.798367in}{3.934194in}}%
\pgfpathlineto{\pgfqpoint{4.844378in}{3.941686in}}%
\pgfpathlineto{\pgfqpoint{4.890390in}{3.959999in}}%
\pgfpathlineto{\pgfqpoint{4.936401in}{3.960787in}}%
\pgfpathlineto{\pgfqpoint{4.982412in}{3.989201in}}%
\pgfpathlineto{\pgfqpoint{5.028423in}{3.977935in}}%
\pgfpathlineto{\pgfqpoint{5.074434in}{3.993058in}}%
\pgfpathlineto{\pgfqpoint{5.120445in}{3.991950in}}%
\pgfpathlineto{\pgfqpoint{5.166456in}{4.018365in}}%
\pgfpathlineto{\pgfqpoint{5.212468in}{4.005974in}}%
\pgfpathlineto{\pgfqpoint{5.258479in}{4.042823in}}%
\pgfpathlineto{\pgfqpoint{5.304490in}{4.020002in}}%
\pgfpathlineto{\pgfqpoint{5.350501in}{4.048212in}}%
\pgfpathlineto{\pgfqpoint{5.396512in}{4.032921in}}%
\pgfpathlineto{\pgfqpoint{5.442523in}{4.041442in}}%
\pgfpathlineto{\pgfqpoint{5.488534in}{4.049079in}}%
\pgfpathlineto{\pgfqpoint{5.534545in}{4.056000in}}%
\pgfusepath{stroke}%
\end{pgfscope}%
\begin{pgfscope}%
\pgfpathrectangle{\pgfqpoint{0.800000in}{0.528000in}}{\pgfqpoint{4.960000in}{3.696000in}}%
\pgfusepath{clip}%
\pgfsetrectcap%
\pgfsetroundjoin%
\pgfsetlinewidth{1.505625pt}%
\definecolor{currentstroke}{rgb}{1.000000,0.498039,0.054902}%
\pgfsetstrokecolor{currentstroke}%
\pgfsetdash{}{0pt}%
\pgfpathmoveto{\pgfqpoint{1.025455in}{0.696000in}}%
\pgfpathlineto{\pgfqpoint{1.071466in}{1.148080in}}%
\pgfpathlineto{\pgfqpoint{1.117477in}{1.451580in}}%
\pgfpathlineto{\pgfqpoint{1.163488in}{1.694376in}}%
\pgfpathlineto{\pgfqpoint{1.209499in}{1.877241in}}%
\pgfpathlineto{\pgfqpoint{1.255510in}{2.055053in}}%
\pgfpathlineto{\pgfqpoint{1.301521in}{2.182181in}}%
\pgfpathlineto{\pgfqpoint{1.347532in}{2.227115in}}%
\pgfpathlineto{\pgfqpoint{1.393544in}{2.319081in}}%
\pgfpathlineto{\pgfqpoint{1.439555in}{2.390304in}}%
\pgfpathlineto{\pgfqpoint{1.485566in}{2.463870in}}%
\pgfpathlineto{\pgfqpoint{1.531577in}{2.532183in}}%
\pgfpathlineto{\pgfqpoint{1.577588in}{2.585943in}}%
\pgfpathlineto{\pgfqpoint{1.623599in}{2.637518in}}%
\pgfpathlineto{\pgfqpoint{1.669610in}{2.687167in}}%
\pgfpathlineto{\pgfqpoint{1.715622in}{2.735600in}}%
\pgfpathlineto{\pgfqpoint{1.761633in}{2.777365in}}%
\pgfpathlineto{\pgfqpoint{1.807644in}{2.814781in}}%
\pgfpathlineto{\pgfqpoint{1.853655in}{2.853715in}}%
\pgfpathlineto{\pgfqpoint{1.899666in}{2.887820in}}%
\pgfpathlineto{\pgfqpoint{1.945677in}{2.924371in}}%
\pgfpathlineto{\pgfqpoint{1.991688in}{2.953796in}}%
\pgfpathlineto{\pgfqpoint{2.037699in}{2.999363in}}%
\pgfpathlineto{\pgfqpoint{2.083711in}{3.019911in}}%
\pgfpathlineto{\pgfqpoint{2.129722in}{3.050141in}}%
\pgfpathlineto{\pgfqpoint{2.175733in}{3.087656in}}%
\pgfpathlineto{\pgfqpoint{2.221744in}{3.105668in}}%
\pgfpathlineto{\pgfqpoint{2.267755in}{3.128055in}}%
\pgfpathlineto{\pgfqpoint{2.313766in}{3.157854in}}%
\pgfpathlineto{\pgfqpoint{2.359777in}{3.178001in}}%
\pgfpathlineto{\pgfqpoint{2.405788in}{3.203487in}}%
\pgfpathlineto{\pgfqpoint{2.451800in}{3.222475in}}%
\pgfpathlineto{\pgfqpoint{2.497811in}{3.245809in}}%
\pgfpathlineto{\pgfqpoint{2.543822in}{3.262181in}}%
\pgfpathlineto{\pgfqpoint{2.589833in}{3.284394in}}%
\pgfpathlineto{\pgfqpoint{2.635844in}{3.302345in}}%
\pgfpathlineto{\pgfqpoint{2.681855in}{3.320660in}}%
\pgfpathlineto{\pgfqpoint{2.727866in}{3.340172in}}%
\pgfpathlineto{\pgfqpoint{2.773878in}{3.399116in}}%
\pgfpathlineto{\pgfqpoint{2.819889in}{3.374117in}}%
\pgfpathlineto{\pgfqpoint{2.865900in}{3.392532in}}%
\pgfpathlineto{\pgfqpoint{2.911911in}{3.411894in}}%
\pgfpathlineto{\pgfqpoint{2.957922in}{3.423308in}}%
\pgfpathlineto{\pgfqpoint{3.003933in}{3.438148in}}%
\pgfpathlineto{\pgfqpoint{3.049944in}{3.457905in}}%
\pgfpathlineto{\pgfqpoint{3.095955in}{3.540160in}}%
\pgfpathlineto{\pgfqpoint{3.141967in}{3.488406in}}%
\pgfpathlineto{\pgfqpoint{3.187978in}{3.501548in}}%
\pgfpathlineto{\pgfqpoint{3.233989in}{3.514067in}}%
\pgfpathlineto{\pgfqpoint{3.280000in}{3.529285in}}%
\pgfpathlineto{\pgfqpoint{3.326011in}{3.546479in}}%
\pgfpathlineto{\pgfqpoint{3.372022in}{3.559345in}}%
\pgfpathlineto{\pgfqpoint{3.418033in}{3.579415in}}%
\pgfpathlineto{\pgfqpoint{3.464045in}{3.584314in}}%
\pgfpathlineto{\pgfqpoint{3.510056in}{3.600429in}}%
\pgfpathlineto{\pgfqpoint{3.556067in}{3.610240in}}%
\pgfpathlineto{\pgfqpoint{3.602078in}{3.624621in}}%
\pgfpathlineto{\pgfqpoint{3.648089in}{3.635630in}}%
\pgfpathlineto{\pgfqpoint{3.694100in}{3.657229in}}%
\pgfpathlineto{\pgfqpoint{3.740111in}{3.658555in}}%
\pgfpathlineto{\pgfqpoint{3.786122in}{3.671420in}}%
\pgfpathlineto{\pgfqpoint{3.832134in}{3.678769in}}%
\pgfpathlineto{\pgfqpoint{3.878145in}{3.689292in}}%
\pgfpathlineto{\pgfqpoint{3.924156in}{3.700227in}}%
\pgfpathlineto{\pgfqpoint{3.970167in}{3.712239in}}%
\pgfpathlineto{\pgfqpoint{4.016178in}{3.721249in}}%
\pgfpathlineto{\pgfqpoint{4.062189in}{3.734670in}}%
\pgfpathlineto{\pgfqpoint{4.108200in}{3.741751in}}%
\pgfpathlineto{\pgfqpoint{4.154212in}{3.751785in}}%
\pgfpathlineto{\pgfqpoint{4.200223in}{3.759916in}}%
\pgfpathlineto{\pgfqpoint{4.246234in}{3.771279in}}%
\pgfpathlineto{\pgfqpoint{4.292245in}{3.784651in}}%
\pgfpathlineto{\pgfqpoint{4.338256in}{3.786773in}}%
\pgfpathlineto{\pgfqpoint{4.384267in}{3.809299in}}%
\pgfpathlineto{\pgfqpoint{4.430278in}{3.810410in}}%
\pgfpathlineto{\pgfqpoint{4.476289in}{3.822495in}}%
\pgfpathlineto{\pgfqpoint{4.522301in}{3.828103in}}%
\pgfpathlineto{\pgfqpoint{4.568312in}{3.831374in}}%
\pgfpathlineto{\pgfqpoint{4.614323in}{3.864670in}}%
\pgfpathlineto{\pgfqpoint{4.660334in}{3.849010in}}%
\pgfpathlineto{\pgfqpoint{4.706345in}{3.861364in}}%
\pgfpathlineto{\pgfqpoint{4.752356in}{3.873758in}}%
\pgfpathlineto{\pgfqpoint{4.798367in}{3.956244in}}%
\pgfpathlineto{\pgfqpoint{4.844378in}{3.890930in}}%
\pgfpathlineto{\pgfqpoint{4.890390in}{3.900616in}}%
\pgfpathlineto{\pgfqpoint{4.936401in}{3.910026in}}%
\pgfpathlineto{\pgfqpoint{4.982412in}{3.973747in}}%
\pgfpathlineto{\pgfqpoint{5.028423in}{3.924179in}}%
\pgfpathlineto{\pgfqpoint{5.074434in}{3.927312in}}%
\pgfpathlineto{\pgfqpoint{5.120445in}{3.934553in}}%
\pgfpathlineto{\pgfqpoint{5.166456in}{3.942517in}}%
\pgfpathlineto{\pgfqpoint{5.212468in}{3.955135in}}%
\pgfpathlineto{\pgfqpoint{5.258479in}{3.956983in}}%
\pgfpathlineto{\pgfqpoint{5.304490in}{3.964282in}}%
\pgfpathlineto{\pgfqpoint{5.350501in}{3.999520in}}%
\pgfpathlineto{\pgfqpoint{5.396512in}{3.982484in}}%
\pgfpathlineto{\pgfqpoint{5.442523in}{3.995049in}}%
\pgfpathlineto{\pgfqpoint{5.488534in}{3.997729in}}%
\pgfpathlineto{\pgfqpoint{5.534545in}{4.015298in}}%
\pgfusepath{stroke}%
\end{pgfscope}%
\begin{pgfscope}%
\pgfsetrectcap%
\pgfsetmiterjoin%
\pgfsetlinewidth{0.803000pt}%
\definecolor{currentstroke}{rgb}{0.000000,0.000000,0.000000}%
\pgfsetstrokecolor{currentstroke}%
\pgfsetdash{}{0pt}%
\pgfpathmoveto{\pgfqpoint{0.800000in}{0.528000in}}%
\pgfpathlineto{\pgfqpoint{0.800000in}{4.224000in}}%
\pgfusepath{stroke}%
\end{pgfscope}%
\begin{pgfscope}%
\pgfsetrectcap%
\pgfsetmiterjoin%
\pgfsetlinewidth{0.803000pt}%
\definecolor{currentstroke}{rgb}{0.000000,0.000000,0.000000}%
\pgfsetstrokecolor{currentstroke}%
\pgfsetdash{}{0pt}%
\pgfpathmoveto{\pgfqpoint{5.760000in}{0.528000in}}%
\pgfpathlineto{\pgfqpoint{5.760000in}{4.224000in}}%
\pgfusepath{stroke}%
\end{pgfscope}%
\begin{pgfscope}%
\pgfsetrectcap%
\pgfsetmiterjoin%
\pgfsetlinewidth{0.803000pt}%
\definecolor{currentstroke}{rgb}{0.000000,0.000000,0.000000}%
\pgfsetstrokecolor{currentstroke}%
\pgfsetdash{}{0pt}%
\pgfpathmoveto{\pgfqpoint{0.800000in}{0.528000in}}%
\pgfpathlineto{\pgfqpoint{5.760000in}{0.528000in}}%
\pgfusepath{stroke}%
\end{pgfscope}%
\begin{pgfscope}%
\pgfsetrectcap%
\pgfsetmiterjoin%
\pgfsetlinewidth{0.803000pt}%
\definecolor{currentstroke}{rgb}{0.000000,0.000000,0.000000}%
\pgfsetstrokecolor{currentstroke}%
\pgfsetdash{}{0pt}%
\pgfpathmoveto{\pgfqpoint{0.800000in}{4.224000in}}%
\pgfpathlineto{\pgfqpoint{5.760000in}{4.224000in}}%
\pgfusepath{stroke}%
\end{pgfscope}%
\begin{pgfscope}%
\pgfsetbuttcap%
\pgfsetmiterjoin%
\definecolor{currentfill}{rgb}{1.000000,1.000000,1.000000}%
\pgfsetfillcolor{currentfill}%
\pgfsetfillopacity{0.800000}%
\pgfsetlinewidth{1.003750pt}%
\definecolor{currentstroke}{rgb}{0.800000,0.800000,0.800000}%
\pgfsetstrokecolor{currentstroke}%
\pgfsetstrokeopacity{0.800000}%
\pgfsetdash{}{0pt}%
\pgfpathmoveto{\pgfqpoint{0.897222in}{3.725543in}}%
\pgfpathlineto{\pgfqpoint{2.014507in}{3.725543in}}%
\pgfpathquadraticcurveto{\pgfqpoint{2.042285in}{3.725543in}}{\pgfqpoint{2.042285in}{3.753321in}}%
\pgfpathlineto{\pgfqpoint{2.042285in}{4.126778in}}%
\pgfpathquadraticcurveto{\pgfqpoint{2.042285in}{4.154556in}}{\pgfqpoint{2.014507in}{4.154556in}}%
\pgfpathlineto{\pgfqpoint{0.897222in}{4.154556in}}%
\pgfpathquadraticcurveto{\pgfqpoint{0.869444in}{4.154556in}}{\pgfqpoint{0.869444in}{4.126778in}}%
\pgfpathlineto{\pgfqpoint{0.869444in}{3.753321in}}%
\pgfpathquadraticcurveto{\pgfqpoint{0.869444in}{3.725543in}}{\pgfqpoint{0.897222in}{3.725543in}}%
\pgfpathlineto{\pgfqpoint{0.897222in}{3.725543in}}%
\pgfpathclose%
\pgfusepath{stroke,fill}%
\end{pgfscope}%
\begin{pgfscope}%
\pgfsetrectcap%
\pgfsetroundjoin%
\pgfsetlinewidth{1.505625pt}%
\definecolor{currentstroke}{rgb}{0.121569,0.466667,0.705882}%
\pgfsetstrokecolor{currentstroke}%
\pgfsetdash{}{0pt}%
\pgfpathmoveto{\pgfqpoint{0.925000in}{4.050389in}}%
\pgfpathlineto{\pgfqpoint{1.063889in}{4.050389in}}%
\pgfpathlineto{\pgfqpoint{1.202778in}{4.050389in}}%
\pgfusepath{stroke}%
\end{pgfscope}%
\begin{pgfscope}%
\definecolor{textcolor}{rgb}{0.000000,0.000000,0.000000}%
\pgfsetstrokecolor{textcolor}%
\pgfsetfillcolor{textcolor}%
\pgftext[x=1.313889in,y=4.001778in,left,base]{\color{textcolor}\rmfamily\fontsize{10.000000}{12.000000}\selectfont mergesort}%
\end{pgfscope}%
\begin{pgfscope}%
\pgfsetrectcap%
\pgfsetroundjoin%
\pgfsetlinewidth{1.505625pt}%
\definecolor{currentstroke}{rgb}{1.000000,0.498039,0.054902}%
\pgfsetstrokecolor{currentstroke}%
\pgfsetdash{}{0pt}%
\pgfpathmoveto{\pgfqpoint{0.925000in}{3.856716in}}%
\pgfpathlineto{\pgfqpoint{1.063889in}{3.856716in}}%
\pgfpathlineto{\pgfqpoint{1.202778in}{3.856716in}}%
\pgfusepath{stroke}%
\end{pgfscope}%
\begin{pgfscope}%
\definecolor{textcolor}{rgb}{0.000000,0.000000,0.000000}%
\pgfsetstrokecolor{textcolor}%
\pgfsetfillcolor{textcolor}%
\pgftext[x=1.313889in,y=3.808105in,left,base]{\color{textcolor}\rmfamily\fontsize{10.000000}{12.000000}\selectfont bmergesort}%
\end{pgfscope}%
\end{pgfpicture}%
\makeatother%
\endgroup%

\subsubsection{Memory}
%% Creator: Matplotlib, PGF backend
%%
%% To include the figure in your LaTeX document, write
%%   \input{<filename>.pgf}
%%
%% Make sure the required packages are loaded in your preamble
%%   \usepackage{pgf}
%%
%% Also ensure that all the required font packages are loaded; for instance,
%% the lmodern package is sometimes necessary when using math font.
%%   \usepackage{lmodern}
%%
%% Figures using additional raster images can only be included by \input if
%% they are in the same directory as the main LaTeX file. For loading figures
%% from other directories you can use the `import` package
%%   \usepackage{import}
%%
%% and then include the figures with
%%   \import{<path to file>}{<filename>.pgf}
%%
%% Matplotlib used the following preamble
%%   
%%   \makeatletter\@ifpackageloaded{underscore}{}{\usepackage[strings]{underscore}}\makeatother
%%
\begingroup%
\makeatletter%
\begin{pgfpicture}%
\pgfpathrectangle{\pgfpointorigin}{\pgfqpoint{6.400000in}{4.800000in}}%
\pgfusepath{use as bounding box, clip}%
\begin{pgfscope}%
\pgfsetbuttcap%
\pgfsetmiterjoin%
\definecolor{currentfill}{rgb}{1.000000,1.000000,1.000000}%
\pgfsetfillcolor{currentfill}%
\pgfsetlinewidth{0.000000pt}%
\definecolor{currentstroke}{rgb}{1.000000,1.000000,1.000000}%
\pgfsetstrokecolor{currentstroke}%
\pgfsetdash{}{0pt}%
\pgfpathmoveto{\pgfqpoint{0.000000in}{0.000000in}}%
\pgfpathlineto{\pgfqpoint{6.400000in}{0.000000in}}%
\pgfpathlineto{\pgfqpoint{6.400000in}{4.800000in}}%
\pgfpathlineto{\pgfqpoint{0.000000in}{4.800000in}}%
\pgfpathlineto{\pgfqpoint{0.000000in}{0.000000in}}%
\pgfpathclose%
\pgfusepath{fill}%
\end{pgfscope}%
\begin{pgfscope}%
\pgfsetbuttcap%
\pgfsetmiterjoin%
\definecolor{currentfill}{rgb}{1.000000,1.000000,1.000000}%
\pgfsetfillcolor{currentfill}%
\pgfsetlinewidth{0.000000pt}%
\definecolor{currentstroke}{rgb}{0.000000,0.000000,0.000000}%
\pgfsetstrokecolor{currentstroke}%
\pgfsetstrokeopacity{0.000000}%
\pgfsetdash{}{0pt}%
\pgfpathmoveto{\pgfqpoint{0.800000in}{0.528000in}}%
\pgfpathlineto{\pgfqpoint{5.760000in}{0.528000in}}%
\pgfpathlineto{\pgfqpoint{5.760000in}{4.224000in}}%
\pgfpathlineto{\pgfqpoint{0.800000in}{4.224000in}}%
\pgfpathlineto{\pgfqpoint{0.800000in}{0.528000in}}%
\pgfpathclose%
\pgfusepath{fill}%
\end{pgfscope}%
\begin{pgfscope}%
\pgfsetbuttcap%
\pgfsetroundjoin%
\definecolor{currentfill}{rgb}{0.000000,0.000000,0.000000}%
\pgfsetfillcolor{currentfill}%
\pgfsetlinewidth{0.803000pt}%
\definecolor{currentstroke}{rgb}{0.000000,0.000000,0.000000}%
\pgfsetstrokecolor{currentstroke}%
\pgfsetdash{}{0pt}%
\pgfsys@defobject{currentmarker}{\pgfqpoint{0.000000in}{-0.048611in}}{\pgfqpoint{0.000000in}{0.000000in}}{%
\pgfpathmoveto{\pgfqpoint{0.000000in}{0.000000in}}%
\pgfpathlineto{\pgfqpoint{0.000000in}{-0.048611in}}%
\pgfusepath{stroke,fill}%
}%
\begin{pgfscope}%
\pgfsys@transformshift{0.979443in}{0.528000in}%
\pgfsys@useobject{currentmarker}{}%
\end{pgfscope}%
\end{pgfscope}%
\begin{pgfscope}%
\definecolor{textcolor}{rgb}{0.000000,0.000000,0.000000}%
\pgfsetstrokecolor{textcolor}%
\pgfsetfillcolor{textcolor}%
\pgftext[x=0.979443in,y=0.430778in,,top]{\color{textcolor}\rmfamily\fontsize{10.000000}{12.000000}\selectfont \(\displaystyle {0}\)}%
\end{pgfscope}%
\begin{pgfscope}%
\pgfsetbuttcap%
\pgfsetroundjoin%
\definecolor{currentfill}{rgb}{0.000000,0.000000,0.000000}%
\pgfsetfillcolor{currentfill}%
\pgfsetlinewidth{0.803000pt}%
\definecolor{currentstroke}{rgb}{0.000000,0.000000,0.000000}%
\pgfsetstrokecolor{currentstroke}%
\pgfsetdash{}{0pt}%
\pgfsys@defobject{currentmarker}{\pgfqpoint{0.000000in}{-0.048611in}}{\pgfqpoint{0.000000in}{0.000000in}}{%
\pgfpathmoveto{\pgfqpoint{0.000000in}{0.000000in}}%
\pgfpathlineto{\pgfqpoint{0.000000in}{-0.048611in}}%
\pgfusepath{stroke,fill}%
}%
\begin{pgfscope}%
\pgfsys@transformshift{1.899666in}{0.528000in}%
\pgfsys@useobject{currentmarker}{}%
\end{pgfscope}%
\end{pgfscope}%
\begin{pgfscope}%
\definecolor{textcolor}{rgb}{0.000000,0.000000,0.000000}%
\pgfsetstrokecolor{textcolor}%
\pgfsetfillcolor{textcolor}%
\pgftext[x=1.899666in,y=0.430778in,,top]{\color{textcolor}\rmfamily\fontsize{10.000000}{12.000000}\selectfont \(\displaystyle {2000}\)}%
\end{pgfscope}%
\begin{pgfscope}%
\pgfsetbuttcap%
\pgfsetroundjoin%
\definecolor{currentfill}{rgb}{0.000000,0.000000,0.000000}%
\pgfsetfillcolor{currentfill}%
\pgfsetlinewidth{0.803000pt}%
\definecolor{currentstroke}{rgb}{0.000000,0.000000,0.000000}%
\pgfsetstrokecolor{currentstroke}%
\pgfsetdash{}{0pt}%
\pgfsys@defobject{currentmarker}{\pgfqpoint{0.000000in}{-0.048611in}}{\pgfqpoint{0.000000in}{0.000000in}}{%
\pgfpathmoveto{\pgfqpoint{0.000000in}{0.000000in}}%
\pgfpathlineto{\pgfqpoint{0.000000in}{-0.048611in}}%
\pgfusepath{stroke,fill}%
}%
\begin{pgfscope}%
\pgfsys@transformshift{2.819889in}{0.528000in}%
\pgfsys@useobject{currentmarker}{}%
\end{pgfscope}%
\end{pgfscope}%
\begin{pgfscope}%
\definecolor{textcolor}{rgb}{0.000000,0.000000,0.000000}%
\pgfsetstrokecolor{textcolor}%
\pgfsetfillcolor{textcolor}%
\pgftext[x=2.819889in,y=0.430778in,,top]{\color{textcolor}\rmfamily\fontsize{10.000000}{12.000000}\selectfont \(\displaystyle {4000}\)}%
\end{pgfscope}%
\begin{pgfscope}%
\pgfsetbuttcap%
\pgfsetroundjoin%
\definecolor{currentfill}{rgb}{0.000000,0.000000,0.000000}%
\pgfsetfillcolor{currentfill}%
\pgfsetlinewidth{0.803000pt}%
\definecolor{currentstroke}{rgb}{0.000000,0.000000,0.000000}%
\pgfsetstrokecolor{currentstroke}%
\pgfsetdash{}{0pt}%
\pgfsys@defobject{currentmarker}{\pgfqpoint{0.000000in}{-0.048611in}}{\pgfqpoint{0.000000in}{0.000000in}}{%
\pgfpathmoveto{\pgfqpoint{0.000000in}{0.000000in}}%
\pgfpathlineto{\pgfqpoint{0.000000in}{-0.048611in}}%
\pgfusepath{stroke,fill}%
}%
\begin{pgfscope}%
\pgfsys@transformshift{3.740111in}{0.528000in}%
\pgfsys@useobject{currentmarker}{}%
\end{pgfscope}%
\end{pgfscope}%
\begin{pgfscope}%
\definecolor{textcolor}{rgb}{0.000000,0.000000,0.000000}%
\pgfsetstrokecolor{textcolor}%
\pgfsetfillcolor{textcolor}%
\pgftext[x=3.740111in,y=0.430778in,,top]{\color{textcolor}\rmfamily\fontsize{10.000000}{12.000000}\selectfont \(\displaystyle {6000}\)}%
\end{pgfscope}%
\begin{pgfscope}%
\pgfsetbuttcap%
\pgfsetroundjoin%
\definecolor{currentfill}{rgb}{0.000000,0.000000,0.000000}%
\pgfsetfillcolor{currentfill}%
\pgfsetlinewidth{0.803000pt}%
\definecolor{currentstroke}{rgb}{0.000000,0.000000,0.000000}%
\pgfsetstrokecolor{currentstroke}%
\pgfsetdash{}{0pt}%
\pgfsys@defobject{currentmarker}{\pgfqpoint{0.000000in}{-0.048611in}}{\pgfqpoint{0.000000in}{0.000000in}}{%
\pgfpathmoveto{\pgfqpoint{0.000000in}{0.000000in}}%
\pgfpathlineto{\pgfqpoint{0.000000in}{-0.048611in}}%
\pgfusepath{stroke,fill}%
}%
\begin{pgfscope}%
\pgfsys@transformshift{4.660334in}{0.528000in}%
\pgfsys@useobject{currentmarker}{}%
\end{pgfscope}%
\end{pgfscope}%
\begin{pgfscope}%
\definecolor{textcolor}{rgb}{0.000000,0.000000,0.000000}%
\pgfsetstrokecolor{textcolor}%
\pgfsetfillcolor{textcolor}%
\pgftext[x=4.660334in,y=0.430778in,,top]{\color{textcolor}\rmfamily\fontsize{10.000000}{12.000000}\selectfont \(\displaystyle {8000}\)}%
\end{pgfscope}%
\begin{pgfscope}%
\pgfsetbuttcap%
\pgfsetroundjoin%
\definecolor{currentfill}{rgb}{0.000000,0.000000,0.000000}%
\pgfsetfillcolor{currentfill}%
\pgfsetlinewidth{0.803000pt}%
\definecolor{currentstroke}{rgb}{0.000000,0.000000,0.000000}%
\pgfsetstrokecolor{currentstroke}%
\pgfsetdash{}{0pt}%
\pgfsys@defobject{currentmarker}{\pgfqpoint{0.000000in}{-0.048611in}}{\pgfqpoint{0.000000in}{0.000000in}}{%
\pgfpathmoveto{\pgfqpoint{0.000000in}{0.000000in}}%
\pgfpathlineto{\pgfqpoint{0.000000in}{-0.048611in}}%
\pgfusepath{stroke,fill}%
}%
\begin{pgfscope}%
\pgfsys@transformshift{5.580557in}{0.528000in}%
\pgfsys@useobject{currentmarker}{}%
\end{pgfscope}%
\end{pgfscope}%
\begin{pgfscope}%
\definecolor{textcolor}{rgb}{0.000000,0.000000,0.000000}%
\pgfsetstrokecolor{textcolor}%
\pgfsetfillcolor{textcolor}%
\pgftext[x=5.580557in,y=0.430778in,,top]{\color{textcolor}\rmfamily\fontsize{10.000000}{12.000000}\selectfont \(\displaystyle {10000}\)}%
\end{pgfscope}%
\begin{pgfscope}%
\definecolor{textcolor}{rgb}{0.000000,0.000000,0.000000}%
\pgfsetstrokecolor{textcolor}%
\pgfsetfillcolor{textcolor}%
\pgftext[x=3.280000in,y=0.251766in,,top]{\color{textcolor}\rmfamily\fontsize{10.000000}{12.000000}\selectfont Input Size}%
\end{pgfscope}%
\begin{pgfscope}%
\pgfsetbuttcap%
\pgfsetroundjoin%
\definecolor{currentfill}{rgb}{0.000000,0.000000,0.000000}%
\pgfsetfillcolor{currentfill}%
\pgfsetlinewidth{0.803000pt}%
\definecolor{currentstroke}{rgb}{0.000000,0.000000,0.000000}%
\pgfsetstrokecolor{currentstroke}%
\pgfsetdash{}{0pt}%
\pgfsys@defobject{currentmarker}{\pgfqpoint{-0.048611in}{0.000000in}}{\pgfqpoint{-0.000000in}{0.000000in}}{%
\pgfpathmoveto{\pgfqpoint{-0.000000in}{0.000000in}}%
\pgfpathlineto{\pgfqpoint{-0.048611in}{0.000000in}}%
\pgfusepath{stroke,fill}%
}%
\begin{pgfscope}%
\pgfsys@transformshift{0.800000in}{1.939163in}%
\pgfsys@useobject{currentmarker}{}%
\end{pgfscope}%
\end{pgfscope}%
\begin{pgfscope}%
\definecolor{textcolor}{rgb}{0.000000,0.000000,0.000000}%
\pgfsetstrokecolor{textcolor}%
\pgfsetfillcolor{textcolor}%
\pgftext[x=0.501581in, y=1.890938in, left, base]{\color{textcolor}\rmfamily\fontsize{10.000000}{12.000000}\selectfont \(\displaystyle {10^{4}}\)}%
\end{pgfscope}%
\begin{pgfscope}%
\pgfsetbuttcap%
\pgfsetroundjoin%
\definecolor{currentfill}{rgb}{0.000000,0.000000,0.000000}%
\pgfsetfillcolor{currentfill}%
\pgfsetlinewidth{0.803000pt}%
\definecolor{currentstroke}{rgb}{0.000000,0.000000,0.000000}%
\pgfsetstrokecolor{currentstroke}%
\pgfsetdash{}{0pt}%
\pgfsys@defobject{currentmarker}{\pgfqpoint{-0.048611in}{0.000000in}}{\pgfqpoint{-0.000000in}{0.000000in}}{%
\pgfpathmoveto{\pgfqpoint{-0.000000in}{0.000000in}}%
\pgfpathlineto{\pgfqpoint{-0.048611in}{0.000000in}}%
\pgfusepath{stroke,fill}%
}%
\begin{pgfscope}%
\pgfsys@transformshift{0.800000in}{3.702282in}%
\pgfsys@useobject{currentmarker}{}%
\end{pgfscope}%
\end{pgfscope}%
\begin{pgfscope}%
\definecolor{textcolor}{rgb}{0.000000,0.000000,0.000000}%
\pgfsetstrokecolor{textcolor}%
\pgfsetfillcolor{textcolor}%
\pgftext[x=0.501581in, y=3.654057in, left, base]{\color{textcolor}\rmfamily\fontsize{10.000000}{12.000000}\selectfont \(\displaystyle {10^{5}}\)}%
\end{pgfscope}%
\begin{pgfscope}%
\pgfsetbuttcap%
\pgfsetroundjoin%
\definecolor{currentfill}{rgb}{0.000000,0.000000,0.000000}%
\pgfsetfillcolor{currentfill}%
\pgfsetlinewidth{0.602250pt}%
\definecolor{currentstroke}{rgb}{0.000000,0.000000,0.000000}%
\pgfsetstrokecolor{currentstroke}%
\pgfsetdash{}{0pt}%
\pgfsys@defobject{currentmarker}{\pgfqpoint{-0.027778in}{0.000000in}}{\pgfqpoint{-0.000000in}{0.000000in}}{%
\pgfpathmoveto{\pgfqpoint{-0.000000in}{0.000000in}}%
\pgfpathlineto{\pgfqpoint{-0.027778in}{0.000000in}}%
\pgfusepath{stroke,fill}%
}%
\begin{pgfscope}%
\pgfsys@transformshift{0.800000in}{0.706796in}%
\pgfsys@useobject{currentmarker}{}%
\end{pgfscope}%
\end{pgfscope}%
\begin{pgfscope}%
\pgfsetbuttcap%
\pgfsetroundjoin%
\definecolor{currentfill}{rgb}{0.000000,0.000000,0.000000}%
\pgfsetfillcolor{currentfill}%
\pgfsetlinewidth{0.602250pt}%
\definecolor{currentstroke}{rgb}{0.000000,0.000000,0.000000}%
\pgfsetstrokecolor{currentstroke}%
\pgfsetdash{}{0pt}%
\pgfsys@defobject{currentmarker}{\pgfqpoint{-0.027778in}{0.000000in}}{\pgfqpoint{-0.000000in}{0.000000in}}{%
\pgfpathmoveto{\pgfqpoint{-0.000000in}{0.000000in}}%
\pgfpathlineto{\pgfqpoint{-0.027778in}{0.000000in}}%
\pgfusepath{stroke,fill}%
}%
\begin{pgfscope}%
\pgfsys@transformshift{0.800000in}{1.017266in}%
\pgfsys@useobject{currentmarker}{}%
\end{pgfscope}%
\end{pgfscope}%
\begin{pgfscope}%
\pgfsetbuttcap%
\pgfsetroundjoin%
\definecolor{currentfill}{rgb}{0.000000,0.000000,0.000000}%
\pgfsetfillcolor{currentfill}%
\pgfsetlinewidth{0.602250pt}%
\definecolor{currentstroke}{rgb}{0.000000,0.000000,0.000000}%
\pgfsetstrokecolor{currentstroke}%
\pgfsetdash{}{0pt}%
\pgfsys@defobject{currentmarker}{\pgfqpoint{-0.027778in}{0.000000in}}{\pgfqpoint{-0.000000in}{0.000000in}}{%
\pgfpathmoveto{\pgfqpoint{-0.000000in}{0.000000in}}%
\pgfpathlineto{\pgfqpoint{-0.027778in}{0.000000in}}%
\pgfusepath{stroke,fill}%
}%
\begin{pgfscope}%
\pgfsys@transformshift{0.800000in}{1.237547in}%
\pgfsys@useobject{currentmarker}{}%
\end{pgfscope}%
\end{pgfscope}%
\begin{pgfscope}%
\pgfsetbuttcap%
\pgfsetroundjoin%
\definecolor{currentfill}{rgb}{0.000000,0.000000,0.000000}%
\pgfsetfillcolor{currentfill}%
\pgfsetlinewidth{0.602250pt}%
\definecolor{currentstroke}{rgb}{0.000000,0.000000,0.000000}%
\pgfsetstrokecolor{currentstroke}%
\pgfsetdash{}{0pt}%
\pgfsys@defobject{currentmarker}{\pgfqpoint{-0.027778in}{0.000000in}}{\pgfqpoint{-0.000000in}{0.000000in}}{%
\pgfpathmoveto{\pgfqpoint{-0.000000in}{0.000000in}}%
\pgfpathlineto{\pgfqpoint{-0.027778in}{0.000000in}}%
\pgfusepath{stroke,fill}%
}%
\begin{pgfscope}%
\pgfsys@transformshift{0.800000in}{1.408411in}%
\pgfsys@useobject{currentmarker}{}%
\end{pgfscope}%
\end{pgfscope}%
\begin{pgfscope}%
\pgfsetbuttcap%
\pgfsetroundjoin%
\definecolor{currentfill}{rgb}{0.000000,0.000000,0.000000}%
\pgfsetfillcolor{currentfill}%
\pgfsetlinewidth{0.602250pt}%
\definecolor{currentstroke}{rgb}{0.000000,0.000000,0.000000}%
\pgfsetstrokecolor{currentstroke}%
\pgfsetdash{}{0pt}%
\pgfsys@defobject{currentmarker}{\pgfqpoint{-0.027778in}{0.000000in}}{\pgfqpoint{-0.000000in}{0.000000in}}{%
\pgfpathmoveto{\pgfqpoint{-0.000000in}{0.000000in}}%
\pgfpathlineto{\pgfqpoint{-0.027778in}{0.000000in}}%
\pgfusepath{stroke,fill}%
}%
\begin{pgfscope}%
\pgfsys@transformshift{0.800000in}{1.548017in}%
\pgfsys@useobject{currentmarker}{}%
\end{pgfscope}%
\end{pgfscope}%
\begin{pgfscope}%
\pgfsetbuttcap%
\pgfsetroundjoin%
\definecolor{currentfill}{rgb}{0.000000,0.000000,0.000000}%
\pgfsetfillcolor{currentfill}%
\pgfsetlinewidth{0.602250pt}%
\definecolor{currentstroke}{rgb}{0.000000,0.000000,0.000000}%
\pgfsetstrokecolor{currentstroke}%
\pgfsetdash{}{0pt}%
\pgfsys@defobject{currentmarker}{\pgfqpoint{-0.027778in}{0.000000in}}{\pgfqpoint{-0.000000in}{0.000000in}}{%
\pgfpathmoveto{\pgfqpoint{-0.000000in}{0.000000in}}%
\pgfpathlineto{\pgfqpoint{-0.027778in}{0.000000in}}%
\pgfusepath{stroke,fill}%
}%
\begin{pgfscope}%
\pgfsys@transformshift{0.800000in}{1.666052in}%
\pgfsys@useobject{currentmarker}{}%
\end{pgfscope}%
\end{pgfscope}%
\begin{pgfscope}%
\pgfsetbuttcap%
\pgfsetroundjoin%
\definecolor{currentfill}{rgb}{0.000000,0.000000,0.000000}%
\pgfsetfillcolor{currentfill}%
\pgfsetlinewidth{0.602250pt}%
\definecolor{currentstroke}{rgb}{0.000000,0.000000,0.000000}%
\pgfsetstrokecolor{currentstroke}%
\pgfsetdash{}{0pt}%
\pgfsys@defobject{currentmarker}{\pgfqpoint{-0.027778in}{0.000000in}}{\pgfqpoint{-0.000000in}{0.000000in}}{%
\pgfpathmoveto{\pgfqpoint{-0.000000in}{0.000000in}}%
\pgfpathlineto{\pgfqpoint{-0.027778in}{0.000000in}}%
\pgfusepath{stroke,fill}%
}%
\begin{pgfscope}%
\pgfsys@transformshift{0.800000in}{1.768299in}%
\pgfsys@useobject{currentmarker}{}%
\end{pgfscope}%
\end{pgfscope}%
\begin{pgfscope}%
\pgfsetbuttcap%
\pgfsetroundjoin%
\definecolor{currentfill}{rgb}{0.000000,0.000000,0.000000}%
\pgfsetfillcolor{currentfill}%
\pgfsetlinewidth{0.602250pt}%
\definecolor{currentstroke}{rgb}{0.000000,0.000000,0.000000}%
\pgfsetstrokecolor{currentstroke}%
\pgfsetdash{}{0pt}%
\pgfsys@defobject{currentmarker}{\pgfqpoint{-0.027778in}{0.000000in}}{\pgfqpoint{-0.000000in}{0.000000in}}{%
\pgfpathmoveto{\pgfqpoint{-0.000000in}{0.000000in}}%
\pgfpathlineto{\pgfqpoint{-0.027778in}{0.000000in}}%
\pgfusepath{stroke,fill}%
}%
\begin{pgfscope}%
\pgfsys@transformshift{0.800000in}{1.858487in}%
\pgfsys@useobject{currentmarker}{}%
\end{pgfscope}%
\end{pgfscope}%
\begin{pgfscope}%
\pgfsetbuttcap%
\pgfsetroundjoin%
\definecolor{currentfill}{rgb}{0.000000,0.000000,0.000000}%
\pgfsetfillcolor{currentfill}%
\pgfsetlinewidth{0.602250pt}%
\definecolor{currentstroke}{rgb}{0.000000,0.000000,0.000000}%
\pgfsetstrokecolor{currentstroke}%
\pgfsetdash{}{0pt}%
\pgfsys@defobject{currentmarker}{\pgfqpoint{-0.027778in}{0.000000in}}{\pgfqpoint{-0.000000in}{0.000000in}}{%
\pgfpathmoveto{\pgfqpoint{-0.000000in}{0.000000in}}%
\pgfpathlineto{\pgfqpoint{-0.027778in}{0.000000in}}%
\pgfusepath{stroke,fill}%
}%
\begin{pgfscope}%
\pgfsys@transformshift{0.800000in}{2.469915in}%
\pgfsys@useobject{currentmarker}{}%
\end{pgfscope}%
\end{pgfscope}%
\begin{pgfscope}%
\pgfsetbuttcap%
\pgfsetroundjoin%
\definecolor{currentfill}{rgb}{0.000000,0.000000,0.000000}%
\pgfsetfillcolor{currentfill}%
\pgfsetlinewidth{0.602250pt}%
\definecolor{currentstroke}{rgb}{0.000000,0.000000,0.000000}%
\pgfsetstrokecolor{currentstroke}%
\pgfsetdash{}{0pt}%
\pgfsys@defobject{currentmarker}{\pgfqpoint{-0.027778in}{0.000000in}}{\pgfqpoint{-0.000000in}{0.000000in}}{%
\pgfpathmoveto{\pgfqpoint{-0.000000in}{0.000000in}}%
\pgfpathlineto{\pgfqpoint{-0.027778in}{0.000000in}}%
\pgfusepath{stroke,fill}%
}%
\begin{pgfscope}%
\pgfsys@transformshift{0.800000in}{2.780384in}%
\pgfsys@useobject{currentmarker}{}%
\end{pgfscope}%
\end{pgfscope}%
\begin{pgfscope}%
\pgfsetbuttcap%
\pgfsetroundjoin%
\definecolor{currentfill}{rgb}{0.000000,0.000000,0.000000}%
\pgfsetfillcolor{currentfill}%
\pgfsetlinewidth{0.602250pt}%
\definecolor{currentstroke}{rgb}{0.000000,0.000000,0.000000}%
\pgfsetstrokecolor{currentstroke}%
\pgfsetdash{}{0pt}%
\pgfsys@defobject{currentmarker}{\pgfqpoint{-0.027778in}{0.000000in}}{\pgfqpoint{-0.000000in}{0.000000in}}{%
\pgfpathmoveto{\pgfqpoint{-0.000000in}{0.000000in}}%
\pgfpathlineto{\pgfqpoint{-0.027778in}{0.000000in}}%
\pgfusepath{stroke,fill}%
}%
\begin{pgfscope}%
\pgfsys@transformshift{0.800000in}{3.000666in}%
\pgfsys@useobject{currentmarker}{}%
\end{pgfscope}%
\end{pgfscope}%
\begin{pgfscope}%
\pgfsetbuttcap%
\pgfsetroundjoin%
\definecolor{currentfill}{rgb}{0.000000,0.000000,0.000000}%
\pgfsetfillcolor{currentfill}%
\pgfsetlinewidth{0.602250pt}%
\definecolor{currentstroke}{rgb}{0.000000,0.000000,0.000000}%
\pgfsetstrokecolor{currentstroke}%
\pgfsetdash{}{0pt}%
\pgfsys@defobject{currentmarker}{\pgfqpoint{-0.027778in}{0.000000in}}{\pgfqpoint{-0.000000in}{0.000000in}}{%
\pgfpathmoveto{\pgfqpoint{-0.000000in}{0.000000in}}%
\pgfpathlineto{\pgfqpoint{-0.027778in}{0.000000in}}%
\pgfusepath{stroke,fill}%
}%
\begin{pgfscope}%
\pgfsys@transformshift{0.800000in}{3.171530in}%
\pgfsys@useobject{currentmarker}{}%
\end{pgfscope}%
\end{pgfscope}%
\begin{pgfscope}%
\pgfsetbuttcap%
\pgfsetroundjoin%
\definecolor{currentfill}{rgb}{0.000000,0.000000,0.000000}%
\pgfsetfillcolor{currentfill}%
\pgfsetlinewidth{0.602250pt}%
\definecolor{currentstroke}{rgb}{0.000000,0.000000,0.000000}%
\pgfsetstrokecolor{currentstroke}%
\pgfsetdash{}{0pt}%
\pgfsys@defobject{currentmarker}{\pgfqpoint{-0.027778in}{0.000000in}}{\pgfqpoint{-0.000000in}{0.000000in}}{%
\pgfpathmoveto{\pgfqpoint{-0.000000in}{0.000000in}}%
\pgfpathlineto{\pgfqpoint{-0.027778in}{0.000000in}}%
\pgfusepath{stroke,fill}%
}%
\begin{pgfscope}%
\pgfsys@transformshift{0.800000in}{3.311136in}%
\pgfsys@useobject{currentmarker}{}%
\end{pgfscope}%
\end{pgfscope}%
\begin{pgfscope}%
\pgfsetbuttcap%
\pgfsetroundjoin%
\definecolor{currentfill}{rgb}{0.000000,0.000000,0.000000}%
\pgfsetfillcolor{currentfill}%
\pgfsetlinewidth{0.602250pt}%
\definecolor{currentstroke}{rgb}{0.000000,0.000000,0.000000}%
\pgfsetstrokecolor{currentstroke}%
\pgfsetdash{}{0pt}%
\pgfsys@defobject{currentmarker}{\pgfqpoint{-0.027778in}{0.000000in}}{\pgfqpoint{-0.000000in}{0.000000in}}{%
\pgfpathmoveto{\pgfqpoint{-0.000000in}{0.000000in}}%
\pgfpathlineto{\pgfqpoint{-0.027778in}{0.000000in}}%
\pgfusepath{stroke,fill}%
}%
\begin{pgfscope}%
\pgfsys@transformshift{0.800000in}{3.429171in}%
\pgfsys@useobject{currentmarker}{}%
\end{pgfscope}%
\end{pgfscope}%
\begin{pgfscope}%
\pgfsetbuttcap%
\pgfsetroundjoin%
\definecolor{currentfill}{rgb}{0.000000,0.000000,0.000000}%
\pgfsetfillcolor{currentfill}%
\pgfsetlinewidth{0.602250pt}%
\definecolor{currentstroke}{rgb}{0.000000,0.000000,0.000000}%
\pgfsetstrokecolor{currentstroke}%
\pgfsetdash{}{0pt}%
\pgfsys@defobject{currentmarker}{\pgfqpoint{-0.027778in}{0.000000in}}{\pgfqpoint{-0.000000in}{0.000000in}}{%
\pgfpathmoveto{\pgfqpoint{-0.000000in}{0.000000in}}%
\pgfpathlineto{\pgfqpoint{-0.027778in}{0.000000in}}%
\pgfusepath{stroke,fill}%
}%
\begin{pgfscope}%
\pgfsys@transformshift{0.800000in}{3.531418in}%
\pgfsys@useobject{currentmarker}{}%
\end{pgfscope}%
\end{pgfscope}%
\begin{pgfscope}%
\pgfsetbuttcap%
\pgfsetroundjoin%
\definecolor{currentfill}{rgb}{0.000000,0.000000,0.000000}%
\pgfsetfillcolor{currentfill}%
\pgfsetlinewidth{0.602250pt}%
\definecolor{currentstroke}{rgb}{0.000000,0.000000,0.000000}%
\pgfsetstrokecolor{currentstroke}%
\pgfsetdash{}{0pt}%
\pgfsys@defobject{currentmarker}{\pgfqpoint{-0.027778in}{0.000000in}}{\pgfqpoint{-0.000000in}{0.000000in}}{%
\pgfpathmoveto{\pgfqpoint{-0.000000in}{0.000000in}}%
\pgfpathlineto{\pgfqpoint{-0.027778in}{0.000000in}}%
\pgfusepath{stroke,fill}%
}%
\begin{pgfscope}%
\pgfsys@transformshift{0.800000in}{3.621606in}%
\pgfsys@useobject{currentmarker}{}%
\end{pgfscope}%
\end{pgfscope}%
\begin{pgfscope}%
\definecolor{textcolor}{rgb}{0.000000,0.000000,0.000000}%
\pgfsetstrokecolor{textcolor}%
\pgfsetfillcolor{textcolor}%
\pgftext[x=0.446026in,y=2.376000in,,bottom,rotate=90.000000]{\color{textcolor}\rmfamily\fontsize{10.000000}{12.000000}\selectfont Memory (B)}%
\end{pgfscope}%
\begin{pgfscope}%
\pgfpathrectangle{\pgfqpoint{0.800000in}{0.528000in}}{\pgfqpoint{4.960000in}{3.696000in}}%
\pgfusepath{clip}%
\pgfsetrectcap%
\pgfsetroundjoin%
\pgfsetlinewidth{1.505625pt}%
\definecolor{currentstroke}{rgb}{0.121569,0.466667,0.705882}%
\pgfsetstrokecolor{currentstroke}%
\pgfsetdash{}{0pt}%
\pgfpathmoveto{\pgfqpoint{1.025455in}{1.507666in}}%
\pgfpathlineto{\pgfqpoint{1.071466in}{1.300720in}}%
\pgfpathlineto{\pgfqpoint{1.117477in}{1.551074in}}%
\pgfpathlineto{\pgfqpoint{1.163488in}{1.615408in}}%
\pgfpathlineto{\pgfqpoint{1.209499in}{1.887782in}}%
\pgfpathlineto{\pgfqpoint{1.255510in}{1.964764in}}%
\pgfpathlineto{\pgfqpoint{1.301521in}{2.038148in}}%
\pgfpathlineto{\pgfqpoint{1.347532in}{2.138880in}}%
\pgfpathlineto{\pgfqpoint{1.393544in}{2.228727in}}%
\pgfpathlineto{\pgfqpoint{1.439555in}{2.307238in}}%
\pgfpathlineto{\pgfqpoint{1.485566in}{2.405402in}}%
\pgfpathlineto{\pgfqpoint{1.531577in}{2.447223in}}%
\pgfpathlineto{\pgfqpoint{1.577588in}{2.508440in}}%
\pgfpathlineto{\pgfqpoint{1.623599in}{2.563769in}}%
\pgfpathlineto{\pgfqpoint{1.669610in}{2.616128in}}%
\pgfpathlineto{\pgfqpoint{1.715622in}{2.665135in}}%
\pgfpathlineto{\pgfqpoint{1.761633in}{2.711640in}}%
\pgfpathlineto{\pgfqpoint{1.807644in}{2.755059in}}%
\pgfpathlineto{\pgfqpoint{1.853655in}{2.795748in}}%
\pgfpathlineto{\pgfqpoint{1.899666in}{2.834573in}}%
\pgfpathlineto{\pgfqpoint{1.945677in}{2.885624in}}%
\pgfpathlineto{\pgfqpoint{1.991688in}{2.908418in}}%
\pgfpathlineto{\pgfqpoint{2.037699in}{2.941880in}}%
\pgfpathlineto{\pgfqpoint{2.083711in}{2.974258in}}%
\pgfpathlineto{\pgfqpoint{2.129722in}{3.005627in}}%
\pgfpathlineto{\pgfqpoint{2.175733in}{3.035469in}}%
\pgfpathlineto{\pgfqpoint{2.221744in}{3.064191in}}%
\pgfpathlineto{\pgfqpoint{2.267755in}{3.091467in}}%
\pgfpathlineto{\pgfqpoint{2.313766in}{3.118461in}}%
\pgfpathlineto{\pgfqpoint{2.359777in}{3.144028in}}%
\pgfpathlineto{\pgfqpoint{2.405788in}{3.169383in}}%
\pgfpathlineto{\pgfqpoint{2.451800in}{3.193212in}}%
\pgfpathlineto{\pgfqpoint{2.497811in}{3.216898in}}%
\pgfpathlineto{\pgfqpoint{2.543822in}{3.239650in}}%
\pgfpathlineto{\pgfqpoint{2.589833in}{3.261636in}}%
\pgfpathlineto{\pgfqpoint{2.635844in}{3.283221in}}%
\pgfpathlineto{\pgfqpoint{2.681855in}{3.304007in}}%
\pgfpathlineto{\pgfqpoint{2.727866in}{3.324245in}}%
\pgfpathlineto{\pgfqpoint{2.773878in}{3.344058in}}%
\pgfpathlineto{\pgfqpoint{2.819889in}{3.363277in}}%
\pgfpathlineto{\pgfqpoint{2.865900in}{3.389250in}}%
\pgfpathlineto{\pgfqpoint{2.911911in}{3.401279in}}%
\pgfpathlineto{\pgfqpoint{2.957922in}{3.419218in}}%
\pgfpathlineto{\pgfqpoint{3.003933in}{3.436660in}}%
\pgfpathlineto{\pgfqpoint{3.049944in}{3.453884in}}%
\pgfpathlineto{\pgfqpoint{3.095955in}{3.470396in}}%
\pgfpathlineto{\pgfqpoint{3.141967in}{3.486966in}}%
\pgfpathlineto{\pgfqpoint{3.187978in}{3.502867in}}%
\pgfpathlineto{\pgfqpoint{3.233989in}{3.518834in}}%
\pgfpathlineto{\pgfqpoint{3.280000in}{3.534170in}}%
\pgfpathlineto{\pgfqpoint{3.326011in}{3.549129in}}%
\pgfpathlineto{\pgfqpoint{3.372022in}{3.564096in}}%
\pgfpathlineto{\pgfqpoint{3.418033in}{3.578631in}}%
\pgfpathlineto{\pgfqpoint{3.464045in}{3.592896in}}%
\pgfpathlineto{\pgfqpoint{3.510056in}{3.606761in}}%
\pgfpathlineto{\pgfqpoint{3.556067in}{3.620448in}}%
\pgfpathlineto{\pgfqpoint{3.602078in}{3.634229in}}%
\pgfpathlineto{\pgfqpoint{3.648089in}{3.647372in}}%
\pgfpathlineto{\pgfqpoint{3.694100in}{3.660487in}}%
\pgfpathlineto{\pgfqpoint{3.740111in}{3.673127in}}%
\pgfpathlineto{\pgfqpoint{3.786122in}{3.685812in}}%
\pgfpathlineto{\pgfqpoint{3.832134in}{3.698351in}}%
\pgfpathlineto{\pgfqpoint{3.878145in}{3.710568in}}%
\pgfpathlineto{\pgfqpoint{3.924156in}{3.722413in}}%
\pgfpathlineto{\pgfqpoint{3.970167in}{3.734431in}}%
\pgfpathlineto{\pgfqpoint{4.016178in}{3.746032in}}%
\pgfpathlineto{\pgfqpoint{4.062189in}{3.757573in}}%
\pgfpathlineto{\pgfqpoint{4.108200in}{3.768831in}}%
\pgfpathlineto{\pgfqpoint{4.154212in}{3.779926in}}%
\pgfpathlineto{\pgfqpoint{4.200223in}{3.790916in}}%
\pgfpathlineto{\pgfqpoint{4.246234in}{3.801698in}}%
\pgfpathlineto{\pgfqpoint{4.292245in}{3.812489in}}%
\pgfpathlineto{\pgfqpoint{4.338256in}{3.823130in}}%
\pgfpathlineto{\pgfqpoint{4.384267in}{3.833367in}}%
\pgfpathlineto{\pgfqpoint{4.430278in}{3.843672in}}%
\pgfpathlineto{\pgfqpoint{4.476289in}{3.853690in}}%
\pgfpathlineto{\pgfqpoint{4.522301in}{3.863827in}}%
\pgfpathlineto{\pgfqpoint{4.568312in}{3.873538in}}%
\pgfpathlineto{\pgfqpoint{4.614323in}{3.883272in}}%
\pgfpathlineto{\pgfqpoint{4.660334in}{3.892836in}}%
\pgfpathlineto{\pgfqpoint{4.706345in}{3.902423in}}%
\pgfpathlineto{\pgfqpoint{4.752356in}{3.915286in}}%
\pgfpathlineto{\pgfqpoint{4.798367in}{3.921407in}}%
\pgfpathlineto{\pgfqpoint{4.844378in}{3.930509in}}%
\pgfpathlineto{\pgfqpoint{4.890390in}{3.939594in}}%
\pgfpathlineto{\pgfqpoint{4.936401in}{3.948483in}}%
\pgfpathlineto{\pgfqpoint{4.982412in}{3.957227in}}%
\pgfpathlineto{\pgfqpoint{5.028423in}{3.966003in}}%
\pgfpathlineto{\pgfqpoint{5.074434in}{3.974721in}}%
\pgfpathlineto{\pgfqpoint{5.120445in}{3.983215in}}%
\pgfpathlineto{\pgfqpoint{5.166456in}{3.991657in}}%
\pgfpathlineto{\pgfqpoint{5.212468in}{3.999882in}}%
\pgfpathlineto{\pgfqpoint{5.258479in}{4.008308in}}%
\pgfpathlineto{\pgfqpoint{5.304490in}{4.016398in}}%
\pgfpathlineto{\pgfqpoint{5.350501in}{4.024524in}}%
\pgfpathlineto{\pgfqpoint{5.396512in}{4.032406in}}%
\pgfpathlineto{\pgfqpoint{5.442523in}{4.040404in}}%
\pgfpathlineto{\pgfqpoint{5.488534in}{4.048281in}}%
\pgfpathlineto{\pgfqpoint{5.534545in}{4.056000in}}%
\pgfusepath{stroke}%
\end{pgfscope}%
\begin{pgfscope}%
\pgfpathrectangle{\pgfqpoint{0.800000in}{0.528000in}}{\pgfqpoint{4.960000in}{3.696000in}}%
\pgfusepath{clip}%
\pgfsetrectcap%
\pgfsetroundjoin%
\pgfsetlinewidth{1.505625pt}%
\definecolor{currentstroke}{rgb}{1.000000,0.498039,0.054902}%
\pgfsetstrokecolor{currentstroke}%
\pgfsetdash{}{0pt}%
\pgfpathmoveto{\pgfqpoint{1.025455in}{1.422372in}}%
\pgfpathlineto{\pgfqpoint{1.071466in}{0.741232in}}%
\pgfpathlineto{\pgfqpoint{1.117477in}{0.894023in}}%
\pgfpathlineto{\pgfqpoint{1.163488in}{0.696000in}}%
\pgfpathlineto{\pgfqpoint{1.209499in}{1.101989in}}%
\pgfpathlineto{\pgfqpoint{1.255510in}{1.106773in}}%
\pgfpathlineto{\pgfqpoint{1.301521in}{1.086524in}}%
\pgfpathlineto{\pgfqpoint{1.347532in}{1.174533in}}%
\pgfpathlineto{\pgfqpoint{1.393544in}{1.253461in}}%
\pgfpathlineto{\pgfqpoint{1.439555in}{1.325008in}}%
\pgfpathlineto{\pgfqpoint{1.485566in}{1.463361in}}%
\pgfpathlineto{\pgfqpoint{1.531577in}{1.450713in}}%
\pgfpathlineto{\pgfqpoint{1.577588in}{1.506589in}}%
\pgfpathlineto{\pgfqpoint{1.623599in}{1.558663in}}%
\pgfpathlineto{\pgfqpoint{1.669610in}{1.607420in}}%
\pgfpathlineto{\pgfqpoint{1.715622in}{1.653257in}}%
\pgfpathlineto{\pgfqpoint{1.761633in}{1.696505in}}%
\pgfpathlineto{\pgfqpoint{1.807644in}{1.737440in}}%
\pgfpathlineto{\pgfqpoint{1.853655in}{1.776297in}}%
\pgfpathlineto{\pgfqpoint{1.899666in}{1.813277in}}%
\pgfpathlineto{\pgfqpoint{1.945677in}{1.889500in}}%
\pgfpathlineto{\pgfqpoint{1.991688in}{1.882276in}}%
\pgfpathlineto{\pgfqpoint{2.037699in}{1.914576in}}%
\pgfpathlineto{\pgfqpoint{2.083711in}{1.945568in}}%
\pgfpathlineto{\pgfqpoint{2.129722in}{1.975355in}}%
\pgfpathlineto{\pgfqpoint{2.175733in}{2.004026in}}%
\pgfpathlineto{\pgfqpoint{2.221744in}{2.031662in}}%
\pgfpathlineto{\pgfqpoint{2.267755in}{2.058335in}}%
\pgfpathlineto{\pgfqpoint{2.313766in}{2.084110in}}%
\pgfpathlineto{\pgfqpoint{2.359777in}{2.109046in}}%
\pgfpathlineto{\pgfqpoint{2.405788in}{2.133195in}}%
\pgfpathlineto{\pgfqpoint{2.451800in}{2.156606in}}%
\pgfpathlineto{\pgfqpoint{2.497811in}{2.179323in}}%
\pgfpathlineto{\pgfqpoint{2.543822in}{2.201385in}}%
\pgfpathlineto{\pgfqpoint{2.589833in}{2.222829in}}%
\pgfpathlineto{\pgfqpoint{2.635844in}{2.243688in}}%
\pgfpathlineto{\pgfqpoint{2.681855in}{2.263995in}}%
\pgfpathlineto{\pgfqpoint{2.727866in}{2.283777in}}%
\pgfpathlineto{\pgfqpoint{2.773878in}{2.303060in}}%
\pgfpathlineto{\pgfqpoint{2.819889in}{2.321870in}}%
\pgfpathlineto{\pgfqpoint{2.865900in}{2.362047in}}%
\pgfpathlineto{\pgfqpoint{2.911911in}{2.358158in}}%
\pgfpathlineto{\pgfqpoint{2.957922in}{2.375677in}}%
\pgfpathlineto{\pgfqpoint{3.003933in}{2.392804in}}%
\pgfpathlineto{\pgfqpoint{3.049944in}{2.409556in}}%
\pgfpathlineto{\pgfqpoint{3.095955in}{2.425949in}}%
\pgfpathlineto{\pgfqpoint{3.141967in}{2.441999in}}%
\pgfpathlineto{\pgfqpoint{3.187978in}{2.457720in}}%
\pgfpathlineto{\pgfqpoint{3.233989in}{2.473124in}}%
\pgfpathlineto{\pgfqpoint{3.280000in}{2.488224in}}%
\pgfpathlineto{\pgfqpoint{3.326011in}{2.503032in}}%
\pgfpathlineto{\pgfqpoint{3.372022in}{2.517560in}}%
\pgfpathlineto{\pgfqpoint{3.418033in}{2.531817in}}%
\pgfpathlineto{\pgfqpoint{3.464045in}{2.545813in}}%
\pgfpathlineto{\pgfqpoint{3.510056in}{2.559558in}}%
\pgfpathlineto{\pgfqpoint{3.556067in}{2.573060in}}%
\pgfpathlineto{\pgfqpoint{3.602078in}{2.586329in}}%
\pgfpathlineto{\pgfqpoint{3.648089in}{2.599372in}}%
\pgfpathlineto{\pgfqpoint{3.694100in}{2.612196in}}%
\pgfpathlineto{\pgfqpoint{3.740111in}{2.624809in}}%
\pgfpathlineto{\pgfqpoint{3.786122in}{2.637217in}}%
\pgfpathlineto{\pgfqpoint{3.832134in}{2.649428in}}%
\pgfpathlineto{\pgfqpoint{3.878145in}{2.661447in}}%
\pgfpathlineto{\pgfqpoint{3.924156in}{2.673280in}}%
\pgfpathlineto{\pgfqpoint{3.970167in}{2.684933in}}%
\pgfpathlineto{\pgfqpoint{4.016178in}{2.696412in}}%
\pgfpathlineto{\pgfqpoint{4.062189in}{2.707721in}}%
\pgfpathlineto{\pgfqpoint{4.108200in}{2.718865in}}%
\pgfpathlineto{\pgfqpoint{4.154212in}{2.729849in}}%
\pgfpathlineto{\pgfqpoint{4.200223in}{2.740679in}}%
\pgfpathlineto{\pgfqpoint{4.246234in}{2.751357in}}%
\pgfpathlineto{\pgfqpoint{4.292245in}{2.761888in}}%
\pgfpathlineto{\pgfqpoint{4.338256in}{2.772276in}}%
\pgfpathlineto{\pgfqpoint{4.384267in}{2.782525in}}%
\pgfpathlineto{\pgfqpoint{4.430278in}{2.792639in}}%
\pgfpathlineto{\pgfqpoint{4.476289in}{2.802621in}}%
\pgfpathlineto{\pgfqpoint{4.522301in}{2.812475in}}%
\pgfpathlineto{\pgfqpoint{4.568312in}{2.822203in}}%
\pgfpathlineto{\pgfqpoint{4.614323in}{2.831810in}}%
\pgfpathlineto{\pgfqpoint{4.660334in}{2.841297in}}%
\pgfpathlineto{\pgfqpoint{4.706345in}{2.850668in}}%
\pgfpathlineto{\pgfqpoint{4.752356in}{2.871071in}}%
\pgfpathlineto{\pgfqpoint{4.798367in}{2.869074in}}%
\pgfpathlineto{\pgfqpoint{4.844378in}{2.878113in}}%
\pgfpathlineto{\pgfqpoint{4.890390in}{2.887047in}}%
\pgfpathlineto{\pgfqpoint{4.936401in}{2.895878in}}%
\pgfpathlineto{\pgfqpoint{4.982412in}{2.904608in}}%
\pgfpathlineto{\pgfqpoint{5.028423in}{2.913239in}}%
\pgfpathlineto{\pgfqpoint{5.074434in}{2.921775in}}%
\pgfpathlineto{\pgfqpoint{5.120445in}{2.930216in}}%
\pgfpathlineto{\pgfqpoint{5.166456in}{2.938566in}}%
\pgfpathlineto{\pgfqpoint{5.212468in}{2.946825in}}%
\pgfpathlineto{\pgfqpoint{5.258479in}{2.954996in}}%
\pgfpathlineto{\pgfqpoint{5.304490in}{2.963081in}}%
\pgfpathlineto{\pgfqpoint{5.350501in}{2.971082in}}%
\pgfpathlineto{\pgfqpoint{5.396512in}{2.978999in}}%
\pgfpathlineto{\pgfqpoint{5.442523in}{2.986836in}}%
\pgfpathlineto{\pgfqpoint{5.488534in}{2.994593in}}%
\pgfpathlineto{\pgfqpoint{5.534545in}{3.002273in}}%
\pgfusepath{stroke}%
\end{pgfscope}%
\begin{pgfscope}%
\pgfsetrectcap%
\pgfsetmiterjoin%
\pgfsetlinewidth{0.803000pt}%
\definecolor{currentstroke}{rgb}{0.000000,0.000000,0.000000}%
\pgfsetstrokecolor{currentstroke}%
\pgfsetdash{}{0pt}%
\pgfpathmoveto{\pgfqpoint{0.800000in}{0.528000in}}%
\pgfpathlineto{\pgfqpoint{0.800000in}{4.224000in}}%
\pgfusepath{stroke}%
\end{pgfscope}%
\begin{pgfscope}%
\pgfsetrectcap%
\pgfsetmiterjoin%
\pgfsetlinewidth{0.803000pt}%
\definecolor{currentstroke}{rgb}{0.000000,0.000000,0.000000}%
\pgfsetstrokecolor{currentstroke}%
\pgfsetdash{}{0pt}%
\pgfpathmoveto{\pgfqpoint{5.760000in}{0.528000in}}%
\pgfpathlineto{\pgfqpoint{5.760000in}{4.224000in}}%
\pgfusepath{stroke}%
\end{pgfscope}%
\begin{pgfscope}%
\pgfsetrectcap%
\pgfsetmiterjoin%
\pgfsetlinewidth{0.803000pt}%
\definecolor{currentstroke}{rgb}{0.000000,0.000000,0.000000}%
\pgfsetstrokecolor{currentstroke}%
\pgfsetdash{}{0pt}%
\pgfpathmoveto{\pgfqpoint{0.800000in}{0.528000in}}%
\pgfpathlineto{\pgfqpoint{5.760000in}{0.528000in}}%
\pgfusepath{stroke}%
\end{pgfscope}%
\begin{pgfscope}%
\pgfsetrectcap%
\pgfsetmiterjoin%
\pgfsetlinewidth{0.803000pt}%
\definecolor{currentstroke}{rgb}{0.000000,0.000000,0.000000}%
\pgfsetstrokecolor{currentstroke}%
\pgfsetdash{}{0pt}%
\pgfpathmoveto{\pgfqpoint{0.800000in}{4.224000in}}%
\pgfpathlineto{\pgfqpoint{5.760000in}{4.224000in}}%
\pgfusepath{stroke}%
\end{pgfscope}%
\begin{pgfscope}%
\pgfsetbuttcap%
\pgfsetmiterjoin%
\definecolor{currentfill}{rgb}{1.000000,1.000000,1.000000}%
\pgfsetfillcolor{currentfill}%
\pgfsetfillopacity{0.800000}%
\pgfsetlinewidth{1.003750pt}%
\definecolor{currentstroke}{rgb}{0.800000,0.800000,0.800000}%
\pgfsetstrokecolor{currentstroke}%
\pgfsetstrokeopacity{0.800000}%
\pgfsetdash{}{0pt}%
\pgfpathmoveto{\pgfqpoint{0.897222in}{3.725543in}}%
\pgfpathlineto{\pgfqpoint{2.014507in}{3.725543in}}%
\pgfpathquadraticcurveto{\pgfqpoint{2.042285in}{3.725543in}}{\pgfqpoint{2.042285in}{3.753321in}}%
\pgfpathlineto{\pgfqpoint{2.042285in}{4.126778in}}%
\pgfpathquadraticcurveto{\pgfqpoint{2.042285in}{4.154556in}}{\pgfqpoint{2.014507in}{4.154556in}}%
\pgfpathlineto{\pgfqpoint{0.897222in}{4.154556in}}%
\pgfpathquadraticcurveto{\pgfqpoint{0.869444in}{4.154556in}}{\pgfqpoint{0.869444in}{4.126778in}}%
\pgfpathlineto{\pgfqpoint{0.869444in}{3.753321in}}%
\pgfpathquadraticcurveto{\pgfqpoint{0.869444in}{3.725543in}}{\pgfqpoint{0.897222in}{3.725543in}}%
\pgfpathlineto{\pgfqpoint{0.897222in}{3.725543in}}%
\pgfpathclose%
\pgfusepath{stroke,fill}%
\end{pgfscope}%
\begin{pgfscope}%
\pgfsetrectcap%
\pgfsetroundjoin%
\pgfsetlinewidth{1.505625pt}%
\definecolor{currentstroke}{rgb}{0.121569,0.466667,0.705882}%
\pgfsetstrokecolor{currentstroke}%
\pgfsetdash{}{0pt}%
\pgfpathmoveto{\pgfqpoint{0.925000in}{4.050389in}}%
\pgfpathlineto{\pgfqpoint{1.063889in}{4.050389in}}%
\pgfpathlineto{\pgfqpoint{1.202778in}{4.050389in}}%
\pgfusepath{stroke}%
\end{pgfscope}%
\begin{pgfscope}%
\definecolor{textcolor}{rgb}{0.000000,0.000000,0.000000}%
\pgfsetstrokecolor{textcolor}%
\pgfsetfillcolor{textcolor}%
\pgftext[x=1.313889in,y=4.001778in,left,base]{\color{textcolor}\rmfamily\fontsize{10.000000}{12.000000}\selectfont mergesort}%
\end{pgfscope}%
\begin{pgfscope}%
\pgfsetrectcap%
\pgfsetroundjoin%
\pgfsetlinewidth{1.505625pt}%
\definecolor{currentstroke}{rgb}{1.000000,0.498039,0.054902}%
\pgfsetstrokecolor{currentstroke}%
\pgfsetdash{}{0pt}%
\pgfpathmoveto{\pgfqpoint{0.925000in}{3.856716in}}%
\pgfpathlineto{\pgfqpoint{1.063889in}{3.856716in}}%
\pgfpathlineto{\pgfqpoint{1.202778in}{3.856716in}}%
\pgfusepath{stroke}%
\end{pgfscope}%
\begin{pgfscope}%
\definecolor{textcolor}{rgb}{0.000000,0.000000,0.000000}%
\pgfsetstrokecolor{textcolor}%
\pgfsetfillcolor{textcolor}%
\pgftext[x=1.313889in,y=3.808105in,left,base]{\color{textcolor}\rmfamily\fontsize{10.000000}{12.000000}\selectfont bmergesort}%
\end{pgfscope}%
\end{pgfpicture}%
\makeatother%
\endgroup%

\subsection{Analysis}
Since we are practically using only half the memory we see a huge
\textbf{decrease}. The other metrics are very similar. To improve the time
we can use \textbf{timsort} which runs \textit{Insertion sort} upto a point
and switches to merge sort.
\subsubsection{Time}
%% Creator: Matplotlib, PGF backend
%%
%% To include the figure in your LaTeX document, write
%%   \input{<filename>.pgf}
%%
%% Make sure the required packages are loaded in your preamble
%%   \usepackage{pgf}
%%
%% Also ensure that all the required font packages are loaded; for instance,
%% the lmodern package is sometimes necessary when using math font.
%%   \usepackage{lmodern}
%%
%% Figures using additional raster images can only be included by \input if
%% they are in the same directory as the main LaTeX file. For loading figures
%% from other directories you can use the `import` package
%%   \usepackage{import}
%%
%% and then include the figures with
%%   \import{<path to file>}{<filename>.pgf}
%%
%% Matplotlib used the following preamble
%%   
%%   \makeatletter\@ifpackageloaded{underscore}{}{\usepackage[strings]{underscore}}\makeatother
%%
\begingroup%
\makeatletter%
\begin{pgfpicture}%
\pgfpathrectangle{\pgfpointorigin}{\pgfqpoint{6.400000in}{4.800000in}}%
\pgfusepath{use as bounding box, clip}%
\begin{pgfscope}%
\pgfsetbuttcap%
\pgfsetmiterjoin%
\definecolor{currentfill}{rgb}{1.000000,1.000000,1.000000}%
\pgfsetfillcolor{currentfill}%
\pgfsetlinewidth{0.000000pt}%
\definecolor{currentstroke}{rgb}{1.000000,1.000000,1.000000}%
\pgfsetstrokecolor{currentstroke}%
\pgfsetdash{}{0pt}%
\pgfpathmoveto{\pgfqpoint{0.000000in}{0.000000in}}%
\pgfpathlineto{\pgfqpoint{6.400000in}{0.000000in}}%
\pgfpathlineto{\pgfqpoint{6.400000in}{4.800000in}}%
\pgfpathlineto{\pgfqpoint{0.000000in}{4.800000in}}%
\pgfpathlineto{\pgfqpoint{0.000000in}{0.000000in}}%
\pgfpathclose%
\pgfusepath{fill}%
\end{pgfscope}%
\begin{pgfscope}%
\pgfsetbuttcap%
\pgfsetmiterjoin%
\definecolor{currentfill}{rgb}{1.000000,1.000000,1.000000}%
\pgfsetfillcolor{currentfill}%
\pgfsetlinewidth{0.000000pt}%
\definecolor{currentstroke}{rgb}{0.000000,0.000000,0.000000}%
\pgfsetstrokecolor{currentstroke}%
\pgfsetstrokeopacity{0.000000}%
\pgfsetdash{}{0pt}%
\pgfpathmoveto{\pgfqpoint{0.800000in}{0.528000in}}%
\pgfpathlineto{\pgfqpoint{5.760000in}{0.528000in}}%
\pgfpathlineto{\pgfqpoint{5.760000in}{4.224000in}}%
\pgfpathlineto{\pgfqpoint{0.800000in}{4.224000in}}%
\pgfpathlineto{\pgfqpoint{0.800000in}{0.528000in}}%
\pgfpathclose%
\pgfusepath{fill}%
\end{pgfscope}%
\begin{pgfscope}%
\pgfsetbuttcap%
\pgfsetroundjoin%
\definecolor{currentfill}{rgb}{0.000000,0.000000,0.000000}%
\pgfsetfillcolor{currentfill}%
\pgfsetlinewidth{0.803000pt}%
\definecolor{currentstroke}{rgb}{0.000000,0.000000,0.000000}%
\pgfsetstrokecolor{currentstroke}%
\pgfsetdash{}{0pt}%
\pgfsys@defobject{currentmarker}{\pgfqpoint{0.000000in}{-0.048611in}}{\pgfqpoint{0.000000in}{0.000000in}}{%
\pgfpathmoveto{\pgfqpoint{0.000000in}{0.000000in}}%
\pgfpathlineto{\pgfqpoint{0.000000in}{-0.048611in}}%
\pgfusepath{stroke,fill}%
}%
\begin{pgfscope}%
\pgfsys@transformshift{0.979443in}{0.528000in}%
\pgfsys@useobject{currentmarker}{}%
\end{pgfscope}%
\end{pgfscope}%
\begin{pgfscope}%
\definecolor{textcolor}{rgb}{0.000000,0.000000,0.000000}%
\pgfsetstrokecolor{textcolor}%
\pgfsetfillcolor{textcolor}%
\pgftext[x=0.979443in,y=0.430778in,,top]{\color{textcolor}\rmfamily\fontsize{10.000000}{12.000000}\selectfont \(\displaystyle {0}\)}%
\end{pgfscope}%
\begin{pgfscope}%
\pgfsetbuttcap%
\pgfsetroundjoin%
\definecolor{currentfill}{rgb}{0.000000,0.000000,0.000000}%
\pgfsetfillcolor{currentfill}%
\pgfsetlinewidth{0.803000pt}%
\definecolor{currentstroke}{rgb}{0.000000,0.000000,0.000000}%
\pgfsetstrokecolor{currentstroke}%
\pgfsetdash{}{0pt}%
\pgfsys@defobject{currentmarker}{\pgfqpoint{0.000000in}{-0.048611in}}{\pgfqpoint{0.000000in}{0.000000in}}{%
\pgfpathmoveto{\pgfqpoint{0.000000in}{0.000000in}}%
\pgfpathlineto{\pgfqpoint{0.000000in}{-0.048611in}}%
\pgfusepath{stroke,fill}%
}%
\begin{pgfscope}%
\pgfsys@transformshift{1.899666in}{0.528000in}%
\pgfsys@useobject{currentmarker}{}%
\end{pgfscope}%
\end{pgfscope}%
\begin{pgfscope}%
\definecolor{textcolor}{rgb}{0.000000,0.000000,0.000000}%
\pgfsetstrokecolor{textcolor}%
\pgfsetfillcolor{textcolor}%
\pgftext[x=1.899666in,y=0.430778in,,top]{\color{textcolor}\rmfamily\fontsize{10.000000}{12.000000}\selectfont \(\displaystyle {2000}\)}%
\end{pgfscope}%
\begin{pgfscope}%
\pgfsetbuttcap%
\pgfsetroundjoin%
\definecolor{currentfill}{rgb}{0.000000,0.000000,0.000000}%
\pgfsetfillcolor{currentfill}%
\pgfsetlinewidth{0.803000pt}%
\definecolor{currentstroke}{rgb}{0.000000,0.000000,0.000000}%
\pgfsetstrokecolor{currentstroke}%
\pgfsetdash{}{0pt}%
\pgfsys@defobject{currentmarker}{\pgfqpoint{0.000000in}{-0.048611in}}{\pgfqpoint{0.000000in}{0.000000in}}{%
\pgfpathmoveto{\pgfqpoint{0.000000in}{0.000000in}}%
\pgfpathlineto{\pgfqpoint{0.000000in}{-0.048611in}}%
\pgfusepath{stroke,fill}%
}%
\begin{pgfscope}%
\pgfsys@transformshift{2.819889in}{0.528000in}%
\pgfsys@useobject{currentmarker}{}%
\end{pgfscope}%
\end{pgfscope}%
\begin{pgfscope}%
\definecolor{textcolor}{rgb}{0.000000,0.000000,0.000000}%
\pgfsetstrokecolor{textcolor}%
\pgfsetfillcolor{textcolor}%
\pgftext[x=2.819889in,y=0.430778in,,top]{\color{textcolor}\rmfamily\fontsize{10.000000}{12.000000}\selectfont \(\displaystyle {4000}\)}%
\end{pgfscope}%
\begin{pgfscope}%
\pgfsetbuttcap%
\pgfsetroundjoin%
\definecolor{currentfill}{rgb}{0.000000,0.000000,0.000000}%
\pgfsetfillcolor{currentfill}%
\pgfsetlinewidth{0.803000pt}%
\definecolor{currentstroke}{rgb}{0.000000,0.000000,0.000000}%
\pgfsetstrokecolor{currentstroke}%
\pgfsetdash{}{0pt}%
\pgfsys@defobject{currentmarker}{\pgfqpoint{0.000000in}{-0.048611in}}{\pgfqpoint{0.000000in}{0.000000in}}{%
\pgfpathmoveto{\pgfqpoint{0.000000in}{0.000000in}}%
\pgfpathlineto{\pgfqpoint{0.000000in}{-0.048611in}}%
\pgfusepath{stroke,fill}%
}%
\begin{pgfscope}%
\pgfsys@transformshift{3.740111in}{0.528000in}%
\pgfsys@useobject{currentmarker}{}%
\end{pgfscope}%
\end{pgfscope}%
\begin{pgfscope}%
\definecolor{textcolor}{rgb}{0.000000,0.000000,0.000000}%
\pgfsetstrokecolor{textcolor}%
\pgfsetfillcolor{textcolor}%
\pgftext[x=3.740111in,y=0.430778in,,top]{\color{textcolor}\rmfamily\fontsize{10.000000}{12.000000}\selectfont \(\displaystyle {6000}\)}%
\end{pgfscope}%
\begin{pgfscope}%
\pgfsetbuttcap%
\pgfsetroundjoin%
\definecolor{currentfill}{rgb}{0.000000,0.000000,0.000000}%
\pgfsetfillcolor{currentfill}%
\pgfsetlinewidth{0.803000pt}%
\definecolor{currentstroke}{rgb}{0.000000,0.000000,0.000000}%
\pgfsetstrokecolor{currentstroke}%
\pgfsetdash{}{0pt}%
\pgfsys@defobject{currentmarker}{\pgfqpoint{0.000000in}{-0.048611in}}{\pgfqpoint{0.000000in}{0.000000in}}{%
\pgfpathmoveto{\pgfqpoint{0.000000in}{0.000000in}}%
\pgfpathlineto{\pgfqpoint{0.000000in}{-0.048611in}}%
\pgfusepath{stroke,fill}%
}%
\begin{pgfscope}%
\pgfsys@transformshift{4.660334in}{0.528000in}%
\pgfsys@useobject{currentmarker}{}%
\end{pgfscope}%
\end{pgfscope}%
\begin{pgfscope}%
\definecolor{textcolor}{rgb}{0.000000,0.000000,0.000000}%
\pgfsetstrokecolor{textcolor}%
\pgfsetfillcolor{textcolor}%
\pgftext[x=4.660334in,y=0.430778in,,top]{\color{textcolor}\rmfamily\fontsize{10.000000}{12.000000}\selectfont \(\displaystyle {8000}\)}%
\end{pgfscope}%
\begin{pgfscope}%
\pgfsetbuttcap%
\pgfsetroundjoin%
\definecolor{currentfill}{rgb}{0.000000,0.000000,0.000000}%
\pgfsetfillcolor{currentfill}%
\pgfsetlinewidth{0.803000pt}%
\definecolor{currentstroke}{rgb}{0.000000,0.000000,0.000000}%
\pgfsetstrokecolor{currentstroke}%
\pgfsetdash{}{0pt}%
\pgfsys@defobject{currentmarker}{\pgfqpoint{0.000000in}{-0.048611in}}{\pgfqpoint{0.000000in}{0.000000in}}{%
\pgfpathmoveto{\pgfqpoint{0.000000in}{0.000000in}}%
\pgfpathlineto{\pgfqpoint{0.000000in}{-0.048611in}}%
\pgfusepath{stroke,fill}%
}%
\begin{pgfscope}%
\pgfsys@transformshift{5.580557in}{0.528000in}%
\pgfsys@useobject{currentmarker}{}%
\end{pgfscope}%
\end{pgfscope}%
\begin{pgfscope}%
\definecolor{textcolor}{rgb}{0.000000,0.000000,0.000000}%
\pgfsetstrokecolor{textcolor}%
\pgfsetfillcolor{textcolor}%
\pgftext[x=5.580557in,y=0.430778in,,top]{\color{textcolor}\rmfamily\fontsize{10.000000}{12.000000}\selectfont \(\displaystyle {10000}\)}%
\end{pgfscope}%
\begin{pgfscope}%
\definecolor{textcolor}{rgb}{0.000000,0.000000,0.000000}%
\pgfsetstrokecolor{textcolor}%
\pgfsetfillcolor{textcolor}%
\pgftext[x=3.280000in,y=0.251766in,,top]{\color{textcolor}\rmfamily\fontsize{10.000000}{12.000000}\selectfont Input Size}%
\end{pgfscope}%
\begin{pgfscope}%
\pgfsetbuttcap%
\pgfsetroundjoin%
\definecolor{currentfill}{rgb}{0.000000,0.000000,0.000000}%
\pgfsetfillcolor{currentfill}%
\pgfsetlinewidth{0.803000pt}%
\definecolor{currentstroke}{rgb}{0.000000,0.000000,0.000000}%
\pgfsetstrokecolor{currentstroke}%
\pgfsetdash{}{0pt}%
\pgfsys@defobject{currentmarker}{\pgfqpoint{-0.048611in}{0.000000in}}{\pgfqpoint{-0.000000in}{0.000000in}}{%
\pgfpathmoveto{\pgfqpoint{-0.000000in}{0.000000in}}%
\pgfpathlineto{\pgfqpoint{-0.048611in}{0.000000in}}%
\pgfusepath{stroke,fill}%
}%
\begin{pgfscope}%
\pgfsys@transformshift{0.800000in}{1.104461in}%
\pgfsys@useobject{currentmarker}{}%
\end{pgfscope}%
\end{pgfscope}%
\begin{pgfscope}%
\definecolor{textcolor}{rgb}{0.000000,0.000000,0.000000}%
\pgfsetstrokecolor{textcolor}%
\pgfsetfillcolor{textcolor}%
\pgftext[x=0.501581in, y=1.056236in, left, base]{\color{textcolor}\rmfamily\fontsize{10.000000}{12.000000}\selectfont \(\displaystyle {10^{6}}\)}%
\end{pgfscope}%
\begin{pgfscope}%
\pgfsetbuttcap%
\pgfsetroundjoin%
\definecolor{currentfill}{rgb}{0.000000,0.000000,0.000000}%
\pgfsetfillcolor{currentfill}%
\pgfsetlinewidth{0.803000pt}%
\definecolor{currentstroke}{rgb}{0.000000,0.000000,0.000000}%
\pgfsetstrokecolor{currentstroke}%
\pgfsetdash{}{0pt}%
\pgfsys@defobject{currentmarker}{\pgfqpoint{-0.048611in}{0.000000in}}{\pgfqpoint{-0.000000in}{0.000000in}}{%
\pgfpathmoveto{\pgfqpoint{-0.000000in}{0.000000in}}%
\pgfpathlineto{\pgfqpoint{-0.048611in}{0.000000in}}%
\pgfusepath{stroke,fill}%
}%
\begin{pgfscope}%
\pgfsys@transformshift{0.800000in}{2.367849in}%
\pgfsys@useobject{currentmarker}{}%
\end{pgfscope}%
\end{pgfscope}%
\begin{pgfscope}%
\definecolor{textcolor}{rgb}{0.000000,0.000000,0.000000}%
\pgfsetstrokecolor{textcolor}%
\pgfsetfillcolor{textcolor}%
\pgftext[x=0.501581in, y=2.319624in, left, base]{\color{textcolor}\rmfamily\fontsize{10.000000}{12.000000}\selectfont \(\displaystyle {10^{7}}\)}%
\end{pgfscope}%
\begin{pgfscope}%
\pgfsetbuttcap%
\pgfsetroundjoin%
\definecolor{currentfill}{rgb}{0.000000,0.000000,0.000000}%
\pgfsetfillcolor{currentfill}%
\pgfsetlinewidth{0.803000pt}%
\definecolor{currentstroke}{rgb}{0.000000,0.000000,0.000000}%
\pgfsetstrokecolor{currentstroke}%
\pgfsetdash{}{0pt}%
\pgfsys@defobject{currentmarker}{\pgfqpoint{-0.048611in}{0.000000in}}{\pgfqpoint{-0.000000in}{0.000000in}}{%
\pgfpathmoveto{\pgfqpoint{-0.000000in}{0.000000in}}%
\pgfpathlineto{\pgfqpoint{-0.048611in}{0.000000in}}%
\pgfusepath{stroke,fill}%
}%
\begin{pgfscope}%
\pgfsys@transformshift{0.800000in}{3.631237in}%
\pgfsys@useobject{currentmarker}{}%
\end{pgfscope}%
\end{pgfscope}%
\begin{pgfscope}%
\definecolor{textcolor}{rgb}{0.000000,0.000000,0.000000}%
\pgfsetstrokecolor{textcolor}%
\pgfsetfillcolor{textcolor}%
\pgftext[x=0.501581in, y=3.583012in, left, base]{\color{textcolor}\rmfamily\fontsize{10.000000}{12.000000}\selectfont \(\displaystyle {10^{8}}\)}%
\end{pgfscope}%
\begin{pgfscope}%
\pgfsetbuttcap%
\pgfsetroundjoin%
\definecolor{currentfill}{rgb}{0.000000,0.000000,0.000000}%
\pgfsetfillcolor{currentfill}%
\pgfsetlinewidth{0.602250pt}%
\definecolor{currentstroke}{rgb}{0.000000,0.000000,0.000000}%
\pgfsetstrokecolor{currentstroke}%
\pgfsetdash{}{0pt}%
\pgfsys@defobject{currentmarker}{\pgfqpoint{-0.027778in}{0.000000in}}{\pgfqpoint{-0.000000in}{0.000000in}}{%
\pgfpathmoveto{\pgfqpoint{-0.000000in}{0.000000in}}%
\pgfpathlineto{\pgfqpoint{-0.027778in}{0.000000in}}%
\pgfusepath{stroke,fill}%
}%
\begin{pgfscope}%
\pgfsys@transformshift{0.800000in}{0.601709in}%
\pgfsys@useobject{currentmarker}{}%
\end{pgfscope}%
\end{pgfscope}%
\begin{pgfscope}%
\pgfsetbuttcap%
\pgfsetroundjoin%
\definecolor{currentfill}{rgb}{0.000000,0.000000,0.000000}%
\pgfsetfillcolor{currentfill}%
\pgfsetlinewidth{0.602250pt}%
\definecolor{currentstroke}{rgb}{0.000000,0.000000,0.000000}%
\pgfsetstrokecolor{currentstroke}%
\pgfsetdash{}{0pt}%
\pgfsys@defobject{currentmarker}{\pgfqpoint{-0.027778in}{0.000000in}}{\pgfqpoint{-0.000000in}{0.000000in}}{%
\pgfpathmoveto{\pgfqpoint{-0.000000in}{0.000000in}}%
\pgfpathlineto{\pgfqpoint{-0.027778in}{0.000000in}}%
\pgfusepath{stroke,fill}%
}%
\begin{pgfscope}%
\pgfsys@transformshift{0.800000in}{0.724144in}%
\pgfsys@useobject{currentmarker}{}%
\end{pgfscope}%
\end{pgfscope}%
\begin{pgfscope}%
\pgfsetbuttcap%
\pgfsetroundjoin%
\definecolor{currentfill}{rgb}{0.000000,0.000000,0.000000}%
\pgfsetfillcolor{currentfill}%
\pgfsetlinewidth{0.602250pt}%
\definecolor{currentstroke}{rgb}{0.000000,0.000000,0.000000}%
\pgfsetstrokecolor{currentstroke}%
\pgfsetdash{}{0pt}%
\pgfsys@defobject{currentmarker}{\pgfqpoint{-0.027778in}{0.000000in}}{\pgfqpoint{-0.000000in}{0.000000in}}{%
\pgfpathmoveto{\pgfqpoint{-0.000000in}{0.000000in}}%
\pgfpathlineto{\pgfqpoint{-0.027778in}{0.000000in}}%
\pgfusepath{stroke,fill}%
}%
\begin{pgfscope}%
\pgfsys@transformshift{0.800000in}{0.824180in}%
\pgfsys@useobject{currentmarker}{}%
\end{pgfscope}%
\end{pgfscope}%
\begin{pgfscope}%
\pgfsetbuttcap%
\pgfsetroundjoin%
\definecolor{currentfill}{rgb}{0.000000,0.000000,0.000000}%
\pgfsetfillcolor{currentfill}%
\pgfsetlinewidth{0.602250pt}%
\definecolor{currentstroke}{rgb}{0.000000,0.000000,0.000000}%
\pgfsetstrokecolor{currentstroke}%
\pgfsetdash{}{0pt}%
\pgfsys@defobject{currentmarker}{\pgfqpoint{-0.027778in}{0.000000in}}{\pgfqpoint{-0.000000in}{0.000000in}}{%
\pgfpathmoveto{\pgfqpoint{-0.000000in}{0.000000in}}%
\pgfpathlineto{\pgfqpoint{-0.027778in}{0.000000in}}%
\pgfusepath{stroke,fill}%
}%
\begin{pgfscope}%
\pgfsys@transformshift{0.800000in}{0.908760in}%
\pgfsys@useobject{currentmarker}{}%
\end{pgfscope}%
\end{pgfscope}%
\begin{pgfscope}%
\pgfsetbuttcap%
\pgfsetroundjoin%
\definecolor{currentfill}{rgb}{0.000000,0.000000,0.000000}%
\pgfsetfillcolor{currentfill}%
\pgfsetlinewidth{0.602250pt}%
\definecolor{currentstroke}{rgb}{0.000000,0.000000,0.000000}%
\pgfsetstrokecolor{currentstroke}%
\pgfsetdash{}{0pt}%
\pgfsys@defobject{currentmarker}{\pgfqpoint{-0.027778in}{0.000000in}}{\pgfqpoint{-0.000000in}{0.000000in}}{%
\pgfpathmoveto{\pgfqpoint{-0.000000in}{0.000000in}}%
\pgfpathlineto{\pgfqpoint{-0.027778in}{0.000000in}}%
\pgfusepath{stroke,fill}%
}%
\begin{pgfscope}%
\pgfsys@transformshift{0.800000in}{0.982026in}%
\pgfsys@useobject{currentmarker}{}%
\end{pgfscope}%
\end{pgfscope}%
\begin{pgfscope}%
\pgfsetbuttcap%
\pgfsetroundjoin%
\definecolor{currentfill}{rgb}{0.000000,0.000000,0.000000}%
\pgfsetfillcolor{currentfill}%
\pgfsetlinewidth{0.602250pt}%
\definecolor{currentstroke}{rgb}{0.000000,0.000000,0.000000}%
\pgfsetstrokecolor{currentstroke}%
\pgfsetdash{}{0pt}%
\pgfsys@defobject{currentmarker}{\pgfqpoint{-0.027778in}{0.000000in}}{\pgfqpoint{-0.000000in}{0.000000in}}{%
\pgfpathmoveto{\pgfqpoint{-0.000000in}{0.000000in}}%
\pgfpathlineto{\pgfqpoint{-0.027778in}{0.000000in}}%
\pgfusepath{stroke,fill}%
}%
\begin{pgfscope}%
\pgfsys@transformshift{0.800000in}{1.046652in}%
\pgfsys@useobject{currentmarker}{}%
\end{pgfscope}%
\end{pgfscope}%
\begin{pgfscope}%
\pgfsetbuttcap%
\pgfsetroundjoin%
\definecolor{currentfill}{rgb}{0.000000,0.000000,0.000000}%
\pgfsetfillcolor{currentfill}%
\pgfsetlinewidth{0.602250pt}%
\definecolor{currentstroke}{rgb}{0.000000,0.000000,0.000000}%
\pgfsetstrokecolor{currentstroke}%
\pgfsetdash{}{0pt}%
\pgfsys@defobject{currentmarker}{\pgfqpoint{-0.027778in}{0.000000in}}{\pgfqpoint{-0.000000in}{0.000000in}}{%
\pgfpathmoveto{\pgfqpoint{-0.000000in}{0.000000in}}%
\pgfpathlineto{\pgfqpoint{-0.027778in}{0.000000in}}%
\pgfusepath{stroke,fill}%
}%
\begin{pgfscope}%
\pgfsys@transformshift{0.800000in}{1.484779in}%
\pgfsys@useobject{currentmarker}{}%
\end{pgfscope}%
\end{pgfscope}%
\begin{pgfscope}%
\pgfsetbuttcap%
\pgfsetroundjoin%
\definecolor{currentfill}{rgb}{0.000000,0.000000,0.000000}%
\pgfsetfillcolor{currentfill}%
\pgfsetlinewidth{0.602250pt}%
\definecolor{currentstroke}{rgb}{0.000000,0.000000,0.000000}%
\pgfsetstrokecolor{currentstroke}%
\pgfsetdash{}{0pt}%
\pgfsys@defobject{currentmarker}{\pgfqpoint{-0.027778in}{0.000000in}}{\pgfqpoint{-0.000000in}{0.000000in}}{%
\pgfpathmoveto{\pgfqpoint{-0.000000in}{0.000000in}}%
\pgfpathlineto{\pgfqpoint{-0.027778in}{0.000000in}}%
\pgfusepath{stroke,fill}%
}%
\begin{pgfscope}%
\pgfsys@transformshift{0.800000in}{1.707251in}%
\pgfsys@useobject{currentmarker}{}%
\end{pgfscope}%
\end{pgfscope}%
\begin{pgfscope}%
\pgfsetbuttcap%
\pgfsetroundjoin%
\definecolor{currentfill}{rgb}{0.000000,0.000000,0.000000}%
\pgfsetfillcolor{currentfill}%
\pgfsetlinewidth{0.602250pt}%
\definecolor{currentstroke}{rgb}{0.000000,0.000000,0.000000}%
\pgfsetstrokecolor{currentstroke}%
\pgfsetdash{}{0pt}%
\pgfsys@defobject{currentmarker}{\pgfqpoint{-0.027778in}{0.000000in}}{\pgfqpoint{-0.000000in}{0.000000in}}{%
\pgfpathmoveto{\pgfqpoint{-0.000000in}{0.000000in}}%
\pgfpathlineto{\pgfqpoint{-0.027778in}{0.000000in}}%
\pgfusepath{stroke,fill}%
}%
\begin{pgfscope}%
\pgfsys@transformshift{0.800000in}{1.865097in}%
\pgfsys@useobject{currentmarker}{}%
\end{pgfscope}%
\end{pgfscope}%
\begin{pgfscope}%
\pgfsetbuttcap%
\pgfsetroundjoin%
\definecolor{currentfill}{rgb}{0.000000,0.000000,0.000000}%
\pgfsetfillcolor{currentfill}%
\pgfsetlinewidth{0.602250pt}%
\definecolor{currentstroke}{rgb}{0.000000,0.000000,0.000000}%
\pgfsetstrokecolor{currentstroke}%
\pgfsetdash{}{0pt}%
\pgfsys@defobject{currentmarker}{\pgfqpoint{-0.027778in}{0.000000in}}{\pgfqpoint{-0.000000in}{0.000000in}}{%
\pgfpathmoveto{\pgfqpoint{-0.000000in}{0.000000in}}%
\pgfpathlineto{\pgfqpoint{-0.027778in}{0.000000in}}%
\pgfusepath{stroke,fill}%
}%
\begin{pgfscope}%
\pgfsys@transformshift{0.800000in}{1.987532in}%
\pgfsys@useobject{currentmarker}{}%
\end{pgfscope}%
\end{pgfscope}%
\begin{pgfscope}%
\pgfsetbuttcap%
\pgfsetroundjoin%
\definecolor{currentfill}{rgb}{0.000000,0.000000,0.000000}%
\pgfsetfillcolor{currentfill}%
\pgfsetlinewidth{0.602250pt}%
\definecolor{currentstroke}{rgb}{0.000000,0.000000,0.000000}%
\pgfsetstrokecolor{currentstroke}%
\pgfsetdash{}{0pt}%
\pgfsys@defobject{currentmarker}{\pgfqpoint{-0.027778in}{0.000000in}}{\pgfqpoint{-0.000000in}{0.000000in}}{%
\pgfpathmoveto{\pgfqpoint{-0.000000in}{0.000000in}}%
\pgfpathlineto{\pgfqpoint{-0.027778in}{0.000000in}}%
\pgfusepath{stroke,fill}%
}%
\begin{pgfscope}%
\pgfsys@transformshift{0.800000in}{2.087568in}%
\pgfsys@useobject{currentmarker}{}%
\end{pgfscope}%
\end{pgfscope}%
\begin{pgfscope}%
\pgfsetbuttcap%
\pgfsetroundjoin%
\definecolor{currentfill}{rgb}{0.000000,0.000000,0.000000}%
\pgfsetfillcolor{currentfill}%
\pgfsetlinewidth{0.602250pt}%
\definecolor{currentstroke}{rgb}{0.000000,0.000000,0.000000}%
\pgfsetstrokecolor{currentstroke}%
\pgfsetdash{}{0pt}%
\pgfsys@defobject{currentmarker}{\pgfqpoint{-0.027778in}{0.000000in}}{\pgfqpoint{-0.000000in}{0.000000in}}{%
\pgfpathmoveto{\pgfqpoint{-0.000000in}{0.000000in}}%
\pgfpathlineto{\pgfqpoint{-0.027778in}{0.000000in}}%
\pgfusepath{stroke,fill}%
}%
\begin{pgfscope}%
\pgfsys@transformshift{0.800000in}{2.172148in}%
\pgfsys@useobject{currentmarker}{}%
\end{pgfscope}%
\end{pgfscope}%
\begin{pgfscope}%
\pgfsetbuttcap%
\pgfsetroundjoin%
\definecolor{currentfill}{rgb}{0.000000,0.000000,0.000000}%
\pgfsetfillcolor{currentfill}%
\pgfsetlinewidth{0.602250pt}%
\definecolor{currentstroke}{rgb}{0.000000,0.000000,0.000000}%
\pgfsetstrokecolor{currentstroke}%
\pgfsetdash{}{0pt}%
\pgfsys@defobject{currentmarker}{\pgfqpoint{-0.027778in}{0.000000in}}{\pgfqpoint{-0.000000in}{0.000000in}}{%
\pgfpathmoveto{\pgfqpoint{-0.000000in}{0.000000in}}%
\pgfpathlineto{\pgfqpoint{-0.027778in}{0.000000in}}%
\pgfusepath{stroke,fill}%
}%
\begin{pgfscope}%
\pgfsys@transformshift{0.800000in}{2.245414in}%
\pgfsys@useobject{currentmarker}{}%
\end{pgfscope}%
\end{pgfscope}%
\begin{pgfscope}%
\pgfsetbuttcap%
\pgfsetroundjoin%
\definecolor{currentfill}{rgb}{0.000000,0.000000,0.000000}%
\pgfsetfillcolor{currentfill}%
\pgfsetlinewidth{0.602250pt}%
\definecolor{currentstroke}{rgb}{0.000000,0.000000,0.000000}%
\pgfsetstrokecolor{currentstroke}%
\pgfsetdash{}{0pt}%
\pgfsys@defobject{currentmarker}{\pgfqpoint{-0.027778in}{0.000000in}}{\pgfqpoint{-0.000000in}{0.000000in}}{%
\pgfpathmoveto{\pgfqpoint{-0.000000in}{0.000000in}}%
\pgfpathlineto{\pgfqpoint{-0.027778in}{0.000000in}}%
\pgfusepath{stroke,fill}%
}%
\begin{pgfscope}%
\pgfsys@transformshift{0.800000in}{2.310040in}%
\pgfsys@useobject{currentmarker}{}%
\end{pgfscope}%
\end{pgfscope}%
\begin{pgfscope}%
\pgfsetbuttcap%
\pgfsetroundjoin%
\definecolor{currentfill}{rgb}{0.000000,0.000000,0.000000}%
\pgfsetfillcolor{currentfill}%
\pgfsetlinewidth{0.602250pt}%
\definecolor{currentstroke}{rgb}{0.000000,0.000000,0.000000}%
\pgfsetstrokecolor{currentstroke}%
\pgfsetdash{}{0pt}%
\pgfsys@defobject{currentmarker}{\pgfqpoint{-0.027778in}{0.000000in}}{\pgfqpoint{-0.000000in}{0.000000in}}{%
\pgfpathmoveto{\pgfqpoint{-0.000000in}{0.000000in}}%
\pgfpathlineto{\pgfqpoint{-0.027778in}{0.000000in}}%
\pgfusepath{stroke,fill}%
}%
\begin{pgfscope}%
\pgfsys@transformshift{0.800000in}{2.748167in}%
\pgfsys@useobject{currentmarker}{}%
\end{pgfscope}%
\end{pgfscope}%
\begin{pgfscope}%
\pgfsetbuttcap%
\pgfsetroundjoin%
\definecolor{currentfill}{rgb}{0.000000,0.000000,0.000000}%
\pgfsetfillcolor{currentfill}%
\pgfsetlinewidth{0.602250pt}%
\definecolor{currentstroke}{rgb}{0.000000,0.000000,0.000000}%
\pgfsetstrokecolor{currentstroke}%
\pgfsetdash{}{0pt}%
\pgfsys@defobject{currentmarker}{\pgfqpoint{-0.027778in}{0.000000in}}{\pgfqpoint{-0.000000in}{0.000000in}}{%
\pgfpathmoveto{\pgfqpoint{-0.000000in}{0.000000in}}%
\pgfpathlineto{\pgfqpoint{-0.027778in}{0.000000in}}%
\pgfusepath{stroke,fill}%
}%
\begin{pgfscope}%
\pgfsys@transformshift{0.800000in}{2.970639in}%
\pgfsys@useobject{currentmarker}{}%
\end{pgfscope}%
\end{pgfscope}%
\begin{pgfscope}%
\pgfsetbuttcap%
\pgfsetroundjoin%
\definecolor{currentfill}{rgb}{0.000000,0.000000,0.000000}%
\pgfsetfillcolor{currentfill}%
\pgfsetlinewidth{0.602250pt}%
\definecolor{currentstroke}{rgb}{0.000000,0.000000,0.000000}%
\pgfsetstrokecolor{currentstroke}%
\pgfsetdash{}{0pt}%
\pgfsys@defobject{currentmarker}{\pgfqpoint{-0.027778in}{0.000000in}}{\pgfqpoint{-0.000000in}{0.000000in}}{%
\pgfpathmoveto{\pgfqpoint{-0.000000in}{0.000000in}}%
\pgfpathlineto{\pgfqpoint{-0.027778in}{0.000000in}}%
\pgfusepath{stroke,fill}%
}%
\begin{pgfscope}%
\pgfsys@transformshift{0.800000in}{3.128485in}%
\pgfsys@useobject{currentmarker}{}%
\end{pgfscope}%
\end{pgfscope}%
\begin{pgfscope}%
\pgfsetbuttcap%
\pgfsetroundjoin%
\definecolor{currentfill}{rgb}{0.000000,0.000000,0.000000}%
\pgfsetfillcolor{currentfill}%
\pgfsetlinewidth{0.602250pt}%
\definecolor{currentstroke}{rgb}{0.000000,0.000000,0.000000}%
\pgfsetstrokecolor{currentstroke}%
\pgfsetdash{}{0pt}%
\pgfsys@defobject{currentmarker}{\pgfqpoint{-0.027778in}{0.000000in}}{\pgfqpoint{-0.000000in}{0.000000in}}{%
\pgfpathmoveto{\pgfqpoint{-0.000000in}{0.000000in}}%
\pgfpathlineto{\pgfqpoint{-0.027778in}{0.000000in}}%
\pgfusepath{stroke,fill}%
}%
\begin{pgfscope}%
\pgfsys@transformshift{0.800000in}{3.250920in}%
\pgfsys@useobject{currentmarker}{}%
\end{pgfscope}%
\end{pgfscope}%
\begin{pgfscope}%
\pgfsetbuttcap%
\pgfsetroundjoin%
\definecolor{currentfill}{rgb}{0.000000,0.000000,0.000000}%
\pgfsetfillcolor{currentfill}%
\pgfsetlinewidth{0.602250pt}%
\definecolor{currentstroke}{rgb}{0.000000,0.000000,0.000000}%
\pgfsetstrokecolor{currentstroke}%
\pgfsetdash{}{0pt}%
\pgfsys@defobject{currentmarker}{\pgfqpoint{-0.027778in}{0.000000in}}{\pgfqpoint{-0.000000in}{0.000000in}}{%
\pgfpathmoveto{\pgfqpoint{-0.000000in}{0.000000in}}%
\pgfpathlineto{\pgfqpoint{-0.027778in}{0.000000in}}%
\pgfusepath{stroke,fill}%
}%
\begin{pgfscope}%
\pgfsys@transformshift{0.800000in}{3.350956in}%
\pgfsys@useobject{currentmarker}{}%
\end{pgfscope}%
\end{pgfscope}%
\begin{pgfscope}%
\pgfsetbuttcap%
\pgfsetroundjoin%
\definecolor{currentfill}{rgb}{0.000000,0.000000,0.000000}%
\pgfsetfillcolor{currentfill}%
\pgfsetlinewidth{0.602250pt}%
\definecolor{currentstroke}{rgb}{0.000000,0.000000,0.000000}%
\pgfsetstrokecolor{currentstroke}%
\pgfsetdash{}{0pt}%
\pgfsys@defobject{currentmarker}{\pgfqpoint{-0.027778in}{0.000000in}}{\pgfqpoint{-0.000000in}{0.000000in}}{%
\pgfpathmoveto{\pgfqpoint{-0.000000in}{0.000000in}}%
\pgfpathlineto{\pgfqpoint{-0.027778in}{0.000000in}}%
\pgfusepath{stroke,fill}%
}%
\begin{pgfscope}%
\pgfsys@transformshift{0.800000in}{3.435536in}%
\pgfsys@useobject{currentmarker}{}%
\end{pgfscope}%
\end{pgfscope}%
\begin{pgfscope}%
\pgfsetbuttcap%
\pgfsetroundjoin%
\definecolor{currentfill}{rgb}{0.000000,0.000000,0.000000}%
\pgfsetfillcolor{currentfill}%
\pgfsetlinewidth{0.602250pt}%
\definecolor{currentstroke}{rgb}{0.000000,0.000000,0.000000}%
\pgfsetstrokecolor{currentstroke}%
\pgfsetdash{}{0pt}%
\pgfsys@defobject{currentmarker}{\pgfqpoint{-0.027778in}{0.000000in}}{\pgfqpoint{-0.000000in}{0.000000in}}{%
\pgfpathmoveto{\pgfqpoint{-0.000000in}{0.000000in}}%
\pgfpathlineto{\pgfqpoint{-0.027778in}{0.000000in}}%
\pgfusepath{stroke,fill}%
}%
\begin{pgfscope}%
\pgfsys@transformshift{0.800000in}{3.508802in}%
\pgfsys@useobject{currentmarker}{}%
\end{pgfscope}%
\end{pgfscope}%
\begin{pgfscope}%
\pgfsetbuttcap%
\pgfsetroundjoin%
\definecolor{currentfill}{rgb}{0.000000,0.000000,0.000000}%
\pgfsetfillcolor{currentfill}%
\pgfsetlinewidth{0.602250pt}%
\definecolor{currentstroke}{rgb}{0.000000,0.000000,0.000000}%
\pgfsetstrokecolor{currentstroke}%
\pgfsetdash{}{0pt}%
\pgfsys@defobject{currentmarker}{\pgfqpoint{-0.027778in}{0.000000in}}{\pgfqpoint{-0.000000in}{0.000000in}}{%
\pgfpathmoveto{\pgfqpoint{-0.000000in}{0.000000in}}%
\pgfpathlineto{\pgfqpoint{-0.027778in}{0.000000in}}%
\pgfusepath{stroke,fill}%
}%
\begin{pgfscope}%
\pgfsys@transformshift{0.800000in}{3.573428in}%
\pgfsys@useobject{currentmarker}{}%
\end{pgfscope}%
\end{pgfscope}%
\begin{pgfscope}%
\pgfsetbuttcap%
\pgfsetroundjoin%
\definecolor{currentfill}{rgb}{0.000000,0.000000,0.000000}%
\pgfsetfillcolor{currentfill}%
\pgfsetlinewidth{0.602250pt}%
\definecolor{currentstroke}{rgb}{0.000000,0.000000,0.000000}%
\pgfsetstrokecolor{currentstroke}%
\pgfsetdash{}{0pt}%
\pgfsys@defobject{currentmarker}{\pgfqpoint{-0.027778in}{0.000000in}}{\pgfqpoint{-0.000000in}{0.000000in}}{%
\pgfpathmoveto{\pgfqpoint{-0.000000in}{0.000000in}}%
\pgfpathlineto{\pgfqpoint{-0.027778in}{0.000000in}}%
\pgfusepath{stroke,fill}%
}%
\begin{pgfscope}%
\pgfsys@transformshift{0.800000in}{4.011555in}%
\pgfsys@useobject{currentmarker}{}%
\end{pgfscope}%
\end{pgfscope}%
\begin{pgfscope}%
\definecolor{textcolor}{rgb}{0.000000,0.000000,0.000000}%
\pgfsetstrokecolor{textcolor}%
\pgfsetfillcolor{textcolor}%
\pgftext[x=0.446026in,y=2.376000in,,bottom,rotate=90.000000]{\color{textcolor}\rmfamily\fontsize{10.000000}{12.000000}\selectfont Time (ns)}%
\end{pgfscope}%
\begin{pgfscope}%
\pgfpathrectangle{\pgfqpoint{0.800000in}{0.528000in}}{\pgfqpoint{4.960000in}{3.696000in}}%
\pgfusepath{clip}%
\pgfsetrectcap%
\pgfsetroundjoin%
\pgfsetlinewidth{1.505625pt}%
\definecolor{currentstroke}{rgb}{0.121569,0.466667,0.705882}%
\pgfsetstrokecolor{currentstroke}%
\pgfsetdash{}{0pt}%
\pgfpathmoveto{\pgfqpoint{1.025455in}{0.696000in}}%
\pgfpathlineto{\pgfqpoint{1.071466in}{1.180350in}}%
\pgfpathlineto{\pgfqpoint{1.117477in}{1.488448in}}%
\pgfpathlineto{\pgfqpoint{1.163488in}{1.717473in}}%
\pgfpathlineto{\pgfqpoint{1.209499in}{1.871489in}}%
\pgfpathlineto{\pgfqpoint{1.255510in}{2.005253in}}%
\pgfpathlineto{\pgfqpoint{1.301521in}{2.113700in}}%
\pgfpathlineto{\pgfqpoint{1.347532in}{2.163819in}}%
\pgfpathlineto{\pgfqpoint{1.393544in}{2.229182in}}%
\pgfpathlineto{\pgfqpoint{1.439555in}{2.285866in}}%
\pgfpathlineto{\pgfqpoint{1.485566in}{2.347192in}}%
\pgfpathlineto{\pgfqpoint{1.531577in}{2.410916in}}%
\pgfpathlineto{\pgfqpoint{1.577588in}{2.481843in}}%
\pgfpathlineto{\pgfqpoint{1.623599in}{2.556214in}}%
\pgfpathlineto{\pgfqpoint{1.669610in}{2.569878in}}%
\pgfpathlineto{\pgfqpoint{1.715622in}{2.606493in}}%
\pgfpathlineto{\pgfqpoint{1.761633in}{2.691278in}}%
\pgfpathlineto{\pgfqpoint{1.807644in}{2.680799in}}%
\pgfpathlineto{\pgfqpoint{1.853655in}{2.717871in}}%
\pgfpathlineto{\pgfqpoint{1.899666in}{2.796657in}}%
\pgfpathlineto{\pgfqpoint{1.945677in}{2.776964in}}%
\pgfpathlineto{\pgfqpoint{1.991688in}{2.806306in}}%
\pgfpathlineto{\pgfqpoint{2.037699in}{2.844544in}}%
\pgfpathlineto{\pgfqpoint{2.083711in}{2.886645in}}%
\pgfpathlineto{\pgfqpoint{2.129722in}{2.915195in}}%
\pgfpathlineto{\pgfqpoint{2.175733in}{2.941124in}}%
\pgfpathlineto{\pgfqpoint{2.221744in}{2.962074in}}%
\pgfpathlineto{\pgfqpoint{2.267755in}{3.022435in}}%
\pgfpathlineto{\pgfqpoint{2.313766in}{3.008499in}}%
\pgfpathlineto{\pgfqpoint{2.359777in}{3.034051in}}%
\pgfpathlineto{\pgfqpoint{2.405788in}{3.047441in}}%
\pgfpathlineto{\pgfqpoint{2.451800in}{3.063828in}}%
\pgfpathlineto{\pgfqpoint{2.497811in}{3.082060in}}%
\pgfpathlineto{\pgfqpoint{2.543822in}{3.103273in}}%
\pgfpathlineto{\pgfqpoint{2.589833in}{3.118105in}}%
\pgfpathlineto{\pgfqpoint{2.635844in}{3.157159in}}%
\pgfpathlineto{\pgfqpoint{2.681855in}{3.152806in}}%
\pgfpathlineto{\pgfqpoint{2.727866in}{3.173421in}}%
\pgfpathlineto{\pgfqpoint{2.773878in}{3.190268in}}%
\pgfpathlineto{\pgfqpoint{2.819889in}{3.204300in}}%
\pgfpathlineto{\pgfqpoint{2.865900in}{3.226132in}}%
\pgfpathlineto{\pgfqpoint{2.911911in}{3.265864in}}%
\pgfpathlineto{\pgfqpoint{2.957922in}{3.253948in}}%
\pgfpathlineto{\pgfqpoint{3.003933in}{3.270515in}}%
\pgfpathlineto{\pgfqpoint{3.049944in}{3.285631in}}%
\pgfpathlineto{\pgfqpoint{3.095955in}{3.299684in}}%
\pgfpathlineto{\pgfqpoint{3.141967in}{3.318836in}}%
\pgfpathlineto{\pgfqpoint{3.187978in}{3.334501in}}%
\pgfpathlineto{\pgfqpoint{3.233989in}{3.352929in}}%
\pgfpathlineto{\pgfqpoint{3.280000in}{3.370017in}}%
\pgfpathlineto{\pgfqpoint{3.326011in}{3.386209in}}%
\pgfpathlineto{\pgfqpoint{3.372022in}{3.395684in}}%
\pgfpathlineto{\pgfqpoint{3.418033in}{3.406184in}}%
\pgfpathlineto{\pgfqpoint{3.464045in}{3.446286in}}%
\pgfpathlineto{\pgfqpoint{3.510056in}{3.429436in}}%
\pgfpathlineto{\pgfqpoint{3.556067in}{3.440134in}}%
\pgfpathlineto{\pgfqpoint{3.602078in}{3.450934in}}%
\pgfpathlineto{\pgfqpoint{3.648089in}{3.458854in}}%
\pgfpathlineto{\pgfqpoint{3.694100in}{3.486312in}}%
\pgfpathlineto{\pgfqpoint{3.740111in}{3.480044in}}%
\pgfpathlineto{\pgfqpoint{3.786122in}{3.536617in}}%
\pgfpathlineto{\pgfqpoint{3.832134in}{3.498190in}}%
\pgfpathlineto{\pgfqpoint{3.878145in}{3.508131in}}%
\pgfpathlineto{\pgfqpoint{3.924156in}{3.519731in}}%
\pgfpathlineto{\pgfqpoint{3.970167in}{3.527689in}}%
\pgfpathlineto{\pgfqpoint{4.016178in}{3.551317in}}%
\pgfpathlineto{\pgfqpoint{4.062189in}{3.543943in}}%
\pgfpathlineto{\pgfqpoint{4.108200in}{3.578718in}}%
\pgfpathlineto{\pgfqpoint{4.154212in}{3.561364in}}%
\pgfpathlineto{\pgfqpoint{4.200223in}{3.571489in}}%
\pgfpathlineto{\pgfqpoint{4.246234in}{3.579427in}}%
\pgfpathlineto{\pgfqpoint{4.292245in}{3.587965in}}%
\pgfpathlineto{\pgfqpoint{4.338256in}{3.600628in}}%
\pgfpathlineto{\pgfqpoint{4.384267in}{3.602418in}}%
\pgfpathlineto{\pgfqpoint{4.430278in}{3.689030in}}%
\pgfpathlineto{\pgfqpoint{4.476289in}{3.621870in}}%
\pgfpathlineto{\pgfqpoint{4.522301in}{3.628029in}}%
\pgfpathlineto{\pgfqpoint{4.568312in}{3.638102in}}%
\pgfpathlineto{\pgfqpoint{4.614323in}{3.636155in}}%
\pgfpathlineto{\pgfqpoint{4.660334in}{3.641166in}}%
\pgfpathlineto{\pgfqpoint{4.706345in}{3.660107in}}%
\pgfpathlineto{\pgfqpoint{4.752356in}{3.656133in}}%
\pgfpathlineto{\pgfqpoint{4.798367in}{3.663675in}}%
\pgfpathlineto{\pgfqpoint{4.844378in}{3.672434in}}%
\pgfpathlineto{\pgfqpoint{4.890390in}{3.680058in}}%
\pgfpathlineto{\pgfqpoint{4.936401in}{3.707916in}}%
\pgfpathlineto{\pgfqpoint{4.982412in}{3.693469in}}%
\pgfpathlineto{\pgfqpoint{5.028423in}{3.701214in}}%
\pgfpathlineto{\pgfqpoint{5.074434in}{3.708278in}}%
\pgfpathlineto{\pgfqpoint{5.120445in}{3.714053in}}%
\pgfpathlineto{\pgfqpoint{5.166456in}{3.733253in}}%
\pgfpathlineto{\pgfqpoint{5.212468in}{3.731426in}}%
\pgfpathlineto{\pgfqpoint{5.258479in}{3.747259in}}%
\pgfpathlineto{\pgfqpoint{5.304490in}{3.757516in}}%
\pgfpathlineto{\pgfqpoint{5.350501in}{3.772923in}}%
\pgfpathlineto{\pgfqpoint{5.396512in}{3.773625in}}%
\pgfpathlineto{\pgfqpoint{5.442523in}{3.782087in}}%
\pgfpathlineto{\pgfqpoint{5.488534in}{3.791666in}}%
\pgfpathlineto{\pgfqpoint{5.534545in}{3.818418in}}%
\pgfusepath{stroke}%
\end{pgfscope}%
\begin{pgfscope}%
\pgfpathrectangle{\pgfqpoint{0.800000in}{0.528000in}}{\pgfqpoint{4.960000in}{3.696000in}}%
\pgfusepath{clip}%
\pgfsetrectcap%
\pgfsetroundjoin%
\pgfsetlinewidth{1.505625pt}%
\definecolor{currentstroke}{rgb}{1.000000,0.498039,0.054902}%
\pgfsetstrokecolor{currentstroke}%
\pgfsetdash{}{0pt}%
\pgfpathmoveto{\pgfqpoint{1.025455in}{1.082450in}}%
\pgfpathlineto{\pgfqpoint{1.071466in}{1.487440in}}%
\pgfpathlineto{\pgfqpoint{1.117477in}{1.759327in}}%
\pgfpathlineto{\pgfqpoint{1.163488in}{1.976833in}}%
\pgfpathlineto{\pgfqpoint{1.209499in}{2.140650in}}%
\pgfpathlineto{\pgfqpoint{1.255510in}{2.299940in}}%
\pgfpathlineto{\pgfqpoint{1.301521in}{2.413826in}}%
\pgfpathlineto{\pgfqpoint{1.347532in}{2.454080in}}%
\pgfpathlineto{\pgfqpoint{1.393544in}{2.536466in}}%
\pgfpathlineto{\pgfqpoint{1.439555in}{2.600271in}}%
\pgfpathlineto{\pgfqpoint{1.485566in}{2.666173in}}%
\pgfpathlineto{\pgfqpoint{1.531577in}{2.727371in}}%
\pgfpathlineto{\pgfqpoint{1.577588in}{2.775531in}}%
\pgfpathlineto{\pgfqpoint{1.623599in}{2.821734in}}%
\pgfpathlineto{\pgfqpoint{1.669610in}{2.866211in}}%
\pgfpathlineto{\pgfqpoint{1.715622in}{2.909600in}}%
\pgfpathlineto{\pgfqpoint{1.761633in}{2.947014in}}%
\pgfpathlineto{\pgfqpoint{1.807644in}{2.980533in}}%
\pgfpathlineto{\pgfqpoint{1.853655in}{3.015411in}}%
\pgfpathlineto{\pgfqpoint{1.899666in}{3.045963in}}%
\pgfpathlineto{\pgfqpoint{1.945677in}{3.078708in}}%
\pgfpathlineto{\pgfqpoint{1.991688in}{3.105067in}}%
\pgfpathlineto{\pgfqpoint{2.037699in}{3.145888in}}%
\pgfpathlineto{\pgfqpoint{2.083711in}{3.164296in}}%
\pgfpathlineto{\pgfqpoint{2.129722in}{3.191376in}}%
\pgfpathlineto{\pgfqpoint{2.175733in}{3.224984in}}%
\pgfpathlineto{\pgfqpoint{2.221744in}{3.241120in}}%
\pgfpathlineto{\pgfqpoint{2.267755in}{3.261175in}}%
\pgfpathlineto{\pgfqpoint{2.313766in}{3.287870in}}%
\pgfpathlineto{\pgfqpoint{2.359777in}{3.305919in}}%
\pgfpathlineto{\pgfqpoint{2.405788in}{3.328750in}}%
\pgfpathlineto{\pgfqpoint{2.451800in}{3.345760in}}%
\pgfpathlineto{\pgfqpoint{2.497811in}{3.366663in}}%
\pgfpathlineto{\pgfqpoint{2.543822in}{3.381330in}}%
\pgfpathlineto{\pgfqpoint{2.589833in}{3.401230in}}%
\pgfpathlineto{\pgfqpoint{2.635844in}{3.417310in}}%
\pgfpathlineto{\pgfqpoint{2.681855in}{3.433717in}}%
\pgfpathlineto{\pgfqpoint{2.727866in}{3.451197in}}%
\pgfpathlineto{\pgfqpoint{2.773878in}{3.504001in}}%
\pgfpathlineto{\pgfqpoint{2.819889in}{3.481606in}}%
\pgfpathlineto{\pgfqpoint{2.865900in}{3.498103in}}%
\pgfpathlineto{\pgfqpoint{2.911911in}{3.515449in}}%
\pgfpathlineto{\pgfqpoint{2.957922in}{3.525673in}}%
\pgfpathlineto{\pgfqpoint{3.003933in}{3.538968in}}%
\pgfpathlineto{\pgfqpoint{3.049944in}{3.556667in}}%
\pgfpathlineto{\pgfqpoint{3.095955in}{3.630354in}}%
\pgfpathlineto{\pgfqpoint{3.141967in}{3.583991in}}%
\pgfpathlineto{\pgfqpoint{3.187978in}{3.595764in}}%
\pgfpathlineto{\pgfqpoint{3.233989in}{3.606979in}}%
\pgfpathlineto{\pgfqpoint{3.280000in}{3.620611in}}%
\pgfpathlineto{\pgfqpoint{3.326011in}{3.636015in}}%
\pgfpathlineto{\pgfqpoint{3.372022in}{3.647541in}}%
\pgfpathlineto{\pgfqpoint{3.418033in}{3.665520in}}%
\pgfpathlineto{\pgfqpoint{3.464045in}{3.669909in}}%
\pgfpathlineto{\pgfqpoint{3.510056in}{3.684345in}}%
\pgfpathlineto{\pgfqpoint{3.556067in}{3.693134in}}%
\pgfpathlineto{\pgfqpoint{3.602078in}{3.706018in}}%
\pgfpathlineto{\pgfqpoint{3.648089in}{3.715880in}}%
\pgfpathlineto{\pgfqpoint{3.694100in}{3.735229in}}%
\pgfpathlineto{\pgfqpoint{3.740111in}{3.736417in}}%
\pgfpathlineto{\pgfqpoint{3.786122in}{3.747942in}}%
\pgfpathlineto{\pgfqpoint{3.832134in}{3.754525in}}%
\pgfpathlineto{\pgfqpoint{3.878145in}{3.763952in}}%
\pgfpathlineto{\pgfqpoint{3.924156in}{3.773748in}}%
\pgfpathlineto{\pgfqpoint{3.970167in}{3.784508in}}%
\pgfpathlineto{\pgfqpoint{4.016178in}{3.792580in}}%
\pgfpathlineto{\pgfqpoint{4.062189in}{3.804603in}}%
\pgfpathlineto{\pgfqpoint{4.108200in}{3.810947in}}%
\pgfpathlineto{\pgfqpoint{4.154212in}{3.819935in}}%
\pgfpathlineto{\pgfqpoint{4.200223in}{3.827220in}}%
\pgfpathlineto{\pgfqpoint{4.246234in}{3.837399in}}%
\pgfpathlineto{\pgfqpoint{4.292245in}{3.849378in}}%
\pgfpathlineto{\pgfqpoint{4.338256in}{3.851279in}}%
\pgfpathlineto{\pgfqpoint{4.384267in}{3.871458in}}%
\pgfpathlineto{\pgfqpoint{4.430278in}{3.872454in}}%
\pgfpathlineto{\pgfqpoint{4.476289in}{3.883280in}}%
\pgfpathlineto{\pgfqpoint{4.522301in}{3.888304in}}%
\pgfpathlineto{\pgfqpoint{4.568312in}{3.891234in}}%
\pgfpathlineto{\pgfqpoint{4.614323in}{3.921062in}}%
\pgfpathlineto{\pgfqpoint{4.660334in}{3.907033in}}%
\pgfpathlineto{\pgfqpoint{4.706345in}{3.918100in}}%
\pgfpathlineto{\pgfqpoint{4.752356in}{3.929203in}}%
\pgfpathlineto{\pgfqpoint{4.798367in}{4.003097in}}%
\pgfpathlineto{\pgfqpoint{4.844378in}{3.944586in}}%
\pgfpathlineto{\pgfqpoint{4.890390in}{3.953264in}}%
\pgfpathlineto{\pgfqpoint{4.936401in}{3.961693in}}%
\pgfpathlineto{\pgfqpoint{4.982412in}{4.018777in}}%
\pgfpathlineto{\pgfqpoint{5.028423in}{3.974372in}}%
\pgfpathlineto{\pgfqpoint{5.074434in}{3.977179in}}%
\pgfpathlineto{\pgfqpoint{5.120445in}{3.983665in}}%
\pgfpathlineto{\pgfqpoint{5.166456in}{3.990800in}}%
\pgfpathlineto{\pgfqpoint{5.212468in}{4.002104in}}%
\pgfpathlineto{\pgfqpoint{5.258479in}{4.003759in}}%
\pgfpathlineto{\pgfqpoint{5.304490in}{4.010298in}}%
\pgfpathlineto{\pgfqpoint{5.350501in}{4.041865in}}%
\pgfpathlineto{\pgfqpoint{5.396512in}{4.026604in}}%
\pgfpathlineto{\pgfqpoint{5.442523in}{4.037860in}}%
\pgfpathlineto{\pgfqpoint{5.488534in}{4.040261in}}%
\pgfpathlineto{\pgfqpoint{5.534545in}{4.056000in}}%
\pgfusepath{stroke}%
\end{pgfscope}%
\begin{pgfscope}%
\pgfsetrectcap%
\pgfsetmiterjoin%
\pgfsetlinewidth{0.803000pt}%
\definecolor{currentstroke}{rgb}{0.000000,0.000000,0.000000}%
\pgfsetstrokecolor{currentstroke}%
\pgfsetdash{}{0pt}%
\pgfpathmoveto{\pgfqpoint{0.800000in}{0.528000in}}%
\pgfpathlineto{\pgfqpoint{0.800000in}{4.224000in}}%
\pgfusepath{stroke}%
\end{pgfscope}%
\begin{pgfscope}%
\pgfsetrectcap%
\pgfsetmiterjoin%
\pgfsetlinewidth{0.803000pt}%
\definecolor{currentstroke}{rgb}{0.000000,0.000000,0.000000}%
\pgfsetstrokecolor{currentstroke}%
\pgfsetdash{}{0pt}%
\pgfpathmoveto{\pgfqpoint{5.760000in}{0.528000in}}%
\pgfpathlineto{\pgfqpoint{5.760000in}{4.224000in}}%
\pgfusepath{stroke}%
\end{pgfscope}%
\begin{pgfscope}%
\pgfsetrectcap%
\pgfsetmiterjoin%
\pgfsetlinewidth{0.803000pt}%
\definecolor{currentstroke}{rgb}{0.000000,0.000000,0.000000}%
\pgfsetstrokecolor{currentstroke}%
\pgfsetdash{}{0pt}%
\pgfpathmoveto{\pgfqpoint{0.800000in}{0.528000in}}%
\pgfpathlineto{\pgfqpoint{5.760000in}{0.528000in}}%
\pgfusepath{stroke}%
\end{pgfscope}%
\begin{pgfscope}%
\pgfsetrectcap%
\pgfsetmiterjoin%
\pgfsetlinewidth{0.803000pt}%
\definecolor{currentstroke}{rgb}{0.000000,0.000000,0.000000}%
\pgfsetstrokecolor{currentstroke}%
\pgfsetdash{}{0pt}%
\pgfpathmoveto{\pgfqpoint{0.800000in}{4.224000in}}%
\pgfpathlineto{\pgfqpoint{5.760000in}{4.224000in}}%
\pgfusepath{stroke}%
\end{pgfscope}%
\begin{pgfscope}%
\pgfsetbuttcap%
\pgfsetmiterjoin%
\definecolor{currentfill}{rgb}{1.000000,1.000000,1.000000}%
\pgfsetfillcolor{currentfill}%
\pgfsetfillopacity{0.800000}%
\pgfsetlinewidth{1.003750pt}%
\definecolor{currentstroke}{rgb}{0.800000,0.800000,0.800000}%
\pgfsetstrokecolor{currentstroke}%
\pgfsetstrokeopacity{0.800000}%
\pgfsetdash{}{0pt}%
\pgfpathmoveto{\pgfqpoint{0.897222in}{3.725543in}}%
\pgfpathlineto{\pgfqpoint{2.014507in}{3.725543in}}%
\pgfpathquadraticcurveto{\pgfqpoint{2.042285in}{3.725543in}}{\pgfqpoint{2.042285in}{3.753321in}}%
\pgfpathlineto{\pgfqpoint{2.042285in}{4.126778in}}%
\pgfpathquadraticcurveto{\pgfqpoint{2.042285in}{4.154556in}}{\pgfqpoint{2.014507in}{4.154556in}}%
\pgfpathlineto{\pgfqpoint{0.897222in}{4.154556in}}%
\pgfpathquadraticcurveto{\pgfqpoint{0.869444in}{4.154556in}}{\pgfqpoint{0.869444in}{4.126778in}}%
\pgfpathlineto{\pgfqpoint{0.869444in}{3.753321in}}%
\pgfpathquadraticcurveto{\pgfqpoint{0.869444in}{3.725543in}}{\pgfqpoint{0.897222in}{3.725543in}}%
\pgfpathlineto{\pgfqpoint{0.897222in}{3.725543in}}%
\pgfpathclose%
\pgfusepath{stroke,fill}%
\end{pgfscope}%
\begin{pgfscope}%
\pgfsetrectcap%
\pgfsetroundjoin%
\pgfsetlinewidth{1.505625pt}%
\definecolor{currentstroke}{rgb}{0.121569,0.466667,0.705882}%
\pgfsetstrokecolor{currentstroke}%
\pgfsetdash{}{0pt}%
\pgfpathmoveto{\pgfqpoint{0.925000in}{4.050389in}}%
\pgfpathlineto{\pgfqpoint{1.063889in}{4.050389in}}%
\pgfpathlineto{\pgfqpoint{1.202778in}{4.050389in}}%
\pgfusepath{stroke}%
\end{pgfscope}%
\begin{pgfscope}%
\definecolor{textcolor}{rgb}{0.000000,0.000000,0.000000}%
\pgfsetstrokecolor{textcolor}%
\pgfsetfillcolor{textcolor}%
\pgftext[x=1.313889in,y=4.001778in,left,base]{\color{textcolor}\rmfamily\fontsize{10.000000}{12.000000}\selectfont timsort}%
\end{pgfscope}%
\begin{pgfscope}%
\pgfsetrectcap%
\pgfsetroundjoin%
\pgfsetlinewidth{1.505625pt}%
\definecolor{currentstroke}{rgb}{1.000000,0.498039,0.054902}%
\pgfsetstrokecolor{currentstroke}%
\pgfsetdash{}{0pt}%
\pgfpathmoveto{\pgfqpoint{0.925000in}{3.856716in}}%
\pgfpathlineto{\pgfqpoint{1.063889in}{3.856716in}}%
\pgfpathlineto{\pgfqpoint{1.202778in}{3.856716in}}%
\pgfusepath{stroke}%
\end{pgfscope}%
\begin{pgfscope}%
\definecolor{textcolor}{rgb}{0.000000,0.000000,0.000000}%
\pgfsetstrokecolor{textcolor}%
\pgfsetfillcolor{textcolor}%
\pgftext[x=1.313889in,y=3.808105in,left,base]{\color{textcolor}\rmfamily\fontsize{10.000000}{12.000000}\selectfont bmergesort}%
\end{pgfscope}%
\end{pgfpicture}%
\makeatother%
\endgroup%

\subsubsection{Memory}
%% Creator: Matplotlib, PGF backend
%%
%% To include the figure in your LaTeX document, write
%%   \input{<filename>.pgf}
%%
%% Make sure the required packages are loaded in your preamble
%%   \usepackage{pgf}
%%
%% Also ensure that all the required font packages are loaded; for instance,
%% the lmodern package is sometimes necessary when using math font.
%%   \usepackage{lmodern}
%%
%% Figures using additional raster images can only be included by \input if
%% they are in the same directory as the main LaTeX file. For loading figures
%% from other directories you can use the `import` package
%%   \usepackage{import}
%%
%% and then include the figures with
%%   \import{<path to file>}{<filename>.pgf}
%%
%% Matplotlib used the following preamble
%%   
%%   \makeatletter\@ifpackageloaded{underscore}{}{\usepackage[strings]{underscore}}\makeatother
%%
\begingroup%
\makeatletter%
\begin{pgfpicture}%
\pgfpathrectangle{\pgfpointorigin}{\pgfqpoint{6.400000in}{4.800000in}}%
\pgfusepath{use as bounding box, clip}%
\begin{pgfscope}%
\pgfsetbuttcap%
\pgfsetmiterjoin%
\definecolor{currentfill}{rgb}{1.000000,1.000000,1.000000}%
\pgfsetfillcolor{currentfill}%
\pgfsetlinewidth{0.000000pt}%
\definecolor{currentstroke}{rgb}{1.000000,1.000000,1.000000}%
\pgfsetstrokecolor{currentstroke}%
\pgfsetdash{}{0pt}%
\pgfpathmoveto{\pgfqpoint{0.000000in}{0.000000in}}%
\pgfpathlineto{\pgfqpoint{6.400000in}{0.000000in}}%
\pgfpathlineto{\pgfqpoint{6.400000in}{4.800000in}}%
\pgfpathlineto{\pgfqpoint{0.000000in}{4.800000in}}%
\pgfpathlineto{\pgfqpoint{0.000000in}{0.000000in}}%
\pgfpathclose%
\pgfusepath{fill}%
\end{pgfscope}%
\begin{pgfscope}%
\pgfsetbuttcap%
\pgfsetmiterjoin%
\definecolor{currentfill}{rgb}{1.000000,1.000000,1.000000}%
\pgfsetfillcolor{currentfill}%
\pgfsetlinewidth{0.000000pt}%
\definecolor{currentstroke}{rgb}{0.000000,0.000000,0.000000}%
\pgfsetstrokecolor{currentstroke}%
\pgfsetstrokeopacity{0.000000}%
\pgfsetdash{}{0pt}%
\pgfpathmoveto{\pgfqpoint{0.800000in}{0.528000in}}%
\pgfpathlineto{\pgfqpoint{5.760000in}{0.528000in}}%
\pgfpathlineto{\pgfqpoint{5.760000in}{4.224000in}}%
\pgfpathlineto{\pgfqpoint{0.800000in}{4.224000in}}%
\pgfpathlineto{\pgfqpoint{0.800000in}{0.528000in}}%
\pgfpathclose%
\pgfusepath{fill}%
\end{pgfscope}%
\begin{pgfscope}%
\pgfsetbuttcap%
\pgfsetroundjoin%
\definecolor{currentfill}{rgb}{0.000000,0.000000,0.000000}%
\pgfsetfillcolor{currentfill}%
\pgfsetlinewidth{0.803000pt}%
\definecolor{currentstroke}{rgb}{0.000000,0.000000,0.000000}%
\pgfsetstrokecolor{currentstroke}%
\pgfsetdash{}{0pt}%
\pgfsys@defobject{currentmarker}{\pgfqpoint{0.000000in}{-0.048611in}}{\pgfqpoint{0.000000in}{0.000000in}}{%
\pgfpathmoveto{\pgfqpoint{0.000000in}{0.000000in}}%
\pgfpathlineto{\pgfqpoint{0.000000in}{-0.048611in}}%
\pgfusepath{stroke,fill}%
}%
\begin{pgfscope}%
\pgfsys@transformshift{0.979443in}{0.528000in}%
\pgfsys@useobject{currentmarker}{}%
\end{pgfscope}%
\end{pgfscope}%
\begin{pgfscope}%
\definecolor{textcolor}{rgb}{0.000000,0.000000,0.000000}%
\pgfsetstrokecolor{textcolor}%
\pgfsetfillcolor{textcolor}%
\pgftext[x=0.979443in,y=0.430778in,,top]{\color{textcolor}\rmfamily\fontsize{10.000000}{12.000000}\selectfont \(\displaystyle {0}\)}%
\end{pgfscope}%
\begin{pgfscope}%
\pgfsetbuttcap%
\pgfsetroundjoin%
\definecolor{currentfill}{rgb}{0.000000,0.000000,0.000000}%
\pgfsetfillcolor{currentfill}%
\pgfsetlinewidth{0.803000pt}%
\definecolor{currentstroke}{rgb}{0.000000,0.000000,0.000000}%
\pgfsetstrokecolor{currentstroke}%
\pgfsetdash{}{0pt}%
\pgfsys@defobject{currentmarker}{\pgfqpoint{0.000000in}{-0.048611in}}{\pgfqpoint{0.000000in}{0.000000in}}{%
\pgfpathmoveto{\pgfqpoint{0.000000in}{0.000000in}}%
\pgfpathlineto{\pgfqpoint{0.000000in}{-0.048611in}}%
\pgfusepath{stroke,fill}%
}%
\begin{pgfscope}%
\pgfsys@transformshift{1.899666in}{0.528000in}%
\pgfsys@useobject{currentmarker}{}%
\end{pgfscope}%
\end{pgfscope}%
\begin{pgfscope}%
\definecolor{textcolor}{rgb}{0.000000,0.000000,0.000000}%
\pgfsetstrokecolor{textcolor}%
\pgfsetfillcolor{textcolor}%
\pgftext[x=1.899666in,y=0.430778in,,top]{\color{textcolor}\rmfamily\fontsize{10.000000}{12.000000}\selectfont \(\displaystyle {2000}\)}%
\end{pgfscope}%
\begin{pgfscope}%
\pgfsetbuttcap%
\pgfsetroundjoin%
\definecolor{currentfill}{rgb}{0.000000,0.000000,0.000000}%
\pgfsetfillcolor{currentfill}%
\pgfsetlinewidth{0.803000pt}%
\definecolor{currentstroke}{rgb}{0.000000,0.000000,0.000000}%
\pgfsetstrokecolor{currentstroke}%
\pgfsetdash{}{0pt}%
\pgfsys@defobject{currentmarker}{\pgfqpoint{0.000000in}{-0.048611in}}{\pgfqpoint{0.000000in}{0.000000in}}{%
\pgfpathmoveto{\pgfqpoint{0.000000in}{0.000000in}}%
\pgfpathlineto{\pgfqpoint{0.000000in}{-0.048611in}}%
\pgfusepath{stroke,fill}%
}%
\begin{pgfscope}%
\pgfsys@transformshift{2.819889in}{0.528000in}%
\pgfsys@useobject{currentmarker}{}%
\end{pgfscope}%
\end{pgfscope}%
\begin{pgfscope}%
\definecolor{textcolor}{rgb}{0.000000,0.000000,0.000000}%
\pgfsetstrokecolor{textcolor}%
\pgfsetfillcolor{textcolor}%
\pgftext[x=2.819889in,y=0.430778in,,top]{\color{textcolor}\rmfamily\fontsize{10.000000}{12.000000}\selectfont \(\displaystyle {4000}\)}%
\end{pgfscope}%
\begin{pgfscope}%
\pgfsetbuttcap%
\pgfsetroundjoin%
\definecolor{currentfill}{rgb}{0.000000,0.000000,0.000000}%
\pgfsetfillcolor{currentfill}%
\pgfsetlinewidth{0.803000pt}%
\definecolor{currentstroke}{rgb}{0.000000,0.000000,0.000000}%
\pgfsetstrokecolor{currentstroke}%
\pgfsetdash{}{0pt}%
\pgfsys@defobject{currentmarker}{\pgfqpoint{0.000000in}{-0.048611in}}{\pgfqpoint{0.000000in}{0.000000in}}{%
\pgfpathmoveto{\pgfqpoint{0.000000in}{0.000000in}}%
\pgfpathlineto{\pgfqpoint{0.000000in}{-0.048611in}}%
\pgfusepath{stroke,fill}%
}%
\begin{pgfscope}%
\pgfsys@transformshift{3.740111in}{0.528000in}%
\pgfsys@useobject{currentmarker}{}%
\end{pgfscope}%
\end{pgfscope}%
\begin{pgfscope}%
\definecolor{textcolor}{rgb}{0.000000,0.000000,0.000000}%
\pgfsetstrokecolor{textcolor}%
\pgfsetfillcolor{textcolor}%
\pgftext[x=3.740111in,y=0.430778in,,top]{\color{textcolor}\rmfamily\fontsize{10.000000}{12.000000}\selectfont \(\displaystyle {6000}\)}%
\end{pgfscope}%
\begin{pgfscope}%
\pgfsetbuttcap%
\pgfsetroundjoin%
\definecolor{currentfill}{rgb}{0.000000,0.000000,0.000000}%
\pgfsetfillcolor{currentfill}%
\pgfsetlinewidth{0.803000pt}%
\definecolor{currentstroke}{rgb}{0.000000,0.000000,0.000000}%
\pgfsetstrokecolor{currentstroke}%
\pgfsetdash{}{0pt}%
\pgfsys@defobject{currentmarker}{\pgfqpoint{0.000000in}{-0.048611in}}{\pgfqpoint{0.000000in}{0.000000in}}{%
\pgfpathmoveto{\pgfqpoint{0.000000in}{0.000000in}}%
\pgfpathlineto{\pgfqpoint{0.000000in}{-0.048611in}}%
\pgfusepath{stroke,fill}%
}%
\begin{pgfscope}%
\pgfsys@transformshift{4.660334in}{0.528000in}%
\pgfsys@useobject{currentmarker}{}%
\end{pgfscope}%
\end{pgfscope}%
\begin{pgfscope}%
\definecolor{textcolor}{rgb}{0.000000,0.000000,0.000000}%
\pgfsetstrokecolor{textcolor}%
\pgfsetfillcolor{textcolor}%
\pgftext[x=4.660334in,y=0.430778in,,top]{\color{textcolor}\rmfamily\fontsize{10.000000}{12.000000}\selectfont \(\displaystyle {8000}\)}%
\end{pgfscope}%
\begin{pgfscope}%
\pgfsetbuttcap%
\pgfsetroundjoin%
\definecolor{currentfill}{rgb}{0.000000,0.000000,0.000000}%
\pgfsetfillcolor{currentfill}%
\pgfsetlinewidth{0.803000pt}%
\definecolor{currentstroke}{rgb}{0.000000,0.000000,0.000000}%
\pgfsetstrokecolor{currentstroke}%
\pgfsetdash{}{0pt}%
\pgfsys@defobject{currentmarker}{\pgfqpoint{0.000000in}{-0.048611in}}{\pgfqpoint{0.000000in}{0.000000in}}{%
\pgfpathmoveto{\pgfqpoint{0.000000in}{0.000000in}}%
\pgfpathlineto{\pgfqpoint{0.000000in}{-0.048611in}}%
\pgfusepath{stroke,fill}%
}%
\begin{pgfscope}%
\pgfsys@transformshift{5.580557in}{0.528000in}%
\pgfsys@useobject{currentmarker}{}%
\end{pgfscope}%
\end{pgfscope}%
\begin{pgfscope}%
\definecolor{textcolor}{rgb}{0.000000,0.000000,0.000000}%
\pgfsetstrokecolor{textcolor}%
\pgfsetfillcolor{textcolor}%
\pgftext[x=5.580557in,y=0.430778in,,top]{\color{textcolor}\rmfamily\fontsize{10.000000}{12.000000}\selectfont \(\displaystyle {10000}\)}%
\end{pgfscope}%
\begin{pgfscope}%
\definecolor{textcolor}{rgb}{0.000000,0.000000,0.000000}%
\pgfsetstrokecolor{textcolor}%
\pgfsetfillcolor{textcolor}%
\pgftext[x=3.280000in,y=0.251766in,,top]{\color{textcolor}\rmfamily\fontsize{10.000000}{12.000000}\selectfont Input Size}%
\end{pgfscope}%
\begin{pgfscope}%
\pgfsetbuttcap%
\pgfsetroundjoin%
\definecolor{currentfill}{rgb}{0.000000,0.000000,0.000000}%
\pgfsetfillcolor{currentfill}%
\pgfsetlinewidth{0.803000pt}%
\definecolor{currentstroke}{rgb}{0.000000,0.000000,0.000000}%
\pgfsetstrokecolor{currentstroke}%
\pgfsetdash{}{0pt}%
\pgfsys@defobject{currentmarker}{\pgfqpoint{-0.048611in}{0.000000in}}{\pgfqpoint{-0.000000in}{0.000000in}}{%
\pgfpathmoveto{\pgfqpoint{-0.000000in}{0.000000in}}%
\pgfpathlineto{\pgfqpoint{-0.048611in}{0.000000in}}%
\pgfusepath{stroke,fill}%
}%
\begin{pgfscope}%
\pgfsys@transformshift{0.800000in}{2.507159in}%
\pgfsys@useobject{currentmarker}{}%
\end{pgfscope}%
\end{pgfscope}%
\begin{pgfscope}%
\definecolor{textcolor}{rgb}{0.000000,0.000000,0.000000}%
\pgfsetstrokecolor{textcolor}%
\pgfsetfillcolor{textcolor}%
\pgftext[x=0.501581in, y=2.458934in, left, base]{\color{textcolor}\rmfamily\fontsize{10.000000}{12.000000}\selectfont \(\displaystyle {10^{4}}\)}%
\end{pgfscope}%
\begin{pgfscope}%
\pgfsetbuttcap%
\pgfsetroundjoin%
\definecolor{currentfill}{rgb}{0.000000,0.000000,0.000000}%
\pgfsetfillcolor{currentfill}%
\pgfsetlinewidth{0.602250pt}%
\definecolor{currentstroke}{rgb}{0.000000,0.000000,0.000000}%
\pgfsetstrokecolor{currentstroke}%
\pgfsetdash{}{0pt}%
\pgfsys@defobject{currentmarker}{\pgfqpoint{-0.027778in}{0.000000in}}{\pgfqpoint{-0.000000in}{0.000000in}}{%
\pgfpathmoveto{\pgfqpoint{-0.000000in}{0.000000in}}%
\pgfpathlineto{\pgfqpoint{-0.027778in}{0.000000in}}%
\pgfusepath{stroke,fill}%
}%
\begin{pgfscope}%
\pgfsys@transformshift{0.800000in}{0.711728in}%
\pgfsys@useobject{currentmarker}{}%
\end{pgfscope}%
\end{pgfscope}%
\begin{pgfscope}%
\pgfsetbuttcap%
\pgfsetroundjoin%
\definecolor{currentfill}{rgb}{0.000000,0.000000,0.000000}%
\pgfsetfillcolor{currentfill}%
\pgfsetlinewidth{0.602250pt}%
\definecolor{currentstroke}{rgb}{0.000000,0.000000,0.000000}%
\pgfsetstrokecolor{currentstroke}%
\pgfsetdash{}{0pt}%
\pgfsys@defobject{currentmarker}{\pgfqpoint{-0.027778in}{0.000000in}}{\pgfqpoint{-0.000000in}{0.000000in}}{%
\pgfpathmoveto{\pgfqpoint{-0.000000in}{0.000000in}}%
\pgfpathlineto{\pgfqpoint{-0.027778in}{0.000000in}}%
\pgfusepath{stroke,fill}%
}%
\begin{pgfscope}%
\pgfsys@transformshift{0.800000in}{1.164051in}%
\pgfsys@useobject{currentmarker}{}%
\end{pgfscope}%
\end{pgfscope}%
\begin{pgfscope}%
\pgfsetbuttcap%
\pgfsetroundjoin%
\definecolor{currentfill}{rgb}{0.000000,0.000000,0.000000}%
\pgfsetfillcolor{currentfill}%
\pgfsetlinewidth{0.602250pt}%
\definecolor{currentstroke}{rgb}{0.000000,0.000000,0.000000}%
\pgfsetstrokecolor{currentstroke}%
\pgfsetdash{}{0pt}%
\pgfsys@defobject{currentmarker}{\pgfqpoint{-0.027778in}{0.000000in}}{\pgfqpoint{-0.000000in}{0.000000in}}{%
\pgfpathmoveto{\pgfqpoint{-0.000000in}{0.000000in}}%
\pgfpathlineto{\pgfqpoint{-0.027778in}{0.000000in}}%
\pgfusepath{stroke,fill}%
}%
\begin{pgfscope}%
\pgfsys@transformshift{0.800000in}{1.484978in}%
\pgfsys@useobject{currentmarker}{}%
\end{pgfscope}%
\end{pgfscope}%
\begin{pgfscope}%
\pgfsetbuttcap%
\pgfsetroundjoin%
\definecolor{currentfill}{rgb}{0.000000,0.000000,0.000000}%
\pgfsetfillcolor{currentfill}%
\pgfsetlinewidth{0.602250pt}%
\definecolor{currentstroke}{rgb}{0.000000,0.000000,0.000000}%
\pgfsetstrokecolor{currentstroke}%
\pgfsetdash{}{0pt}%
\pgfsys@defobject{currentmarker}{\pgfqpoint{-0.027778in}{0.000000in}}{\pgfqpoint{-0.000000in}{0.000000in}}{%
\pgfpathmoveto{\pgfqpoint{-0.000000in}{0.000000in}}%
\pgfpathlineto{\pgfqpoint{-0.027778in}{0.000000in}}%
\pgfusepath{stroke,fill}%
}%
\begin{pgfscope}%
\pgfsys@transformshift{0.800000in}{1.733909in}%
\pgfsys@useobject{currentmarker}{}%
\end{pgfscope}%
\end{pgfscope}%
\begin{pgfscope}%
\pgfsetbuttcap%
\pgfsetroundjoin%
\definecolor{currentfill}{rgb}{0.000000,0.000000,0.000000}%
\pgfsetfillcolor{currentfill}%
\pgfsetlinewidth{0.602250pt}%
\definecolor{currentstroke}{rgb}{0.000000,0.000000,0.000000}%
\pgfsetstrokecolor{currentstroke}%
\pgfsetdash{}{0pt}%
\pgfsys@defobject{currentmarker}{\pgfqpoint{-0.027778in}{0.000000in}}{\pgfqpoint{-0.000000in}{0.000000in}}{%
\pgfpathmoveto{\pgfqpoint{-0.000000in}{0.000000in}}%
\pgfpathlineto{\pgfqpoint{-0.027778in}{0.000000in}}%
\pgfusepath{stroke,fill}%
}%
\begin{pgfscope}%
\pgfsys@transformshift{0.800000in}{1.937301in}%
\pgfsys@useobject{currentmarker}{}%
\end{pgfscope}%
\end{pgfscope}%
\begin{pgfscope}%
\pgfsetbuttcap%
\pgfsetroundjoin%
\definecolor{currentfill}{rgb}{0.000000,0.000000,0.000000}%
\pgfsetfillcolor{currentfill}%
\pgfsetlinewidth{0.602250pt}%
\definecolor{currentstroke}{rgb}{0.000000,0.000000,0.000000}%
\pgfsetstrokecolor{currentstroke}%
\pgfsetdash{}{0pt}%
\pgfsys@defobject{currentmarker}{\pgfqpoint{-0.027778in}{0.000000in}}{\pgfqpoint{-0.000000in}{0.000000in}}{%
\pgfpathmoveto{\pgfqpoint{-0.000000in}{0.000000in}}%
\pgfpathlineto{\pgfqpoint{-0.027778in}{0.000000in}}%
\pgfusepath{stroke,fill}%
}%
\begin{pgfscope}%
\pgfsys@transformshift{0.800000in}{2.109266in}%
\pgfsys@useobject{currentmarker}{}%
\end{pgfscope}%
\end{pgfscope}%
\begin{pgfscope}%
\pgfsetbuttcap%
\pgfsetroundjoin%
\definecolor{currentfill}{rgb}{0.000000,0.000000,0.000000}%
\pgfsetfillcolor{currentfill}%
\pgfsetlinewidth{0.602250pt}%
\definecolor{currentstroke}{rgb}{0.000000,0.000000,0.000000}%
\pgfsetstrokecolor{currentstroke}%
\pgfsetdash{}{0pt}%
\pgfsys@defobject{currentmarker}{\pgfqpoint{-0.027778in}{0.000000in}}{\pgfqpoint{-0.000000in}{0.000000in}}{%
\pgfpathmoveto{\pgfqpoint{-0.000000in}{0.000000in}}%
\pgfpathlineto{\pgfqpoint{-0.027778in}{0.000000in}}%
\pgfusepath{stroke,fill}%
}%
\begin{pgfscope}%
\pgfsys@transformshift{0.800000in}{2.258229in}%
\pgfsys@useobject{currentmarker}{}%
\end{pgfscope}%
\end{pgfscope}%
\begin{pgfscope}%
\pgfsetbuttcap%
\pgfsetroundjoin%
\definecolor{currentfill}{rgb}{0.000000,0.000000,0.000000}%
\pgfsetfillcolor{currentfill}%
\pgfsetlinewidth{0.602250pt}%
\definecolor{currentstroke}{rgb}{0.000000,0.000000,0.000000}%
\pgfsetstrokecolor{currentstroke}%
\pgfsetdash{}{0pt}%
\pgfsys@defobject{currentmarker}{\pgfqpoint{-0.027778in}{0.000000in}}{\pgfqpoint{-0.000000in}{0.000000in}}{%
\pgfpathmoveto{\pgfqpoint{-0.000000in}{0.000000in}}%
\pgfpathlineto{\pgfqpoint{-0.027778in}{0.000000in}}%
\pgfusepath{stroke,fill}%
}%
\begin{pgfscope}%
\pgfsys@transformshift{0.800000in}{2.389623in}%
\pgfsys@useobject{currentmarker}{}%
\end{pgfscope}%
\end{pgfscope}%
\begin{pgfscope}%
\pgfsetbuttcap%
\pgfsetroundjoin%
\definecolor{currentfill}{rgb}{0.000000,0.000000,0.000000}%
\pgfsetfillcolor{currentfill}%
\pgfsetlinewidth{0.602250pt}%
\definecolor{currentstroke}{rgb}{0.000000,0.000000,0.000000}%
\pgfsetstrokecolor{currentstroke}%
\pgfsetdash{}{0pt}%
\pgfsys@defobject{currentmarker}{\pgfqpoint{-0.027778in}{0.000000in}}{\pgfqpoint{-0.000000in}{0.000000in}}{%
\pgfpathmoveto{\pgfqpoint{-0.000000in}{0.000000in}}%
\pgfpathlineto{\pgfqpoint{-0.027778in}{0.000000in}}%
\pgfusepath{stroke,fill}%
}%
\begin{pgfscope}%
\pgfsys@transformshift{0.800000in}{3.280410in}%
\pgfsys@useobject{currentmarker}{}%
\end{pgfscope}%
\end{pgfscope}%
\begin{pgfscope}%
\pgfsetbuttcap%
\pgfsetroundjoin%
\definecolor{currentfill}{rgb}{0.000000,0.000000,0.000000}%
\pgfsetfillcolor{currentfill}%
\pgfsetlinewidth{0.602250pt}%
\definecolor{currentstroke}{rgb}{0.000000,0.000000,0.000000}%
\pgfsetstrokecolor{currentstroke}%
\pgfsetdash{}{0pt}%
\pgfsys@defobject{currentmarker}{\pgfqpoint{-0.027778in}{0.000000in}}{\pgfqpoint{-0.000000in}{0.000000in}}{%
\pgfpathmoveto{\pgfqpoint{-0.000000in}{0.000000in}}%
\pgfpathlineto{\pgfqpoint{-0.027778in}{0.000000in}}%
\pgfusepath{stroke,fill}%
}%
\begin{pgfscope}%
\pgfsys@transformshift{0.800000in}{3.732732in}%
\pgfsys@useobject{currentmarker}{}%
\end{pgfscope}%
\end{pgfscope}%
\begin{pgfscope}%
\pgfsetbuttcap%
\pgfsetroundjoin%
\definecolor{currentfill}{rgb}{0.000000,0.000000,0.000000}%
\pgfsetfillcolor{currentfill}%
\pgfsetlinewidth{0.602250pt}%
\definecolor{currentstroke}{rgb}{0.000000,0.000000,0.000000}%
\pgfsetstrokecolor{currentstroke}%
\pgfsetdash{}{0pt}%
\pgfsys@defobject{currentmarker}{\pgfqpoint{-0.027778in}{0.000000in}}{\pgfqpoint{-0.000000in}{0.000000in}}{%
\pgfpathmoveto{\pgfqpoint{-0.000000in}{0.000000in}}%
\pgfpathlineto{\pgfqpoint{-0.027778in}{0.000000in}}%
\pgfusepath{stroke,fill}%
}%
\begin{pgfscope}%
\pgfsys@transformshift{0.800000in}{4.053660in}%
\pgfsys@useobject{currentmarker}{}%
\end{pgfscope}%
\end{pgfscope}%
\begin{pgfscope}%
\definecolor{textcolor}{rgb}{0.000000,0.000000,0.000000}%
\pgfsetstrokecolor{textcolor}%
\pgfsetfillcolor{textcolor}%
\pgftext[x=0.446026in,y=2.376000in,,bottom,rotate=90.000000]{\color{textcolor}\rmfamily\fontsize{10.000000}{12.000000}\selectfont Memory (B)}%
\end{pgfscope}%
\begin{pgfscope}%
\pgfpathrectangle{\pgfqpoint{0.800000in}{0.528000in}}{\pgfqpoint{4.960000in}{3.696000in}}%
\pgfusepath{clip}%
\pgfsetrectcap%
\pgfsetroundjoin%
\pgfsetlinewidth{1.505625pt}%
\definecolor{currentstroke}{rgb}{0.121569,0.466667,0.705882}%
\pgfsetstrokecolor{currentstroke}%
\pgfsetdash{}{0pt}%
\pgfpathmoveto{\pgfqpoint{1.025455in}{1.508163in}}%
\pgfpathlineto{\pgfqpoint{1.071466in}{0.761899in}}%
\pgfpathlineto{\pgfqpoint{1.117477in}{0.744703in}}%
\pgfpathlineto{\pgfqpoint{1.163488in}{0.696000in}}%
\pgfpathlineto{\pgfqpoint{1.209499in}{0.902028in}}%
\pgfpathlineto{\pgfqpoint{1.255510in}{1.294453in}}%
\pgfpathlineto{\pgfqpoint{1.301521in}{1.264953in}}%
\pgfpathlineto{\pgfqpoint{1.347532in}{1.544178in}}%
\pgfpathlineto{\pgfqpoint{1.393544in}{1.508163in}}%
\pgfpathlineto{\pgfqpoint{1.439555in}{1.612399in}}%
\pgfpathlineto{\pgfqpoint{1.485566in}{1.707723in}}%
\pgfpathlineto{\pgfqpoint{1.531577in}{2.084449in}}%
\pgfpathlineto{\pgfqpoint{1.577588in}{1.876944in}}%
\pgfpathlineto{\pgfqpoint{1.623599in}{1.952810in}}%
\pgfpathlineto{\pgfqpoint{1.669610in}{2.023844in}}%
\pgfpathlineto{\pgfqpoint{1.715622in}{2.090624in}}%
\pgfpathlineto{\pgfqpoint{1.761633in}{2.153632in}}%
\pgfpathlineto{\pgfqpoint{1.807644in}{2.213270in}}%
\pgfpathlineto{\pgfqpoint{1.853655in}{2.269881in}}%
\pgfpathlineto{\pgfqpoint{1.899666in}{2.323757in}}%
\pgfpathlineto{\pgfqpoint{1.945677in}{2.375151in}}%
\pgfpathlineto{\pgfqpoint{1.991688in}{2.424281in}}%
\pgfpathlineto{\pgfqpoint{2.037699in}{2.471339in}}%
\pgfpathlineto{\pgfqpoint{2.083711in}{2.569212in}}%
\pgfpathlineto{\pgfqpoint{2.129722in}{2.559887in}}%
\pgfpathlineto{\pgfqpoint{2.175733in}{2.601658in}}%
\pgfpathlineto{\pgfqpoint{2.221744in}{2.641920in}}%
\pgfpathlineto{\pgfqpoint{2.267755in}{2.680781in}}%
\pgfpathlineto{\pgfqpoint{2.313766in}{2.718333in}}%
\pgfpathlineto{\pgfqpoint{2.359777in}{2.754662in}}%
\pgfpathlineto{\pgfqpoint{2.405788in}{2.789845in}}%
\pgfpathlineto{\pgfqpoint{2.451800in}{2.823952in}}%
\pgfpathlineto{\pgfqpoint{2.497811in}{2.857047in}}%
\pgfpathlineto{\pgfqpoint{2.543822in}{2.889189in}}%
\pgfpathlineto{\pgfqpoint{2.589833in}{2.920431in}}%
\pgfpathlineto{\pgfqpoint{2.635844in}{2.950821in}}%
\pgfpathlineto{\pgfqpoint{2.681855in}{2.980406in}}%
\pgfpathlineto{\pgfqpoint{2.727866in}{3.009226in}}%
\pgfpathlineto{\pgfqpoint{2.773878in}{3.037320in}}%
\pgfpathlineto{\pgfqpoint{2.819889in}{3.064724in}}%
\pgfpathlineto{\pgfqpoint{2.865900in}{3.091471in}}%
\pgfpathlineto{\pgfqpoint{2.911911in}{3.117592in}}%
\pgfpathlineto{\pgfqpoint{2.957922in}{3.143115in}}%
\pgfpathlineto{\pgfqpoint{3.003933in}{3.168067in}}%
\pgfpathlineto{\pgfqpoint{3.049944in}{3.192473in}}%
\pgfpathlineto{\pgfqpoint{3.095955in}{3.216357in}}%
\pgfpathlineto{\pgfqpoint{3.141967in}{3.267619in}}%
\pgfpathlineto{\pgfqpoint{3.187978in}{3.262643in}}%
\pgfpathlineto{\pgfqpoint{3.233989in}{3.285085in}}%
\pgfpathlineto{\pgfqpoint{3.280000in}{3.307085in}}%
\pgfpathlineto{\pgfqpoint{3.326011in}{3.328659in}}%
\pgfpathlineto{\pgfqpoint{3.372022in}{3.349824in}}%
\pgfpathlineto{\pgfqpoint{3.418033in}{3.370594in}}%
\pgfpathlineto{\pgfqpoint{3.464045in}{3.390986in}}%
\pgfpathlineto{\pgfqpoint{3.510056in}{3.411011in}}%
\pgfpathlineto{\pgfqpoint{3.556067in}{3.430682in}}%
\pgfpathlineto{\pgfqpoint{3.602078in}{3.450013in}}%
\pgfpathlineto{\pgfqpoint{3.648089in}{3.469015in}}%
\pgfpathlineto{\pgfqpoint{3.694100in}{3.487699in}}%
\pgfpathlineto{\pgfqpoint{3.740111in}{3.506074in}}%
\pgfpathlineto{\pgfqpoint{3.786122in}{3.524152in}}%
\pgfpathlineto{\pgfqpoint{3.832134in}{3.541942in}}%
\pgfpathlineto{\pgfqpoint{3.878145in}{3.559452in}}%
\pgfpathlineto{\pgfqpoint{3.924156in}{3.576692in}}%
\pgfpathlineto{\pgfqpoint{3.970167in}{3.593670in}}%
\pgfpathlineto{\pgfqpoint{4.016178in}{3.610392in}}%
\pgfpathlineto{\pgfqpoint{4.062189in}{3.626868in}}%
\pgfpathlineto{\pgfqpoint{4.108200in}{3.643104in}}%
\pgfpathlineto{\pgfqpoint{4.154212in}{3.659108in}}%
\pgfpathlineto{\pgfqpoint{4.200223in}{3.674885in}}%
\pgfpathlineto{\pgfqpoint{4.246234in}{3.690441in}}%
\pgfpathlineto{\pgfqpoint{4.292245in}{3.705784in}}%
\pgfpathlineto{\pgfqpoint{4.338256in}{3.720919in}}%
\pgfpathlineto{\pgfqpoint{4.384267in}{3.735851in}}%
\pgfpathlineto{\pgfqpoint{4.430278in}{3.750586in}}%
\pgfpathlineto{\pgfqpoint{4.476289in}{3.765129in}}%
\pgfpathlineto{\pgfqpoint{4.522301in}{3.779485in}}%
\pgfpathlineto{\pgfqpoint{4.568312in}{3.793658in}}%
\pgfpathlineto{\pgfqpoint{4.614323in}{3.807653in}}%
\pgfpathlineto{\pgfqpoint{4.660334in}{3.821475in}}%
\pgfpathlineto{\pgfqpoint{4.706345in}{3.835128in}}%
\pgfpathlineto{\pgfqpoint{4.752356in}{3.848616in}}%
\pgfpathlineto{\pgfqpoint{4.798367in}{3.861943in}}%
\pgfpathlineto{\pgfqpoint{4.844378in}{3.875112in}}%
\pgfpathlineto{\pgfqpoint{4.890390in}{3.888128in}}%
\pgfpathlineto{\pgfqpoint{4.936401in}{3.900993in}}%
\pgfpathlineto{\pgfqpoint{4.982412in}{3.913712in}}%
\pgfpathlineto{\pgfqpoint{5.028423in}{3.926288in}}%
\pgfpathlineto{\pgfqpoint{5.074434in}{3.938723in}}%
\pgfpathlineto{\pgfqpoint{5.120445in}{3.951022in}}%
\pgfpathlineto{\pgfqpoint{5.166456in}{3.963186in}}%
\pgfpathlineto{\pgfqpoint{5.212468in}{3.975219in}}%
\pgfpathlineto{\pgfqpoint{5.258479in}{4.001477in}}%
\pgfpathlineto{\pgfqpoint{5.304490in}{3.998902in}}%
\pgfpathlineto{\pgfqpoint{5.350501in}{4.010558in}}%
\pgfpathlineto{\pgfqpoint{5.396512in}{4.022093in}}%
\pgfpathlineto{\pgfqpoint{5.442523in}{4.033510in}}%
\pgfpathlineto{\pgfqpoint{5.488534in}{4.044812in}}%
\pgfpathlineto{\pgfqpoint{5.534545in}{4.056000in}}%
\pgfusepath{stroke}%
\end{pgfscope}%
\begin{pgfscope}%
\pgfpathrectangle{\pgfqpoint{0.800000in}{0.528000in}}{\pgfqpoint{4.960000in}{3.696000in}}%
\pgfusepath{clip}%
\pgfsetrectcap%
\pgfsetroundjoin%
\pgfsetlinewidth{1.505625pt}%
\definecolor{currentstroke}{rgb}{1.000000,0.498039,0.054902}%
\pgfsetstrokecolor{currentstroke}%
\pgfsetdash{}{0pt}%
\pgfpathmoveto{\pgfqpoint{1.025455in}{1.754249in}}%
\pgfpathlineto{\pgfqpoint{1.071466in}{0.761899in}}%
\pgfpathlineto{\pgfqpoint{1.117477in}{0.984499in}}%
\pgfpathlineto{\pgfqpoint{1.163488in}{0.696000in}}%
\pgfpathlineto{\pgfqpoint{1.209499in}{1.287484in}}%
\pgfpathlineto{\pgfqpoint{1.255510in}{1.294453in}}%
\pgfpathlineto{\pgfqpoint{1.301521in}{1.264953in}}%
\pgfpathlineto{\pgfqpoint{1.347532in}{1.393173in}}%
\pgfpathlineto{\pgfqpoint{1.393544in}{1.508163in}}%
\pgfpathlineto{\pgfqpoint{1.439555in}{1.612399in}}%
\pgfpathlineto{\pgfqpoint{1.485566in}{1.813965in}}%
\pgfpathlineto{\pgfqpoint{1.531577in}{1.795539in}}%
\pgfpathlineto{\pgfqpoint{1.577588in}{1.876944in}}%
\pgfpathlineto{\pgfqpoint{1.623599in}{1.952810in}}%
\pgfpathlineto{\pgfqpoint{1.669610in}{2.023844in}}%
\pgfpathlineto{\pgfqpoint{1.715622in}{2.090624in}}%
\pgfpathlineto{\pgfqpoint{1.761633in}{2.153632in}}%
\pgfpathlineto{\pgfqpoint{1.807644in}{2.213270in}}%
\pgfpathlineto{\pgfqpoint{1.853655in}{2.269881in}}%
\pgfpathlineto{\pgfqpoint{1.899666in}{2.323757in}}%
\pgfpathlineto{\pgfqpoint{1.945677in}{2.434806in}}%
\pgfpathlineto{\pgfqpoint{1.991688in}{2.424281in}}%
\pgfpathlineto{\pgfqpoint{2.037699in}{2.471339in}}%
\pgfpathlineto{\pgfqpoint{2.083711in}{2.516491in}}%
\pgfpathlineto{\pgfqpoint{2.129722in}{2.559887in}}%
\pgfpathlineto{\pgfqpoint{2.175733in}{2.601658in}}%
\pgfpathlineto{\pgfqpoint{2.221744in}{2.641920in}}%
\pgfpathlineto{\pgfqpoint{2.267755in}{2.680781in}}%
\pgfpathlineto{\pgfqpoint{2.313766in}{2.718333in}}%
\pgfpathlineto{\pgfqpoint{2.359777in}{2.754662in}}%
\pgfpathlineto{\pgfqpoint{2.405788in}{2.789845in}}%
\pgfpathlineto{\pgfqpoint{2.451800in}{2.823952in}}%
\pgfpathlineto{\pgfqpoint{2.497811in}{2.857047in}}%
\pgfpathlineto{\pgfqpoint{2.543822in}{2.889189in}}%
\pgfpathlineto{\pgfqpoint{2.589833in}{2.920431in}}%
\pgfpathlineto{\pgfqpoint{2.635844in}{2.950821in}}%
\pgfpathlineto{\pgfqpoint{2.681855in}{2.980406in}}%
\pgfpathlineto{\pgfqpoint{2.727866in}{3.009226in}}%
\pgfpathlineto{\pgfqpoint{2.773878in}{3.037320in}}%
\pgfpathlineto{\pgfqpoint{2.819889in}{3.064724in}}%
\pgfpathlineto{\pgfqpoint{2.865900in}{3.123257in}}%
\pgfpathlineto{\pgfqpoint{2.911911in}{3.117592in}}%
\pgfpathlineto{\pgfqpoint{2.957922in}{3.143115in}}%
\pgfpathlineto{\pgfqpoint{3.003933in}{3.168067in}}%
\pgfpathlineto{\pgfqpoint{3.049944in}{3.192473in}}%
\pgfpathlineto{\pgfqpoint{3.095955in}{3.216357in}}%
\pgfpathlineto{\pgfqpoint{3.141967in}{3.239740in}}%
\pgfpathlineto{\pgfqpoint{3.187978in}{3.262643in}}%
\pgfpathlineto{\pgfqpoint{3.233989in}{3.285085in}}%
\pgfpathlineto{\pgfqpoint{3.280000in}{3.307085in}}%
\pgfpathlineto{\pgfqpoint{3.326011in}{3.328659in}}%
\pgfpathlineto{\pgfqpoint{3.372022in}{3.349824in}}%
\pgfpathlineto{\pgfqpoint{3.418033in}{3.370594in}}%
\pgfpathlineto{\pgfqpoint{3.464045in}{3.390986in}}%
\pgfpathlineto{\pgfqpoint{3.510056in}{3.411011in}}%
\pgfpathlineto{\pgfqpoint{3.556067in}{3.430682in}}%
\pgfpathlineto{\pgfqpoint{3.602078in}{3.450013in}}%
\pgfpathlineto{\pgfqpoint{3.648089in}{3.469015in}}%
\pgfpathlineto{\pgfqpoint{3.694100in}{3.487699in}}%
\pgfpathlineto{\pgfqpoint{3.740111in}{3.506074in}}%
\pgfpathlineto{\pgfqpoint{3.786122in}{3.524152in}}%
\pgfpathlineto{\pgfqpoint{3.832134in}{3.541942in}}%
\pgfpathlineto{\pgfqpoint{3.878145in}{3.559452in}}%
\pgfpathlineto{\pgfqpoint{3.924156in}{3.576692in}}%
\pgfpathlineto{\pgfqpoint{3.970167in}{3.593670in}}%
\pgfpathlineto{\pgfqpoint{4.016178in}{3.610392in}}%
\pgfpathlineto{\pgfqpoint{4.062189in}{3.626868in}}%
\pgfpathlineto{\pgfqpoint{4.108200in}{3.643104in}}%
\pgfpathlineto{\pgfqpoint{4.154212in}{3.659108in}}%
\pgfpathlineto{\pgfqpoint{4.200223in}{3.674885in}}%
\pgfpathlineto{\pgfqpoint{4.246234in}{3.690441in}}%
\pgfpathlineto{\pgfqpoint{4.292245in}{3.705784in}}%
\pgfpathlineto{\pgfqpoint{4.338256in}{3.720919in}}%
\pgfpathlineto{\pgfqpoint{4.384267in}{3.735851in}}%
\pgfpathlineto{\pgfqpoint{4.430278in}{3.750586in}}%
\pgfpathlineto{\pgfqpoint{4.476289in}{3.765129in}}%
\pgfpathlineto{\pgfqpoint{4.522301in}{3.779485in}}%
\pgfpathlineto{\pgfqpoint{4.568312in}{3.793658in}}%
\pgfpathlineto{\pgfqpoint{4.614323in}{3.807653in}}%
\pgfpathlineto{\pgfqpoint{4.660334in}{3.821475in}}%
\pgfpathlineto{\pgfqpoint{4.706345in}{3.835128in}}%
\pgfpathlineto{\pgfqpoint{4.752356in}{3.864853in}}%
\pgfpathlineto{\pgfqpoint{4.798367in}{3.861943in}}%
\pgfpathlineto{\pgfqpoint{4.844378in}{3.875112in}}%
\pgfpathlineto{\pgfqpoint{4.890390in}{3.888128in}}%
\pgfpathlineto{\pgfqpoint{4.936401in}{3.900993in}}%
\pgfpathlineto{\pgfqpoint{4.982412in}{3.913712in}}%
\pgfpathlineto{\pgfqpoint{5.028423in}{3.926288in}}%
\pgfpathlineto{\pgfqpoint{5.074434in}{3.938723in}}%
\pgfpathlineto{\pgfqpoint{5.120445in}{3.951022in}}%
\pgfpathlineto{\pgfqpoint{5.166456in}{3.963186in}}%
\pgfpathlineto{\pgfqpoint{5.212468in}{3.975219in}}%
\pgfpathlineto{\pgfqpoint{5.258479in}{3.987123in}}%
\pgfpathlineto{\pgfqpoint{5.304490in}{3.998902in}}%
\pgfpathlineto{\pgfqpoint{5.350501in}{4.010558in}}%
\pgfpathlineto{\pgfqpoint{5.396512in}{4.022093in}}%
\pgfpathlineto{\pgfqpoint{5.442523in}{4.033510in}}%
\pgfpathlineto{\pgfqpoint{5.488534in}{4.044812in}}%
\pgfpathlineto{\pgfqpoint{5.534545in}{4.056000in}}%
\pgfusepath{stroke}%
\end{pgfscope}%
\begin{pgfscope}%
\pgfsetrectcap%
\pgfsetmiterjoin%
\pgfsetlinewidth{0.803000pt}%
\definecolor{currentstroke}{rgb}{0.000000,0.000000,0.000000}%
\pgfsetstrokecolor{currentstroke}%
\pgfsetdash{}{0pt}%
\pgfpathmoveto{\pgfqpoint{0.800000in}{0.528000in}}%
\pgfpathlineto{\pgfqpoint{0.800000in}{4.224000in}}%
\pgfusepath{stroke}%
\end{pgfscope}%
\begin{pgfscope}%
\pgfsetrectcap%
\pgfsetmiterjoin%
\pgfsetlinewidth{0.803000pt}%
\definecolor{currentstroke}{rgb}{0.000000,0.000000,0.000000}%
\pgfsetstrokecolor{currentstroke}%
\pgfsetdash{}{0pt}%
\pgfpathmoveto{\pgfqpoint{5.760000in}{0.528000in}}%
\pgfpathlineto{\pgfqpoint{5.760000in}{4.224000in}}%
\pgfusepath{stroke}%
\end{pgfscope}%
\begin{pgfscope}%
\pgfsetrectcap%
\pgfsetmiterjoin%
\pgfsetlinewidth{0.803000pt}%
\definecolor{currentstroke}{rgb}{0.000000,0.000000,0.000000}%
\pgfsetstrokecolor{currentstroke}%
\pgfsetdash{}{0pt}%
\pgfpathmoveto{\pgfqpoint{0.800000in}{0.528000in}}%
\pgfpathlineto{\pgfqpoint{5.760000in}{0.528000in}}%
\pgfusepath{stroke}%
\end{pgfscope}%
\begin{pgfscope}%
\pgfsetrectcap%
\pgfsetmiterjoin%
\pgfsetlinewidth{0.803000pt}%
\definecolor{currentstroke}{rgb}{0.000000,0.000000,0.000000}%
\pgfsetstrokecolor{currentstroke}%
\pgfsetdash{}{0pt}%
\pgfpathmoveto{\pgfqpoint{0.800000in}{4.224000in}}%
\pgfpathlineto{\pgfqpoint{5.760000in}{4.224000in}}%
\pgfusepath{stroke}%
\end{pgfscope}%
\begin{pgfscope}%
\pgfsetbuttcap%
\pgfsetmiterjoin%
\definecolor{currentfill}{rgb}{1.000000,1.000000,1.000000}%
\pgfsetfillcolor{currentfill}%
\pgfsetfillopacity{0.800000}%
\pgfsetlinewidth{1.003750pt}%
\definecolor{currentstroke}{rgb}{0.800000,0.800000,0.800000}%
\pgfsetstrokecolor{currentstroke}%
\pgfsetstrokeopacity{0.800000}%
\pgfsetdash{}{0pt}%
\pgfpathmoveto{\pgfqpoint{0.897222in}{3.725543in}}%
\pgfpathlineto{\pgfqpoint{2.014507in}{3.725543in}}%
\pgfpathquadraticcurveto{\pgfqpoint{2.042285in}{3.725543in}}{\pgfqpoint{2.042285in}{3.753321in}}%
\pgfpathlineto{\pgfqpoint{2.042285in}{4.126778in}}%
\pgfpathquadraticcurveto{\pgfqpoint{2.042285in}{4.154556in}}{\pgfqpoint{2.014507in}{4.154556in}}%
\pgfpathlineto{\pgfqpoint{0.897222in}{4.154556in}}%
\pgfpathquadraticcurveto{\pgfqpoint{0.869444in}{4.154556in}}{\pgfqpoint{0.869444in}{4.126778in}}%
\pgfpathlineto{\pgfqpoint{0.869444in}{3.753321in}}%
\pgfpathquadraticcurveto{\pgfqpoint{0.869444in}{3.725543in}}{\pgfqpoint{0.897222in}{3.725543in}}%
\pgfpathlineto{\pgfqpoint{0.897222in}{3.725543in}}%
\pgfpathclose%
\pgfusepath{stroke,fill}%
\end{pgfscope}%
\begin{pgfscope}%
\pgfsetrectcap%
\pgfsetroundjoin%
\pgfsetlinewidth{1.505625pt}%
\definecolor{currentstroke}{rgb}{0.121569,0.466667,0.705882}%
\pgfsetstrokecolor{currentstroke}%
\pgfsetdash{}{0pt}%
\pgfpathmoveto{\pgfqpoint{0.925000in}{4.050389in}}%
\pgfpathlineto{\pgfqpoint{1.063889in}{4.050389in}}%
\pgfpathlineto{\pgfqpoint{1.202778in}{4.050389in}}%
\pgfusepath{stroke}%
\end{pgfscope}%
\begin{pgfscope}%
\definecolor{textcolor}{rgb}{0.000000,0.000000,0.000000}%
\pgfsetstrokecolor{textcolor}%
\pgfsetfillcolor{textcolor}%
\pgftext[x=1.313889in,y=4.001778in,left,base]{\color{textcolor}\rmfamily\fontsize{10.000000}{12.000000}\selectfont timsort}%
\end{pgfscope}%
\begin{pgfscope}%
\pgfsetrectcap%
\pgfsetroundjoin%
\pgfsetlinewidth{1.505625pt}%
\definecolor{currentstroke}{rgb}{1.000000,0.498039,0.054902}%
\pgfsetstrokecolor{currentstroke}%
\pgfsetdash{}{0pt}%
\pgfpathmoveto{\pgfqpoint{0.925000in}{3.856716in}}%
\pgfpathlineto{\pgfqpoint{1.063889in}{3.856716in}}%
\pgfpathlineto{\pgfqpoint{1.202778in}{3.856716in}}%
\pgfusepath{stroke}%
\end{pgfscope}%
\begin{pgfscope}%
\definecolor{textcolor}{rgb}{0.000000,0.000000,0.000000}%
\pgfsetstrokecolor{textcolor}%
\pgfsetfillcolor{textcolor}%
\pgftext[x=1.313889in,y=3.808105in,left,base]{\color{textcolor}\rmfamily\fontsize{10.000000}{12.000000}\selectfont bmergesort}%
\end{pgfscope}%
\end{pgfpicture}%
\makeatother%
\endgroup%

\end{document}
